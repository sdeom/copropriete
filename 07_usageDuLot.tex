\chapter{L'usage du lot}

En principe, chaque copropriétaire peut utiliser et user comme il l'entend des parties privatives de son lot. Mais cette liberté d'usage comporte des limites que ne peut transgresser le copropriétaire sous peine de porter atteinte aux droits des autres membres du syndicat ou à la destination de l'immeuble. L'usage peut-il aller jusqu'à la modification de l'affectation que les documents contractuels de la copropriété ont assigné aux parties privatives ? Cette question importante et complexe devra faire l'objet d'un examen particulier.

\section{La protection de la liberté d'usage des parties privatives}

	\subsection{Le principe de liberté d’usage et de modification des parties privatives}
	
		\subsubsection{La liberté quant à l’usage des parties privatives}
		
			L'article 2 al.2 de la loi de 1965 dispose que : << les parties privatives sont la propriété exclusive de chaque copropriétaire. >> Or l'on sait que l'\emph{usus}, le droit d'user de la chose, constitue l'un des trois attributs du droit de propriété.
			
			Par ailleurs, l'article 9 de la même loi dispose que chaque copropriétaire use et jouit librement des parties privatives, sous la condition de ne porter atteinte ni aux droits des autres copropriétaires, ni à la destination de l'immeuble.
			
			Le copropriétaire peut par conséquent :
			\begin{itemize}
				\item utiliser les locaux comme bon lui semble pour les habiter ou pour l'exercice de sa profession dans la mesure où le règlement de copropriété et la destination de l'immeuble le permettent,
				\item introduire chez lui les personnes de son choix ou laisser son lot inoccupé,
				\item détenir ou non un animal familier.
			\end{itemize}
		
		\subsubsection{La liberté quant à la modification des parties privatives}
		
			Le copropriétaire peut également, sans autorisation :
			\begin{itemize}
				\item modifier la distribution des pièces à l'intérieur de son appartement;
				\item effectuer des travaux ou des aménagements intérieurs --- suppression ou création de cloisons\footnote{Paris, 27 mai 1981, D.1982 I.R. 143 \no46}, créer des placards, poser des revêtements muraux, mettre des lambris, poser des appareils sanitaires, installer une cave à mazout dans sa cave\footnote{T.G.I. Paris, 13 mars 1975, D.1976 I.R. 70} ou apporter toute amélioration qu'il juge utile
				\item aménager sa vitrine à son gré , si le lot est à usage de boutique,  sans que puisse lui être opposée la destination de l'immeuble (T.G.I. Paris 8ème chambre, 4 février 1975, D.1976 I.R. 70); étant entendu qu'il s'agit là d'aménagements à l'intérieur du lot et non en façade.
				\item supprimer des éléments d'équipement communs compris dans les parties privatives de son lot ou ne pas les utiliser : suppression de radiateurs à la condition de ne pas déséquilibrer le système de distribution de chaleur et de ne pas porter atteinte aux conditions de jouissance des autres lots (Versailles, 17 mars 1986, Revue des Loyers 1987, 108) et de continuer à participer aux charges afférentes à l'élément d'équipement supprimé.
			\end{itemize}
		
			Toutefois, le copropriétaire devra solliciter l’autorisation du Syndicat des Copropriétaires dès lors que les travaux envisagés portent atteinte aux parties communes ou à l’aspect extérieur de l’immeuble :
			\begin{itemize}
				\item o création d’une évacuation d’eaux usées qui s’accompagne de passage de la canalisation à travers le gros oeuvre du plancher, cette canalisation étant elle-même branchée sur une canalisation commune\footnote{PARIS 23\degres Ch 12 jan 1994 Dalloz 1994 Somm p 128}.
				\item création d'un plancher intermédiaire qui s'ancre sur le gros oeuvre de l'immeuble constitue une atteinte aux parties communes qui, à ce titre doit être autorisée par l'Assemblée Générale.
				\item dans les immeubles anciens la suppression de cloisons, parties privatives, peut avoir pour conséquence de fragiliser l'immeuble et à ce titre nécessite une autorisation de l'assemblée, alors même qu'il s'agit de travaux qui ne paraissent à première analyse qu'affecter les parties privatives.
			\end{itemize}
			
			On rencontre également dans certains règlements de copropriété des clauses restrictives quant à la faculté pour un copropriétaire de créer des revêtements en carrelage. Ces clauses doivent être considérées comme valables dès lors que ces dallages rapportés peuvent avoir pour conséquence une gêne pour les autres copropriétaires. La Cour de Paris a même admis, qu’en l’absence de toute restriction dans le règlement de copropriété, la pose de marbre constituait une atteinte aux parties communes.
		
		\subsubsection{La liberté d’affectation des parties privatives}
		
			On verra au chapitre suivant qu’après de longues hésitations, la jurisprudence a admis qu’un copropriétaire pouvait librement modifier l’affectation de ses parties privatives, et ainsi transformer :
			\begin{itemize}
				\item un local accessoire (grenier, cave) en local principal (aménagement d’un duplex ou d’un souplex),
				\item un local à usage de commerce en local à usage d’habitation, ou inversement.
			\end{itemize}
			
			Toutefois cette liberté d’affectation doit se faire dans le respect des clauses restrictives du règlement de copropriété, lorsque celles-ci sont justifiées par la destination de l’immeuble.
		
	\subsection{Les droits d’usage du lot soumis a une protection spécifique du législateur}
		
		\subsubsection{L’Affectation privative et la liberté religieuse}
		
			Une jurisprudence relativement fournie se développe en ce qui concerne les problèmes posés par la contradiction pouvant exister entre les droits collectifs et la liberté religieuse.
			Certes, en application de l'article 9 de la Convention de sauvegarde des droits de l'homme et des libertés fondamentales on doit admettre avec la Cour européenne des droits de l’Homme que la liberté de religion implique celle d'extérioriser cette religion individuellement et en privé, ou de manière collective, en public et dans le cercle de ceux dont on partage la foi (arrêt X... c. Grèce du 25 mai 1993, paragraphe 31, requête \no 14307/88) et que la liberté de manifester la religion par le culte et l'accomplissement des rites en fait partie intégrante (arrêt X... et autres c. Grèce du 26 septembre 1996, paragraphe 36, requête \no 18748/91).
			
			Ainsi, s’il a été jugé par le passé\footnote{Aix, 10 juin 1981, Bull. Aix, 2, 81 \no79. Paris 23\degres 12 juin 1992 Loyers et Copropriété octobre 1992 \no 398} que la possibilité laissée par le règlement d'exercer des activités professionnelles ou commerciales ne permet pas l'installation d'un temple ou d'un lieu de culte; le contraire doit être admis présentement; ce n’est pas parce que l’installation de lieu de culte n’entre pas dans les prévisions du règlement de copropriété, qu’une association peut se voir interdire d’affecter ses locaux du rez-de-chaussée à usage de culte, dès lors que ces locaux sont à vocation commerciale ou artisanale\footnote{Civ 20 juillet 1994 Myriam Zana c/ 10 rue Bayre, Dalloz 1995, Jur. P. 408}.
	
			Toutefois, cette liberté de culte ne concerne que les parties privatives, et peut être restreinte par des clauses limitatives justifiées par la destination de l’immeuble et figurant dans le règlement de copropriété.
			
			C’est précisément sous le visa de l'article 9 de la Convention européenne que la Cour de cassation avait cassé le 18 décembre 2002 un arrêt d'appel obligeant le bailleur institutionnel à l'installation d'un mécanisme de fermeture mécanique pour permettre aux croyants juifs de ne pas avoir à mettre en oeuvre un dispositif électrique la veille du chabbat, en jugeant que : "... les pratiques dictées par les convictions religieuses des preneurs, n'entrent pas, sauf convention expresse, dans le champ contractuel du bail et ne font naître à la charge du bailleur aucune obligation spécifique".
			
			Un autre arrêt de la cour de cassation\footnote{3e CIV. - 8 juin 2006. \no 05-14.774. - C.A. Aix-en-Provence, 18 janvier 2005} apporte une solution du même ordre :
			\begin{quote}
				\itshape
				Ayant retenu à bon droit que la liberté religieuse, pour fondamentale qu'elle soit, ne pouvait avoir effet de rendre licites les violations des dispositions d'un règlement de copropriété, une cour d'appel qui a relevé qu'une cabane édifiée sur un balcon à l'occasion d'une fête juive faisait partie des usages prohibées par ce règlement et portait atteinte à l'harmonie générale de l'immeuble, puisqu'elle était visible de la rue, en a exactement déduit que l'assemblée générale était fondée à mandater son syndic pour agir en justice en vue de son enlèvement.
			\end{quote}
		
		\subsubsection{La domiciliation d’entreprise et l’exercice d’une activité par un entrepreneur individuel (art. L123-10 du code de commerce)}
		
			La création d’entreprise, encouragée par le législateur, se heurte à deux contraintes pratiques :
			\begin{itemize}
				\item d’une part, l’inscription personnelle au registre du commerce ou au répertoire des métiers impose une domiciliation, et l’immatriculation d’une société est subordonnée à l’indication d’un siège social --- l’entrepreneur ou la société doit justifier de la jouissance effective du local ainsi déclaré ;
				\item d’autre part, l’article L 631-7 du Code de la construction et de l’habitation (CCH) interdit le changement d’affectation de locaux à usage d’habitation à Paris, dans les villes de la région parisienne dans un rayon de 50 kilomètres autour de l’emplacement des anciennes fortifications et dans les communes de plus de 10 000 habitants, sauf autorisation administrative spéciale.
			\end{itemize}
		
			C’est pourquoi la loi \no 84-1149 du 21 décembre 1984 avait modifié l’ordonnance \no 58-1352 du 27 décembre 1958 (article 1er ter) en permettant l’installation du siège d’une entreprise dans le local d’habitation de l’entrepreneur ou du représentant légal d’une personne morale pendant une durée
			maximale de deux ans, nonobstant toute clause ou stipulation contraire, et notamment malgré les clauses du règlement de copropriété
			
			En outre, par dérogation aux dispositions de l'article L. 631-7, l'exercice d'une activité professionnelle, y compris commerciale, est autorisé dans une partie d'un local à usage d'habitation, dès lors que l'activité considérée n'est exercée que par le ou les occupants ayant leur résidence principale dans ce local et ne conduit à y recevoir ni clientèle ni marchandises (loi \no 98-546 du 2 juillet 1998)
			
			Ce régime a été assez profondément modifié par la loi \no 2003-721 du 1er août 2003 pour l'initiative économique (loi Dutreil), donc l’objectif est de faciliter la domiciliation de l'entreprise dans le local d'habitation de son dirigeant. L’ordonnance \no 2005-655 du 8 juin 2005 a apporté quelques modifications supplémentaires à ce régime
			
			\paragraph{La domiciliation de société (article L.~123-11-1 du Code de commerce)}
			
				\begin{quote}
					\itshape
					La personne morale qui demande son immatriculation au registre du commerce et des sociétés est autorisée à installer son siège au domicile de son représentant légal et y exercer une activité, sauf dispositions législatives ou stipulations contractuelles contraires.
					
					Lorsque la personne morale est soumise à des dispositions législatives ou stipulations contractuelles mentionnées à l'alinéa précédent, son représentant légal peut en installer le siège à son domicile, pour une durée ne pouvant ni excéder cinq ans à compter de la création de celle-ci, ni dépasser le terme légal, contractuel ou judiciaire de l'occupation des locaux.
					
					Dans ce cas, elle doit, préalablement au dépôt de sa demande d'immatriculation, notifier par écrit au bailleur, au syndicat de la copropriété ou au représentant de l'ensemble immobilier son intention d'user de la faculté ainsi prévue.
					
					Avant l'expiration de la période mentionnée au deuxième alinéa, la personne doit, sous peine de radiation d'office, communiquer au greffe du tribunal les éléments justifiant son changement de situation, selon les modalités fixées par décret en Conseil d'État.
					
					Il ne peut résulter des dispositions du présent article ni le changement de destination de l'immeuble, ni l'application du statut des baux commerciaux.
				\end{quote}
				
				En d’autres termes, une société peut toujours être domiciliée au siège de son représentant légal. L’exercice de l’activité dans ce local doit respecter les clauses du règlement de copropriété. En revanche, la domiciliation de l’entreprise au domicile du représentant est possible, nonobstant les clauses contraires du règlement de copropriété :
				\begin{itemize}
					\item à condition qu’il n’y soit exercé aucune activité,
					\item pendant une durée maximale de 5 ans,
					\item sous condition d’une notification préalable au syndic.
				\end{itemize}
				
			\paragraph{L'exercice d'une activité individuelle (article L.~123-10 du Code de commerce)}
			
				\begin{quote}
					\itshape
					Les personnes physiques demandant leur immatriculation au registre du commerce et des sociétés ou au répertoire des métiers doivent déclarer l'adresse de leur entreprise et en justifier la jouissance.
					
					Les personnes physiques peuvent déclarer l'adresse de leur local d'habitation et y exercer une activité, dès lors qu'aucune disposition législative ou stipulation contractuelle ne s'y oppose.
				
					Lorsqu'elles ne disposent pas d'un établissement, les personnes physiques peuvent, à titre exclusif d'adresse de l'entreprise, déclarer celle de leur local d'habitation. Cette déclaration n'entraîne ni changement d'affectation des locaux, ni application du statut des baux commerciaux.
				\end{quote}
				
				Il ne s’agit plus ici de la simple domiciliation mais de l’exercice effectif d’une activité dans le local d'habitation de l'entrepreneur.
				
				Elle n’est possible \textbf{qu’en l’absence de disposition législative ou stipulation contractuelle s’y opposant}. L’exercice d’une activité doit donc être compatible avec le règlement de copropriété.
				
				Deux conditions doivent en outre être respectées dans les communes de plus de \nombre{200 000} habitants et dans les départements des Hauts de Seine, de la Seine-Saint-Denis et du Val de Marne :
				\begin{itemize}
					\item l'activité ne doit être exercée que par le ou les occupants ayant leur résidence principale dans ce local,
					\item l'activité ne doit pas comporter de réception de la clientèle ou de marchandises.
				\end{itemize}
		
	\subsection{L’interdiction faite au syndicat de porter atteinte au droit de jouissance du copropriétaire sur son lot (art.~26 al.~2 de loi du 10 juillet 1965)}
		
		Non seulement le copropriétaire est libre d'user à sa guise des parties privatives de son lot, mais encore il ne pourrait être contraint de modifier cet usage par une décision du syndicat.
		
		A cet égard, il se trouve protégé par l'article 26 al.2 de la loi selon lequel :
		\begin{quote}
			<< {\itshape L'assemblée générale ne peut, à quelque majorité que ce soit, imposer à un copropriétaire une modification à la destination de ses parties privatives ou aux modalités de leur jouissance, telles qu'elles résultent du règlement de copropriété.} >>
		\end{quote}
		
		Ainsi, les modalités de jouissance des locaux privés, c'est-à-dire toutes leurs conditions d'usage, ne peuvent elles être en principe modifiées sans l'accord du propriétaire intéressé.
		
		Cette protection de la liberté individuelle en ce qui concerne les modalités de jouissance des parties privatives était autrefois très strictement entendue, tout copropriétaire pouvait s'opposer :
		\begin{itemize}
			\item à la suppression du chauffage collectif\footnote{Civ. 3ème, 28 novembre 1973, Bull. Civ. III \no618},
			\item à la mise en place d'un dispositif de fermeture automatique des portes de l'immeuble,
			\item à la suppression de boîtes aux lettres individuelles\footnote{Civ. 3ème, 10 mai 1989, Gaz. Pal. 1984 2 Pan. 28},
			\item à la suppression de sonnettes d’appel\footnote{T.G.I. Paris, 26 mars 1976, D.1976 I.R. 307}.
		\end{itemize}
		
		Mais à l'heure actuelle, la position des tribunaux s'est assouplie s'il existe des circonstances dans lesquelles on peut considérer que les suppressions et aménagements projetés constituent des améliorations.
		
		Il en est ainsi en cas de suppression du chauffage collectif en vue de lui substituer un chauffage individuel plus économique ou plus performant\footnote{Civ. 3ème, 13 décembre 1983, Bull. Civ. III \no258, Rep. DEFRENOIS 1984, art.33379 \no84, obs. H. SOULEAU; Civ. 3ème, 24 janvier 1984, Rep. DEFRENOIS 1984, art.33342 \no55, obs. H. SOULEAU; Civ. 3ème, 26 novembre 1985, Bull. Civ. III \no156, Rep. DEFRENOIS 1986, art.33745 \no62, obs. H.SOULEAU; Civ. 3ème, 4 janvier 1989, D.1989 I.R. 26, Rep. DEFRENOIS 1989 art.34470 \no17, obs. H.SOULEAU}.

\section{Les restrictions au droit d'usage des parties privatives}
	
	L'usage des parties privatives n'est pas discrétionnaire. Il ne peut s'exercer qu'à l'intérieur de certaines limites.
	\begin{itemize}
		\item Les unes tiennent au respect des obligations de voisinage (\ref{07_II_A}).
		\item D'autres à des clauses restrictives du règlement concernant les conditions d'usage et de jouissance des parties privatives (\ref{07_II_B}).
		\item D'autres enfin à des décisions prises par l'assemblée générale. Certaines de ces restrictions entraînent une interdiction d'agir, d'autres une interdiction de s'opposer à l'activité d'autrui.
	\end{itemize}
	
	\subsection{Les restrictions tenant aux obligations de voisinage}\label{07_II_A}
	
		L'immeuble en copropriété est le lieu par excellence où l'on constate, à l'état dense, le phénomène du voisinage.
		Le copropriétaire est un voisin privilégié. Ce que Madame KISCHINEWSKY-BROQUISSE exprime en d'autres termes : << La copropriété donne l'image d'une promiscuité exacerbée de voisinage. >>
		
		Pour que la vie des différents occupants soit facilitée et que la présence des voisins ne soit pas une source d'exacerbation et d'inconfort, chacun doit respecter certaines obligations et tenir compte des droits concurrents des autres copropriétaires.
		
		\subsubsection{La responsabilité pour troubles de voisinage s'applique au sein de la copropriété.}
		
			La théorie prétorienne des troubles de voisinage soumet la personne qui est l'auteur d'un trouble anormal de voisinage à réparer le préjudice causé: « nul ne doit causer à autrui un trouble anormal de voisinage »\footnote{2è Civ., 19 novembre 1986, Bull. 1986, II, \no 172, pourvoi \no 84-16.379 ; 3è Civ., 13 avril 2005, Bull. 2005, III, \no 89, pourvoi \no 03-20.575}.
			
			Point n'est besoin de prouver la faute de l'auteur du trouble. Seul le caractère anormal de celui-ci est pris en considération.
	
			Bien qu'il y ait eu discussion sur le point de savoir si une responsabilité d'ordre délictuel pouvait jouer dans les rapports entre personnes soumises à un même contrat, le droit positif a admis que pouvait être condamné à réparation celui qui, dans l'usage qu'il fait de ses locaux, trouble de manière anormale les copropriétaires voisins dans l'exercice de leur propre droit d'usage. La jurisprudence est assez copieuse à cet égard. Le trouble peut être sonore, visuel, olfactif, ou même, selon une jurisprudence plus récente, esthétique\footnote{3è Civ., 9 mai 2001, pourvoi \no 99-16.260 ; 2è Civ., 24 février 2005, Bull. 2005, II, \no 50, pourvoi \no 04-10.362.}. En matière de copropriété la jurisprudence est abondante. Elle sanctionne des troubles provenant :
			\begin{itemize}
				\item de travaux effectués dans les parties privatives du lot voisin\footnote{Civ. 3ème, 6 mai 1975, Bull. Civ. III \no157; Civ. 3ème, 3 juillet 1979, D. 1980 I.R. 236} ;
				\item de bruits violents ou renouvelés provenant par exemple d'aboiements d'un chien\footnote{Paris, 22 juin 1962, Gaz. Pal. 1962, 2, 318} ou d'un chenil\footnote{TGI Paris 9 février 1977; D.S. 1978 IR 118} ; de chants et danses nocturnes\footnote{Riom, 3 mai 1966, A.J.P.I. 1967 p.38 note CABANAC}, d'une boulangerie fonctionnant la nuit\footnote{Civ. 3ème, 20 février 1973, Gaz. Pal. 1973, 1, 471}, de l'occupation en surnombre des chambres de service\footnote{Versailles, 23 octobre 1983, D.1984 I.R. 282} ;
				\item de bruits de piano\footnote{TGI Paris 21 déc 1984, Inf. Rap. Copropriété 1985 p 185}.
				\item de bruits provenant d'une cuisine installée dans une salle à manger d'origine\footnote{PARIS 23\degres Chambre 24 mars 1988; Juris Data n. 027267}.
				\item d'odeurs de mazout\footnote{T.G.I. Paris, 11 juillet 1974, D.1976 I.R.61}, d'odeurs de restaurant\footnote{T.G.I. Marseille, 6 mars 1986, Ann. Loyers 1986 p.ll0 \no16861} ;
				\item d'activités contraires à la décence minimum : prostitution\footnote{T.G.I. Grenoble, 6 avril 1964, J.C.P. 1964 II 13663, note SAVATIER, A.J.P.I. Janvier 1965 p.23, note CABANAC }, exploitation d'un sex-shop\footnote{Paris 16 juin 1978 : D.S. 1980 IR 235, note Giverdon ; Civ. 3ème, 24 février 1981, Gaz. Pal. 1981 2 Pan. Jur. 221 }, massages thaïlandais\footnote{Paris, 13 octobre 1982, D.1983 I.R. 336}.
			\end{itemize}
	
			Ces diverses activités entraînent la responsabilité de leur auteur alors même qu'elles auraient été approuvées par l'assemblée générale\footnote{Civ. 3ème, 3 juillet 1979 (solution implicite), D.1980 I.R. 236}.
			
			La responsabilité pour troubles de voisinage suppose rapportée la preuve du caractère anormal de ce trouble, c'est-à-dire excédant les inconvénients normaux de voisinage. Ce trouble anormal s’appréciant de façon objective, notamment en ce qui concerne le bruit au regard de la qualité de la construction (immeuble mal insonorisé bien que conforme aux normes de l’époque de construction et indépendamment des qualités ou handicaps de la personne qui subit ce trouble\footnote{Cour d'Appel Paris, pôle 4, 2\degres Ch. 20 janvier 2016, \no 14/14691 JurisData \no 2016-000817 Loyers et Copropriété fev 2016 \no 110}.
			
			Or, la victime est exonérée de cette preuve si le trouble provient d'un usage contraire aux dispositions du règlement.
			
			Par contre l’installation d’une antenne relais de téléphone mobile sur la toiture non accessible de l’immeuble n’a pas été considérée comme un trouble anormal de voisinage pouvant justifier la demande de dépose par un copropriétaire\footnote{CA Paris 23\degres Ch A 7 mai 2002, Loyers et Copropriété 2002 \no 272}.
			
			Nul n’est assuré de conserver son ensoleillement et donc la construction de l’immeuble voisin ne constitue pas un trouble anormal du voisinage\footnote{Cass. Civ. 3e 21 octobre 2009}. Pour autant, une construction permettant d’avoir une vue directe et plongeante chez un voisin prive celui-ci de jouir pleinement de son droit de propriété\footnote{3è Civ., 7 février 2007, pourvoi \no 07-21.405.}.
			
			Enfin, le Syndicat peut aussi se prévaloir de la théorie du trouble anormal de voisinage, y compris contre un copropriétaire, ce qui lui permet notamment en cas de désordres dus aux travaux réalisés par un copropriétaire, de mettre en cause la responsabilité du propriétaire actuel du lot, quand bien même il n’aurait pas réalisé les travaux\footnote{Cour de cassation chambre civile 3 11 mai 2017 \no 16-14339 FS-P+B+I : JurisData \no 2017-008883 et \no 16-14.665}.
		
		\subsubsection{La responsabilité civile de droit commun s’applique également entre copropriétaires}
		
			\paragraph{Pour les désordres matériels}
	
				De la même façon, le copropriétaire est responsable des désordres matériels qu’il cause aux lots privatifs ou aux parties communes\footnote{Civ 3\degres 27 fév 2001 AJDI oct 2002 p. 677 note Capoulade à propos de la chute d’une cheminée privative.}.
				Dans une copropriété à deux personnes, les copropriétaires n’ont individuellement aucune qualité pour répondre des désordres provenant des parties communes et donc l’assignation de ces copropriétaires n’est pas recevable\footnote{Cass. Civ. 3e 15 décembre 2010}.
			
			\paragraph{Pour les manquements juridiques du copropriétaire}
			
			Le copropriétaire peut engager sa responsabilité en tant que copropriétaire indépendamment des troubles de voisinage, notamment par ce que le Président Capoulade appelle ses « manquements juridiques. »
			
			C’est ainsi notamment qu’il doit garder sa sérénité dans l’appréciation de la gestion de la copropriété : par exemple il ne peut demander d’inscrire à l’ordre du jour d’une assemblée générale un projet de résolution dont la rédaction présenterait un caractère diffamatoire\footnote{Civ 3\degres 4 avril 2001 AJDI oct 2002 p. 677 note Capoulade.}.
		
		\subsubsection{La responsabilité du copropriétaire peut également être engagée sur un fondement contractuel, pour violation des clauses du règlement de copropriété relatives au respect de la tranquillité des autres copropriétaires}
		
			La plupart des règlements contiennent des clauses (souvent considérées, à tort, comme des clauses de style\footnote{La clause de style est une clause non motivée qui n’a pas de portée juridique.}) interdisant aux copropriétaires de troubler la tranquillité des autres ou d'apporter une gêne à l'habitation par :
			\begin{itemize}
				\item les bruits, les odeurs, les vibrations ;
				\item es activités professionnelles ou commerciales susceptibles d'entraîner des bruits, odeurs, \etc ;
				\item l'occupation des locaux d'habitation par des personnes de mauvaise vie et moeurs ;
				\item l'étendage du linge aux fenêtres, \etc
			\end{itemize}
			Ces dispositions permettent de condamner celui qui est à l'origine de tels troubles alors même que ceux-ci n'excéderaient pas les inconvénients normaux de voisinage.
			
			Ainsi, la Cour de cassation a-t-elle décidé à propos d'une boulangerie dont une turbine provoquait bruits et vibrations << que par application du règlement de copropriété, le commerce exploité dans le lot ne devait apporter, même par des conditions normales de fonctionnement, aucune gêne d'habitation, notamment par les bruits >>\footnote{Civ. 3ème, 13 janvier 1964 Inf. Rap. Copropr. mai 1964 p.376.}.
			
			Les règlements de copropriété contiennent tous une clause selon laquelle les appartements ne pourront être occupés que par des personnes de bonnes vie et mœurs : le copropriétaire doit, en vertu de ces dispositions, être condamné à faire cesser les troubles provenant de l'utilisation d'un studio par son locataire << en un lieu que la morale réprouve >>\footnote{Paris 8ème ch. 8 mars 1963, D.1963 Som.74}.
			
			De la même manière, la clause stipulant dans un immeuble commercial que les commerces exploités devront obligatoirement être "propres et de bon ton" interdit d'exploiter une sex-shop ou un porno-shop\footnote{Paris, 22 juin 1978, D.1980 I.R. 235, obs. GIVERDON Même si, par application de l’article 290 Quater C.G.I. le sex-shop est un Etablissement de Spectacles comme un autre (tenu à ce titre de délivrer un billet à chaque spectateur avant son entrée dans la salle (CRIM 20 août 1994).}.
			Dans tous ces cas, la simple constatation d'un trouble, même normal, mais contraire aux dispositions du règlement suffit à entraîner la condamnation à réparer.
			
			Cette condamnation, tout comme en matière de droit commun de la responsabilité contractuelle, peut consister dans la suppression de la cause du trouble (au besoin sous astreinte), accompagnée ou non de dommages et intérêts : cessation sous astreinte de l'activité de racolage, mise en place d'une isolation acoustique\footnote{Paris, 23 mars 1983, Gaz. Pal. 1983, 457}.
			
			Par ailleurs, les limitations à l'exercice du droit d'usage peuvent être valablement sanctionnées par une clause pénale.
		
	\subsection{Les restrictions résultant de clauses licites du règlement de copropriété}\label{07_II_B}
		
		A la différence des précédentes, ces clauses ne sont plus strictement motivées par des considérations tenant au voisinage, mais elles tendent à conférer à l'immeuble des caractéristiques propres et un type d'occupation ou d'activités déterminé dont les locaux pourront être le lieu.
		
		A cet égard, on rencontre dans les règlements des clauses de caractère plus ou moins général qui indiquent ou prohibent certaines utilisations des différentes parties de l'immeuble et qui constituent l'élément essentiel de la destination de celui-ci. Ces clauses ont directement ou indirectement pour effet de restreindre la liberté d'usage des parties privatives (\ref{07_II_B_1}).
		
		Le règlement peut également contenir des clauses restrictives ou prohibitives, licites si elles sont justifiées par la destination de l’immeuble (\ref{07_II_B_2}).
		
		\subsubsection{Les clauses indicatives}\label{07_II_B_1}
		
			\paragraph{Les clauses relatives a l’occupation de l’immeuble}
			
				Ces clauses participent à la détermination de la destination générale de l'immeuble : clause d’habitation bourgeoise, d’usage mixte habitation et professionnel, d’usage de bureau, \etc (cf. infra chapitre VI.). Elles ont pour effet de limiter les activités qu’il est possible d’exercer dans les lots principaux
				\begin{itemize}
					\item la simple clause d’habitation bourgeoise est compatible avec une activité professionnelle libérale,
					\item la clause d’habitation bourgeoise exclusive interdit toute activité,
					\item la clause d’usage mixte affecte le plus souvent le rez de chaussée au commerce, à l’exclusion des étages.
				\end{itemize}
				
				Toutefois, ces clauses ne peuvent interdire la domiciliation d’une entreprise (cf supra)
				
			\paragraph{Les clauses de réglementation des commerces}
				
				Par exemple, on trouve fréquemment dans les règlements une clause n'autorisant l'exploitation de commerces qu'au rez-de-chaussée de l'immeuble. Si l'immeuble est un centre commercial, le règlement peut prévoir que les commerces installés originairement ne pourront pas être modifiés.
				
				En revanche, les clauses de non concurrence sont illicites, car elles ne peuvent être justifiées par la destination de l’immeuble.
				
			\paragraph{Les clauses concernant l'affectation des parties privatives }
				
				Outre les clauses susvisées qui concernent la destination générale de l'immeuble (immeuble d'habitation mixte, commercial \etc), il en est d'autres qui précisent l'affectation de chaque local privatif : grenier, chambre de service, garage, emplacement de stationnement, local d'habitation, boutique, restaurant \etc
				
				Ces indications relèvent normalement de l'état descriptif de division mais elles peuvent être reprises dans le règlement de copropriété. Leur valeur obligatoire a fait l'objet de nombreuses controverses et d'une évolution spectaculaire de la jurisprudence qui seront évoquées au chapitre suivant. Indiquons à présent que l'affectation des parties privatives n'est pas fixée de façon intangible et qu'une modification est aujourd'hui admise.
				
				Si l'indication de l'affectation d'un local commercial est établie de façon précise par la fixation d'un commerce déterminé --- librairie, salon de coiffure --- il ne semble pas non plus que cette affectation soit intangible, au moins dans une certaine mesure.
		
		\subsubsection{Les clauses prohibitives}\label{07_II_B_2}
		
			Les clauses prohibitives, c'est-à-dire celles qui interdisent tel ou tel usage des parties privatives, ne peuvent avoir d'effet qu'autant qu'elles sont licites. Elles font partie de ces restrictions aux droits des copropriétaires qui, selon l'article 8 al.2 de la loi, doivent être justifiées par la destination de l'immeuble.
			On rencontre surtout de telles clauses en matière d'activités professionnelles ou commerciales.
			
			\paragraph{Locaux professionnels}
			
				Ces clauses visent à concilier au sein d'un immeuble les conditions d'habitation de certains copropriétaires et les impératifs professionnels d'autres occupants. Il en est ainsi des clauses interdisant les activités qui sont de nature à gêner les voisins par le bruit : professeur de chant, de musique, de danse, de gymnastique.
			
			\paragraph{Locaux commerciaux}
			
				On relève dans les règlements de copropriété de nombreuses limitations aux activités commerciales dans les locaux privatifs.
				\begin{itemize}
					\item Tout d'abord, le règlement peut exclure l'exploitation de certains commerces précisément nommés : commerce en meublé ou garni (T.G.I. Paris, 16 mai 1976, D.1976 I.R. 315), exploitation d'une laverie (Caen, 24 mai 1965, Gaz. Pal. 1965, 2, 370, note CABANAC), d'une boîte de nuit (Civ. 3ème, 20 juin 1973, J.C.P. 1973 II 17504, note GUILLOT).
					\item L'exclusion peut ensuite résulter des nuisances provoquées par certains commerces. C'est ainsi que de nombreux règlements interdisent les commerces << malodorants, insalubres ou dangereux >>. 
				\end{itemize}
				
				Cette clause est appréciée de façon assez libérale par les tribunaux. L'exploitation d'une blanchisserie automatique (Civ. lère, 10 février 1964, A.J.P.I. 1964 II 763) ou d'une boucherie (T.G.I. Seine, 9 juillet 1963, A.J.P.I. 1964 II 298) peuvent être compatibles avec cette clause, mais non l’exploitation d’une blanchisserie-teinturerie (Civ. 3ème, 18 février 1987, Revue des Loyers 1987, 221).
		
				Mais pour éviter tout risque de conflit, le copropriétaire qui veut exercer une activité commerciale dans son lot peut demander à l'assemblée générale d'une part de constater, à la majorité de l'article 24, qu'une telle activité est conforme à la destination de l'immeuble et aux clauses du règlement, et d'autre part l'autorisation d'exploiter le commerce envisagé. Après l'expiration du délai de deux mois pour contester une telle décision, celle-ci ne pourra plus être remise en cause.
				
				L'interdiction en matière commerciale contenue dans le règlement de copropriété ne doit pas être discriminatoire, c'est-à-dire ne s'appliquer qu'à un seul lot et non pas à l'ensemble des copropriétaires (Paris, 22 janvier 1968, A.J.P.I. 1969 p.24, note BOUYEURE).
				
				Les différentes clauses évoquées ici n'interdisent pas positivement mais fixent limitativement le domaine au sein duquel des activités commerciales ou professionnelles peuvent se déployer. Hors des limites ainsi fixées, l'usage des lots devient illicite\footnote{Il incombe au notaire de se renseigner sur l’activité qu’un acquéreur profane en droit entend exercer dans les locaux acquis et en cas de clause restrictive d’attirer l’attention de cet acquéreur sur ces restrictions contractuelles (Cour d'Appel Orléans 28 juin 2008 \no 07/01300 inédit).}.
		
	\subsection{Le changement d'affectation des parties privatives}\label{07_II_C}
			
		Pour améliorer leurs conditions d'habitat, rentabiliser leurs locaux ou étendre leurs activités professionnelles ou commerciales, certains copropriétaires souhaitent modifier l'affectation des parties privatives de leur lot, telle qu'elle avait été établie originairement dans le règlement de copropriété.
		
		Ils veulent, par exemple, transformer un local d'habitation en un local commercial dans un immeuble qui est déjà à destination mixte ou transformer un grenier en appartement dans un immeuble à usage principal d'habitation.
		
		L'hypothèse est que cette transformation respecte au moins apparemment la destination générale de l'immeuble et qu'aucune restriction licite contenue dans le règlement de copropriété ne s'y oppose. Si tel n'était pas le cas, une autorisation donnée à l'unanimité devrait être obtenue.
		
		A ce sujet il convient de bien distinguer les deux notions d’affectation et de destination. Si les notions d’affectation et de destination sont étymologiquement synonymes, elles font l’objet d’une approche différente, sinon opposée, dans les textes régissant le droit de la copropriété.
		
		La destination de l’immeuble s’attache au bâtiment dans son entier ; elle ne s’apprécie pas local par local mais sur l’ensemble de l’immeuble. Ainsi l’ensemble de l’immeuble a-t-il une destination d’habitation, commerciale ou professionnelle. L’immeuble peut par ailleurs avoir une destination mixte telle que la destination d’habitation d’une part et une destination commerciale d’autre part. Sur ce point, il est fréquent que les rez-de-chaussée de copropriété soient destinés à un usage commercial tandis que les étages sont destinés à un usage d’habitation.

		Les dispositions du règlement de copropriété, notamment celles concernant l’affectation des parties privatives (boutiques, magasins, bureaux…) permettent de déduire la destination générale de l’immeuble (habitation, professionnelle, commerciale, mixte).
		
		Néanmoins, alors que la destination de l’immeuble est en principe « figée », la jurisprudence a fini par admettre, au contraire, une certaine liberté dans le changement d’affectation des parties privatives.
		
		\subsubsection{Les données du problème.}
		
			Il faut d'abord souligner que le Législateur n'est pas directement intervenu sur cette question. En revanche, la loi du 10 juillet 1965 contient des dispositions qui peuvent être utilisées en sens contraire,
			\begin{itemize}
				\item soit pour justifier l'intangibilité de l'affectation fixée originairement par le règlement,
				\item soit, à l'inverse, pour légitimer, dans certaines limites, des possibilités de modification.
			\end{itemize}
			
			\paragraph{Arguments en faveur du maintien de l'affectation fixée originairement}
			
				Le premier principe avancé pour imposer l'intangibilité de l'affectation des parties privatives est offert par l'article 8 al.ler de la loi.
				\begin{quote}
					<< {\itshape Un règlement conventionnel de copropriété détermine la destination des parties privatives ainsi que les conditions de leur jouissance.} >>
				\end{quote}
				
				On peut en déduire que l'affectation de ces parties est dotée de la force obligatoire des conventions (art.1134 du Code civil), lesquelles ne peuvent en principe être modifiées qu'avec l'accord unanime des contractants.
				
				Cette interdiction de principe pour un copropriétaire de modifier son lot est par ailleurs de nature à éliminer le risque d'une atteinte progressive et sournoise à la destination générale de l'immeuble qui de mixte, pourra devenir, par transformation individuelle et successive de locaux d'habitation en locaux commerciaux, un immeuble purement commercial et perdre ainsi son caractère initial.
				
				De tels changements engendreraient par ailleurs des bouleversements dans les conditions d'occupation originaires déterminées par le règlement de copropriété.
			
			\paragraph{Arguments en faveur d'une modification de l'affectation des parties privatives}

				Il existe, pour combattre les arguments en faveur de l'interdiction de tout changement de l'affectation des parties privatives, d'autres arguments de textes dont la valeur doit être prise en considération.
				
				Tout d'abord, l'article 9 de la loi édicte que :
				\begin{quote}
					<< {\itshape Chaque copropriétaire use et jouit librement des parties privatives \lips à condition de ne porter atteinte ni aux droits des autres copropriétaires, ni à la destination de l'immeuble.} >>
				\end{quote}
				
				Par ailleurs, le souci du Législateur de préserver au maximum les prérogatives de chaque copropriétaire sur son lot apparaît aussi dans l'alinéa 2 de l'article 8 selon lequel :
				\begin{quote}
					<< {\itshape Le Règlement de Copropriété ne peut imposer aucune restriction aux droits des copropriétaires en dehors de celles qui seraient justifiées par la destination de l'immeuble, telle qu'elle est définie aux actes, par ses caractères ou sa situation.} >>
				\end{quote}
				
				En d'autres termes, toute restriction aux droits des copropriétaires contenue dans le règlement doit être dépourvue d'effets si elle n'est pas justifiée par la destination de l'immeuble.
				
				Tant que la destination de l'immeuble n'est pas affectée, les droits de chacun, notamment celui de modifier l'affectation d'un local privatif, peuvent se déployer librement. C'est dire si cette notion de destination de l'immeuble est appelée à jouer un rôle capital pour la solution du problème.
				
				A ce fondement textuel, on ajoute généralement que la reconnaissance du droit de modifier le lot dans le respect de la destination de l'immeuble permet de faire évoluer la copropriété selon des facteurs qui sont apparus depuis sa création et permet d'éviter une "glaciation" des parties privatives prélude à une fossilisation du bien immobilier dans son ensemble.
		
		\subsubsection{Évolution de la jurisprudence.}
		
			Le sujet a donné lieu à de vives controverses doctrinales, et aujourd’hui encore toutes les analyses ne convergent pas parfaitement\footnote{Cf. essentiellement AUBERT Le changement d'affectation d'un lot en copropriété, Administrer juillet 1992 p. 2; ATIAS, Pour servir à la pratique du changement de la destination des parties privatives, JCP 1987, Ed N.I.375; KISCHINEWSKY-BROQUISSE, Destination de l'immeuble et affectation des parties privatives RD imm. 17 (3) juill -sep 1995 p. 421.}. La jurisprudence a beaucoup varié, hésité, jusqu'en 1986, date à laquelle elle a trouvé une certaine unité.

			\paragraph{Jurisprudence antérieure à 1986}
			
				Dans une première phase, assez permissive, la Cour de cassation avait décidé que la modification de la destination des parties privatives pouvait valablement résulter d'une décision de l'assemblée générale prise à la majorité de l'article 25 dès lors que ni la destination de l'immeuble, ni les parties privatives des autres lots n'étaient affectés\footnote{Civ. 3ème, 11 février 1975, Bull. Civ. III \no51, J.C.P. 1975 II 18084, obs. GUILLOT, Civ. 3ème, 29 novembre 1977, J.C.P. 1978 Ed. not. II p.185, note ATIAS, Rep. DEFRENOIS 1978, art.31808 \no48, obs. H.SOULEAU).}.
				
				Dans une deuxième phase, la Cour de cassation, faisant primer le principe de la force obligatoire du contrat qu'est le règlement de copropriété a jugé que l'affectation des parties privatives expressément et clairement déterminée dans le règlement ne pouvait être modifiée, sauf décision unanime de l'assemblée\footnote{Civ. 3ème, 2 octobre 1979, D.1980, I.R. 233, J.C.P. 1979 II 19289, note GUILLOT; Civ. 3ème, 16 octobre 1979, J.C.P. 1980 II 12282; Civ. 3ème, 30 octobre 1984, Bull. Civ. III \no179, Rep. DEFRENOIS 1985, art.33543 \no52, obs. H.SOULEAU}.
			
			\paragraph{Les 3 arrêts de principe du 10 décembre 1986}
			
				Dans trois arrêts rendus par la troisième chambre civile de la Cour de cassation le 10 décembre 1986, le droit pour un copropriétaire de modifier unilatéralement l'affectation donnée par le règlement de copropriété aux parties privatives de son lot a été reconnu à la condition que soient respectés :
				\begin{itemize}
					\item la destination générale de l'immeuble (habitation, commerce, ou mixte) ;
					\item les droits des autres copropriétaires ;
					\item les clauses restrictives licites contenues dans le règlement de copropriété.
				\end{itemize}
				
				Si le changement d'usage ou d'affectation s'avère compatible avec ces trois exigences, aucun obstacle d'ordre juridique tiré du droit de la copropriété ne peut être élevé à l'encontre d'une telle opération.
				
				\subparagraph{L'arrêt LA ROTONDE. Cinéma La Rotonde, Bull. Civ. III \no179, D.1987, 146, note GIVERDON, Rep. DEFRENOIS 1988 art.34202 \no22, obs H. SOULEAU}
				
				Le premier arrêt a ainsi admis la transformation d'une salle de spectacle en restaurant, alors que le règlement interdisait une telle modification : << {\itshape un changement de la nature de l'activité commerciale, dans un lot où le règlement de copropriété autorise l'exercice du commerce, n'implique pas, par lui même, une modification de la destination de l'immeuble et peut s'effectuer librement, sous réserve de ne porter atteinte ni aux droits des autres copropriétaires, ni à des limitations conventionnelles justifiées par la destination de l'immeuble }>>.
				
				\subparagraph{L'arrêt HUA c/ Syndicat avenue d'Ivry. (Epoux HUA, Administrer, mars 1987, p.31, obs. GUILLOT, Rep. DEFRENOIS 1988, art.34202, obs H. SOULEAU)}
				
				Dans cet arrêt a été admise la transformation d'un lot affecté par le règlement à l'usage exclusif de station-service en une boutique de produits alimentaires et d'objets exotiques.
				
				\subparagraph{L'arrêt Alexandre DUMAS. (Résidence A.DUMAS, Bull. Civ. III \no180, Rep. DEFRENOIS 1988, art.34202, obs. H.SOULEAU)}
				
				Dans le troisième arrêt la Cour de cassation a rejeté le pourvoi formé contre un arrêt d'appel ayant admis qu'un copropriétaire pouvait librement équiper et aménager des locaux dénommés << grenier >> dans l'état descriptif de division en locaux destinés à être habités, dès lors que la destination de l'immeuble est à usage principal d'habitation.
			
			\paragraph{L’évolution postérieure}
			
				Depuis ces trois arrêts, la jurisprudence semblait s'être fixée : elle refusait la thèse contractuelle de l’état descriptif de division en admettant la possibilité pour un copropriétaire de transformer l'affectation des parties privatives, dès lors que sont respectés la destination de l'immeuble, les droits des autres copropriétaires et les limitations conventionnelles.
				
				Elle écartait sans autre forme de discussion les clauses limitant le droit à changement d'affectation dès lors que ces clauses ne lui paraissent pas justifiées par la destination de l'immeuble.
				
				\subparagraph{Civ 3\degres 8 juillet 1987 rue Campagne Première}
				
				Cassation d’un arrêt ayant refusé la transformation en local d'habitation d'un local dénommé atelier par l’État descriptif de division, alors que le Règlement de Copropriété stipulait que les lots à usage d'atelier devaient conserver cette affectation. Sans s'arrêter aux termes du Règlement de Copropriété le juge du fond devait rechercher << en quoi la transformation à usage d'habitation du lot litigieux était contraire à la destination de l'immeuble lui-même et portait atteinte aux droits des autres copropriétaires >>.
				
				\subparagraph{Civ 3\degres 25 janvier 1995 Loyers et Copropriété 1995 \no 286}
				
				Pour accueillir la demande de copropriétaires tendant à la remise de locaux en leur état d'origine, l'arrêt retient que le Règlement de Copropriété ayant défini l'utilisation de chaque lot, notamment pour les caves et les garages, les propriétaires de ces lots doivent s'y conformer et, en cas de non respect de leurs obligations, réparer le préjudice causé par le paiement de dommages-intérêts et restituer aux lieux la destination d'origine.
	
				En statuant ainsi, sans rechercher en quoi les aménagements réalisés portaient atteinte aux droits des autres copropriétaires ou à la destination de l'immeuble, la cour d'appel n'a pas donné de base légale à sa décision de ce chef.
			
			\paragraph{Tempéraments au principe l’arrêt du 6 juillet 2017}
			
				\subparagraph{L’état descriptif de division peut être contractualisé par le règlement de copropriété}
				
				Un arrêt du 6 juillet 2017 de la cour de cassation\footnote{3\degres civ., 6 juill. 2017, \no 16-16849, JCP N 2017, \no29, act. 729 et JCP N 2017 Chr 1347 Droit des Biens Hugues PERINET Marquet} semble constituer un important revirement quant à la valeur de l’état descriptif de division … et un retour en arrière sur la « liberté » d’affectation des lots.
				
				Dans le cas d’espèce le Règlement de copropriété stipulait que l'immeuble était destiné à un usage professionnel de bureaux commerciaux ou d'habitation en ce qui concernait les locaux situés aux étages et combles, sans autre précision. Par contre l’état descriptif de division définissait les lots du Rez-de-chaussée comme étant à usage professionnel ou commercial et les lots à partir du 2ème étage comme étant à usage d’habitation. Essentiellement le Règlement de copropriété stipulait : « {\itshape L'état descriptif de division, ci-inclus, dont chaque copropriétaire a eu connaissance et accepté les termes, a même valeur contractuelle que le règlement lui-même : il détermine l'affectation particulière de chaque lot dépendant du groupe de bâtiments que son propriétaire s'oblige à respecter} ».
				
				La Cour de cassation dans cet arrêt constate que l’état descriptif de division a reçu valeur contractuelle de par ces dispositions du Règlement de copropriété et que les mentions de l’état descriptif de division n’étant pas en contradiction avec les stipulations contractuelles, les copropriétaires devaient respecter l’affectation prévue par l’état descriptif de division.
				
				Bien évidemment, en l’absence de cette « contractualisation » de l’état descriptif de division la jurisprudence précédente continuera de s’appliquer, comme le note M Périnet Marquet dans son commentaire de cet arrêt.
				
				\subparagraph{Une décision d’assemblée générale devenue définitive peut prohiber un changement d’affectation du lot}
				
				Le fait, pour un copropriétaire, de demander l'autorisation des autres copropriétaires pour exercer une nouvelle activité dans les parties privatives de son lot, l'oblige à respecter la décision de l'assemblée générale, quand bien même il n'avait pas à solliciter une telle autorisation ; aussi, le refus opposé par l'assemblée générale , une fois définitive, fait obstacle à l'exercice de cette nouvelle activité. En l'espèce, un établissement public de santé avait sollicité de l'assemblée générale des copropriétaires l'autorisation de changer l'affectation de ses lots de commerces en hôpital de jour ; le syndicat des copropriétaires, se prévalant du refus de cette assemblée, l'avait assigné en cessation de cette nouvelle activité . Mais l'établissement public s'était, de sa propre initiative, assujetti à l'accord des copropriétaires pour exercer son activité d'hôpital de jour et que la décision de l'assemblée générale refusant ce changement d'affectation n'avait pas été contestée et devenue définitive, elle s'imposait au propriétaire du lot
				(Cass. civ. 3, 8 juin 2017, \no 16-16.566, FS-P+B)
			
		\subsubsection{Analyse des décisions publiées}
			
			a. Transformation d'un lot d'habitation ou mixte en lot commercial ou en local d’activité
			
				UNE TELLE TRANSFORMATION N'EST CONCEVABLE QUE DANS UN IMMEUBLE ADMETTANT DEJA L’EXERCICE DE COMMERCES OU D’ACTIVITES
				
				Par conséquent, un local commercial ne peut être créé dans un immeuble à usage d'habitation et professionnel (Civ 3\degres 28 avril 1993 Loyers et Copropriété juillet 1993 \no 273)
				
				Dans un immeuble à destination habitation et commerces, la création d’un lieu de culte a été jugée
				- tantôt contraire à la destination de l’immeuble (Paris 23\degres Chambre 12 juin 1992; Loyers et Copropriété oct. 92 \no 398),
				- tantôt conforme à celle-ci, le règlement de copropriété ne comportant pas de restriction, autre qu’industrielles, à l’affectation des locaux (Civ 8 mars 1995, Loyers et Copropriété 1995 \no 287)
				De même, a été admise la transformation d’un commerce en locaux politiques dans un immeuble à usage mixte (Paris 1\degres Chambre 26 juin 1992 - Revue Droit Immobilier 92.537).
				
				PAR AILLEURS, S'IL EXISTE DES CLAUSES CANTONNANT L'EXERCICE DES COMMERCES A CERTAINES PARTIES DE L'IMMEUBLE (REZ-DE-CHAUSSEE PAR EXEMPLE), ELLES DEVRONT ETRE RESPECTEES.
				
				C'est ainsi que des locaux d'habitation situés ailleurs que dans les parties réservées aux activités commerciales, par exemple dans les étages, ne pourront, sauf décision unanime, être utilisés à des fins commerciales (Paris 23ème ch. 16 mai 1986, Rev. Droit Immobilier 1986 p.388).
				En effet la transformation de locaux d'habitation en locaux commerciaux bouleverse radicalement les conditions d'occupation originairement établies. Le voisinage et l'activité d'un commerce sont en effet fort différents de ceux d'un local à usage d'habitation (Civ. 3ème, 4 janvier 1990, Loyers et Copropriété mars 1990 \no140).
	
				En revanche, un appartement ou un lot mixte situé au niveau de l'immeuble affecté aux activités commerciales peut librement être transformé en local commercial, cette transformation étant conforme à la destination générale de l'immeuble (Paris 23ème ch.B, 19 mai 1989, Loyers et Copropriété juillet 89 \no344; Civ. 3ème, 3 décembre 1989, D.1989, I.R. 322).
				De même lorsque le Règlement de Copropriété précise que le lot est à destination mixte, l'une des affectations prévues pourra être étendue à l'ensemble du lot (Paris 23\degres Ch 19 mai 1989 D. 1990 IR 125 et Civ 3\degres, 6 dec 1989, Administrer pct 1990 p 49 pour l'extension de l'activité commerciale à l'ensemble du lot alors que le Règlement de Copropriété prévoyait une affectation mixte habitation et commerciale).
				Cependant, la disparition de locaux d'habitation pourrait contrevenir aux dispositions générales relatives à l'habitat. (article 631-7 C.C.H.).
			
			b. Modification de la nature du commerce exercé dans un lot affecté à une activité commerciale déterminée
			
				La stipulation de la nature du commerce qui peut être exploité dans un lot n'empêche pas le copropriétaire d'y exercer un commerce différent qui ne porte atteinte
				o ni à la destination de l'immeuble,
				o ni aux droits des autres copropriétaires
				o ni aux stipulations du règlement de copropriété justifiées par la destination de l’immeuble.
				Ainsi ont été validés les changements d’affectation suivants :
				o Modification d’une salle de spectacle en commerce (La rotonde)
				o Transformation d'un garage en commerce de produits exotiques (Hua) ou en local d'enseignement (Civ 3\degres 27 nov. 1991, AJPI avril 92 p. 278).
				o Création d'un restaurant dans un immeuble dont le Règlement de Copropriété autorise toute affectation sauf industrielle (Paris 23\degres Chambre 18 décembre 1991; Loyers et Copropriété mars 1992 \no 129)
				o De même le locataire de locaux situés dans un immeuble en copropriété peut exploiter un restaurant à la place d'une boulangerie dès lors que cette activité est conforme à l’usage mixte des locaux et qu'elle ne trouble pas les ne nuit pas aux droits des autres copropriétaires. Cass. 3e civ. 24 mars 2015 \no 13-25.528 (\no 366 F-D), Sté Ziah c/ Sté
				En revanche un restaurant ne peut être ouvert alors que le Règlement de Copropriété interdit les commerces gênants (Civ 3\degres 18 décembre 1991, Loyers et Copropriété février 1992 \no 82).
			
			c. Modification de l'affectation de lots accessoires : caves, greniers, garages etc...

				Tel copropriétaire, voulant valoriser son patrimoine immobilier ou le rendre plus adapté à ses besoins, désire transformer un grenier en studio, une cave en annexe de local commercial.
				Depuis le revirement jurisprudentiel de 1986, il peut opérer ce changement à la triple condition, déjà connue, de ne contrevenir ni à la destination de l'immeuble, ni aux droits des autres copropriétaires, ni aux clauses restrictives ou prohibitives licites et justifiées par la destination de l'immeuble contenues dans le règlement de copropriété.
				AINSI, LA COUR DE CASSATION A-T-ELLE ADMIS QUE DANS UN IMMEUBLE D'HABITATION :
				o un grenier pouvait être transformé en logement (Civ. 3ème, 10 décembre 1986 précité),
				o un atelier à usage de remise en logement (Civ. 3ème, 8 juillet 1987, Bull. Civ. III \no141 PARIS 23\degres CHAMBRE 7 DECEMBRE 1992; LOYERS ET COPROPRIETE AVRIL 1993 \no 146).
				o une Cave en appartement (Paris 19\degres Chambre, 27 février 1992, \no 221)
				o une remise en logement (Civ 3\degres 18 décembre 1991, Loyers et Copropriété 92 \no 80).
				DE LA MEME MANIERE, DANS UN IMMEUBLE A USAGE COMMERCIAL,
				o une cave peut être transformée en annexe d'un local commercial. (Civ 3\degres 21 oct. 1992 Loyers et Copropriété janvier 1993 \no 34).
				o une remise en local annexe d'un commerce (Paris 7\degres Chambre 14 novembre 1990, Loyers et Copropriété janvier 1991 \no 36).
				MAIS CE CHANGEMENT NE RESPECTE PAS TOUJOURS LES TROIS CONDITIONS POSEES PAR LA JURISPRUDENCE
				o Si un règlement stipule que seuls les locaux du rez-de-chaussée peuvent être affectés à des fins commerciales, un copropriétaire ne pourra affecter une cave à cet usage (Civ. 3ème, 18 février 1987, Administrer août septembre 1987, p.48, obs. GUILLOT).
				o la transformation d'un lot à usage de garage en cuisine porte atteinte aux droits des autres copropriétaires si elle entraîne des nuisances pour les voisins et porte atteinte au droit d'accès à la chaudière en tout temps et à toute heure des autres copropriétaires (Civ. 3ème, 4 janvier 1990, Loyers et Copropriété mars 1990 \no140).
				o Par ailleurs, dans un immeuble à destination bourgeoise, les copropriétaires doivent, lors de la création du lot, s’assurer que celui-ci respecte les critères de « décence », tels qu’ils résultent du Code de la Santé Publique ou du CCH : l’aménagement d’une cave à usage d’habitation, en violation des dispositions du Code de la construction et de l’Habitation, porte atteinte à la destination bourgeoise de l’immeuble.
				(Cour de cassation, Chambre civile 3, 6 septembre 2018, 17-22.172, Inédit, société Tsinga :
			
			d. Transformation d’un local commercial en local d’habitation

			La mode est aujourd’hui à la transformation de locaux commerciaux en locaux d’habitation. Elle est en principe jugée conforme à la destination de l’immeuble (Paris 23\degres Chambre A 26 mars 2003, Administrer nov 2003 p 43, note Bouyeure).
			Cependant, ces transformations ne sont pas toujours compatibles avec les trois conditions mises par la jurisprudence à leur licéité.
			Le respect des clauses restrictives licites du règlement et des droits des autres copropriétaires peut interdire certaines modifications projetées : Des "hobby-room" destinés à l'exercice d'un passe-temps ordinaire ne peuvent être transformés en locaux d'habitation si le règlement l'interdit et si la transformation est de nature à troubler la tranquillité des autres copropriétaires (Civ. 3ème, 27 avril 1988, Administrer avril 1989 p.43, Rep. DEFRENOIS 1989, art.34554, \no67, obs. H. SOULEAU).
			D. Modalités juridiques de l'opération : La question de l'intervention de l'assemblée générale.
			a. Intervention de l’Assemblée sur le changement d’affectation
			L'intervention de l'assemblée n'est pas une condition de validité du changement d'affectation, puisque ce changement peut être fait librement dans le respect de la destination de l’immeuble et des clauses licites du règlement de copropriété. Le copropriétaire, en l’absence de restriction dans le Règlement de copropriété, n’a pas à solliciter d’autorisation de l’assemblée générale ; toutefois s’il demande quand même cette autorisation et qu’elle lui est refusée, il devra respecter la décision de l’assemblée générale … sauf à en demander l’annulation au juge dans le délai de deux mois de la notification du procès-verbal298.
			Cependant un accord préalable de l'assemblée paraît souhaitable, le copropriétaire prenant des risques importants s'il effectue un changement d’affectation sans avoir consulté l'assemblée. La notion de destination de l'immeuble est trop floue pour qu'il puisse juger à coup sûr que les modifications sont compatibles avec celle-ci. La question sera facilement résolue dès lors que le changement d’usage nécessitera la réalisation de travaux touchant aux parties communes ou à l’aspect extérieur de l’immeuble : le copropriétaire ne demandera pas l’autorisation de changer l’affectation … mais l’autorisation de réaliser les travaux correspondants.
			Par souci de sécurité, le copropriétaire aura donc tout intérêt à saisir l'assemblée générale avant d'entreprendre ses travaux (Paris, 29 novembre 1989, D.1989, I.R. 122; CIV 3\degres 27 novembre 1991, Loyers et Copropriété 1992 \no 81).
			298 3\degres civ. 8 juin 2017, \no 16-16566, au Bulletin : Il s’agissait en l’espèce de transformer une activité commerciale en activité d’hôpital de jour alors que le Règlement de copropriété ne comportait pas de restriction d’usage et qu’aucune autorisation ne paraissait nécessaire, mais en demandant l’autorisation « le copropriétaire s'était, de sa propre initiative, assujetti à l'accord des copropriétaires pour exercer son activité d'hôpital de jour ».

			Celle-ci constatera cette compatibilité et manifestera ainsi son absence d'opposition au changement d'affectation ou, au contraire, estimera que le projet est de nature à porter atteinte à la destination de l'immeuble et refusera de donner son accord. Après l'expiration d'un délai de deux mois à compter de la notification de la résolution, celle-ci devient définitive.
			Mais au cours de ce délai, tout copropriétaire opposant ou absent peut attaquer une décision favorable en démontrant que le changement projeté était en réalité incompatible avec la destination de l'immeuble. Pendant ce même délai, le copropriétaire qui se serait heurté à un refus de l'assemblée pourrait démontrer que la décision est abusive parce que non fondée sur une atteinte à la destination de l'immeuble.
			Sur la majorité à obtenir, il apparaît que celle de l'article 24 soit suffisante puisqu'il ne s'agit que de constater la conformité de la modification avec la destination de l'immeuble et non pas d'autoriser une transformation que le requérant a en principe le droit d'effectuer. C'est la position qui a été adoptée par la Cour de Cassation qui, dans un arrêt du 27 novembre 1991 (Loyers et Copropriété février 1992 \no 81) a considéré que le changement d'affectation d'une partie privative pouvait faire l'objet d'un vote autrement qu'à l'unanimité.
			Attention, toutefois : en cas de refus non fondé, l’absence de contestation de la résolution par le copropriétaire demandeur lui interdira, pour l’avenir, le changement d’affectation du lot, quand bien même l’autorisation n’était pas requise (Cass. civ. 3, 8 juin 2017, \no 16-16.566, FS-P+B)
			b. Intervention de l’assemblée sur les travaux, si ceux-ci affectent l’aspect extérieur de l’immeuble ou les parties communes
			En revanche, si le changement d’affectation suppose des travaux modifiant l’aspect extérieur de l’immeuble ou les parties communes, l’autorisation préalable de l’assemblée générale sur ces travaux est indispensable. Peu importe que ces travaux soient rendus obligatoires pour l'exploitation du local; le copropriétaire doit solliciter l'autorisation préalable à leur réalisation.
			Cette autorisation doit être donnée à la majorité de l’article 25, mais ne peut être valablement refusée, sous peine d’engager la responsabilité du Syndicat des Copropriétaires, que si ces travaux portent atteinte à la solidité des parties communes ou à la destination de l’immeuble
			En cas de refus, le copropriétaire pourra solliciter une autorisation judiciaire de réaliser les travaux.
			Pour un ex, cf PARIS 23\degres 25 nov 1994 (21 rue Massenet).
			En l'espèce existe un local boutique avec un local d'habitation non érigé en un lot distinct décrit à l'État descriptif de division comme trois pièces (cuisine et deux chambres). Pas de restriction au Règlement de Copropriété quant à l'affectation des locaux commerciaux. L'acquéreur veut transformer l'ensemble en restaurant et doit faire des travaux quant à la sécurité (création d'une sortie de secours) et à la ventilation (appareil à poser en toiture). Le Règlement de Copropriété précise que l'autorisation de travaux peut être demandé au syndic. Le propriétaire obtient l'accord du syndic et réalise ces travaux. L’Assemblée Générale décide de remettre en cause ces travaux.

			La décision de la Cour de Paris :
			- Sur le changement d’affectation
			Sous réserve de ne pas porter atteinte aux droits des autres copropriétaires, les trois pièces constituant des locaux accessoires à la boutique, peuvent être librement utilisés comme annexe de l'activité commerciale exercée dans le lot à usage mixte dont elles font partie, dès lors que cette activité - soumise à aucune restriction conventionnelle - est compatible avec la destination de l'immeuble
			- Sur les travaux nécessaires au changement d'activité.
			Même si le Règlement de Copropriété autorise le syndic à donner un accord, cette clause ne peut avoir effet puisque seule l’Assemblée Générale peut être habilitée de par les termes de la loi.
			Peu importe que ces travaux soient rendus obligatoires pour l'exploitation du local; le copropriétaire doit solliciter l'autorisation préalable à leur réalisation
			- Sur la modification d'aspect de la vitrine.
			Quelle que soit la valeur esthétique de l'aménagement réalisé, il s'agit de travaux affectant la façade et à ce titre soumis à autorisation préalable
	
\section{Le droit d’usage des parties communes}

	Prises séparément des parties privatives, les parties communes peuvent s'envisager sous divers angles :
	\begin{itemize}
		\item ou bien comme des parties matérielles de l'immeuble affectées à l'usage de tous ou de certains,
		\item ou bien en tant que quote-part abstraite, attachée à une partie privative d'un lot.
	\end{itemize}
	Dans les deux cas, les droits du copropriétaire sur ces « parties communes » est limité par la situation d’indivision forcée et organisée dans laquelle il se trouve : il ne peut céder ces parties communes (\vref{07_III_A}), et son droit d’usage est limité par les droits concurrents des autres copropriétaires (\vref{07_III_B}). Il faut ajouter à cette prérogative celle d’exécuter des travaux sur les parties communes (\vref{07_III_C}) et le droit d’agir en justice (\vref{07_III_D}).
	
	\subsection{Le droit d'usage et de jouissance des parties communes : un droit indivis et limité}\label{07_III_A}
	
		L'article 9 alinéa ler de la loi du 10 juillet 1965 confère au copropriétaire le droit de jouir et d'user librement des parties communes sous la double condition de ne porter atteinte ni aux droits des autres copropriétaires, ni à la destination de l'immeuble.
		
		Le droit de jouissance de chacun est à la fois identique et concurrent à celui des autres : c'est la loi de l'indivision.
		
		Ce droit présente donc un contenu positif : l'exercice d'actes de maîtrise sur telle ou telle chose commune. Il est en revanche soumis à des limitations tenant à l'existence des autres copropriétaires et à la destination de l'immeuble.
		
		L'étendue et la consistance du droit d'usage sur les parties communes sont fonction de ce qui est nécessaire pour l'usage normal des parties privatives du lot.
		Chaque copropriétaire peut retirer de telle ou telle partie commune toutes les utilités dont elle est susceptible pour l'exercice de son droit sur la partie privative de son lot : ainsi, au cas où un lot est commercial, un couloir peut être utilisé en tant que lieu de passage pour les besoins du commerce (pour un commerce de charcuterie, Civ. 3ème, 31 janvier 1984, Rep. DEFRENOIS 1984, art.33242 \no54, obs. H.SOULEAU; Gaz. Pal. 1984 Pan. 185, Voir également PARIS 23\degres Ch. 27 mars 1991; Loyers et Copropriété 1991 \no 226).
		
		\subsubsection{Un droit d'usage des parties communes dans le respect de leur destination.}
		
			Cet usage ne peut être exercé qu'en respectant la nature ou la destination des parties communes :
			o une cour ne peut, sauf stipulation du règlement de copropriété, être utilisée en tant que parking (Civ 3\degres 13 octobre 1965; Gaz. Pal 1966, 1, 41),
			o un couloir en tant que lieu de dépôt même temporaire d'objets ou de marchandises (Civ III 5 avril 1968; Gaz. Pal 1968, 2, somm. 3)
		
		\subsubsection{L’interdiction d'annexer les parties communes.}
	
			A fortiori, un copropriétaire ne peut-il annexer à son profit telle ou telle parcelle de parties communes (cour, passage, couloir).
			Le Syndicat des Copropriétaires pourra contraindre le copropriétaire à restituer la partie commune indûment appropriée ainsi qu'obtenir la démolition des ouvrages réalisés qui matérialisent cette appropriation.
	
			L’action en revendication de la partie commune n’est prescrite qu’après 30 ans, le délai de l’article 42 al 1 de la loi (dix ans) étant écarté, car il ne concerne que les actions personnelles nées de l’application de la loi du 10 juillet 1965 Or, l’action tendant à faire cesser une appropriation par un copropriétaire d’une partie commune est une action réelle.
			Toutefois, il est parfois difficile d’établir la distinction entre une action aux fins de remise en état, action personnelle qui serait prescrite par 10 ans, et une action en revendication, la cour de cassation faisant une interprétation « extensive » de l’action en revendication.
			La Cour de cassation a affirmé récemment qu’en perçant une partie commune, les copropriétaires s’étaient « livrés à un acte de propriétaire qui par l’effet de l’usucapion était de nature à transférer la propriété de cette partie commune en une partie privative », en conséquence, la Cour de cassation a retenu que l’action tendant à obtenir la remise de la partie commune en « son état initial était une action réelle qui se prescrivait par trente ans » (Cass. civ. 3ème, 12 janvier 2010 \no 09-11.514).
			La jurisprudence antérieure avait déjà retenu concernant des installations d’évacuation et d’extraction des fumées qu’il s’agissait « d’un accaparement de parties communes par ce copropriétaire » et que « la Cour d’appel a retenu à bon droit que l’action du syndicat visant à obtenir la restitution de celles-ci [les parties communes] s’analysait en une action réelle soumise à la prescription trentenaire » (Cass. civ. 3ème, 4 novembre 2009, \no06-21.647), et avait également admis, selon les mêmes motifs, une action en démolition d’une cloison sur parties communes après plus de 10 ans, cf. Civ 3\degres 24 février 1993, Defrénois 1993. 1ère Partie \no 63 p. 777, note Aubert)
		
		\subsubsection{Les parties communes à jouissance privative.}
		
		Dans certains cas, l'usage d'une partie commune est réservé exclusivement à un ou plusieurs copropriétaires déterminés soit par le règlement de copropriété, soit par une décision de l'assemblée générale prise à la majorité de l'article 26 : place de stationnement affectée, parcelle de jardin, terrasse.
		Ces parties communes à usage privatif font dès lors l'objet d'une utilisation exclusive de la part du copropriétaire bénéficiaire du droit.
		Mais ce dernier n'a pas pour autant la maîtrise totale de cette partie. Il ne peut l'utiliser que conformément à la nature et à la destination de celle-ci : lui serait interdite la transformation d'un jardin d'agrément en parking ou d'une terrasse en annexe de son appartement. (cf. supra chapitre V)
		
	\subsection{Un droit d’usage s’exerçant en conformité avec le règlement de copropriété et les décisions d’assemblée générale}\label{07_III_B}
		
		\subsubsection{Les restrictions du Règlement de Copropriété quant aux modalités de jouissance des parties communes}
		
			\paragraph{Le règlement de copropriété peut, au chapitre des "conditions de jouissance des parties communes" fixer les modalités de jouissance des parties communes}
			
				Par exemple, le règlement de copropriété peut autoriser les véhicules automobiles à stationner dans la cour ou à y pénétrer pour leur chargement ou déchargement ; autoriser l'apposition d'enseignes, plaques ou panneaux publicitaires pour les lots commerciaux ou professionnels.
				
				De même, les dispositions du Règlement de Copropriété, surtout dans les Grands Ensembles, seront éventuellement complétées par un Règlement Intérieur, sorte de cahier des charges de l'usage des parties communes.
		
			\paragraph{Le Règlements de Copropriété comportent souvent des dispositions restreignant les droits des copropriétaires sur les parties communes.}
			
				Ces restrictions doivent être justifiées par la destination de l'immeuble :
				
				Par exemple, doit être considérée comme licite une clause faisant interdiction d'emprunter le hall d'accès avec des paniers ou cabas dès lors que l'on se trouve dans un immeuble de grand standing.
				
				Bien souvent également le Règlement de Copropriété limite l'accès des véhicules automobiles aux cours communes, qui ne pourra se faire par exemple que pour les déménagements ou emménagements ou pour une période de temps limitée aux besoins de chargement et de déchargement des véhicules (Sur ce sujet cf. J.D. LACHKAR : Réglementation de la circulation et du stationnement dans les parkings privés, in Administrer janvier 1990 p. 31).
			
			\paragraph{Le Règlement de Copropriété peut assortir ses prescriptions de sanctions qui constituent des clauses pénales}
			
				Ce sont le plus souvent des amendes forfaitaires par infraction constatée. Toutefois, ces clauses pénales sont soumises au pouvoir modérateur du juge, si elles sont manifestement excessive (art 1152 du Code Civil)
				
				Dans les Grands Ensembles en Copropriété le stationnement irrégulier est une véritable plaie. Certes, il existe une procédure légale prévue par la loi du 31 décembre 1970 dont l'article 3 est ainsi rédigé :
				\begin{quote}
					<< {\itshape A la demande du maître des lieux et sous sa responsabilité, peuvent être mis en fourrière, aliénés et éventuellement livrés à la destruction les véhicules laissés sans droit, dans les lieux publics ou privés où ne s'applique pas le code de la route} >>.
				\end{quote}
				
				Mais il s'agit là d'une procédure lourde et rarement efficace (sauf pour les véhicules ventouses).
				
				Aussi la pratique de la rédaction d'actes a t'elle prévu de sanctionner le stationnement abusif dans les Grands Ensembles, en insérant dans les Règlements de Copropriété dans le Règlement de Copropriété une clause pénale imposant le paiement d'une amende forfaitaire par infraction relevée. Le Syndicat fait alors assermenter le gardien de l'immeuble devant le Président du tribunal d'Instance. Le gardien relève les infractions. Le syndic obtient à la Préfecture l'identité du propriétaire du véhicule ... Il ne lui reste plus qu'à faire application de la clause pénale en réclamant au copropriétaire le paiement de l'amende forfaitaire.	
		
		\subsubsection{Les décisions d'assemblées générales modifiant les conditions de jouissance des parties communes}
		
			Les dispositions du règlement relatives aux modalités d'usage et de jouissance des parties communes peuvent être modifiées à la majorité des copropriétaires représentant les deux tiers des voix (art.26 b).
			
			Les copropriétaires devront en conséquence respecter les décisions d'Assemblée modifiant l'usage et la jouissance des parties communes.
			
			\begin{exemple}
				Par exemple, l'Assemblée Générale peut décider à la double majorité de l'article 26 de transformer un local à vélos en local à poubelles (PARIS 10 mars 1988 : Loyers et Copropriété 1988 \no 238).
				
				Elle pourra encore décider de transformer un jardin d'agrément en parking (LIMOGES 1er juillet 1966; D 67, 11).
			\end{exemple}
			
			De même l'Assemblée pourra aliéner des parties communes qui ne sont pas indispensables à la conservation de l'immeuble (article 26 dernier alinéa).
			
			Cette possibilité de modification --- ou d'aliénation --- se trouve toutefois limitée par l'interdiction qui est faite à l'assemblée d'imposer au copropriétaire une modification à ses parties privatives ou aux modalités de leur jouissance tel qu'il résulte du règlement de copropriété. On rencontre assez fréquemment le problème :
			\begin{itemize}
				\item \textbf{De la fermeture de la porte d'accès à la Cour de l'immeuble.}
				
				Constatant que l'ouverture permanente de cette porte entraîne des risques pour la sécurité des occupants, l'Assemblée décide qu'elle devra être fermée en permanence. Une telle décision risque d'entraîner un préjudice pour les copropriétaires commerçants dont l'activité s'exerce en fond de cour (cas de figure classique dans le quartier du Marais ou de la Bastille à Paris).
				
				\item \textbf{De la suppression des boites aux lettres}, dans un immeuble composé essentiellement de bureau, obligeant les copropriétaires à faire porter directement leur courrier par la poste, ce qui modifie les modalités de jouissance des lots.
				
				\item \textbf{De la suppression du service de la concierge}, laquelle constitue une modification des modalités de jouissance des lots dès lors que ce service figurait dans le règlement de copropriété
			\end{itemize}
		
	\subsection{Les travaux sur les parties communes}\label{07_III_C}
		
		%CF . POLYCOPIE \no2 (LES TRAVAUX).
		
		Les travaux sur parties communes peuvent être diligentés par le syndicat, après décision de l’Assemblée Générale, mais également par un copropriétaire individuel, également a près autorisation de l’assemblée générale.
		
		\subsubsection{Les travaux à l'initiative du Syndicat}
		
			En principe, les travaux exécutés sur les parties communes - qu'il s'agisse d'entretien ou d'amélioration relèvent de la compétence du syndicat.
			
			La décision d'entreprendre de tels travaux doit être prise en assemblée générale aux majorités prévues par la loi (art.24, 25, ou 26) et le financement doit être réparti entre tous les copropriétaires selon des proportions légalement précisées.
		
		\subsubsection{Travaux sur l'initiative d'un ou plusieurs copropriétaires}
		
			Mais il peut arriver qu'un copropriétaire désire accomplir à ses frais des aménagements impliquant des travaux affectant les parties communes ou l'aspect extérieur de l'immeuble : percement ou agrandissement de fenêtre dans un mur commun, branchement d'eau, de gaz ou d'électricité sur des réseaux ou canalisations communs, apposition de plaques ou enseignes sur des parties communes.
			
			Dans ce cas, l'opération est réalisable à condition que l'assemblée générale lui en accorde l'autorisation à la majorité de l'article 25, c'est-à-dire à la majorité absolue de tous les copropriétaires, avec possibilité d’un second vote.
			
			Mais le pouvoir de l'assemblée n'est pas discrétionnaire. Si elle refuse d'autoriser les travaux alors que ceux-ci sont conformes à la destination de l'immeuble, le copropriétaire peut exercer un recours devant le Tribunal de Grande Instance aux fins d'autorisation (art.30 al.4). Le juge pourra alors autoriser les travaux aux conditions qu'il fixera.
		
	\subsection{Le droit d’agir en justice (art. 15 alinéa 2 de la loi du 10 juillet 1965)}\label{07_III_D}
		
		A priori l'action individuelle d'un copropriétaire devrait se limiter à la sauvegarde des parties privatives de son lot, le syndicat ayant seul qualité pour agir au cas d'atteinte aux parties communes. Telle pourrait être l'interprétation de l'article 15 al.2 de la loi du 10 juillet 1965 qui édicte que :
		\begin{quote}
			<< {\itshape Tout copropriétaire peut exercer seul les actions concernant la propriété ou la jouissance de son lot.} >>
		\end{quote}
		
		Cependant la Cour de cassation n'a pas voulu restreindre le droit individuel d'accès à la justice au cas où le copropriétaire subi un préjudice personnel du fait d'une atteinte aux parties communes de l'immeuble.
		
		La position de la Cour de Cassation a été marquée par une évolution certaine\footnote{Cf : Troisième Partie, Chapitre I, Section III : LA CAPACITE D’ESTER EN JUSTICE DU SYNDICAT ET DES COPROPRIETAIRES}. En effet, après avoir longtemps exigé la démonstration de ce préjudice personnel pour reconnaitre l’intérêt à agir d’un copropriétaire, elle a fini par abandonner purement et simplement ce critère.
		
		\subsubsection{Action en sanction de la violation du règlement de copropriété}
		
			Le règlement de copropriété ayant valeur contractuelle, chaque copropriétaire a donc le droit d'en exiger le respect par les autres.
			
			L'action individuelle des copropriétaires est recevable sans qu'ils soient astreints à démontrer qu'ils subissent un préjudice personnel et spécial distinct de celui dont souffre la collectivité des membres du syndicat, leur intérêt à agir trouvant sa source dans le respect du règlement de copropriété\footnote{Cass. 3e civ., 22 mars 2000 : D. 2000, inf. rap., p. 115 – Cass. 3e civ., 17 nov. 2004, FS P + B, JCP N 2005, 1206, note A. Djigo ; AJDI 2005, p. 581, obs. Cl. Giverdon. – Cass. 3e civ., 13 sept. 2006}.
		
		\subsubsection{Action sanctionnant l’atteinte aux parties communes}
		
			Chaque copropriétaire a le droit d'exiger la cessation d'une atteinte aux parties communes par un autre copropriétaire, sans être astreint á démontrer qu'il subit un préjudice personnel et distinct de celui dont souffre la collectivité des membres du syndicat\footnote{Cass. 3e civ., 17 nov. 2004, FS P + B ; JCP N 2005, 1206, note A. Djigo ; AJDI 2005, p. 581, obs. Cl. Giverdon. – Cass. 3e civ., 13 sept. 2006 . Cass. 3e civ., 10 janv. 2001, \no 99-11.607, P+B : JurisData \no 2001-007713 ; Loyers et copr. 2001, comm. 105, G. Vigneron. – Cass. 3e civ., 5 juill. 2006, \no 05-14.579 : JurisData \no 2006-034721 ; Constr.-Urb. 2006, comm. 218, D. Sizaire. – Cass. 3e civ., 13 sept. 2006, \no 05-13.073 : JurisData \no 2006-034950}
	
			Ainsi, lorsque l'action du copropriétaire tend à la cessation d'une atteinte aux parties communes, elle était considéré comme recevable sans nécessité de preuve d'un préjudice personnel\footnote{
			Cass. 3e civ., 26 janv. 2017, \no 15-24.030, F-D (pourvoi c/ CA Saint-Denis, 22 mai 2015) : JurisData \no 2017-001112
			
			Tout copropriétaire peut exercer seul, à charge d'en informer le syndic, les actions concernant la propriété ou la jouissance de son lot, lequel comprend une partie privative et une quote-part de parties communes. Il peut par conséquent agir seul, contre un autre copropriétaire, en cas d'empiètement sur les parties communes, sans avoir à justifier de son intérêt à agir.}.
			
			Par un arrêt récent, non publié (Cass. 3e civ., 15 nov. 2018, \no 17-13.514), la Cour de cassation a, à rebours de sa jurisprudence traditionnelle, exigé la démonstration de l’intérêt personnel du copropriétaire alors que celui-ci agissait pour voir cesser une véritable annexion de parties communes :
			\begin{quote}
				{\itshape Attendu que, pour condamner la société Les Glaciers italiens à libérer un accès à la copropriété, ainsi que l’ensemble des parties communes obstruées, l’arrêt retient qu’en s’appropriant l’entrée de service de l’immeuble, qu’elle a englobée dans une terrasse à usage privatif, et en obstruant l’entrée principale en y entreposant des cartons, cette société a porté atteinte à la copropriété et, au regard du caractère absolu du droit de propriété, il est indifférent que Mme Y… ne subisse aucun préjudice, ou que les clés du local commercial lui ait été remises pour qu’elle puisse l’emprunter pour accéder au second étage ;
				
				Qu’en statuant ainsi, tout en constatant que Mme Y… ne justifiait pas d’un préjudice personnellement éprouvé dans la propriété ou la jouissance de ses lots et indépendant de celui subi par la collectivité des copropriétaires, la cour d’appel a violé les textes susvisés ;}
			\end{quote}\index{Jurisprudence!Glaciers italiens}
			
			Par ailleurs, la Cour de cassation semble maintenir son exigence relative à la preuve d’un préjudice distinct lorsque l’atteinte aux parties communes est le fait d’un tiers à la copropriété :
			\begin{quote}
				{\bfseries Cass. 3e civ., 22 sept. 2004 : Juris-Data \no 2004-024872 ; AJDI 2005, p. 141, obs. Cl. Giverdon ; Administrer juin 2005, p. 45, obs. J.-R. Bouyeure} {\itshape à propos de la démolition du mur partie commune par le propriétaire voisin. Ainsi que l'indique la Cour de cassation dans cette affaire, le copropriétaire dont les lots sont à l'opposé du mur démoli en partie, qui n'invoque qu'une atteinte au droit de propriété indivis des copropriétaires, sans prouver subir de préjudice propre dans la jouissance ou la propriété de ses parties privatives ou des parties communes du fait des travaux exécutés par le tiers à la copropriété, est irrecevable en sa demande individuelle de remise en état des lieux}
			\end{quote}
			
		\subsubsection{Nécessité d'information du syndic (article 51 du décret du 17 mars 1967)}
		
			Le copropriétaire qui exerce seul les actions concernant la propriété ou la jouissance de son lot, doit en informer le syndic par \lrar.
		
			La cour de Cassation, ajoutant au texte, exige même désormais que le syndicat soit attrait en la cause (cf Poly II).