\chapter{La cession du lot}

Comme tout bien susceptible d'appropriation privée, le lot peut faire l'objet d'actes de disposition et
notamment de cession : vente, échange, donation.

Ce droit est d'ailleurs consacré par l'article 9 de la loi du 10 juillet 1965 qui dispose :
\begin{quote}
	Chaque copropriétaire dispose des parties privatives comprises dans son lot.
\end{quote}

Ce texte comporte une légère inexactitude car la cession ne porte pas seulement sur les parties privatives
mais aussi sur la quote-part de parties communes y afférentes, c'est-à-dire sur la totalité du lot.

Le droit de céder s'exerce en principe librement. Cependant, des restrictions particulières au statut de la
copropriété viennent le limiter, si bien que la liberté de vendre un lot n'est pas aussi étendue que celle
d'aliéner tout autre bien immobilier. Par ailleurs, la cession du lot comporte des formalités spécifiques.

Enfin, les effets d'une telle cession suscitent des difficultés particulières qu'il conviendra d'examiner.

\section{La liberté de céder son lot}

	Si la liberté de céder son lot constitue le principe, des restrictions de plus en plus nombreuses sont venues
	la limiter, faisant ainsi apparaître la différence irréductible existant entre la situation d'un copropriétaire
	et celle d'un propriétaire.
	
	\subsection{Liberté de cession et restrictions issues du règlement de copropriété}
	
		Pendant longtemps, le caractère absolu de ce principe a été affirmé par la jurisprudence.
		
		Le fondement textuel d'une telle position était le suivant : l'article 9 al. 1\ier{} de la loi, dans une première
		proposition reconnaît au copropriétaire le droit de disposer sans restriction du lot alors que la deuxième
		phrase de ce texte soumet le droit d'en user ou d'en jouir à la double condition de ne porter atteinte ni
		aux droits des autres copropriétaires, ni à la destination de l'immeuble.
		
		La troisième chambre civile de la Cour de cassation a sur ce fondement décidé dans un arrêt très remarqué
		en date du 17 juillet 1972 que « la notion de destination de l'immeuble ne concerne que l'usage et la
		jouissance de l'immeuble ; elle ne saurait justifier une restriction au droit de disposer librement des
		lots\footnote{D.1972, 727, note E.F.; J.C.P. 1972, II, 17241, note E.J. GUILLOT, Rep. DEFRENOIS 1973 art.30293 \no 11, obs. H.	SOULEAU}. »
		
		Il résultait de cette jurisprudence que toute clause ou décision limitant le droit de disposer était par nature
		illicite. Mais, cette conception exagérément littérale (art.9 de la loi), et absolue s'est révélée artificielle et
		quelque fois nuisible dans les hypothèses où la limitation au droit de disposer se justifiait par la sauvegarde
		de la disposition de l'immeuble.
		
		On retrouve ici la notion de destination de l’immeuble : une atteinte au droit de libre cession peut être
		justifiée, dès lors que la destination de l’immeuble le justifie.
		
		C'est pourquoi, après avoir affirmé le 17 juillet 1972\footnote{Cf. \emph{supra}} que cette destination était une notion étrangère à
		la disposition des lots, la Cour de cassation a depuis un arrêt en date du 10 mars 1981 admis la validité
		d'une clause interdisant de vendre des chambres de service séparément des appartements dès lors qu'une
		telle interdiction se trouvait justifiée par la destination de l'immeuble.
		
		En l’espèce il s'agissait d'une copropriété de taille réduite, la destination de cet immeuble ne pouvait pas
		permettre une subdivision indéfinie des parties privatives; l'absence de lots accessoires tels que chambres
		de service, garages ou débarras entraînerait un encombrement des parties communes et gênerait
		l'utilisation normale de l'immeuble\footnote{Civ. 3ème 10 mars 1981, Bull. Civ. III \no52, Rep. DEFRENOIS 1981, art.32797, obs. H.SOULEAU}
		
		Il convient donc de distinguer entre les clauses licites au droit de disposer et les clauses illicites.
		
		\subsubsection{Restrictions licites au droit de disposer}
		
			Aujourd'hui sont donc reconnues valables les clauses du règlement de copropriété restreignant la liberté
			de disposer lorsqu'elles se justifient par la sauvegarde de la destination de l'immeuble.
			
			Si cette condition est remplie, sont donc considérées comme licites :
			\begin{description}
				\item[les clauses d'inaliénabilité temporaire] dans un immeuble comportant deux
				appartements construits pour couples amis, la clause interdisant la cession pendant un
				délai raisonnable pour permettre aux uns de tenter d'acquérir la part des autres est
				valable\footnote{exemple donné par CH. ATIAS, La Copropriété des Immeubles Bâtis, Sirey 1989,				\no63)} ;
				
				\item[les clauses interdisant la division des lots] dans les immeubles où la multiplication des
				unités familiales altérerait le caractère de l'habitat et serait incompatible avec les
				équipements et installations existants qui ont été conçus pour un nombre déterminé
				d'utilisateurs ;
				
				\item[les clauses de préemption ou pactes de préférence] conférant au copropriétaire originaire
				un droit de priorité pour l'acquisition d'un lot cédé par un membre du syndicat, ce n'est
				qu'au cas de refus des titulaires de ce droit que le vendeur recouvre la liberté de vendre à
				qui il veut ;
				
				\item[les clauses d'exclusivité] portant sur des locaux accessoires tels que caves, greniers,
				chambres de service ou garages interdisant de vendre séparément ces biens à d'autres
				personnes que des copropriétaires ;
				
				\item[les clauses d'indivisibilité] interdisant de vendre des locaux accessoires séparément de
				l'appartement principal\footnote{Cf. Civ. 3\ieme{}, 10 mars 1981 précité}.
			\end{description}
			
			Bien entendu, si ces clauses restrictives ne se justifient pas par la nécessité de la
			sauvegarde de la destination de l'immeuble, elles doivent être annulées.
			
		\subsubsection{Restrictions illicites au droit de disposer}
		
			En revanche, il existe des clauses qui par nature sont illicites et que ne peut valider le
			recours à la destination de l'immeuble : ce sont celles qui auraient pour résultat de rendre
			le copropriétaire prisonnier de son lot, c'est-à-dire mis dans l'impossibilité de vendre. Il
			s'agit :
			
			\begin{description}
				\item[des clauses d'inaliénabilité perpétuelle] ;
	
				\item[des clauses de préemption ou de préférence] portant sur les locaux principaux --- si
				aucun des copropriétaires ne désire acheter, le cédant est totalement privé du
				droit de disposer\footnote{CA Toulouse, 10 jan 2011 (Loyers et Copropriété Juin 2011 \no 187) ; Civ. 3\ieme{} 29 mai 1979 ; JCPN 1979, II, p. 237, note Lafond)} ;
				
				\item[des clauses d'agrément] qui imposent au copropriétaire cédant l'assentiment du
				syndicat sur la personne de l'acquéreur --- si l'assemblée refuse systématiquement
				tous les candidats qui lui sont présentés par le cédant, celui-ci se trouve dans
				l'impossibilité de vendre ;
				
				\item[de la clause faisant obligation au copropriétaire d’un lot de le louer] au syndicat
				pour être affecté au logement du gardien de l’immeuble\footnote{Paris 23\ieme{} Ch. 19 fév. 1997 Loyers et Copropriété 1997 \no 186}.
			\end{description}
	
	\subsection[Les droits de préemption]{Les droits de préemption : atteinte a la liberté de céder par substitution d’acquéreur}
		
		\subsubsection{Le droit de préemption au profit du locataire}\footnote{Sur l’ensemble des droits de préemption urbains et au profit des locataires, consulter la remarquable étude de M \nom{Casteran} et Mme \nom{Lambret-Borderie} au JCP, Ed N. 2012 – Études \no 1237}
		
			\paragraph{Droit de préemption du locataire ou occupant de bonne foi en cas de division de l’immeuble en vue de la vente par lots (Loi \no 1351 du 31 décembre 1975 modifiée en dernier lieu par la loi \no 526 du 22 juin 1982).}
			
				Ce texte a pour origine les abus de marchands de biens vis-à-vis des locataires de la loi de 48. Il s’applique
				lorsque le propriétaire unique d’un immeuble établit un État descriptif de division par lots et met en vente
				ces lots.
				
				\begin{quote}
					« Préalablement à la conclusion de toute vente d'un ou plusieurs locaux à usage d'habitation ou à usage
					mixte d'habitation et professionnel, consécutive à la division initiale ou à la subdivision de tout ou partie d'un immeuble par lots, le bailleur doit, à peine de nullité de la vente, faire connaître par lettre
					recommandée avec demande d'avis de réception, à chacun des locataires ou occupants de bonne foi, l'indication du prix et des conditions de la vente projetée pour le local qu'il occupe. Cette notification vaut offre de vente au profit de son destinataire ».
				\end{quote}
				
				\subparagraph{Condition de mise en œuvre du droit de préemption}
				
				Pour que ce texte s’applique trois conditions doivent être réunies :
				\begin{itemize}
					\item il doit y avoir vente\footnote{Il n’y a pas vente en cas de partage (même avec soulte), de cession de parts ou d’échange, ni a fortiori si la cession est faite à titre gratuit.} portant sur des locaux à usage d'habitation\footnote{Le droit existe en cas de vente d’un local accessoire à usage d’habitation (chambre de service) et même pour un garage accessoire d’un appartement en location.} ou à usage mixte d'habitation et professionnelle ;
					\item la vente doit être consécutive à la division\footnote{La publication de l'État descriptif de division préalablement à la vente n’est pas obligatoire : hypothèse de la promesse synallagmatique de vente avant publication.} de l’immeuble par lots de copropriété\footnote{Ce qui implique que l’immeuble soit collectif.} ;
					\item il doit s’agir de la première vente après division de l’immeuble\footnote{La Cour de Cassation considère que s’il existe deux bâtiments et que l’un de ces bâtiments est vendu « en bloc », le droit de préemption ne bénéficie pas aux locataires de ce bâtiment. 5 juillet 1995 (Bull. civ. 1995, III, \no 173 ; JCP G 1995, IV, 2199).}.
				\end{itemize}
				
				Ce droit de préemption n’est pas limité à la première vente d’un lot de l’immeuble divisé mais s’applique
				à la première vente de chacun des lots issus de la division de l’immeuble, en sorte que le droit de
				préemption peut s’exercer dans l’immeuble sur une longue période de temps. Il s’applique sur la vente
				d’un ou plusieurs lots (vente en bloc), mais le prix doit alors être ventilé par lot.
				
				\subparagraph{Bénéficiaire du droit de préemption}
				
				Le bénéficiaire est le locataire ou l’occupant de bonne foi (titulaire d’un bail « loi de 48 » bénéficiant du
				droit au maintien dans les lieux). Le bénéficiaire doit occuper effectivement les lieux.
				
				En cas de pluralité de titulaires chacun d’eux (époux par exemple) bénéficie de la procédure de
				préemption.
				
				\subparagraph{Procédure de préemption}
				
				La notification doit être faite par lettre recommandée avec demande d'avis de réception préalablement à
				la conclusion de la vente. Par cette notification le vendeur offre au locataire ou occupant de bonne foi la
				faculté d’acquérir le local qu’il occupe.
				
				L’offre précise le prix et les conditions de la vente et, à peine de nullité, reproduit les différents textes
				l'article 10, \I{} de la loi du 31 décembre 1975 qui développe le droit de préemption.
				
				Le locataire dispose d’un délai de deux mois pour accepter l’offre d’acquérir. La vente devra être réalisée
				à son profit dans le délai de deux mois de son acceptation (quatre mois si le locataire déclare recourir à un
				prêt).
				
				\subparagraph{Sanction en cas de non respect du droit de préemption}
				
				En cas de non-respect de la procédure, comme en cas de notification irrégulière, la nullité de la vente est
				encourue\footnote{Cass. 3e civ., 11 juin 1997 : Loyers et copr. 1997, comm. 251, obs. B. Vial-Pédroletti}. Elle doit être demandée par le locataire dans le délai de cinq ans de l’article 1304 du Code	civil.
				
				Par contre le locataire ne bénéficie pas d’un droit de substitution.
				
				Toutefois si la vente se fait pour un prix plus avantageux que le prix offert, il n’y a pas lieu à annulation de la vente mais en ce cas, le locataire bénéficiera d’un nouveau droit de préemption.
			
			\paragraph{Le droit de préemption du locataire en cas de congé pour	vendre (loi du 6 juillet 1989, article 15 \II)}
			
				\par Ce droit de préemption bénéficie au locataire « habitation » qui se voit notifié en fin de bail un congé pour vendre. Ce droit a été modifié en dernier lieu par la loi ALUR.
				
				Le bailleur n’a aucune obligation de donner congé pour vendre : il peut parfaitement vendre directement
				son bien occupé. Auquel cas le bail se poursuivra avec l’acquéreur $\dots$ qui pourra par exemple attendre
				l’issue du bail pour donner congé pour habiter lui-même, donc sans droit de préemption pour le locataire.
				
				Cependant, s’il étend vendre libre, le lot doit être offert en priorité au locataire qui occupe les lieux .
				
				\subparagraph{Conditions de mise en œuvre}
				
				\begin{itemize}
					\item Existence d’un bail soumis à la loi du 6 juillet 1989 (locaux d’habitation principale ou à usage mixte professionnel et d'habitation).
					
					Le bail doit être en cours, qu’il s’agisse d’un bail reconduit ou renouvelé. Si le bail a été résilié ou
					annulé, le locataire --– devenu simple occupant --– ne peut prétendre au bénéfice du droit de
					préemption ; il en ira de même si le locataire a préalablement donné congé.
					
					\item  La cession échappe au droit de préemption du locataire en cas de cession intervenant entre parents
					jusqu'au troisième degré inclus, sous la condition que l'acquéreur occupe le logement pendant une
					durée qui ne peut être inférieure à deux ans à compter de l'expiration du délai de préavis.
					
					\item  Le droit de préemption du locataire est écarté en cas de préemption par la commune ou vente à un
					OPHLM.
				\end{itemize}
				
				\subparagraph{Procédure de préemption}
				
				Le congé est délivré par lettre recommandée avec AR ou acte d’huissier, ou depuis la loi ALUR par a remise
				en main propre contre récépissé ou émargement, six mois au moins avant le terme du terme du bail\footnote{Si ce délai n’est pas respecté, le congé est nul.}.
				
				Le congé doit être motivé, comporter l’offre de vente (prix et modalités de paiement, commission de l’agent
				immobilier, conditions de la vente avec diagnostics) et reproduire les dispositions de l’article 15 alinéa 1 à
				5 de la loi de 1989, c'est-à-dire les dispositions qui exposent le mécanisme de ce droit de préemption.

				Par contre le vendeur n’a pas l’obligation de joindre le Règlement de copropriété au congé si la copropriété
				préexiste. Le vendeur n’a pas davantage l’obligation de joindre le mesurage « Carrez » à ce stade de la
				vente.
				
				Le congé vaut offre de vente, laquelle est « valable pendant les deux premiers mois du délai de préavis” >>.
				Elle doit donc être maintenue pendant toute la durée de ce délai. Aucune rétractation n'est possible de la
				part du bailleur.
				
				Si le locataire accepte l’offre, la vente est parfaite il ne peut pas se rétracter et le bailleur ne peut pas
				davantage révoquer son offre\footnote{Par contre le locataire disposera du délai de sept jours de la remise de l’acte authentique pour se rétracter (\article{L}{271-1} du \CCH).}. Son droit d’occupation est prorogé jusqu’à la réalisation de la vente, qui
				doit intervenir dans un délai préfix de deux mois à compter de l’acceptation (délai porté à 4 mois si le
				locataire fait connaître au bailleur on intention de souscrire un prêt).
				
				\subparagraph{Sanction}
				
				Si le congé pour vendre ne respecte pas les dispositions légales, il est nul : par exemple si le congé n’est
				pas motivé ou si le bailleur n’a pas l’intention réel de vendre. Pour autant le bailleur n’a pas à justifier qu’il a d’ores et déjà trouvé un autre acquéreur.
				
				Si le congé a été donné frauduleusement (prix artificiellement gonflé, local conservé par le bailleur puis
				reloué à un loyer supérieur) ou sans respect préjudiciable des obligations légales, le locataire peut obtenir
				des Dommages Intérêts. En outre, la loi ALUR \no 2014-366 du 24 mars 2014 a instauré une sanction pénale.
				
				L'article 15-\IV{} nouveau de la loi \no 89-462 du 6 juillet 1989 prévoit que :
				\begin{quote}
					Le fait pour un bailleur de délivrer un congé justifié frauduleusement par sa décision de reprendre ou de
					vendre le logement est puni d'une amende pénale dont le montant ne peut être supérieur à \montant{6 000} pour une personne physique et à \montant{30 000} pour une personne morale.
				\end{quote}
				
				\subparagraph{Droit de préemption subsidiaire} Si le propriétaire décide de vendre à un tiers à des conditions ou à un
				prix plus avantageux, ce prix ou ces conditions doivent être notifiés au locataire soit par le bailleur, soit
				par le notaire (si le bailleur n'y a pas préalablement procédé). Cette notification est imposée à peine de
				nullité de la vente. Effectuée soit à l'adresse indiquée par le locataire au bailleur, soit à l'adresse des locaux
				dont la location avait été consentie, elle vaut offre de vente au profit du locataire et est valable pendant
				une durée d'un mois à compter de sa réception.
			
			\paragraph{Le droit de préemption du locataire ou occupant de bonne foi en cas de « vente à la	découpe » (loi \nom{Aurillac} du 13 juin 2006)}
			
			\par
			Cette loi trouve son origine dans les ventes d’importants patrimoines réalisées par les grands investisseurs
			(Établissements Financiers et Assureurs, notamment) d’immeubles entiers, sans division préalable de
			ceux-ci en lots de copropriété : il est plus rapide de vendre un immeuble entier plutôt que des lots de
			copropriété.
			
			Ces « ventes en bloc » dites encore « ventes à la découpe » ont été dénoncées médiatiquement :
			reproche étant fait aux acquéreurs (parfois fonds spéculatifs américains) de maltraiter les locataires en
			proposant l’acquisition des lots à des prix prohibitifs et les mêmes media ont réclamé une loi de protection
			des locataires. La loi a été modifiée par la loi ALUR
			181 

			\subparagraph{A) CONDITIONS DE MISE EN OEUVRE}.
			Cette loi crée au profit des locataires à usage d'habitation ou à usage mixte d'habitation et professionnel,
			un nouveau droit de préemption en cas de vente de l’immeuble « dans sa totalité et en une seule fois ».
			C’est l’hypothèse de la « vente en bloc »
			Ce droit de préemption ne peut jouer que si l’immeuble comporte plus de CINQ logements, en ce compris
			les logements non donnés à bail (seuil abaissé de 10 à 5 logements par la loi ALUR)
			Pour bénéficier du droit de préemption l’occupant doit justifier d’un bail en cours (d’habitation ou
			d’habitation et professionnel) à la date de signature de l’acte de vente ou être titulaire d’un droit au
			maintien dans les lieux (loi de 48).
			B) PROCEDURE DE PREEMPTION
			Ce droit de préemption sera proposé avant la vente de l’immeuble : le vendeur fait connaître par lettre
			recommandée A.R. à ses locataires le prix et les conditions de la vente de l’immeuble dans son ensemble
			et du logement en particulier. A cette offre le propriétaire joint le diagnostic technique (constat de l'état
			apparent de la solidité du clos et du couvert et de celui de l'état des conduites et canalisations collectives)
			et une copie du projet de Règlement de copropriété.
			Le locataire dispose d’un délai de quatre mois pour accepter cette offre d’acquérir et la vente se fera dans
			les deux mois qui suivent.
			Si le locataire ne préempte pas dans le délai de 4 mois le vendeur pourra passer la vente « en bloc ».
			Si un locataire préempte les autres lots seront alors vendus « en copropriété ». Si ces lots sont vendus
			moins cher qu’au prix proposé précédemment, le locataire pourra se substituer à l’acquéreur au prix de
			l’acte de vente (après que cette vente lui ait été notifié le locataire disposera d’un délai d’un mois
			seulement pour se substituer à l’acquéreur du lot).
			C) LA PROROGATION DES BAUX EN COURS PENDANT SIX ANS.
			Toutefois le vendeur peut échapper à cette obligation de mise en oeuvre du droit de préemption s’il
			s’oblige“à proroger les contrats de bail à usage d'habitation en cours à la date de la conclusion de la vente
			afin de permettre à chaque locataire ou occupant de bonne foi de disposer du logement qu'il occupe pour
			une durée de six ans à compter de la signature de l'acte authentique de vente qui contiendra la liste des
			locataires concernés par un engagement de prorogation de bail”. Cet engagement figurera dans l’acte de
			vente et sera transmis de plein droit à son acquéreur.
			En sorte que si la vente intervient alors que le bail du locataire doit s’achever dans les deux ans, ce locataire
			bénéficiera au total d’un bail de huit ans !
			Le bailleur a le plus grand intérêt à notifier par LAR son engagement de prorogation afin que son droit à
			délivrer congé après ces 6 années ne soit pas contesté par le locataire.
			D) SANCTION EN CAS DE NON RESPECT DU DROIT DE PREEMPTION OU DE PROROGATION
			DES BAUX.
			droit de la copropriété année 2018-2019
			166
			L’absence de notification dans l’un ou l’autre cas définis précédemment entraîne la nullité de la vente qui
			pourra être demandée par le locataire en place.
		
		\subsubsection{Le droit de préemption urbain}
		
			Le droit de préemption urbain permet à la commune d'acquérir prioritairement un bien foncier ou
			immobilier lorsque celui-ci est sur le point d'être vendu, afin de lui permettre de réaliser ses projets
			d’aménagement.
			
			Sont exclus du droit de préemption les successions, les donations portant sur des immeubles ou droits
			sociaux (SCI) entre parents jusqu’au 6ème degré ou entre personnes ayant des liens issus d’un mariage ou
			d’un pacs, le partage, les immeubles faisant l'objet d'un contrat de vente d'immeubles à construire, les
			donations.
			
			La commune ne peut en principe exercer son droit que sur les biens immobiliers dont la construction est
			achevée depuis au moins 4 ans (date de la DAACT) qui font l'objet d'une cession volontaire ou forcée à titre
			onéreux (vente, échange, apport en société $\dots$).
			
			L'\article{L}{211-4} du Code de l'urbanisme exclut en outre du champ d'application du droit de préemption
			urbain l’aliénation d'un ou plusieurs lots constitués soit par un seul local à usage d'habitation, à usage
			professionnel ou à usage professionnel et d'habitation, soit par un tel local et ses locaux accessoires, soit
			par un ou plusieurs locaux accessoires d'un tel local, compris dans un bâtiment soumis […$\dots$] au régime de
			la copropriété depuis plus de dix ans (date de publication du règlement de copropriété).
			
			En revanche, le DPU est applicable :
			\begin{itemize}
				\item en cas de vente d’un lot de copropriété « habitation » d’un immeuble achevé depuis plus de
				4 ans, et dont le règlement de copropriété a été publié il y a moins de 10 ans (donc à un
				immeuble neuf soumis au régime après \VEFA{} entre la 5\ieme{} et la 10\ieme{} année après dépôt du
				règlement de copropriété) ;
				\item  en cas de vente d’un lot affecté à une activité commerciale ;
				\item  en cas de délibération motivée de la Commune ou de l’EPCI de créer un droit de préemption
				dit « renforcé » ;
				\item  dans les ZAD, c'est-à-dire les Zones d’Aménagement Différé, secteur où une collectivité locale,
				un établissement public y ayant vocation ou une Société d'économie mixte (SEM) titulaire
				d'une convention d'aménagement dispose, pour une durée de 14 ans, d'un droit de
				préemption sur toutes les ventes et cessions à titre onéreux de biens immobiliers ou de droits
				sociaux.
			\end{itemize}
			
		\subsubsection{Le droit de préemption au profit des autres copropriétaires}
			
			Il s’agit ici d’un texte ajouté à la loi sur la Copropriété par et qui constitue L’ article 8-1 de la Loi \no 65-557 du 10 juillet 1965, introduite par la loi MOLLE du 25 mars 2009, ouvre la possibilité d’un droit de
			préemption au profit des autres copropriétaires sur les lots de stationnement.
			
			\begin{quote}
				« Le règlement de copropriété des immeubles dont le permis de construire a été délivré conformément à
				un plan local d'urbanisme ou d'autres documents d'urbanisme imposant la réalisation d'aires de
				stationnement peut prévoir une clause attribuant un droit de priorité aux copropriétaires à l'occasion de la
				vente de lots exclusivement à usage de stationnement au sein de la copropriété.
			
				Dans ce cas, le vendeur doit, préalablement à la conclusion de toute vente d'un ou plusieurs lots à usage
				de stationnement, faire connaître au syndic par lettre recommandée avec demande d'avis de réception son
				intention de vendre, en indiquant le prix et les conditions de la vente.
				
				Cette information est transmise sans délai à chaque copropriétaire par le syndic par lettre recommandée
				avec demande d'avis de réception, aux frais du vendeur. Elle vaut offre de vente pendant une durée de deux
				mois à compter de sa notification ».
			\end{quote}
		
			\paragraph{Origine de ce texte}
			
			\par Aux termes de l'article 8 de la loi du 10 juillet 1965 les clauses restrictives du règlement de copropriété ne
			sont admises que si elles sont justifiées par la destination de l'immeuble, alors que l'article 9 pose le
			principe selon lequel chaque copropriétaire dispose librement de son lot. En application de ces dispositions
			légales la jurisprudence est le plus souvent hostile aux clauses restrictives au droit de disposer librement
			de leur lot par les copropriétaires. Elle considère principalement que le droit de préemption donné aux
			copropriétaires est étranger à l'objet même du syndicat des copropriétaires.
			
			Or revendre un parking à un tiers étranger au syndicat de copropriété alors que le permis de construire a
			imposé la réalisation d'un certain nombre de places de stationnement constitue un véritable
			détournement des autorisations obtenues.
			
			\paragraph{Le mécanisme mis en place}
			
			\par Ce mécanisme se ne s'applique que si le copropriétaire vend séparément son parking des autres lots qu'il
			peut posséder. Il devra adresser une lettre recommandée au syndic pour l'informer de ce qu'il vend son
			emplacement de stationnement et du prix auquel il se propose de réaliser cette vente.
			
			Le syndic devra alors, à son tour, adresser « sans délai » une lettre recommandée à tous les
			copropriétaires, « aux frais du vendeur » ( toutefois, cet envoi ne doit pas donner à perception
			d’honoraire, n’étant pas prévu dans le contrat type).
			
			Dans les copropriétés importantes (où les mutations de parkings seuls sont les plus fréquentes) cette
			procédure aura un effet dissuasif sur le copropriétaire compte tenu des frais et éventuellement honoraires
			qu'elle va engendrer pour lui.
			
			Une fois « cette notification faite par le syndic, le copropriétaire devra attendre l'expiration d'un délai de
			deux mois pour savoir si son parking a été préempté ou s’il récupère la libre disposition de celui-ci.
	
			Toutefois, la loi MOLLE prévoit simplement la possibilité d’insérer une telle clause dans le règlement de
			copropriété. Il faut donc faire une modification du règlement de copropriété qui, en principe, requière
			l’unanimité (modification des droits du copropriétaire sur son lot)\footnote{en ce sens : réponse ministérielle, Question \no \nombre{120 883}, publiée au JO le 3 janvier 2012, page 86}.
		
		\subsubsection{Le droit de préemption en cas de vente par le syndicat de copropriété du droit de surélever (art. 35 de la loi du 10 juillet 1965)}
	
			\begin{quote}
				\textbf{Article 35 de la loi alinéa premier}\newline
				La surélévation ou la construction de bâtiments aux fins de créer de nouveaux locaux à usage privatif ne
				peut être réalisée par les soins du syndicat que si la décision en est prise à la majorité prévue à l’article 26.
			\end{quote}
			
			La décision d’aliéner aux mêmes fins le droit de surélever un bâtiment existant exige la majorité prévue à
			l’article 26, et, si l’immeuble comprend plusieurs bâtiments, la confirmation par une assemblée spéciale
			des copropriétaires des lots composant le bâtiment à surélever, statuant à la majorité indiquée ci-dessus.
			
			Les copropriétaires de l'étage supérieur du bâtiment surélevé bénéficient d'un droit de priorité à l'occasion
			de la vente par le syndicat des locaux privatifs créés. Préalablement à la conclusion de toute vente d'un ou
			plusieurs lots, le syndic notifie à chaque copropriétaire de l'étage supérieur du bâtiment surélevé l'intention
			du syndicat de vendre, en indiquant le prix et les conditions de la vente. Cette notification vaut offre de
			vente pendant une durée de deux mois à compter de sa notification.
			
			Les copropriétaires de l'étage supérieur du bâtiment à surélever bénéficient du même droit de priorité à
			l'occasion de la cession par le syndicat de son droit de surélévation. << Ce droit de priorité s'exerce dans les
			mêmes conditions que celles prévues au quatrième alinéa >>.
			
			\bigskip
			Avant la loi ALUR, les propriétaires du dernier étage disposaient d’un droit de véto en cas de surélévation.
			Ce droit a été transformé en simple droit de « priorité », purgé par le syndic (et non le notaire !), qui doit
			notifier à ces propriétaires « le prix et les conditions de la vente ». Ce n’est qu’une fois ce « droit de
			priorité » purgé que le droit peut être offert à la vente aux tiers.
			
			Toutefois, le texte ne précise ni comment le syndic fixe le prix de vente avant la tenue de l’assemblée
			générale, ni comment traiter l’éventuel conflit entre différents propriétaires du dernier étage, ni si l’offre
			doit comporter un prix de vente « global » ou divisé.
	
	\subsection{L’interdiction d’acquérir}
	
		Sous couvert d’un chapitre relatif à la lutte contre « les acquéreurs déstabilisateurs en copropriété », la loi
		ALUR a introduit une double restriction au droit d’acquérir un lot. L’objectif de ces dispositions est
		d’éviter, dans une copropriété en difficulté notamment, les acquisitions massives de lots par des
		marchands de sommeil, qui diviserons les lots pour louer , sans pour autant payer leurs charges.
		Le dispositif a été complété et renforcé par la loi ELAN.
		
		\subsubsection{Lutte contre les marchands de sommeil}
		
			La loi ALUR a ajouté au Code Pénal des peines complémentaires à l’encontre de personnes condamnées
			pour avoir soumis « une personne, dont la vulnérabilité ou l'état de dépendance sont apparents ou connus
			de l'auteur, à des conditions de travail ou d'hébergement incompatibles avec la dignité humaine ». La
			sanction principale est puni de cinq ans d'emprisonnement et de 150 000 euros d'amende. » ( infraction
			défini à l’article L.~225-14).
			
			Cette peine complémentaire sanctionne précisément les pratiques des marchands de sommeil :
			\begin{itemize}
				\item Le fait de soumettre une personne, dont la vulnérabilité ou l'état de dépendance sont apparents
				ou connus de l'auteur, à des conditions de travail ou d'hébergement incompatibles avec la
				dignité humaine est puni de cinq ans d'emprisonnement et de 150 000 euros d'amende (article
				225-14 du Code Pénal).
				
				\item  Le fait de mettre à disposition aux fins d'habitation, à titre gratuit ou onéreux des caves, sous-sols,
				combles, pièces dépourvues d'ouverture sur l'extérieur et autres locaux par nature impropres à
				l'habitation et de ne pas déférer à l’injonction préfectorale de faire cesser la situation (\article{L}{1337-4} du	Code de la santé publique).
				
				\item  Le refus délibéré et sans motif légitime de réaliser des travaux prescrit par le Maire pour faire
				cesser la situation d'insécurité constatée par la commission de sécurité dans un établissement
				recevant du public à usage total ou partiel d'hébergement (\article{L}{123-3} du \CCH).
			
				\item  Le refus délibéré et sans motif légitime, constaté après mise en demeure, d'exécuter les travaux
				prescrits par le Maire par suite d’un arrêté de péril ou d’insalubrité (\article{L}{511-6} du \CCH)
			\end{itemize}
			
			Selon l’actuel \article{L}{225-26} du Code pénal, doivent être obligatoirement prononcées les peines
			complémentaires suivantes (sauf décision motivée).
			\begin{quote}
				1\degre{} La confiscation de tout ou partie de leurs biens, quelle qu'en soit la nature, meubles ou immeubles, divis
				ou indivis, ayant servi à commettre l'infraction. Lorsque les biens immeubles qui appartenaient à la
				personne condamnée au moment de la commission de l'infraction ont fait l'objet d'une expropriation pour
				cause d'utilité publique, le montant de la confiscation en valeur prévue au neuvième alinéa de l'article
				131-21 est égal à celui de l'indemnité d'expropriation ;
				
				2\degre{} L'interdiction pour une durée de dix ans au plus d'acheter un bien immobilier à usage d'habitation ou
				un fonds de commerce d'un établissement recevant du public à usage total ou partiel d'hébergement ou
				d'être usufruitier d'un tel bien ou fonds de commerce. Cette interdiction porte sur l'acquisition ou
				l'usufruit d'un bien ou d'un fonds de commerce soit à titre personnel, soit en tant qu'associé ou mandataire
				social de la société civile immobilière ou en nom collectif se portant acquéreur ou usufruitier, soit sous
				forme de parts immobilières ; cette interdiction ne porte toutefois pas sur l'acquisition ou l'usufruit d'un
				bien immobilier à usage d'habitation à des fins d'occupation à titre personnel ;
				
				3\degre{} La confiscation de tout ou partie des biens leur appartenant ou, sous réserve des droits du propriétaire
				de bonne foi, dont elles ont la libre disposition, quelle qu'en soit la nature, meubles ou immeubles, divis
				ou indivis.
			\end{quote}
			
			Il est en conséquence inséré au sein du Code des procédures civiles d'exécution, un \article{L}{322-7-1} : la
			personne qui est condamnée à l'une des peines complémentaires précitées ne peut se porter enchérisseur
			pendant la durée de cette peine pour l'acquisition d'un bien immobilier à usage d'habitation ou d'un fonds
			de commerce d'un établissement recevant du public à usage total ou partiel d'hébergement, sauf dans le
			cas d'une acquisition pour une occupation à titre personnel.
			
			Le Code de la Construction et de l'Habitation comprend également un \article{L}{551-1} qui impose au notaire
			qui établit l’acte de vente de vérifier si l’acquéreur a fait l’objet d’une condamnation prévue par l’\article{L}{225-19}. Si tel est le cas, il ne pourra recevoir l’acte de vente, et la promesse sera résiliée aux torts de
			l’acquéreur.

			Le notaire doit interroger l'Association pour le développement du service notarial placée sous le contrôle
			du Conseil supérieur du notariat, qui demande consultation du bulletin \no 2 du casier judiciaire de
			l'acquéreur au casier judiciaire national automatisé.
			
			Enfin, depuis la loi ELAN, le syndic doit signaler au procureur de la République les faits qui sont susceptibles
			de constituer une des infractions prévues aux articles 225-14 du Code pénal, \article{L}{1337-4} du Code de la santé
			publique et \article{L}{123-3}, L. 511-6 et L. 521-4 du Code de la construction et de l'habitation (L. \no 65-557, 10 juill.
			1965, art. 18-1-1.), à l’exception des syndics non professionnels. La disposition a été introduite dans la Loi
			du 10 juillet 1965, mais également dans la loi HOGUET (L. \no 70-9, 2 janv. 1970, JO 4 janv., art. 8-2-1, nouv.),
			elle est donc contrôlée par le CNTGI et la DGCCRF.
			
			Les personnes qui se livrent à des activités d'entremise et de gestion des immeubles et fonds de commerce
			doivent effectuer ce même signalement
		
		\subsubsection{Interdiction d’acquérir un nouveau lot concernant un copropriétaire débiteur de charges}
		
			L’article 55 de la loi ALUR, ajoutant un paragraphe \II{} à l’article 20 de la loi du 10 juillet 1965, a été introduit
			dans le but d’interdire aux copropriétaires « déstabilisateurs » d’acheter de nouveaux lots dans la
			copropriété –-- en pratique est visé tout copropriétaire présentant un risque identifiable de ne pas acquitter
			ses charges.
			
			Préalablement à la signature de la vente, le notaire doit indiquer au syndic le nom du futur acheteur, ainsi
			que celui de son partenaire pacsé ou de son époux, ou encore de tous ses associés et mandataires sociaux
			s’il s’agit d’une SCI ou d’une SNC, afin de savoir si l’une de ces personnes est déjà copropriétaire dans
			l’immeuble et s’il est à jour du paiement de ses charges.
			
			Le syndic dispose d’un mois pour répondre. S’il indique l’une de ces personnes a fait l’objet d’une mise en
			demeure de payer restée infructueuse plus de 45 jours, le notaire doit informer les parties de
			l’impossibilité de régulariser la vente. Le candidat acquéreur (ou son conjoint, pacsé $\dots$) devra s’acquitter
			des sommes dues dans les 30 jours de la notification par le notaire de son refus d’instrumenter ; seul un
			certificat du syndic attestant de cet acquit permettra au notaire de régulariser la vente. A défaut, la
			promesse de vente sera résiliée, aux torts de l’acquéreur, lui faisant ainsi perdre la somme versée à titre
			d’indemnité d’immobilisation ou de dépôt de garantie.
			
			Le dispositif, peu efficace contre les marchands de sommeil, est en revanche redoutable contre les
			copropriétaires « mauvais payeurs » qui souhaitent acheter un lot annexe dans la copropriété

\section{La cession des parties communes}

	\subsection{Interdiction de cession de la quote-part de partie commune séparément des parties privatives (article 6 de la loi du 10 juillet 1965)}
	
		\begin{quote}
			<< Les parties communes et les droits qui leur sont accessoires ne peuvent faire l'objet, séparément des
			parties privatives, d'une action en partage ni d'une licitation forcée >>.
		\end{quote}
		
		Le copropriétaire n'a de droit de cession que sur son lot, considéré en tant qu'une entité indivisible
		composé des parties privatives et de la quote-part de parties communes.
		
		On peut cependant s'interroger sur les droits accessoires aux parties privatives.
		\begin{itemize}
			\item Lorsque de tels droits sont constitués en lots privatifs, leur aliénation ne posera aucun
			problème puisque le lot comportera à la fois un droit purement privatif (celui de construire
			par exemple) et une quote part de parties communes (les tantièmes dont ce lot est affecté).
			
			\item Lorsque le lot privatif comportera en même temps un volume privatif, la jouissance d'une
			partie de l'immeuble (jardin) et le droit accessoire proprement dit : le copropriétaire pourra
			dans le respect de l'article 6 de la loi diviser son lot en deux nouveaux lots en répartissant
			les tantièmes existants entre ces deux lots, le premier comprenant le volume privatif et le
			second la jouissance du jardin avec droit de construire.
			
			\item En revanche lorsque ces droits accessoires ne sont pas constitués en lot, mais se
			<< découvrent >> à la lecture des dispositions du Règlement de Copropriété, ou dans la
			description du lot, le copropriétaire ne sans enfreindre les dispositions de l'article 6 de la
			loi, ne céder que ces droits accessoires en conservant pour lui le volume seul constitué en
			lot privatif doté de quotes-parts de parties communes.
		\end{itemize}
		
		\textbf{Exception :} l’\article{L}{615-10} du \CCH{} (article 72 de la loi ALUR)
		\begin{quote}
			– \I. – Par dérogation à l’article 6 de la loi \no 65-557 du 10 juillet 1965 fixant le statut de la copropriété des
			immeubles bâtis, une possibilité d’expropriation des parties communes est instaurée à titre expérimental
			et pour une durée de dix ans à compter de la promulgation de la loi \no du pour l’accès au logement
			et un urbanisme rénové. Dans ce cas, l’article L. 13-10 du code de l’expropriation pour cause d’utilité
			publique est applicable
		\end{quote}
		
		Cette procédure qui concerne le dispositif législatif mis en place pour parer à la carence des
		syndicats de copropriété fait l’objet de commentaires au Chapitre du cours relatif à la disparition
		d’un syndicat des copropriétaires et du dernier Chapitre du cours (Poly 2) relatif aux copropriétés
		en difficulté.
	
	\subsection{Interdiction de cession des parties communes matérielles du syndicat}
	
		Quant aux parties communes matérielles, il va de soi qu'un copropriétaire ne peut en disposer
		personnellement, par exemple vendre des locaux de services communs.
		
		Ce droit de disposer de parties communes déterminées appartient au syndicat statuant à la majorité de
		l'article 26 (majorité des copropriétaires représentant les deux tiers des voix), étant précisé que s'il s'agit
		de parties communes dont la conservation est nécessaire au respect de la destination de l'immeuble,
		l'unanimité est requise (article 26 dernier alinéa).
		
		La vente de parties communes, à la supposer parfaite, apporte une modification au règlement de
		copropriété qui n’est pas opposable aux acquéreurs qu’à dater de sa publication au fichier immobilier
		Le transfert de propriété de peut s’opérer qu’à compter de la création de nouveaux lots de copropriété et
		l’attribution au lot créé de tantièmes de parties communes.
		
		La décision qui arrête le principe de la vente n’est qu’une décision préparatoire qui n’opère pas de
		transfert de propriété, celle-ci requérant une nouvelle décision d’assemblée statuant sur la vente et sur
		toutes les modifications du règlement de copropriété et de l’état descriptif de division en résultant\footnote{CA Paris 2 juillet 2009 JD 09-378843}.
	
	\subsection{L’expropriation du syndicat}
	
		Quatre types d’expropriation peuvent frapper le Syndicat
		\begin{itemize}
			\item  l'expropriation pour cause d’utilité publique ;
			\item  l'expropriation pour carence (\article{L}{615-6} du \CCH, dite Loi \nom{Borloo})
			\item  l’expropriation en cas de péril imminent ( art. , dite loi Vivien)
			\item  l’expropriation des parties communes à titre expérimental, pour les copropriétés
			<< irrémédiablement dégradées >>
		\end{itemize}
		
		\subsubsection{L'expropriation pour cause d’utilité publique}
		
		La procédure d'expropriation permet à une collectivité territoriale de s'approprier des biens immobiliers
		privés, afin de réaliser un projet d'aménagement dans un but d'utilité publique. Une opération
		d'expropriation ne peut être légalement déclarée d'utilité publique que si les atteintes à la propriété privé,
		le coût financier et éventuellement les inconvénients d'ordre social qu'elle comporte ne sont pas excessifs
		eu égard à l'intérêt qu'elle présente.
		
		La procédure d'expropriation se décompose en deux phases.
		\begin{enumerate}
			\item \textbf{La phase administrative} dont la finalité est la déclaration d'utilité publique du projet prononcé par arrêté
			préfectoral (enquête d'utilité publique) et la détermination des parcelles à exproprier définies par un
			arrêté préfectoral de cessibilité (enquête parcellaire) ; elle aboutit à deux arrêté préfectoraux, l’arrêté de
			déclaration d’utilité publique et l’arrêté de cessibilité. Chacune de ces décisions est susceptible de recours
			devant le Juge administratif dans le délai de deux mois suivant la notification de la décision.
			
			\item \textbf{La phase judiciaire}, qui correspond à la procédure de transfert de propriété des biens et d'indemnisation
			des propriétaires. Cette procédure est instruite par le juge de l'expropriation dès la transmission du dossier
			administratif finalisé par le préfet au juge de l'expropriation. Dans un délai qui ne peut excéder 6 mois à
			compter de la date de l'arrêté de cessibilité, et si l'acquisition des parcelles n'a pas pu se faire à l'amiable,
			l'expropriant saisit le préfet aux fins de transmettre le dossier au juge de l'expropriation (au greffe du
			tribunal de grande instance), afin que celui-ci prononce l'ordonnance d'expropriation.
		\end{enumerate}
		
		C'est en effet le préfet, exclusivement, qui saisit le juge de l'expropriation sur demande de l'expropriant.
	
		Le principal effet de l'ordonnance d'expropriation est de transférer à l'expropriant la propriété de
		l'immeuble exproprié. Mais la prise de possession est subordonnée au fait que l'indemnité d'expropriation
		ait été payée ou consignée.
		
		Cette expropriation peut porter sur une partie commune (cas le plus fréquent, expropriation d’une partie
		du terrain), auquel cas seul le Syndicat est poursuivi dans le cadre de la procédure d’expropriation, et
		l’indemnité sera répartie entre tous les copropriétaires au prorata des tantièmes. Cependant
		l'indemnisation du syndicat des copropriétaires pour l'expropriation de parties communes n'exclut pas
		nécessairement celle de chaque copropriétaire pour la dévalorisation de la partie privative de son lot\footnote{Cour de cassation, 3e Chambre civ., 11 octobre 2006 (\no de pourvoi: 05-16.037)}.
		
		Si l’expropriation touche aussi une partie privative, ou une partie commune à jouissance privative ( par
		exemple un lot parking), l’expropriation devra être poursuivie contre le Syndicat et le copropriétaire à titre
		individuel\footnote{voir sur ce sujet le mémoire d’anne Gazeau: Le statut de la copropriété dans la procédure
			D’expropriation https://dumas.ccsd.cnrs.fr/dumas-01701078}.
		
		L’expropriation de la partie commune a pour conséquence le retrait de la partie de terrain expropriée (au
		delà de la ligne divisoire, qui doit figurer dans l’enquête parcellaire), donc une scission de la copropriété
		(art 16-2 de la loi du 10 juillet 1965 issu de la loi \no96-987 du 14 novembre 1996). Elle nécessite donc, en
		principe, une adaptation du règlement consécutive à la scission.
		
		\subsubsection{L’expropriation du Syndicat en état de carence (loi \no 2003-710 du 1er août 2003 d'orientation et de programmation pour la ville et la rénovation urbaine, dite « loi Borloo » : article L.615-6 et L.~615-7 du \CCH)}
		
		L’état de carence peut être prononcé par le président du tribunal de grande instance, suite au rapport de
		l’expert désigné par ses soins, lorsqu' en raison de graves difficultés financières et de gestion un
		propriétaire, un syndicat des copropriétaires est dans l’impossibilité d’assurer la réalisation des travaux
		nécessaire à la conservation des immeubles et à la sécurité des occupants.
		
		Lorsque l’état de carence du ou des immeubles a été déclaré (par ordonnance du Président du TGI rendue
		au vue du rapport de l’expert et les parties entendues), l’expropriation est poursuivie, dans les conditions
		fixées par le code de l’expropriation pour cause d’utilité publique, au bénéfice de la commune ou de
		l’établissement public de coopération intercommunale compétent en matière de logement pour la mise
		en oeuvre d’actions ou d’opérations de rénovation urbaine ou de politique locale de l’habitat. Cette
		expropriation porte sur les parties communes et sur les lots privatifs.
		
		Le cas échéant, dans l'ordonnance prononçant l'état de carence, le président du tribunal de grande
		instance désigne un administrateur provisoire mentionné à l'article 29-1 de la loi \no 65-557 du 10 juillet
		1965 précitée pour préparer la liquidation des dettes de la copropriété et assurer les interventions
		urgentes de mise en sécurité.

		Ces dispositions ont été renforcées par la loi ALUR\footnote{Voir Poly 2 – Les Copropriétés en difficulté}, et par la Loi ELAN.
		Ainsi, depuis la loi ELAN :
		\begin{itemize}
			\item l’agence nationale de l’habitat (Anah) doit encourager et faciliter l’exécution d’opérations de
			résorption d’une copropriété dont l’état de carence a été déclaré conformément à l’\article{L}{615-6} du Code de la construction et de l’habitation (CCH, art. L. 321-1, \I, mod.) ;
			\item la procédure d’expropriation doit être menée de façon contradictoire à l’encontre du Syndicat,
			mais aussi de chacun des copropriétaires, ce qui n’était pas le cas jusqu’à la loi ELAN (le rapport
			de l’expert doit leur être notifié, et ils sont nécessairement assignés individuellement dans le cadre
			de la procédure).
		\end{itemize}
		
		\subsubsection{L’expropriation des copropriétés à usage d’habitation insalubres ou dangereux (Dite expropriation << Loi \nom{Vivien} >> L 1331-25 et L 1331-28 du CCH, article L 511-2 du CCH)}
		
		Cette expropriation concerne visés les immeubles déclarés insalubres à titre irrémédiable en
		application des articles L. 1331-25 et L. 1331-28 du code de la santé publique, ou frappés d’un arrêté de
		péril et interdits définitivement à l’habitation, en application de l’\article{L}{511-2} du \CCH. Ce même article
		précise que peuvent aussi être expropriés selon ce mode dérogatoire des immeubles qui ne sont eux-mêmes
		ni insalubres, ni impropres à l'habitation, lorsque leur expropriation est indispensable à la
		démolition des immeubles insalubres ou d'immeubles menaçant ruine, ainsi que des terrains où sont situés
		les immeubles déclarés insalubres ou menaçant ruine lorsque leur acquisition est nécessaire à la
		résorption de l'habitat insalubre.
		
		La DUP peut être signée par le préfet sur la seule base des arrêtés d'insalubrité irrémédiable ou d'un arrêté
		de péril ayant ordonné la démolition du bâtiment ou prononcé une interdiction définitive d'habiter.
		
		La DUP désigne le bénéficiaire de l'expropriation, mentionne obligatoirement les offres de relogement
		faites aux occupants (y compris les propriétaires). Elle est notifiée au Syndicat et aux copropriétaires.
		
		La même DUP déclare cessibles les terrains et immeubles visés dans l'arrêté (cette cession peut intervenir
		au profit d’un acquéreur privé) et fixe le montant des indemnités provisionnelles dues aux propriétaires
		(et titulaires de baux commerciaux) qui ne peuvent être inférieures à l'évaluation des Domaines.
		
		L’indemnité d’expropriation est fixée à la valeur dite de << récupération foncière >>, c’est à dire à la valeur du
		terrain nu, déduction faite des travaux de démolition. L’indemnité est réduite du montant des frais de
		relogement des occupants assuré, lorsque le propriétaire n'y a pas procédé. Seuls les occupants de bonne
		fois, depuis au moins 2 ans avant la notification de l'arrêté d’insalubrité ou de péril, ont droit à une
		indemnité fixée selon le droit commun (ou si l’immeuble n’est exproprié que par voie de conséquence
		dans le cadre d’une opération de RHI, sans être lui-même insalubre ou frappé de péril).
		
		\subsubsection{L’expropriation des parties communes a titre expérimental}

		C’est un dispositif << révolutionnaire >> introduit par la loi ALUR dans l’\article{L}{615-10} du \CCH, qui déroge
		explicitement à l’article 6, puisqu'il aboutit précisément à la dissociation des parties communes et
		privatives. Le but est de réduire le coût du portage par l’opérateur en cas de réhabilitation d’une
		copropriété en difficulté ne nécessitant pas l’expulsion des occupants. Il aboutit à la disparition du Syndicat
		(voir le chapitre I).
		
		\begin{quote}
			Article L 615-10 CCH (nouveau)
			<< \I. – Par dérogation à l’article 6 de la loi \no 65-557 du 10 juillet 1965 fixant le statut de
			la copropriété des immeubles bâtis, une possibilité d’expropriation des parties
			communes est instaurée à titre expérimental et pour une durée de dix ans à compter de
			la promulgation de la loi \no du pour l’accès au logement et un urbanisme
			rénové. Dans ce cas, l’article L. 13-10 du code de l’expropriation pour cause d’utilité
			publique est applicable
			
			\medskip
			« \II. – Lorsque le projet mentionné au \V{} de l’article L. 615-6 du présent code prévoit
			l’expropriation de l’ensemble des parties communes, la commune ou l’établissement
			public de coopération intercommunale compétent en matière d’habitat peut confier
			l’entretien de ces biens d’intérêt collectif à un opérateur ou désigner un opérateur au
			profit duquel l’expropriation est poursuivie.
			
			« Au moment de l’établissement du contrat de concession ou de la prise de
			possession par l’opérateur, l’état descriptif de division de l’immeuble est mis à jour ou
			établi s’il n’existe pas. Aux biens privatifs mentionnés dans l’état de division est attachée
			une servitude des biens d’intérêt collectif. Les propriétaires de ces biens privatifs sont
			tenus de respecter un règlement d’usage établi par l’opérateur.
			
			« En contrepartie de cette servitude, les propriétaires sont tenus de verser à
			l’opérateur une redevance mensuelle proportionnelle à la superficie de leurs parties
			privatives. Cette redevance, dont les modalités de révision sont prévues par décret,
			permet à l’opérateur de couvrir les dépenses nécessaires à l’entretien, à l’amélioration
			et à la conservation de parties communes de l’immeuble et des équipements communs.
			
			« Pour les propriétaires occupants, cette redevance ouvre droit aux allocations de
			logement prévues aux articles L. 542-1 à L. 542-9 et L. 831-1 à L. 835-7 du code de la
			sécurité sociale.
			
			\medskip
			« \III. – L’opérateur est chargé d’entretenir et de veiller à la conservation des biens
			d’intérêt collectif. Il est responsable des dommages causés aux propriétaires de parties
			privatives ou aux tiers par le vice de construction ou le défaut d’entretien des biens
			d’intérêt collectif, sans préjudice de toutes actions récursoires.
			
			« Il réalise un diagnostic technique des parties communes, établit un plan
			pluriannuel de travaux actualisé tous les trois ans et provisionne, dans sa comptabilité,
			des sommes en prévision de la réalisation des travaux.
			
			\medskip
			« \IV. – Le droit de préemption urbain renforcé prévu à l’article L. 211-4 du code de
			l’urbanisme peut lui être délégué.
			
			\medskip
			« \V. – Dans le cadre de l’expérimentation prévue au présent article, en cas de déséquilibre
			financier important, l’opérateur peut demander à la commune ou à l’établissement
			public de coopération intercommunale compétent en matière d’habitat à l’origine de
			l’expérimentation de procéder à l’expropriation totale de l’immeuble. Un nouveau
			projet d’appropriation publique doit alors être approuvé dans les conditions prévues
			au V de l’article L. 615-6 du présent code. La procédure est poursuivie dans les
			conditions prévues à l’article L. 615-7.
			
			\medskip
			« \VI. – Après avis favorable de la commune ou de l’établissement public de coopération
			intercommunale compétent en matière d’habitat à l’origine de l’expérimentation et des
			propriétaires des biens privatifs, l’immeuble peut faire l’objet d’une nouvelle mise en
			copropriété à la demande de l’opérateur. Les propriétaires versent alors une indemnité
			au propriétaire de ces biens d’intérêt collectif équivalente à la valeur initiale
			d’acquisition des parties communes ayant initialement fait l’objet de l’expropriation,
			majorée du coût des travaux réalisés, de laquelle est déduit le montant total des
			redevances versées à l’opérateur. Cette indemnité est répartie selon la quote-part des
			parties communes attribuée à chaque lot dans le projet de règlement de copropriété.
		\end{quote}
		
		C’est donc une loi à caractère temporaire (par contre l’expropriation pratiquée est définitive). Seule est
		prévue au \VI, la possibilité de recréer une copropriété à la demande de l’opérateur lorsque les travaux de
		remise en état des parties communes auront été réalisés ; auquel cas les copropriétaires,
		proportionnellement à leurs tantièmes verseront à l’opérateur « une indemnité ( ?) » correspondant au
		coût d’achat des parties communes par l’Opérateur, majoré du coût des travaux réalisés par l’Opérateur
		et diminué le montant des redevances payés par les propriétaires à l’Opérateur pour la conservation de
		ces (ex)parties communes.
		
		\par Le texte précise les modalités d’application et les conséquences quant à la gestion de ces parties
		communes expropriées.
		\begin{itemize}
			\item  La procédure normale d’expropriation est mise en œuvre (avec bien évidemment une DUP).
			
			\item  La Commune (ou l’EPCI) confie l’entretien des parties communes expropriées à un Opérateur qui
			établira une servitude au profit des parties privatives sur les parties communes expropriées et
			rédigera un règlement d’usage. En contrepartie de cette servitude les copropriétaires devront
			payer une « redevance » à l’opérateur qui couvrira non seulement les dépenses d’entretien et de
			conservation de ces parties communes mais également les travaux d’amélioration de ces parties
			communes et de leurs équipements.
			
			\item  Enfin le texte précise que cette expropriation partielle (ne portant que sur les parties communes)
			peut être le premier pas vers une expropriation totale de l’immeuble (donc des parties privatives)
			si l’Opérateur ne parvient pas à équilibrer la gestion de ces parties communes.
			
			En fait ce texte est l’aboutissement d’une longue réflexion du Ministère du Logement qui avait suscité les
			plus grandes réserves de la part des juristes mais qui avait été reprise dans le rapport \nom{Braye}. Sa
			constitutionnalité est douteuse (cette procédure n’aboutit-elle pas, en fait, à l’expropriation du lot?), et le
			décret d’application reste en attente.
		\end{itemize}
	
\section[Lors de la promesse de vente]{Les informations et vérifications accomplies lors de la promesse de vente du lot de copropriété}

	Les formalités destinées à la bonne information de l’acquéreur --- auparavant limitées à la transmission du
	règlement de copropriété --- se sont multipliées ces dernières années : obligation de faire figurer la
	superficie de la partie privative du lot vendu, diagnostic technique obligatoire concernant les parties
	privatives et les parties communes $\dots$
	La loi ALUR a considérablement renforcé cette obligation
	d’information de l’acquéreur, et la place désormais au stade de la promesse de vente : un ensemble
	d’information concernant la situation technique, juridique, et financière de la copropriété doit être
	transmise dès la promesse de vente, et ce n’est qu’à compter de la remise de l’ensemble des éléments
	d’informations que courra le délai de rétractation de l’acquéreur.
	
	Simultanément, la loi ALUR a obligé le notaire à effectuer certaines vérifications avant la vente concernant
	l’acquéreur lui-même, afin de lutter contre les acquéreurs « déstabilisateurs » (cf supra, section II C)
	
	\subsection{Les informations a communiquer a l’acquéreur lors de la promesse de vente}
	
		La liste des informations à communiquer à l’acquéreur résulte de la loi ALUR du 24 août 2014, amendée
		par l’ordonnance numéro 2015-1075 du 27 août 2015 dite de « simplification » qui a simplifié le mode de
		communication de ces informations et a introduit certaines dérogations. Ces obligations figurent dans
		l’\article{L}{721-2} et l’\article{L}{721-3} du Code de la Construction et de l’Habitation.
		
		Les documents et informations visés à l’\article{L}{721-2} ne doivent plus être « annexés » à la promesse de
		vente, ils doivent désormais être remis à l’acquéreur « au plus tard à la date de signature de la
		promesse ».
		
		Ils peuvent être remis à l'acquéreur en amont du compromis de vente, par tous moyens, y compris par un
		procédé dématérialisé », c’est-à-dire par voie électronique « sous réserve de l’acceptation expresse par
		l’acquéreur ». L’acquéreur doit alors attester les avoir reçus.
		
		Si les documents n’ont pas été joints à la promesse de vente, ils doivent être joints à l’acte authentique de
		vente.
		
		Le délai de rétractation, initialement de sept jours, porté à dix jours par la loi \nom{Macron} du 6 août 2015.
		(\article{L}{271-1} modifié du \CCH) ne court qu’à compter du lendemain de la communication de ces
		documents (excepté pour la notice d’information). En d’autres termes, l’omission de l’un de ces
		documents permet à l’acquéreur de se dégager de la promesse de vente, jusqu’au jour de la signature de
		la vente – voire dans les 10 jours suivant cette vente-, sans indemnité, en faisant jouer son droit de
		rétractation.
		
		\subsubsection{Documents concernant l’organisation de la copropriété}
			
			Doivent être communiqués à ce titre
			
			\paragraph{La fiche synthétique de la copropriété}
			
			\par Le syndic est tenu de réaliser une fiche synthétique de la copropriété regroupant les données
			financières et techniques essentielles relatives à la copropriété et à son bâti (art 8-2 modifié de
			la Loi \no 65-557 du 10 juillet 1965), sauf pour les copropriétés autres que d’habitation.
			
			Cette fiche doit être réalisée dans des délais qui varient en fonction de la taille de la copropriété :
			\begin{itemize}
				\item plus de 200 lots de copropriété, à partir du 31 décembre 2016 ;
				\item plus de 50 lots et jusqu'à 200 lots de copropriété, à partir du 31 décembre 2017 ;
				\item jusqu'à 50 lots de copropriété, à partir du 31 décembre 2018.
			\end{itemize}
			
			Cette fiche doit être mise à disposition des copropriétaires qui en font la demande par tous
			moyens, et remise lors de la promesse de vente. Elle peut extraite du registre national des
			copropriétés si la copropriété est déjà immatriculée sur ce registre. Elle doit comporter la date de
			délivrance et la signature du syndic si c’est lui qui l’établit. Elle doit être mise à jour tous les ans,
			dans le délai de 2 mois suivant la notification du procès-verbal de l'assemblée générale au cours
			de laquelle les comptes de l'exercice clos ont été approuvés.
			
			La fiche synthétique est le seul document dont l’absence n’a pas pour effet de reporter le point
			de départ du délai de rétractation. En revanche, Le défaut de réalisation de la fiche synthétique
			est un motif de révocation du syndic. Les contrats de syndic prévoient obligatoirement une
			pénalité financière forfaitaire automatique à l'encontre du syndic chaque fois que celui-ci ne met
			pas la fiche synthétique à disposition d'un copropriétaire dans un délai de quinze jours à compter
			de la demande. Cette pénalité est déduite de la rémunération du syndic lors du dernier appel de
			charges de l'exercice.
			
			Le contenu de la fiche synthétique a été fixé par le Décret \no 2016-1822 du 21 décembre 2016.
			\begin{itemize}
				\item Identification de la copropriété, du syndic ou de l'administrateur provisoire.
					\newline La fiche synthétique doit mentionner :
					\begin{itemize}
						\item le nom d'usage, s'il y a lieu, et l'adresse(s) du syndicat de copropriétaires ;
						\item l'adresse(s) du ou des immeubles (si différente de celle du syndicat) ;
						\item le numéro d'immatriculation de la copropriété et la date de sa dernière mise à jour ;
						\item la date d'établissement du règlement de copropriété et le numéro identifiant d'établissement
						(SIRET) du syndicat ;
						\item le nom, prénom et adresse du représentant légal de la copropriété (syndic ou administrateur
						provisoire) et le numéro identifiant d'établissement (SIRET) du représentant légal ;
						\item le cadre d'intervention du représentant légal (mandat de syndic ou mission d'administration
						provisoire).
					\end{itemize}
			
				\item Organisation juridique de la copropriété
					\newline La fiche synthétique doit mentionner :
					\begin{itemize}
						\item la nature du syndicat (principal/secondaire/coopératif) ou résidence-services ;
						\item s'il s'agit d'un syndicat secondaire, numéro d'immatriculation au registre national des
						copropriétés du syndicat principal du syndicat de copropriétaires.
					\end{itemize}
				
				\item Caractéristiques techniques de la copropriété
					\newline La fiche synthétique doit mentionner :
					\begin{itemize}
						\item le nombre total de lots inscrit dans le règlement de copropriété ;
						\item le nombre total de lots à usage d'habitation, de commerces et de bureaux inscrit dans le
						règlement de copropriété ;
						\item le nombre de bâtiments ;
						\item la période de construction des bâtiments.
					\end{itemize}
				
				\item  Équipements de la copropriété
					La fiche synthétique doit mentionner :
					\begin{itemize}
						\item le type de chauffage et, pour un chauffage collectif (partiel ou total) non urbain le type d'énergie utilisée ;
						\item le nombre d'ascenseurs.
					\end{itemize}
				
				\item  Caractéristiques financières de la copropriété
					\begin{itemize}
						\item les dates de début et de fin de l'exercice comptable et la date de l'assemblée générale ayant approuvé les comptes ;
						\item le montant des charges pour opérations courantes ;
						\item le montant des charges pour travaux et opérations exceptionnelles ;
						\item le montant des dettes fournisseurs, rémunérations et autres ;
						\item le montant des impayés ;
						\item le nombre de copropriétaires débiteurs dont la dette dépasse \montant{300} ;
						\item le montant du fonds de travaux.
					\end{itemize}
			\end{itemize}
			
			Toutefois, les syndicats comportant moins de 10 lots à usage de logements, de bureaux ou de commerces,
			dont le budget prévisionnel moyen sur une période de 3 exercices consécutifs est inférieur à \montant{15 000}, ne
			sont pas tenus de fournir le nombre de copropriétaires débiteurs et le montant des impayés.
			
			\paragraph{L’état descriptif de division et le règlement de copropriété et ses modificatifs}
			Doivent être remis
			\begin{itemize}
				\item le règlement de copropriété et l’état descriptif de division ;
				\item ainsi que tous les actes modificatifs publiés même s’ils ne concernent pas directement les lots
			vendus.
			\end{itemize}
			
			Le notaire devra donc veiller à dénoncer d’éventuels modificatifs de l’état descriptif de division dont le
			syndic pourrait ne pas avoir connaissance (état descriptif de division modificatif consécutif à une division
			de lot, par ex.), en levant une fiche d’immeuble.

			Il devra également veiller à interroger le syndic sur les éventuels modificatifs votés et qui ne seraient pas
			publiés.
			
			\paragraph{Les procès-verbaux des 3 dernières assemblées générales}
			Le vendeur doit communiquer les procès-verbaux des trois dernières années, « sauf s’il n’a pas été en
			mesure de les obtenir du syndic ».
			
			\paragraph{La notice d’information générale sur les droits et obligations du copropriétaire}
			En attente d’un arrêté.
			
		\subsubsection{Les documents concernant la situation technique de l’immeuble}
		
			\paragraph{Le carnet d’entretien de l’immeuble}
			
			\par Le carnet d'entretien est un document obligatoire, qui recense toutes les informations
			permettant le suivi des travaux et des contrats de maintenance concernant l'immeuble et ses
			équipements.
			
			\subparagraph{Informations obligatoires}
			Le carnet d'entretien mentionne :
			\begin{itemize}
				\item l'adresse de l'immeuble,
				\item l'identité du syndic en exercice,
				\item les références des contrats d'assurance souscrits par le syndicat des copropriétaires, avec leurs
				dates d'échéance,
				\item l'année de réalisation des gros travaux (ravalement de façade, réfection de toiture, remplacement
				de chaudière, d'ascenseur ou de canalisations par exemple), et ceux apparaissant nécessaires au
				vu du diagnostic technique global (DTG),
				\item l'identité des entreprises qui ont réalisé ces travaux,
				\item la référence des contrats d'assurance dommage-ouvrage dont la garantie est en cours,
				\item s'ils existent, les références des contrats d'entretien et de maintenance des équipements
				communs (ascenseur, chaudière...) avec leurs dates d'échéance, ainsi que l’échéancier du
				programme pluriannuel de travaux décidé en assemblée générale.
			\end{itemize}
				
			\subparagraph{Informations complémentaires}
			
			\par Le carnet d'entretien doit également mentionner toutes les informations complémentaires que
			les copropriétaires décident d'y faire figurer lors d'un vote en assemblée générale à la majorité
			simple.

			Il peut notamment s'agir d'informations relatives :
			\begin{itemize}
				\item à la construction de l'immeuble ;
				\item aux études techniques réalisées.
			\end{itemize}
		
			La loi \no 2018-1021 du 23 novembre 2018 dite ELAN a généralisé le carnet d’entretien, en l’imposant pour
			tout logement, donc également pour la partie privative du lot. Toutefois, cette obligation généralisée ne
			concerne que les logements neufs (permis postérieur au 1\ier{} janvier 2020), ou les mutations de lots dans
			les immeubles antérieures à compter de 2025.
			
			Ce carnet d’entretien est dématérialisé, sa transmission concernant les parties communes incombe au
			syndic. Toutefois, l’entrée en vigueur du texte est subordonnée à un décret.
			
			Ainsi, l’\article{L}{111-10-5} du \CCH{} dispose désormais :
			\begin{quote}
				\I{} - Il est créé pour tout logement un carnet numérique d'information, de suivi et d'entretien de
				ce logement.
				
				Ce carnet permet de connaître l'état du logement et du bâtiment, lorsque le logement est soumis au
				statut de la copropriété, ainsi que le fonctionnement de leurs équipements et d'accompagner
				l'amélioration progressive de leur performance environnementale. Ce carnet permet l'accompagnement
				et le suivi de l'amélioration de la performance énergétique et environnementale du bâtiment et du
				logement pour toute la durée de vie de celui-ci.
				
				Les éléments contenus dans le carnet n'ont qu'une valeur informative.
				
				Le carnet numérique d'information, de suivi et d'entretien est un service en ligne sécurisé qui regroupe les
				informations visant à améliorer l'information des propriétaires, des acquéreurs et des occupants des
				logements.
				
				L'opérateur de ce service le déclare auprès de l'autorité administrative et assure la possibilité de
				récupérer les informations et la portabilité du carnet numérique sans frais de gestion supplémentaires.
				
				Le carnet numérique intègre le dossier de diagnostic technique mentionné à l'article L. 271-4 et, lorsque
				le logement est soumis au statut de la copropriété, les documents mentionnés à l'article L. 721-2
				
				Le carnet numérique d'information, de suivi et d'entretien du logement est obligatoire pour toute
				construction neuve dont le permis de construire est déposé à compter du 1er janvier 2020 et pour tous les
				logements et immeubles existants faisant l'objet d'une mutation à compter du 1er janvier 2025.
				
				Le carnet numérique d'information, de suivi et d'entretien du logement est établi et mis à jour :
				\newline 1\degre{} Pour les constructions neuves, par le maître de l'ouvrage qui renseigne le carnet numérique
				d'information, de suivi et d'entretien et est tenu de le transmettre à son acquéreur à la livraison du
				logement ;
				\newline 2\degre{} Pour les logements existants, par le propriétaire du logement. Le syndicat des copropriétaires
				transmet au propriétaire les informations relatives aux parties communes.
				
				Le carnet est transféré à l'acquéreur du logement au plus tard lors de la signature de l'acte de
				mutation.
			\end{quote}
			
			\paragraph{Le DPE ou le diagnostic technique global de l'article L. 731-1 s’il a été établi DPE ou DTG ?}
			
			\par S’il est réalisé (il n’est pas encore obligatoire dans toutes les copropriété, voir chapitre TRAVAUX) le
			diagnostic technique global doit être communiqué pour faire partir le délai de rétractation. L’obligation de
			communiquer le plan pluriannuel de travaux normalement établi à la suite.
			
			Le DTG comprend :
			\begin{quote}
				\emph{1\degre{} Une analyse de l'état apparent des parties communes et des équipements communs de l'immeuble ;
				\newline 2\degre{} Un état de la situation du syndicat des copropriétaires au regard des obligations légales et
				réglementaires au titre de la construction et de l'habitation ;
				\newline 3\degre{} Une analyse des améliorations possibles de la gestion technique et patrimoniale de l'immeuble ;
				\newline4\degre{} Un diagnostic de performance énergétique de l'immeuble tel que prévu aux articles L. 134-3 ou L. 134-4-1 du présent code. L'audit énergétique prévu au même article L. 134-4-1 satisfait cette obligation\footnote{
					Article \articleCodifie{L}{134-3} : En cas de vente de tout ou partie d'un immeuble bâti, le diagnostic de
					performance énergétique est communiqué à l'acquéreur dans les conditions et selon les modalités
					prévues aux articles \articleCodifie{L}{271-4} à \articleCodifie{L}{271-6}. Lorsque l'immeuble est offert à la vente ou à la location, le	propriétaire tient le diagnostic de performance énergétique à la disposition de tout candidat acquéreur ou locataire.
				}.
				\newline \medskip
				Il fait apparaître une évaluation sommaire du coût et une liste des travaux nécessaires à la conservation
				de l'immeuble, en précisant notamment ceux qui devraient être menés dans les dix prochaines années.}
			\end{quote}
			
			A défaut, c’est le DPE qui doit être transmis à l’acquéreur. Celui doit avoir été établi dans toutes les
			copropriété depuis le 31 décembre 2017. Il est inclus dans le DTG s’il en a été établi un.
			
			Établi par un diagnostiqueur professionnel, le DPE indique la quantité annuelle d’énergie consommée ou
			estimée pour une utilisation standardisée du bâtiment, ainsi qu’une classification du bâtiment en fonction
			de la quantité d’émission de gaz à effet de serre, le tout afin de connaitre sa performance énergétique. Il
			est accompagné de recommandations du diagnostiqueur destinées à améliorer la performance
			énergétique du bâtiment (\article{L}{134-1} du \CCH). Sa durée de validité est de dix ans.
			
			En cas de vente ou de location de tout ou partie d’un immeuble bâti, le vendeur ou le bailleur a donc
			l’obligation d’annexer à la promesse de vente, à l’acte authentique de vente ou au contrat de bail, un
			diagnostic de performance énergétique (inclus ou non dans le DTG), sauf exceptions prévues par les textes
			(articles L.271-4 du CCH pour la vente et L.134-3-1 du CCH pour la location).
			
			\subparagraph{Quelle est la Valeur du DPE ? (inclus ou non dans le DTG)}
			
			La loi ELAN modifie les articles \articleCodifie{L}{271-4} et \articleCodifie{L}{134-3-1} du \CCH{} afin que
			les informations contenues dans le diagnostic de performance énergétique à compter du 1er janvier 2021
			soient opposables aux vendeurs et aux bailleurs.
			
			Jusqu’alors, le DPE n’avait qu’une valeur informative, si bien que l’acquéreur ne pouvait, en principe, se
			prévaloir à l’encontre du vendeur ou du bailleur des informations qu’il contient. L’acquéreur ou le locataire
			peut en revanche se retourner contre le diagnostiqueur afin d’engager sa responsabilité délictuelle.
			Cependant, les nouveaux articles \articleCodifie{L}{271-4} et \articleCodifie{L}{134-3-1} du \CCH{}, issus de loi ELAN, suppriment le caractère informatif du DPE et rendent ses informations opposables au vendeur et au bailleur.
			
			Autrement dit, le vendeur ou le bailleur engagera sa responsabilité contractuelle envers l’acquéreur ou le
			locataire en cas d’information erronée figurant dans le DPE, à la condition que ladite information erronée
			leur cause effectivement un préjudice pouvant résulter, par exemple, de la perte de chance d’acquérir à
			un prix moindre ou de négocier à la baisse le montant des loyers.
			
			En revanche, les nouveaux articles \articleCodifie{L}{271-4} et \articleCodifie{L}{134-3-1} du \CCH{} prévoient que les recommandations du	diagnostiqueur accompagnant le DPE conserveront un caractère informatif et ne seront pas opposables.
			
			La loi fixe l’entrée en vigueur de ces nouvelles dispositions au 1er janvier 2021 « AFIN DE LAISSER LE
			TEMPS NÉCESSAIRE AU PLAN DE FIABILISATION DES DIAGNOSTICS ENGAGE PAR LE GOUVERNEMENT DE
			PRODUIRE TOUS SES EFFETS » (Rapp. Commission mixte paritaire, 2017-2018, art. 55 bis C). Par
			conséquent, seront opposables les informations contenues dans les DPE établis à compter de cette date.
			
		\subsubsection{Les informations financières}
		
			Les principales informations financières concernant la situation du vendeur et de la copropriété sont
			désormais transmises lors de la promesse de vente –-- au point que les praticiens parlent d’un « pré état
			daté », même si ce terme ne figure nulle part.
			
			Doivent être communiqués :
			\begin{itemize}
				\item  le montant des charges courantes du budget prévisionnel et des charges hors budget prévisionnel
				payées par le copropriétaire vendeur au titre des deux exercices comptables précédant la vente ;
				\item  l'état global des impayés de charges au sein du syndicat et de la dette vis-à-vis des fournisseurs ;
				\item  lorsque le syndicat des copropriétaires dispose d'un fonds de travaux, le montant de la part du
				fonds de travaux rattachée au lot principal vendu et le montant de la dernière cotisation au fonds
				versée par le copropriétaire vendeur au titre de son lot ;
				\item  les sommes qui seront dues au syndicat par l'acquéreur (mais non les sommes restant dues par le
				vendeur, cette obligation a été supprimée en 2015, ce montant étant de toutes façons prélevé sur
				le prix de vente, sans solidarité de l’acquéreur).
			\end{itemize}
			
			Les informations financières concernant le syndicat de copropriété doivent être communiquées à la date
			du dernier arrêté des comptes.
			
		\subsubsection{Formalités simplifiées pour certaines ventes}
			
			\paragraph{Acquéreur déjà copropriétaire dans l'immeuble :}
			N’ont pas à être communiques
			\begin{itemize}
				\item  le règlement de copropriété, l'état descriptif de division,
				\item  les procès-verbaux d'assemblées générales,
				\item  le carnet d'entretien de l'immeuble.
			\end{itemize}
			Seules les informations financières doivent donc lui être fournies.
			
			\paragraph{Lot annexe (parking, cave, grenier, débarras, placard, remise, garage ou cellier)}
			Seules les informations financières de la copropriété ainsi que le règlement de copropriété doivent être
			maintenant fournis à l'acquéreur.
		
	\subsection{Les vérifications imposées pour la lutte contre les acquéreurs déstabilisateurs}
		
		En raison des limitations imposées au droit d’acquérir par la loi ALUR, deux vérifications s’imposent au
		notaire dès le stade de la promesse de vente, ou, en l’absence d’avant contrat, lors de la vente.
		\begin{itemize}
			\item Vérification de l’absence de dettes de charges du candidat acquéreur (art 20 \II{} de la Loi du 10
			juillet 1965), par interrogation du syndic qui, dans le mois suivant le courrier du notaire, doit
			indiquer si le futur acquéreur a déjà un lot dans la copropriété, et si tel est le cas, qu’il n’a pas fait
			l’objet d’une mise en demeure de payer ses charges restée infructueuse plus de 45 jours
			
			\item Vérification, au casier judiciaire, de l’absence de condamnation du candidat acquéreur, à
			l’interdiction de l’article 5\degre{} bis de l’article 225-19 du Code pénal (marchands de sommeil)
			Cf. Supra, Section II C
		\end{itemize}
		
	\subsection[Loi << \nom{Carrez} >>]{Les dispositions légales destinées a faire connaitre la surface réelle des lots vendus. (loi 18 décembre 1996 améliorant la protection des acquéreurs de lots de copropriété dite loi « carrez », modifiant l’article 46 de la loi du 10 juillet 2010)}
		
		L’idée maîtresse de cette loi était la nécessité de protéger l’acquéreur d’un lot immobilier, là où les
		dispositions du code civil sont insuffisantes.
		
		L'article 1583 du code civil dispose que la vente est parfaite s'il y a accord sur la chose et sur le prix. Le
		vendeur est tenu de délivrer la contenance telle que portée au contrat (article 1616 du code civil).
		
		Sauf le cas de vente << à la mesure >>, une marge d'un vingtième en plus ou en moins est tolérée. Au-delà,
		une action en réduction de prix est ouverte à l'acquéreur et une action en augmentation de prix est
		ouverte au vendeur (article 1619 du code civil) ; dans cette dernière hypothèse cependant l'acquéreur a
		la ressource de se désister du contrat en remboursant le prix et les frais (article 1620 du code civil). Ces
		actions sont prescrites par un délai d'un an.
		
		Enfin, les parties ont toujours la liberté de stipuler une clause de non garantie, expressément prévue à
		l'article 1619 précité du code civil.
		
		Jusqu’à l’entrée en vigueur de cette loi, la superficie de l'immeuble vendu n'était pratiquement jamais
		mentionnée dans l'acte définitif. Quand bien même la superficie était mentionnée dans l'acte authentique,
		l'acquéreur se heurtait à la clause de non-garantie, devenue systématique : << l'immeuble est vendu dans
		son état actuel, sans garantie de la contenance indiquée, la différence avec celle réelle, même supérieure
		à 1/20\ieme{} devant faire le profit ou la perte de l'acquéreur >>. Cette clause est considérée comme valable par
		une jurisprudence constante.
		
		Cette absence de garantie paraît tout à fait anormale dans les grandes villes où les appartements sont
		vendus à un prix calculé par référence à la valeur au \metreCarre. Iniquité d'autant plus flagrante, qu'à l'opposé, les
		acquéreurs d'appartements neufs sont privilégiés. En effet, les articles L. 26 1 -11 et suivants du code de
		la construction et de l'habitation applicables à la vente en l'état futur d'achèvement dans le secteur
		protégé (habitation ou habitation et professionnel) prévoient la mention de la superficie de l'immeuble
		vendu et dans le contrat préliminaire, et dans le contrat définitif.
		
		Le législateur a donc choisi d’instaurer une protection spécifique de l’acquéreur d’un lot en copropriété
		quant à la surface vendue, et a inséré cette disposition dans la loi sur la copropriété\footnote{La loi ALUR avait ajouté l’obligation d’indiquer en outre la « surface habitable » du lot. Cette ineptie législative a été supprimée par	la loi du \no 2014-1545,du 20 déc. 2014.}.
		
		\paragraph{Article 46 de la loi du 10 juillet 1965}
		
		\begin{quote}
			{\emph Toute promesse unilatérale de vente ou d'achat, tout contrat réalisant ou constatant la vente d'un lot ou
			d'une fraction de lot mentionne la superficie de la partie privative de ce lot ou de cette fraction de lot. La
			nullité de l'acte peut être invoquée sur le fondement de l'absence de toute mention de superficie.
			
			Cette superficie est définie par le décret en Conseil d'État prévu à l'article 47.
			
			Les dispositions du premier alinéa ci-dessus ne sont pas applicables aux caves, garages, emplacements de
			stationnement ni aux lots ou fractions de lots d'une superficie inférieure à un seuil fixé par le décret en
			Conseil d'État prévu à l'article 47.
			
			Le bénéficiaire en cas de promesse de vente, le promettant en cas de promesse d'achat ou l'acquéreur peut
			intenter l'action en nullité, au plus tard à l'expiration d'un délai d'un mois à compter de l'acte authentique
			constatant la réalisation de la vente.
			
			La signature de l’acte authentique constatant la réalisation de la vente mentionnant la superficie de la
			partie privative du lot ou de la fraction de lot entraîne la déchéance du droit à engager ou à poursuivre une
			action en nullité de la promesse ou du contrat qui l’a précédé, fondée sur l’absence de mention de cette
			superficie.
			
			Si la superficie est supérieure à celle exprimée dans l’acte, l’excédent de mesure ne donne lieu à aucun
			supplément de prix.
			
			Si la superficie est inférieure de plus d’un vingtième à celle exprimée dans l’acte, le vendeur, à la demande
			de l’acquéreur, supporte une diminution du prix proportionnelle à la moindre mesure.
			
			L’action en diminution du prix doit être intentée par l’acquéreur dans un délai d’un an à compter de l’acte
			authentique constatant la réalisation de la vente, à peine de déchéance.}
		\end{quote}
	
		La loi utilise le terme de << superficie >> et non celui de << surface >> (par référence à la surface habitable du code
		de la construction et de l'habitation) ou de << contenance >> (par référence à la garantie de contenance du
		code civil) pour harmoniser la loi nouvelle avec les autres dispositions de la loi sur la copropriété.
		
		Cette superficie est définie par le Décret qui a modifié le Décret du 17 mars 1967, articles 4-1 à 4-3,
		s’inspirant largement de la définition du \CCH{} à propos de la SHON.
			\begin{itemize}
				\item La superficie des planchers des locaux clos et couverts après déduction des surfaces occupées par
					les murs, cloisons, marches et cages d'escaliers, gaines, embrasures de portes et de fenêtres.
				\item Il n'est pas tenu compte des planchers des parties des locaux d'une hauteur inférieure à 1,80 mètre.
				\item Les lots ou fractions de lots d'une superficie inférieure à 8 mètres carrés n’étant pas pris en
					compte pour le calcul de la superficie (art. 4-2 du décret).
			\end{itemize}
		
		\subsubsection{Champ d’application : les actes concernés}
		
			\paragraph{La loi ne s’applique qu’aux lots de copropriété et ne s’applique pas en \VEFA{}}
			
			\par Il a été jugé que si par suite de la réunion de tous les lots entre les mains d’une seule personne, il n’y a plus
			de copropriété et dès lors il n’y a plus obligation, lors de la revente de procéder à un mesurage des anciens
			lots au titre de la loi Carrez\footnote{Civ. 3ème Ch. 28 janvier 2009, \no 06-19650}.
			
			Par ailleurs, après nombre de discussions et arrêts contradictoires des cours d’appel, la cour de cassation
			considère que les dispositions de la loi Carrez ne s’appliquent pas à la vente en \VEFA\footnote{3\ieme{} Ch. 11 janvier 2012, \no 10-22.924 au Bulletin}.
			
			Par contre, et à défaut de clause de non garantie insérée à l’acte de vente, les dispositions des articles
			1619 et s. sur la vente à la mesure et la faculté de réduction de prix dans le délai d’un an du jour du contrat
			s’applique à la \VEFA, sous réserve que le point de départ de ce délai de déchéance est la date de remise
			des clés\footnote{3\ieme{} Ch. Civ. 24 nov 1999 JCP 2000, I, 237 ; 3\ieme{} Ch. Civ. 8 oct 2013, Pourvoi \no 12-23275, non publié}
			
			\paragraph{Obligation de mentionner la « superficie » dans l’avant contrat et dans l’acte de vente}
			
			\par Le législateur a tenu à imposer cette mention le plus en amont possible, car elle est de nature à influencer
			le consentement définitif (obligation précontractuelle de renseignement).
			
			La superficie doit être mentionnée dans les avant-contrats. L'article 46 vise expressément les << promesses
			unilatérales de vente ou d'achat >>.
			
			La superficie doit être également mentionnée dans << tout contrat réalisant ou constatant la vente d'un lot >>.
			
			A ce titre, la superficie aurait sans doute dû être mentionnée dans le congé valant offre de vente dans le
			cadre du droit de préemption reconnu au locataire par la loi du 6 juillet 1989, complétée par la loi du 21
			juillet 1994. Toutefois la loi SRU a dispensé le bailleur de cette obligation $\dots$ à titre rétroactif :
			\begin{quote}
				Cf. Art. 15 de la loi du 6 juillet 1989 modifié
			
				« Les dispositions de l'article 46 de la loi no 65-557 du 10 juillet 1965 fixant le statut de la copropriété des	immeubles bâtis ne sont pas applicables au congé fondé sur la décision de vendre le logement. »
			\end{quote}
	
			La même dispense ne s’applique pas à l'offre de vente au locataire en cas de première vente après division
			ou subdivision de l'immeuble, prévue par l'article 10 de la loi du 31 décembre 1975 alors même que la
			vente sera parfaite du seul fait de l'acceptation du locataire
			
			\paragraph{Les modalités de la remise du « certificat carrez »}
			
			\par La mention de la superficie se fait par la remise par le notaire « contre émargement ou récépissé, une copie
			simple de l'acte signé ou un certificat reproduisant la clause de l'acte mentionnant la superficie de la partie
			privative du lot ou de la fraction du lot vendu, ainsi qu'une copie des dispositions de l'article 46 de la loi du
			10 juillet 1965 lorsque ces dispositions ne sont pas reprises intégralement dans l'acte ou le certificat. »
			(art.4-3 D)
			Le vendeur est libre de procéder comme il l'entend au métrage de son lot : il peut le mesurer lui-même.
			En pratique, la surface est « offerte » par l’agent immobilier chargé de la vente du lot ou le propriétaire
			fait le plus souvent appel à un géomètre expert\footnote{
				La cour de cassation (3\ieme Civ. - 21 juin 2006) a précisé que Le mesurage de la superficie de la partie privative d'un lot de copropriété en application de l'article 46 de la loi du 10 juillet 1965, qui est une prestation topographique n'ayant pas pour objet la délimitation des	propriétés, ne relève pas de la compétence exclusive des géomètres experts
			}, un métreur vérificateur ou à un architecte.
			
		\subsubsection{Les lots concernés par l'obligation de métrage}
			
			\paragraph{Exclusion des lots ou fraction de lots de moins de \surface{8}}
			
			\par La loi vise la << vente d'un lot ou d'une fraction de lot >>. En pratique la fraction de lot désigne le lot lui même $\dots$
			
			Un arrêt\footnote{3\ieme{} Ch Civ. 28 janvier 2015 \no 13-26035, au Bulletin} du 28 janvier 2015 apporte une réponse sans ambiguïté à une question à portée générale :
			qu’est-ce qu’un lot ou une fraction de lot inférieure à \surface{8} ? Doit-on désigner par là une partie du lot située à l’extérieur du lot ou toute partie du lot inférieure à \surface{8} ? En l’espèce le lot était désigné comme comportant deux loggias (l’une de \surface{6,27} et l’autre de \surface{6,69}). Ces loggias incluses dans le lot étaient	donc fermées. L’acquéreur affirme que ces loggias étant chacune d’une superficie inférieure à \surface{8}	auraient dû être exclues du calcul de la surface vendue. La cour de cassation répond que « \emph{la cour d’appel ayant constaté que les deux loggias étaient incluses dans le lot et qu’elles étaient closes et habitables en a déduit à bon droit que ces loggias devaient être prises en compte pour le calcul de la superficie des parties	privatives vendues} ».
			
			La superficie du lot doit être indiquée quelle que soit la nature de celui-ci : lot d'habitation, lot
			professionnel ou commercial ou même lot industriel ou $\dots$ transitoire.
			
			\paragraph{Exclusion des lots accessoires}
			
			L'obligation de métrage est exclue pour les lots dits << accessoires >>, c'est-à-dire, selon les termes mêmes
			de la loi, les lots désignés comme : << caves, garages, emplacements de stationnement >> ou les lots ou
			fractions de lots d'une superficie inférieure à 8 \metreCarre{} » (la liste est limitative).

			Cette dispense de métrage s'applique aussi bien lorsque le vendeur cède une cave à titre exclusif, que
			lorsqu'il cède une cave à titre accessoire à son appartement.
			
			Toutefois, le lot « cave » a pu être transformé par son occupant et devenir de la sorte un complément du
			lot principal. En ce cas, et contrairement à ce qu’a pu juger par le passé la Cour de Paris, la superficie de la
			cave ne doit pas être écartée, mais au contraire comprise dans la loi Carrez. C’est ce que décide la cour de
			cassation\footnote{Civ. 3\ieme{} Ch. 2 octobre 2013, \no de pourvoi: 12-21918, au Bulletin} dans un arrêt du 2 octobre 2013 :
			\begin{quote}
				\emph{Ayant exactement retenu que pour l'application de l'article 46 de la loi du 10 juillet 1965, il y avait
				lieu de prendre en compte le bien tel qu'il se présentait matériellement au moment de la vente, la
				cour d'appel, qui, procédant à la recherche prétendument omise, a souverainement estimé que le
				local situé au sous-sol, annexe de la pièce du rez-de-chaussée à laquelle il était directement relié,
				n'était plus une cave comme l'énonçaient le règlement de copropriété et l'acte de vente mais avait
				été aménagé et transformé en réserve, et qui n'était pas tenue de répondre à un moyen inopérant
				relatif au caractère inondable de ce sous-sol, en a déduit à bon droit que ce local devait être pris
				en compte pour le calcul de la superficie des parties privatives vendues.}
			\end{quote}
			
			Il en ira de même si le copropriétaire a transformé ses caves en bureaux, ceci quand bien même aucune
			autorisation de l’Administration n’a été sollicitée\footnote{civ. 3\ieme{} Ch. 7 février 2012, Pourvoi \no 11-11.608, F-D}.
			
			Mais l’inverse n’est pas toujours juste : en l’espèce le propriétaire d’un local d’habitation transforme celui-ci
			en garage pour sa voiture de collection $\dots$ La Cour de Cassation\footnote{Arrêt précité du 7 février 2012} retient, nonobstant cette
			transformation, que la surface du lot doit être incluse dans la superficie de la loi \nom{Carrez} ) au double motif
			:
			\begin{itemize}
				\item de première part que le lot est toujours défini comme un local d'habitation dans le règlement de
				copropriété ;
				\item de seconde part que l'acquéreur peut toujours et à tout moment réaffecter ce lot à usage
				d'habitation.
			\end{itemize}
			
			\paragraph{Les surfaces à prendre en compte}
			
			Doit être mesurée « la superficie privative des planchers des locaux clos et couverts après déduction des
			surfaces occupées par les murs, cloisons, marches et cages d'escaliers, gaines, embrasures de portes et de
			fenêtres ».
			
			Deux hypothèses doivent être envisagées :
			\begin{itemize}
				\item .. Lorsque le lot a été modifié dans sa consistance sans annexion de parties communes (création
				d’une mezzanine par exemple\footnote{
					civ., 3e ch., 13 avril 2005, \no 03-21004 et 03-21015 ; Contra voir cependant les arrêts de la CA de Paris du 20 septembre 2007 et 5
					juin 2012 (RG 10/24914) Sten Mickael X… c/ SARL LOFT, inédit qui considère que la mezzanine n’avait pas d’existence juridique faute
					d’autorisation administrative et d’autorisation de l’assemblée générale).
				}, ou encore transformation d’un abris non clos en local clos et couvert\footnote{civ. 3\ieme{} Ch. 6 septembre 2011 – \no 994 F-D}, il y a lieu de prendre en compte le bien tel qu'il se présentait matériellement au	moment de la vente.
				
				\item Le lot a été modifié par incorporation de parties communes, sans modification de l’état descriptif
				de division : en ce cas il n’y a pas à tenir compte des surfaces irrégulièrement annexées.
			\end{itemize}
			
		\subsubsection{Les sanctions}
			
			\paragraph{La vente ou la promesse de vente est nulle, faute de mention de la superficie}
			
				\subparagraph{S’agissant de la promesse de vente.}
				
				L’absence de mention de la superficie du lot objet de la promesse de vente est sanctionnée par la nullité
				de la promesse de vente, sauf régularisation lors de la signature de l’acte authentique de vente :
				\begin{quote}
					On ne peut rattraper l’omission de la superficie dans la promesse de vente par la signature postérieure
					d’un acte complémentaire « indissociable » de la promesse de vente\footnote{civ. 3\ieme{} Ch. 14 mars 2019, \no 18-10214 – JCP N 2019, \no 13, act. 338.}.
				\end{quote}
				En sorte que l’acquéreur pourra refuser de régulariser l’acte authentique.
				
				\subparagraph{S’agissant de l’acte authentique de vente.}
				
				\par En premier lieu et aux termes de l’alinéa 5 de l’article 46 :
				\begin{quote}
					« \emph{La signature de l'acte authentique constatant la réalisation de la vente et mentionnant la
				superficie entraîne la déchéance du droit à engager ou à poursuivre l'action en nullité de la promesse
				ou du contrat qui l'a précédé, fondée sur l'absence de mention de superficie} »
				\end{quote}
				
				En sorte que si l’acquéreur, nonobstant l’omission de la superficie dans la promesse de vente, accepte de
				signer l’acte authentique avec mention de la superficie, il perd le droit d’agir en annulation.
				
				\bigskip
				En second lieu, si la mention est omise dans l’acte authentique, l’acquéreur pourra demander au juge
				d’annuler la vente (ceci quand bien même la superficie aurait-elle figuré dans la promesse de vente).
				
				Le délai d'action est d'un mois au plus à compter de << l'acte authentique constatant la réalisation de la
				vente >>, ceci afin de limiter le contentieux et la durée de fragilité du contrat. Ce délai est certainement un
				délai préfixe, seulement susceptible d'interruption par voie d’assignation en justice (articles 2242 à 2250
				du code civil).
				
				Le but de cette règle est de limiter le contentieux en faisant jouer au notaire un rôle de << filtre >> : il ne
				laissera pas passer un acte sans mention de superficie.
			
			\paragraph{Sanction de l'erreur sur la superficie dans l'acte authentique : l'action en	réduction de prix}
			
				\begin{quote}
					<< \emph{Si la superficie est inférieure de plus de 1/20\ieme{} à celle exprimée dans l'acte, le vendeur, à
					la demande de l'acquéreur, supporte une diminution de prix proportionnelle à la moindre	mesure.} »
				\end{quote}
				
				Il n'existe pas de sanction si la superficie réelle est supérieure à celle mentionnée dans l'acte authentique,
				contrairement aux dispositions du code civil.
				
				L'action en réduction de prix n'est ouverte que si la différence entre la superficie mentionnée et la
				superficie réelle est supérieure à 1/20\ieme, marge de tolérance reprise du code civil. En effet, il est souhaitable	que l'action ne soit ouverte que dans les cas où l'acquéreur supporte une perte financière substantielle. D'autre part, cette tolérance permet un métrage par le vendeur lui-même.
				
				Pour déterminer les conditions d'ouverture de l'action, il faut calculer cette différence en nature et non
				en valeur, contrairement au code civil\footnote{
					Soit un appartement acheté \montant{300 000} pour une superficie de 100 \metreCarre{} et en conséquence un prix du \metreCarre{} de \montant{30 000}. Si la superficie
					réelle est seulement de 92 \metreCarre, donc inférieure de plus de \pourcent{5} à la surface mentionnée à l’acte de vente, le vendeur sera débiteur de 100-
					92=8\metreCarre{} x \prixSurface{30 000} = \montant{240 000}
				}.
				
				Le délai d'action est d'un an à compter de l'acte authentique constatant la réalisation de la vente, à peine
				de déchéance. C'est une transposition des solutions de droit commun. Il s'agit donc là encore d'un délai
				préfix.
				
				L’action est ouverte à l’acquéreur quand bien même aurait-il acquis en sachant que la surface portée à
				l’acte était supérieure à la surface Carrez : la bonne ou la mauvaise foi n’a pas à être recherchée dans une
				action en réduction du prix\footnote{Cour de Cass. Ch. civ 3 - 10 décembre 2015 Numéro de pourvoi 14-13.832 Non publié au Bulletin (\no 1403 FS-PB)}.
				
				La cour de cassation impose au juge du fond de caractériser la nature des surfaces déduites, en application
				de l'art. 4-1 du décret \no 67-223 du 17 mars 1967 ; il ne peut donc se contenter de la certification de
				superficie établie par le cabinet d'expertise faisant apparaître que la superficie au sens de l'art. 46 telle
				que définie par l'art. 1er du décret \no 97-532 du 23 mai 1997 est inférieure de plus d'un vingtième à celle
				exprimée dans l'acte\footnote{Civ 3\ieme{}, 7 nov 2001 – Dalloz 2003 somm p. 1332}.
			
			\paragraph{Exclusion de l’action en garantie des vices cachés}
			
			Quand bien même les parties communes extérieures aux lots vendus ne doivent pas entrer dans le calcul
			de la loi Carrez, il arrive que le « technicien » tienne compte des surfaces « absorbées » dans son calcul.
			L’acquéreur peut-il alors prétendre avoir un titre sur ces parties communes ainsi privatisées ? La cour de
			cassation\footnote{Civ. 3\ieme{} Ch. 8 oct. 2013, \no de pourvoi: 12-19854 , non publié} donne une réponse dans les relations entre vendeur et acquéreur, l’acquéreur s’étant
			retourné vers le vendeur sur la base du vice caché au motif que celui-ci lui avait dissimulé que les surfaces
			absorbées (paliers et escalier) n’étaient pas privatives : « Attendu qu'ayant, par motifs adoptés, retenu
			que le fait que le rez-de-chaussée, les paliers, et les escaliers intérieurs constituaient des parties
			communes n'avait pas été caché à M. X... puisque l'acte authentique ne les mentionnait pas
			comme lots vendus, la cour d'appel procédant à la recherche prétendument omise a légalement
			justifié sa décision de rejeter la demande fondée sur la garantie des vices cachés »
			
			D) SORT DES FRAIS ACCESSOIRES.
			
			L’acquéreur a supporté un prix supérieur au prix qu’il aurait dû payer. Il a en conséquence payé des droits
			et émoluments supérieurs à ce qu’il aurait dû payer dès lors que ces droits et émoluments ont pour
			assiette le prix de vente du lot. Peut-il demander au vendeur le remboursement de ces frais
			complémentaires ? La cour de cassation répond négativement\footnote{
				Civ.3ème - 22 novembre 2006 : « Viole les dispositions de l’article 46, alinéa 7, de la loi du 10 juillet 1965 une cour d’appel qui, sur ce fondement, condamne le vendeur d’un appartement dans un immeuble en copropriété à payer à l’acquéreur le montant de frais afférents au surplus indu du prix de vente >>
			} :
			
		\subsubsection{Le recours en garantie contre les professionnels}
			
			L’action de la loi Carrez est une action en restitution du prix. Dès lors, même si l’erreur de calcul est
			imputable à un professionnel, le vendeur ne saurait lui demander garantie des sommes qu’il a dû
			rembourser à l’acquéreur.
			
			« La restitution à laquelle le vendeur est condamné à la suite de la diminution du prix prévue par l’article
			46, alinéa 7, de loi du 10 juillet 1965, résultant de la délivrance d’une moindre mesure par rapport à la
			superficie convenue, ne constitue pas un préjudice indemnisable. Elle ne peut, dès lors, donner lieu à
			garantie de la part du professionnel de mesurage »\footnote{3e Civ. - 25 octobre 2006}.
			
			La diminution du prix de vente dû à la méconnaissance des dispositions de la loi Carrez ne constitue pas
			un préjudice indemnisable par le notaire.
			Cependant, celui-ci peut être tenu à indemniser l’acquéreur en tant que débiteur subsidiaire lorsque le
			vendeur est insolvable\footnote{Cass. Civ. 1e 25 mars 2010}.
			S’agissant du technicien auquel avait été simplement confié la mission de déterminer la superficie privative
			du lot objet de la vente, en l’occurrence une société spécialisée en diagnostics, on citera un arrêt\footnote{Civ. 3\ieme{} Ch. 18 septembre 2013 \no 12-24077 inédit} du 18 septembre 2013 qui casse un arrêt de la cour de Bordeaux ayant condamné ce technicien à indemniser le
			vendeur pour avoir pris en compte la superficie de locaux d'un immeuble attenant ne faisant pas partie de
			la copropriété : « la société X n'était pas tenue, dans le cadre de la mission de mesurage qui lui avait été
			confiée, de procéder à l'analyse juridique du lot objet de la vente, la cour d'appel a violé les textes
			susvisés ».
			
			A) UNE PREMIERE EVOLUTION JURISPRUDENTIELLE : VERS L’INDEMNISATION D’UNE PERTE DE	CHANCE
			
			Un arrêt de la cour de cassation\footnote{Civ 3\ieme{} Ch 2 juillet 2014, pourvoi: 12-26619, Non publié} du 2 juillet 2014, bien que non publié au Bulletin, paraît apporter
			un assouplissement à sa position antérieure quant à l’indemnisation du vendeur tenu à restitution
			du « surplus du prix » par suite d’une erreur de calcul de surface par le professionnel :
			Mais attendu qu'ayant retenu que si la preuve d'une faute de la société François C... lors des opérations
			de calcul de la superficie des appartements vendus se trouvait rapportée, il n'existait aucun lien de
			causalité direct entre cette faute et le préjudice invoqué par les consorts Y...- X..., la cour d'appel, qui n'a
			pas modifié l'objet du litige et qui a statué sur les dernières conclusions déposées en retenant que les
			consorts Y...- X... demandaient l'indemnisation d'une « perte de surface » et non pas d'une perte de
			chance, a pu en déduire, abstraction faite d'un motif erroné mais surabondant tenant au fait que la perte
			de surface alléguée avait pour cause exclusive le défaut d'exercice de l'action prévue par l'article 46 de
			la loi du 10 juillet 1965, que la demande en dommages-intérêts des consorts Y...- X... à l'égard de la
			société François C... devait être rejetée ;
			Mais il est vrai, dans le cas analysé, que d’une part l’action n’avait pas été introduite par le vendeur mais
			par l’acquéreur, et que d’autre part l’acquéreur n’avait pas agi en restitution de prix contre le vendeur
			dans le délai d’un an.
			
			B) LA CONSECRATION PAR LA COUR DE CASSATION DU DROIT A INDEMNITE POUR PERTE DE
			CHANCE..
			
			Un arrêt\footnote{chambre civile 3 , 28 janvier 2015 , \no de pourvoi: 13-27397, Publié au bulletin} du 28 janvier 2015, publié au Bulletin, nous apprend qu’en définitive tout est question de
			rédaction de la demande.
			En l’espèce le vendeur n’a pas demandé condamnation du professionnel (le métreur) à restitution de la
			différence de prix … mais à Dommages et Intérêts pour perte de chance de vendre le bien au même prix !
			La Cour d’Appel ayant fait droit à la demande du vendeur sur cette base de la perte de chance, ce dernier
			régularise un pourvoi … rejeté par la cour de cassation !
			« Mais attendu qu'ayant retenu, à bon droit, que, si la restitution, à laquelle le vendeur est tenu en
			vertu de la loi à la suite de la diminution du prix résultant d'une moindre mesure par rapport à la
			superficie convenue, ne constitue pas, par elle-même, un préjudice indemnisable permettant une
			action en garantie, le vendeur peut se prévaloir à l'encontre du mesureur ayant réalisé un mesurage
			erroné, d'une perte de chance de vendre son bien au même prix pour une surface moindre, la cour
			d'appel a souverainement apprécié l'étendue du préjudice subi par Mme X.. »
			
			Certes les Tribunaux considèrent habituellement qu’une perte de chance ne peut permettre une
			indemnisation totale (sauf à considérer que la « chance » était quasiment certaine). Mais récupérer –-- ne
			serait-ce que \pourcent{50} --- de la différence de prix ce n’est déjà pas trop mal $\dots$ et en fait sous couvert d’une qualification juridique distincte, nous sommes en présence d’un véritable revirement de jurisprudence.
	
\section{Les formalités lors de cession du lot}

	Il convient que le syndic soit informé du changement du titulaire du lot afin qu'il puisse s'adresser à
	l'acquéreur en tant que copropriétaire dans les différents actes de la vie de la copropriété. Par ailleurs, le
	syndic devra être avisé de la vente pour tenter de récupérer sur le prix les sommes restant dues au syndicat
	par le vendeur.
	
	Enfin, cédant et cessionnaire doivent être renseignés exactement sur les dépenses déjà engagées par le
	syndicat, mais qui ne constituent pas encore des créances liquides et exigibles : cette information
	permettra d'établir un juste prix.
	A. NOTIFICATION AU SYNDIC DE TOUT TRANSFERT DE PROPRIETE D'UN LOT OU D'UNE
	FRACTION DE LOT (ARTICLE 6 AL 1 ET 3 DU DECRET DU 17 MARS 1967).
	Le syndic doit impérativement être informé de tout mutation (à titre onéreux ou gratuit, particulier ou
	universel) afin qu’il connaisse l'identité du nouveau copropriétaire, le convoque aux assemblées et appelle
	les charges auprès de lui. C’est la fonction de la « notification » de l’article 6 du décret.
	" Tout transfert de propriété d'un lot ou d'une fraction de lot, toute constitution sur ces derniers d'un droit
	d'usufruit, de nue propriété, d'usage ou d'habitation, tout transfert de l'un de ces droits est notifié, sans
	délai, au syndic, soit par les parties, soit par le notaire qui établit l'acte, soit par l'avocat ou soit par l’avoué
	qui a obtenu la décision judiciaire, acte ou décision qui suivant le cas, réalise, atteste, constate ce transfert
	ou cette constitution.
	Cette notification doit être faite indépendamment de l'avis de mutation prévu à l'article 20 de la loi du 10
	juillet 1965 modifiée ".
	1. Contenu (art. 6 alinéa 2) :
	droit de la copropriété année 2018-2019
	194
	"Cette notification comporte la désignation du lot ou de la fraction de lot ainsi que l'indication des nom,
	prénom, domicile réel ou élu de l'acquéreur et, le cas échéant, du mandataire commun prévu à l'article 23
	de la loi."
	Elle est faite sous forme de lettre recommandée avec avis de réception ou par télécopie avec récépissé
	depuis le 1er avril 2007(s’agissant d’une notification, il y a lieu d’rappliquer les dispositions de l’article 64
	du Décret).
	S'agissant du "transfert de propriété", c'est en principe l'acte sous seing privé qui a cet effet translatif,
	c'est-à-dire qu'il faudrait notifier le compromis209 En pratique cependant, la notification n'est souvent faite
	qu'après la passation de l'acte authentique : les actes sous seing privé sont le plus souvent assortis de
	conditions suspensives qui ne seront définitivement levées qu’au jour de la signature de l’acte
	authentique.
	2. Conséquences de la notification.
	Seule la notification d’une mutation opérée selon le formalisme prévu par l’article 6 du décret de 1967
	rend cette mutation opposable au syndicat des copropriétaires quand bien même le syndic aurait eu
	connaissance de la vente par d’autres moyens210
	Le syndic pourra mettre à jour la liste des copropriétaires qui devra désormais comprendre l'acquéreur
	tant pour la gestion documentaire et comptable que pour les convocations aux assemblées.
	Art. 32. – Le syndic établit et tient à jour une liste de tous les copropriétaires avec l'indication des lots qui
	leur appartiennent, ainsi que de tous les titulaires des droits visés à l'article 6 ci-dessus ; il mentionne leur
	état civil ainsi que leur domicile réel élu.
	A défaut de notification de la vente, le vendeur est toujours copropriétaire au regard du syndicat des
	copropriétaires, en sorte qu’il sera valablement convoqué aux assemblées générales et tenu au paiement
	des charges.
	B. L’AVIS DE MUTATION DONNE AU SYNDIC EN VUE DE FORMER OPPOSITION AU
	VERSEMENT DES FONDS (ARTICLE 20 DE LA LOI DU 10 JUILLET 1965)
	209 Aix, 4ème ch., 23 mars 1983, Bull. Aix Janvier 1983 \no30 p.46.
	210 Civ.3ème 22 mars 2000, Civ.3ème 26 sept 2007
	droit de la copropriété année 2019-2020
	195
	"Lors de la mutation à titre onéreux d'un lot, et si le vendeur n'a pas présenté au notaire un certificat du
	syndic ayant moins d'un mois de date, attestant qu'il est libre de toute obligation à l'égard du syndicat, avis
	de la mutation doit être donné par le notaire au syndic de l'immeuble, par lettre recommandée avec avis
	de réception. Avant l'expiration d'un délai de quinze jours à compter de la réception de cet avis, le syndic
	peut former, au domicile élu, par acte extrajudiciaire, opposition au versement des fonds, dans la limite ciaprès
	pour obtenir le paiement des sommes restant dues par l'ancien propriétaire. Cette opposition
	contient élection de domicile dans le ressort du tribunal de grande instance de la situation de l'immeuble.
	Et, à peine de nullité, énonce le montant et les causes de la créance. Les effets de l'opposition sont limités
	au montant ainsi énoncé.
	Tout paiement ou transfert amiable ou judiciaire du prix opéré en violation des dispositions de l'alinéa
	précédent est inopposable au syndic ayant régulièrement fait opposition.
	L'opposition régulière vaut au profit du syndicat mise en oeuvre du privilège mentionné à l'article 19-1 ".
	L'avis de mutation prévu par l'article 20 de la loi a une finalité très précise. Il doit permettre au syndic,
	avant que le prix de vente ne soit versé au vendeur, de paralyser un tel versement afin d'obtenir paiement
	des sommes restant dues au syndicat par le cédant.
	Il ne concerne donc que les mutations à titre onéreux, par opposition à la notification de l’article 6 du
	décret.
	1. Le processus de l'article 20 de la loi.
	Cet avis et l'opposition éventuelle qu'il peut susciter s'inscrivent dans un processus comportant plusieurs
	phases et alternatives.
	1. Le vendeur doit présenter au notaire un certificat ayant moins d'un mois de date, attestant
	qu'il est libre de toute obligation à l'égard du syndicat. Si tel est le cas, l'acte de vente est
	passé sans autres formalités. En pratique, c'est le notaire qui demande le certificat au
	syndic211.
	211 Cour d’Appel de PARIS – 27 mars 2002 : cet arrêt a rappelé que l’avis doit être envoyé au syndic avant la signature
	de l’acte de vente et en tout état de cause avant la remise du prix au vendeur au risque de remettre des sommes indues
	au vendeur débiteur de la copropriété.
	droit de la copropriété année 2018-2019
	196
	2. A défaut pour le vendeur de pouvoir présenter ce certificat, avis de mutation doit être
	donné au syndic, par lettre recommandée à la diligence du notaire (le texte antérieur à la
	réforme du 21 juillet 1994 disait "à la requête de l'acquéreur").
	3. La réception de cet avis fait courir un délai de quinze jours (au paravent il n'était que de
	huit jours, délai très court qui explique la modification légale), au cours duquel le syndic
	peut, par acte d'huissier, former opposition au versement des fonds pour obtenir le
	paiement des sommes liquides et exigibles restant dues par l'ancien propriétaire. Cette
	opposition devra mentionner, à peine de nullité, le montant et les causes de la créance.
	2. Effets de l’opposition
	Initialement, cette opposition rendait indisponible la totalité du prix de vente entre les mains de
	l'acquéreur ou, plus précisément, du notaire ; si bien que tout paiement était en principe inopposable au
	syndicat. La loi du 21 juillet 1994 limite le "blocage" du prix au seul montant de l'opposition pratiquée par
	le syndic.
	Depuis cette même loi, l'opposition régulièrement pratiquée confère un véritable privilège au Syndicat des
	Copropriétaires, privilège plaçant le syndicat en tête de tous les créanciers sur les immeubles, juste après
	les salariés et les frais de justice qui bénéficient d'un privilège général (sur les meubles et immeubles)..
	L'article 5-1 du Décret d'Application de la loi exige que l'opposition du syndic énonce de manière précise
	le montant et les causes de la créance, en distinguant les sommes dues selon les différents rangs de sûreté
	dont le syndicat est susceptible de bénéficier :
	- Créances du syndicat afférentes aux charges et travaux de l'année courante et des deux
	dernières années échues.
	- Créances du syndicat afférentes aux deux années antérieures aux deux dernières années
	échues.
	- Créances de toute nature du syndicat garanties par une hypothèque légale et non comprises
	dans les créances privilégiées ci-dessus. (En pratique cela devrait concerner les hypothèques
	légales prises pour la période remontant à plus de quatre ans).
	- Créances de toute nature non comprises dans les créances ci-dessus.
	3. Auteur de l’opposition
	L’article 5-1 du Décret précise :
	" Si le lot fait l'objet d'une vente devant le tribunal sur licitation ou sur saisie immobilière, l'avis de mutation
	prévu à l'article 20 est donné au syndic par le notaire ou l'avocat du demandeur ou du créancier
	poursuivant; s'il fait l'objet d'une expropriation pour cause d'utilité publique ou de l'exercice d'un droit de
	préemption, l'avis de mutation est donné au syndic par l'autorité expropriante, le titulaire du droit de
	droit de la copropriété année 2019-2020
	197
	préemption ou le notaire. Si l'acte est reçu en la forme administrative, l'avis de mutation est donné au
	syndic par l'autorité qui authentifie la convention
	C. LES INFORMATIONS A FOURNIR A L’ACHETEUR LORS DE LA CESSION
	De nouvelles formalités sont imposées lors de la signature de l’acte authentique de vente : la dénonciation
	du règlement de copropriété pour son opposabilité à l’acquéreur (I), la mise à jour des informations
	financières via l’état daté, permettant de faire les comptes entre le syndicat de copropriété et le vendeur
	(2).
	1. Dénonciation du règlement de copropriété et de l'état descriptif de division
	à l'acquéreur
	Si, comme il est de règle, le règlement de copropriété et l'état descriptif de division ont fait l'objet d'une
	publication au fichier immobilier, leur opposabilité à l'égard du nouveau propriétaire est automatique à
	dater de cette publication.
	L'article 4 du décret impose néanmoins pour plus de sûreté que l'acte mentionne expressément que
	l'acquéreur a eu préalablement connaissance du règlement, de ses modificatifs ainsi que de l'état
	descriptif et les actes qui l'ont modifié. Mais l'omission de cette mention n'est assortie d'aucune sanction
	puisque l'opposabilité de ces documents résulte de la publication. Simplement, si l'acquéreur subit un
	préjudice du fait de cette opposabilité, il peut mettre en jeu la responsabilité du rédacteur de l'acte (agent
	immobilier, notaire, avocat) qui a omis la mention exigée212
	Si le règlement et l'état descriptif n'ont pas été publiés, ils ne s'imposeront à l'acquéreur que s'il est
	expressément constaté dans l'acte de cession qu'il en a eu préalablement connaissance et qu'il a adhéré
	aux obligations qui en résultent (art.4 al.3 D.17 mars 1967).
	Si cette double mention n'est pas portée à l'acte, on doit alors considérer que les documents en question
	ne sont pas opposables à l'acquéreur213 .
	Dans une certaine mesure, l’existence de l’état descriptif de division est une condition nécessaire à la
	détermination de l’objet de la vente (identification de l’appartement et des quote parts de parties
	communes), donc à la validité même de la vente. Toutefois, la jurisprudence louvoie sur le sujet :
	212 GIVORD et GIVERDON \no193
	213 GIVORD et GIVERDON \no193 in fine
	droit de la copropriété année 2018-2019
	198
	La vente d’un lot de copropriété n’est pas réalisée, en l’absence de la détermination de la quote-part de
	parties communes afférente au lot qui constituait un élément essentiel de la convention : en l’absence de
	détermination suffisante de l’objet de la vente, celle-ci n’est pas parfaite214.
	La vente (de combles) est parfaite entre les parties dès lors que l’objet de la vente est déterminable, même
	en l’absence de décision de l’assemblée générale devenue définitive approuvant l’état descriptif de
	division créant le nouveau lot et lui attribuant des millièmes de parties communes215.
	2. L’état daté (art. 5 du décret \no67-223 du 17 mars 1967, en sa rédaction issue
	du décret du 27 mars 2004)
	Avant la réforme du 21 juillet 1994, l'application de cet article se confondait avec la mise en oeuvre des
	dispositions de l'article 20 de la loi (avis de mutation à titre onéreux). Désormais les choses sont
	parfaitement distinctes : l'article 5 nouveau du Décret s'applique dans tous les cas de mutation ou de
	constitution de droit réel sur le lot; qu'il y ait vente ou non.
	Avant l'établissement d'un acte réalisant ou constatant le transfert de propriété d'un lot ou d'une fraction
	de lot, ou la constitution sur ces derniers d'un droit réel, le syndic doit adresser au notaire chargé de
	recevoir l'acte, à la demande de ce dernier ou à celle du copropriétaire, un état daté qui en vue de
	l'information des parties comporte trois parties, avec des objets distincts :
	- Première partie : Déterminer les dettes du copropriétaire vendeur envers le syndicat.
	- Deuxième partie : Déterminer les dettes du syndicat envers le copropriétaire vendeur.
	- Troisième partie : Déterminer les sommes qui devront incomber à l’acquéreur.
	a. 1ère partie : les dettes du copropriétaire envers le Syndicat
	1\degre{} Dans la première partie, le syndic indique, d'une manière même approximative et sous réserve de
	l'apurement des comptes, les sommes pouvant rester dues, pour le lot considéré, au syndicat par le
	copropriétaire cédant, au titre :
	a) Des provisions exigibles du budget prévisionnel ;
	b) Des provisions exigibles des dépenses non comprises dans le budget prévisionnel ;
	c) Des charges impayées sur les exercices antérieurs ;
	d) Des sommes mentionnées à l'article 33 de la loi du 10 juillet 1965 ;
	e) Des avances exigibles.
	Ces indications sont communiquées par le syndic au notaire ou au propriétaire cédant, à charge pour eux
	de les porter à la connaissance, le cas échéant, des créanciers inscrits.
	214 Cass. Civ. 3e 11 février 2009
	215 Cass. Civ. 3e 10 septembre 2008
	droit de la copropriété année 2019-2020
	199
	b. 2ème partie : les dettes du Syndicat des Copropriétaires envers le
	copropriétaire vendeur
	2\degre{} Dans la deuxième partie, le syndic indique, d'une manière même approximative et sous réserve de
	l'apurement des comptes, les sommes dont le syndicat pourrait être débiteur, pour le lot considéré, à
	l'égard du copropriétaire cédant, au titre :
	a) Des avances mentionnées à l'article 45-1 ;
	b) Des provisions du budget prévisionnel pour les périodes postérieures à la période en cours et rendues
	exigibles en raison de la déchéance du terme prévue par l'article 19-2 de la loi du 10 juillet 1965.
	c. Troisième partie : les sommes qui devront incomber à l’acquéreur
	3\degre{} Dans la troisième partie, le syndic indique les sommes qui devraient incomber au nouveau copropriétaire,
	pour le lot considéré, au titre :
	a) De la reconstitution des avances mentionnées à l'article 45-1 et ce d'une manière même approximative
	b) Des provisions non encore exigibles du budget prévisionnel ;
	c) Des provisions non encore exigibles dans les dépenses non comprises dans le budget prévisionnel.
	Dans une annexe à la troisième partie de l'état daté, le syndic indique la somme correspondant, pour les
	deux exercices précédents, à la quote-part afférente au lot considéré dans le budget prévisionnel et dans
	le total des dépenses hors budget prévisionnel. Il mentionne, s'il y a lieu, l'objet et l'état des procédures en
	cours dans lesquelles le syndicat est partie.
	Cet état daté fait l’objet d’un formulaire qui a été rédigé en concertation avec le conseil supérieur du
	Notariat et les différents syndicats représentatifs de la profession de syndic.
	Ainsi informées, les parties pourront prendre en compte ces éléments pour évaluer le prix de vente du lot,
	et éventuellement décider -dans leurs relations - ce que l'acquéreur remboursera au vendeur ou ce que le
	vendeur s'engage à supporter (par exemple au titre de travaux non encore exécutés et dont les charges
	ne sont pas encore exigibles).
	Etant bien entendu que les conventions entre vendeur et acquéreur ne sont aucunement opposables à la
	copropriété qui a l’obligation de faire application des dispositions des articles 6-2 et 6-3 quant à la
	répartition des dettes et créances entre vendeur et acquéreur.
	Ainsi, le Décret organise en réalité la bonne application d'un principe général du droit contractuel en
	général et plus particulièrement du droit de la vente d'immeuble : le vendeur d'un immeuble - donc d'un
	droit de la copropriété année 2018-2019
	200
	lot de copropriété - doit informer son acquéreur, ne serait-ce que parce que l'acquéreur doit bénéficier
	d'une information loyale et complète pour donner un accord en connaissance de cause216
	D. LES BASES IMMOBILIERES DU NOTARIAT
	Le Décret \no 2013-803, complété par deux arrêtés du 30 septembre 2016 oblige les notaires, à compter du
	1er janvier 2017, à alimenter deux bases de données distinctes tenues par le Conseil Supérieur du Notariat
	:
	- Une base des avant-contrats
	- Une base des actes authentiques
	Les renseignements ainsi transmis (dans les 60 jours de leur signature pour les actes authentiques et dans
	les 30 jours de leur signature ou de leur remise au notaire pour les avant-contrats) permettent au C.S.N.
	d’établir des tableaux de résultats statistiques pour l’ensemble des mutations portant sur une période
	d’un ou plusieurs trimestres consécutifs, relevées dans un cadre territorial de référence (unité urbaine
	d’un arrondissement de la Ville de Paris, par exemple) portant sur au moins une vingtaine de biens. Ces
	données statistiques devant être transmises à toute personne en faisant la demande : ces renseignements
	précisant le prix et les caractéristiques essentielles de chaque bien217.

\section{Les effets de la cession}

	Il ne sera pas fait état ici des effets généraux de la vente : le vendeur est tenu des obligations de délivrance
	et de garantie, l'acheteur de celle de payer le prix.
	Mais la vente a aussi des effets concernant le syndicat : d’une part, il modifie la composition de celui-ci (I)
	et d’autre part l’acquéreur devient débiteur des charges de copropriété à compter de la notification de la
	mutation (II)
	A. EFFETS QUANT A LA COMPOSITION DU SYNDICAT.
	L'acquéreur se substitue au vendeur en tant que copropriétaire. Il devient titulaire des droits du cédant en
	tant que membre du syndicat : participation aux assemblées générales, droit d'en contester les décisions,
	etc...
	Plusieurs questions se posent à l’occasion de la cession d’un lot :
	216 cf. L'intéressante étude Madame Muriel Fabre-Magnan intitulé : De l'Obligation d'information dans les Contrats aux éditions L.G.D.J.
	1992.
	217 Cf. Rapport \no 2621 sur le proje- de loi de modernisation des professions judiciaires et juridiques réglementées : « La diffusion d’une
	information pertinente sur l’évolution du marché immobilier est de nature à la fluidifier et donc, à favoriser l’accès à la propriété de nos
	concitoyens » JCPN 2016 \no 40, act. 1072
	droit de la copropriété année 2019-2020
	201
	1. Si le cédant n'a pas informé le syndic de la cession
	Bien évidemment, s'agissant d'une mutation à titre onéreux, le mécanisme de l'article 20 permet de penser
	qu'il n'y aura pas de difficulté à ce que le syndic soit immédiatement informé de la cession. Par contre dans
	tous les autres cas, le syndic n'est informé que par la mise en oeuvre des dispositions de l'article 6 du
	Décret.
	Or, ces dispositions sont parfois ignorées par les personnes chargées d'établir les actes constatant la
	mutation : en cas de succession, trop fréquemment le notaire oublie d’informer le syndic; de même en cas
	de démembrement de la propriété du lot (nue-propriété et usufruit; vente en crédit-bail ...).
	En ce cas il a été jugé à maintes reprises que le syndic convoquait régulièrement à l'Assemblée l'ancien
	propriétaire du lot218 : c’est en effet la notification de la mutation qui, selon l’expression de MM LAFOND
	et STEMMER “ confère à l’acquéreur la qualité de copropriétaire à l’égard tant du syndicat que des autres
	copropriétaires ”219
	Toutefois, la Cour de Paris220 a estimé que la notification de l’article 6 du Décret n’ayant pas la même
	finalité que l’article 20 de la loi, le syndic doit tenir compte d’une mutation portée à sa connaissance par
	lettre simple et non par lettre recommandée alors qu’il a bien reçu cette lettre simple.
	2. Si la cession est notifiée entre la convocation de l’assemblée générale et la
	tenue de celle-ci
	Si le syndic n'a pas été avisé de la mutation, celle-ci est inopposable au syndicat. Le syndic peut alors
	valablement continuer à s'adresser à l'ancien copropriétaire pour la convocation aux assemblées générales
	ou le recouvrement des charges221.
	Si la notification de la vente lui est faite plus de quinze jours avant la tenue de l'assemblée générale le
	syndic devra convoquer l’acquéreur, même si le vendeur avait été précédemment convoqué.
	218 Civ 3\ieme{} 6 nov 1991, Administrer juillet 1992 p. 32
	219 Cour d’Appel de PARIS – 28 mars 2002 : la cour a rappelé que le critère de l’opposabilité du transfert du lot au syndic
	n’était pas la publication de l’acte de transfert mais sa notification au syndic.
	220 Paris 23\ieme{} Chambre 13 mars 1991 Dalloz 1992, Somm p 132.
	221 Civ 3\ieme{}, 21 juin 1995 et Versailles 22 mars 2004, Loyers et Copropriété oct 2004 \no 170 : « Le syndic convoque
	valablement le vendeur à l'assemblée générale dès lors qu’à la date d’envoi de la convocation il n’avait pas reçu la
	notification du transfert de propriété ».
	droit de la copropriété année 2018-2019
	202
	3. La contestation de l’assemblée générale par l’acquéreur.
	C'est le propriétaire qui doit exercer l'action en contestation.
	LE CEDANT N’A PLUS QUALITE POUR DEMANDER L’ANNULATION D’UNE ASSEMBLEE GENERALE.
	Cependant, le vendeur peut conserver intérêt à contester l'Assemblée Générale. En conséquence, s'il était
	encore copropriétaire lors de l'Assemblée, et sous réserve qu'il justifie d'un intérêt à agir, il pourra
	contester les décisions de l'Assemblée Générale alors même qu'il aura vendu son lot depuis la tenue de
	l'assemblée.
	LE CESSIONNAIRE DU LOT NE PEUT AGIR EN NULLITE D'UNE DELIBERATION D'ASSEMBLEE VOTEE
	ALORS QU’IL N’AVAIT PAS ENCORE ACQUIS LA PROPRIETE DU LOT.
	L'acquéreur ne peut contester une décision d'Assemblée tenue avant que la cession soit devenue effective,
	même s'il représentait son vendeur lors de cette assemblée. Cependant, doit être considéré comme
	valablement poursuivie par l’acquéreur l’action intentée par le vendeur en contestation d'une délibération
	de l'assemblée générale alors qu’avait été précisé à l’acte d’acquisition qu'il reprendrait la procédure222
	LE CESSIONNAIRE PEUT SOLLICITER L’ANNULATION DE L’ASSEMBLEE GENERALE SI LA
	NOTIFICATION DE LA MUTATION EST INTERVENUE AVANT SA TENUE
	Si le transfert a été notifié au syndic plus de trois semaines avant la tenue de l'assemblée générale, et que
	le syndic n’a pas re-convoqué l’acquéreur, celui-ci pourra demander l’annulation pour défaut de
	convocation223.
	Si l'acquéreur n'a pas été convoqué, sans qu'il y ait eu faute du syndic, il pourra contester l'Assemblée pour
	tout motif autre que l'absence de convocation régulière.
	Si le procès verbal de l’Assemblée n’est pas notifié à l’acquéreur, celui-ci pourra contester l’assemblée
	générale pendant un délai de 10 ans !
	B. EFFETS DE LA VENTE SUR LE PAIEMENT DES CHARGES.
	222 Cour d'appel de Paris, 19e ch. B, 12 octobre 1995 Recueil Dalloz 1996, Somm. p. 91
	223 Civ 3\ieme{} 22 juin 1994, JCP N 1994, II, p. 336.
	droit de la copropriété année 2019-2020
	203
	Un point essentiel est de déterminer quelles sont les charges et dettes qui restent dues par le cédant et
	celles qui incombent au cessionnaire.
	1. Suspension des effets de la vente à la notification du transfert de propriété.
	En premier lieu, comme rappelé précédemment, tant que la notification de la vente (article 6 du Décret)
	n’a pas été faite au syndic, le vendeur demeure débiteur des charges de copropriété224 :
	« Tant que la notification au syndic du transfert de propriété d'un lot ou d'une fraction de lot n'a pas été
	opérée en application de l'art. 6 du décret \no 67-223 du 17 mars 1967, le transfert de propriété est
	inopposable au syndic qui peut valablement recouvrer auprès du vendeur les charges dues par ce dernier
	sans tenir compte de la vente intervenue ».
	D'après les textes (art.5 du décret de 1967), le cédant est débiteur des charges et sommes devenues
	liquides et exigibles avant qu'il ne perde sa qualité de propriétaire. Le cessionnaire sera tenu de celles qui
	sont devenues liquides et exigibles après qu'il ait acquis la qualité de copropriétaire.
	A l'égard du syndicat, le moment à prendre en considération est celui de la notification du transfert au
	syndic (notification dont il a été dit précédemment qu’elle est exigée par les dispositions de l’article 6 du
	décret de 1967).
	La Cour de Cassation a affirmé que le syndicat des copropriétaires, qui oppose à l'acquéreur
	l'inopposabilité du transfert de propriété intervenu à défaut de notification de la mutation, ne peut lui
	réclamer le paiement des charges de copropriété225
	2. Exigibilité des charges au regard de la jurisprudence antérieure à la loi SRU
	Les décisions rendues avant l’intervention de la loi SRU posaient le principe selon lequel ce ne sont pas les
	décisions votées qui rendent la créance exigible : l’exigibilité résulte de l’appel de fonds; en sorte que le
	vendeur n’était pas tenu vis à vis de la copropriété des travaux votés tant que le syndic n’a pas appelé les
	provisions correspondant au coût de ces travaux 226
	Par conséquent, le débiteur des charges de copropriété est celui qui est propriétaire au moment où l'appel
	de fonds est lancé par l'assemblée
	224 PARIS, 23\ieme{} Chambre B 23 septembre 1994 Dalloz 1996 Somm p 159.
	225 civile 3\ieme{}, 8 juillet 2015 \no de pourvoi: 14-12995 - publié au bulletin - Cassation
	226 PARIS 8\ieme{} Chambre 6 fév 1997, Dalloz 1997 IR p 63.
	droit de la copropriété année 2018-2019
	204
	Il s'en suit que si les appels de fonds sont “ lancés ” avant la cession, les sommes appelées restent dues
	par le cédant, bien que les travaux ne soient exécutés qu'après. Le syndic pourra former sur ces sommes
	opposition sur le prix de vente en vertu de l'article 20 de la loi.
	3. Exigibilité des appels de charges postérieurement à la loi SRU
	Les textes combinés de la loi SRU et du Décret du 27 mai 2004 permettent aujourd’hui de répondre sans
	la moindre hésitation à l’imputabilité des appels de charges entre vendeur et acquéreur.
	Article 6-2 du Décret du 17 mars 1967 (rédaction du 27 mai 2004) :
	1\degre{} Le paiement de la provision exigible du budget prévisionnel, en application du troisième alinéa de l'article
	14-1 de la loi du 10 juillet 1965, incombe au vendeur ;
	2\degre{} Le paiement des provisions des dépenses non comprises dans le budget prévisionnel incombe à celui,
	vendeur ou acquéreur, qui est copropriétaire au moment de l'exigibilité ;
	3\degre{} Le trop ou moins perçu sur provisions, révélé par l'approbation des comptes, est porté au crédit ou au
	débit du compte de celui qui est copropriétaire lors de l'approbation des comptes.
	Certes, ce texte ne précise pas la date d’exigibilité dans chaque cas, alors que le principe demeure que le
	syndic ne peut exiger que les sommes … exigibles. Reprenons cependant successivement ces trois
	dispositions :
	LA PROVISION SUR BUDGET.(ART 14-1 DE LA LOI)
	D’après cet article, la provision sur charge est « exigible le premier jour de chaque trimestre ou le premier
	jour de la période fixée par l'assemblée générale »
	Pour savoir qui est débiteur il suffit donc de savoir qui est copropriétaire au premier jour du trimestre ou
	à la date d’exigibilité fixée par l'assemblée générale.
	LES TRAVAUX HORS BUDGET.(ART.14-2 DE LA LOI)
	D’après cet article, les dépenses sur travaux, qui ne peuvent être comprises dans le budget prévisionnel
	sont « exigibles selon les modalités votées par l'assemblée générale. »
	droit de la copropriété année 2019-2020
	205
	Si l’assemblée générale vote des travaux, elle doit impérativement fixer elle-même la date d’exigibilité
	des appels de fonds correspondant à ces travaux.
	Certes se pose la question de savoir qui sera débiteur si l’assemblée générale ne respectant pas les
	dispositions légales oublie de voter sur les modalités d’exigibilité des travaux.
	Deux hypothèses sont envisageables :
	- soit on en revient à la jurisprudence actuelle et on considère qu’est débiteur celui
	qui est copropriétaire au jour où l’appel de fonds est lancé,
	- soit on considère que faute par l’assemblée générale d’avoir voté les modalités
	d’exigibilité des fonds travaux … ceux-ci ne sont pas exigibles. Sanction
	particulièrement rigoureuse pour le syndic qui sera dans l’impossibilité de faire
	exécuter les décisions d’assemblée générale.
	L’APUREMENT DES COMPTES.
	Le décret précise que dans les relations entre le syndicat des copropriétaires et les copropriétaires il n’y a
	pas lieu à faire un compte prorata : la règle est simple, c’est celui qui est copropriétaire à la date de
	l’approbation des comptes (donc lors de l’assemblée générale approuvant les comptes) qui fait son affaire
	personnelle du trop versé en application des appels provisionnels ou de l’insuffisance de ces appels
	provisionnels.
	ABSENCE DE CONSEQUENCE DES APPELS DE FONDS POUR LE FONDS DE TRAVAUX
	La loi ALUR oblige la quasi-totalité des copropriétés comportant un ou plusieurs logements à créer un
	« fonds de travaux » qui se traduira par des appels décidés en assemblée générale227.
	mais il n’y aura pas lieu à modification du Décret du fait de la création du fonds de travaux puisque celuici
	est attaché au lot et reste acquis au syndicat des copropriétaires en cas de vote du lot.
	4. Les Conventions entre les parties concernant les charges.
	Les règles qui viennent d'être exposées ne concernent que les rapports des parties avec le syndicat.
	227 Voir infra le II de l’article 14-2 de la loi après modification par la loi ALUR.
	droit de la copropriété année 2018-2019
	206
	Mais il est possible, dans les rapports du vendeur avec l'acheteur, que la répartition soit aménagée
	différemment, c'est-à-dire que la contribution finale de chacun soit fixée librement par accord contractuel.
	Ainsi, les parties peuvent-elles convenir que les créances du syndicat exigibles lors de la mutation seront
	prises en charge par l'acquéreur. Cette clause sera obligatoire entre les parties mais inopposable au
	syndicat, à l'égard duquel le vendeur restera seul débiteur.
	Il s’agit là de l’application d’une règle de droit :
	Article 1165 du Code Civil : “ Les conventions n’ont effet qu’entre les parties contractantes; elles ne nuisent point
	au tiers, et elles ne lui profitent que dans le cas prévu par l’article 1121 (stipulation pour autrui) ”.
	Le décret du 27 mai 2004 renforce cette règle puisqu’il édicte :
	Article 6-3 :
	Toute convention contraire aux dispositions de l'article 6-2 n'a d'effet qu'entre les parties à la mutation
	à titre onéreux.
	Il est vrai que le syndic aura intérêt à se rapprocher le plus possible des conventions des parties lorsque
	celles-ci lui permettront d’obtenir paiement au moment de la mutation, surtout, lorsque la convention
	stipule que les travaux votés restent à la charge du vendeur aux termes du contrat, alors que l’appel de
	fonds correspondant n’est pas encore exigible.
	En ce cas le syndic en demandera le paiement au notaire qui doit recevoir les fonds et on voit mal pour
	quelles raisons le vendeur s’y opposerait. Pour autant, le syndic ne pourra pas faire l’opposition de l’article
	20 de la loi alors que les fonds ne sont pas exigibles.
	C. EFFETS DES DECISIONS JUDICIAIRES SUR LA VENTE DU LOT
	1. Les conséquences de l’annulation de la vente
	On relève sur ce sujet un arrêt de la cour de Colmar en date du 8 novembre 1996228 :
	228 Recueil Dalloz 1997, Sommaires commentés p. 245
	droit de la copropriété année 2019-2020
	207
	En cas de résolution de la vente d'un lot de copropriété, le vendeur est censé être resté
	propriétaire de l'appartement en cause ;
	En l'état d'un jugement prévoyant que le transfert de propriété au profit des bénéficiaires de
	la promesse de vente initiale interviendrait soit par la signature d'un acte notarié, soit par la
	publication du jugement qui en tiendrait lieu, le propriétaire reste tenu au paiement des
	charges de copropriété jusqu'à la notification informant le syndic du transfert de propriété de
	l'appartement.
	2. Les conséquences de l’annulation des travaux
	Il arrive que, soit à la suite d’une décision de justice, soit à la suite d’un vote en A.G. la décision de travaux
	soit annulée et les appels de fonds remboursés. A notre sens, le syndic rembourse le “ copropriétaire ”,
	c’est à dire le propriétaire du lot au moment de l’annulation. Mais, à notre connaissance, la question n’a
	pas été jugée.
	Symétriquement, si un nouvel appel de fonds est voté en « substitution » du premier, annulé, et n’est pas
	contesté, le nouveau copropriétaire en sera redevable.229 :
	3. Affectation des remboursements obtenus par le syndicat des
	copropriétaires après la vente du lot.
	Fréquemment le syndicat des copropriétaires, tout en engageant une procédure contre l’assureur ou
	contre le responsable décide de réaliser les travaux et fait un appel de fonds auprès des copropriétaires.
	Ultérieurement le syndicat des copropriétaires obtiendra condamnation de cet assureur ou du
	responsable au remboursement des sommes dont il a fait l’avance. Ce remboursement pourra être effectif
	alors que le copropriétaire ayant fait l’avance des travaux a cédé son lot à un acquéreur. Le syndicat des
	copropriétaires devra répartir les sommes reçues entre les copropriétaires : doit-il faire ce
	remboursement à celui qui avait fait l’avance provisionnel, à celui qui était copropriétaire lors de la
	réception des fonds ou à celui qui est copropriétaire à la date d’approbation des comptes de l’exercice au
	cours duquel le remboursement est intervenu (les fonds ont pu être reçus en janvier 2011 et les comptes
	ne seront approuvés qu’en juin 2012 (or entre temps l’acquéreur a peut être revendu).
	La réponse est apportée par un arrêt de cassation230 du 19 décembre 2012 qui casse l’arrêt de la cour
	ayant condamné le syndicat des copropriétaires à rembourser le copropriétaire ayant fait l’avance des
	fonds : « Qu’en statuant ainsi, alors que le trop perçu sur provisions qui apparaît après la mutation à titre
	onéreux de lots de copropriété est porté au crédit de celui qui est copropriétaire lors de l’approbation des
	comptes, la cour d’appel a violé le texte susvisé ».
	229 3\ieme{} Chambre civile 6 octobre 1999, Administrer jan 2000 p 30, note Capoulade.
	230 3\ieme{} Chambre civile 19 décembre 2012, \no 11-17178 (Cité Rivern c/ Itraco) au Bulletin
	droit de la copropriété année 2018-2019
	208
	On peut sans doute s’étonner de voir la cour de cassation décider que si le syndicat des
	copropriétaires obtient remboursement par un tiers des travaux votés, les fonds reçus des copropriétaires
	pour la réalisation de ces travaux deviennent des « trop perçus », mais il convient d’en déduire purement
	et simplement que les fonds reçus d’un tiers par le syndicat doivent être attribués à celui qui est
	copropriétaire lors de l’approbation des comptes comportant ces « produits ». D’où un conseil important :
	le syndic ne doit pas distribuer les fonds lorsqu’il les perçoit ; il ne pourra les imputer (donc les distribuer)
	qu’après approbation des comptes et à la date de celle-ci !