\chapter{Parties communes et parties privatives}

	L'article 664 de Code Civil ne réglait que le mode de réparation et de reconstruction des parties de
	l'immeuble en copropriété.
	
	L'article 5 de la loi de 1938 faisait la distinction entre les choses communes et les choses << affectées à
	l'usage exclusif >> d'un copropriétaire.
	
	Il permettait de distinguer les parties communes des parties privatives en fonction de l'usage qui en était
	fait :
	\begin{itemize}
		\item usage exclusif = parties privatives
		\item usage commun = parties communes
	\end{itemize}
	
	Ce texte d'autre part prévoyait une présomption de communauté pour tout ce qui n'était pas privatif.
	
	La loi du 10 juillet 1965 apporte des définitions plus précises des parties communes et privatives et donne
	une énumération des parties communes qui révèle le souci du législateur de faciliter la rédaction des
	règlements de copropriété.
	
	L'article 2 donne une définition des parties privatives plus précise que celle de 1938, mais c’est toujours
	l'usage exclusif qui détermine le caractère privatif.
	\begin{quote}
		Sont privatives les parties des bâtiments et des terrains réservées à l'usage exclusif d'un copropriétaire déterminé.
	\end{quote}

	L'article 3 donne une définition plus large des parties communes que celle donnée par la loi de 1938 :
	\begin{quote}
		sont communes les parties de bâtiments et de terrains affectées à l'usage ou à l'utilité de tous les
		copropriétaires ou de plusieurs d'entre eux. 
	\end{quote}

	Ce n'est donc plus l'usage seulement qui détermine le caractère commun, mais également l'utilité commune.
	
	Ces dispositions ne sont pas d’Ordre Public (elles ne sont pas comprises dans l'énumération de l'article 43
	de la loi). Cela signifie que l'on peut y déroger soit dans le Règlement de Copropriété soit par décision
	d'assemblée générale. Ce caractère supplétif est renforcé par les termes mêmes de l'article 3 de la loi qui,
	avant de donner une liste de parties communes, précise :
	\begin{quote}
		dans le silence ou la contradiction des titres,	sont réputées parties communes \dots
	\end{quote}
	
	Cependant la liberté des auteurs de Règlement de Copropriété se trouve limitée par la nature des choses :
	il est de l'essence même de certaines parties d'immeubles d'être communes à plusieurs lots ( la toiture,
	le gros œuvre, les fondations de l'immeuble). De même l'affectation de certaines parties de l'immeuble
	implique leur classement en parties communes, comme par exemple le sol de la cour donnant accès à
	l'immeuble depuis le porche sur rue.
	
	La loi \no 2018-1021 du 23 novembre 2018 dite ELAN a, en consacrant une jurisprudence antérieure de la
	cour de Cassation :
	\begin{enumerate}
		\item complété la liste des parties figurant dans l’article 3;
		\item consacré les « parties communes spéciales », c’est-à-dire des parties
		communes entre certains copropriétaires seulement (article 6-2) ;
		\item  consacré la possibilité de procéder à un démembrement de la
		propriété des parties communes, en créant des « parties communes
		à jouissance exclusive >> (article 6-3).
	\end{enumerate}

\section{Les parties privatives}

	\subsection{Definition des parties privatives : parties a l’usage exclusif d’un	coproprietaire}

		L'article 2 dispose :
		\begin{quote}
			Sont privatives les parties des bâtiments et des terrains réservées à l'usage exclusif d'un copropriétaire déterminé
		\end{quote}
		C'est donc l'usage exclusif qui détermine le caractère privatif. L'usage exclusif, c'est celui qui est
		incompatible avec l'usage d'un autre copropriétaire : le copropriétaire a sur ses parties privatives un
		véritable monopole.
		
		Selon l'article 2 peuvent être privatives non seulement des parties de bâtiments mais également des
		parties de terrains : ainsi, le Règlement de Copropriété peut classer dans les parties privatives des jardins,
		terrasses ou cours d'immeuble.
		
		Le plus souvent ces terrasses ou jardins ne sont pas classés en parties privatives, mais en parties communes
		à jouissance exclusive. Cette jouissance exclusive, même si elle interdit la jouissance concurrente des
		autres copropriétaires n'a pas pour effet de déclasser ces parties communes en parties privatives.
	
	\subsection{Qualification usuelle des parties privatives}
		La loi ne donne aucune énumération des parties privatives, même à titre subsidiaire. C'est donc le
		Règlement de copropriété qui détermine les parties privatives.
		
		En cas d'ambiguïté du Règlement de Copropriété, lorsque ses clauses sont obscures ou contradictoires ou
		encore lorsqu'il n'existe pas de Règlement de Copropriété les Tribunaux qualifient habituellement de
		parties privatives :
		\begin{itemize}
			\item les parquets, revêtements de sol, y compris ceux des balcons et terrasses, mais pas le gros œuvre de ces balcons et terrasses\footnote{TGI Paris 17 mars 1975 et TGI Paris 10 mars 1973} ;
			\item les enduits intérieurs des murs et plafonds, les corniches, moulures, lambris, revêtements
			intérieurs ;
			\item les cloisons intérieures non porteuses ;
			\item les fenêtres et porte-fenêtre avec leur bâti, vitres, volets, persiennes, barres d'appui de
			fenêtres et balcons, les bannes ou stores ;
			\item les menuiseries intérieures (placards, portes y compris la porte palière) ;
			\item les devantures et vitrines des magasins et boutiques ;
			\item les installations de cuisine et sanitaires (éviers, baignoires, lavabos), les canalisations et
			branchements des appareils jusqu'aux canalisations communes ;
			\item les installations de chauffage et de production d'eau chaude individuelle ;
			\item les installations d'eau, d'électricité et de gaz jusqu'au branchement.
		\end{itemize}

\section{Les parties communes}

	\subsection{Definition des parties communes}
	
		\subsubsection{Définition des parties communes : parties affectées à l’usage ou à l’utilité de tous}
		
			L’article 3 de la loi du 10 juillet 1965 dispose « sont parties communes les parties des bâtiments et des
			terrains affectées à l'usage ou à l'utilité de tous les copropriétaires ou de plusieurs d'entre eux ».En
			pratique, presque tout le bâti relève des « parties communes » : gros murs, fondations, planchers, toiture
			et sol…
			Ces parties communes se caractérisent non seulement par une « utilité » commune, mais par la propriété
			indivise des copropriétaires sur celles-ci : le droit du copropriétaire sur ces parties communes se traduit
			par la « quote- indivise » ou les « tantièmes » qui sont attachés à son lot (\emph{lot \no 1 : un appartement et les	125/ \nombre{1 000}\ieme des parties communes})
			
		\subsubsection{L’énumération de l’article 3 et son caractère supplétif}
		
			L'article 3 de la loi, deuxième alinéa comporte une liste d’éléments de bâtis « réputés parties communes »
			« Dans le silence ou la contradiction des titres sont réputées parties communes... ». Cette énumération
			n'est donc que supplétive de la volonté des parties, c'est à dire de l'auteur du règlement de Copropriété.
	
			Le Règlement de Copropriété peut donc prévoir de classer en parties privatives des parties d'immeuble
			qui dans le silence ou la contradiction du Règlement de Copropriété auraient été placées en parties
			communes.
			Il en est ainsi par exemple de l'étanchéité et du gros oeuvre d'une terrasse que le
			Règlement de Copropriété classe expressément en partie privative alors que dans le
			silence contractuel la terrasse est classée en partie commune dans son gros oeuvre et son
			étanchéité\footnote{Civ 3\degre{} Ch 13 nov 1975, pourvoi: 74-11703 ; Bull Civ \no 331 p 251}.
			Il en sera de même d'un gros mur que le Règlement de Copropriété aura classé en partie
			privative, en sorte que si ce gros mur est mitoyen, en cas de travaux à réaliser sur ce mur,
			le voisin dirigera son action contre le copropriétaire titulaire du lot comprenant ce mur et
			non contre le syndicat des copropriétaires\footnote{LYON 30 Auch 1973, JCP 74 II 17715}.
		
		\subsubsection{L’absence de partie commune par « subsidiarité »}
		
			Les Tribunaux ont considéré pendant de nombreuses années que toute partie de bâtiment qui n'est pas
			affectée à l'usage exclusif d'un copropriétaire est une partie commune. C’était notamment la position de
			la Cour de Cassation dans un arrêt du 16 janvier 1979\footnote{Administrer 1979 \no 94}.
			Cependant la Cour de Cassation est revenue sur cette jurisprudence à propos d'une terrasse classée partie
			commune par le juge du fond, car non affectée à un copropriétaire par le Règlement de Copropriété :
			l’arrêt a été cassé au motif que le juge du fond aurait dû rechercher à l'usage et à l'utilité de qui cette
			partie de l'immeuble était réservée\footnote{Civ 3\degre{} Ch – 14 février 1990 - pourvoi: 88-17781 Inf. Rapides de la Copropriété. 1990 p 281}.
			Un arrêt du 6 mai 1993\footnote{PARIS 14 décembre 1990 - D. 91 IR 15} de la Cour de PARIS se référant à l'arrêt de Cassation du 14 février 1990 éclaire
			bien cette jurisprudence : "Un comble qui ne peut servir qu'à assurer la réparation de la toiture de
			l'immeuble, incontestablement partie commune, constitue lui-même une partie commune, peu important
			qu'il existe une trappe d'accès dans le lot privatif du copropriétaire défendeur, dès lors qu'un accès est
			effectivement nécessaire, étant observé de surcroît que cette trappe permet d'accéder à l'ensemble du
			comble et non seulement à la partie située au-dessus de son appartement".
			La jurisprudence la plus récente sur la question\footnote{Cass. 3e civ., 30 mai 1995, pourvoi: 93-16347 somJCP N 1995 / \no 26 / p. 981. et Dalloz 1994, Somm p.203 Les conditions à remplir pour que le s combles de l'immeuble constituent des parties privatives. La pose	de chiens-assis dans les combles peut-elle être assimilée à une surélévation ? par Jean-Robert Bouyeure} admet qu’il n’y a pas de parties communes par	subsidiarité et qu’il convient à chaque fois de rechercher si la partie d’immeuble concernée est à usage exclusif\footnote{Cf. l’intéressant article de Me Hocquard in IRC 2006 \no 515 p 29} d’un copropriétaire (auquel cas elle est bien une partie privative) ou si elle est à l’usage de tous ou de plusieurs (auquel cas elle est une partie commune).
	
	\subsection{Enumeration des parties communes (art 3 alinea 2 de la loi)}
	
		\subsubsection{Le sol}
		
		Le sol de la Copropriété est présumé commun ; mais comme dit précédemment il pourrait être classé
		partie privative. Ce dernier cas se présente parfois dans le domaine des copropriétés horizontales.
		Cependant, l’Administration refuse de délivrer le permis de construire dans le cadre de copropriétés
		horizontales dès lors que le sol est partie privative ou à jouissance privative. Elle considère en effet
		que dans ces deux hypothèses il y a réalisation d’un lotissement.
		Il arrive parfois que le sol soit qualifié de << partie commune à jouissance privative >>, ce qui ne lui fait pas
		perdre son caractère de partie commune\footnote{Cf infra le << problème des parties communes à jouissance privative >>.}.
		
		\subsubsection{Les cours}
		
		Les cours posent souvent problème du fait du stationnement des véhicules. Un copropriétaire ne peut
		prétendre encombrer les parties communes et à ce titre ne peut laisser stationner un véhicule dans la cour
		de l'immeuble. Mais les copropriétaires peuvent décider en assemblée générale d'autoriser le
		stationnement des véhicules.
		
		La Cour de PARIS a estimé illicite la décision d'assemblée générale concédant un droit d'usage
		exclusif sur le sol de la cour pour y faire stationner des véhicules, dès lors que ce droit d'usage exclusif
		était consenti à certains copropriétaires seulement\footnote{PARIS 23\degre{} 7 mai 1993 - Loyers et Copropriété Août Sep 93 \no 316}.
		
		\subsubsection{Les parcs et jardins, les voies d'accès}
		
		\subsubsection{Le gros œuvre des bâtiments}
		
		\subsubsection{Les murs}
		
		Sont présumés communs les murs qui constituent le gros œuvre de la construction (gros murs ou murs de
		refend, les poteaux, poutres et poutrelles verticales ou horizontales, métalliques ou en maçonnerie).
		
		Le plus souvent les revêtements muraux comme les revêtements de sol sont placés dans les parties
		privatives. Toutefois, dans le silence du Règlement de copropriété, s’agissant d’un flocage « amiante »,
		destiné à assurer la protection de l’immeuble contre l’incendie, même si le flocage est situé au plancher
		haut de locaux privatifs, il doit être considéré comme partie commune\footnote{Cass. 3e civ., 7 mai 2003 : JurisData \no 2003-018913 ; JCP G 2003, p. 1869 ; Loyers et copr. 2003, comm.}. En sorte que le désamiantage
		devra être pris en charge par le syndicat des copropriétaires.
		
		\subsubsection{Les planchers}
		
		Il s’agit du gros œuvre du plancher : poutres, solives, hourdis. Il convient d'y assimiler les voûtes des caves.
		
		\subsubsection{Le toit, les terrasses}
		
		Le toit étant partie commune, son entretien est à la charge du Syndicat. Mais on peut imaginer que le
		Règlement de Copropriété classe en partie privative le toit d'un bâtiment constitué d'un seul lot de
		copropriété, ou la verrière surplombant un appartement\footnote{CA Lyon 30 novembre 2011 à propos d’un « ciel vitré » : Les ciels vitrés, ayant pour utilité de faire pénétrer la lumière dans les appartements du dernier étage, constituent des parties privatives au même titre que les fenêtres lesquelles sont qualifiées comme telles par le règlement de copropriété, même s'ils sont incorporés dans le pan de la toiture, dès lors que leur utilité principale est d'éclairer des appartements privés et non de couvrir l'immeuble.}. Peut également se poser la question des
		« Vélux » fenêtres de toit : alors même que le Règlement de copropriété les qualifie de parties privatives
		la Cour de Paris a considéré que le syndicat des copropriétaires devait répondre des infiltrations d’eau
		ayant leur siège à la jonction des velux parties privatives et de la toiture commune\footnote{Paris, 23\degre{} Chambre A, 25 septembre 2002, 92 rue Rochechouart, Jurisdata 2002-188829 ; cf. également, dans le silence du Règlement de copropriété, la qualification des velux en parties communes par la Cour de Pau, en ce qu’ils assurent non seulement l’éclairage d’un lot mais également le couvert de l’immeuble (Pau, 1\degre{} Chambre, 15 février 2013, JurisData 2013-01064).}.
		
		En ce qui concerne la terrasse d'un bâtiment quand bien même cette terrasse est classée en partie
		privative, seul le revêtement est partie privative. Par contre le dispositif d'étanchéité et la couche
		d'étanchéité sont des parties communes, de même que le gros oeuvre de la terrasse et les maçonneries
		sous-jacentes\footnote{Civ. 16 nov. 1976 D 77 IR 62}.
		Les structures métalliques de la terrasse lorsqu’elles sont ornements extérieurs des façades doivent
		également être placées dans les parties communes\footnote{Paris 23\degre{} Ch A 11 déc 1996; Loyers et Copropriété 1997 \no 119}.
		
		\subsubsection{Les balcons et bow-windows}
		
		Une jurisprudence abondante a classé les balcons en parties privatives au motif que ces balcons sont à
		l'usage exclusif du lot dont ils dépendent, mais la jurisprudence la plus récente a tendance à classer ces
		balcons en parties communes. Par exemple un balcon qui est le prolongement extérieur de la dalle de
		l'étage sera très certainement classé en partie commune quant à son gros oeuvre. De même un balcon,
		pièce rapportée en façade dont il constitue l'ornementation contribue à l'harmonie générale et à ce titre
		sera également classé en partie commune.
		
		Mais bien évidemment ce classement par le juge du fond ne se fait que dans le silence ou la contradiction
		des titres. Le plus souvent le Règlement de Copropriété opère une distinction entre le gros œuvre du
		balcon (effectivement classé en partie commune) et les balustrades et garde-corps (classés en parties
		privatives).
		
		Les bow-windows semblent devoir être placés en parties privatives pour leurs parties vitrées et en parties
		communes pour la dormant métallique qui concourt à l'harmonie, et qui, non-manœuvrable constitue
		partie de la façade de l'immeuble\footnote{Par assimilation avec les châssis vitrés aménagés dans les boxes d’un immeuble, cf. 3e civ., 4 juill. 1990 : Loyers et copr. 1990, comm. 399 ; D. 1991, p. 76}.
		
		\subsubsection{Les escaliers}
		
		Les escaliers sont présumés parties communes comme faisant partie du gros œuvre des bâtiments. Il en
		est de même des marches et contremarches.
		
		A défaut de disposition expresse du Règlement de Copropriété, l'escalier est une partie commune à tous
		les copropriétaires, en sorte que tous doivent y participer $\dots$ et tous pourront l'emprunter.
		
		On ne saurait en effet, ni dans le Règlement de Copropriété, ni par décision d' assemblée générale, limiter
		de façon discriminatoire les droits d'usage d'un copropriétaire sur les parties communes au motif qu'il n'a
		pas l'utilité réelle de ces parties communes (par ex. par la pose d’un code à la porte d'accès à l'escalier de
		l'immeuble en refusant la clé aux copropriétaires du rez de chaussée).
		
		\subsubsection{Canalisations traversant les locaux privés}
		
		\paragraph{La présomption de parties communes ne s'applique qu'aux canalisations qui
		traversent (sans s’y arrêter) les locaux privatifs}.\footnote{Cf Les canalisations en copropriété, Me Guilhem GIL, I.R.C. \no 595 – jan/fev 2014 p. 17}
	
		En effet << ces canalisations, quand bien même elles sont situées pour parties à l'intérieur de lots privatifs,
		sont, par leur fonction propre, rattachées à l'élément commun qu'elles prolongent, et non au local privatif
		qu'elles traversent, sans servir à l'usage du propriétaire du lot>>\footnote{CA PARIS 7\degre{} Ch du 3 mars 1993}.
		D'où une controverse pour les canalisations qui desservent les parties privatives sans en ressortir. Il
		incombe aux rédacteurs de Règlement de Copropriété d'être particulièrement clairs sur ce point et de
		qualifier de privatives ou communes ces canalisations.
		
		Mais encore une fois, le juge ne pourra aller contre une classification précise du Règlement de Copropriété.
		Par exemple à propos de canalisations classées parties privatives par le Règlement de Copropriété dans
		leur partie qui traverse les lots privatifs et ce alors même que ces canalisations n'ont aucune utilité pour
		ces lots.
		
		\paragraph{Canalisations encastrées dans le plancher commun}
		
		Assez souvent se pose la question des canalisations ne desservant qu’un lot privatif encastrées
		dans le plancher partie commune.
		Si le Règlement de Copropriété précise que les canalisations ne desservant qu’un seul lot sont
		privatives, il ne fait aucun doute qu’encastrées ou non dans le gros oeuvre, ces canalisations
		demeurent privatives.
		
		Si le Règlement de copropriété exclue des parties privatives les canalisations qui ne se trouvent pas
		à l'intérieur des lots qu'elles desservent, la Cour d’Appel en déduit exactement que ces canalisations
		encastrées dans le plancher partie commune se trouvant à l’extérieur des parties privatives, constitue
		des parties communes\footnote{Civ. 3\degre{} Ch. 26 janvier 2017, pourvoi \no 15-2825, non publié}.
		
		Si le Règlement de Copropriété reste muet sur la qualification de ces canalisations, il convient à
		notre sens d’appliquer les dispositions de l’article 2 de la loi et de les qualifier de privatives ;
		cependant on citera un arrêt du 1er juillet 2003 qui a reproché à la cour d’appel dans une telle
		hypothèse de n’avoir pas recherché si la canalisation encastrée dans le plancher partie commune
		n'était pas elle-même une partie commune\footnote{Civ 3\degre{} Ch 1er juillet 2000 », pourvoi \no 01-03430, inédit.}.
		
		\subsubsection{Coffres, gaines et têtes de cheminées}
		
		Ces coffres, gaines et têtes de cheminées sont la plus souvent incorporés dans les gros murs.
		
		\subsubsection{Locaux et services communs}
		
		Il s’agit de la loge du concierge, le local social, le local a bicyclettes, \emph{etc.}
		
		Il existe une jurisprudence abondante sur le caractère privatif ou commun de la loge de concierge dont le
		promoteur fait très souvent un lot privatif : par le passé certains arrêts ont décidé en ce cas que même
		affectée de tantièmes de copropriété, la loge pouvait être classée partie commune, puisqu’elle était
		d’intérêt collectif.
		
		Aujourd’hui la solution inverse est affirmée avec vigueur\footnote{CA Paris 23\degre{} Ch. A, 28 mai 1997 Recueil Dalloz 1997, Sommaires commentés p. 325} :
		\begin{quote}
			<< La loge du concierge, qui comprend d'une manière indissociable des parties privatives et une
			quote-part correspondante des parties communes, ne peut être classée comme partie
			commune sans perdre sa nature et son indépendance ; En raison de cette incompatibilité, entre
			la notion de lot et la qualification de partie commune, il y a lieu de juger qu'il existe une erreur
			de rédaction évidente dans le règlement de copropriété classant la loge de gardien parmi les
			parties communes, cette erreur pouvant faire l'objet d'une correction par décision de
			l'assemblée générale des copropriétaires de l'immeuble statuant aux conditions de majorité de
			l'art. 26 de la loi \no 65-557 du 10 juill. 1965 >>
		\end{quote}
	
		Il est donc incontestable que si la loge de concierge est constituée en partie privative dotée de tantièmes
		généraux, le juge ne peut requalifier cette loge de partie commune\footnote{Civ 3\degre{} 19 nov 1986; Civ 3\degre{} 4 jan 1989; Civ 27 juin 1990 , Rev Droit Imm. 90, 526}, même si le règlement de copropriété
		fait état d’une loge dans les parties communes\footnote{Paris 23\degre{} Chambre 28 mai 1997 Loyers et Copropriété 1997 \no 297, qui précise que l’assemblée générale peut corriger le règlement de copropriété sur ce point à la majorité de l’article 26 de la loi.}.
		
		Pour autant, si le Règlement de Copropriété prévoit qu'il existera dans l'immeuble une loge de concierge,
		le local même classé en partie privative peut-il être considéré comme une partie privative à affectation
		déterminée et « perpétuelle »?
		
		Une réponse négative a été apportée par la Cour de Cassation dans un arrêt du 4 novembre 2004\footnote{Civ - Troisième chambre Arrêt \no 1101 du 4 novembre 2004}, motif
		pris que le règlement de copropriété ne peut porter atteinte au droit de disposition de son lot par un
		copropriétaire.
		
		En l’espèce le promoteur, tombé depuis lors en liquidation de biens, avait conservé des lots utilisés
		par le syndicat des copropriétaires comme conciergeries dans la plus grande copropriété de France
		(La Rouvière à Marseille). Après la vente de ces lots à la barre du tribunal, l’adjudicataire avait initié
		une procédure d’expulsion de ces locaux à l’encontre du syndicat des copropriétaires. Pour refuser
		l’expulsion, la cour d’appel avait retenu : « que l’affectation des lots est justifiée par la destination
		de l’immeuble, que l’article 55 du règlement de copropriété selon lequel << le concierge habitera
		obligatoirement au rez-de-chaussée dans des locaux spécialement affectés à cet effet >> est licite et
		opposable à l’acquéreur, que les caractéristiques de l’immeuble imposent la présence à demeure de
		concierges, que les locaux du rez-de-chaussée ont été immédiatement affectés en conciergeries par
		le promoteur-constructeur et équipés d’un matériel spécifique, qu’il les a loués verbalement au
		syndicat des copropriétaires et que l’attention de l’adjudicataire a été attirée par le cahier des
		charges sur leur usage impératif ».
		
		Cassation au visa des articles 544 du Code civil, et de l’article 9 de la loi du 10 juillet 1965 au motif
		lapidaire : Qu’en statuant ainsi, alors que cette stipulation du règlement de copropriété ne pouvait
		avoir pour effet d'instituer de restriction aux droits de copropriétaires sur leur lot, la cour d'appel a
		violé les textes susvisés.
		
		En revanche le Promoteur-Vendeur pourra être condamné pour publicité mensongère, et le cas échéant à
		livrer une loge à la Copropriété s’il a pris cet engagement envers les acquéreurs.
		L'analyse conjuguée de l'acte de vente (descriptif annexé à une vente en l'état futur d'achèvement, par
		exemple) et du Règlement de Copropriété permettra de déterminer les obligations du vendeur vis-à-vis
		de ses acquéreurs. Bien souvent le Descriptif lui-même est muet sur certains locaux collectifs ou éléments
		d'équipement alors que le Règlement de Copropriété traitera de ces parties de l'immeuble.
		De plus, et alors même que le Promoteur Vendeur a souscrit des engagements envers les acquéreurs à
		titre individuel et non envers la personne morale du syndicat des copropriétaires, les Tribunaux ont déclaré
		le syndicat des copropriétaires recevable à agir en justice pour avoir condamnation du Promoteur Vendeur
		à lui livrer ces parties d'immeubles\footnote{Civ. 3\degre{} 19 fèv 1980, Bull Civ III \no 41 p 29}.
		
		\subsubsection{Les passages et corridors}
		
		Présumés parties communes, ils sont souvent placés en parties communes spéciales aux bâtiments dont
		ils dépendent.
		
		\subsubsection{Les parties communes par incorporation}
		
		La loi ELAN a ajouté un point à cette énumération : est également réputé parties communes : « tout
		élément incorporé dans les parties communes. » Cette présomption peut être particulièrement utile pour
		les réseaux de chauffage au sol noyés dans la dalle partie commune, mais également aux éventuels bowwindows,
		rampe d’accès PMR, voire ascenseur réalisé par un copropriétaire, si le règlement de copropriété
		ne précise rien à ce sujet.
		
		\subsection{Les parties communes « speciales »}
		
		\subsubsection{Définition}
		
		L'article 3 ébauche une distinction entre les parties communes à tous les copropriétaires et les parties
		communes à "plusieurs d'entre eux". Effectivement le Règlement de Copropriété peut distinguer les
		parties qui sont communes à tous les copropriétaires (parties communes « générales ») et les parties qui
		sont spéciales à certains d'entre eux (parties communes « spéciales »),
		Cette distinction a été consacrée dans l’article 6-2 de la loi \no 65-557 du 10 juillet 1965 dans sa rédaction
		issue de la loi loi \no 2018-1021 du 23 novembre 2018 dite ELAN (rectifiée par l’ordonnance)
		Article 6-2 Créé par LOI \no2018-1021 du 23 novembre 2018 - art. 209 (V) -Ordonnance \no2019-1101 du 30
		octobre 2019 - art. 5.
		
		Les parties communes spéciales sont celles affectées à l'usage et ou à l'utilité de plusieurs
		copropriétaires. Elles sont la propriété indivise de ces derniers.
		La création de parties communes spéciales est indissociable de l'établissement de charges spéciales à
		chacune d'entre elles.
		
		Les décisions afférentes aux seules parties communes spéciales peuvent être prises soit au cours d'une
		assemblée spéciale, soit au cours de l'assemblée générale de tous les copropriétaires. Seuls prennent
		part au vote les copropriétaires à l'usage et ou à l'utilité desquels sont affectées ces parties communes.
		
		En outre, l’article 6-4 dispose désormais
		
		Article 6-4 Créé par LOI \no2018-1021 du 23 novembre 2018 - art. 209 (V)
		L'existence des parties communes spéciales et de celles à jouissance privative est subordonnée à leur
		mention expresse dans le règlement de copropriété.
		
		Lorsque des parties communes sont instituées (par exemple par Bâtiment), le copropriétaire a à la fois des
		droits indivis dans les « parties communes générales » (en général le sol, les espaces verts et les voiries)
		et des droits indivis spécifiques dans son Bâtiment. En revanche, il ne détient aucun droit dans le Bâtiment
		voisin.
		Ex : lot \no1 : un appartement au Rez-de-chaussée du Bâtiment B et les 125/10.000 du sol et les 200/1000
		des parties communes du Bâtiment A
		
		\subsubsection{Conséquences de la création de parties communes spéciales}
		
		Il résulte de la création de parties communes spéciales à certains copropriétaires :
		\begin{itemize}
			\item Une spécialisation des charges : il avait déjà été jugé par la Cour de cassation que « la
			création dans le règlement de copropriété de parties communes spéciales avait pour
			corollaire l’instauration de charges spéciales et déduit que les charges afférentes à la
			réfection de l’étanchéité de la terrasse devaient être réparties entre les propriétaires de ce
			bloc » ; le principe est consacré par l’article 6-2 (auparavant : 3è Chambre civile, 6 juin 2007
			(Bull. \no 98)
			\item  Une spécialisation des droits de vote en assemblée générale (ou une assemblée spéciale):
			seuls les copropriétaires ayant des tantièmes dans les parties communes spéciales peuvent
			prendre des décisions à leur sujet, puisque les tantièmes de propriété déterminent les
			droits de vote. L’article 6-3 consacre même la faculté d’une assemblée spéciale, incertaine
			jusqu’alors (en ce sens avant la loi \no 2018-1021 du 23 novembre 2018 dite ELAN CA Paris
			23è chambre B 29 juin 2000, loyers et copropriété février 2001 comm. \no53 ; CA Paris 23è
			chambre section A 15 mai 2002, AJDI septembre 2002, page 618) ; à la condition toutefois
			que la décision ne concerne que les parties communes spéciales ( le ravalement, par
			exemple, peut aussi concerner l’aspect extérieur de l’ensemble immobilier ; la vente d’une
			partie commune spéciale implique ensuite un modificatif au règlement de copropriété qui
			concerne tous les copropriétaires - CA Paris 23è chambre, 4 septembre 2008, Administrer
			mai 2009 page 49)
			\item  Une limitation des droits des copropriétaires : lorsqu'une partie commune est qualifiée de
			particulière à certains copropriétaires, les autres copropriétaires de l'immeuble n'ont
			aucun droit de propriété indivise sur cette partie. Aussi le prix de vente des parties
			communes spéciales doit-il être réparti entre les copropriétaires, ou l’accès au Bâtiment
			peut être interdit aux copropriétaires n’ayant pas de droit dans celui-ci.
		\end{itemize}
		
		\subsubsection{Adaptation des règlements de copropriété}
		
		La loi \no 2018-1021 du 23 novembre 2018 dite ELAN consacre l’existence des parties communes spéciales,
		mais impose leur mention dans le règlement de copropriété (elles apparaissaient parfois seulement dans
		l’état descriptif de division) et impose également la création explicite d’une clef de charge constitutive.
		
		Il peut donc être nécessaire d’adapter le règlement de copropriété existant.
		
		Cette adaptation a été prévue par la loi ELAN dans son article 209 – \II (disposition transitoire)
		
		Les syndicats des copropriétaires disposent d’un délai de trois ans à compter de la promulgation de la
		présente loi pour mettre, le cas échéant, leur règlement de copropriété en conformité avec les dispositions
		de l’article 6-4 de la loi \no 65-557 du 10 juillet 1965 fixant le statut de la copropriété des immeubles bâtis.
		
		A cette fin, le syndic inscrit à l’ordre du jour de chaque assemblée générale des copropriétaires la question
		de la mise en conformité du règlement de copropriété. La décision de mise en conformité du règlement
		de copropriété est prise à la majorité des voix exprimées des copropriétaires présents ou représentés.

\section{Droits accessoires aux parties communes}

	Un certain nombre de droits sont définis comme des droits « accessoires » aux parties communes, c’està-
	dire attaché au sol ou au gros oeuvre du Bâtiment. Par application du principe selon lequel « l’accessoire
	suit le principal », ces droits sont eux même réputés parties communes. L’article 3 établit une liste de ces
	droits « réputés parties communes », mais elle n’est pas limitative (A) Cependant, tout comme les parties
	communes « matérielles » de l’immeuble ces droits peuvent être « privatisés », c’est-à-dire réservés à
	l’usage exclusif d’un copropriétaire, soit par intégration dans la partie privative du lot (B), soit par
	convention (C)
	
	\subsection{Definition des droits accessoires aux parties communes}
	
		\subsubsection{Enumération des droits accessoires (art 3 alinéa 3)}
		
			L'article 3 al 3 de la loi donne la liste des « droits accessoires » aux parties communes, donc des droits
			immatériels réputés propriété indivise à tous. Seul le Syndicat, peut donc en disposer, moyennant une
			contrepartie financière, à la majorité de l’article 26 de la loi sous réserve que cette cession des droits ne
			porte pas atteinte aux modalités de jouissance des autres lots\footnote{Versailles 29 nov 2004, Administrer \no 375 mars 2005.} ou à la destination de l’immeuble.
			
			Cette énumération a été complétée par la loi \no 2018-1021 du 23 novembre 2018 dite ELAN
			Article 3- modifié loi \no 2018-1021 du 23 novembre 2018 dite ELAN
			" Sont réputés droits accessoires aux parties communes dans le silence ou la contradiction des titres
			- le droit de surélever un bâtiment affecté à l'usage commun ou comportant plusieurs locaux qui
			constituent des parties privatives différentes, ou d'en affouiller le sol ;
			- le droit d'édifier des bâtiments nouveaux dans des cours, parcs ou jardins constituant des parties
			communes ;
			- le droit d'affouiller de tels cours, parcs ou jardins ;
			- le droit de mitoyenneté afférent aux parties communes.
			« – le droit d’affichage sur les parties communes ; (art. 59 bis F loi ELAN)
			« – le droit de construire afférent aux parties communes. » ; (art. 59 bis F loi ELAN)
		
		\subsubsection{Sur le caractère non limitatif des droits accessoires énumérés par l’article 3}
		
			La jurisprudence a ajouté à cette liste d’autres droits accessoires aux parties communes, que le syndicat
			peut céder à un copropriétaire à la majorité de l’article 26 de la loi, affirmant que cette liste n’est pas
			limitative\footnote{3\degre{} Chambre civile 24 mai 2006. \no 05-14.038. au Bulletin - C.A. Paris, 27 janvier 2005} :
			\begin{quote}
				« La liste des droits accessoires aux parties communes définis par l'article 3 de la loi du 10 juillet 1965 n'est
			pas limitative.
			\end{quote}
		
			Dès lors, une cour d'appel qui, interprétant souverainement les stipulations d'un règlement de copropriété,
			retient que la faculté que s'y réserve un propriétaire de clore une terrasse dont il a la jouissance privative
			constitue un droit accessoire, en déduit exactement, par application de l'article 37 de cette loi, que cette
			faculté est devenue caduque si ce droit n'a pas été exercé dans les dix années qui suivent la convention ».
			
			C’est ainsi que la jurisprudence antérieure à la loi ELAN avait qualifié le droit « administratif » de construire
			de droit accessoire aux parties communes- à l’époque où il existait encore une limitation de la superficie
			constructible à la parcelle par le Coefficient d’Occupation des sols\footnote{Civ 3\degre{} 10 janvier 2001 (D. 2002 p. 1521, note Giverdon) : le droit de construire sur le sol commun résultant des règles	d’urbanisme constitue un droit accessoire aux parties communes. » et , 3\degre{} Ch. Civ. 1er oct 2013, \no pourvoi : 12-21785, non publié. }. En d’autres termes, lorsqu’un
			copropriétaire entend déposer une demande de permis de construire, il utilise les droits de construire
			propriété du syndicat, à moins que ce droit lui ait été donné par le Règlement de Copropriété. Il ne pourra
			le faire que dans la mesure où le syndicat les lui aura expressément cédés. Il en était ainsi dans l’hypothèse
			d’un changement d’utilisation d’un lot dès lors que ce changement avait une influence sur le COS de
			l’immeuble, ceci quand bien même le Règlement de Copropriété autoriserait-il toute affectation du lot\footnote{114}.
			
			Cette jurisprudence a été consacrée par la loi \no 2018-1021 du 23 novembre 2018 dite ELAN qui inclus
			désormais dans les droits accessoires aux parties communes « le droit de construire afférent aux parties
			communes ». Ce droit ne se confond pas avec le droit de surélever un Bâtiment ou d’en construire un
			nouveau.
			
			Cependant, la disparition du COS limite la portée de cette disposition, puisque, désormais, le droit de
			construire sur la parcelle est limitée essentiellement par le gabarit ou le prospect. Ainsi, désormais, la
			réalisation d’une mezzanine est sans conséquence sur la constructibilité du Bâtiment, et ne nécessite plus
			de cession préalable du droit de construire 115. En revanche, la surélévation d’un Bâtiment exclusivement
			privatif, qui n’est pas en principe un droit « accessoire » appartenant au syndicat des copropriétaires116 (cf
			la rédaction de l’art 3 al 3) peut porter atteinte à la constructibilité via les règles de gabarit $\dots$
	
	\subsection[Conventions de l'article 37]{Les conventions relatives aux droits accessoires sur parties	communes (art. 37 Et 37-1 de la loi \no 65-557 du 10 juillet 1965 )}
	
	L’article 37 de la loi a été ainsi rédigé dans la loi \no 65-557 du 10 juillet 1965
	Toute convention par laquelle un propriétaire ou un tiers se réserve l'exercice de l'un des
	droits accessoires visés à l'article 3, autre que le droit de mitoyenneté, devient caduque si
	ce droit n'a pas été exercé dans les dix années qui suivent ladite convention.
	
	Si la convention est antérieure à la promulgation de la présente loi, le délai de dix ans court
	de ladite promulgation.
	
	Avant l'expiration de ce délai, le syndicat peut, statuant à la majorité prévue à l'article 25,
	s'opposer à l'exercice de ce droit, sauf à en indemniser le titulaire dans le cas où ce dernier
	justifie que la réserve du droit comportait une contrepartie à sa charge.
	
	Toute convention postérieure à la promulgation de la présente loi, et comportant réserve
	de l'un des droits visés ci-dessus, doit indiquer, à peine de nullité, l'importance et la
	consistance des locaux à construire, et la modification que leur exécution entraînerait dans
	les droits et charges des copropriétaires.
	
	On voit ici la méfiance du législateur à l'égard des propriétaires d'immeubles qui veulent vendre
	l'immeuble par lots en conservant certains droits dans cet immeuble. Rédacteurs du Règlement de
	Copropriété auquel ils feront adhérer leurs acquéreurs, ils prévoient de se réserver des droits exorbitants
	tels le droit d'affouiller le sol ou de surélever l'immeuble.
	114 Plus récemment, pour la réalisation d’une nouvelle construction
	115 Cf. IRC … 2018, par Cédric Jobelot
	116 Versailles 7 novembre 1983 . Rev. Loyers 1985, p 386

	Or de telles conventions sont inscrites dans les Règlements de Copropriété dans l'intérêt du bénéficiaire
	exclusivement et en conséquence n'étaient aucunement justifiées au regard de l'objet du syndicat des
	copropriétaires. C’est pourquoi le législateur en a limité la portée, au point de procéder, sous certaines
	conditions à une véritable << expropriation privée >>\footnote{L'expression est de \nom{Cabanac}.}.
	Seul le droit de mitoyenneté échappait à cette censure.
	
	\subsubsection{L’article 37 prévoit tout d’abord une limitation de la validité de ces conventions à 10	ans}
	
	Depuis le 10 juillet 1975 toutes les conventions antérieures au 10 juillet 1965 par lesquelles un
	copropriétaire ou un tiers se réservait l'exercice d'un des accessoires sont caduques.
	Pour les règlements postérieurs à la loi du 10 juillet 1965, la caducité est également encourue si le droit
	réservé à un copropriétaire par le règlement de copropriété n’a pas été mis en oeuvre dans le délai de 10
	ans suivant la rédaction du règlement de copropriété.
	
	\subsubsection{Mentions obligatoires}
	
	 La convention doit indiquer, à peine de nullité :
	- l'importance des locaux à construire
	- la consistance des locaux à construire
	- les tantièmes de copropriété et de charges affectés aux lots ont construire.
	
	En d'autres termes la Convention ne sera valable que dans la mesure où son bénéficiaire aura permis aux
	autres copropriétaires, soit lors de leur acquisition dans l'immeuble, soit lors du vote en assemblée
	générale, de savoir avec précision la nature et même l'aspect des constructions qui risquent d'être un jour
	réalisées et de connaître les répercussions que cette construction aura dans les droits et charges de la
	copropriété.
	
	Par contre, pour être valable, la convention ne doit pas obligatoirement préciser la superficie des locaux
	qui seront réalisés. Malgré l'opinion contraire émise par MM LAFOND et STEMMER, on ne saurait ajouter
	au texte une condition qui n'y est pas précisée (Cour de Versailles 23 avril 1992) De la même façon il ne
	semble pas qu'il soit nécessaire de préciser l'affectation projetée pour ces locaux, affectation et
	consistance étant deux choses distinctes.
	
	\subsubsection{Le droit d'opposition du syndicat.}

	Le syndicat peut à la majorité de l'art. 25 (501/l000\degre{}) s'opposer à l'exercice de ce droit. Si le syndicat exerce
	ce droit d'opposition, il n'a pas à motiver sa décision.
	En cas d'exercice du droit d'opposition, le copropriétaire bénéficiaire de ce droit ne sera pas indemnisé, à
	moins de justifier que la réserve de droit inscrite à son profit comportait une contrepartie à sa charge.
	
	\subsubsection{Disparition des conventions de l’article 37 sur les « droits de construire » à compter de la loi \no 2018-1021 du 23 novembre 2018 dite ELAN}
	
	La très grande fragilité des conventions de l’article 37 depuis la loi \no 65-557 du 10 juillet 1965, ont peu à
	peu fait tomber celles-ci en désuétude, en tous cas pour ce qui concerne les conventions relatives aux
	« droits de construire » (affouiller, surélever, édifier), au profit d’une autre technique notariale, celle du
	lot « transitoire ». La loi \no 2018-1021 du 23 novembre 2018 dite ELAN acte ces évolutions et interdit,
	pour l’avenir, les conventions portant sur ces droits (article 37-1 de la loi \no 65-557 du 10 juillet 1965)
	« Art. 37-1. – Par dérogation à l’article 37, les droits de construire,d’affouiller et de
	surélever ne peuvent faire l’objet d'une convention par laquelle un propriétaire ou un tiers
	se les réserverait. Ces droits peuvent toutefois constituer la partie privative d’un lot
	transitoire. (art. 59 bis F)
	II (nouveau). – Les conventions par lesquelles un tiers ou un copropriétaire s’est réservé,
	dans les conditions prévues à l’article 37 de la loi \no 65-557 du 10 juillet 1965 fixant le statut
	de la copropriété des immeubles bâtis, dans sa rédaction antérieure à la publication de la
	présente loi, l’exercice d’un droit de construire, d’affouiller ou de surélever, demeurent
	valables.
	
	\paragraph{Les conventions antérieures à la loi ELAN}
	Celles-ci demeurent valables si elles respectent les dispositions de l’article 37 de la loi, donc si le droit de
	construire, d’affouiller ou de surélever est parfaitement défini, dès lors que ce droit a été constitué depuis
	moins de dix ans et n’a pas été remis en cause par une assemblée générale. Compte tenu de la caducité
	de l’article 37, toutes ces conventions auront néanmoins disparu au plus tard en 2028.
	
	\paragraph{Les conventions postérieures à la loi ELAN}
	Celles-ci sont désormais impossibles si elles portent sur un droit de construire, d’affouiller ou de surélever
	l’immeuble. Ce qui signifie qu’elles demeurent valables si elles portent sur un autre doit accessoire.
	Cette disposition concerne manifestement le droit de mitoyenneté afférent aux parties communes (prévu
	dans la loi de 1965) que le droit d’affichage qualifié de droit accessoire par la loi ELAN.
	Mais cette disposition concerne-t-elle également le droit administratif de construire ? La réponse semble
	devoir être négative de par la généralité des termes employés : « le droit de construire » pouvant être
	aussi bien le droit d’édifier de nouveaux bâtiments que le droit administratif de construire.

	\subsection{L’integration de droits reputes accessoires aux parties communes a des lots privatifs}
	
		Cette faculté est désormais expressément prévue par l’article 37-1 de la loi \no 65-557 du 10 juillet 1965
		Issu de la loi ELAN:
		Article 37-1 Créé par LOI \no2018-1021 du 23 novembre 2018 - art. 208 (V)
		Par dérogation à l'article 37, les droits de construire, d'affouiller et de surélever ne peuvent faire l'objet
		d'une convention par laquelle un propriétaire ou un tiers se les réserverait. Ces droits peuvent toutefois
		constituer la partie privative d'un lot transitoire.
		Ce faisant la loi ELAN entérine une pratique notariale constante, validée par la jurisprudence.
		
		\subsubsection{Fondement de la pratique notariale}
		
			L’ article 3 précise que << Sont réputés droits accessoires aux parties communes $\dots$ >> les droits de construire
			(surélever, affouiller, édifier), de mitoyenneté et d’affichage. Il établit un présomption réfragable, et il
			n’est pas d’ordre public.
			
			Il est donc tout à fait possible pour le rédacteur du règlement de copropriété de considérer que les droits
			ainsi décrits (notamment le droit de construire au sens large) peuvent être placés par le règlement de
			copropriété en droits accessoires à des parties privatives et non pas en droits accessoires à des parties
			communes.
			
			Aussi, pour éviter la rigueur des dispositions de l'article 37, la pratique notariale a souvent placé les droits
			« accessoires » dans la constitution même du lot privatif; quant à la jurisprudence elle a parfois qualifié
			ces droits de << droits accessoires à des parties privatives >> alors même que ces droits n'ont pas été
			expressément placés comme constitutifs de lots de copropriété.
			
			L’assemblée générale peut également céder un tel droit, après l’avoir constitué en un « lot transitoire ».
		
		\subsubsection{Consécration par la jurisprudence}
		
			Mieux encore qu’un droit accessoire aux parties privatives, le droit de construire est en effet constitutif
			d’une partie privative au sens de l’article 1er de la loi du 10 juillet 1965 (Civ. 3ème 30 juin 1998 ; Loyers et
			Copropriété 1998 \no 254 ou encore Civ 3ème 30 sep 1998 ; Loyers et Copropriété 1999 \no 25).
			Exemple : Lot \no 1 ; droit d'édifier un bâtiment de x étages sur le terrain délimité par les lettres ABCD
			sur le plan annexé ainsi que x x/1000 èmes des parties communes générales de l'immeuble.
			Il est possible également d’intégrer ce droit dans la définition du lot privatif figurant à l’état descriptif de
			division, en modulant en fonction les tantièmes affectés au lot.
			Exemple : Lot \no X, "Un appartement avec terrasse et droit d'édifier un étage supplémentaire sur
			cette terrasse".
		
			En ce cas, le copropriétaire n’est pas tenu de demander au syndicat la cession du droit de construire
			comme dans l’hypothèse précédente. Il devra en revanche, à l’époque où il entendra entreprendre les
			travaux, obtenir l’autorisation de l’assemblée si la construction est susceptible d’affecter les parties
			communes ou l’aspect extérieur de l’immeuble, en application de l’article 25 b de la loi.
			Une décision de la 3ème Chambre Civile de la Cour de Cassation en date du 8 juin 2011 a même admis que
			le promoteur n’était pas soumis aux règles d’autorisation de la copropriété « puisqu’en vertu du règlement
			de copropriété, son titulaire bénéficiait du droit d’édification de tous bâtiments et constructions ».\footnote{
			Une société d’HLM avait édifié une copropriété par tranches successives, se réservant dans le règlement de copropriété et dans	l’état descriptif de division, le « droit d’édification de tous bâtiments et constructions », sous la forme d’un lot transitoire assorti	de tantièmes de parties communes.
			}
			On notera cependant un arrêt de la Cour d’Appel de Paris en date du 21 janvier 2015 RG \no 13/03562
			rendu sur un lot simplement défini comme « la propriété exclusive et particulière d'un droit à construire
			sur la partie de terrain cadastrée (xx) avec droit à la jouissance exclusive et particulière dudit terrain » qui
			considère que : « la consistance de ce droit n'étant pas clairement définie dans le règlement de
			copropriété, sa mise en oeuvre nécessitait une autorisation de l'assemblée générale des copropriétaires »
			(majorité de l’article 25-II-b).
		
		\subsubsection{Le « Lot transitoire » consacré et encadré par la loi ELAN}
		
			La loi \no 2018-1021 du 23 novembre 2018 dite ELAN, tout en consacrant la technique du lot transitoire, a
			tâché d’en encadré la pratique, dans un effort assez semblable à celui de l’article 37 en 1965, en réaction
			à cette jurisprudence libérale.
		
			Dans le cadre de la mise en œuvre de son droit à construire institué en lot transitoire, la Société d’HLM a supprimé un certain
			nombre d’espaces et aménagement (voie de circulation goudronnée et ses équipements, un rond-point) que les demandeurs à la
			procédure considéraient partie communes ou parties à usage commun.
			
			A ce titre, ces derniers soutenaient qu’une autorisation de l’assemblée générale aurait été nécessaire et sollicitaient le
			rétablissement desdits aménagements.
			
			La Cour de Cassation, tout comme la Cour d’Appel a néanmoins jugé :
			\begin{itemize}
				\item D’une part, que le lot transitoire considéré ne faisait état d’aucune partie commune générale ou spéciale.
				\item  D’autre part, que les aménagements supprimés n’étaient que des aménagements provisoires dont les copropriétaires
				n’avaient qu’un droit de jouissance temporaire puisque lesdits équipements n’avaient pas la qualité de parties
				communes et étaient appelés à disparaître en fonction de la poursuite du programme de construction.
				\item  Qu’en conséquence, le droit à construire sur le lot transitoire n’était pas soumis aux règles d’autorisation de la
				copropriété.
			\end{itemize}
			
			L’arrêt considéré ne se prononce pas sur l’implication des constructions litigieuses sur les parties communes ou l’aspect extérieur de
			l’immeuble, mais uniquement sur la suppression d’aménagements existants.
		
			Or, en ce qui concerne de tels aménagements, la Cour d’appel puis la Cour de Cassation, ne les ont pas retenus en tant que parties
			communes ou éléments d’équipement communs mais en tant qu’aménagements provisoires dont les copropriétaires n’avaient qu’un
			droit de jouissance temporaire.
			
			Article 1 alinéa 3-issu de la loi \no 2018-1021 du 23 novembre 2018 dite ELAN
			Ce lot peut être un lot transitoire. Il est alors formé d'une partie privative constituée d'un droit de
			construire précisément défini quant aux constructions qu'il permet de réaliser sur une surface
			déterminée du sol, et d'une quote-part de parties communes correspondante. La création et la
			consistance du lot transitoire sont stipulées dans le règlement de copropriété.
			
			Article 37-1 Créé par LOI \no2018-1021 du 23 novembre 2018 - art. 208 (V)
			Par dérogation à l'article 37, les droits de construire, d'affouiller et de surélever ne peuvent faire l'objet
			d'une convention par laquelle un propriétaire ou un tiers se les réserverait. Ces droits peuvent toutefois
			constituer la partie privative d'un lot transitoire.
			
			Ainsi :
			\begin{itemize}
				\item La technique du lot transitoire est consacrée et devient la seule technique possible
				\item  Ce droit est encadré : le droit de construire doit être précisément défini quant aux constructions
				qu'il permet de réaliser et d'une quote-part de parties communes correspondante. La référence au
				sol, qui figurait dans le texte de la loi ELAN, a été vivement condamnée par la pratique et par la
				doctrine \footnote{Cf. notamment l’article du Professeur POUMAREDE à la Revue de Droit Immobilier, Dalloz, 2019, 44} puisqu’ainsi rédigé le texte ne permettait plus de créer de lots transitoires de
				surélévation pourtant relativement fréquents en pratique à une époque où la disparition de terrains
				à bâtir conduit le même législateur à favoriser la surélévation des immeubles. C’est pourquoi
				l’Ordonnance du 30 octobre 2019 rectifie le tir en supprimant les mots malvenus « sur une surface
				déterminée du sol ».
				\item  Ce lot transitoire doit être mentionné dans le règlement de copropriété. Or, la plupart du temps, il
				ne figurait que dans l’état descriptif de division
				\item  L’article 206 \II de la loi \no 2018-1021 du 23 novembre 2018 dite ELAN fait obligation au syndicat de
				copropriété de mettre, dans un délai de 3 ans à compter de la promulgation de la loi, les règlements
				de copropriété en conformité avec les dispositions relatives aux lots transitoires, donc l’obligation
				faite à l’assemblée générale de définir la consistance du lot transitoire figurant dans l’état descriptif
				de division de l’immeuble existant. Toutefois, ce texte ne dit rien sur les conséquences de l’absence
				de mise à jour ( inexistence du droit comme contraire à l’article 1 ? Mais l’article n’est pas d’ordre
				public $\dots$), ou la compétence du Juge en cas de refus d’adaptation.
			\end{itemize}

\section{Parties communes a jouissance privative}\footnote{Sur la question lire l’excellent article de Jean-Marc Roux in Loyers et Copropriété d’octobre 2004 Et 9.}

	Bien que le législateur n’ait aucunement évoqué cette possibilité, la pratique (c’est à dire les rédacteurs
	des règlements de copropriété) a imaginé de placer une partie d’immeuble en partie commune, mais d’en
	donner la jouissance privative à un lot déterminé.

	C’est ainsi que le sol, un jardin, une cour, une terrasse pourront être en tout ou en partie classés en
	<< parties communes à jouissance privative >>.
	
	On s’interroge sur la nature juridique d’un droit de jouissance sur partie commune. A l’évidence il ne s’agit
	pas d’une servitude : nous ne sommes pas en présence de propriétés distinctes : le titulaire d’un droit de
	jouissance exclusive n’est pas propriétaire, quand bien même ce droit de jouissance doit être l’accessoire
	d’un lot.
	
	Les auteurs se sont interrogés sur la régularité et la qualification d’un tel droit de jouissance privatif. Mais
	la jurisprudence a très tôt reconnu la validité de cette institution. C’est la jurisprudence qui a élaboré
	progressivement le régime de ces « droits de jouissance ».
	
	La loi \no 2018-1021 du 23 novembre 2018 dite ELAN a consacré cette institution :
	\begin{quote}
		Art. 6‑3 de la loi \no 65-557 du 10 juillet 1965 \\
		Les parties communes à jouissance privative sont les parties communes affectées à l’usage et
		à l’utilité exclusifs d’un lot. Elles appartiennent indivisément à tous les copropriétaires. Le droit
		de jouissance privative est nécessairement accessoire au lot de copropriété auquel il est
		attaché. Il ne peut en aucun cas constituer la partie privative d’un lot.
	\end{quote}
	
	\subsection{Le droit de jouissance exclusive est un droit reel}
	
		\subsubsection{Le droit de jouissance est transmissible aux propriétaires successifs du lot}
		
			Si ce droit de jouissance est accordé à un copropriétaire et non attaché à un lot, il ne s’agit plus alors d’un
			droit réel, mais d’un droit personnel qui disparaît avec le décès de son bénéficiaire ou la cession du lot à
			un tiers.
			
			Au contraire, lorsque le « droit de jouissance exclusif » attaché au lot, suit celui-ci entre les mains de tout
			ayant droit du copropriétaire du lot : constituant un droit réel, il est transmissible à tous les acquéreurs
			successifs\footnote{Civ 3\degre{} 17 juin 1997, Loyers et Copropriété 1997 \no 296}, et ce même si son titulaire y renonce, tant que cette renonciation ne se traduit pas par un
			modificatif au règlement de copropriété\footnote{Civ.3ème 21 juin 2006}.
			Cette création d’un nouveau démembrement de la propriété, non prévu par le Code Civil, n’était pas
			évident, car la liste des droits réels prévus par le code civil a longtemps été considérée comme limitative.
			Cependant, le premier arrêt MAISON DE LA POESIE\footnote{Note 14 Cass. 3e civ., 31 oct. 2012, \no 11-16.304 : JurisData \no 2012-024285 ; Bull. civ. 2012, III, \no 159 ; JCP G 2012, \no 52, 1400,
				note F.-X. Testu ; Defrénois 2013, p. 13, obs. L. Tranchant ; LPA 16 janv. 2013, p. 11, note F.-X.Agostini. – V. sur cette décision, H. Périnet-
				Marquet, La liberté de création des droits réels est consacrée : Constr.-urb. 2013, Repère 1 . – V. précédemment Cass. 3e civ., 23 mai
				2012, \no 11-13.202 : JurisData \no 2012-010886 ; Bull. civ. 2012, III, \no 84 ; LPA 24 oct. 2012, p. 12, note J.-F. Barbieri. – V. sur cette
				décision, F. Danos, Perpétuité, droits réels sur la chose d'autrui et droit de superficie : Defrénois 2012, art. 40637, p. 106 ; ZALEWSKISICARD  Vivien 03/10/2016} en date du 31 octobre 2012, a admis qu’ « un
			propriétaire peut consentir, sous réserve des règles d'ordre public, un droit réel conférant le bénéfice d'une
			jouissance spéciale de son bien ». Il s’agissait précisément d’un droit de jouissance exclusif, cet arrêt
			consacrant le fait que la liste des droits réels n’était pas limitative.
		
		\subsubsection{Le droit de jouissance est perpétuel}
		
			Si le droit de jouissance exclusif est un démembrement de la propriété des parties communes, il devrait,
			en principe, ne pas avoir un caractère perpétuel : le droit de propriété a vocation a se reconstituer
			pleinement à terme, comme prévu par l’article 619 (usufruit) ou 625 ( droit d’usage et d’habitation).
			Pour autant, les droits de jouissance exclusifs sur les parties communes ne sont jamais assortis d’une
			durée. Il est vrai que ce droit s’exerce lui-même sur des parties communes en indivision « perpétuelle »,
			ce qui est déjà une anomalie au regard des règles du Code Civil. Cette question de la perpétuité du droit
			de jouissance exclusif a donné lieu à plusieurs arrêts récents :
			
			\begin{itemize}
				\item \item Dans un arrêt du 28 janvier 2015, la cour de cassation\footnote{3\degre{} Ch Civ 28 janvier 2015 (14-10013) Sur le site de la Cour de Cassation} affirme que : « lorsque le propriétaire
				consent un droit réel, conférant le bénéfice d’une jouissance spéciale de son bien, ce droit, s’il n’est pas limité
				dans le temps par la volonté des parties, ne peut être perpétuel et s’éteint dans les conditions prévues par
				les articles 619 et 625 du code civil » (l’article 619 est relatif à l’usufruit et l’article 625 au droit d’usage
				et d’habitation. Cette formule, très générale, semblait pouvoir s’appliquer au droit de jouissance
				privative sur une partie commune. Dans le cas d’espèce, le syndicat des copropriétaires avait
				consenti à l’EDF (aujourd’hui ERDF) un droit d’usage sur un lot composé d’un transformateur sans
				limitation de durée, et ce droit a été considéré comme éteint au bout de 30 ans.
				
				\item Dans un second arrêt MAISON DE LA POESIE en date du 8 septembre 2016\footnote{Propriété - Maison de Poésie II : combien de temps dure la perpétuité en France ? - Note sous arrêt par Julien Laurent 07/11/2016	La Semaine Juridique - Edition générale}, la cour de Cassation	a précisé que la stipulation par laquelle un droit réel de jouissance spéciale est consenti pour la	durée d'une Fondation, ne rend pas ce droit perpétuel, lui permettant d'échapper au terme
				trentenaire que fixent les articles 619 et 625 du Code civil. En d’autres termes, elle considère que
				si le droit été stipulé comme perpétuel, il aurait été soumis à extinction au bout de 30 ans$\dots$ Même
				si, en pratique, il suffit de proroger la durée de vie de la personne morale pour que ce droit soit
				effectivement perpétuel. Ainsi, dans cet arrêt, la Cour de cassation consacre la « quasi perpétuité »
				de ce démembrement de propriété, mais refuse d’aller jusqu’à la perpétuité. Ce positionnement
				est manifestement en décalage avec les arrêts rendus en matière de droit de jouissance sur parties
				privatives en copropriété.
				
				\item Dans un arrêt du 7 juin 2018, publié au Bulletin\footnote{civ. 3\degre{} Ch. 7 Juin 2018, \no 17-17.240, P+B+R+I , Semaine Juridique Edition Générale \no 36, 3 Septembre 2018, 893,	Commentaire Hugues Périnet-Marquet : « Est perpétuel un droit réel attaché à un lot de copropriété conférant le bénéfice	d'une jouissance spéciale d'un autre lot ; la cour d'appel a retenu que les droits litigieux, qui avaient été établis en faveur}, la troisième chambre civile a consacré la	perpétuité du droit réel attaché à un lot de copropriété, ceci quand bien même en l’espèce il ne s’agissait pas du droit de jouissance consenti à un lot sur une partie commune, mais du droit de
				jouissance consenti à l’ensemble des copropriétaires successifs sur une piscine dépendant d’un lot
				privatif En effet, avec le Professeur Périnet-Marquet, on peut penser que : « l’arrêt laisse entendre
				clairement que, dans le cadre des copropriétés, les droits conférés soit sur des lots soit, sans doute,
				sur des parties communes, présentent un caractère perpétuel qui ne peut être remis en cause ».
				Pour lever toute équivoque il eut été bon que la loi ELAN, en créant l’article 6-3 dans la loi de 65 relatif aux
				parties communes à jouissance privative, précisa qu’il s’agissait d’un droit réel perpétuel, mais la mention
				n’y figure pas.
			\end{itemize}
		
		\subsubsection{Une jouissance privative peut faire l’objet d’une prescription acquisitive trentenaire}
		
			Un copropriétaire qui manifeste sans équivoque son intention de se comporter en seul bénéficiaire de la
			jouissance d’une partie commune, en interdisant l’accès aux autres copropriétaires, étant allés jusqu’à
			fermer à clé l’accès à cette cour (Civ. 3\degre{} 24 octobre 2007, pourvoi \no 06-19260)
			« Un droit de jouissance privatif sur des parties communes est un droit réel et perpétuel qui peut
			s’acquérir par usucapion »\footnote{Civ. 3\degre{} 24 octobre 2007, pourvoi \no 06-19260}
			
			On notera également – mais sur un fondement plus traditionnel - l’hypothèse où le syndicat acquiert la
			prescription d’une cave appartenant à un propriétaire voisin dont il a eu la jouissance pendant plus de 30
			ans\footnote{Paris 23\degre{} Ch A 7 mai 2002 Loyers et Copropriété 2002 \no 270}.
			
			Cependant la jurisprudence est particulièrement exigeante pour accepter la prescription acquisitive
			(jouissance trentenaire paisible, publique, continue, non équivoque et à titre de propriétaire
			conformément à l’article 2230 du code civil).
			
			De plus, il est de jurisprudence constante qu’une simple tolérance à l’occupation d’une partie commune
			(autorisée par l’assemblée générale) ne donne pas vocation à son bénéficiaire d’en acquérir la propriété
			ou la jouissance exclusive par usucapion, car nul ne peut prescrire contre son titre\footnote{3e civ., 6 mai 2014, \no 13-16.790, F-D (JCPN du 6 juin 2014) ; 3\degre{} civ. 6 septembre 2018 \no pourvoi: 17-22180 Non publié au bulletin Rejet}.
		
		\subsubsection{}
		
			En tant que droit réel accessoire aux parties communes le droit de jouissance exclusive peut également être concédé par décision d’assemblée générale :
			
			des autres lots de copropriété et constituaient une charge imposée à certains lots, pour l'usage et l'utilité des autres lots
			appartenant à d'autres propriétaires, étaient des droits réels \emph{sui generis} trouvant leur source dans le règlement de
			copropriété et que les parties avaient ainsi exprimé leur volonté de créer des droits et obligations attachés aux lots des
			copropriétaires ; il en résulte que ces droits sont perpétuels
		
			La décision sera alors prise à la double majorité de l’article 26 de la loi\footnote{PARIS 23\degre{} Ch A 21 déc 1994 D 97 IR 244}. Cette cession ne devant pas	cependant être confondue, ni avec la cession de la propriété de la partie commune, ni avec une jouissance	précaire et révocable accordée personnellement à un copropriétaire déterminé.
		
		\subsubsection{Ce droit de jouissance ne peut être remis en cause en cas de non usage}
		
			Les copropriétaires ne peuvent imposer au lot bénéficiaire la perte de ce de droit de jouissance, même
			s’il n’est plus mis en œuvre depuis 30 ans – en l’espèce, la station essence n’utilisait plus le droit de
			jouissance sur la cour\footnote{Civ 3\degre{} 4 mars 1992, Rép. Déf. 1992-1140, obs Henri Souleau ; D 1992-2-386 Ch. Atias.}
	
	\subsection{Les attributs du droit de jouissance}
	
		\subsubsection{Ce droit emporte la jouissance exclusive de la partie commune}
		
			Le copropriétaire qui a la jouissance exclusive d'une partie du sol est en droit de s'opposer au passage des
			occupants de l'immeuble par cette partie commune à jouissance privative, ce quand bien même le passage
			est demandé pour les besoins de l'entretien de l'immeuble.
			
			En l'espèce, "le titulaire d'un droit à jouissance exclusive du jardin, est en droit de s'opposer, sans
			commettre d'abus, au passage quotidien des conteneurs à ordures ménagères, dont le transit
			pouvait, par un aménagement approprié, s'effectuer par les parties communes de l'immeuble non
			affectées à un droit de jouissance privative".\footnote{PARIS 14 décembre 1990 - D. 91 IR 15}
			
			C’est ainsi que la Cour de Cassation a jugé\footnote{Civ 3 19 déc 1990, D 91 IR p 15} que : 
			\begin{quote}
				Le droit de jouissance, affecté d'une quote-part
				des parties communes correspondant aux charges que son titulaire supporte pour la conservation
				et l'entretien de la cour, n'est pas assimilable à un droit de propriété et ne donne pas à son titulaire
				la possibilité de transformer en local clos privatif une cour rangée par le règlement de copropriété
				dans les parties communes.
			\end{quote}
		
			Le copropriétaire titulaire d’un droit de jouissance exclusif a qualité à agir pour faire cesser un
			empiétement sur la partie commune sur laquelle s’exerce son droit de jouissance exclusif\footnote{(Cass. 3e civ., 15 décembre 2016, \no 15-22.583, F-D : JurisData \no 2016-028801 ; Loyers et copr.
			2016, comm. 55, note A. Lebatteux)}
			\begin{quote}
				Sur le moyen unique, ci-après annexé :
			
				Attendu, selon l'arrêt attaqué (Pau, 29 mai 2015), que la SCI S… et M. X..., copropriétaires de lots
				dans le bâtiment A d'un immeuble en copropriété, ont obtenu, par délibération de l'assemblée
				générale des copropriétaires de la résidence C$\dots$ (le syndicat des copropriétaires) du 12 décembre
				2005, réitérée le 19 décembre 2008, l'autorisation d'affouiller le sol d'un terrain affecté à la
				jouissance exclusive du bâtiment A pour y construire une piscine ; que M. Y$\dots$, copropriétaire de deux
				lots dans un autre bâtiment de cet immeuble, se plaignant de l'empiétement de la piscine sur le
				jardin affecté à son usage privatif, a assigné la SCI S…, M. X$\dots$ et le syndicat des copropriétaires en
				annulation de la délibération de l'assemblée générale du 19 décembre 2008 et démolition de la
				piscine par les deux premiers ;
				
				Si un droit de jouissance exclusive sur des parties communes n'est pas un droit de propriété, le
				titulaire de ce droit réel et perpétuel a qualité et intérêt à assurer la défense en justice, sur le
				fondement de l'article 15 de la loi du 10 juillet 1965 $\dots$
			\end{quote}
		
		\subsubsection{Le droit de jouissance n’est pas un droit de propriété, en sorte que la partie communes	reste indivise entre tous les copropriétaires}
		
			\begin{itemize}
				\item Le droit de jouissance n’emporte pas requalification de la partie commune, ce n’est qu’un droit de
				« superficie\footnote{Civ 3\degre{} 29 oct 1973 Bull Civ III \no 552} ».
				
				La question pouvait se poser compte tenu de la rédaction de l’article 2 : « sont parties privatives
				les parties des bâtiments et des terrains réservés à l’usage exclusif d’un copropriétaire ». Sur ces
				dispositions un copropriétaire dont le lot est constitué d’une maison et de la jouissance privative
				d’un terrain, affirme qu’il a en fait la propriété du terrain correspondant. La cour de cassation
				répond :
				\begin{quote}
					qu’ayant, par motifs propres et adoptés, relevé que les lots des copropriétaires étaient
					composés du droit à la jouissance exclusive et privative d’une parcelle de terrain sur
					lesquels est implantée chaque maison et la propriété privative des constructions ainsi que
					de millièmes de parties communes, la cour d’appel a retenu, à bon droit et sans
					dénaturation, que seul un droit réel de jouissance était conféré aux copropriétaires et que
					le sol était une partie commune\footnote{Civ. 3ème Ch. 2 octobre 2013, \no pourvoi 12-17084 – au Bulletin}.
				\end{quote}
				
				Ce principe est rappelé dans l’article 6-3 de la loi de 1965 issu de la loi loi \no 2018-1021 du 23
				novembre 2018 dite ELAN :
				[Les parties communes indivises] appartiennent indivisément à tous les copropriétaires.
				
				\item  Le droit de jouissance ne confère pas le droit de construire.
				
				La jouissance privative d’une partie commune n’entraîne en aucun cas un doit de construire, lequel
				demeure un accessoire des parties communes\footnote{Civ 3\degre{} 22 juillet 1987 D 87 IR p 187}. De la même façon le bénéficiaire ne peut affouiller le sol
				par exemple pour y creuser une piscine.
				
				Il est même douteux que le copropriétaire bénéficiaire de ce droit de jouissance ait la faculté de clore la
				partie dont il a la jouissance\footnote{Civ 3\degre{} 22 juillet 1987 D 87 IR p 187}.
			
				\item  Le droit de jouissance ne permet pas le retrait de la copropriété.
				Un tel droit de jouissance n’étant pas un droit de propriété ne permet pas à son titulaire de retirer le lot
				de la copropriété pour cette partie de jouissance\footnote{Civ 3\degre{} 29 janvier 1997, Administrer juin 1997 p. 42 note Capoulade} quand bien même serait-elle assortie du droit de
				construire.
			\end{itemize}
		
		\subsubsection{Le droit de jouissance s’exerce dans le respect de la destination de la partie commune}
		
			Le droit de jouissance exclusif sur une partie commune ne modifie pas la destination de cette partie
			commune ; il doit donc être conforme aux dispositions de l’acte qui l’a institué\footnote{Civ. 3\degre{}, 20 mars 2002, Loyers et Copropriété 2002, \no 159 ; Civ. 3\degre{} 26 mai 2006 ; Juris-Data \no 2006-302912 ; Civ. 3\degre{}, 8 nov 2006,Administrer jan 2007 p. 59 ; Versailles 30 jan 2012, Loyers et Copr 2012 \no 152}.
		
			Notons qu’il n’est pas toujours facile de définir ce qui relève de la jouissance privative d’une partie
			commune : par exemple lorsque le jardin est à jouissance privative … l’arbre est-il privatif ? La réponse
			semble devoir être négative\footnote{PARIS 23\degre{} Ch B 11 avril 2002 Loyers et Copropriété 2002 \no 268}. Il a été même été jugé que la décision de l'assemblée générale d’effectuer
			des plantations dans un jardin à jouissance privative en remplacement de végétaux malades ne portait pas
			atteinte aux modalités de jouissance de son lot par le copropriétaire\footnote{Administrer \no 375 mars 2005}.
		
		\subsubsection{Le droit de jouissance peut générer une obligation spécifique d’entretien}
		
			En principe, le copropriétaire titulaire du droit de jouissance privative doit assurer l’entretien du
			revêtement superficiel, et même sa remise en état à la suite de travaux engagés par le Syndicat des
			Copropriétaires sur l’étanchéité partie commune :
			\begin{itemize}
				\item les travaux rendus nécessaires par l'état de la terrasse relèvent du syndicat pour le grosoeuvre
				et l'étanchéité de l'ouvrage, et du bénéficiaire du droit d'usage exclusif pour le surplus
				(Cass. 3e civ., 18 déc. 1996 : Juris-Data \no 1996-005007 ; Loyers et copr. 1997, comm. 90. –
				CA Paris, 20 juin 2001 : Juris-Data \no 2001-146857).
				\item le syndicat décideur des travaux est fondé à réclamer au copropriétaire le coût des
				dépenses engagées pour la remise en état du revêtement superficiel, de même que pour la
				remise en place des autres installations privatives réalisées sur la terrasse (Cass. 3e civ.,
				30 avr. 2002 : Juris-Data \no 2002-014270 ; Administrer nov. 2002, p. 36. – CA Aix-en-
				Provence, 1re ch. civ., 6 mai 1997 : D. 1998, somm. 122. – CA Versailles, 20 déc. 1990 :
				Administrer mars 1991, p. 64).
				\item Les frais de pose de carrelage des terrasses doivent, après la réfection de leur étanchéité, être
				supportés par les seuls copropriétaires qui en ont la jouissance exclusive et l'obligation
				d'entretien (CA Aix-en-Provence, 6 mai 1997 : D. 1998, somm. p. 122. – CA Reims, 27 sept.
				2004 : JurisData \no 2004-266119 ; JCP G 2005, IV, 2402).
				\item doit être qualifiée de partie commune à usage exclusif d'un copropriétaire le toit-terrasse
				d'un immeuble. Une distinction doit cependant être faite entre le revêtement superficiel et
				tous ses accessoires, partie privative et l'ossature de l'immeuble, y compris l'étanchéité qui y
				est incorporée qui est une partie commune. Lorsque des travaux d'étanchéité sont effectués
				sur un toit-terrasse, partie commune à jouissance exclusive, les frais de dépose et de repose
				des aménagements privatifs mis en place, en l'absence de disposition contraire du règlement
				de copropriété, doivent rester à la charge du copropriétaire bénéficiaire du droit de
				jouissance exclusive. (Cour d'appel Bastia, 11 Juin 2008, \no 06/01099)
				Le copropriétaire ne peut pas refuser l’accès au Syndicat des Copropriétaires pour effectuer les travaux
				sur les parties communes dont il a la jouissance, en revanche le syndic doit respecter les modalités prévues
				par l’article 9 de la loi du 10 juillet 1965 (cf infra, poly 2 : les travaux)
			\end{itemize}
		
			Ce principe est consacré par l’ordonnance du 30 octobre 2019 qui complète l’article 6-3 par un dernier
			alinéa
			\begin{quote}
				Le règlement de copropriété précise, le cas échéant, les charges que le titulaire de ce droit de
				jouissance privative supporte.
			\end{quote}
	
	\subsection{Le droit de jouissance d’une partie commune est necessairement l’accessoire d’un lot comportant des parties privatives}
	
		\subsubsection{Le droit de jouissance est indivisible par rapport au lot auquel il est attribué}\footnote{Civ 3\degre{} 8 juillet 1992 RDI 1992 p 364}
		
			Le lot ne doit pas être constitué du seul droit de jouissance : le droit de jouissance est présenté
			comme le complément d’un lot ; par exemple : << lot \no 1 appartement au rez-de- chaussée et
			jouissance exclusive d’une cour >>.
			
			En effet, un lot de copropriété comporte nécessairement une partie privative et une quote-part de parties
			communes ; or, un lot ne comportant qu’un droit de jouissance ne comporte pas de partie privative ; selon
			l’expression de M \nom{CAPOULADE} : « Il n’en représente qu’une dépendance, un complément ou un
			accessoire ».
			
			Cette question est à l’origine d’une procédure complexe dans une affaire Fournier c/ Syndicat Pauline
			Borghèse, dans laquelle un copropriétaire détenait un lot composé « du droit de jouissance du jardin et
			de 48/10.000 ème de propriété du sol et des parties communes\footnote{Civ.3ème 6 novembre 2002 puis Civ3ème 1er Mars 2006} >>.
			La cour de cassation dans un arrêt du 6 novembre 2002 constate qu’un lot de jouissance $\dots$ ne peut être
			un lot de copropriété ; elle casse. La cour de renvoi (Paris 23 juin 2004) en déduit que le lot a
			disparu. Cassation à nouveau le 1er mars 2006. L’arrêt ne nie pas l’existence du droit de jouissance
			mais considère que n’étant pas attaché à un lot, les tantièmes affectés au lot doivent être retranchés
			des tantièmes de propriété. Le Président \nom{Capoulade}\footnote{Administrer février 2003 p. 45} et M \nom{Vigneron}\footnote{Loyers et Copropriété mai 2006 \no 112} en déduisent que ces
			tantièmes n’ont d’existence qu’en tant que tantièmes de charges (Civ. 3\degre{} 4 mai 1995, Bull. civ. \no
			113).
			
			Un arrêt de la 3ème Chambre de la Cour de Cassation\footnote{Civ. 3\degre{} Arrêt \no 546 Syndicat des Copropriétaire Les Rotondes}, en date du 6 juin 2007 semble apporter un point
			final à cette discussion :
			\begin{quote}
				« Un droit de jouissance exclusif sur des parties communes n’est pas un droit de propriété et ne peut	constituer la partie privative d’un lot ».
			\end{quote}
			
			Cette décision est d’autant plus importante qu’il ne s’agissait pas en l’espèce d’un lot de jouissance du sol
			mais d’un « lot de jouissance exclusive et particulière d’emplacement de stationnement ». On peut penser
			que si l’auteur de l’état descriptif de division avait simplement qualifié le lot « d’emplacement de
			parking », nous aurions été en présence d’un véritable lot comportant une partie privative et non un
			simple lot de jouissance.
			
			Pour autant un lot constitué à la fois d’un droit de jouissance exclusive et d’un droit de construire, même
			non défini dans sa consistance future, n’est pas un « lot de jouissance » mais un « lot transitoire », c’est-à-
			dire un lot comme les autres comportant bien une partie privative\footnote{Civ 3\degre{} - Arrêt \no 699 du 8 juin 2011 (10-20.276) Bulletin numérique des arrêts publiés}
			\begin{quote}
				De la même façon, ce droit de jouissance ne peut être aliéné séparément du lot auquel	il est attaché\footnote{Paris 23\degre{} Ch 16 fév 2006, SCI ALTAIR c/ Syndicat 34 rue Guynemer}, même pour être rattaché à un autre lot.
			\end{quote}
			
			Toutefois la cour de cassation dans un arrêt du 17 décembre 2013 a admis implicitement que le droit de
			jouissance appartenant à un propriétaire peut être partagé entre deux lots, avec l’accord de l’assemblée
			générale du syndicat des copropriétaires\footnote{Cass. 3e civ., 17 déc. 2013, \no 12-23.670, FS-P+B}.
			Jurisprudence sans doute caduque avec l’entrée en vigueur de la loi ELAN puisque le nouvel article 6-3 de
			la loi du 10 juillet 1965 édicte que « Le droit de jouissance privative est nécessairement accessoire au lot
			de copropriété auquel il est attaché ».
		
		\subsubsection{Un « lot de jouissance » ne peut être doté de tantièmes de copropriété}
		
			Ce principe figure désormais dans l’article 6-3 issu de la loi loi \no 2018-1021 du 23 novembre 2018 dite
			ELAN : « Le droit de jouissance privative est nécessairement accessoire au lot de copropriété auquel il est
			attaché. Il ne peut en aucun cas constituer la partie privative d’un lot. »
		
			Un droit de jouissance exclusif sur les parties communes n'étant pas un droit de propriété ne peut
			constituer la partie privative d'un lot de copropriété. En conséquence un copropriétaire est fondé à
			demander au tribunal qu’il annule les tantièmes de ce lot\footnote{Civ. 3\degre{} 4 nov 2014, Pourvoi \no 13-22243, non publié au Bulletin}.
			\begin{quote}
				Vu les articles 1 alinéa 2 et 2 de la loi du 10 juillet 1965 ;
				 
				Attendu que pour rejeter la demande de M. X... tendant à l'annulation des tantièmes de copropriété afférents à
				l'espace dit « lot 19 », la cour d'appel retient que le lot \no 19 ne comporte qu'un jardin et une cour sans aucune
				édification, de sorte qu'il est improprement désigné comme lot de parties privatives et constitue en réalité un droit
				de jouissance exclusif sur des parties communes, mais qu'il est désormais admis que le droit de jouissance exclusif
				d'un copropriétaire soit affecté d'une quote-part des parties communes correspondant aux charges que son titulaire
				doit supporter sans pour autant être assimilé à un droit de propriété ;
				
				Qu'en statuant ainsi, alors qu'elle avait constaté que le lot \no 19 était constitué d'un droit de jouissance exclusif sur		des parties communes et d'une quote-part de parties communes dans la propriété du sol, la cour d'appel n'a pas tiré	les conséquences légales de ses constatations
			\end{quote}
			
			Cet arrêt rend extrêmement compliquée la cession du droit de jouissance par le syndicat des
			copropriétaires à un copropriétaire : en effet, pour réaliser cette cession, il est traditionnellement procédé
			à la création d’un lot » droit de jouissance privative de la terrasse », qui subsiste un instant de raison, puis
			qui est réuni au lot principal. Cette technique semble condamnée si l’on ne peut assortir des tantièmes un
			lot de jouissance $\dots$
		
		\subsubsection{Quel est le sort d’un « droit de jouissance » qui ne peut être rattaché à un lot ?}
		
			Par ailleurs, la question qui se pose inévitablement est le sort des « lots de jouissance » séparées des lots
			principaux, tels que les parkings, qui ont été constitué au fil des années. Sont-ils, du fait de leur irrégularité,
			dépourvu de toute existence et dispensés de toute participation aux charges ? La Cour de cassation a
			répondu de façon pour le moins ambiguë dans un arrêt du 2 décembre 2009, 08-20.310, Publié au bulletin,
			concernant des lots de jouissance de stationnement :
			
			\begin{quote}
				Attendu que le syndicat fait grief à l'arrêt de refuser de constater l'inexistence du droit de jouissance exclusive de M.	X$\dots$ sur les emplacements de stationnement, alors, selon le moyen, qu'un droit de jouissance exclusive sur une partie		commune n'est pas un droit de propriété et ne peut constituer la partie privative d'un lot ; qu'il en résulte que, privé		de cause et d'objet, ce droit disparaît avec le lot le constituant ; qu'en refusant dès lors de constater l'inexistence du	droit de jouissance de M. X$\dots$ sur les emplacements de parking, la cour d'appel a violé l'article 1er de la loi du 10 juillet	1965 ;
				
				Mais attendu que la cour d'appel a exactement retenu que si le seul droit de jouissance exclusif sur un ou plusieurs
				emplacements de stationnement ne conférait pas la qualité de copropriétaire, son titulaire bénéficiait néanmoins
				d'un droit réel et perpétuel et qu'il n'y avait pas lieu de constater que le droit de jouissance exclusif de M. X... sur	ces emplacements avait disparu ;
				
				D'où il suit que le moyen n'est pas fondé ;
			
				Sur le second moyen $\dots$
				
				Attendu qu'ayant relevé que selon les stipulations du règlement de copropriété les bénéficiaires de droit de
				jouissance exclusif sur les emplacements de stationnement n'étaient redevables que des frais d'entretien et de
				réparation de ces emplacements, la cour d'appel a retenu à bon droit, sans dénaturation, que la délibération \no 2 de
				l'assemblée générale du 4 juin 1998 mettant à la charge de ses bénéficiaires une quote part des charges communes,
				alors qu'ils n'avaient pas la qualité de copropriétaires, devait être annulée.
			\end{quote}
			
			Ainsi, la Cour de cassation quelque peu embarrassée reconnaît avoir créé un « \nom{ojni} » (objet juridique non
			identifié) : le droit de stationnement est un droit de jouissance exclusif qui doit être reconnu, alors même
			qu’il ne peut être constitutif d’un lot, au titre duquel le bénéficiaire peut être redevable de frais d’entretien
			et de réparation, mais cette obligation ne peut être traduite en quote-parts de charges communes
			générales ! Le titulaire d’un tel lot se trouve donc dans la situation d’un tiers à la copropriété qui devrait à
			celle-ci une redevance pour la jouissance dont il bénéficie, les modalités de calcul de cette redevance
			n’étant toutefois pas précisées\footnote{Doctrine sur le sujet : Loyers et Copropriété \no 10, Octobre 2015, dossier 6 Droit de jouissance exclusif et copropriété : une histoire	tourmentée Etude par Jacques LAFOND docteur en droit - avocat honoraire au barreau de Paris ; D. SIzaire. ss Cass. 3e civ., 1er mars 2006 \no 04-18.547 : JurisData \no 2006-032514 ; Constr.-Urb. 2006, comm. 139 ; V. C. Calfayan, La notion de partie privative à l'épreuve du droit de jouissance exclusive sur une partie commune : JCP N 2007, \no 1273. – Y. Stemmer, note ss Cass. 3e civ., 24 oct. 2007, \no 06-19.260 : JurisData \no 2007-041007 ; JCP N 2007, 1328. – C. Atias, Jouissance réelle ne vaut pas propriété virtuelle : D. 2007, p. 2358. ; S. Lelièvre et S. Chaix-Bryan, Le droit de jouissance exclusive d'une partie commune : la fin d'un questionnement ? : Defrénois 2007, art. 38637, p. 1173.G. Gil, Les « lots » constitués par la jouissance exclusive d'une partie commune : le point de la situation :	Administrer juin 2013, p. 7 et s. – V. aussi J.-R. Bouyeure, Réflexion sur les conséquences de la nullité des lots dont la partie privative est constituée par un droit de jouissance exclusive sur une partie commune : Administrer oct. 2007, p. 70. – Pour une recension des diverses solutions adoptées, par la jurisprudence, V. V. Matet, La pratique notariale du droit de jouissance exclusif sur les parties communes :	l'histoire d'un inventeur dépassé par sa créature : Bull. Cridon Bordeaux-Toulouse, févr. 2010, 610.B. Kan-Balivet, La nature juridique du droit de jouissance exclusive sur les parties communes : Defrénois 2008, art. 38825, p. 1765. – R. Boffa, La nature juridique du droit de	jouissance exclusive sur les parties communes : LPA 10 nov. 2010, p. 3.}.
	
	\subsection{La consecration dans la loi \no 65-557 du 10 juillet 1965 (article 6-3 et 6-4)}
	
		\subsubsection{La loi \no 2018-1021 du 23 novembre 2018 dite ELAN}
		
			La loi ELAN du 23 novembre 2018 consacre le droit de jouissance privatif sur parties communes, mais
			l’encadre dans les nouveaux articles 6-3 et 6-4 de la Loi 65-557 du 10 juillet 1965 :
			\begin{quote}
				Art. 6‑3. – Les parties communes à jouissance privative sont les parties communes affectées
				à l’usage et à l’utilité exclusifs d’un lot. Elles appartiennent indivisément à tous les
				copropriétaires. Le droit de jouissance privative est nécessairement accessoire au lot de
				copropriété auquel il est attaché. Il ne peut en aucun cas constituer la partie privative d’un
				lot.
			
				Art. 6‑4. – L’existence des parties communes spéciales et de celles à jouissance privative
				est subordonnée à leur mention expresse dans le règlement de copropriété.
			\end{quote}
			
			L’article 6-3 est une consécration de la jurisprudence antérieure, mais a le mérite de stabiliser une situation
			juridique devenue incertaine avec les arrêts « Maison de la Poésie ».
			En revanche, l’article 6-4, en imposant que le droit de jouissance privatif figure non seulement dans la
			description du lot (donc dans l’état descriptif de division) mais également dans le règlement de copropriété
			risque de poser problème, si les modalités de cette intégration ne sont pas précisées.
			En effet, il est prévu dans les dispositions transitoires de la loi ELAN
			II (nouveau). – Les syndicats des copropriétaires disposent d’un délai de trois ans à compter de la
			promulgation de la présente loi pour mettre, le cas échéant, leur règlement de copropriété en conformité
			avec les dispositions de l’article 6-4 de la loi \no 65-557 du 10 juillet 1965 fixant le statut de la copropriété
			des immeubles bâtis.
			
			À cette fin, le syndic inscrit à l’ordre du jour de chaque assemblée générale des copropriétaires la question
			de la mise en conformité du règlement de copropriété. La décision de mise en conformité du règlement de
			copropriété est prise à la majorité des voix exprimées des copropriétaires présents ou représentés.
			Mais que se passera-t-il si le syndicat des copropriétaires refuse cette intégration ? … La décision de
			l’assemblée générale serait-elle valable, mais à charge dans ce cas pour le syndicat des copropriétaires
			d’indemniser le titulaire du droit de jouissance ? Le copropriétaire pourrait demander son annulation pour
			atteinte aux modalités de jouissance de son lot, mais rien n’est prévu pour permettre au magistrat de se
			substituer en ce cas à l’assemblée générale $\dots$
			
			Enfin, les titulaires de droit de jouissance devront être particulièrement attentifs à la régularisation de
			cette situation, sous peine de voir leur droit frappé, au 24 novembre 2021, de caducité.
		
		\subsubsection{L’ordonnance du 30 octobre 2019}
		
			L’Ordonnance du 30 octobre 2019 ajoute un alinéa au nouvel article 6–3 aux termes duquel : « le règlement
			de copropriété précise le cas échéant les charges que le titulaire de ce droit de jouissance privative
			supporte».
			
			Rappelons que par arrêt du 27 mars 2008 , la Cour de cassation a approuvé chaleureusement la cour
			d’appel d’avoir énoncé « exactement » qu’un « droit de jouissance exclusive d’un copropriétaire pouvait
			être affecté d’une quote-part de parties communes correspondant aux charges que son titulaire devait
			supporter sans pour autant être assimilé à un droit de propriété ».
			
			En pratique, lorsque le géomètre définit le lot en y mentionnant la jouissance privative (par exemple un
			appartement de 5 pièces avec jouissance exclusive d’une terrasse) il affecte le lot d’un nombre de
			tantièmes de propriété (et de charges communes) correspondant à ce droit de jouissance privative en
			sorte qu’il n’y a pas lieu à ajouter dans le règlement de copropriété des charges complémentaires à
			supporter par le bénéficiaire de cette jouissance privative.
			
			Par contre, cette nouvelle disposition présente un intérêt lorsqu’un copropriétaire se voit céder une
			jouissance privative par l’assemblée générale : le bout de jardin devant son appartement par exemple.
			Dans cette hypothèse il n’y a pas lieu à création d’un nouveau lot e (qui serait d’ailleurs un lot de
			jouissance) mais l’assemblée générale peut parfaitement prévoir que ce droit de jouissance est assorti de
			X tantièmes des charges communes générales. Au besoin l’assemblée générale pourra adopter un état de
			répartition des charges modificatif majorant la quote-part de charges du lot du copropriétaire attributaire
			de ce droit de jouissance qui, rappelons-le, est et demeure nécessairement l’accessoire d’un lot de
			copropriété.
			
			Cette modification de la répartition des charges sera votée à la même majorité que la décision de cession
			de la jouissance privative, par application de l’article 11 de la loi du 10 juillet 1965 : lorsque des (…) actes
			d'acquisition ou de disposition sont décidés par l'assemblée générale statuant à la majorité exigée par la
			loi, la modification de la répartition des charges ainsi rendue nécessaire peut être décidée par l'assemblée
			générale statuant à la même majorité.
			
			Par ailleurs, cet article peut également se lire comme consacrant la faculté de prévoir, dans le règlement
			de copropriété, que les charges d’entretien de la partie superficielle ( de la toiture, du jardin) incomberont
			au bénéficiaire du lot, comme l’a admis la jurisprudence.

\section{Les parties mitoyennes}

	L'article 7 de la loi dispose :
	\begin{quote}
		Les cloisons ou murs, séparant des parties privatives et non compris dans le gros œuvre, sont présumés mitoyens entre les locaux qu'ils séparent.
	\end{quote}
	
	Par conséquent il existe, à côté des parties communes et des parties privatives, des parties mitoyennes.
	Bien que l'article 43 de la loi incorpore les dispositions de l'article 7 de la loi dans les textes d'Ordre Public,
	la rédaction de ce dernier article autorise l'auteur du Règlement de Copropriété à classer ces cloisons ou
	murs dans les parties communes : la présomption disparaît en présence de la preuve contraire.
	Pour que cette présomption de mitoyenneté s'applique, encore faut-il qu’il s’agisse de cloison ou de mur
	non porteur et non de gros œuvre.
	
	Le texte ne vise que les cloisons et murs non porteurs séparant des parties privatives entre elles. Par contre
	il ne vise pas les cloisons et murs non porteurs situés entre une partie privative et une partie commune. Il
	n’y a pas à ce jour de jurisprudence qui donne la qualification à retenir pour ce dernier type de séparation.
	Dans le silence du Règlement de Copropriété ces cloisons et murs non porteurs séparant un lot des parties
	communes pourront être qualifiés selon le cas de parties communes ou de parties privatives ; la porte
	palière étant pour sa part une partie privative, puisqu’à l’usage exclusif du copropriétaire.

\section{Les servitudes}

	\subsection{Servitude de vue}

		Un arrêt de cassation\footnote{Civ. 3\degre{} 2 décembre 1980, JCP 1981 Ed N. II. 266, note Stemmer} a considéré qu'il ne pouvait y avoir de servitude de vue à l'intérieur d'une
		Copropriété, fut elle horizontale.\footnote{Cf également : Civ 3\degre{} 18 déc 1991 JCP N 92 IV \no 247}
		
		Civ 3ème Chambre 19 juillet 1995 : \\
		Cassation d’un arrêt de la Cour d’Appel ayant retenu l’existence d’une servitude de vue alors que
		dans le régime de la copropriété des immeubles bâtis, les lots ne sont séparés par aucune ligne
		divisoire et que la totalité du sol est partie commune, et que les juges du fond ne relèvent pas si la
		vue directe s'exerçait sur un fonds distinct et indépendant.
	
	\subsection{Servitude de passage}
		
		Il peut paraître utile d'imposer à un lot privatif une servitude de passage soit au profit d'un autre lot, soit
		au profit des parties communes. Si l'on oblige le propriétaire à laisser le droit de passage, cette obligation
		disparaîtra avec le propriétaire obligé, par exemple lorsqu'il vendra son lot. Si par contre on créée une
		servitude de passage, celle-ci se perpétuera quel que soit le propriétaire du lot. La finalité dans l'intérêt de
		la Copropriété est de perpétuer cette obligation de passage au détriment d'un lot et non pas d'une
		personne.
		
		Aussi nombre de Règlements de Copropriétés comportent ce type de clause ou encore on voit des
		Assemblées Générales qui cèdent à un copropriétaire les combles d'un immeuble avec une servitude de
		passage pour l'entretien de la toiture.
		
		La question s'est posée assez rapidement de la validité de ces servitudes à l'intérieur d'une même
		copropriété : en effet, qui dit copropriété dit essentiellement existence d'un « héritage » unique puisque
		nous n'avons qu'une seule propriété partagée entre plusieurs lots, chaque lot comportant une partie
		privative et une quote-part de parties communes.
		
		\subsubsection{La jurisprudence antérieure au 30 juin 2004.}
		
			La réponse donnée a été simple, sinon simpliste : une servitude ne pouvant exister qu'entre deux fonds
			distincts, il ne peut en être créée entre parties communes et lots privatifs. Pas plus qu'il ne peut être créé
			de telles servitudes sur une partie privative au profit d'une autre partie privative, car les parties communes
			comprises dans le lot sont indivises.
			
			Ce principe appliqué en matière de copropriété a été affirmé de par la Cour de Cassation à partir de 1984,
			malgré de vives critiques doctrinales154. Cette jurisprudence est restée constante pendant près de 20 ans\footnote{Civ 3ème 10 janvier 1984 D. 85, 335 puis Civ 3\degre{} 11 jan 1989 RDI 89 p 243, Civ 3\degre{} 6 mars 1991; 51 Civ 3\degre{} 18 juin 1997 Somm. JCP N 1997 \no 47 p. 1430 (passage d’un lot sur un autre).}.
			
			\begin{quote}
				(Il y a) incompatibilité entre la division d'un immeuble en lots de copropriété et la création d'une servitude			sur une partie commune au profit d'un lot.
			\end{quote}
			
			Les opposants à la jurisprudence\footnote{J.-L. Aubert, Rev. Administrer, mai 1993, p. 11, et notes sous Cass. 3e civ., 10 janv. 1984, D. 1985.335, et 30 juin 1992,		D.1993.156 ; P. Capoulade et C. Giverdon, RD imm. 1991.247, 374 ; ibid. 1992.535 ; C. Giverdon, Rev. Huissiers 1993.1333		; E.-J. Guillot, Rev. Administrer, juin 1993, p. 28 ; J. Lafond, JCP éd. N 1990. Prat. 1503, spéc. p. 451 ; Ann. Loyers 1986.115,		obs. R. Martin ; H. Souleau, note sous Cass. 3e civ., 6 mars 1991, D.1991.355 ; Defrénois1992.1140 ; F. Zenati, RTD civ.		1990.} précitée ont fait valoir avec le Professeur Aubert que la règle de l'article	667 implique qu'il y a dualité de propriétaires, en sorte que s'il ne peut y avoir servitude lorsqu'il y a un seul propriétaire (elle serait d'ailleurs inutile), par contre la servitude peut exister lorsque les deux fonds	n'appartiennent pas << entièrement et exclusivement >> au même propriétaire. Or, en Copropriété il y a	nécessairement pluralité de propriétaires puisque si chaque copropriétaire est propriétaire indivis des
			parties communes, il est par contre seul propriétaire de ses parties privatives.
		
		\subsubsection{L’arrêt du 30 juin 2004 de la 3ème chambre civile de la cour de cassation.}
		
			Cette décision\footnote{Juris-Data \no 2004-024376 ; H. Périnet Marquet, Droit des biens : JCP G 2004, \no 43, I, 171, spéc. p. 1905 ; Constr.-urb. 2004,	comm. 161, note D. Sizaire ; Loyers et copr. 2004, comm. 196, note G. Vigneron ; Defrénois 2005, \no 10, p. 861, note G. Daublon et	B. Gelot} constitue un véritable revirement de jurisprudence puisqu’elle admet la constitution d’une servitude de passage entre deux lots :
			\begin{quote}
				Mais attendu que le titulaire d'un lot de copropriété disposant d'une propriété exclusive sur
				la partie privative de son lot et d'une propriété indivise sur la quote part de partie commune
				attachée à ce lot, la division d'un immeuble en lots de copropriété n'est pas incompatible
				avec l'établissement de servitudes entre les parties privatives de deux lots, ces héritages
				appartenant à des propriétaires distincts.
			\end{quote}
			
			On notera la précision des termes de la cour de cassation : celle-ci ne valide la servitude qu’à la condition
			qu’elle soit établie entre deux lots privatifs. Effectivement les parties communes ne constituent pas des
			« héritages » distincts. La cour de cassation a confirmé son revirement ultérieurement, à propos d’une
			servitude de passage de canalisations\footnote{Cass. 3e civ., 13 sept. 2005, pourvoi \no 04-15.742}.
			
			Des copropriétaires qui ne peuvent accéder à leurs lots que par une cour privative attachée au lot
			appartenant à un autre copropriétaire sont enclavés et peuvent donc réclamer un passage suffisant pour
			assurer la desserte de leur fonds alors que le règlement de copropriété est muet . Il y a une servitude
			lorsque celle-ci est nécessaire à l’exploitation normale du lot\footnote{Cass. Civ. 3e 19 janvier 2010 pourvoi: 09-12522}.
			\begin{quote}
				Attendu qu'ayant constaté que, pour accéder au lot privatif \no 17 leur appartenant, les époux X...
				n'avaient d'autre possibilité que de passer par la cour \no 2 qui était une partie privative attachée
				au lot \no 12 appartenant à M. Y..., la cour d'appel, qui a souverainement relevé que le lot \no 17
				était enclavé, en a déduit exactement que les époux X... étaient fondés à réclamer sur la cour n'appartenant à leur voisin M. Y... un passage suffisant pour assurer la desserte de leur fonds
			\end{quote}


	\subsection{Le nouvel article 6-1 a de la loi du 10 juillet 1965}
	
		L’Ordonnance du 30 octobre 2019 a créé un article 6-1-A (de la loi \no 65-557 du 10 juillet 1965), aux termes
		duquel :
		\begin{quote}
			« Aucune servitude ne peut être établie sur une partie commune au profit d’un lot ».
		\end{quote}
		Ce faisant, le gouvernement a voulu inscrire dans le marbre de la loi une jurisprudence parfaitement
		établie.
		
		Ce principe appliqué en matière de copropriété a été affirmé par la Cour de Cassation à partir de 1984\footnote{Civ 3ème 10 janvier 1984 D. 85, 335 puis Civ 3\degre{} 11 jan 1989 RDI 89 p 243, Civ 3\degre{} 6 mars 1991} :
		\begin{quote}
			(Il y a) incompatibilité entre la division d'un immeuble en lots de copropriété et la création d'une servitude
			sur une partie commune au profit d'un lot\footnote{Certes, l’article L 615-10 du CCH qui prévoit – à titre expérimental et pour dix ans – l’expropriation des parties communes au profit d’un opérateur désigné par la commune ou l’EPCI, édicte que : « Au  moment de l'établissement du contrat de concession ou de la prise de possession par l'opérateur, l'état		descriptif de division de l'immeuble est mis à jour ou établi s'il n'existe pas. Aux biens privatifs mentionnés dans l'état de division est attachée une servitude des biens d'intérêt collectif. Les propriétaires de ces biens privatifs sont tenus de respecter un règlement d'usage établi par l'opérateur ». Mais cette expropriation des parties communes fait disparaître la copropriété !}.
		\end{quote}
		
		Il est toutefois regrettable que le gouvernement ne soit pas allé jusqu’à reproduire intégralement la
		jurisprudence relative aux servitudes à l’intérieur d’un immeuble en copropriété, et n’ait pas précisé que
		si aucune servitude ne peut être établie sur une partie commune au profit d’un lot, par contre il peut être
		créé des servitudes entre les parties privatives de deux lots, conformément à la jurisprudence issue de
		l’arrêt du 30 juin 2004 de la 3ème chambre civile de la cour de cassation\footnote{Juris-Data \no 2004-024376 ; H. Périnet Marquet, Droit des biens : JCP G 2004, \no 43, I, 171, spéc. p. 1905 ; Constr.-urb. 2004, comm. 161, note D. Sizaire ; Loyers et copr. 2004, comm. 196, note G. Vigneron ; Defrénois 2005, \no 10, p. 861, note G. Daublon et B. Gelot,} :
		\begin{quote}
			« Mais attendu que le titulaire d'un lot de copropriété disposant d'une propriété exclusive sur la partie
			privative de son lot et d'une propriété indivise sur la quote-part de partie commune attachée à ce lot, la
			division d'un immeuble en lots de copropriété n'est pas incompatible avec l'établissement de servitudes
			entre les parties privatives de deux lots, ces héritages appartenant à des propriétaires distincts ».
		\end{quote}
		
		Cette rédaction ne remettrait d’ailleurs pas en cause la règle selon laquelle il ne peut y avoir de servitude
		de vue au sein d’une copropriété horizontale puisque ces servitudes ne sont pas établies entre les parties
		privatives de deux lots mais sont établies sur le fonds voisin (article 673 Code civil) comme le rappelait la
		Cour de cassation dans un arrêt du 19 juillet 1995\footnote{civile 3e chambre, 19 juillet 1995–Administrer novembre 1996, page 57, commentaire \nom{CAPOULADE}}. Or il n’y a pas de fonds voisin puisque le terrain est
		bien celui d’une même copropriété.

	
\section{Le jeu de la prescription acquisitive}

	Une fraction d’immeuble peut-elle faire l’objet d’une prescription acquisitive, et être ainsi « requalifiée »
	de partie commune à privative (ou inversement) ?
	
	La question concerne en réalité l’affectation privative de parties communes ou à l’inverse l’affectation
	commune d’une partie privative.
	
	La Cour de cassation donne une réponse positive dans les deux cas, tout en se montrant très exigeante sur
	la preuve de la possession trentenaire :
	\begin{itemize}
		\item - usucapion trentenaire d’une partie commune par un	copropriétaire\footnote{Civ. 3 e , 26 mai 1993, \no 91-11.185, RDI 1993. 411, obs. P. Capoulade et C. Giverdon} ou prescription abrégée par juste titre\footnote{Civ. 3e , 30 avr. 2003, \no 01-15.078, D. 2003. 2047, obs. B. Mallet-Bricout} ;
		\item  usucapion trentenaire d’une partie privative par le syndicat des	copropriétaires\footnote{Civ. 3e , 8 oct. 2015, FS-P+B, \no 14-16.071; Dalloz Actualité, note Le Rudelier}
	\end{itemize}.
	
	Concrètement, comment faire reconnaître un droit de propriété privative sur une partie commune par
	usucapion ?
	
	Pour pouvoir publier son titre, il faut justifier au Service de publicité Foncière d’un « effet relatif » sur
	l’origine du lot. Aussi en pratique bien souvent le « bénéficiaire » demande à l’assemblée générale
	d’approuver la création d’un lot nouveau à partir des parties communes correspondant à la fraction
	accaparée depuis plus de 30 ans et la cession de ce nouveau lot à titre gratuit ou pour 1 \euro symbolique.
	
	Cette procédure présente deux inconvénients :
	\begin{itemize}
		\item d’une part elle ne correspond pas à la réalité puisqu’il n’y a pas de cession ;
		\item d’autre part elle renvoie aux dispositions de l’article 26 a) de la loi et à la double majorité.
	\end{itemize}
	
	Pour sa part Jacques Lafond au Jurisclasseur Copropriété F 91-40 mutations concernant les parties
	communes propose une solution différente.
	\begin{enumerate}
		\item  Le « bénéficiaire » fait établir un acte de notoriété par notaire. Certes un tel acte ne vaut pas preuve
		de la propriété, mais il a pour mérite, outre la participation de témoins, de viser et d’annexer
		éventuellement tous les éléments de preuve de la prescription acquisitive.
		\item Le « bénéficiaire » fait établir un modificatif à l’état descriptif de division avec création d’un nouveau
		lot privatif.
		\item Le « bénéficiaire » demande l’inscription à l’ordre du jour de la prochaine assemblée générale de
		résolutions aux fins :
		\begin{itemize}
			\item de constater le jeu de la prescription au profit du copropriétaire concerné ;
			\item  d'approuver le projet de modification du règlement de copropriété et de l'état descriptif de division.
		\end{itemize}
	\end{enumerate}
	
	S’agissant de la constatation d’un droit et non de la cession d’une partie commune, faute de majorité
	spécifique dans la loi \no 65-557 du 10 juillet 1965 la question sera posée à la majorité de l’article 24. En cas
	de refus de l’assemblée générale le copropriétaire pourra demander au juge de reconnaître son droit et
	pour modifier l’état descriptif (en application de l’article 3 du Décret de 67).
	
	Il faut noter que la prescription acquisitive peut jouer dans le sens inverse. En effet, la Cour de cassation a
	récemment indiqué (Cour de cassation 3e chambre civile 8 Octobre 2015) « qu'aucune disposition ne
	s'oppose à ce qu'un syndicat de copropriétaires acquière par prescription la propriété d'un lot », alors
	même que le pourvoi avait soutenu qu’il ne pouvait être de l’objet du syndicat des copropriétaires de
	porter atteinte aux droits privatives d’un copropriétaire. Pour autant, il risque d’être difficile de démontrer
	la possession non équivoque du syndicat des copropriétaires : quel sera l’acte de possession matérielle
	indiquant que le syndicat des copropriétaires entend utiliser le lot privatif de telle façon qu’il doit être
	considéré comme à l’utilité de tous ?