\chapter[Naissance et disparition du Syndicat]{Naissance et disparition du syndicat des copropriétaires}

\section[Naissance et immatriculation]{Naissance et immatriculation du syndicat des copropriétaires}

	\subsection{Naissance de plein droit lorsque sont réunies les conditions de l’article 1\ier{} et immatriculation}

		\subsubsection{La constitution du syndicat des copropriétaires de plein droit}
		
			Le syndicat des copropriétaires est une des rares personnes morales dépourvues d’acte de naissance ou de constitution. En effet, le Syndicat des Copropriétaires existe dès lors que les conditions prévues par l’article 1\ier{} sont réunies, puisque le statut de la copropriété est d’application impérative.
			
			De ce fait, le législateur n’a pas voulu conditionner la naissance du Syndicat à une quelconque formalité --- il n’existe pas de déclaration préalable à la mise en copropriété contrairement à ce qui s’impose dans certains pays --- de peur de permettre à celui qui devrait accomplir cette formalité --- le promoteur notamment, en cas de construction de l’immeuble, de différer le plus longtemps possible cette formalité afin d’en éviter les contraintes.
			
			Dès qu’il nait à la vie civile, le syndicat des copropriétaires a la personnalité civile, ceci sans aucune forme particulière.
			
			Il suffit donc que l’immeuble soit divisé par lots appartenant à plusieurs personnes pour que le statut s’applique, ceci quand bien même il n’a pas été établi de Règlement de copropriété, d’état descriptif de division et quand bien même aucun syndic n’a été élu :
			
			Cette naissance de plein droit a bien évidemment des effets importants : par exemple la nécessité de s’adresser au syndicat des copropriétaires pour avoir réparation des parties communes ou remboursement des sommes avancées pour le compte du syndicat.\footnote{
			Civ. 3\degre{} Ch. 11 janvier 2012, \no 10-24413, au Bulletin ; D 2012, 219 note Y. Rouquet. En l’espèce l’un des deux copropriétaires avait réalisé des travaux conservatoires sur le sol commun et assigné l’autre copropriétaire pour avoir remboursement de sa quote-part : irrecevabilité de la demande qui aurait dû être engagée contre le syndicat des copropriétaires}
		
		\subsubsection{L’immatriculation obligatoire de certaines copropriétés}
		
			Le fait que les syndicats existent de plein droit dès que se trouvent remplies les conditions posées à l’article 1\ier{} de la loi a permis de constater une méconnaissance des copropriétés par les pouvoirs publics. Certes leur nombre est à peu près connu par l’enquête réalisée tous les cinq ans par l’INSEE et le Fichier des Logements par Commune (FILCOM) recoupé avec les données fiscales donne une idée approximative de ce nombre.
			
			Cette approximation n’avait pas grande importance à l’époque où les copropriétés relevaient totalement du droit privé, si ce n’est parfois la difficulté d’identifier le représentant du syndicat des copropriétaires.
			
			Mais depuis 1994 les autorités publiques sont amenées à participer au redressement des copropriétés en difficulté. Il est donc nécessaire de connaître beaucoup plus précisément non seulement le nombre de copropriétés mais également l’état de ces copropriétés. Suivant la proposition du rapport BRAYE, la loi ALUR a adopté l’obligation d’immatriculation des copropriétés sur un Registre spécifique. Ce sont les nouveaux articles L 711-1 à L 711-7 du \CCH, complété par le décret \no 2016-1167 du 26 août 2016 relatif au registre national d'immatriculation des syndicats de copropriétaires.
			
			\paragraph{Qui tient le Registre des Syndicats de Copropriété ?}
			
				Le registre des syndicats de copropriétaires est tenu par un établissement public de l'État (le Teneur du Registre). Par arrêté en date du 26 octobre 2016, l’ANAH a été désignée en cette qualité à compter du 1\ier{} novembre 2016.
			
			\paragraph{Le Registre des Copropriétés est un Registre dématérialisé}
			
				Le dépôt du dossier et les modifications qui y sont apportées sont dématérialisées ; la déclaration est donc faite par le $\dots$ « Télédéclarant ».
				
			\paragraph{Quels sont les syndicats concernés ?}
			
				Uniquement les syndicats dont les immeubles sont à usage total ou partiel d’habitation (au moins un lot à usage d’habitation).
			
			\paragraph{A qui incombe l’obligation d’immatriculation ?}
				
				La réponse est sans équivoque : Pour les immeubles à construire ou mis en copropriété, cette obligation incombe au notaire qui reçoit l’EDD et le RC\footnote{
					Immatriculation des syndicats de copropriétaires : le rôle du notaire ; Jacques Lafond, JCPN \no 38 – 23 Septembre 2016 ts
				} ; pour les immeubles déjà existants, il incombe au syndic d’effectuer cette immatriculation et de transmettre les actualisations nécessaires. A défaut il s’expose à une amende (pouvant aller jusqu’à \montant{20} par lot) assortie d’une astreinte recouvrées par le Teneur du Registre (art. L 711-6 CCH). Si un acte est reçu par un notaire sur un immeuble non immatriculé, le notaire
				procèdera à l’immatriculation et le syndic paiera l’amende ! Bien évidemment le syndic ne pourra pas se faire rembourser cette amende par le syndicat des copropriétaires. De plus le syndicat non immatriculé ou dont les données ne sont pas à jour ne pourra percevoir de subventions de l’Etat.
				
			\paragraph{Le contenu de l’immatriculation.}
			
				Les informations à porter au Registre sont fixées par Décret en Conseil d’Etat ; il s’agit du Décret \no 2016-1167 du 26 août 2016.
				
				Elles sont de deux catégories (art. L 711-1 CCH) :
				\begin{itemize}
					\item  d’une part les informations permettant d'identifier le syndicat, de préciser son mode de gestion et de connaître les caractéristiques financières et techniques de la copropriété et de son bâti, notamment le nom, l'adresse et la date de création du syndicat ainsi que, le cas échéant, le nom du syndic et le nombre et la nature des lots
					
					\item d’autre part les informations financières, les informations sur une mise sous administration judiciaire, la mise en oeuvre d’une procédure de carence ou de sauvegarde.
							
					Mais les informations « financières » seront limitées si l’immeuble comporte moins de dix lots à usage de logements, de bureaux ou de commerces, dont le budget prévisionnel moyen sur une période de trois ans consécutifs est inférieur à \montant{15 000}.
				\end{itemize}
			
			\paragraph{Les modalités de télédéclaration}
		
				La télédéclaration se fait en plusieurs temps.
				
				\subparagraph{La création d’un compte de télédéclarant.}
				Soit le télédéclarant est un « professionnel » : notaire, syndic, mandataire ad hoc, administrateur judiciaire ; auquel cas il aura un compte général auquel il « rattachera » la copropriété concernée.
				Soit le déclarant est un syndic bénévole, auquel cas il remplira directement le formulaire de sa copropriété.
				
				\subparagraph{La déclaration de la copropriété concernée.}
				Le télédéclarant fournit au Teneur de Registre tous les documents et renseignements prévus par l’arrêté du 26 octobre 2016, Annexe 1 à 7 (certaines annexes concernant les renseignements à fournir chaque année et celles à fournir en cas de changement de syndic, tant par le syndic sortant que par le nouveau syndic).
			
				\subparagraph{L’immatriculation du syndicat des copropriétaires}
				Une fois complétés les renseignements demandés, l’ANAH fournit au Télédéclarant un \no d’im\-ma\-triculation qui sera unique pour le syndicat des copropriétaires tout au long de son existence.
				
				\subparagraph{Conservation des données pendant 5 ans}
								L’article R 711-15 prévoit que les données fournies sont conservées pendant 5 ans
			
			\paragraph{La mise à jour des informations}
			
				Doivent être portées au Registre :
				\begin{itemize}
					\item toute modification dans les données du Registre ;
					\item à l’issue de chaque exercice comptable les données financières actualisées.
				\end{itemize}
			
				C’est ainsi qu’en cas de changement de syndic, le syndic sortant doit informer le Teneur de Registre de la cessation de ses fonctions et son successeur doit à son tour informer le Teneur de Registre de ce qu’il est le nouveau représentant légal du syndicat des copropriétaires.
				
				Précisons que dores et déjà il est possible d’accéder au site du Teneur de Registre et de télécharger les formulaires à compléter sur le site \url{http://www.registre-coproprietes.gouv.fr/} .
			
			\paragraph{Qui peut accéder aux informations portées au Registre ?}
			
				\begin{enumerate}
					\item \textbf{Le notaire}, qui doit porter le numéro d'immatriculation de la copropriété dans les actes de vente.
					\item \textbf{Les copropriétaires} ont un droit d'accès aux données relatives au syndicat dont ils font partie et peuvent solliciter le syndic aux fins de rectification des données erronées.
					\item \textbf{L'État} et ses services ainsi que ses opérateurs, les EPCI compétents en matière d'habitat, les départements et les régions obtiennent du teneur du registre communication des informations du répertoire relatives à chaque copropriété située sur leur territoire.
					\item \textbf{Des tiers}, selon des conditions précisées par décret en Conseil d'État pris après avis de la Commission nationale de l'informatique et des libertés (CNIL).
				\end{enumerate}
			
			\paragraph{Date d’entrée en vigueur de ces nouvelles dispositions.}
				\begin{enumerate}[label=\alph*)]
					\item Tous les immeubles à usage total ou partiel d’habitation dès à présent soumis au statut de la copropriété doivent être immatriculés avant le 31 décembre 2018.
					\item Pour les immeubles neufs ou mis en copropriété : l’immatriculation devra se faire automatiquement à compter du 1er décembre 2016.
				\end{enumerate}
		
	\subsection{Naissance a la date de la division de l’immeuble par lots}
	
		Ne sont soumis de plein droit au statut de la copropriété que tous « les immeubles bâtis dont la propriété est répartie entre plusieurs personnes $\dots$ ».
		
		Par conséquent, le Syndicat des Copropriétaires prend naissance à la plus tardive des deux dates suivantes :
		\begin{itemize}
			\item division de l’immeuble par lots comprenant chacun une quote part de parties communes et une partie privative, division qui se traduit en principe par la rédaction d’un Règlement de copropriété et d’un État descriptif de division ;
			\item existence d’un immeuble bâti, pour les immeubles vendus en VEFA.
		\end{itemize}
		
		\subsubsection{Vente par appartements d'un immeuble ancien déjà construit}
		
			L’article 1\ier{}-1 (rédaction de la loi ELAN) consacre la jurisprudence existante et édicte :
			\begin{quote}
				« En cas de mise en copropriété d’un immeuble bâti existant, l’ensemble du statut s’applique à compter du premier transfert de propriété d’un lot ».
			\end{quote}
			
			C'est le cas de l'immeuble de rapport vendu par appartements par son propriétaire.
			
			Celui-ci va procéder ou plutôt faire procéder par un ou plusieurs hommes de l'art (géomètre, expert, notaire, avocat) à l'établissement d'un Règlement de Copropriété, et d’un état descriptif de division, puis vendre un par un les lots issus de cette division.
	
			Le syndicat des copropriétaires va naître dès la première vente entre l'ancien propriétaire qui sera propriétaires de tous les lots à l'exclusion d'un seul et l'acquéreur de ce lot : en effet dès cette première vente la propriété de l'immeuble est répartie entre plusieurs personnes.
			En pratique on relève très souvent des modifications au Règlement de Copropriété et parfois des contradictions qui tiennent au fait que l'acquéreur a étudié le Règlement de Copropriété qui lui est soumis par le vendeur avant la signature de l'acte authentique de vente et tente d'obtenir divers avantages que le vendeur, désireux d'en finir pour réaliser la vente, accepte de prendre en compte.
		
		\subsubsection{Dissolution de la Société d'Attribution ou retrait d'associé (Sociétés du Titre \II{} de la loi du 16 juillet 1971)}
		
			Ces sociétés constituées pour la construction ou l'acquisition d'un immeuble en vue de sa division par fractions destinées à être attribuées aux associés en propriété ont donc pour vocation d'être dissoutes pour laisser place au régime de la Copropriété.
			
			L'immeuble construit, les associés réunis en Assemblée Générale Extraordinaire approuvent les comptes de construction et décident de la liquidation de la Société. Ils désignent un liquidateur qui aura notamment pour mission d'établir un projet de partage que signeront (ou pourront contester) individuellement les associés. Les Associés se réuniront pour approuver les comptes de liquidation. Il ne restera plus alors qu'à signer l'Acte de Partage. Par cette signature de l'acte de partage chaque associé se voit attribuer la propriété du lot dont il n'avait jusque lors que la jouissance.
			
			Toutefois les associés peuvent préférer rester en société nonobstant l'achèvement de l'immeuble et l'approbation des comptes de construction. Dans cette hypothèse cependant tout associé peut demander à se retirer de la Société et à se voir attribuer la propriété de son appartement : il exerce la faculté de retrait anticipé. Le retrait est simplement constaté par acte authentique signé par le retrayant et le représentant légal de la société.
			
			L'associé ne peut demander son retrait que s'il remplit trois conditions :
			\begin{itemize}
				\item être à jour de tous les appels de fonds nécessaires à la réalisation de l'objet social (la jurisprudence ne prend en considération que les appels de fonds nécessaires à la construction de l’immeuble et non les appels de fonds relatifs à la gestion de l’immeuble une fois construit).
				\item justifier que l'achèvement de l'immeuble social et sa conformité avec l’État descriptif de division ont été constatés par l'Assemblée Générale des Associés.
				\item justifier que l'Assemblée Générale des Associés a décidé des comptes définitifs de l'opération.
			\end{itemize}
			
			Toutefois l'associé peut demander au Tribunal de Grande Instance de faire ces constatations et mettre en œuvre la procédure de retrait malgré le refus ou la passivité des organes de direction de la Société.
			
			Dans les deux cas évoqués (dissolution ou retrait anticipé), la loi sur la copropriété s'appliquera à compter de la signature de l'acte authentique :
			\begin{itemize}
				\item en cas de liquidation il n'existe plus que des copropriétaires.
				\item en cas de retrait la copropriété fait coexister la société et le retrayant ; c'est alors une Copropriété à deux personnes.
			\end{itemize}
		
			Relevons toutefois un intéressant arrêt de la Cour d'Appel de Versailles du 27 juin 1991\footnote{Rev. Dr. Imm 1991, 3\degre{} T, p. 337} qui a affirmé que la Copropriété naissait le jour de l'Assemblée ayant approuvé les comptes de liquidation. C'est en effet cette Assemblée qui procède à l'attribution des lots conformément aux statuts et à l’État descriptif de division. Mais cette décision demeure isolée. L’opinion généralement admise étant que la copropriété naît au moment de l’attribution à chaque associé de sa part divise\footnote{Cf. Givors, Giverdon, Capoulade, La Copropriété Dalloz Ed 2018 \no 111.31}.
		
		\subsubsection{Partage en nature de l'immeuble ou licitation des lots}
		
			Lorsqu'un immeuble échoit à plusieurs héritiers, ceux-ci se trouvent placés sous le régime de l'indivision régie par les articles 812 et suivants du Code Civil. Tant que l'immeuble demeure en indivision, il appartient à une même famille et n'est donc pas soumis au statut de la Copropriété : il n’y a pas attribution des parts aux lots, et chacun est propriétaire d’une quote part virtuelle de l’immeuble.
			
			Cependant, << Nul n'est tenu de rester dans l'indivision >> en sorte que tout indivisaire est en droit de demander à sortir de l'indivision et à se faire attribuer, après partage de l'immeuble, la quote-part d'immeuble à laquelle il a droit.
			
			Préalablement au partage de l'immeuble les héritiers feront établir par le notaire un Règlement de Copropriété et un état descriptif de division.
			
			Si les héritiers ou co-indivisaires sont d'accord sur l'attribution des lots l'immeuble sera partagé en nature. En cas de désaccord des héritiers, les lots seront vendus sur licitation, les héritiers pouvant bien entendu se porter acquéreurs de lots.\footnote{Toulouse 18 Jan. 88 - juris Data \no 043735}
			
			Par contre, dès la première attribution (signature de l'acte de partage ou licitation) l'immeuble entre dans le champ d'application de la loi du 10 juillet 1965.

			Relevons que le nouvel article 1\ier{}-1 de la loi n’envisage comme point de départ du statut de la copropriété dans un immeuble existant que l’hypothèse du premier transfert de propriété du lot, alors qu’il eut été plus exhaustif d’ajouter « ou de la première attribution d’un lot ».
		
		\subsubsection{Vente d’un lot de S.C.I. d’attribution sur poursuites judiciaires.}
		
			Hypothèse plus originale où la Copropriété est imposée par voie de justice : la vente sur poursuites judiciaires d'une partie du patrimoine d'une Société d'Attribution. On peut envisager effectivement le cas où un créancier d'une société d'attribution ayant inscrit une hypothèque conventionnelle ou judiciaire sur l'immeuble propriété de la Société poursuit la vente judiciaire d'une partie de l'immeuble seulement (la valeur de cette partie d'immeuble étant suffisante pour l'indemniser de sa créance). Dans cette hypothèse l'adjudicataire sera copropriétaire de l'immeuble avec la Société saisie.
		
		\subsubsection{Surélévation d'un immeuble appartenant à un propriétaire}
		
			Dans cette hypothèse, le propriétaire d'un immeuble construit va céder à un tiers le droit de surélever son immeuble en y construisant un ou plusieurs étages supplémentaires. Une convention de copropriété devra être établie avec rédaction d'un Règlement de Copropriété et d'un État descriptif de division.
			
			Le statut de la copropriété s’appliquera dès la cession du lot « droit de surélever ».
		
		\subsubsection{Location vente ou location attribution}
		
			Si la location-vente est fréquente appliquée à un lot d'un immeuble déjà soumis au droit de la Copropriété, il arrive également que l'immeuble soit réalisé par une société qui soumettra l'ensemble au régime de la location-vente. Hypothèse qui se rencontre essentiellement dans le domaine de la Coopérative H.L.M.
			
			La société d'H.L.M. demeurera propriétaire jusqu'à ce que soit réalisée la promesse de vente. Mais dès l'origine un état descriptif de division et un Règlement de Copropriété auront été établis. Rappelons que la location accession est régie par les dispositions de la loi du 17 juillet 1984 qui emporte différentes conséquences sur le statut de la Copropriété, par exemple participation du locataire acquéreur aux Assemblées Générales de copropriété et même au Conseil Syndical.
			
	\subsection{Naissance a la date de l’achèvement de l’immeuble}
	
		\subsubsection{L’article 1\ier{}-1 de la loi du 10 juillet 1965}
	
			Pour la réalisation d’un immeuble à construire, les promoteurs peuvent constituer entre eux une société, de vente d'immeuble (art. L 210-1 et s. CCH pour les sociétés civile) , dont l'objet est l'acquisition et la
			vente d'appartements. Les fonds nécessaires à la construction étant réunis, partie en fonds propres des promoteurs et partie grâce aux prêts promoteurs consentis par les banques.
			
			Le candidat acquéreur n'est pas associé à la Société de construction mais achète directement son local qui lui sera livré soit achevé soit en l'état futur d'achèvement,
			
			Si l'immeuble est vendu << achevé >>, la Copropriété prend naissance dès la première vente.
			
			Si l'immeuble est vendu en l'état futur d'achèvement la Copropriété prend naissance dès achèvement de l'immeuble.
			C’est ce qui est expressément mentionné dans l’article 1\ier{}–1 nouveau de la loi du 10 juillet 1965 qui est ainsi rédigé :
			« Pour les immeubles à construire, le fonctionnement de la copropriété découlant de la personnalité morale du syndicat de copropriétaires prend effet lors de la livraison du premier lot ».
			Ce faisant le législateur de 2018 n’a fait que consacrer légalement une jurisprudence relativement constante\footnote{Civ 3\degre{} 20 déc 1976, JCP 1978, II, 18800, Obs. Guillot.}, quand bien même certaines décisions de cours d’appel avaient cru devoir affirmer que le statut aurait été applicable dès la première vente en l’état futur\footnote{Paris 23\degre{} Ch. 22 sep 1995, Loyers et Copropriété 1996 \no 80 ; Versailles 22 mai 1984, Rev. Dr. imm. 1984 p. 351 ; Aix 16 avril 1992 Loyers et Copropriété 1993 \no 275.}.
	
		\subsubsection{Construction d'un bâtiment unique.}
		
			S'agissant de vente en l'état futur d'achèvement le décret d'application de la loi du 3 janvier donne une définition de cet achèvement :
			\begin{quote}
			Art 261-1 C.C.H.\\
			<< L'immeuble ($\dots$) est réputé achevé ($\dots$) lorsque sont exécutés les ouvrages et sont installés les éléments d'équipement qui sont indispensables à l'utilisation, conformément à sa destination, de l'immeuble faisant l'objet du contrat. Pour l'appréciation de cet achèvement les défauts de conformité avec les prévisions du contrat ne sont pas pris en considération lorsqu'ils n'ont pas un caractère substantiel, ni les malfaçons qui ne rendent pas les ouvrages ou éléments ci-dessus précisés impropres à leur utilisation >>.
			\end{quote}

			Cet achèvement n’est donc pas subordonné à la livraison de tous les lots, ou encore à l’obtention du certificat de conformité.
		
		\subsubsection{L’immeuble vendu en l’état futur d’inachèvement}
		
			La loi ELAN a modifié le \CCH sur la vente en l’état futur d’achèvement, en ajoutant à l’article L 261-15 qui traite du contrat préliminaire à la VEFA un \II{} ainsi rédigé :
			\begin{quote}
				« \II{}. – Le contrat préliminaire peut prévoir qu’en cas de conclusion de la vente, l’acquéreur se réserve l’exécution de travaux de finition ou d’installation d’équipements qu’il se procure par lui-même. Le contrat comporte alors une clause en caractères très apparents stipulant que l’acquéreur accepte la charge, le coût et les responsabilités qui résultent de ces travaux, qu’il réalise après la livraison de l’immeuble. ($\dots$)
				
				Un décret en Conseil d’État précise les conditions d’application du présent \II{}, notamment la nature des travaux dont l’acquéreur peut se réserver l’exécution »
			\end{quote}
			
			Bien évidemment ce texte n’entrera en application qu’après publication du Décret.
			
			Pour autant il est ici question de travaux d’achèvement et non de parachèvement en sorte que les travaux que l’acquéreur se réserve peuvent porter sur des ouvrages ou éléments d’équipement indispensables à l’utilisation du lot (équipements de la salle de bains par exemple).
			
			Dans cette hypothèse faudra-t’il attendre que l’acquéreur ait réalisé ces travaux ou éléments d’équipement indispensables pour faire application du statut ?
			
			Pour des raisons pratiques et de texte, la réponse nous paraît devoir être négative et la date à retenir pour l’application du statut devrait être la livraison du premier lot $\dots$ même non achevé.
		
		\subsubsection{Pluralité de bâtiments}
		
			Il est très délicat en revanche de déterminer l'achèvement d'un ensemble de bâtiments livrés successivement (Problème des tranches successives de construction) et il est en conséquence difficile de déterminer la date à laquelle le statut de la Copropriété va recevoir application.
			
			Or, la loi s'applique à << tout immeuble bâti ou groupe d'immeubles bâtis >>.
			
			Ce texte est susceptible de deux interprétations :
			\begin{itemize}
				\item soit l'on considère que la loi n'est applicable qu'à l'achèvement du groupe d'immeubles --- lorsque le groupe est construit ;
				\item soit l'on considère que la loi s'applique dès achèvement du premier bâtiment --- dès qu'il y a un immeuble bâti.
			\end{itemize}
		
			Dans un arrêt du 4 décembre 1979\footnote{3\degre{} Ch Civ. B \no 218 p. 171} la Cour de Cassation a admis implicitement que la loi devait recevoir application dès achèvement du premier bâtiment.
			Alors que les autres bâtiments n'étaient pas construits, ce premier bâtiment achevé s'était constitué en syndicat secondaire. Une telle constitution d'un syndicat secondaire n'était possible que dans l'hypothèse où la loi de 1965 recevait application. C'est ce qu'a décidé la Cour de Cassation.
			
			Cette solution n'est plus guère discutée aujourd'hui : en fait elle constitue une réponse pratique à un problème difficile. La solution contraire aurait créé un << vide juridique >> : quel aurait été en effet le régime applicable au bâtiment achevé qui, par suite des retards souvent apportés à la réalisation des travaux de construction par << tranches successives >>, serait resté seul achevé pendant plusieurs mois, voire plusieurs années ?
			
			Au demeurant la rédaction du nouvel article 1\ier{}-1 qui fait application du statut dès la livraison du premier lot ne fait pas de distinction entre la VEFA passée dans le cadre d’un bâtiment unique ou dans le cas d’une pluralité de bâtiments.
		
			La jurisprudence précitée devrait continuer de s’appliquer : le statut s’applique dès la livraison du premier lot dans le premier bâtiment ; pour les bâtiments non construits, le vendeur sera considéré comme propriétaire des lots correspondant à ces bâtiments, mais ne participera qu’aux charges générales.
		
		\subsubsection{Le problème du lot transitoire}
			
			\paragraph{Historique}
			
				Étant admis que la loi reçoit application dès achèvement du premier bâtiment, doit-elle être appliquée au seul bâtiment achevé ou à l'ensemble des lots, que ceux-ci soient construits ou non ?
				
				Ce d'autant que la technique le plus souvent appliquée dans la Construction par << Tranches Successives >> est celle dite du lot transitoire\footnote{Autrement appelé selon Mr \nom{CAPOULADE} dans son étude pour le Rapport de la Cour de Cassation de 1994 : << lot d'attente >>, << lot par anticipation >> ou << lot parthénogénique >>}.
				
				\begin{quote}
					En pratique le Promoteur de l'ensemble constitue des lots intermédiaires, généralement un lot par bâtiment à construire. Par exemple si cinq bâtiments sont prévus, l’État descriptif de division comprendra cinq lots intermédiaires 100, 200, 300, 400 et 500.

					Chacun de ces lots étant défini non par l'addition d'une partie privative et de tantièmes généraux de copropriété, mais par un droit de construire un bâtiment qui comprendra un nombre de tantièmes généraux de la Copropriété, correspondant à l’ensemble des tantièmes du futur bâtiment.
				
					Lors de l'édification du premier bâtiment le lot 100 est << éclaté >> en autant de lots qu'il y aura de locaux principaux et accessoires dans ce bâtiment, de telle sorte que le lot 100 sera annulé et remplacé par exemple par les lots 101 à 152. Avec l'édification du deuxième bâtiment le lot 200 sera remplacé par les lots 201 à 260, etc.
				\end{quote}
				
				Ce lot transitoire est-il véritablement un lot de copropriété alors qu'il ne comprend pas de parties privatives ? La doctrine avait en effet douté de l'applicabilité de la loi à ce type de lot qui ne comporte pas de véritable "partie privative", ou plutôt, dont la partie privative ne présente pas de consistance matérielle.\footnote{Giverdon et Bouyeure, ADMINISTRER déc 81, p. 7.}
				
				La Cour de Cassation a finalement admis l’existence des lots transitoires, notamment par un important arrêt du 15 novembre 1989\footnote{Bulletin \no 213, p. 117; D 90. J. 216, note Giverdon et Capoulade} à propos de la saisie immobilière de lots transitoires
				\begin{quote}
					<< Retenant que chacun des lots saisis, placé par l'auteur de la division sous le régime de la copropriété, comprenait selon l'état descriptif, le droit exclusif d'utiliser une surface déterminée du sol pour y édifier des constructions conformément à un permis de construire délivré..., ainsi qu'une quote-part de la propriété du sol et des parties communes, la cour d'appel a exactement décidé... que le lot privatif du débiteur constituait un immeuble par nature pouvant faire l'objet d'une saisie immobilière >>.
				\end{quote}
				Et surtout par un arrêt SCI SALENGRO du 3 juin 1991\footnote{3\degre{} Ch. Civ. du 3 juin 1991 (JCP 92, \I{}, 3560)} aux termes duquel le lot transitoire ou intermédiaire est un véritable lot de copropriété, en sorte que dès l'achèvement du premier bâtiment le régime de la Copropriété va s'appliquer à l'ensemble immobilier, parties bâties et non bâties et que la SCI promotrice, en sa qualité de propriétaire des lots transitoires participera aux Assemblées Générales de la Copropriété.
				
				On retiendra également un arrêt du 14 novembre 1991 par lequel la cour de cassation a rejeté un pourvoi à l'encontre d'un arrêt par lequel la cour d'appel
				\begin{quote}
					« Retenant à bon droit que les lots dits << transitoires >> ne sont pas assujettis à un régime particulier, en a exactement déduit$\dots$ que la SCI était copropriétaire au sens de la loi du 10 juillet 1965 »
				\end{quote}.
	
			\paragraph{Le nouvel article 1\ier{} alinéas 2 et 3} 

				La loi ELAN a tenu compte de cette jurisprudence pour ajouter un nouvel alinéa à l’article 1er qui est ainsi rédigé :
				\begin{quote}
					« Ce lot peut être un lot transitoire. Il est alors formé d’une partie privative constituée d’un droit de construire précisément défini quant aux constructions qu’il permet de réaliser sur une surface déterminée du sol, et d’une quote-part de parties communes correspondante. (art. 59 bis D)
					
					« La création et la consistance du lot transitoire sont stipulées dans le règlement de copropriété. »
				\end{quote}
			
			\paragraph{La construction future doit être précisément définie}
	
				Ce nouvel article est très important.
				
				La jurisprudence ayant validé le lot transitoire acceptait parfaitement que celui-ci ne soit en aucune façon défini. Par exemple, un arrêt du 3 novembre 2016\footnote{Civ. 3\degre{} Ch. 16 novembe 2016, Pourvoi \no 15-14895 15-15113, Inédit} a considéré qu’était parfaitement valable le lot appartenant à une société , constitué du droit de construire sur une partie du terrain, sans aucune précision quant à la future construction alors que le demandeur au pourvoi avait fait valoir que : « le droit de construire mentionné dans la description d'un lot de copropriété ne peut en constituer la partie privative qu'à la condition d'être concrètement défini dans sa consistance et son implantation » . De façon lapidaire la cour de cassation a simplement répondu : « a cour d'appel a retenu, à bon droit, que le lot de la société constituait un lot privatif composé pour sa partie privative du droit exclusif d'utiliser le sol pour édifier une construction et d'une quote-part de parties communes et en a exactement déduit que la société était titulaire d'un lot transitoire.
				
				Jurisprudence d’autant plus favorable au titulaire du lot que le droit à construire sur le lot transitoire n'est pas soumis aux règles d'autorisation de la copropriété de l’article 25 dès lors qu'en vertu du règlement de copropriété son titulaire bénéficie du droit de construite « tous bâtiments et constructions »\footnote{Cass. Civ. 3e 8 juin 2011 ; \no de pourvoi: 10-20276 Publié au bulletin Revue des Loyers juillet 2011 page 322, note Jean-Marc \nom{Roux}}. De même il a été jugé que le promoteur, qui a édifié sur le lot transitoire un bâtiment à usage de garage et qui n'a fait qu'user d'un droit reconnu par le règlement de copropriété, n'était pas tenu de solliciter pour construire l'autorisation de l'assemblée générale dans la mesure où la nature et l’affectation étaient définies\footnote{Cass. Civ. 3e 4 novembre 2010}.
				
				Cette jurisprudence avait un effet déstabilisateur à l’intérieur de la copropriété du fait de la totale liberté dont bénéficiait le titulaire du lot transitoire.
				Le nouvel article 1er alinéa 2 réagit contre cette jurisprudence en imposant justement que le lot transitoire soit précisément défini :
				\begin{itemize}
					\item quant à sa surface au sol ;
					\item quant à la construction à réaliser ;
					\item quant aux quotes-parts attribuées à ce lot transitoire.
				\end{itemize}
				
				De plus, ce lot transitoire doit non seulement figurer comme tel dans l’état descriptif de division mais il doit également être mentionné et décrit dans le règlement de copropriété.\footnote{Nous verrons ultérieurement que l’état descriptif de division ne se confond pas avec le règlement de copropriété.}
			
			\paragraph{Dispositions transitoires prévues par la loi ELAN}
	
				En principe ces nouvelles dispositions ne devraient s’appliquer qu’aux lots transitoires créés postérieurement à la publication de la loi ELAN, en application du principe de non-rétroactivité des lois.
				
				La loi ELAN en décide autrement en imposant des dispositions transitoires ainsi rédigées :
				\begin{quote}
					– Les syndicats des copropriétaires disposent d’un délai de trois ans à compter de la promulgation de la présente loi pour mettre, le cas échéant, leur règlement de copropriété en conformité avec les dispositions relatives au lot transitoire de l’article 1er de la loi \no 65-557 du 10 juillet 1965 fixant le statut de la copropriété des immeubles bâtis.
					
					À cette fin et si nécessaire, le syndic inscrit à l’ordre du jour de chaque assemblée générale des copropriétaires organisée dans ce délai de trois ans la question de la mise en conformité du règlement de copropriété. La décision de mise en conformité du règlement de copropriété est prise à la majorité des voix exprimées des copropriétaires présents ou représentés.
				\end{quote}
				
				Ce texte posera certainement des difficultés d’application : Sera-t’il seulement applicable lorsque l’état descriptif de division comportera un lot transitoire non défini ou pourra-t’il être recevoir application dans l’hypothèse où le droit de construire figurera seulement dans le règlement de copropriété.
				
				En tout cas dès la première assemblée générale suivant la publication de la loi ELAN les syndics devront vérifier si les états descriptif de division comprennent des lots transitoires non précisément définis ni quant à la surface au sol ni quant à la construction projetée et en ce cas proposer à l’assemblée générale de modifier le règlement de copropriété pour y inscrire la définition du lot transitoire.
				
				Comment se fera cette définition ? Il semble difficile d’imposer au titulaire du lot une description du bâtiment à construire.
				
				A quelle majorité l’assemblée générale se prononcera sur cette modification du règlement de copropriété qui ne concerne ni l’administration de l’immeuble ni l’usage des parties communes , seules hypothèses de modifications du règlement de copropriété à la double majorité de l’article 26 de la loi ? Doit-on considérer que cette adaptation du règlement de copropriété relève de la majorité de l’article 24 f) de la loi ?
				Ce droit de construire sera-t’il caduc si l’assemblée générale n’a pas défini le lot transitoire dans le délai de 3 ans de la publication de la loi ?

			\paragraph{Jurisprudence complémentaire sur le lot transitoire}
			
				La jurisprudence, tirant les conséquences de l’’existence du lot provisoire, condamne les clauses du règlement de copropriété qui ne font participer aux assemblées générales et aux charges que les propriétaires des lots construits\footnote{Civ 3\degre{} 30 juin 1998 - Administrer janvier 1999 p. 62, note \nom{CAPOULADE}.} : le titulaire d’un lot transitoire doit obligatoirement être convoqué aux assemblées générales (à défaut il pourra en demander l’annulation pendant dix ans), et doit participer aux charges communes générales (assurances et frais de gestion notamment) $\dots$
				
				La modification d’un droit de construire attaché à un lot dit << transitoire >> pour l’étendre à des activités annexes et/ou complémentaires relève de la double majorité de l’article 26 de la loi du 10 juillet 1965 et non de l’unanimité\footnote{Cass. Civ. 3e 30 novembre 2010 30 novembre 2010 \no de pourvoi: 09-72386, non publié}.
	
\section{Disparition du syndicat des coproprietaires}

	Nous venons de voir précédemment dans quelles conditions un immeuble qui appartenait à une seule personne se trouve du fait de sa division entre plusieurs propriétaires soumis au statut de la copropriété.
	
	L’hypothèse inverse peut se rencontrer : un immeuble divisé en plusieurs lots appartenant à des copropriétaires différents va devenir la propriété d’une seule personne\footnote{Il ne faut pas confondre cette hypothèse avec la situation qui résulte de l’application des dispositions de l’article 28 de la loi sur la copropriété en cas de retrait d’un ou plusieurs lots de la copropriété d’origine. En effet aux termes de cette loi, la scission entraîne la disparition du syndicat des copropriétaires d’origine. Cette question sera traitée ultérieurement.}.
	
	La disparition du syndicat peut également résulter de son $\dots$ annulation par le juge !
	
	\subsection{La disparition du syndicat pour l’avenir}
	
		\subsubsection{La dissolution de plein droit du syndicat des copropriétaires, et sa survie pour les besoins de la liquidation}
		
			La réunion de tous les lots entre les mains d’un même propriétaire entraîne, de plein droit, la disparition du Syndicat\footnote{Cass. Civ. 3e 28 janvier 2009 pourvoi: 06-19650 Publié au bulletin} du fait que disparaît l’une : « la réunion de tous les lots entre les mains d'un même propriétaire avait entraîné de plein droit la disparition de la copropriété ».

			C’est pourquoi la cour de cassation\footnote{Civ 3ème Ch 4 juillet 2007, pourvoi: 06-11015 Publié au bulletin, Loyers et Copropriété 2007 \no 204} dans un arrêt du 4 juillet 2007 pose le principe selon lequel la réunion de tous les lots entraîne la disparition de la copropriété.
			
			Ce principe étant posé, reste à en découvrir les conséquences.
			
			\vskip
			
			Aux termes de l’article 1844–8 du code civil la société civile se survie pour les besoins de sa liquidation. Aucune disposition spécifique n’a été prévue dans la loi du 10 juillet 1965. Un syndicat des copropriétaires n’est pas une société ! Dès lors peut-on soutenir que le syndicat des copropriétaires se survie pour les besoins de sa liquidation ?
			
			\vskip
			
			Une réponse positive à cette question a été donnée par la cour de cassation\footnote{Civ 3ème Ch 5 déc 2007, pourvoi: 07-11188 07-11204 Publié au bulletin Loyers et Copropriété 2008 \no 43} dans un arrêt du 5 décembre 2007 dont les enseignements sont les suivants :
			\begin{itemize}
				\item la personnalité morale du syndicat subsiste pour les besoins de sa liquidation,
				\item la liquidation du syndicat des copropriétaires ne peut se faire sous le régime du droit de la copropriété - le liquidateur peut être désigné à l’unanimité des anciens propriétaires
				\item le Syndicat peut également être représenté par un mandataire ad hoc dans les procédures en cours, à la demande de tout intéressé.\footnote{Cour de cassation chambre civile 3 27 avril 2017 \no de pourvoi: 16-11278 Non publié au bulletin Jurisdata}
			\end{itemize}
			La demande était fondée sur l’article 1844-8 code civil. La cour de cassation ne vise pas cet arrêt : cela signifie-t’il qu’elle écarte effectivement les règles de liquidation des sociétés civiles ?
			
			\begin{quote}
				(la cour d’appel ayant) retenu à bon droit qu'en cas de réunion de tous les lots entre les mains d'une même personne, aucune disposition de la loi du 10 juillet 1965 n'avait vocation à régir la liquidation de la copropriété et que sa personnalité morale subsistait pour les besoins de sa liquidation, et constaté que l'assemblée générale tenue entre tous les anciens copropriétaires après la vente des lots avait désigné à l'unanimité M. Z$\dots$ aux fonctions de liquidateur amiable, la cour d'appel en a exactement déduit, abstraction faite d'un motif surabondant relatif à la représentation de chaque copropriétaire par le liquidateur, que la fin de non-recevoir soulevée par M. X$\dots$ devait être rejetée
			\end{quote}
			
		\subsubsection{Disparition d’une copropriété horizontale}
	
			Dans une copropriété horizontale les propriétaires de pavillons souhaitent mettre fin à la copropriété.
		
			Si cette demande est consécutive à la cession des VRD à la Commune, les copropriétaires décideront simplement de se partager le terrain d’assiette de la copropriété, en sorte que chaque propriétaire
			possèdera en pleine propriété son pavillon et le terrain d’assiette de la construction (et éventuellement le jardin).
			
			Il ne s’agira pas d’une scission de la copropriété, mais d’un partage pur et simple qui peut s’effectuer à la suite d’une décision unanime de tous les copropriétaires décidant de s’attribuer mutuellement le sol suivie d’un acte notarié de partage auquel comparaîtront tous les copropriétaires. Cette décision sera complétée par la désignation d’un liquidateur du syndicat des copropriétaires.
			
			S’il subsiste des VRD communs, l’assemblée générale pourra décider de les apporter en propriété à une ASL qu’elle constituera concomitamment\footnote{Cf. La revue fiscale du patrimoine \no 7-8, Juillet 2012, form. 7 par Jacques Lafond}. Mais attention lorsque les biens des copropriétaires sont grevés d’hypothèques.
	
		\subsubsection{Le statut de la copropriété ne s’applique pas à la liquidation du syndicat}
		
			Dès lors que la loi du 10 juillet 1965 ne s’applique plus à l’immeuble dont la propriété n’appartient qu’à une seule personne, on voit mal pour quelle raison le statut de la copropriété survivrait à la disparition du syndicat des copropriétaires.
			
			C’est la position qu’adopte la cour de cassation dans l’arrêt précité du 5 décembre 2007 :
			\begin{quote}
				« (la cour d’Appel) ayant retenu à bon droit à qu’en cas de réunion de tous les lots entre les mains d’une même personne, aucune disposition de la loi du 10 juillet 65 n’avait vocation à régir la liquidation de la copropriété $\dots$ »
			\end{quote}
			
			Dès lors en cas de dissolution du syndicat des copropriétaires, il appartient :
			\begin{itemize}
				\item - soit à l’assemblée générale du syndicat des copropriétaires avant la cession du dernier lot à une seule et même personne ;
				\item soit aux anciens copropriétaires agissant à l’unanimité ;
				\item soit à tout intéressé --- et principalement au propriétaire unique --- une fois la dissolution de plein droit réalisée ;		
			\end{itemize}
			de désigner ou de faire désigner par le juge statuant sur requête un liquidateur au syndicat des copropriétaires.
			
			Ce liquidateur pouvant bien évidemment être désigné avant dissolution du syndicat des copropriétaires par l’assemblée générale des copropriétaires sous condition suspensive de la réunion de tous les lots entre les mains d’une seule et même personne. Il peut être également dit par l’assemblée que le liquidateur ne prendra ses fonctions qu’au jour où l’ensemble des lots aura été réuni entre les mains d’une seule personne.

			Ce liquidateur se verra conférer les pouvoirs habituels du liquidateur d’une personne morale.
		
		\subsubsection{Les conséquences de la liquidation du Syndicat des Copropriétaires}
		
			De questions complémentaires doivent être résolues :
			\begin{itemize}
				\item qui doit être convoqué par le liquidateur pour l’approbation de ses comptes de liquidation ?
				\item les droits et actions du syndicat des copropriétaires sont-ils transmis de plein droit à l’acquéreur unique de tous les lots ?
			\end{itemize}
			Ici encore nous ne disposions d’aucune jurisprudence sur la question. Certes, l’article 28 de la loi, en cas de scission d’une copropriété, affirme bien que celle-ci emporte dissolution du syndicat d’origine.
			
			\paragraph{Qui est concerné par les opérations de liquidation ?}
			
			Le liquidateur devra convoquer en assemblée l’ensemble des personnes qui étaient copropriétaires au jour de la dissolution de la copropriété. C’est entre ces personnes qu’il fait les comptes et c’est à elles qu’il demande le complément nécessaire à l’apurement des dettes du syndicat en cours de liquidation.
			
			\paragraph{Les droits et obligations des copropriétaires sont-ils transmis à l’acquéreur unique ?}
			
				\subparagraph{Principe général}
				
				Contrairement à une opinion qui avait été émise par certains auteurs, l’acquéreur unique ne se trouve pas aux droits et obligations du syndicat des copropriétaires.
			
				Certes, il convient de citer un arrêt de la cour de cassation\footnote{Civ 3ème 12 septembre 200, Loyers et Copropriété 2007 \no 228. Bull Civ.} en date du 12 septembre 2007 qui reconnaît à l’acquéreur unique le bénéfice de l’action à l’encontre de l’assureur dommages ouvrage appartenant précédemment au syndicat des copropriétaires.
		
				Mais cet arrêt ne signifie en aucune façon que l’acquéreur se trouve au droit du syndicat des copropriétaires : il reprend purement et simplement le principe relatif à l’assurance dommages ouvrage
				qui veut que le l’action bénéficie aux propriétaires successifs de l’immeuble pendant la durée de la garantie décennale.
				
				Dans un arrêt de cassation\footnote{Civ. 3ème 2 octobre 2013, \no de pourvoi: 12-17098, non publié au Bulletin ; Loyers et Copropriété jan 2014 \no 27} du 2 octobre 2013, la Cour de Cassation, au visa de l’article 1165 du code civil et 14 de la loi de 1965, rappelle : «qu’en l'absence de clause expresse, la société bénéficiaire de l'apport, venue aux droits de la société apporteuse, n'était pas tenue de plein droit des obligations personnelles du syndicat ». La cour de cassation prend sa décision sur le fondement de l’art. 1165 du code civil aux termes duquel les convenbtions n’ont d’effet qu’entre les parties.
				
				\subparagraph{Le nouvel article 28 de la loi}
				
				Il est dérogé au principe que nous venons d’évoquer par le nouvel article 28 de la loi en matière de scission de copropriété. En effet cet article comprend désormais trois nouveaux paragraphes ainsi rédigés :
				\begin{quote}
					« La répartition des créances et des dettes est effectuée selon les principes suivants :
				
					« 1\degre{} Les créances du syndicat initial sur les copropriétaires anciens et actuels et les hypothèques du syndicat initial sur les lots des copropriétaires sont transférées de plein droit aux syndicats issus de la division auquel le lot est rattaché, en application du 3\degre{} de l’article 1251 du code civil ;
				
					« 2\degre{} Les dettes du syndicat initial sont réparties entre les syndicats issus de la division à hauteur du montant des créances du syndicat initial sur les copropriétaires transférées aux syndicats issus de la division. » ;
				\end{quote}
				
				On peut dès lors considérer que ce nouveau texte est une exception au principe général qui ne s’applique qu’à l’hypothèse de la scission d’une copropriété.
				
				Il est regrettable que nous ayons désormais deux régimes distincts de liquidation d’un syndicat des copropriétaires selon les causes de cette disparition !
				
				Ce nouveau texte légal est au demeurant surprenant puisqu’il fait « application » de l’article 1251 3\degre{} (ancien) du code civil selon lequel la subrogation a lieu « au profit de celui qui, tenu pour d’autres au paiement de la dette, avait intérêt de l’acquitter » : certes désormais le syndicat issu de la scission est bien tenu pour le syndicat d’origine en liquidation, des dettes de ce dernier, mais la transmission des créances de l’ancien syndicat sur les copropriétaires a lieu au profit du nouveau syndicat avant que ce dernier se soit acquitté des dettes de l’ancien syndicat.

	\subsection{La disparition du syndicat du fait de son annulation par le juge}
	
		Cette question concerne l’annulation d’un syndicat secondaire irrégulièrement créé.
		
		Dans une affaire PIGUET c/ Syndicat Le Consul\footnote{Civ. 3\degre{} 20 mai 2009 pourvoi: 07-22051 08-10043 08-10495 Publié au bulletin Loyers et Copropriété sep 2009 \no 218}, le Règlement de copropriété avait prévu l’existence d’un syndicat principal et de plusieurs syndicats secondaires alors que les conditions matérielles de séparation des bâtiments n’étaient pas réunies (cf. Troisième Partie – Le Syndicat secondaire).
		
		Quarante ans après l’entrée en application du Règlement de copropriété les consorts PIGUET ont demandé au juge d’annuler la clause du Règlement de copropriété prévoyant la constitution des syndicats secondaires. La Cour d’Appel annule la clause du Règlement de copropriété mais refuse de déclarer inexistants les syndicats secondaires : « Le syndicat secondaire institué par une clause réputée non écrite du Règlement de copropriété est censé n’avoir jamais existé ».
		
		La Cour d’Appel refuse de prononcer cette annulation affirmant que l’annulation du syndicat secondaire doit être prononcée pour l’avenir seulement (à compter de la décision annulant les syndicats secondaires). Pourvoi des consorts Piguet : en statuant ainsi la cour d’appel a violé les dispositions de l’article 43 de la loi. La Cour de Cassation rejette le pourvoi :
		\begin{quote}
			« Attendu que la Cour d’appel retient à bon droit que, même s’ils ont été institués par une clause du Règlement de copropriété ultérieurement réputée non écrite, les syndicats secondaires n’en ont pas moins acquis dès leur constitution et jusqu’à la décision ordonnant leur suppression, une personnalité juridique opposable aux tiers »
		\end{quote}.
		
		Si les termes de cet arrêt ne sont guère convaincants sur le plan des principes, la solution n’en est pas moins heureuse au plan pratique.
	
	\subsection{La question de la mise en redressement judiciaire ou liquidation du syndicat}
	
		\subsubsection{Discussion antérieure à la loi du 21 juillet 1994}
		
			Le problème se pose lorsque le syndicat est débiteur d'un tiers, par exemple d'un entrepreneur pour des travaux réalisés dans l'immeuble. Les copropriétaires ne paient pas leur quote-part de charges relative à ces travaux. Comment le créancier peut-il recouvrer sa créance ?

			La pratique a imaginé dans ce cas que le créancier assigne le syndicat des copropriétaires en liquidation judiciaire. En effet la loi sur la faillite de 1967, comme la loi du 25 janvier 1985 sur le redressement et la liquidation judiciaires des entreprises ont édicté que ce régime s'appliquait tant aux sociétés commerciales qu'aux sociétés civiles, associations ou syndicats.
			
			Cependant la doctrine admettait difficilement qu'un syndicat de copropriétaires qui n'est normalement propriétaire d'aucun bien (puisque les parties privatives comme les parties communes sont la propriété des copropriétaires), puisse être soumis à un régime de $\dots$ la liquidation de biens.
			
			De même un syndicat peut-il réellement se trouver en état de cessation de paiements dès lors que les débiteurs véritables sont les copropriétaires eux-mêmes ?
			
			De plus, admettre une procédure de redressement se concevait parfaitement dans la mesure où elle pouvait aboutir à un Plan de Redressement. Mais dans le cas contraire, le redressement se trouvait transformé en liquidation; d'où disparition de la personne morale.
			
			Or, peut-on faire disparaître un syndicat de copropriétaires autrement que par la réunion de toutes les parties d’immeubles entre les mains d'une seule personne ? Liquider un syndicat de Copropriété, c'est en définitive donner naissance à un nouveau syndicat remplaçant de plein droit celui qui existait antérieurement, sans pour autant que soient réglées les difficultés à l'origine de la liquidation de biens du premier syndicat.
			
			Par contre, refuser la procédure légale de redressement ou de liquidation d'un syndicat paraissait difficile dès lors que la loi s'appliquait à toutes les personnes morales sans distinction. De plus, il est vrai que la procédure de redressement présentait de sérieux avantages, surtout pour les syndicats en graves difficultés financières : on a d'ailleurs vu des syndics d'importantes résidences banlieusardes à Sarcelles ou près de Marseille, déposer le bilan de leur Copropriété.
			
			La loi du 21 juillet 1994 modifiant le statut de la Copropriété a tranché la question\footnote{Étant précisé que cet article était à l’origine l’article 29-4 de la loi et qu’il est devenu l’article 29-6 lors de la modification apportée par la loi SRU, puis 29-14 après publication de la loi ALUR !} :
			\begin{quote}
				ARTICLE 29-14 DE LA LOI : « Les procédures prévues au livre VI du code de commerce ne sont pas applicables aux syndicats de copropriétaires ».
			\end{quote}
			
			Par la même loi du 21 juillet 1994, le Parlement a doté les Copropriétés d'une arme nouvelle qui s'apparente à la procédure de l'Administrateur ad hoc prévue par la loi du 1er mars 1984 pour les Entreprises en Difficulté et que nous retrouverons ultérieurement lorsque nous étudierons l'administration judiciaire des syndicats de Copropriété.

			Précisons que la loi ALUR a modifié très sensiblement la procédure l’administration des copropriétés en difficulté la rapprochant de plus en plus de la procédure de redressement judiciaire des sociétés ; elle a elle-même été remaniée par la loi du 27 janvier 2017 dite Égalité et Citoyenneté.
			
		\subsubsection{L’action des créanciers contre les copropriétaires}
		
			Les créanciers –-- tant que le syndicat des copropriétaires n’est pas soumis à la procédure des articles 29-1 et suivants de la loi\footnote{Dispositions qui font l’objet de la dernière partie du cours (Cf. Poly 2)} --- peuvent bien évidemment agir contre le syndicat des copropriétaires pour obtenir sa condamnation au paiement de leur créance.
			
			Notamment ces créanciers disposent de la faculté de mettre en oeuvre les dispositions de l’article 29-1 A aux fins de désignation d’un mandataire ad hoc à copropriété en pré-difficulté\footnote{Cf. infra Chapitre XII, Section I}
			Le syndicat des copropriétaires n’a pas de patrimoine et ne peut pas être mis en en règlement judiciaire ou en liquidation de biens.
			
			Certes, le syndic devra recouvrer auprès des copropriétaires les sommes auquel il aura été condamné et nous avons vu qu’en ce cas ces condamnations étaient charges communes générales réparties entre tous les copropriétaires au prorata de leurs tantièmes généraux de copropriété, quelle que soit la cause de la condamnation au profit du tiers créancier.
			
			Si le syndic omet de recouvrer les sommes dues il néglige d’exercer les droits et actions des copropriétaires et nous verrons qu’en ce cas un administrateur judiciaire pourra être désigné à la copropriété, à la requête du créancier sur le fondement de l’article 24 f) du décret du 17 mars 1967\footnote{Civ 3\degre{} Ch 16 sep 2003, Loyers et Copropriété, comm 244}. Pour autant la question s’est rapidement posée de savoir si les créanciers du syndicat des copropriétaires ne disposaient pas d’une action directe à l’encontre des copropriétaires eux-mêmes.
			
			La cour de cassation devait répondre positivement à cette question en décidant qu’un créancier peut toujours agir contre les copropriétaires pour recouvrer sa créance (Civ 3\degre{} 30 oct 84, JCP 85 G, IV, 12; Civ 3\degre{} 7 nov 1990, Loy. et Cop. 1991 \no 41),
			La question posée a été de connaître le fondement juridique de cette action : action directe ou action oblique de l’article 1166 du code civil ?
			La cour de Paris dans un arrêt du 12 janvier 2000\footnote{Loyers et copr., 2000, comm. \no 156, note G. Vigneron} y a vu une application de l’action oblique : « Le créancier est fondé à exercer par la voie de l'action oblique, aux lieu et place du syndicat défaillant, l'action en recouvrement des quote-parts des charges communes contre chacun des copropriétaires pour obtenir le paiement des sommes qui lui sont dues ».
			
			La Cour de cassation a validé cette interprétation\footnote{3\degre{} Civ 26 oct 2005, pourvoi: 04-16664 Publié au bulletin ; Loyers et Copropriété 2006 \no 21} en écartant expressément l’action directe :
			\begin{quote}
				« Le syndicat des copropriétaires étant une personne morale de droit privé dont le patrimoine est distinct de celui de ses membres, le créancier dudit syndicat dispose d’une action oblique et non d’une action directe à l’égard des copropriétaires en paiement des sommes qui lui sont dues »
			\end{quote}
			
			Cette solution n’est pas idéale dans la mesure où les fonds obtenus par la voie de l’action oblique ne reviennent pas directement au créancier mais au syndicat des copropriétaires, à charge pour celui-ci de régler le créancier.
			
			Bien évidemment le créancier ne pourra pas demander à un copropriétaire de payer la totalité de sa créance puisqu’il n’y a pas de solidarité entre copropriétaires, il ne pourra demander paiement que de sa créance à hauteur de la quote-part du copropriétaire dans les parties communes générales et dans la mesure bien évidemment où ce copropriétaire n’a pas répondu aux appels de fonds du syndic destinés à régler le créancier : l’action oblique ne peut être mise en oeuvre que si le copropriétaire est effectivement débiteur du syndicat. En sorte que les copropriétaires peuvent s’exonérer en faisant la preuve qu’ils ont réglé cette quote-part entre les mains du syndic de la copropriété.
			
			La Cour de Paris, sur un appel de référé a d’ailleurs estimé que le créancier (en l’occurrence le chauffagiste) justifiait d’un intérêt légitime pour exiger du syndic la communication des noms et adresses des copropriétaires débiteurs sans que le syndic puisse utilement opposer que les coordonnées des copropriétaires pouvaient être obtenues au cadastre et à la conservation des hypothèques (documents qui ne permettent pas de savoir si les copropriétaires sont à jour ou non envers le syndicat) ni le secret professionnel\footnote{Paris 14\degre{} Chambre 28 avril 2006, Loyers et Copropriété 2006 \no 209}.
	
	\subsection{La disparition du syndicat en cas d’expropriation de l’immeuble pour carence et en cas d’expropriation des parties communes}
	
		Nous verrons à propos des copropriétés en difficulté que la loi \nom{Borloo} a institué une procédure de carence qui permet d’exproprier les copropriétaires lorsque ceux-ci n’entretiennent pas leur immeuble compromettant de la sorte la santé et la sécurité des occupants.
	
		Bien évidemment l’expropriation porte sur la totalité de l’immeuble, parties communes et parties privatives. Ce qui a pour conséquence de faire disparaître le syndicat des copropriétaires, tous les lots étant réunis entre les mains du bénéficiaire de l’expropriation (la Commune ou l’Opérateur choisi par la Commune).

		Cette procédure a été considérablement modifiée par la loi ALUR, article 72 qui, à titre expérimental, a créé une avant dernière étape avant l’expropriation totale de l’immeuble : l’expropriation des parties communes.
	
		Ainsi, la loi ALUR (article 72)\footnote{Numérotation de la Petite Loi (article 37 d’origine)} a institué, dans le cadre des remèdes à l’état de carence des copropriétés un article L 615-10 au CCH dont le I est ainsi rédigé :
		\begin{quote}
			Art. L. 615-10. CCH\\
		– I. – Par dérogation à l’article 6 de la loi \no 65-557 du 10 juillet 1965 fixant le statut de la copropriété des immeubles bâtis, une possibilité d’expropriation des parties communes est instaurée à titre expérimental et pour une durée de dix ans à compter de la promulgation de la loi \no du pour l’accès au logement et un urbanisme rénové. Dans ce cas, l’article L. 13-10 du code de l’expropriation pour cause d’utilité publique est applicable
		\end{quote}
		
		Ce texte signifie que pendant les dix années à venir, le Maire ou l’EPCI pourra lancer une procédure en expropriation des parties communes de l’immeuble.
		
		Le texte prévoit l’expropriation complète ou l’expropriation « de l’ensemble des parties communes ». Dans le dernier cas ces parties communes seront entretenues ou cédées à un Opérateur choisi par la Commune ou l’EPCI.
		
		A l’issue de cette procédure d’expropriation de l’ensemble des parties communes l’immeuble ne comportera plus que des propriétés privatives : celles d’origine restant appartenir aux copropriétaires d’origine et les anciennes parties communes devenues propriété de la Ville ou de l’Opérateur acquéreur de ces parties communes.
		
		La disparition du caractère commun de ces parties expropriées est d’autant plus évidente que le texte précise bien que ces parties d’immeuble devenues propriété de l’Opérateur vont se trouver grevées de servitudes au profit des locaux privatifs.
		
		Or, pour qu’il y ait copropriété, il faut qu’existent des parties communes dans l’immeuble.
		En sorte que la mise en œuvre de cette procédure entraîne la disparition du syndicat des copropriétaires.
		
		Le législateur était parfaitement conscient de cette situation puisque le même article L 615-10
		comprend un \VI{} ainsi rédigé :
		\begin{quote}
			\VI{}. – Après avis favorable de la commune ou de l’établissement public de coopération intercommunale compétent en matière d’habitat à l’origine de l’expérimentation et des propriétaires des biens privatifs, l’immeuble peut faire l’objet d’une nouvelle mise en copropriété à la demande de l’opérateur.
		\end{quote}
	
		Ceci explique d’ailleurs que la même loi ALUR précise au \VI{} de L. 615-6 du \CCH :
		\begin{quote}
			\VI{}. – Le cas échéant, dans l’ordonnance prononçant l’état de carence, le président du tribunal de grande instance désigne un administrateur provisoire mentionné à l’article 29-1 de la loi \no 65-557 du 10 juillet 1965 précitée pour préparer la liquidation des dettes de la copropriété et assurer les interventions urgentes de mise en sécurité.
			
			Sans préjudice des dispositions des articles L. 615-7 à L. 615-10 du présent code, la personnalité morale du syndicat subsiste après expropriation pour les besoins de la liquidation des dettes jusqu’à ce que le président du tribunal de grande instance mette fin à la mission de l’administrateur provisoire.
		\end{quote}
		
		C’est donc l’administrateur provisoire qui sera en ce cas liquidateur de plein droit du syndicat des copropriétaires.
	
	\subsection{Disparition par réunion des lots en une seule main}
	
		Il résulte de cette exigence de « division » de l’immeuble entre plusieurs propriétaires qu’une copropriété peut disparaître dans l'hypothèse inverse : lorsque tous les lots deviennent la propriété d'une seule personne (achat, héritage, $\dots$)\footnote{Cf. sur ce point l'article du Coneiller GUILLOT Administrer avril 1979 : La disparition du Syndicat des Copropriétaires.}, comme prévu désormais par l’article 46-1 de la Loi 65-557 du 10 juillet 1965 (article 39 de l’ordonnance du 30.10.2019).
		
		L’article 39 de l’ordonnance rétablit l’article 46-1 de la loi du 10 juillet 1965 afin de consacrer la jurisprudence de la Cour de cassation sur la réunion de tous les lots en une seule main :
		\begin{itemize}
			\item cette réunion entraîne la disparition de plein droit de la copropriété sans qu’une décision de l’assemblée générale ou du juge ne soit nécessaire pour le constater ;
			\item dans ce cas, le syndicat des copropriétaires, personne morale, subsiste uniquement pour les besoins de sa liquidation, sur le modèle de l’article 1844-8 du code civil en droit des société ;
			\item  le syndic est habilité à procéder aux opérations de liquidation et qu’à défaut, un mandataire ad hoc peut être désigné judiciairement pour représenter le syndicat en liquidation.
		\end{itemize}

		De même, un partage avec attribution divise peut entraîner la disparition du syndicat --- hypothèse réalisée par exemple à Roissy-en-Brie où plusieurs syndicats de copropriété horizontaux ont été purement et simplement dissous après que la Commune ait pris en charge l'ensemble des Voies et Réseaux Divers. Dans ce cas, il subsiste une pluralité de propriétaires, mais chacun est propriétaire, de façon exclusive de sa parcelle : il n’y a plus de parties communes indivises.