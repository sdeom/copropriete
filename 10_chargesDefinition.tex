\chapter{La définition légale et la répartition des charges de copropriété}

\section{SECTION I - GENERALITES}
	
	\subsection{A. DEFINITION DES CHARGES DE COPROPRIETE}
	
		Il n'existe pas de définition des charges communes de la Copropriété.
		Cette définition paraît devoir faire l'objet d'une triple approche :
		- Par leur objet
		- Par leur caractère juridique
		- Par leur caractère comptable
		1. Définition des charges par leur objet :
		Comme l'expose Madame KISCHINEWSKY-BROQUISSE381:
		" La qualité de copropriétaire des parties communes entraîne la charge de participer au financement de toutes dépenses causées par leur usage, leur entretien, leur administration et leur remplacement. Cette question est liée à la possession des choses communes".
		Les charges sont donc toutes les dépenses réalisées dans l'intérêt de la copropriété, c'est à dire afférentes aux parties communes et éléments d'équipement communs de l'immeuble :
		Les dépenses antérieures à la naissance de la copropriété ne sont pas des charges de copropriété (les dépenses de construction notamment Civ. 3\degres 20 déc. 1976 D 77 I.R. 320).
		381 op. cité p. 147
		droit de la copropriété année 2019-2020
		311
		Les dépenses réalisées dans l'intérêt des copropriétaires eux-mêmes ne sont pas des charges de copropriété (travaux réalisés sur parties privatives, même s'ils le sont par l'entreprise habituelle du syndicat).
		Cependant, il faut considérer à part le cas des « Résidences services » (cf. 0) Copropriétés spécialisées qui outre les charges habituelles nécessaires à la conservation de l'immeuble comprennent des charges destinées à la satisfaction des besoins individuels des copropriétaires. Le législateur a consacré ce principe par la loi du 13 juillet 2006 en créant un chapitre IV bis dans la loi du 10 juillet 1965.
		2. Définition des charges par leur nature juridique.
		Les charges communes constituent une créance du syndicat des Copropriétaires à l'encontre des copropriétaires eux-mêmes.
		Civ. 3\degres, 30 oct 1984 Bull Civ \no 180 p. 141:
		" La contribution de chacun des copropriétaires aux charges constitue le soutien de l'obligation du syndicat et correspond automatiquement à une créance de celui-ci sur chacun des copropriétaires ".
		Doivent alors être considérées comme des charges de copropriétés toutes les sommes qui sont dues par les copropriétaires au syndicat.
		Une telle définition est beaucoup plus large que la précédente puisque sont dues au syndicat non seulement les dépenses engagées sur les parties communes ou les éléments d'équipement communs, mais également, et conformément aux dispositions de l'article 35 du Décret du 17 mars 1967 modifié382:
		L'avance constituant la réserve prévue au règlement de copropriété
		Les provisions du budget prévisionnel, alinéas 2 et 3 de l’article 14-1
		Les provisions pour dépenses non comprises dans le buddget prévisionnel
		Les avances correspondant à l’échéancier prévu dans le plan pluriannuel de travaux accepté par l’assemblée
		Les cotisations au fonds de travaux prévus au II de l’article 14-2
		La provision prévue au règlement de copropriété lorsque l’immeuble est mis en copropriété et lorsque cette provision est épuisée le remboursement des dépenses faites par le syndic provisoire
		382 Cf. Infra Chapitre XV, Section IV
		droit de la copropriété année 2019-2020
		312
		3. Définition des charges par leur nature comptable.
		L’article 45-1 du décret \no67-223 du 17 mars 1967 modifié par le décret du 27 mai 2004 donne pour la première fois une définition comptable des charges, provisions et avances :
		Les charges sont les dépenses incombant définitivement aux copropriétaires, chacun pour sa quote-part. L'approbation des comptes du syndicat par l'assemblée générale ne constitue pas une approbation du compte individuel de chacun des copropriétaires.
		Au sens et pour l'application des règles comptables du syndicat :
		- sont nommées provisions sur charges les sommes versées ou à verser en attente du solde définitif qui résultera de l'approbation des comptes du syndicat ;
		- sont nommées avances les fonds destinés, par le règlement de copropriété ou une décision de l'assemblée générale, à constituer des réserves, ou qui représentent un emprunt du syndicat auprès des copropriétaires ou de certains d'entre eux. Les avances sont remboursables ».
		Si l’on retrouve bien dans cette définition la notion de dette du copropriétaire (« incombant aux copropriétaires »), les charges sont cependant définies dans cet article du décret par opposition aux provisions sur charges et aux avances.
	
	\subsection{B. HISTORIQUE DES TEXTES RELATIFS A LA REPARTITION DES CHARGES}
	
		1. Art. 664 du Code Civil.
		Nous avons précisé que le Code Civil ne consacrait que le seul article 664 à la Copropriété. Or cet article ne traitait que du problème de répartition des charges de la "maison" commune.
		Traitement sommaire puisqu'il n'était question que de réparation et reconstruction des gros murs, du toit, des planchers et de l'escalier !
		2. La loi du 28 juin 1938
		Selon l'article 8 de la loi du 28 juin 1938, troisième alinéa :
		droit de la copropriété année 2019-2020
		313
		"Dans le silence ou la contradiction des titres, chacun des copropriétaires est tenu de participer aux charges de conservation, d'entretien et d'administration des parties communes (...) proportionnellement aux valeurs respectives des fractions divises de l'immeuble eu égard à leur étendue et à leur situation".
		Donc la répartition de toutes les charges se faisait, sauf stipulation contraire, au prorata des tantièmes généraux.
		3. L’origine de la loi du 10 juillet 1965.
		Il a semblé aux auteurs de la loi que le système de 1938 était source d'iniquité. Sauf disposition particulière du Règlement de Copropriété, chaque copropriétaire participait à des dépenses, parfois très importantes qui pouvaient n'avoir aucun intérêt (aucune utilité) pour son lot: par exemple le propriétaire du rez-de-chaussée devait payer des charges d'entretien de l'ascenseur!
		Aussi est-il apparu qu'il fallait distinguer entre les catégories de charges en sorte que celles relatives à la conservation, à l'entretien et à l'administration des parties communes soient proportionnelles à la valeur des lots et celles relatives aux services collectifs et équipements communs soient réparties en fonction de l'utilité que ceux-ci représentent pour chaque lot, et fixer dans le règlement de copropriété « non seulement les millièmes de copropriété, mais aussi les millièmes de charges pour l'escalier, l'ascenseur, le chauffage, l'eau chaude, etc.". (Commission des lois).
		C. LES PRINCIPES DE REPARTITION DES CHARGES DANS LA LOI DU 10 JUILLET 1965
		1. Distinction des charges communes générales et des charges « spéciales »
		Article 10 de la Loi du 10 juillet 1965 (rédaction issue de l’Ordonnance du 30 octobre 2019)
		Les copropriétaires sont tenus de participer aux charges entraînées par les services collectifs et les éléments d'équipement commun en fonction de l'utilité objective que ces services et éléments présentent à l'égard de chaque lot, dès lors que ces charges ne sont pas individualisées.
		Ils sont tenus de participer aux charges relatives à la conservation, à l'entretien et à l'administration des parties communes, générales et spéciales, et de verser au fonds de travaux mentionné à l'article 14-2 la cotisation prévue au même article, proportionnellement aux valeurs relatives des parties privatives comprises dans leurs lots, telles que ces valeurs résultent des dispositions de l'article 5.
		droit de la copropriété année 2019-2020
		314
		Le règlement de copropriété fixe la quote-part afférente à chaque lot dans chacune des catégories de charges et indique les éléments pris en considération ainsi que la méthode de calcul ayant permis de fixer les quotes-parts de parties communes et la répartition des charges. Lorsque le règlement de copropriété met à la seule charge de certains copropriétaires les dépenses d'entretien et de fonctionnement entraînées par certains services collectifs ou éléments d'équipements, il peut prévoir que ces copropriétaires prennent seuls part au vote sur les décisions qui concernent ces dépenses. Chacun d'eux dispose d'un nombre de voix proportionnel à sa participation auxdites dépenses.
		2. Etablissement d’un « état de répartition des charges »
		Article 1er alinéas 2 à 4 du décret Du 17 mars 1967 :
		"Le règlement de copropriété (...) comporte (...) l'état de répartition des charges prévu au dernier alinéa de l'article 10 de ladite loi.
		Cet état définit les différentes catégories de charges et distingue celles afférentes à la conservation, à l'entretien et à l'administration de l'immeuble, celles relatives au fonctionnement et à l'entretien de chacun des éléments d'équipement commun et celles entraînées par chaque service collectif.
		L'état de répartition des charges fixe, conformément aux dispositions du troisième alinéa et, s’il y a lieu, du dernier alinéa de l’article 10 de la loi du 10 juillet 1965, la quote-part qui incombe à chaque lot dans chacune des catégories de charges
		".
		Le dernier alinéa de cet article exige donc que le règlement de copropriété comprenne un état de répartition des charges383, c’est-à-dire un tableau qui précise pour chaque lot la quote-part de participation du lot.
		3. Caractère impératif des dispositions relatives à la répartition des charges
		L'article 10 est de ceux que l'article 43 de la loi déclare impératifs. Il ne peut donc y être dérogé par le Règlement de Copropriété. Une répartition non conforme aux critères de l'article 10 est donc réputée non écrite384
		383 A l’origine le Décret précisait qu’à défaut d’état de répartition des charges le règlement de copropriété devait « indiquer les bases selon lesquelles la répartition est faite pour une ou plusieurs catégories de charges. Cette précision a été supprimée par le Décret de 2020 : il ne suffit plus de donner une clé de répartition mais il convient de faire un tableau de répartition entre les lots.
		384 (Civ. 3\degres, 5 juin 1970. J.C.P. 70.II.16.537). Etant rappelé que la repartition du Règlement de copropriété continue de s’appliquer tant que le juge ne l’a pas “annulée”.
		droit de la copropriété année 2019-2020
		315
		Ce qui signifie en pratique que les critères de l'article 10 de la loi s'appliquent alors même que la répartition prévue par le Règlement de Copropriété aurait été antérieure à la promulgation de la loi du 10 juillet 1965 : ce qui était licite, parce que contractuel, avant le 10 juillet 1965 est devenu illicite à cette date.
		
\section{SECTION II. LES CHARGES RELATIVES A LA CONSERVATION, L'ENTRETIEN ET L'ADMINISTRATION DES PARTIES COMMUNES.}
	
	Ces charges sont habituellement appelées les « charges communes générales ».
	
	\subsection{A. DEFINITION DES CHARGES COMMUNES GENERALES}
	
		1. Les charges relatives à la conservation des parties communes.
		Ce sont les dépenses de réparation, de réfection (notamment peinture, ravalement…), de reconstruction des parties communes.
		- Entretien et réparation des terrasses
		Selon la distinction rappelée entre gros oeuvre partie commune et revêtement ou protection partie privative, la réfection du gros oeuvre de la terrasse est charge commune générale, alors même que cette terrasse est partie privative ou à jouissance privative. 385
		- Entretien et réparation de la cage d'escalier.
		Il convient de rappeler qu'au terme de la jurisprudence la plus constante, les charges relatives aux parties d'escalier qui composent le gros oeuvre sont charges générales, 386 alors que les charges relatives au tapis et au revêtement de l'escalier sont charges afférentes à un élément d'équipement.387
		De même sont charges communes générales, les frais d'entretien, réparation, peintures de la loge du gardien ou autres locaux communs (local social; local bicyclettes ou voitures d’enfants, etc.).
		385 Civ. 3\degres 23 jan 1979, Gaz. Pal. 79 1 somm. 221). Civ 3\degres 18 décembre 1996, Loyers et Copr. 1997 \no90
		386 Civ. 3\degres 12 jan 1982; Administrer \no 123; Civ 3\degres 22 mars 1989 Loy et Cop mai 89 \no 244
		387 Civ. 3\degres 23 nov 1977; D.S. 1978 I.R. 294
		droit de la copropriété année 2019-2020
		316
		Ravalement et clause d’unité d’aspect
		S'agissant du ravalement, la totalité de la façade n'est pas toujours partie commune : par exemple la façade du rez-de-chaussée à usage commercial est parfois classée en parties privatives. En ce cas les devis de ravalement devront distinguer entre ce qui est à la charge de l'ensemble des copropriétaires et ce qui est à la charge du ou des copropriétaires du rez-de-chaussée. Il a été jugé que si la distinction n'était pas possible, l'ensemble du ravalement devait alors être réparti en charges générales.388
		Par contre une clause du Règlement de Copropriété dispensant un copropriétaire commerçant de participer aux charges de ravalement de la façade devra être considérée comme nulle, une telle clause étant contraire aux dispositions de l'article 10 alinéa 2. 389
		Il en irait de même d'une clause dispensant les propriétaires de garage de participer aux charges de ravalement dès lors que la façade de l'immeuble serait partie commune à tous les copropriétaires, en ce compris les propriétaires de garage390, ou encore le propriétaire d’un local commercial en rez- de-chaussée, quand bien même il lui incombe d’entretenir sa devanture privative391
		Enfin, pour préserver l’unité de style d’un immeuble, le règlement de copropriété prévoit assez fréquemment que le ravalement des façades comportera également les frais de peinture des garde- corps, fenêtres, persiennes, etc. Alors même que ces éléments sont parties privatives Une telle clause a été considérée comme licite392, mais les frais doivent alors être répartis par une refacturation des coûts réels sur les copropriétaires (ref) en tenant compte par exemple du linéaire de chaque lot.
		- Frais d'éclairage des parties communes.
		La Cour de Cassation a tout d'abord retenu ces charges au nombre de celles entraînées par les éléments d'équipement393; elle a, en dernier lieu394 opté pour leur classement en charges relatives à la conservation de l'immeuble.
		388 TGI Paris 1er mars 1974, AJPI 74 p. 811 note Bouyeure.
		389 Civ 3\degres 9 déc 1987 commenté in RTDI 1988.139 ainsi que le curieux arrêt de la Cour de PARIS 23\degres Chambre, 6 novembre 1992, mal motivé, mais surtout le commentaire du Professeur GIVERDON à la RTDI 93.124
		390 PARIS 23\degres Ch 7 mai 1993, Loyers et Copropriété août-sep 93 \no 318
		391 Ca Paris 7 décembre 1993
		392 PARIS 23\degres A, 19 mars 1997, Loyers et Copropriété 1997 \no 294.
		393 Civ 3\degres 1' mai 1970; D.S. 1971, 95
		394 Civ. 3\degres, 4 jan 1989, D 89, 260 note Souleau; Civ 3\degres 9 déc 1992; Loyers et Copropriété mars 1993 \no 108. Civ 3\degres 27 nov 1991 Loyers et Copropriété fév 1992 \no 85; Paris 23\degres Ch, 7 déc 1993 Loyers et Copropriété 1994 \no 124. Plus récemment, voir PARIS 23\degres Ch, 23 avril 1997, Loyers et Copropriété 1997 \no 272 qui répute non écrite la clause du règlement de copropriété exonérant le propriétaire du rez-de-chaussée de l’immeuble de la participation aux frais d’éclairage de la cage d’escalier.
		droit de la copropriété année 2019-2020
		317
		2. Les charges relatives à l'entretien des parties communes.
		- Achat des produits d'entretien et matériels d'entretien
		Par ex : tondeuse à gazon, ampoules….
		- Entretien des cours et jardins lorsqu'ils sont rangés dans les parties communes,
		Il n’y a pas à prendre en considération la situation particulière d'un lot commercial qui ne bénéficie pas directement de la jouissance de ces jardins.395
		- Salaires et sommes versées aux personnes chargées de l'entretien des parties communes (concierge, entreprise de nettoyage, jardinier).
		Pendant longtemps les tribunaux ont eu tendance à distinguer parmi les tâches du personnel en classant comme charges générales les tâches relatives à l'entretien des parties communes et comme service collectif les tâches relatives à la surveillance de l'immeuble et à la distribution du courrier, en sorte que les locaux commerciaux qui disposent le plus souvent d'un accès distinct de celui de l'immeuble, de boite aux lettres distinctes ou qui assument leur surveillance par leurs propres moyens, se voyaient dispensés de participer aux charges correspondantes. Cette ventilation opérée par les Tribunaux était facilitée par l'examen des contrats d'employés d'immeuble qui affectent des unités de valeur à chacune des taches remplies.396
		Mais depuis la Cour de Cassation, par plusieurs arrêts a adopté une position inverse en décidant que les salaires et cotisations sociales devaient être considérés sans distinction quant à la nature des tâches accomplies, comme des charges afférentes à la conservation et à l'entretien de l'immeuble.397
		Toutefois, cette jurisprudence applicable au personnel du syndicat ne saurait être étendue aux contrats de gardiennage passés par le syndicat avec des entreprises pour la surveillance de telle ou telle partie de l’immeuble (par exemple les parkings). En ce cas l'objet du contrat de gardiennage serait d'assurer un service collectif réparti en fonction de l'utilité que ce service présente pour chaque lot (dans l’exemple supposé, ces charges seraient à répartir entre les propriétaires de parkings398).
		395 Civ 3\degres 20 juillet 1994; Loy. et Copr. Dec 1994 \no 483
		396 dans le sens de cette distinction cf. par exemple Civ. 3\degres 18 juil 1979; D.S. 80 I.R. 238
		397 Soc. 3 mai 1984, Bull. Soc. \no 168 p. 130; Civ. 3\degres, 6 mars 1991, Bull. civ. \no 80 p. 48; Civ. 11 juin 1992 Rep. Defrénois 1993 p. 355; Civ 3\degres 9 déc 1992, Loyers et Copropriété mars 1993 \no 108
		398 PARIS 26 mai 1989, D 1990 Somm p 131.
		droit de la copropriété année 2019-2020
		318
		3. Charges relatives à l'administration des parties communes.
		- Les coûts induits par la propriété des parties communes
		LES IMPOTS
		LES FRAIS D'ACTES DE MUTATION OU DE PUBLICATION.
		Ce sont les frais que l'on qualifie improprement de "frais notariés" entraînés par exemple par l'acquisition d'une partie privative ou par la publication d'un modificatif au Règlement de Copropriété.
		LES PRIMES D'ASSURANCES.
		Un problème peut cependant se poser concernant l'augmentation des primes imposées par l'Assureur du fait de l'aggravation des risques due à l'activité de certains copropriétaires (présence d'un établissement classé dans l’immeuble ; affectation d'un sous-sol à usage de studio d’enregistrement ; cave à usage de dépôt d'objets d'art, etc.).
		En ce cas, il semble que la Copropriété sera tentée de faire jouer la clause d'aggravation des charges,399 lorsqu'elle existe dans le Règlement de Copropriété, mettant à la charge du copropriétaire la surprime générée par son activité. La Cour de Paris400 a refusé cette pratique (s’agissant de l’exploitation d’une discothèque dans l’immeuble) au motif que les frais d’assurances exposés par le syndicat des copropriétaires sont des charges communes au sens des articles 10 et 5 de la loi et ne peuvent à ce titre qu’être réparties entre tous les copropriétaires, sans égard à l’utilisation du lot.
		Interprétation retenue par la cour de cassation à propos du surcoût des primes payées par la copropriété du fait de l’installation d’une discothèque dans la galerie marchande de la copropriété401 ; ce surcoût ne peut être imputé aux seuls copropriétaires de la galerie marchande mais doit être payé comme charge commune générale à l’ensemble des copropriétaires.
		- Les frais de fonctionnement du syndicat des copropriétaires
		399 Voir infra \no 155 « La Clause d’aggravation des charges ».
		400 PARIS 23\degres B 2 mai 1997, Loyers et Copropriété 1997 \no 295.
		401 Cass. Civ.3ème 4 juin 2009
		droit de la copropriété année 2019-2020
		319
		LES HONORAIRES ET FRAIS PAYES AU SYNDIC.402
		LES HONORAIRES D'ARCHITECTE D'ENTRETIEN.
		FRAIS D'ASSEMBLEES GENERALES
		(location de salles, l'envoi des convocations, la notification des proces-verbaux.) Sous réserve toutefois des nouvelles dispositions de l’article 17-1-AA permettant à un copropriétaire de solliciter la convocation à ses frais d’une assemblée générale ne concernant que ses droits ou obligations.
		FRAIS DE FONCTIONNEMENT DU CONSEIL SYNDICAL.
		HONORAIRES DE TECHNICIEN CONSULTE PAR LE CONSEIL SYNDICAL.
		L'article 27 du Décret du 17 mars 1967 prévoit que « Le conseil syndical peut, pour l'exécution de sa mission, prendre conseil auprès de toute personne de son choix. Il peut aussi, sur une question particulière, demander un avis technique à tout professionnel de la spécialité ».
		HONORAIRES DE L’ADMINISTRATEUR PROVISOIRE
		On relèvera une décision403 qui juge que les frais entraînés par la nomination d'un administrateur provisoire de la Copropriété dans l'attente de la désignation d'un syndic doivent être pris en charge par l'ensemble des copropriétaires (et non par les seuls copropriétaires qui ont sollicité cette désignation dans l'intérêt commun).
		Les Frais induits par les procédures judiciaires (Honoraires des avocats et experts, Condamnation du syndicat des copropriétaires)
		Les Honoraires des Avocats et Experts sont incontestablement des charges communes générales. Or cette situation peut paraître inéquitable dans deux cas :
		- les frais de justice exposés par le syndicat des copropriétaires à l’encontre d’un copropriétaire du fait de ce dernier, en particulier en cas de recouvrement de charges, ne devraient être supportés que par le copropriétaire défaillant
		402 Civ 3\degres 1er avril 1987, Bull \no 74.
		403 PARIS 23\degres Chambre, 9 juillet 1993; Loy et Cop déc 93 \no 446
		droit de la copropriété année 2019-2020
		320
		- inversement, il est anormal qu’un copropriétaire engageant une action bien fondée contre le syndicat des copropriétaires ait à payer, dans sa quote part de charges, les frais judiciaires exposés par le syndicat des copropriétaires pour sa défense
		Pour tenir compte de ces deux situations, la loi SRU, complétée par la loi ENL, a créé un article 10-1 de la loi du 10 juillet 1965 qui déroge au principe de la généralité de ces dépenses, modifié à plusieurs reprises et dernièrement par l’ordonnance
		Article 10-1 Modifié par Ordonnance \no2019-1101 du 30 octobre 2019 - art. 10 Par dérogation aux dispositions du deuxième alinéa de l'article 10, sont imputables au seul copropriétaire concerné : a) Les frais nécessaires exposés par le syndicat, notamment les frais de mise en demeure, de relance et de prise d'hypothèque à compter de la mise en demeure, pour le recouvrement d'une créance justifiée à l'encontre d'un copropriétaire ainsi que les droits et émoluments des actes des huissiers de justice et le droit de recouvrement ou d'encaissement à la charge du débiteur ; b) Les frais et honoraires du syndic afférents aux prestations effectuées au profit de ce copropriétaire. Les honoraires et frais perçus par le syndic au titre des prestations qu'il doit effectuer pour l'établissement de l'état daté à l'occasion de la mutation à titre onéreux d'un lot, ou de plusieurs lots objets de la même mutation, ne peuvent excéder un montant fixé par décret ; c) Les dépenses pour travaux d'intérêt collectif réalisés sur les parties privatives en application du c du II de l'article 24 et du f de l'article 25 ; d) Les astreintes , fixées par lot, relatives à des mesures ou travaux prescrits par l'autorité administrative compétente ayant fait l'objet d'un vote en assemblée générale et qui n'ont pu être réalisés en raison de la défaillance du copropriétaire. Le copropriétaire qui, à l'issue d'une instance judiciaire l'opposant au syndicat, voit sa prétention déclarée fondée par le juge, est dispensé, même en l'absence de demande de sa part, de toute participation à la dépense commune des frais de procédure, dont la charge est répartie entre les autres copropriétaires. Le juge peut toutefois en décider autrement en considération de l'équité ou de la situation économique des parties au litige.
	
	\subsection{B. MODALITES DE REPARTITION DES CHARGES GENERALES}
	
		1. Le principe : répartition en proportion de la valeur relative des lots (art. 10 de la loi)
		Comme sous l'empire de la loi de 1938 ces charges sont réparties proportionnellement aux valeurs relatives des parties privatives comprises dans leur lot. L'article 10 de la loi du 10 juillet 1965 ajoute : « telles que ces valeurs résultent des dispositions de l'article 5. »
		Rappelons que l'article 5 de la loi du 10 juillet 1965 modifié par l’Ordonnance du 30 octobre 2019, édicte que :
		"Dans le silence ou la contradiction des titres, la quote-part des parties communes, tant générales que spéciales, afférente à chaque lot est proportionnelle à la valeur relative de chaque partie privative par rapport à l'ensemble
		droit de la copropriété année 2019-2020
		321
		des valeurs desdites parties, telles que ces valeurs résultent lors de l'établissement de la copropriété, de la consistance, de la superficie et de la situation des lots, sans égard à leur utilisation »
		En d'autres termes, c'est le Tableau des Quotes-parts de propriété qui sert normalement de Tableau des quotes-parts de charges dites générales : le copropriétaire qui possède 150/1000èmes des parties communes supportera 150/1000èmes des charges générales de l'immeuble.
		On constate ici une contradiction : l'article 10 de la loi est d'ordre public alors que l'article 5 ne l'est pas. Comment concilier les deux types de dispositions ? Les tribunaux se sont vus soumettre très tôt la difficulté et ont considéré que les trois critères de l'article 5, s'ils étaient facultatifs et supplétifs pour déterminer la valeur relative de chaque lot, étaient d'ordre public pour déterminer la répartition des charges générales entre ces lots.
		La jurisprudence fournit un exemple éloquent404 :
		Le Règlement de Copropriété répartissait les charges conformément à l’état descriptif de division. L’état descriptif attribuait à chaque lot une quote-part fonction des trois critères de l’article 5 (consistance, superficie, situation), mais ajoutait un quatrième critère : celui de la nature d’affectation des lots.
		La Cour de cassation censure l’arrêt d'appel déboutant les copropriétaires demandeurs en modification de la répartition des charges alors que ceux-ci "se référaient au rapport d'expertise constatant que la répartition des charges générales effectuée par le Règlement de Copropriété découlait de la prise en considération dans celui-ci, pour la fixation des quotes-parts, de la destination des parties privatives".
		Dans la pratique et le plus souvent il n'existe pas dans le règlement de copropriété de tableau de répartition des charges générales, mais seulement un renvoi au tableau de répartition des tantièmes de copropriété.
		2. Les exceptions.
		Il serait erroné d’affirmer que toutes les charges relatives à la conservation, à l’entretien et même à l’administration de l’immeuble sont nécessairement réparties proportionnellement à la valeur relative de chaque lot par rapport à la valeur de l’ensemble des lots de la Copropriété.
		Deux exceptions doivent être signalées :
		- L’hypothèse des charges afférentes aux parties communes spéciales
		- L’hypothèse de la spécialisation des charges résultant de la création d’un ou de plusieurs syndicats secondaires.
		404 Civ. 3\degres Chambre 11 janvier 1995 (Revue de Droit Immobilier 1995 p 371)
		droit de la copropriété année 2019-2020
		322
		- Les charges afférentes aux parties communes spéciales.
		Le plus souvent lorsqu'une copropriété comporte plusieurs bâtiments, le Règlement de Copropriété prendra soin de placer chaque bâtiment en parties communes spéciales aux seuls propriétaires de ce bâtiment. Ceci signifie que les propriétaires des autres bâtiments n'auront pas la copropriété de ce bâtiment... et n'en assumeront pas davantage la charge.
		Dès lors seuls les propriétaires de ces parties communes spéciales assumeront les charges afférentes à ces bâtiments :
		« Mais attendu qu’ayant, par motifs adoptés, relevé que le règlement de copropriété définissait des parties communes spéciales propres à l’usage de certains copropriétaires répartis en blocs et prévoyait que la toiture terrasse du bloc C ainsi que les installations et constructions qui s’y trouvent sont la propriété de ce bloc, la cour d’appel a, à bon droit, retenu que la création dans le règlement de copropriété de parties communes spéciales avait pour corollaire l’instauration de charges spéciales »405
		Mais il ne s'agit pas là à proprement parler d'une dérogation aux dispositions conjuguées des articles 10 alinéa 2 et 5 de la loi : ces articles renvoient aux charges relatives à la conservation, à l'entretien et à l'administration des parties communes; or, il peut y avoir des parties communes à tous les copropriétaires ou à certains d'entre eux seulement.
		Dès lors la règle de l'article 10 alinéa 2 peut recevoir application : la participation du copropriétaire dans les charges afférentes aux parties communes spéciales devront être proportionnelles à la valeur du lot par rapport à la valeur totale des parties privatives qui ont des droit de copropriété sur ces parties communes spéciales.
		La loi \no 2018-1021 du 23 novembre 2018 dite ELAN a consacré ces principes dans l’article 6-2 nouveau de la loi \no 65-557 du 10 juillet 1965
		Article 6-2 nouveau créé par LOI \no2018-1021 du 23 novembre 2018 - art. 209 (V) Les parties communes spéciales sont celles affectées à l'usage ou à l'utilité de plusieurs copropriétaires. Elles sont la propriété indivise de ces derniers. La création de parties communes spéciales est indissociable de l'établissement de charges spéciales à chacune d'entre elles. Les décisions afférentes aux seules parties communes spéciales peuvent être prises soit au cours d'une
		405 Civ. 3\degres Ch. Arrêt \no 697 du 8 juin 2011 (10-15.551)
		droit de la copropriété année 2019-2020
		323
		assemblée spéciale, soit au cours de l'assemblée générale de tous les copropriétaires. Seuls prennent part au vote les copropriétaires à l'usage ou à l'utilité desquels sont affectées ces parties communes.
		Ceci explique que l’Ordonnance du 30 octobre 2019 ait modifié le texte de l’article 5 de la loi pour viser la détermination des tantièmes de propriété des parties communes générales d’une part et des parties communes spéciales d’autre part.
	
	\subsection{C. LE PROBLEME DE LA SPECIALISATION DES CHARGES COMMUNES GENERALES}
	
		Bien souvent, le Règlement de Copropriété, sans pour créer de parties communes spéciales, a crée des charges communes spécialisées406.
		Par exemple l’escalier ou l’ascenseur ont été mentionnés dans les parties communes générales mais les charges de l’un ou de l’autre étaient réparties entre certains copropriétaires seulement (ceux dont les lots ont un accès direct à l’escalier et ceux pour lesquels l’ascenseur présente une utilité).
		Plus fréquemment le Règlement de Copropriété, sans classer le garage (niveaux de sous-sol) en parties communes spéciales aux seuls propriétaires de boxes ou d'emplacement, stipulait que les dépenses afférentes à ce garage (électricité, entretien des allées communes et escaliers) seraient réparties entre ces seuls propriétaires.
		C’était l’hypothèse traitée par l’article 24, alinéa 4 de la Loi du 10 juillet 1965. Bien que relatif aux modalités de vote à l'intérieur de l'Assemblée Générale, cet article était ainsi rédigé :
		"Lorsque le règlement de copropriété met à la charge de certains copropriétaires seulement les dépenses d'entretien d'une partie de l'immeuble ou celles d'entretien et de fonctionnement d'un élément d'équipement, il peut être prévu par ledit règlement que ces copropriétaires seuls prennent part au vote sur les décisions qui concernent ces dépenses. Chacun d'eux vote avec un nombre de voix proportionnel à sa participation auxdites dépenses".
		Ainsi, le principe d’une répartition spéciale des charges portant sur les charges d’entretien407 des parties communes ou d’entretien et de fonctionnement des éléments d’équipement commun était admis.
		Depuis 1993 jurisprudence s’est montrée hostile à la spécialisation des charges d’entretien des parties communes générales, et l’Ordonnance du 30 octobre 2019, tout en maintenant la faculté de créer des charges spéciales afférentes aux éléments d’équipement communs sans que ceux-ci soient constitués en parties communes spéciales, a supprimé purement et simplement la faculté de créer des charges spéciales d’entretien des parties communes sans création concomitante de parties communes spéciales.
		406 Cf analyse de Me TALAU sur la spécialisation des charges, in Loyers et Copropriété mars 2013, Etude \no 5
		407 Il fallait entendre ici les charges d’entretien et de conservation …
		droit de la copropriété année 2019-2020
		324
		1. Le principe : un lot ne peut être dispensé de sa participation aux charges communes générales
		Si le Règlement de Copropriété exonère certains lots de participation aux charges, il y a également exception au principe de répartition des charges à proportion de la valeur relative de chaque lot dans l'ensemble des parties communes.
		Par le passé cette exonération pouvait résulter d'une volonté délibérée de faire échapper ces lots au paiement de charges qui auraient normalement dû leur incomber. Cette pratique est forcément condamnée par la loi du 10 juillet 1965.
		2. La question des lots n’ayant pas « l’utilité » de certaines dépenses relevant des charges communes générales
		Le Règlement de Copropriété peut-il valablement exonérer certains lots, du fait de leur situation, de toute participation à certaines charges dites générales comme par exemple l'entretien des jardins sur lesquels ils n'auraient aucune ouverture (hypothèse fréquente pour les commerces dépendant de grands ensembles immobiliers) ?
		De la même façon dès lors que l'on met à la charge d'un commerçant les frais exclusifs d'entretien et de ravalement de sa devanture, ne pourrait-on, par souci d'équité, l'exonérer de participer aux charges de ravalement de l'ensemble ?
		Enfin, bien souvent le Règlement de Copropriété met à la charge des seuls copropriétaires des étages les frais d’entretien d’une cage d’escalier alors même que cet escalier est partie commune à l’ensemble des lots.
		La jurisprudence, après avoir longtemps validé sans distinction les clauses de spécialisation des charges communes, sans distinction, semble aujourd’hui beaucoup plus réticentes à les admettre, au motif qu’elles sont contraires au principe posé par l’article 5 et son caractère impératif. C’est particulièrement vrai pour les clauses formulées sous forme de « dispenses de charge » pour les lots qui n’auraient pas « l’utilité » de certaines dépenses de conservation de l’immeuble
		civ. 3ème 18 mai 1988408
		408 Civ 3\degres Ch, 8 juillet 1998, Planteurs d’Abyssinie c/ syndicat rue du Commerce (Administrer 1999 \no 308 p. 59 note Bouyeure)
		droit de la copropriété année 2019-2020
		325
		Cet arrêt concernait les frais de ravalement d’un porche ouvert dans le bâtiment sur rue mais donnant accès aux seuls lots situés en fond de cour, en sorte que la Cour de Paris avait estimé que ces travaux de ravalement étaient sans utilité pour les lots dépendant du bâtiment sur rue. Le porche figurant cependant au nombre des parties communes générales de l’immeuble. Cassation au motif que les charges communes générales ne peuvent être réparties selon le critère de l’utilité.
		Civ 3ème 8 juillet 1998
		Condamnation de l’exonération de certains lots des frais d’entretien d’un hall d’entrée et d’une cage d’escalier alors qu’il s’agissait de locaux commerciaux.
		Civ 3\degres 4 juillet 2006 AJDI 2006 p 747
		La clause du règlement de copropriété mettant à la charge des seuls lots ayant accès à cette cage les frais d’entretien de celle-ci est réputée non écrite dès lors qu’il n’était pas démontré que le lot « dispensé » n’avait pas accès à cet escalier, il ne pouvait être dispensé de participer à l’entretien de cette cage d’escalier
		Au vu de ces arrêts, certains commentateurs ont affirmé qu’il ne pouvait y avoir de spécialisation des charges sans spécialisation des parties communes, et se sont demandés si les dispositions de l’article 24 alinéa 4 pouvaient recevoir application en dehors de la constitution de des parties communes spéciales ou d’un syndicat secondaire409.
		Cependant, la Cour de Cassation a, dans le même temps, validé des clauses de spécialisation des charges communes générales comme n’étant pas contraire aux principes des articles 10 et 5 de la loi du 10 juillet 1965. Mais à rebours des précédentes, ces clauses ont pour objet d’attribuer des charges communes à certains copropriétaires, et non d’en dispenser d’autres.
		On citera notamment à ce sujet un arrêt très motivé du 16 septembre 2003 de la 3ème Chambre Civile de la Cour de Cassation, selon lequel la clause attribuant aux seuls copropriétaires titulaires d’un droit de jouissance privatif sur la terrasse partie commune la réfection de l’étanchéité était :
		« une clause spéciale licite [devant] recevoir application en ce qu’elle dérogeait aux dispositions générales énoncées par les autres dispositions du règlement de copropriété sur la définition des charges et parties communes générales et que ces dispositions étaient légales et non contraires aux articles 5 et 10 de la loi du 10 juillet 1965, dans la mesure où aucune violation des doubles principes d’utilité, en ce qui concernait les éléments d’équipement commun, et de valeur relative des lots, en ce qui concernait l’administration et l’entretien de l’immeuble, n’était démontrée. »
		Au demeurant, si on restreint l’article 24 alinéa 4 à l’hypothèse de parties communes spéciales ou d’un Syndicat des Copropriétaires secondaire, celui-ci n’a plus de raison d’être
		409 Cf également l’avis du Professeur Giverdon dans le commentaire de l’arrêt du 6 mai 2003.
		droit de la copropriété année 2019-2020
		326
		3. La suppression de l’article 24 alinéa III par l’Ordonnance du 30 octobre 2019 et la nouvelle rédaction de l’article 10 in fine
		L’Ordonnance du 30 octobre 2019 a mis fin à cette faculté de créer des charges « spéciales » pour des parties communes générales. En effet, l’article 24 alinéa 4 a été supprimé et remplacé par un nouvel alinéa dans l’article 10, ainsi rédigé Article 10 Modifié par Ordonnance \no2019-1101 du 30 octobre 2019 - art. 9 « Lorsque le règlement de copropriété met à la seule charge de certains copropriétaires les dépenses d'entretien et de fonctionnement entraînées par certains services collectifs ou éléments d'équipements, il peut prévoir que ces copropriétaires prennent seuls part au vote sur les décisions qui concernent ces dépenses. Chacun d'eux dispose d'un nombre de voix proportionnel à sa participation auxdites dépenses. » En d’autres termes la loi maintient la faculté de créer des charges spécialisées pour les éléments d’équipement (et les services collectifs) sans création de parties communes spéciales, mais, par cohérence avec les nouvelles dispositions de l’article 6-2 selon lesquelles : « La création de parties communes spéciales est indissociable de l'établissement de charges spéciales à chacune d'entre elles » elle crée – sans l’écrire expressément - la réciproque en sorte qu’il ne peut plus y avoir de charges communes spéciales afférentes à l’entretien et à la conservation de l’immeuble sans lla création de parties communes spéciales. En effet, il n’est plus question, dans ce nouvel article de charges incombant à certains copropriétaires seulement pour l’entretien des parties communes. Le rapport au Président de la République pour approbation du projet d’Ordonnance, énonce à ce sujet « Enfin, par cohérence de thématique, le III de l'article 24 est déplacé partiellement au dernier alinéa de l'article 10 de la loi du 10 juillet 1965, en supprimant par ailleurs, la référence aux dépenses d'entretien d'une « partie de l'immeuble » qui supposeront, pour être mises à la charge de certains copropriétaires seulement, la création de parties communes spéciales dans le règlement de copropriété (articles 6-2 et 6-4 de la loi du 10 juillet 1965). »
	
	\subsection{D. LES DISPENSES LEGALES DE PARTICIPER AUX CHARGES COMMUNES GENERALES (ARTICLE 10-1)}
	
		Il convient d’évoquer différentes hypothèses pour lesquels le législateur a lui-même décidé de déroger au principe de l’obligation générale de contribuer aux charges communes générales. Dans la rédaction issu de l’Ordonnance du 30 octobre 2019, ces dérogations sont les suivantes (article 10-1 de la loi \no 65-557 du 10 juillet 1965)
		droit de la copropriété année 2019-2020
		327
		Par dérogation aux dispositions du deuxième alinéa de l'article 10, sont imputables au seul copropriétaire concerné : a) Les frais nécessaires exposés par le syndicat, notamment les frais de mise en demeure, de relance et de prise d'hypothèque à compter de la mise en demeure, pour le recouvrement d'une créance justifiée à l'encontre d'un copropriétaire ainsi que les droits et émoluments des actes des huissiers de justice et le droit de recouvrement ou d'encaissement à la charge du débiteur ; b) Les frais et honoraires du syndic afférents aux prestations effectuées au profit de ce copropriétaire. Les honoraires et frais perçus par le syndic au titre des prestations qu'il doit effectuer pour l'établissement de l'état daté à l'occasion de la mutation à titre onéreux d'un lot, ou de plusieurs lots objets de la même mutation, ne peuvent excéder un montant fixé par décret ; c) Les dépenses pour travaux d'intérêt collectif réalisés sur les parties privatives en application du c du II de l'article 24 et du f de l'article 25 ; d) Les astreintes , fixées par lot, relatives à des mesures ou travaux prescrits par l'autorité administrative compétente ayant fait l'objet d'un vote en assemblée générale et qui n'ont pu être réalisés en raison de la défaillance du copropriétaire. Le copropriétaire qui, à l'issue d'une instance judiciaire l'opposant au syndicat, voit sa prétention déclarée fondée par le juge, est dispensé, même en l'absence de demande de sa part, de toute participation à la dépense commune des frais de procédure, dont la charge est répartie entre les autres copropriétaires. Le juge peut toutefois en décider autrement en considération de l'équité ou de la situation économique des parties au litige.
		Il convient de reprendre successivement ces différents points :
		1. Les frais exposés par le syndicat des copropriétaires pour le recouvrement judiciaire des charges
		Le recouvrement des charges entraîne des frais pour le syndicat des copropriétaires.
		Or le Décret du 26 mars 2015 impose un contrat type au syndic, aux termes duquel il est dit
		article 9 : Le coût des prestations suivantes est imputable au seul copropriétaire concerné et non au syndicat des copropriétaires qui ne peut être tenu d’aucune somme à ce titre (9.1 frais de recouvrement).
		Le même article 9-1 définit désormais les frais et honoraires du syndic imputables au copropriétaire débiteur, savoir :
		- Mise en demeure par lettre recommandée avec accusé de réception
		- Relance après mise en demeure ;
		- Conclusion d’un protocole d’accord par acte sous seing privé ;
		- Frais de constitution d’hypothèque ;
		- Frais de mainlevée d’hypothèque ;
		- Dépôt d’une requête en injonction de payer ;
		droit de la copropriété année 2019-2020
		328
		- Constitution du dossier transmis à l’auxiliaire de justice (uniquement en cas de diligences exceptionnelles) ;
		- Suivi du dossier transmis à l’avocat (uniquement en cas de diligences exceptionnelles).
		Les honoraires d’avocat sont considérés, le plus souvent, comme ne relevant pas de l’article 10-1 s’il s’agit d’imputer cette dépense à un copropriétaire, car le Syndicat des Copropriétaires bénéficie alors déjà de l’article 700 du CPC. En revanche, les tribunaux semblent plus enclins à inclure parmi les « frais nécessaires » dont le copropriétaire vainqueur peut être dispensé les honoraires de l’avocat du Syndicat des Copropriétaires
		2. Les frais et honoraires du syndic afférents aux prestations effectuées au profit de ce copropriétaire. Les honoraires et frais perçus par le syndic au titre des prestations qu'il doit effectuer pour l'établissement de l'état daté à l'occasion de la mutation à titre onéreux d'un lot ou de plusieurs lots objets de la même mutation, ne peuvent excéder un montant fixé par décret ;
		Cette formulation est issue de l’Ordonnance du 30 octobre 2019 : auparavant, seuls étaient visés les honoraires du syndic pris pour l’état daté. Désormais, celui-ci a donc la possibilité de facturer individuellement à un copropriétaire des « prestations individuelles » qui seraient distinctes de son contrat de syndic (intervention pour changement d’un robinet d’eau privatif, maitrise d’ouvrage déléguée sur des travaux réalisé par un copropriétaire sur parties communes, … ?)
		La plafonnement des honoraires pour état daté est prévu depuis la loi LOI \no 2014-366 du 24 mars 2014 dite ALUR… mais le décret n’a été publié que le 21 février 2020 ! Sur ce point, l’ordonnance précise toutefois que les honoraires ne peuvent être perçus qu’une seule fois même si plusieurs lots sont cédés simultanément…
		Ce montant maximum fixé par le Décret n \degres 2020-153 du 21 février 2020 est de 380 € TTC.
		3. Les dépenses pour travaux d’intérêt collectif réalisés sur les parties privatives «en application du c du II de l’article 24 et du f de l’article 25 »
		Nous savons que ces travaux sur parties privatives peuvent être imposés par l’assemblée générale au copropriétaire et qu’ils seront réalisés par le syndicat des copropriétaires qui en assurera la maîtrise d’ouvrage jusqu’à réception (définitive !) de ces travaux.
		Il est évident que s’agissant malgré tout de travaux sur parties privatives ils ne peuvent incomber qu’au copropriétaire … bénéficiaire, quand bien même est-il contraint de supporter ces travaux.
		4. Les astreintes
		L’ordonnance du 30.10.2019 a considérablement simplifié le texte, qui visait auparavant les astreintes prévues à l’article L. 1331-29 du code de la santé publique et aux articles L. 129-2 et L. 511-2 du code de la construction et de l’habitation
		droit de la copropriété année 2019-2020
		329
		Lorsque des mesures ou travaux prescrits par un arrêté pris en application d’un arrêt de péril, d’insalubrité ou de carence et ayant fait l’objet d’un vote en assemblée générale n’ont pu être réalisés du fait de la défaillance dudit copropriétaire, l’astreinte incombe alors directement au copropriétaire défaillant
		5. La dispense du copropriétaire qui a gagné son procès contre le syndicat des copropriétaires de participer à la dépense commune des frais de procédure, dont la charge est répartie entre les autres copropriétaires.
		Toutefois l’alinéa 3 de ce § b de l’article 10-1 permet au juge d’en disposer autrement :
		i. Soit en refusant de laisser à la charge du copropriétaire débiteur « les dépenses nécessaires » au recouvrement des charges,
		ii. Soit en refusant de dispenser le copropriétaire de sa participation aux frais du syndicat.
		En application de cette disposition, il a été jugé que :
		-Le copropriétaire dont les prétentions ont été déclarées fondées, doit, par application des dispositions de l’article 10-1 de la loi, être déclaré dispensé de toute participation à la dépense commune des frais et dépens générés par la procédure410.
		-Compte tenu de l’équité et de la situation économique du Syndicat, il n’y a pas lieu de dispenser le copropriétaire de toute participation à la dépense commune des frais de procédure411.
	
\section{SECTION III. LES CHARGES ENTRAINEES PAR LES SERVICES COLLECTIFS ET LES ELEMENTS D 'EQUIPEMENT COMMUNS.}
	
	Il s’agit des charges le plus souvent dénommées « charges spéciales », bien que cette dénomination soit impropre.
	
	\subsection{A. DEFINITION DES CHARGES DE L’ARTICLE 10 ALINEA 1}
	
		410 CA Versailles 12 juin 2008 JD 2008-379385
		411 CA Versailles 12 octobre 2009 \no 08/02944, JD 2009-379392
		droit de la copropriété année 2019-2020
		330
		Les services collectifs sont les services réalisés dans l'intérêt de la copropriété par des personnes rémunérées à cet effet. Cependant nous observerons que ce domaine a tendance à se rétrécir avec l'abondante jurisprudence des années 90 qui a classé le gardiennage en charges communes générales. 412
		Les éléments d'équipement communs sont les installations413 réalisées dans le but de faciliter ou d'améliorer la vie des occupants de l'immeuble.
		Donnons quelques exemples habituellement retenus par la jurisprudence ou par les auteurs :
		- Ascenseur et monte-charge.
		- Tapis de l'escalier.
		- Chauffage central de l'immeuble et canalisations.
		- Production d’eau chaude, eau froide.
		- Conditionnement d'air et climatisation.
		- Compresseur d'eau.
		- Antennes collectives de la télévision.
		- Interphones et digi-clés.
		- Vide ordures.
		- Vidéophones
		- Alarmes.
		- Trappes de désenfumage.
		- Equipement du local de surveillance d'un IGH
		Ce classement n'est pas toujours évident : par exemple, une clause du Règlement de Copropriété peut classer un tapis d'escalier dans les parties communes (et non dans les éléments d'équipement), en sorte que les charges seront réparties en charges générales. 414
		Sont charges relatives aux éléments d'équipement collectifs, non seulement les charges concernant les équipements eux-mêmes, mais encore
		- celles qui ont pour objet leur remplacement ou l'entretien des accessoires de ces éléments d'équipement.
		412 cf notamment Civ 3\degres 11/0 6/92, Loyers et Copropriété octobre 1992 \no 402..
		413 Monsieur VIGNERON (J.-Cl. Copropriété fasc. 70 \no 59 écrit « éléments autres que les installations immobilières ».
		414 Civ. 3\degres 10 mai 1994; Rev. Dr. Imm. 94.489
		droit de la copropriété année 2019-2020
		331
		C'est ainsi qu'a été jugé que la grille de protection de l'ascenseur ne devait pas être répartie comme les charges d'escalier (en charges générales) mais comme les charges d'ascenseur (en charges afférentes à un élément d'équipement commun):415
		- les énergies consommées (fuel, gaz, électricité pour le chauffage), et l'eau lorsqu'elles ne sont pas comptabilisées "individuellement",
		- les frais du personnel spécialement affecté ou des entreprises extérieures payées pour l'entretien de ces éléments d'équipement entreront dans la catégorie des dépenses relatives à ces éléments d'équipement.
		Par exemple si l'ensemble en Copropriété emploie un chauffagiste, les charges afférentes à ce dernier seront réparties comme les charges de chauffage.
	
	\subsection{B. MODALITES DE REPARTITION DES CHARGES SPECIALES (ARTICLE 10 AL 1)}
	
		Article 10 Modifié par Ordonnance \no2019-1101 du 30 octobre 2019 - art. 9 Les copropriétaires sont tenus de participer aux charges entraînées par les services collectifs et les éléments d'équipement commun en fonction de l'utilité objective que ces services et éléments présentent à l'égard de chaque lot, dès lors que ces charges ne sont pas individualisées.
		L’Ordonnance n’a pas changé substantiellement le texte, sauf en ajoutant, à la fin « dès lors que ces charges ne sont pas individualisées ». Ainsi, désormais, ces charges, s’il s’agit d’une consommation « chiffrable » doivent être réparties prioritairement « au compteur ».
		Qu'est-ce que l'utilité ? C'est l'utilité abstraite (in abstracto), c'est à dire envisagée par rapport au lot et non par rapport à l'occupant. Il s'agit donc d'un critère objectif et non subjectif.
		Par exemple, le chauffage collectif présente une utilité pour le lot qui est desservi, alors même que l'occupant est absent pendant la période de chauffe.
		Par exemple, l'ascenseur présente la même utilité pour deux lots d’habitation bourgeoise de même superficie situés au même étage, alors même que le premier des deux appartements est occupé par une famille nombreuse et le second par un célibataire.
		Par contre, le chauffage collectif ne présente aucune utilité pour l'appartement qui n'en bénéfice pas (ou à tout le moins qui ne dispose pas de réservation pour y être raccordé) et l'ascenseur ne présente aucune utilité pour l'appartement du rez-de-chaussée (à moins que l’ascenseur desserve les caves).
		415 Civ. 3\degres 19 février 1976, JCP 77 II \no 18525.
		droit de la copropriété année 2019-2020
		332
		Même si elle répond à des critères objectifs, l'utilité de chaque élément d'équipement sera fonction de cet élément. Il convient donc d'étudier les méthodes de répartition habituellement admises pour les principaux éléments d'équipement :
		1. Les charges d'ascenseur.
		Les spécialistes admettent que l'utilité de l'ascenseur est proportionnelle à l'étage, à la capacité d'occupation des lots et à leur destination (habitation, professionnelle ou commerciale).
		Ces techniciens spécialistes proposent donc l'application de :
		Un coefficient d'étage
		La progression était comprise entre 1 et 2 pour Morand et entre 1 et 2,5 pour Lucien Arnaud416 s'agissant d'un immeuble de 5 étages. Mais les auteurs plus "récents" ont tendance à « écraser » davantage l'échelle du coefficient d'étage.
		Dans leur étude, MM AZAIS, GOUVERNAIRE et MORAND417 après avoir rappelé qu'il n'existait aucune échelle des coefficients d'étage scientifiquement exacte, ont proposé une distinction subtile, selon que l'escalier est ou non apparent, car dans le cas où l'accès à l'escalier n'est pas apparent l'ascenseur sera systématiquement utilisé, même pour accéder au premier étage !
		Pour les immeubles de grande hauteur, les mêmes auteurs proposent une répartition normale pour les dix premiers étages et au delà une majoration de 0,05 entre le onzième et le quinzième étage et de 0,01 au delà.
		Mr QUIGNARD418 propose un coefficient d’étage très différent à l’issue d’une étude extrêmement savante faisant entrer notamment dans son mode de calcul l’économie de temps obtenue par l’utilisateur en prenant l’ascenseur plutôt qu’en montant à pied :
		R.C. = 0 ; 1er = 1 ; 2\degres = 1,33 ; 3\degres = 1,67 ; 4\degres = 2 ; 5\degres = 2,33 ; 6\degres = 2,67.
		La pratique retient le plus souvent une progression de 0,27 par niveau.
		416 Cf. ARNAUD et BOUYEURE, Millièmes de charges de copropriété, 1982.
		417 Répartition des dépenses de chauffage et d’entretien AJPI 1993, p. 235
		418 In Ascenseur et Copropriété. M Bernard QUIGNARD, expert judiciaire honoraire, fait autorité pour tout ce qui concerne les questions d’ascenseur.
		droit de la copropriété année 2019-2020
		333
		Un coefficient d'occupation :
		Un appartement de 4 pièces est prévu habitable par 5 personnes. Selon MM AZAIS, GOUVERNAIRE et MORAND, nous noterons les coefficients suivants,
		1 pièce entre 1 et 1,50 selon sa superficie. 5 pièces = 4
		2 pièces = 2 6 pièces = 4,50
		3 pièces = 3 7 pièces = 5
		4 pièces = 3,50 8 pièces = 5,50
		Un coefficient de destination
		Par exemple, 1 pour les locaux d'habitation, 2 pour les locaux professionnels et 3 pour les locaux commerciaux.
		Quelle que soit en définitive la répartition retenue, nous n'oublierons pas qu'il s'agit ici de répartir les frais afférents à l'entretien ou au remplacement d'un ascenseur existant; par contre, lorsqu’il s'agit d'installer un ascenseur dans un immeuble qui en est dépourvu, il faudra faire application d'autres critères qui seront étudiés avec les décisions d'Assemblée Générale afférentes aux travaux d'amélioration.419
		2. Le chauffage central.
		a. Appréciation de l’utilité du chauffage pour un lot
		La première question posée est de savoir à partir de quel moment le chauffage collectif est utile pour un lot. Selon l’expression de CAPOULADE, il faut distinguer entre « l’utilité virtuelle » (lorsque le chauffage dessert ou peut desservir un lot) et l’avantage « résiduel et fortuit ». 420.
		1ERE HYPOTHESE : CHAUFFAGE COLLECTIF DANS LES SEULES PARTIES COMMUNES.
		Dans ce cas il existe bien un chauffage collectif mais il ne dessert que les parties communes (hall d’entrée, escalier, loge de concierge...) à l’exclusion des parties privatives qui bénéficient d’un chauffage individuel.
		419 Les travaux d’installation doivent être répartis en fonction de l’avantage que l’installation présente pour chaque lot, alors que les frais de fonctionnement, entretien, réparation, remplacement, seront répartis conformément au critère d’utilité que nous venons de définir.
		420 Cf. Zurfluh, Lebatteux, Barnier Le chauffage collectif dans les immeubles en copropriété : lots non raccordés, lots débranchés, et installations de chaudières collectives; Administrer juillet 1987 p. 16 et s.
		droit de la copropriété année 2019-2020
		334
		Les charges afférentes à ce chauffage collectif limité devront être réparties en charges générales de copropriété puisqu’il s’agit de frais d’entretien et de gestion de la copropriété dans son ensemble.
		2\degres HYPOTHESE : LOT NON DESSERVI PAR LE CHAUFFAGE COLLECTIF MAIS « RACCORDABLE »
		Il s’agit du cas assez fréquent où il existe une « attente » pour le raccordement du chauffage que le copropriétaire n’a pas utilisé, préférant pour convenances personnelles ne pas raccorder son lot (la poissonnerie ne veut pas du chauffage de l’immeuble). En ce cas le copropriétaire a l’utilité objective du chauffage.
		3\degres HYPOTHESE : LOT NON DESSERVI MAIS UN PIQUAGE SUR CANALISATION EST POSSIBLE.
		Les tuyauteries de chauffage passent dans l’appartement, mais il n’y a pas de « réservation » pour un piquage ultérieur. Même si par le passé la Cour de Cassation a pu considérer que dans un tel cas le chauffage était objectivement « utile » pour le lot421, on doit aujourd’hui retenir une position contraire telle qu’affirmée dans un arrêt de 1994422 selon lequel il y a contradiction pour un arrêt de cour d’appel à constater que l’équipement commun de chauffage n’est d’aucune utilité pour un lot et à retenir, cependant, qu’il bénéficie de calories à partir des colonnes montantes de l’immeuble.
		4\degres HYPOTHESE : LE RACCORDEMENT NECESSITE DES TRAVAUX IMPORTANTS
		La réponse est ici beaucoup plus évidente : le chauffage n’a pas d’utilité pour le lot.423
		5\degres HYPOTHESE : LE DEBRANCHEMENT DU CHAUFFAGE COLLECTIF.
		Le copropriétaire considère que le chauffage central est vétuste; il installe son propre système de chauffage pour ses parties privatives et se débranche du chauffage collectif.
		Relevons que le fait de se débrancher implique la réalisation de travaux sur les parties communes et nécessite une autorisation de l’Assemblée générale; autorisation que les copropriétaires devront donner avec circonspection, car de nature à compromettre l’équilibre de l’installation424.
		421 Civ. 3\degres 8 oct 1985 : Rev. Dr. imm. 1986 p 103.
		422 Civ 3\degres 10 mai 1994 (Bd de Charonne) Dalloz 1996, Somm p. 159.
		423 Paris 23\degres ch. 20 sep 1985; Rev. Dr. Imm. 1986 p 103.
		424 En ce sens PARIS 23\degres Ch 9 mai 2001 Loyers et Copropriété oct 2001 \no 240
		droit de la copropriété année 2019-2020
		335
		Le copropriétaire ayant fait le choix de se débrancher, sauf dispense votée par la copropriété,425 devra continuer de payer sa quote-part des frais de chauffage collectif, qui conserve une utilité objective en ce qui le concerne.426
		6\degres HYPOTHESE : COPROPRIETAIRE RELIE QUI N’UTILISE PAS SON LOT.
		Ici encore, réponse évidente : l’utilité objective existe et le copropriétaire est mal venu de prétendre se voir exonérer de sa quote-part de charges au motif que pendant la saison de chauffe il demeurait sous des cieux plus cléments.427
		b. Les méthodes proposées.
		Plusieurs méthodes ont été proposées pour répartir les charges de chauffage de l'immeuble; en outre la question s'est trouvée compliquée par les dispositions légales qui ont été adoptées ensuite du premier "choc pétrolier" de 1973 :
		REPARTITION A LA SURFACE DE CHAUFFE :
		Il s'agit de la répartition en fonction de la surface des radiateurs.428
		Méthode inéquitable, car ne tenant pas compte de la situation des locaux : pour avoir une température identique dans un appartement sous terrasse, à la température d'un appartement à mi hauteur de l'immeuble il faudra davantage de radiateurs dans le premier cas.
		REPARTITION A LA SURFACE CHAUFFEE :
		On tient compte de la surface du lot chauffé en excluant la hauteur des locaux.
		Ce système a été reconnu conforme par la jurisprudence.429 Il est vrai que dans la plupart des immeubles modernes la hauteur sous plafond est identique dans tous les appartements, en sorte que ce qui importe ce n'est pas le volume, mais la surface chauffée.
		425 A l’unanimité selon la Cour de Paris 26 fév 1987; D. 87 IR p67. 426 Civ. 3\degres, 26 octobre 1983; Administrer nov 1984 p. 39 ; 3e civ., 14 déc. 2004 : Administrer avr. 2005, \no 376, somm. p. 35, obs. A. Alfandari).
		427 Civ. 3\degres 19 juin 1979; D 80 IR 238.; voir également QUIGNARD : Un copropriétaire doit-il payer les charges d’ascenseur et de chauffage en cas d’inoccupation de ses locaux; gaz. pal. 31 janvier 1987 p. 6.
		428 Validée par PARIS 23\degres 6 juillet 1994 Loyers et Copropriété dec 94.
		429 Civ. 18 nov. 1980; Gaz. Pal. 81 panorama 94
		droit de la copropriété année 2019-2020
		336
		REPARTITION AU VOLUME CHAUFFE :
		On tient compte du volume chauffé, c'est à dire du volume des pièces composant le local et ceci indépendamment du nombre de radiateurs. C'est l'application de la théorie du « confort thermique fourni à chaque lot ». Cette théorie a été consacrée par la jurisprudence430. "Être utile en ce qui concerne le chauffage collectif, consiste à fournir à chaque lot, suivant sa destination, un confort thermique convenable et identique".
		REPARTITION... EN MILLIEMES GENERAUX :
		Selon cette théorie les millièmes généraux tiennent compte des trois critères (superficie, consistance et situation) qui doivent être pris en compte pour déterminer l'utilité objective du chauffage à l'égard de chaque lot. 431
		Cette méthode s'approche finalement du volume chauffé (consistance du lot). Certes il peut paraître surprenant que la notion d'utilité que le législateur a voulu distincte de la notion de valeur relative des lots, finisse par se confondre avec cette valeur.
		Après avoir marqué son hésitation pour valider un tel système : Favorable dans un arrêt ancien432 mais ayant pris position plus récemment en sens contraire433, la Cour de Cassation a finalement admis une répartition aux tantièmes généraux434 :
		"Attendu qu'ayant relevé que le règlement de copropriété stipulait que les frais de chauffage sont répartis "au prorata des millièmes", à moins qu'il existe des compteurs individuels et constaté que le système de comptage mis en place était inadapté, la cour d'appel a légalement justifié sa décision en retenant que le mode de répartition des frais de chauffage à proportion des quotes-parts de parties communes était celui qui répondait le mieux au critère de l'utilité objective prévu par la loi".
		REPARTITION AU "COMPTEUR DE CALORIES".
		Un compteur calcule et enregistre le nombre de calories consommées. Système très séduisant en apparence mais qui privilégie l'utilisation effective du chauffage collectif par le lot et non plus l'utilité "objective" du chauffage pour chaque lot. Ce système ne paraît pas dès lors devoir être retenu.
		430 Civ. 3\degres, 23 mai 1978, Administrer oct 78 p. 37 ; Paris 23\degres Ch 29 sep 2005, Loyers et Copropriété 2006, comm 20
		431 Cf. Atias, La répartition des charges de chauffage en fonction des millièmes de copropriété. Annale des Loyers 1994 p. 525.
		432 Civ. 3\degres 8 fév 1977, JCP 79 G II 19068 note Atias.
		433 Civ 3\degres 18 déc. 1991 Bull. Civ. \no 324 p 191.
		434 Civ. 3\degres 9 juin 1993 (IRC 1994 \no 384 p. 415, note Capoulade) ; Civ 3\degres 9 nov 1994 Loyers et Copropriété fev 95 \no 80.
		droit de la copropriété année 2019-2020
		337
		En conclusion sur les différentes méthodes de calcul, nous dirons avec MM STEMMER et LAFOND qu’il n’y a pas de principe absolu, le juge déterminant au cas par cas, si la méthode retenue, au regard des caractéristiques de l’immeuble, est ou non conforme au critère de l’utilité.
		c. Economies d’énergie et répartition des charges de chauffage.
		LES TEXTES EN VIGUEUR SUR LES ECONOMIES D'ENERGIE.
		Depuis 1974 des textes successifs ont été adoptés pour favoriser les économies d'énergie :
		Une loi du 29 octobre 1974 (portant RTE Réglementation Thermique a posé le principe de l'installation dans tous les immeubles collectifs pourvus d'un chauffage commun de compteurs de chaleur et d'eau chaude. Cette loi a été reprise dans l'article L 131-3 du code de construction et d'habitation435, avant de devenir l’article L 241-9 CCH qui dans sa rédaction issue de la loi du 8 novembre 2019 (ENERGIE et CLIMAT) complétée par l’Ordonnance \no 2020-71 du 29 janvier 2020 est ainsi rédigé :
		« Tout immeuble collectif d'habitation ou mixte pourvu d'une installation centrale de chauffage doit comporter, quand la technique le permet, une installation permettant de déterminer et de réguler la quantité de chaleur et d'eau chaude fournie à chaque local occupé à titre privatif. Tout immeuble collectif d'habitation ou mixte pourvu d'une installation centrale de froid doit comporter, quand la technique le permet, une installation permettant de déterminer et de réguler la quantité de froid fournie à chaque local occupé à titre privatif. Le propriétaire de l'immeuble ou, en cas de copropriété, le syndicat des copropriétaires représenté par le syndic s'assure que l'immeuble comporte des installations répondant à ces obligations. Nonobstant toute disposition, convention ou usage contraires, les frais de chauffage, de refroidissement et de fourniture d'eau chaude mis à la charge des occupants comprennent, en plus des frais fixes, le coût des quantités de chaleur et de froid calculées comme il est dit ci-dessus. Un décret pris en Conseil d'Etat fixe les conditions d'application du présent article, et notamment la part des frais fixes visés au précédent alinéa, les délais d'exécution des travaux prescrits, les caractéristiques techniques et les fonctionnalités des installations prévues au premier alinéa ainsi que les cas et conditions dans lesquels il peut être dérogé en tout ou partie aux obligations prévues au même premier alinéa, en raison d'une impossibilité technique ou d'un coût excessif au regard des économies attendues. Lorsqu'il n'est pas rentable ou techniquement possible d'utiliser des compteurs individuels pour déterminer la quantité de chaleur, des répartiteurs des frais de chauffage individuels sont utilisés pour déterminer la quantité de chaleur à chaque radiateur, à moins que l'installation de tels répartiteurs ne soit pas rentable ou ne soit pas techniquement possible. Dans ces cas, d'autres méthodes rentables permettant de déterminer la quantité de chaleur fournie à chaque local occupé à titre privatif sont envisagées. Un décret en Conseil d'Etat précise le cadre de mise en place de ces méthodes ».
		435 L’article L 313-1 CCH est abrogé et remplacé par les foyers de termites à compter du 1er juillet 2021
		droit de la copropriété année 2019-2020
		338
		Le but d'une telle disposition est bien évidemment de faire payer par le copropriétaire du lot l'énergie qu'il consomme. Aussi le même article L 131-3 CCH se poursuit par les dispositions suivantes :
		Concrètement les charges de chauffage doivent être réparties conformément aux dispositions de l’article R 241-13 du code de l’énergie :
		Les frais de combustible ou d'énergie sont répartis entre les locaux desservis en distinguant des frais communs et des frais individuels. Les frais communs de combustible ou d'énergie sont obtenus en multipliant le total des dépenses de combustible ou d'énergie par un coefficient égal à 0,30. Dans le cas des immeubles pour lesquels des appareils de mesure tels que ceux visés à l'article R. 241-7 ont déjà été installés, le coefficient choisi entre 0 et 0,50 au moment de l'installation de ces appareils est conservé. Toutefois, l'assemblée générale des copropriétaires ou le gestionnaire d'un immeuble entièrement locatif peut remplacer le coefficient initial par le coefficient de 0,30. Les frais communs sont répartis dans les conditions fixées par le règlement de copropriété ou les documents en tenant lieu. Le total des frais individuels s'obtient par différence entre le total des frais de combustible ou d'énergie et les frais communs. Ce total est réparti en fonction des indications fournies par les appareils prévus à l'article R. 241-7, les situations ou configurations thermiquement défavorables des locaux pouvant être prises en compte.
		Nous exposerons dans le cadre du chapitre consacré aux Travaux du Syndicat des Copropriétaires les dispositions de l’article 24-9 de la loi du 10 juillet 1965 et ses décrets d’application les modalités d’installation des compteurs individuels, qui doivent être en principe des compteurs de chaleur (un par appartement lorsqu’existe une boucle indépendante pour chaque lot) ou, à défaut, des compteurs individuels d’énergie (sur chaque radiateur) et les cas dans lesquels les syndicats sont ou peuvent être dispensés d’une telle installation. De même nous verrons les obligations imposées au fournisseur d’énergie et au syndic quant aux informations dues aux copropriétaires sur leur propre consommation d’énergie, avec l’article 18-1 modifié à compter du 25 octobre 2020 (qu’il s’agisse d’ailleurs du chauffage, de l’ECS ou du froid).
		MODE DE REPARTITION DES CHARGES DE CHAUFFAGE
		I - Si l’immeuble est dépourvu de compteurs, soit parce que leur installation n’a pas encore été décidée, soit parce ces immeubles se trouvent dispensés de les installer, les charges sont réparties conformément aux dispositions du règlement de copropriété.
		II – Si l’immeuble est doté de compteurs de chaleur ou de compteurs individuels d’énergie : il convient de diviser les charges de chauffage (ECS et froid) en deux masses :
		• Les frais de combustible ou d’énergie (R 241-13 CCH)
		droit de la copropriété année 2019-2020
		339
		• Les autres frais de chauffage (conduite et entretien des installations, installations électriques pour le fonctionnement de appareillages, remplacement des installations), que l’on pourra qualifier de frais fixes. (Art. R 241-12 CCH)
		Les frais fixes seront répartis conformément aux dispositions du règlement de copropriété.
		Les frais de combustibles ou d’énergie, sont eux-mêmes divisés en deux sous-masse :
		Première sous masse : les frais communs
		Les frais de combustible ou d’énergie
		:
		- 30 % de la dépense totale de combustible ou d’énergie sont répartis conformément aux dispositions du Règlement de copropriété.
		Deuxième sous masse : les frais individuels
		- 70 % restant (par différence entre le total des frais de combustible ou d’énergie et les frais communs) sont répartis en fonction des compteurs de chaleur ou individuels d’énergie, calories avec possibilité de ventiler en fonction des configurations thermiques défavorables.
		Bien évidemment toutes informations sur cette division doivent être données par le fournisseur d’énergie et être transmises aux copropriétaires.
		1. Eau froide.
		Relevons qu’il ne faut pas confondre les charges d’eau froide avec les charges de « froid » qui font l’objet de dispositions calquées sur les dispositions afférentes au chauffage et à l’ECS.
		Les charges d'eau froide sont réparties selon les compteurs individuels lorsque ceux-ci ont été installés, proportionnellement au nombre de points d'eau ou encore en fonction de la capacité des lots. L'installation de compteurs d'eau froide divisionnaires436 relevait jusqu’à la loi ALUR des dispositions de l’article 25 m) de la loi du 10 juillet 1965. Désormais, renumérotation de l’article 25 oblige, cette installation relève de la majorité de l’article 25 k)
		436 Il ne faut pas confondre l’installation de compteurs individuels d’eau froide à partir du compteur général de l’immeuble du syndicat avec la demande « d’individualisation des contrats de fourniture d’eau et la réalisation des études et travaux nécessaires à cette individualisation » qui permettra au copropriétaire demandeur d’être directement livré par le concessionnaire et qui fait désormais l’objet des dispositions de l’article 25 o).
		droit de la copropriété année 2019-2020
		340
		Cependant, les tribunaux ont pu estimer conforme à l'article 10 alinéa 1er la répartition de l'eau froide aux tantièmes généraux en l'absence de compteurs individuels437.
		En sorte que lorsque le règlement de copropriété stipule qu’à défaut d’installation de compteur individuel les charges d’eau froide sont réparties en tantièmes généraux, le propriétaire de parking a une « utilité objective » de l’eau froide et ne peut prétendre être dispensé438
		Dans ce cas un copropriétaire ne saurait exiger de la Copropriété qu'en ce qui le concerne la répartition soit fonction de sa consommation au motif qu'il a fait installer de sa propre initiative un compteur individuel.439 C’est ce qu’a jugé la Cour de Paris :
		« Si le règlement de copropriété prévoit que les charges d’eau sont charges générales, le fait pour l’assemblée générale d’autoriser un copropriétaire à installer un compteur personnel ne le dispense pas de participer aux charges d’eau à titre de charges générales ».440
		Etant rappelé au demeurant qu'une telle installation, c'est à dire en fait son raccordement sur la conduite commune, nécessite une autorisation préalable de l'Assemblée Générale.
		La loi SRU et la loi sur l’eau (30 décembre 2006) permettent l'individualisation des contrats de fourniture d'eau, en sorte qu’un copropriétaire peut non seulement disposer de son propre compteur, mais encore « sortir » ses charges d’eau purement et simplement de la copropriété. Toutefois cette « désolidarisation » est difficile à réaliser ; elle nécessite une étude préalable qui ne peut être décidée en assemblée générale qu’à la double majorité de l’article 26 de la loi du 10 juillet 1965 (art. 26 d)
		2. Frais de suppresseur.
		Ils doivent être répartis de la même façon que les charges d’eau froide441.
		3. Les charges d'eau chaude.
		437 (Paris 30 septembre 1985 : Informations Rapides de la Copropriété nov. 1985 p. 223
		438 PARIS 23\degres 11 octobre 1989. D 1990 Somm p 131, PARIS 23\degres 17 jan 1996; ce dernier arrêt précisant cependant que le copropriétaire n’a pas rapporté la preuve de l’absence de point d’eau dans les garages.
		439 Civ. 3\degres 12 février 1986; JCP 86 IV 108);
		440 PARIS 8\degres 18 jan 1994, R. Dr. Imm. 94.296
		441 Paris 23\degres 28 octobre 1994, Loyers et Copropriété 1995 \no 240.
		droit de la copropriété année 2019-2020
		341
		Les charges d'eau chaude ont fait l'objet de la réglementation dans le cadre des dispositions destinées à réaliser des économies d'énergie
		Le décret du 19 juin 1975 a rendu obligatoire la pose de compteurs d'eau chaude pour les immeubles construits depuis le 30 juin 1975.
		Pour les immeubles construits antérieurement, ces compteurs d'eau chaude sont en principe obligatoires depuis le 15 septembre 1977, avec certaines dispenses lorsque leur installation peut s'avérer trop onéreuse ou lorsque plus de 15 % des points d'eau sont inaccessibles.
		Le même décret prévoit que la répartition dans les immeubles équipés de compteurs doit se faire en distinguant les frais de combustible ou d'énergie répartis entre les locaux au prorata de la mesure directe des quantités de chaleur nécessaires au chauffage de l'eau et les frais fixes qui continuent à être répartis selon les règles fixées par le règlement de copropriété.
		Pour les immeubles non équipés, le règlement de copropriété peut prévoir que ces charges sont réparties en tantièmes généraux442.
		Par contre, si l'assemblée générale décide de pose de compteurs individuels d’eau chaude conformément aux prescriptions légales, cette installation aura pour conséquence de modifier ipso facto la répartition des charges d’eau chaude sans qu’il soit besoin d’une disposition complémentaire d’assemblée générale modifiant le Règlement de copropriété sur ce point443
		4. Antennes collectives.
		L'utilité objective est la même pour chaque prise de raccordement à cette antenne. En conséquence si un copropriétaire bénéficie de deux prises (dans un duplex par exemple), il est conforme à l'utilité objective de lui attribuer deux parts d'entretien.
		De même peu importe l'affectation effective du lot : si celui-ci est desservi, le copropriétaire doit payer sa participation alors même qu'il n'a pas la télévision (local commercial raccordé par exemple).
		5. Vide-ordures.
		442 TGI Paris 8 nov 1976 D 1978 IR 124.
		443 Civ 3\degres Ch 17 nov 2004 – IRC 2006 \no 515 p 13
		droit de la copropriété année 2019-2020
		342
		Alors même que certains auteurs444 se sont interrogés sur l'opportunité de classer les frais d'entretien du vide-ordures en charges de l'une ou l'autre des catégories de l'article 10, il est certain que le vide-ordures est bien un élément d'équipement commun. Cependant, dès lors que tous les lots bénéficient de cet équipement, il apparaît que la répartition doit se faire en tantièmes généraux : l'élément d'équipement n'ayant pas plus d'utilité pour un lot que pour un autre.
		C'est la solution adoptée par la Cour de Cassation dans un arrêt de 1987 445 à propos non seulement du vide-ordures mais encore des charges d'eau et d'électricité afférentes à l'entretien des parties communes.
		Dans un arrêt plus récent la Cour de PARIS 446 a considéré que les dépenses engagées pour le nettoyage du vide-ordures sont des charges générales dès lors que celui-ci est situé, comme indiqué en l'espèce par le Règlement de Copropriété, dans les parties commune; en sorte que tous les copropriétaires devront y participer, tout comme aux frais de gardiennage, sans pouvoir faire valoir que le gardiennage et les frais de vide ordures ne sont pour eux d'aucune utilité.
		6. Portiers électroniques.
		Ici encore nous nous trouvons en présence d'élément d'équipement dont l'utilité est égale pour tous les lots. Dès lors, qu'il s'agisse d'une installation de claviers à codes de types « Digicode » ou encore d'un « Interphone» ou « Visiophone », la répartition la plus simple et la plus conforme au critère d'utilité sera la répartition en tantièmes généraux.
		On pourrait concevoir une répartition "à l'unité", mais il est évident que de tels éléments d'équipement ont plus d'utilité pour les grands locaux que pour les locaux de petite surface compte tenu du nombre d'occupants potentiel (et en conséquence de visiteurs) que peut recevoir chaque lot.
		7. SERVICE DE SECURITE INCENDIE
		Il s’agit d’une question délicate : un SSI est un poste particulièrement lourd (équipement, personnel de sécurité, etc.) qui peut être imposé du fait qu’un seul bâtiment est placé en IGH alors que la copropriété comporte plusieurs bâtiments ou qu’il existe dans un bâtiment une galerie commerciale.
		Ce Service de Sécurité Incendie constitue à n’en pas douter un élément d’équipement doont les charges doivent être réparties en fonction de son utilité pour chaque lot.
		Pour autant toute la copropriété est susceptible de bénéficier de cette installation.
		444 Monsieur GUILLOT in La difficulté de classement des charges communes dans l'une ou l'autre des catégories définies par les alinéas 1 et 2 de l'article 10 de la loi
		445 Civ. 3\degres du 1er avril 1987, Administrer octobre 1987 p. 44, note Guillot
		446 PARIS 23\degres Ch. 01/07/94, Loyers et copr. janvier1995 \no 33
		droit de la copropriété année 2019-2020
		343
		C’est pourquoi plusieurs arrêts de cours d’appel et un arrêt de cassation ont estimé que le service étant rendu à la collectivité des copropriétaires, son coût peut être réparti en charges communes générales.447
	
\section{SECTION IV -LES CHARGES DES RESIDENCES SERVICES}
	
	Nous traiterons dans la deuxième partie du cours448 de la question spécifique des Résidences Services placées sous le statut de la Copropriété qui, outre les services habituellement fournis par les syndicats de copropriété, offrent à leurs occupants des « services spécifiques, notamment de restauration, de surveillance, d'aide ou de loisirs » (article 41-1 de la loi de 1965), c'est-à-dire des services spécifiques liés à la personne des occupants et non à l’existence des parties communes.
	La loi ENL du 13 juillet 2006 et le Décret d’application \no 2010-391 du 20 avril 2010 ont créé un véritable statut de ces Résidences Services soumises au statut de la Copropriété, qui constituent désormais les articles 41-1 à 41-5 de la loi du 10 juillet 1965 et les articles 39-2 à 39-7 du Décret du 17 mars 1967.
	Ces services spécifiques peuvent être directement fournis par le Syndicat des Copropriétaires lui-même ou « externalisés », c'est-à-dire confiés contractuellement à des tiers.
	La loi ENL a été modifiée par la loi \no2015-1776 du 28 décembre 2015 - art. 14 qui est entrée en vigueur le 28 juin 2016 et uniquement pour les Résidences avec Services dont le Règlement de copropriété a été publié à compter du 1er juillet 2016. En sorte que nous avons deux régimes distincts pour les Résidences Services (sauf décision de l’assemblée générale de modifier le Règlement de copropriété à la double majorité de l’article 26 pour le mettre en conformité avec la nouvelle loi)449
	L’article 41-1 modifié définit ainsi la répartition des charges des Résidence Services : Le règlement de copropriété peut étendre l'objet d'un syndicat de copropriétaires à la fourniture aux résidents de l'immeuble de services spécifiques dont les catégories sont précisées par décret et qui, du fait qu'ils bénéficient par nature à l'ensemble de ses résidents, ne peuvent être individualisés. Les services non individualisables sont fournis en exécution de conventions conclues avec des tiers. Les charges relatives à ces services sont réparties en application du premier alinéa de l'article 10. Les charges de fonctionnement constituent des dépenses courantes, au sens de l'article 14-1.
	Désormais, et pour les nouvelles résidences-services il convient de distinguer deux catégories de services :
	447 cf. Cour d'Appel Paris, 23\degres Ch. B, 4 décembre 2003, Loyers et Copropriété 2004, \no 76 ; Paris 23\degres Ch. B 15 jan 2004, Loyers et Copropriété 2004, \no 280 ; civ. 3\degres Ch. 4 février 2009, Administrer oct 2009 p. 55 ; pour un agent de sécurité incendie : Paris, Cour d'Appel Paris, Pole 4, Chambre 2, Chambre 2, 20 janvier 2016 ; Administrer oct 2016 \no 502, p. 51
	448 Polycopié \no 2 – Chapitre VIII – Les Résidences Services
	449 La loi ALUR oblige le syndic a inscrire chaque année à l’ordre du jour de l’assemblée générale cette adaptation du Règlement de copropriété aux nouvelles dispositions des articles 41-1 à 41-6.
	droit de la copropriété année 2019-2020
	344
	- Les services spécifiques non individualisables
	- Services spécifiques individualisables
	
	\subsection{. SERVICES SPECIFIQUES NON INDIVIDUALISABLES}
	
		La liste est donnée par le décret \no 2016-1446 du 26 octobre 2016 (article 39-2 du Décret du 17 mars 1967) et qui concerne :
		1\degres L'accueil personnalisé et permanent des résidents et de leurs visiteurs ; 2\degres La mise à disposition d'un personnel spécifique attaché à la résidence, le cas échéant complétée par des moyens techniques, permettant d'assurer une veille continue quant à la sécurité des personnes et la surveillance des biens ; 3\degres Le libre accès aux espaces de convivialité et aux jardins aménagés.
		Le syndicat ne peut plus les fournir lui-même ces services (il ne peut plus par exemple avoir d’hôtesse d’accueil salariée). Il doit passer une convention avec une société de prestataires de service qui mettra le personnel nécessaire à sa disposition. Le coût de cette prestation extérieure payée par le syndicat la société de service sera réparti entre les copropriétaires en fonction du critère d’utilité objective de ces services (mais il est vrai que s’agissant des trois catégories énumérées ci-dessus, l’utilité objective est sans doute la même pour l’ensemble des propriétaires des lots).
	
	\subsection{B. LES SERVICES INDIVIDUALISABLES}
	
		Tout ce qui n’est pas service non individualisable ne peut exister qu’à titre de services individualisables. Ce peut être par exemple la restauration des résidents.
		Le syndicat ne peut plus fournir lui-même ces services dont le coût ne peut pas apparaître dans les charges copropriété.
		Par contre, le règlement de copropriété de l’immeuble peut prévoir que le syndicat pourra mettre des locaux communs à la disposition de prestataires de services tiers à la copropriété pour que ceux-ci réalisent ces prestations individuelles.
		Si le règlement de copropriété prévoit cette possibilité, l’assemblée générale décidera de passer une convention avec un ou plusieurs prestataires qu’elle aura choisis librement à la majorité de l’article 25 de la loi.
		Cette mise à disposition se fera à titre gratuit sous forme d’un prêt à usage ou commodat pour une durée qui ne peut excéder 5 années renouvelables.
		Resteront cependant à la charge du syndicat des copropriétaires les frais d’entretien et de fonctionnement de ces parties communes mises à disposition du prestataire de services. Le règlement de copropriété doit préciser le mode de répartition de ses frais d’entretien de fonctionnement des locaux entre les copropriétaires, mais l’opération sera neutre pour le syndicat des copropriétaires dans la mesure où le prêt à usage n’interdît pas de demander remboursement des charges afférentes au local concerné.
		
		Précisons enfin et cela résulte de la loi ENL que ces prestations de services individualisables ne peuvent pas porter sur les services de soins (pas de possibilité de contrat avec les infirmières ; par contre rien n’interdit à un résident de faire appel à une infirmière habitant à proximité de la résidence services450).
		-
		° Immeubles mixtes : Copropriété comportant certains lots en résidence services.
		Il arrive fréquemment que l’on trouve dans un même immeuble soumis au statut de la copropriété des locaux exploités en Résidence-hôtelière et des locaux exploités ou occupés directement par leurs propriétaires (le plus souvent à l’expiration du délai de 9 ans de LMP). En ce cas une société d’exploitation gère la résidence-tourisme. Pour autant, les règles de répartition de charges de la loi de 65 demeurent applicables, en sorte par exemple que le service d’accueil n’a pas d’utilité pour les copropriétaires qui n’ont pas confié la gestion de leurs lots à la Résidence-hôtelière451
		450 D’où la création de Résidences Services dans des immeubles divisés en deux volumes, le volume du rez-de-chaussée étant destiné à l’exercice de professions libérales (médecins, masseurs, infirmiers …) tandis que le volume correspondant aux étages sera à vocation de résidence-service.
		451 3° Ch. Civ. 18 février 2015, n° 13-27104, au Bulletin