\chapter{Les mesures conservatoires en vue du recouvrement des charges}

Le législateur a permis aux créanciers de contraindre les débiteurs défaillants à s'exécuter.
Cependant, afin de garantir les débiteurs des demandes intempestives, les « mesures exécutoires », telles que la saisie et la vente des biens appartenant au débiteur ne peuvent être mises en oeuvre qu’après vérification par le Juge de la régularité de la créance. C’est pourquoi les mesures d’exécution ne peuvent être entreprises qu’au vu d’un « titre exécutoire », c’est-à-dire d’un acte authentique (notarié), ou d’un jugement exécutoire.
Pour les créanciers ne disposant pas d’un tel titre exécutoire (ce qui est précisément le cas du syndicat des copropriétaires pour les appels de charges, jusqu’à obtention d’un jugement de condamnation sur le fond), la loi permet au créancier de prendre des mesures conservatoires avant obtention d'une condamnation par le juge, par exemple par l’inscription d’une hypothèque sur les biens immobiliers dont le débiteur est propriétaire. Le syndicat des copropriétaires dispose de cet arsenal de droit commun contre les copropriétaires récalcitrants.
Il bénéficie de plus de dispositions spécialement édictées à son profit par la loi du 10 juillet 1965 à titre conservatoire (article 19), pouvant être prises à tout moment (donc en l’absence de mutation du lot) à savoir :
- une hypothèque légale sur le lot du copropriétaire débiteur
- un privilège sur les meubles du copropriétaire débiteur
- un privilège sur les loyers dus au copropriétaire débiteur
En outre, depuis la loi du 21 juillet 1994, le syndicat bénéficie d'un privilège pour les charges échues sur le prix de vente du lot (art 20 de la loi du 10 juillet 1965) donc seulement en cas de mutation du lot à titre onéreux.
Cependant, ces mesures spécifiques sont seulement conservatoires : elles ne dispensent pas le syndic d'obtenir un titre exécutoire, c'est à dire un jugement définitif de condamnation qui permettra la remise au syndicat des copropriétaires des sommes qui auront été bloquées par les mesures conservatoires ou par la mise en oeuvre du privilège.(cf. chapitre V)
droit de la copropriété année 2019-2020
443
SECTION I - MESURES CONSERVATOIRES EN L'ABSENCE DE MUTATION
Le syndicat bénéficie des mesures conservatoires de droit commun lui permettant de se garantir avant condamnation (A).
La loi du 10 juillet 1965 donne de plus au syndicat de copropriété la faculté d'une part de prendre une hypothèque légale562 sur le lot (B) et d'autre part, un privilège mobilier sur les meubles meublants ou sur les loyers perçus par le copropriétaire débiteur des charges de copropriété (C).
Les dispositions spécifiques à la copropriété font l’objet de l’article 19 de la loi du 10 juillet 1965 :
- L’hypothèque légale résulte des termes des alinéas 1 à 4
- La saisie des meubles meublants et des loyers est prévue à l’alinéa 5
A. MESURES CONSERVATOIRES DE DROIT COMMUN.
En dehors des mesures spécialement édictées par le législateur de 1965 pour protéger les intérêts de la copropriété, le syndicat peut recourir à l'ensemble des dispositions de droit commun :
En effet aux termes de l'article 67 de la loi du 9 juillet 1991
" Toute personne dont la créance paraît fondée en son principe peut solliciter du juge l'autorisation de pratiquer une mesure conservatoire sur les biens de son débiteur, sans commandement préalable, si elle justifie de circonstances susceptibles d'en menacer le recouvrement. La mesure conservatoire prend la forme d'une saisie conservatoire ou d'une sûreté judiciaire".
En pratique le syndicat des copropriétaires sollicitera du juge de l'Exécution du domicile du débiteur, par voie de requête, de l'autoriser à pratiquer saisie-conservatoire de créance (c'est l'ancienne saisie-arrêt) sur le compte bancaire du copropriétaire ou saisie conservatoire de biens mobiliers ou encore à inscrire une hypothèque sur un immeuble qui ne constitue pas un lot de la Copropriété.
562 Une hypothèque est dite conventionnelle (prévue par le contrat) judiciaire (autorisée par le juge) ou légale (la loi autorise la prise d’hypothèque.
droit de la copropriété année 2019-2020
444
Un principe général du droit veut que les lois spéciales dérogent aux lois générales (specialia generalibus dérogant). On peut se demander si l'article 62 du Décret (qui donne compétence exclusive au juge de situation de l'immeuble pour tous les litiges nés de l'application de la loi du 10 juillet 1965) n’emporte pas dérogation au texte général de compétence en matière de saisie attribution, en sorte que la requête devrait être présentée au Juge de l'Exécution du lieu de situation de l'immeuble et non au Juge de l'Exécution du domicile du copropriétaire débiteur. C'est ce qui avait été admis, dans le cadre de l'ancien code de procédure civile, en matière de saisie-arrêt.
1. La saisie conservatoire de créance.
Une fois obtenue l'autorisation du juge de saisir un montant déterminé, le syndicat dispose d'un délai de trois mois pour pratiquer cette saisie entre les mains du tiers (généralement le banquier du copropriétaire débiteur).
Le syndicat devra impérativement dénoncer cette saisie-conservatoire au copropriétaire débiteur dans les huit jours suivants la saisie.
Dans le mois suivant la saisie, le syndicat doit assigner le copropriétaire pour avoir sa condamnation au paiement (devant le Tribunal du lieu de situation de l'immeuble). Dans les huit jours de cette assignation le syndicat devra notifier au tiers (le banquier dans notre exemple) une copie de cette assignation.
Les délais de trois mois, huit jours, un mois et huit jours sont particulièrement importants : le non-respect d'un seul de ces quatre délais entraîne la caducité de la mesure conservatoire.
Par ailleurs, le copropriétaire débiteur pourra à tout moment saisir le Juge de l'Exécution qui a autorisé la mesure conservatoire pour lui demander de prononcer la mainlevée de la mesure de saisie en exposant que la créance n'était pas fondée dans son principe ou que son recouvrement n'était pas menacé ou encore en invoquant une nullité ou une caducité de la procédure.
Lorsque le syndicat aura obtenu condamnation, il pourra convertir la saisie-conservatoire en saisie-attribution.
2. La saisie conservatoire de biens mobiliers.
Il peut s'agir d'une saisie des meubles détenus par le débiteur lui-même (par exemple les meubles meublant son appartement, ou un véhicule) ou par un tiers pour le compte du copropriétaire débiteur.
Après obtention d'une condamnation le syndicat transformera la saisie conservatoire en saisie vente, après commandement de payer.
droit de la copropriété année 2019-2020
445
3. L’inscription d'une sûreté.
Le syndicat peut, par exemple, se faire autoriser à nantir le fonds de commerce du copropriétaire débiteur qui exploite un commerce ; nantir les parts sociales de la SCI copropriétaire d'un lot, ou encore inscrire une hypothèque sur un bien qui ne dépend pas du syndicat poursuivant. Dans ce dernier cas l'inscription se fera de la même façon qu'indiqué précédemment à propos de la mise en oeuvre de l'hypothèque légale de l'article 19 de la loi. Par contre il devra dénoncer cette inscription dans le délai de huit jours au copropriétaire débiteur et assigner dans le mois.
B. L’HYPOTHEQUE LEGALE SUR LE LOT (ARTICLE 19 ALINEAS 1 A 4)
Les créances de toute nature du syndicat à l'encontre de chaque copropriétaire sont, qu'il s'agisse de provision ou de paiement définitif, garanties par une hypothèque légale sur son lot. L'hypothèque peut être inscrite soit après mise en demeure restée infructueuse d'avoir à payer une dette devenue exigible, soit dès que le copropriétaire invoque les dispositions de l'article 33 de la présente loi.
Le syndic a qualité pour faire inscrire cette hypothèque au profit du syndicat; il peut valablement en consentir la mainlevée et requérir la radiation, en cas d'extinction de la dette, sans intervention de l'Assemblée Générale.
Le copropriétaire défaillant peut, même en cas d'instance au principal, sous conditions d'une offre de paiement suffisante ou d'une garantie équivalente, demander mainlevée totale ou partielle au président du Tribunal de Grande Instance statuant comme en matière de référé.
Aucune inscription ou inscription complémentaire ne peut être requise pour des créances exigibles depuis plus de cinq ans.
1. Les conditions de l’hypothèque légale
LES CREANCES SUSCEPTIBLES D’ETRE GARANTIES
Les créances de toute nature, c'est à dire toute créance que le syndicat peut avoir contre une copropriété (charges ou exécution d'une condamnation judiciaire par exemple). De même cette hypothèque peut être prise non seulement pour les charges échues mais également pour les paiements dus à titre provisionnel.
droit de la copropriété année 2019-2020
446
Créances exigibles : bien que le texte ne le précise pas l'inscription n'est possible que pour les sommes exigibles, sauf hypothèse de l'inscription prise en application de l'article 33 de la loi de 1965 ci-après évoquée.
Créances de moins de cinq ans : L’inscription d'hypothèque n’est pas possible que pour les créances exigibles depuis plus de cinq ans563.
Créance résultant des dispositions de l’article 33 de la loi : En ce cas, il ne s'agit plus de sommes dues par le copropriétaire et qu'il a omis de régler, mais simplement d'une mesure conservatoire que le syndic a l’obligation d’inscrire sur le lot d'un copropriétaire qui a demandé à bénéficier de la faculté de paiement en dix annuités de travaux d'amélioration décidés par l'assemblée générale
LA MISE EN DEMEURE PREALABLE A L’INSCRIPTION DE L’HYPOTHEQUE
Deux exigences :
- Une mise en demeure préalable : c'est à dire que cette inscription ne peut être prise qu'après une mise en demeure. Antérieurement au Décret du 21 octobre 2015 cette mise en demeure devait être faite par exploit d'huissier (ancien article 63 du décret). Depuis le 1er avril 2016, cette mise en demeure peut-être faite valablement par lettre recommandée ou par voie électronique.
- Une Mise en demeure restée infructueuse : ce qui signifie que l'inscription sera prise à l'expiration du délai fixé dans la mise en demeure (en général huit jours).
Relevons le cas particulier de la location accession soumise à la loi du 12 juillet 1984 : aux termes de l'article 32 de cette loi, la mise en demeure doit être notifiée au vendeur... alors même que le premier intéressé est en fait le locataire accédant.
LA MAINLEVEE JUDICIAIRE.
Le copropriétaire n'est pas informé directement de l'inscription de cette hypothèque légale : la loi ne fait aucunement obligation au syndic de lui dénoncer son inscription.
Au demeurant, le syndicat n'est pas tenu d'assigner au fond après la prise d'inscription de l'hypothèque légale.
563 Avec l’Ordonnance du 18 septembre 2019 ramenant le délai de prescription à 5 ans il y a désormais concordance entre les deux délais (article 42 et article 19).
droit de la copropriété année 2019-2020
447
Bien souvent, ce n'est que plusieurs mois, parfois plusieurs années, après l'inscription que le copropriétaire apprend l'existence de l'hypothèque.
Aussi, le législateur autorise-t-il le copropriétaire à demander à tout moment mainlevée judiciaire par assignation devant le Juge des référés : Cette demande pourra donc être formulée devant le juge des référés alors même qu'une procédure en paiement aura été préalablement introduite par le syndicat des copropriétaires.
Cette mainlevée sera totale ou partielle selon l'importance des paiements effectués par le copropriétaire débiteur.
2. Les Formalités matérielles de l'inscription.
LES RENSEIGNEMENTS NECESSAIRES A L’INSCRIPTION DE L’HYPOTHEQUE
Les articles 2146 et s. du Code civil traitent du mode de l'inscription des privilèges et hypothèques. Il serait fastidieux dans le cadre du présent cours de reprendre en détail ces dispositions, mais il faut être conscient qu'une inscription d'hypothèque qu'elle soit légale conventionnelle ou judiciaire, est toujours un acte délicat compte tenu de l'importance et de la nature des renseignements dont le créancier doit disposer :
Etat civil complet du ou des copropriétaires,
Désignation complète du lot,
Date d'acquisition du lot et références de publication,
Références du Règlement de Copropriété et date de publication
Références cadastrales de l'immeuble.
Ceci signifie en pratique la nécessité pour le syndic... ou pour l'avocat ou l’huissier désigné par le syndic pour inscrire l'hypothèque, d'obtenir préalablement différents renseignements du Cadastre et de la Conservation des Hypothèques.
3. Le coût d'inscription.
Très modeste : l'inscription est dispensée de la taxe de publicité foncière (art. 663 et 667 du Code Général des Impôts) et l'assiette du droit est égale à 0,05 % du montant de l'inscription et des frais.
droit de la copropriété année 2019-2020
448
4. Les effets de l'hypothèque légale.
Celle-ci est inscrite à sa date ; si l'immeuble vient à être vendu à la barre du Tribunal, le syndicat verra son privilège reporté sur le prix d'adjudication à hauteur des sommes pour lesquelles l'inscription a été prise et à condition toutefois que ce prix d'adjudication permette le règlement total ou partiel du syndicat à son rang.
L'inscription conserve l'hypothèque et le privilège pendant dix années à compter de la date d'inscription.
AVANTAGES
C'est une garantie peu onéreuse.
C'est une mesure très efficace si l'inscription est prise en rang utile et dans la mesure où les inscriptions antérieures n'absorbent pas la totalité du prix escompté en cas de vente judiciaire.
En pratique le syndic - ou l'avocat - lèvera systématiquement un état des inscriptions existantes sur le lot qui lui permettra de connaître le rang de l'inscription prise et l'importance des inscriptions qui précèdent.
INCONVENIENTS
La pratique révèle les difficultés matérielles de l'inscription et son inefficacité réelle parce que bien souvent, le syndicat n'arrive que derrière les principaux créanciers : prêteur des deniers nécessaires à l'acquisition du lot et Trésor Public, dont le montant des inscriptions est tel qu'il rend illusoire toute récupération de fonds au profit du syndicat.
Enfin, l'hypothèque ne pouvant être prise que pour les créances exigibles ne garantit pas le défaut de paiement postérieur à son inscription : en conséquence, des inscriptions complémentaires doivent être prises au fur et à mesure que le copropriétaire laisse de nouvelles charges non réglées.
Cependant, le syndic veillera à prendre une hypothèque sur le lot du copropriétaire défaillant car il a l'obligation, sous sa seule responsabilité, de mettre en oeuvre tous moyens pour parvenir à la récupération des charges dues et si le débiteur peut être soumis aux dispositions du Livre VI du Code de Commerce, le syndicat sera alors créancier inscrit et à ce titre devra être informé directement du jugement déclaratif.
B. LE PRIVILEGE MOBILIER SUR LES MEUBLES MEUBLANTS OU LES LOYERS DU LOT (ARTICLE 19 ALINEA 5)
droit de la copropriété année 2019-2020
449
Les deux derniers alinéas de l'article 19 de la loi accordent au syndicat un privilège portant :
soit sur les meubles du copropriétaire si celui-ci occupe lui-même le local ou l'a donné en location meublée,
soit sur le prix des loyers dus par le locataire lorsqu'il a donné ce local en location non meublée.
En d'autres termes, dans la première hypothèse le privilège du syndicat porte sur les meubles meublants, dans le second cas, il porte sur les loyers.
C'est une procédure peu onéreuse (il n'est pas nécessaire de recourir à l'avocat) et rapide (par opposition à l'inscription d'hypothèque légale). En contrepartie, cette mesure n'est efficace que si elle porte sur les loyers; par contre nombre de meubles meublants sont insaisissables de par la loi et leur valeur est dérisoire lors de la vente aux enchères.
SECTION II - LE PRIVILEGE SPECIAL DU SYNDICAT DES COPROPRIETAIRES EN CAS DE VENTE DU LOT
A. INTRODUCTION
Une importante réforme législative est intervenue le 21 juillet 1994 à l'occasion de la loi sur l'Habitat qui concerne les mesures conservatoires en cas de vente du lot
Le premier objectif du législateur en matière de copropriété dans le projet de loi relatif à l'Habitat a été d'apporter un remède aux "copropriétés dégradées". charges générées par le chômage et les difficultés économiques.
D'où l'insertion dans la loi sur l'Habitat d'une série de dispositions concernant le droit de la copropriété et comportant essentiellement deux volets :
l'amélioration du recouvrement des charges,
l'administration provisoire des copropriétés en situation de difficulté économique.
droit de la copropriété année 2019-2020
450
1. Le projet de loi d'origine prévoyait de conférer un caractère réel aux charges de copropriété.
Le projet de loi d'origine avait conçu de donner un caractère réel aux charges de copropriété de telle sorte que le débiteur des charges ne soit plus le copropriétaire mais le lot de copropriété, d'où la modification de l'article 20 de la loi dont le premier paragraphe aurait été ainsi rédigé :
"L'obligation de participer aux charges et aux travaux mentionnés aux articles 10 et 30 est attachée au lot et le suit en quelques mains qu'il passe".
En d'autres termes, le législateur créait une solidarité entre vendeur et acquéreur pour le paiement des charges.
Juridiquement, le fait de conférer aux charges de copropriété un caractère réel était irréprochable.
Pourtant, cette notion de caractère réel des charges de copropriété a été purement et simplement abandonnée lors de la discussion du projet de la loi devant le Sénat, sans que l'on sache très bien à l'initiative de qui est dû un tel revirement : aux Sénateurs ou au Gouvernement lui-même.
Parmi les critiques faites à ce projet dans sa rédaction initiale, nous relèverons celles de Me J.R. BOUYEURE 564:
- D'une part l'indétermination du prix de vente du bien puisqu'au prix principal s'ajoute le montant des charges dues qui n'est pas nécessairement connu lors de la signature de la promesse de vente.
- D'autre part : "celui d'exposer l'acquéreur au risque de payer deux fois (au créancier inscrit du vendeur et au syndicat)" les charges dues par le vendeur.
Toujours est-il que le texte adopté par le Parlement repose en définitive sur un fondement juridique totalement différent en instituant pour les charges de copropriété, un privilège immobilier spécial... dispensé d'inscription.
2. La création du privilège spécial occulte du Syndicat
Finalement, le parlement va retenir la création, au profit du Syndicat, d’un privilège spécial sur le prix de vente du lot :
- situé en tête de tous les privilèges immobiliers, primant même celui du prêteur de denier (la banque)
- et dispensé d’inscription !
564 in ADMINISTRER octobre 1993
droit de la copropriété année 2019-2020
451
Il est bien vrai que les banquiers ont fait le siège des parlementaires, qu'ils ont critiqué le caractère occulte du privilège, qu'ils ont considéré comme une régression du régime des sûretés réelles, ajoutant de façon plus pernicieuse que ce dispositif allait les contraindre à "être plus sélectifs dans l'attribution des prêts immobiliers, ce qui était de nature à freiner la reprise de la construction".
Monsieur MARTINON, rapporteur à l'Assemblée Nationale a répondu :
« Le privilège spécial vient garantir le recouvrement de créance née notamment pour entretenir un immeuble et donc, il lui permet de conserver une certaine valeur.
Faute d'entretien, cette valeur, gage de la banque, diminuerait énormément.
Il apparaît normal de réserver un traitement privilégié au syndicat par rapport au prêteur de deniers, à partir du moment où ce dernier sélectionne ses clients alors que la copropriété ne choisit pas ses copropriétaires.
Les pertes subies par un établissement financier peuvent être réparties sur l'ensemble de ses clients alors que celles d'une copropriété ne peuvent être réparties que sur un ensemble beaucoup plus réduit : les copropriétaires solvables.
Enfin, la période pendant laquelle le syndicat est préféré au prêteur de deniers peut être réduite si ce dernier se fait subroger dans les droits du vendeur ».
Le texte finalement adopté par le Parlement le 21 juillet 1994 va constituer un nouvel article 19-1 de la loi ainsi rédigé :
Ce privilège était toutefois limité aux sommes dues au titre des « charges et travaux » à l’exclusion des autres créances : Sont garantis par le privilège immobilier spécial prévu à l'article 2374 du code civil : l'obligation de participer aux charges et aux travaux mentionnés aux articles 10 et 30.
3. Un avenir menacé ?
Dans le projet de l’Ordonnance \no 2006-346 du 23 mars 2006 relative aux sûretés, inspirée directement des travaux de l’Association Henri CAPITANT sous la houlette du Professeur GRIMALDI, les privilèges spéciaux immobiliers disparaissaient.
Disparaissait ainsi le privilège du syndicat des copropriétaires pour ne laisser au syndicat des copropriétaires qu’une hypothèque légale sur l’immeuble pour les charges des quatre dernières années, venant après l’hypothèque légale du prix de vente et avant l’hypothèque légale du prêteur de deniers fournis pour l’acquisition de l’immeuble, cette hypothèque se conservant par son inscription à la diligence du syndic et prenant rang à la date de cette inscription.
droit de la copropriété année 2019-2020
452
Le privilège du syndicat des copropriétaires n’a été sauvé qu’in extremis (lors de l’examen du projet par le Conseil d'Etat.)
De plus, l’Ordonnance du 23 mars 2006 a créé une hypothèque conventionnelle rechargeable. On peut donc imaginer l’acquéreur du lot qui a consenti une hypothèque pour l’achat du lot avec faculté de « recharge ». Il rembourse une partie d’acquisition mais souscrit un prêt complémentaire pour travaux qui se trouve donc garanti par l’hypothèque d’origine. Si le syndicat des copropriétaires inscrit son hypothèque légale, ou fait valoir son privilège après « recharge », les sommes récupérées seront diminuées.
4. Mais un privilège renforcé depuis la loi ALUR et plus encore avec l’ORDONNANCE DU 30 OCTOBRE 2019
EXTENSION DE L’ASSIETTE DU PRIVILEGE PAR LA LOI ALUR
Alors que dans le texte d’origine le Privilège ne portait que sur les « charges et travaux », la loi ALUR étend l’assiette du privilège en y ajoutant les sommes dues au titre du « fonds de travaux », les sommes dues au titre des travaux de restauration immobilière de l’article L. 313-4-2 du Code de l'Urbanisme et diverses autres sommes :
« Sont garantis par le privilège immobilier spécial prévu à l’article 2374 du code civil : l’obligation de participer aux charges et aux travaux mentionnés aux articles 10 et 30, les cotisations au fonds de travaux mentionné à l’article 14-2, les créances afférentes aux travaux de restauration immobilière réalisés en application du c du II de l’article 24, les dommages et intérêts alloués par les juridictions au syndicat des copropriétaires, ainsi que le remboursement des dépens. » ;
PRIVILEGE APPLICABLE A TOUTES LES CREANCES PAR L’ORDONNANCE DU 30 OCTOBRE 2019
Pour sa part, l’Ordonnance du 30 octobre 2019 va étendre le privilège du syndicat à l’ensemble des créances du syndicat des copropriétaires sur le ou les lots vendus en remplaçant le texte qui précède par une phrase aussi brève que d’application étendue565 :
Article 19-1 de la loi du 10 juillet 1965
Toutes les créances mentionnées au premier alinéa de l'article 19 sont garanties par le privilège immobilier spécial prévu à l'article 2374 du code civil.
565 Le texte étant d’application différée n’entre en vigueur qu’au 1er juin 2020, en sorte que jusqu’à cette date l’assiette du privilège reste celle fixée par la loi ALUR ;
droit de la copropriété année 2019-2020
453
B. CARACTERISTIQUES DU PRIVILEGE IMMOBILIER SPECIAL
Ce privilège se trouve à la fois occulte(1) et extrêmement bien placé par rapport aux autres créanciers (2) même si son efficacité suppose la vente du lot (2).
1. Un Privilège spécial immobilier occulte.
Alors que depuis 1955 les gouvernements successifs se sont efforcés d’une part de limiter les privilèges et d’autre part d’obliger tous les créanciers à publier leur privilège autre que les privilèges généraux , la loi du 21 juillet 1994 va aller à contresens puisque :
• un privilège spécial immobilier nouveau est créé.
• ce privilège n'aura pas à être publié.
C’est surtout ce caractère occulte qui est exceptionnel.
Rappelons en effet qu’aux termes de l’article 2377 du code civil : « Entre les créanciers, les privilèges ne produisent d'effet à l'égard des immeubles qu'autant qu'ils sont rendus publics par une inscription au fichier immobilier, de la manière déterminée par les articles suivants et par les articles 2426 et 2428. »
En d’autres termes, lorsque le privilège porte sur un immeuble il prend rang tout comme une hypothèque566 : le rang donné par l’article 2374 du code civil ne s’applique que si deux privilèges différents sont publiés le même jour.
Le privilège du syndicat étant dispensé de publication n’a pas à prendre « rang ».
En effet, la loi complète ces dispositions par l'article 2378 du code civil, pour donner le caractère occulte au Privilège Spécial Immobilier ainsi créé, en sorte que l'article 2378 est rédigé comme suit :
"Sont exceptées de la formalité de l'inscription les créances énumérées à l'article 2375567 et les créances du syndicat des copropriétaires énumérées à l'article 2374".
566 Avec une petite différence : le privilège du vendeur prend rang du jour de la vente si celle-ci est publiée dans les deux mois (ce qui explique que dans les ventes d’immeuble le banquier se fait subroger dans les droits du vendeur.
567 Les frais de justice et les créances des salariés
droit de la copropriété année 2019-2020
454
Et au surplus, ce nouveau privilège accède d'emblée à la plus haute marche du podium des privilèges spéciaux immobiliers de l'article 2374 :
2. L’assiette du Privilège :
Celui-ci est défini par l’article 2374 du code civil (modifié par la loi ALUR) auquel renvoie l’article 19-1 de la loi
Les créanciers privilégiés sur les immeubles sont :
1\degres Le vendeur, sur l'immeuble vendu, pour le paiement du prix ;
S'il y a plusieurs ventes successives dont le prix soit dû en tout ou en partie, le premier vendeur est préféré au second, le deuxième au troisième, et ainsi de suite ;
1\degres bis Conjointement avec le vendeur et, le cas échéant, avec le prêteur de deniers mentionné au 2\degres, le syndicat des copropriétaires, sur le lot vendu, pour le paiement des charges et travaux mentionnés à l’article 10, au c du II de l’article 24 et à l’article 30 de la loi \no 65-557 du 10 juillet 1965 fixant le statut de la copropriété des immeubles bâtis, et des cotisations au fonds de travaux mentionné à l’article 14-2 de la même loi relatifs à l'année courante et aux quatre dernières années échues ainsi que des dommages et intérêts alloués par les juridictions et des dépens .
Toutefois, le syndicat est préféré au vendeur et au prêteur de deniers pour les créances afférentes aux charges et travaux de l'année courante et des deux dernières années échues.
1\degres ter Conjointement avec le vendeur et, le cas échéant, avec le prêteur de deniers mentionné au 2\degres du présent article, l’opérateur mentionné à l’article L. 615-10 du code de la construction et de l’habitation, si le bien vendu est assorti d’une servitude sur des biens d’intérêt collectif.
« Toutefois, l’opérateur est préféré au vendeur et au prêteur de deniers pour les redevances prévues au même article L. 615-10 de l’année courante et des deux dernières années échues »568 (…)
- Les Charges et travaux des articles 10 et 30
L’assiette d’origine du Privilège était limitée aux charges et travaux des articles 10 et 30.
568 Cela concerne les travaux engagés par l’Opérateur dans le cadre d’une copropriété en difficulté pour la réalisation des travaux nécessaire à la conservation et à la mise en sécurité de l’immeuble … que la situation financière de la copropriété ne permet pas au syndicat des copropriétaires d’engager lui-même.
droit de la copropriété année 2019-2020
455
Au titre des charges de l’article 10 on doit entendre le mot « charges » dans un sens large et pas seulement comptable : il s’agit de toutes les sommes dues soit à titre des comptes approuvés soit au titre des provisions sur charges de l’article 14-1 de la loi.
Au titre des travaux on doit entendre toutes les sommes dues soit à titre provisionnel soit après approbation des charges pour les travaux autres que d’améliorations (ce sont des charges de l’article 10) ainsi que pour les travaux d’amélioration votés (ce sont les travaux de l’article 30).
- Les autres créances visées par la loi ALUR
L’article 2374 a été modifié pour tenir compte des dispositions de la loi ALUR en ajoutant le bénéfice du privilège aux « fonds de travaux mentionné à l'article 14-2 de la même loi, relatifs à l'année courante et aux quatre dernières années échues ainsi que des dommages et intérêts alloués par les juridictions et des dépens ».
- La généralisation du privilège
Bien évidemment il va devoir être modifié à nouveau pour tenir compte de la généralisation de l’assiette du privilège par l’Ordonnance du 30 octobre 2019.
En effet le syndicat peut être créancier à d’autres titres que les charges (budget et travaux), le fonds de travaux, les D.I. ou les intérêts légaux, par exemple :
Par exemple le syndicat peut être créancier de frais résultant de l'application d'une clause "d'aggravation des charges" inscrite au Règlement de Copropriété.
Le syndicat peut encore être créancier de condamnations prononcées en application de l’article 700 NCPC
Il peut être créancier parce qu’un copropriétaire a accepté de payer l’architecte de l’immeuble pour contrôler la conformité de travaux entrepris sur parties communes
Etc,
- Les créances afférentes aux travaux de restauration immobilière.
Ici encore il s’agit de protéger les copropriétés qui entrent dans le cadre d’une Restauration Immobilière prévue par un plan de sauvegarde ou qui ont fait l’objet d’un plan de sauvegarde.
L’opérateur définit les travaux et invite le syndicat des copropriétaires à les voter (étant entendu que ces travaux bénéficieront de larges subventions et que les copropriétaires ne supporteront effectivement qu’un « reste à charge ».
droit de la copropriété année 2019-2020
456
Si les travaux sont votés, c’est en définitive ce « reste à charge » qui sera garanti par le Privilège.
3. Un privilège placé en tête des privilèges spéciaux immobiliers (Article 2374 Code Civil modifié) :
Désormais la place du Privilège est fonction de la créance qu’il garantit :
a) Toutes les sommes dues pour l’année en cours et les deux années antérieures.
Le Privilège spécial immobilier vient en tête des privilèges spéciaux immobiliers pour les sommes dues depuis moins de deux ans
Toutefois ce privilège est susceptible d'être primé par les privilèges généraux prévus par l’ article 2375 du Code civil (créances sur la généralité des immeubles) : les frais de justice, les privilèges attachés aux créances des gens de services et aux salariés, le privilège du conjoint survivant, du chef d’entreprise ou exploitant agricole, qui jouent en cas de procédure collective, privilèges de l'AGS), le privilège attaché aux frais de justice, le privilège prévu au profit de certains créanciers de la procédure collective et le privilège des auteurs compositeurs et artistes
Et si l'on reprend le tableau général des privilèges immobiliers toutes causes confondues, le syndicat des copropriétaires viendra :
- après les privilèges de l’article 2375 et à défaut de biens mobiliers pour les payer,
- mais avant tous les créanciers pour toutes les sommes dues au titre de l’année en cours et des deux dernières années, donc avant le banquier qui a prêté les fonds nécessaires à l'acquisition du lot, avant l’Opérateur bénéficiaire du rang 1 ter, comme avant le Trésor Public dont le Privilège général immobilier ne prend rang que loin derrière le Privilège du vendeur d'immeuble.
D'autre part le privilège spécial immobilier du syndicat prime également le privilège du Trésor qui garantit le paiement de l'impôt sur les plus-values.
b) Pour les sommes dues depuis plus de 2 ans et moins de 4 ans
Pour toutes les sommes dues depuis plus de deux ans et depuis moins de quatre ans le syndicat des copropriétaires vient « conjointement »avec le vendeur et le prêteur de deniers.
Conjointement signifie que le syndicat sera payé « au marc le franc »
droit de la copropriété année 2019-2020
457
4. Une mise en oeuvre subordonnée à la vente du lot.
Le privilège ne joue en premier lieu que s’il y a vente du lot.
En conséquence, le privilège ne pourra pas être mis en oeuvre :
o tant que le copropriétaire débiteur demeure propriétaire de son lot.
o en cas de succession ou de donation.
o en cas d’échange, sauf s’il y a soulte au profit du vendeur.
o en cas de mutation de parts au sein d’une société propriétaire du lot.
La cour de cassation a jugé que le privilège s’exerce en cas de vente et non pas en cas de mise en liquidation du débiteur (civ 3ème 15 février 2006)
Le privilège immobilier bénéficiant au syndicat des copropriétaires pour le paiement de charges et travaux ne s'exerce qu'en cas de vente du lot de copropriété569.
En sorte que s’il est indispensable de produire à la Liquidation pour les sommes dues à la date jugement déclaratif, sous peine de voir le syndicat déchu de sa créance :
o D’une part cette production ne sera pas à titre privilégiée (sauf si le syndicat des copropriétaires a pris une hypothèque légale avant le jugement déclaratif).
o D’autre part elle ne vaut pas mise en oeuvre du privilège, lequel ne pourra effectivement être mis en oeuvre qu’au jour de la vente du bien à la barre du Tribunal ou sur autorisation du juge commissaire.
C. DATE ET RANG DU PRIVILEGE DU SYNDICAT DES COPROPRIETAIRES
1. La durée du privilège
Le privilège du Syndicat des Copropriétaires est à double détente puisqu'il convient de distinguer entre les charges afférentes à l'année en cours et aux deux dernières années échues pour lesquelles le syndicat se trouve privilégié sans concours des autres créanciers bénéficiant de privilèges spéciaux sur les immeubles,
569 3ème CIV. - 15 février 2006. \no 04-19.095. - C.A. Grenoble, 30 juin 2004
droit de la copropriété année 2019-2020
458
et les deux années antérieures pour lesquelles le syndicat vient en concurrence avec le vendeur et le prêteur de deniers.
Dans sa rédaction du 15 février 1995, pris en application de la loi sur l’Habitat, l’article 5-2 du décret précisait :
" L'année, au sens de l'article 2103-1\degres bis du code civil, s'entend de l'année civile comptée du 1er janvier au 31 décembre ".
Ce texte a été remplacé par l’article 3 du Décret \no 2010-391 du 10 avril 2010 entré en vigueur le 1er juillet 2010 qui est ainsi rédigé :
« Pour l’application du 1\degres bis de l’article 2374 du code civil, l’année s’entend de l’exercice comptable au sens de l’article 5 du décret du 14 mars 2005 relatif aux comptes du syndicat des copropriétaires. »
Il en résulte deux modifications : l’une de concordance, l’autre de fond :
Concordance : le nouveau texte remplace la référence à l’article 2103-1 code civil abrogée par la référence au nouvel article 2374 code civil
Modification de fond : La période d’application du privilège ne se calcule plus désormais en fonction de l’année civile mais en fonction de l’exercice comptable annuel :
Pour bien comprendre prenons l’exemple d’un exercice annuel allant du 1er janvier au 31 décembre comparé à un exercice annuel allant du 1er octobre au 30 septembre et une vente signifiée au syndic le 15 septembre 2015 :
Période
Texte de 1995
Texte de 2010
Année en cours
01/01/2015 - 15/09/2015
01/10/2014au 15/09/2015
Deux années antérieures
01/01/2013 - 31/12/2014
01/10/2012 au 30/09/2014
Deux années qui précèdent
01/01/2012 - 31/12/2013
01/10/2012 au 30/09/2013
Période sans privilège
Avant le 01/01/ 2012
Avant le 01/10/ 2002
Une difficulté peut cependant se présenter du fait de cette nouvelle rédaction en application de l’article 5 alinéa 1 du Décret comptable qui précise que la durée d’un exercice est normalement de 12 mois, mais que :
« pour le premier exercice, l’assemblée générale des copropriétaires fixe la date de clôture des comptes et la durée de cet exercice qui ne pourra excéder dix-huit mois ».
droit de la copropriété année 2019-2020
459
Dans l’exemple qui précède et en admettant que la copropriété soit née le 1er juin 2009, doit-on considérer que si le premier exercice a commencé le 1er juin 2009 pour se terminer le 30 septembre 2010 (soit une durée de 15 mois) le privilège ne portera que sur la période du 1er octobre 2009 au 30 septembre 2011, la période du 1er juin 2009 au 30 septembre 2009 étant « sans privilège ».
Bien évidemment cette définition de l’année joue aussi bien pour le superprivilège de l’année en cours et des deux années précédentes pour les sommes dues au titre des charges et travaux des articles 10 et 30, que pour le privilège sur quatre ans pour les autres causes.
2. Le rang du privilège : privilège « hors concours » et « en concours »
LE PRIVILEGE "SANS CONCOURS".
Il bénéficie aux sommes dues au titre de l’exercice comptable annuel en cours et aux sommes échues depuis moins de deux exercices comptables antérieurs. Ce délai de deux ans se calcule à "rebours" depuis le jour de la mutation.
En d'autres termes, si l’exercice comptable annuel va du 1er janvier au 31 décembre et la mutation intervient le 10 février 2020, le privilège "sans concours" jouera pour les charges dues :
- En 2020 (c'est l'année en cours).
- Depuis le 1er janvier 2018 (ce sont les deux dernières années échues).
LE PRIVILEGE "EN CONCOURS".
Le syndicat est en concours avec le vendeur et le prêteur de deniers pour les sommes dues depuis plus de deux années d’exercice, mais depuis moins de quatre années d’exercice.
Dans l'exemple qui précède d'une mutation au 10 février 2020, le privilège "en concours" jouera pour les charges et travaux et les autres causes, dus pour 2016 et 2017.
Normalement, vendeurs et prêteurs de deniers ne sont pas en concurrence, (et le vendeur doit primer le prêteur) sauf hypothèse très particulière qui veut que l'acquéreur a emprunté la partie du prix payable comptant et obtenu un terme du vendeur pour le paiement du solde.
droit de la copropriété année 2019-2020
460
En pratique, le syndicat se trouvera presque systématiquement en concours avec le seul prêteur de deniers.
CAS PRATIQUE.
Il résulte de l'article 2285 du Code civil que si divers créanciers ont des droits égaux sur une somme à distribuer (par exemple un prix de vente) et que cette somme est insuffisante pour les payer tous, elle se répartit entre eux au marc le franc. Autrement dit, ils sont payés au prorata de leurs créances respectives, chaque créancier subissant une réduction proportionnelle de sa créance
Prenons un exemple détaillé pour un exercice comptable annuel allant du 1er janvier au 31 décembre et une vente intervenue entre le 1er juin et le 31 décembre 2020 :
La banque est inscrite pour : 200.000 Euros
Le syndicat est créancier de : 20.000 Euros
Trésor Public inscrit pour : 30.000 €
La créance du syndicat des copropriétaires se décomposant comme suit :
- année en cours (2020) : 1.000 Euros
- 2019 : 2.000 Euros
- 2018 : 7.000 Euros dont 5.000 € de DI et dépens
- 2017 : 2.000 Euros
- 2016 : 6.000 Euros dont 3.000 euros de dette contractuelle
- 2015 : 2.000 Euros
La vente est réalisée pour 130.000 €
Prix restant à distribuer après règlement des frais de la vente et des privilèges généraux de l’article 2375 du code civil : 100.000 Euros
Le partage se fera de la façon suivante :
1\degres/ la créance « utile » du syndicat est ramenée à l’année en cours et aux quatre dernières années, soit 18.000 €
droit de la copropriété année 2019-2020
461
2\degres/ Le syndicat des copropriétaires, pour l’année en cours et pour 2019 et 2018 percevra la totalité de sa créance, soit 9.000 Euros
3\degres/ Le reste à distribuer est de : 91.000 Euros
La répartition se fera alors au marc le franc entre le syndicat des copropriétaires et le banquier pour le surplus des sommes privilégiées :
- Créance du syndicat des copropriétaires = 8..000 € et créance de la Banque = 200.000 €.
- Sur le total des créances restant privilégiées (8.000 + 200.000 = 208.000), le syndicat des copropriétaires représente 8.000 €, soit 3.84 % et la banque représente 96.15 %
- Sur les 95.000 € à distribuer la Banque touchera 91342 € (95.000 * 96.15%) et le syndicat des copropriétaires recevra 3.648 € (8.000*3.84%)
Ainsi, au total, le syndicat, pour une créance « utile » de 18.000 Euros, recouvrera 5.000 + 3.648 = 8.648 €.
4\degres/ Pour 2015 et pour les sommes dus par le copropriétaire au syndicat des copropriétaires au titre d’un engagement personnel la créance du syndicat ne profitera d'aucun privilège et passera après les créanciers inscrits (sauf si le syndicat est lui-même inscrit en application de l’article 19 de la loi).
3. Sort du privilège en cas de procédure collective contre un copropriétaire commerçant
Le privilège peut être mis en échec en cas de procédure collective dirigée contre un copropriétaire commerçant :
- Pour les créances antérieures au jugement d’ouverture de la procédure collective, comme le privilège est occulte, le syndic n’est pas toujours avisé de la procédure collective (à moins d’avoir également inscrit une hypothèque), et il omet parfois de déclarer sa créance. Sous cette réserve, le privilège du syndicat doit jouer pleinement au moment de la vente du lot, à son rang c'est-à-dire immédiatement après les privilèges généraux
- Pour les créances postérieures au jugement d’ouverture, si les charges n’ont pas été réglées « à leur échéance », la collocation du syndicat est soumise aux dispositions de l’art. L. 622-17 du code de commerce. Le droit des procédures collectives prévaut donc sur le privilège de l'article 2374, 1\degres bis du Code civil ; pour ses créances postérieures au jugement d'ouverture, le syndicat est placé au troisième rang des créanciers privilégiés, parmi : “les autres créances, selon
droit de la copropriété année 2019-2020
462
leur rang” 570, après notamment le super privilège des salaires, les autres créances de salaires, les prêts consentis ainsi que les créances résultant de l'exécution des contrats poursuivis conformément aux dispositions de l'article L. 622-13 (contrats nécessaires à la poursuite de l’exploitation)
2. Nécessité d’accélérer le recouvrement des charges.
Comme l’a déclaré le Ministre du logement lors du vote de la loi de 1994 en séance de l'Assemblée Nationale, le syndic doit "faire les diligences nécessaires pour récupérer les sommes impayées sinon les créances passeront de la qualité de super privilégiées à celle de privilégiées puis perdront tout privilège".
Ajoutons que le syndicat des copropriétaires aura d’autant plus intérêt à précipiter la procédure de la vente judiciaire que préalablement à la délivrance du commandement de saisie, la copropriété doit être titulaire d’une condamnation au fond et que par application de l’article 10 de l’arrêté du 14 mars 2005, l'assemblée générale doit inscrire la créance du syndicat comme créance douteuse, au compte 49 intitulé « dépréciation des comptes de tiers », sous-compte 491 « copropriétaires » et provisionner le montant nécessaire pour équilibrer les comptes, c'est-à-dire voter un « emprunt » auprès des autres copropriétaires à concurrence de cette créance douteuse : or, il est plus facile de faire un emprunt pour combler 5.000 € que 20.000 € !
D. MISE EN OEUVRE DU PRIVILEGE IMMOBILIER SPECIAL DU SYNDICAT DES COPROPRIETAIRES : L’OPPOSITION SUR LE PRIX DE VENTE DU LOT
Il existait dans la loi de 1965 un article 20 qui sans conférer de privilège au syndicat, lui permettait cependant à diverses conditions de forme et de contenu, de bloquer le prix de vente du lot. Ces dispositions ont été totalement modifiées par la loi du 21 juillet 1994, qui a été légèrement modifiée par la loi SRU, complétée par la loi ALUR et enfin ajournée par l’Ordonnance du 18 septembre 2019 entrée en vigueur le 1er janvier 2020.
Nous étudierons en conséquence l'article 20 dans sa rédaction actuelle
Art. 20 –
570 Cass. avis \no 0020002 P, 21 janv. 2002 : JurisData \no 2002-012806 ; Bull. civ., avis, \no 1 ; D. 2003, p. 1334, obs. C. Giverdon ; Defrénois 2002, art. 37591, p. 1092, obs. P. Théry ; Rev. proc. coll. mars 2002, p. 4, obs. I. Fosset-Lefebvre.
droit de la copropriété année 2019-2020
463
I.-Lors de la mutation à titre onéreux d'un lot, et si le vendeur n'a pas présenté au notaire un certificat du syndic ayant moins d'un mois de date, attestant qu'il est libre de toute obligation à l'égard du syndicat, avis de la mutation doit être donné par le notaire au syndic de l'immeuble par lettre recommandée avec avis de réception dans un délai de quinze jours à compter de la date du transfert de propriété. Avant l'expiration d'un délai de quinze jours à compter de la réception de cet avis, le syndic peut former au domicile élu, par acte extrajudiciaire, opposition au versement des fonds dans la limite ci-après pour obtenir le paiement des sommes restant dues par l'ancien propriétaire. Cette opposition contient élection de domicile dans le ressort du tribunal judiciaire de la situation de l'immeuble et, à peine de nullité, énonce le montant et les causes de la créance. Le notaire libère les fonds dès l'accord entre le syndic et le vendeur sur les sommes restant dues. A défaut d'accord, dans un délai de trois mois après la constitution par le syndic de l'opposition régulière, il verse les sommes retenues au syndicat, sauf contestation de l'opposition devant les tribunaux par une des parties. Les effets de l'opposition sont limités au montant ainsi énoncé.
Tout paiement ou transfert amiable ou judiciaire du prix opéré en violation des dispositions de l'alinéa précédent est inopposable au syndic ayant régulièrement fait opposition.
L'opposition régulière vaut au profit du syndicat mise en oeuvre du privilège mentionné à l'article 19-1.
II. Préalablement à l'établissement de l'acte authentique de vente d'un lot ou d'une fraction de lot, le cas échéant après que le titulaire du droit de préemption instauré en application du dernier alinéa de l'article L. 211-4 du code de l'urbanisme a renoncé à l'exercice de ce droit, le notaire notifie au syndic de la copropriété le nom du candidat acquéreur ou le nom des mandataires sociaux et des associés de la société civile immobilière ou de la société en nom collectif se portant acquéreur, ainsi que le nom de leurs conjoints ou partenaires liés par un pacte civil de solidarité. Dans un délai d'un mois, le syndic délivre au notaire un certificat datant de moins d'un mois attestant : 1\degres Soit que l'acquéreur ou les mandataires sociaux et les associés de la société se portant acquéreur, leurs conjoints ou partenaires liés à eux par un pacte civil de solidarité ne sont pas copropriétaires de l'immeuble concerné par la mutation ; 2\degres Soit, si l'une de ces personnes est copropriétaire de l'immeuble concerné par la mutation, qu'elle n'a pas fait l'objet d'une mise en demeure de payer du syndic restée infructueuse depuis plus de quarante-cinq jours. Si le copropriétaire n'est pas à jour de ses charges au sens du 2\degres du présent II, le notaire notifie aux parties l'impossibilité de conclure la vente. Dans l'hypothèse où un avant-contrat de vente a été signé préalablement à l'acte authentique de vente, l'acquéreur ou les mandataires sociaux et les associés de la société se portant acquéreur, leurs conjoints ou partenaires liés à eux par un pacte civil de solidarité, dont les noms ont été notifiés par le notaire, disposent d'un délai de trente jours à compter de cette notification pour s'acquitter de leur dette vis-à-vis du syndicat. Si aucun certificat attestant du règlement des charges n'est produit à l'issue de ce délai, l'avant-contrat est réputé nul et non avenu aux torts de l'acquéreur.
droit de la copropriété année 2019-2020
464
Par ailleurs, par application de l'article 2461 du Code civil :
"Les créanciers ayant privilège ou hypothèque inscrit sur un immeuble, le suivent en quelque main qu'il passe pour être colloqués et payés suivant l'ordre de leur créance ou inscription".
Dès lors :
Soit la vente est faite à titre amiable, et en cette hypothèse, le Notaire paiera au syndicat des copropriétaires les sommes, objet du privilège,
Soit la vente est faite à la barre du Tribunal, et en cette hypothèse, le paiement se fera selon les règles de la distribution du prix par le juge de l’exécution.
1. Principe : la mise en oeuvre du privilège est réalisée par l’opposition faite par le syndic entre les mains du notaire
AVIS DE MUTATION DU LOT DONNE PAR LE NOTAIRE
S'agissant d'une mutation à titre onéreux :
"Avis de la mutation doit être donnée par le Notaire au syndic de l'immeuble, par lettre recommandée avec avis de réception dans un délai de quinze jours à compter de la date du transfert de propriété ".
Il est clair que pour une vente amiable, la diligence de l'acquéreur est le plus souvent constituée par un acte du Notaire.
Si le notaire tarde à donner cet avis de mutation, il engage sa responsabilité
Bastia(Ch. civ.), 16 novembre 2004 - R.G. \no 03/00549:
« En vertu de l'article 20 de la loi du 10 juillet 1965, en cas de la mutation à titre onéreux d'un lot, le notaire doit donner avis de cette mutation au syndic de l'immeuble, le syndic pouvant, avant l'expiration d'un délai de quinze jours à compter de la réception de cet avis, former opposition au versement des fonds pour obtenir le paiement des sommes restant dues par l'ancien propriétaire.
Dès lors manque à son devoir de conseil et occasionne au syndicat des copropriétaires un préjudice égal au montant des charges de copropriété non recouvrées, le notaire qui effectue cette
droit de la copropriété année 2019-2020
465
notification plus de six mois après l'acte de vente - précisant que le prix se compenserait, lors de la réalisation d'une condition suspensive, avec le montant d'un prêt d'argent souscrit par le vendeur au profit de l'acquéreur - et près de deux mois après l'acte constatant la réalisation de ladite condition suspensive et le paiement du prix par compensation, cette notification tardive ayant empêché le syndic de former valablement opposition avant que l'acte constatant la compensation du prix ne soit reçu et d'y faire ainsi obstacle, même si ce syndic ne pouvait exercer son privilège sur le prix de vente qui n'était pas consigné en l'étude du notaire ».
Il convient de relever ici que l’article 20 ne fait état que du notaire et non de l’avocat poursuivant en cas de vente judiciaire. Cette omission a été réparée par le Décret d’application :
Dans le cas présent il édicte, art 5 - 1 dernier alinéa (décret du 17 mars 1967 modifié) :
" Si le lot fait l'objet d'une vente devant le tribunal sur licitation ou sur saisie immobilière, l'avis de mutation prévu par l'article 20 est donné au syndic, selon le cas soit par le notaire soit par l'avocat du demandeur ou du créancier poursuivant; si le lot fait l'objet d'une expropriation pour cause d'utilité publique ou de l'exercice d'un droit de préemption publique, l'avis de mutation est donné au syndic, selon le cas, soit par le notaire, soit par l'autorité expropriante, soit enfin par le titulaire du droit de préemption; si l'acte est reçu en la forme administrative, l'avis de mutation est donné au syndic par l'autorité qui authentifie la convention ".
Attention cependant au piège de la notification au cas où la vente amiable est reçue par deux notaires : il arrive en ce cas que le notaire de l’acquéreur (qui ne détient pas les fonds) fasse la notification de la vente prévue par l’article 6 du Décret, tandis que l’avis de mutation de l’article 20 sera donné par le notaire du vendeur (qui a effectivement reçu les fonds) : l’opposition doit être faite entre les mains du notaire qui dispose des fonds, donc qui aura notifié l’avis de mutation prévu à l’article 20 de la loi.
Il convient également de relever que si la loi sur la copropriété permet les notifications par voie électronique :
- Aux termes de l’article 42-1 de la loi les notifications concernées sont celles faites par le syndic au copropriétaire ayant donné un accord exprès et non à des tiers.
- Il s’agit ici d’un « avis de mutation » et non d’une notification, contrairement à la « notification » de l’article 6 du Décret.
DELAI D'OPPOSITION DU SYNDIC.
Ce délai est de quinze jours de la notification. Le non-respect du délai légal est très grave : il interdit au syndicat de mettre en oeuvre son privilège !
droit de la copropriété année 2019-2020
466
FORME DE L'OPPOSITION.
L'opposition doit être faite par acte extrajudiciaire (huissier)571
L'acte extrajudiciaire doit contenir élection de domicile dans le ressort du Tribunal du lieu de situation de l'immeuble.
L'acte doit énoncer le montant et les causes de la créance.
Relevons cependant que :
seul est prescrit à peine de nullité l'énoncé du montant et des causes de la créance.
par contre, si l'élection de domicile dans le ressort du lieu de situation de l'immeuble est obligatoire, elle n'est pas prescrite à peine de nullité.
LES CAUSES ET LE MONTANT DE LA CREANCE.
Aux termes des alinéas 1 et 2 de l’article 5-2 du Décret (mis à jour par le Décret de 2019) :
Pour l'application des dispositions de l'article 20 de la loi du 10 juillet 1965 modifiée, il n'est tenu compte que des créances du syndicat effectivement liquides et exigibles à la date de la mutation.
L'opposition éventuellement formée par le syndic doit énoncer d'une manière précise :
1\degres Le montant et les causes des créances de toute nature du syndicat de l'année courante et des deux dernières années échues ;
2\degres Le montant et les causes des créances de toute nature du syndicat des deux années antérieures aux deux dernières années échues ;
3\degres Le montant et les causes des créances de toute nature du syndicat garanties par une hypothèque légale et non comprises dans les créances privilégiées, visées aux 1\degres et 2\degres ci-dessus ;
571 Même si l’opposition du syndicat est notifiée au notaire, il s’agit d’une opposition et non d’une notification au sens du nouvel article 42-1 de la loi \no65-557 du 10 juillet 1965. Au demeurant la loi ALUR n’entendait très certainement viser que les notifications « à l’intérieur » de la copropriété et ne pas viser les notifications faites aux tiers ou par les tiers.
droit de la copropriété année 2019-2020
467
4\degres Le montant et les causes des créances de toute nature du syndicat non comprises dans les créances visées aux 1\degres, 2\degres et 3\degres ci-dessus.
L'opposition doit contenir le montant et les causes de la créance.
Le montant est assez facile à déterminer : c'est le total des sommes dues par le copropriétaire vendeur.
Les "causes" de la créance ont une importance primordiale puisqu'elles déterminent l'étendue du Privilège... et son rang.
La ventilation très précise entre les quatre catégories de créances permet de faire une répartition entre créanciers, soit par le notaire, soit dans le cadre de la distribution du prix par le Juge de l’Exécution du Tribunal Judiciaire, conforme aux dispositions des articles 2374 et 2375 du code civil.
QUE SE PASSE-T-IL EN L'ABSENCE D'OPPOSITION ?
Puisque c'est l'opposition qui vaut mise en oeuvre du Privilège, l'absence d'opposition entraîne la perte pure et simple de ce privilège.
L’OPPOSITION IRREGULIERE
Aux termes de l’atrticle 19 de la loi :
L'opposition régulière vaut au profit du syndicat mise en oeuvre du privilège mentionné à l'article 19-1.
En conséquence, l’opposition irrégulière prive le syndicat des copropriétaires du bénéfice du privilège.
C’est parfaitement logique puisque jusque lors le privilège était « occulte ». Il faut bien qu’il soit révélé par une procédure régulière.
Civ 3\degres, 20 octobre 2005572 :
572 Civ 3\degres, 20 oct 2005 ; Loyers et Copropriété 2006 \no 18
droit de la copropriété année 2019-2020
468
L’opposition est irrégulière si elle ne fait pas apparaître avec précision le montant et les causes des créances du syndicat. Une opposition, même irrégulière, prive le syndicat de former ultérieurement une nouvelle opposition et, à tout le moins, à l’expiration du délai de quinze jours à compter de la date de la première opposition.
3e Civ. - 25 octobre 2006 :
Ayant relevé qu’un acte d’opposition au versement du prix de vente de lots de copropriété formé par un syndic énonçait de manière insuffisante les causes et le montant des créances du syndicat des copropriétaires au regard de l’article 5-1 du décret du 17 mars 1967, une cour d’appel en déduit exactement que ce syndicat n’était pas un créancier privilégié en application de l’article 2103 1\degres bis du code civil 573.
Bien évidemment l’opposition faite irrégulièrement ne fait pas perdre sa créance au syndicat des copropriétaires : simplement, il perd son privilège et s’il n’a pas inscrit l’hypothèque judiciaire, il devient simple créancier chirographaire574.
En cas de vente à la barre du Tribunal et dans la mesure où seuls les créanciers ayant dénoncé leur créance ou les créanciers des articles 2374 et 2375 du code civil peuvent être payés sur le produit de la vente judiciaire, même si le prix reçu permet d’indemniser le syndicat des copropriétaires pour le montant de cette opposition irrégulière, le syndicat des copropriétaires ne sera pas payé !
2. Les suites de l'Opposition.
L’OPPOSITION BLOQUE LA SOMME INDIQUEE PAR LE SYNDIC, MAIS NON LA TOTALITE DU PRIX DE VENTE.
Seule se trouve bloquée la somme correspondant au montant de l'opposition. S'agissant d'une vente amiable, le notaire pourra libérer sans arrière-pensée le montant pour lequel l'opposition n'a pas été pratiquée.
L'opposition bloque les sommes énoncées dans l'opposition, que ces sommes bénéficient ou non du Privilège spécial immobilier.
Le notaire ne pourrait faire la distinction entre les différentes causes de l'opposition et considérer par exemple que les sommes ne bénéficiant d'aucun privilège peuvent être libérées nonobstant leur "énonciation" dans l’Opposition.
573 Il convient de lire désormais article 2374 du code civil.
574 Civ 3\degres Ch 15 décembre 2004 ; Cass. 3e civ. 27 novembre 2013 \no 12-25.824 (\no 1392 FS-PB), Synd. copr. Ensemble immobilier parc Kalliste c/ Sté Arevian
droit de la copropriété année 2019-2020
469
Cour de VERSAILLES du 2 mai 1989 à la Gaz. Pal. 1990, 1, sommaire 174 :
"Commet une faute le notaire qui se dessaisit des fonds provenant de la mutation au mépris d'une opposition sur le prix faite entre ses mains au nom du syndicat".
L’OPPOSITION PEUT FAIRE L’OBJET D’UNE DEMANDE DE MAINLEVEE DEVANT LE JUGE DES REFERES
Si l'opposition est pratiquée en vertu d'un Titre Exécutoire, aucune discussion ne paraît possible sur la régularité des causes de l'Opposition et pour peu que les formes prescrites à peine de nullité aient été observées, le copropriétaire "vendeur" ou "vendu" n'aura aucun recours utile contre l'Opposition.
Mais le plus souvent l'Opposition est pratiquée avant que le syndicat ait obtenu condamnation du copropriétaire vendeur. Ce dernier pourra vouloir contester l'Opposition pratiquée de façon "arbitraire" par le syndic.
Dans sa rédaction d'origine le Décret du 17 mars 1967 (article 57) organisait une procédure de recours à l'encontre de l'opposition de l'article 20 de la loi : Le juge des référés pouvait limiter les effets de l'opposition au montant des sommes restant dues au syndicat par l'ancien propriétaire. Mais ces dispositions ont été abrogées575
Il n'en demeure pas moins que l'opposition peut avoir été pratiquée irrégulièrement :
- Montant d'opposition parfaitement injustifié.
- Non respect des formes édictées à peine de nullité.
Or, le texte de l'article 20 affirme que :
" L'opposition régulière vaut au profit du syndicat mise en oeuvre du Privilège prévu à l'article 19-1"
Dans ce cas, il appartient au copropriétaire de former, selon les dispositions du droit commun, une demande de mainlevée devant le Juge des Référés576
575 L’article 57 a été abrogé pour être recréée par le Décret de 2020 qui concerne les actions en justice où le syndicat est représenté par le président du conseil syndical. 576 Aix en Provence 8 juin 2005 - Le pouvoir de prononcer la mainlevée d'une telle opposition ne relève plus de la compétence du juge de l'exécution depuis l'abrogation de l'article 57 du décret du 17 mars 1957. Les dispositions de la loi du 10 juillet 1965, et notamment l'article 20 de la loi du 10 juillet 1965sur l'opposition du prix de vente d'un lot de copropriété, sont en effet autonomes par rapport aux dispositions de la loi du 9 juillet 1991.
droit de la copropriété année 2019-2020
470
Celui-ci pourra ordonner mainlevée d'une opposition manifestement irrégulière ou en limiter les effets aux sommes effectivement liquides et exigibles. Ce sera le cas par exemple si l’opposition porte à la fois sur les sommes dues au titre du lot vendu et sur des sommes dues au titre d’un lot conservé par le copropriétaire débiteur577
Certes, le syndicat demeurera créancier de l'ancien propriétaire du lot, mais il ne pourra prétendre se faire payer sur le prix de vente ou sur le prix d'adjudication.
3. Que se passe-t-il après mise en oeuvre régulière du privilège ?
Article 19 de la loi :
Le notaire libère les fonds dès l'accord entre le syndic et le vendeur sur les sommes restant dues. A défaut d'accord, dans un délai de trois mois après la constitution par le syndic de l'opposition régulière, il verse les sommes retenues au syndicat, sauf contestation de l'opposition devant les tribunaux par une des parties. Les effets de l'opposition sont limités au montant ainsi énoncé.
Tout paiement ou transfert amiable ou judiciaire du prix opéré en violation des dispositions de l'alinéa précédent est inopposable au syndic ayant régulièrement fait opposition.
EN CAS DE VENTE NOTARIEE.
Normalement le notaire n'accepte de faire une vente notariée que dans la mesure où les fonds provenant de la vente permettront de régler l'intégralité des créanciers inscrits et privilégiés.
Dans le cas contraire, sauf à en être requis, il ne pourra que refuser son concours puisque l'acquéreur se verrait dans l'obligation de purger les inscriptions grevant l'immeuble.
En conséquence, lorsque le notaire recevra l'opposition il aura très probablement les fonds nécessaires pour régler le syndicat du montant de l'ensemble de sa créance.
Dès lors trois hypothèses :
• Le vendeur autorise le notaire à remettre les fonds au syndicat des copropriétaires.
577 CA Metz 23 fév 2010 – Loyers et Copropriété déc 2010 \no 334
droit de la copropriété année 2019-2020
471
Si le syndicat des copropriétaires possède déjà un titre exécutoire contre le copropriétaire, il est manifeste que pour limiter les frais le vendeur autorisera le notaire à remettre les fonds au syndicat des copropriétaires.
De même, le copropriétaire peut spontanément se reconnaître débiteur et en conséquence autoriser le notaire à verser les fonds au syndicat des copropriétaires avant toute condamnation à paiement.
Bien évidemment le notaire ne peut régler le montant de l’opposition au syndic dans le délai de trois mois de l’opposition sans l’accord préalable du vendeur, sinon il engagerait sa responsabilité professionnelle vis à vis du vendeur, quand bien même celui-ci n’aurait pas protesté en apprenant l’opposition pratiquée par le syndic 578.
• Le vendeur ne saisit pas le juge des référés dans le délai de trois mois de la constitution de l’opposition.
Le vendeur conserve le silence ou se contente de protester auprès du notaire sans saisir le juge.
C’est la nouveauté qui a été apportée de la loi ALUR : l’absence de saisine du juge par le débiteur donne un titre au syndicat des copropriétaires puisque le notaire à l’issue du délai de 3 mois de la « constitution » de l’opposition devra remettre les fonds (donc la totalité du montant de l’opposition) au syndicat des copropriétaires.
• Le vendeur saisit le juge des référés dans le délai de trois mois
Deux hypothèses encore :
\degres le juge fait droit à la demande de mainlevée d’opposition.
Par exemple parceque l’opposition pratiquée n’est pas régulière ou parceque le juge des référés considère que le syndicat des copropriétaires n’est pas créancier. Le syndicat des copropriétaires n’a plus qu’à saisir le juge du fond pour obtenir un titre.
\degres le juge rejette totalement ou partiellement la demande de mainlevée.
On se trouve désormais dans une situation équivoque.
578 Cass. 1e civ. 12 juin 2012 \no 11-18.100 (\no 685 F-D), Sandler c/ SCP X
droit de la copropriété année 2019-2020
472
- Soit on considère que le rejet de l’opposition purge la contestation et le notaire peut alors remettre les fonds au syndicat des copropriétaires.
- Soit on considère que le texte doit s’interpréter littéralement (: « il (le notaire) verse les sommes retenues au syndicat, sauf contestation de l’opposition devant les tribunaux par une des parties. » ; en sorte que la contestation soit reçue ou non le notaire ne peut pas libérer les fonds entre les mains du syndicat des copropriétaires !
Mais il est vrai en pratique que si le débiteur saisit le juge des référés, le syndicat des copropriétaires, dépourvu de titre, fera une demande reconventionnelle en paiement contre le copropriétaire vendeur et demandera au juge que le notaire libère les fonds entre ses mains (ne pas oublier alors de mettre le notaire dans la cause).
EN CAS DE VENTE FORCEE.
L'ordonnance du 21 avril 2006 et le décret du 27 juillet 2006 ont profondément modifié la procédure de saisie immobilière579,
L'opposition de l'article 20 est notifiée après le jugement d'adjudication, entre les mains de l'avocat poursuivant.
Toutes les créanciers inscrits et le syndicat des copropriétaires pour sa créance de l’article 20 de la loi doivent produire580 leur créance entre les mains du créancier poursuivant.
\degres Accord entre les créanciers : homologation par le JEX.
Le créancier poursuivant propose un ordre de classement qu’il notifie aux autres créanciers ayant produit. A défaut de contestation dans les quinze jours, le créancier poursuivant demande au juge de l’exécution d’homologuer cet accord. En cas de contestation, le créancier poursuivant convoque les autres créanciers et si un accord est trouvé, il est dressé procès-verbal qui est ensuite homologué par le juge de l’exécution. Si tous les créanciers ayant produit en sont d'accord, le juge autorisera le règlement sur le prix de la vente.
Si par contre tous les créanciers ne sont pas d'accord (comme si un seul d'entre eux fait défaut à l'ordre amiable), s'ouvrira alors la phase d'ordre judiciaire.
579 Cf. L’article de M GELINET : Copropriété et Vente judiciaire d’un lot, AJDI Octobre 2007 p. 724.
580 Seuls peuvent produire leur créance : le créancier poursuivant, les créanciers inscrits sur l’immeuble à la date de publication du commandement, les créanciers intervenus à la procédure de saisie et à condition de s’être inscrits avant la publication de la vente judiciaire, les créanciers de l’article 2374 du code civil (le syndicat des copropriétaires).
droit de la copropriété année 2019-2020
473
\degres A défaut d’accord des créanciers ayant produit
Le créancier poursuivant assigne les autres créanciers devant le juge de l’exécution en proposant un ordre de classement.
En sorte que si le syndicat des copropriétaires n'a pas encore de Titre mais a produit sa créance (soit au titre de l’hypothèque légale, soit au titre de l’article 20 de la loi, il pourra sans doute demander au JEX de valider sa créance).
Bien évidemment le jugement ainsi rendu est susceptible d'appel.