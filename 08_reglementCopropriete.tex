\chapter{Le règlement de copropriété}

L'article 664 du Code civil n'avait prévu aucune réglementation dans les rapports entre les copropriétaires puisqu'il ne traitait que du partage des charges d'entretien de l'immeuble.

Aujourd'hui, l'organisation de la copropriété est prévue dans un acte juridique fondamental : le règlement de copropriété.

Cet acte, qui constitue la charte de la copropriété, détermine les prérogatives de chaque copropriétaire, tant sur les parties privatives que sur les parties communes ainsi que les règles d'organisation et d'administration de la copropriété de l'immeuble.

Le texte essentiel qui régit le règlement de copropriété est l'article 8 de la loi du 10 juillet 1965 aux termes duquel
\begin{quote}
	<< {\itshape Un règlement conventionnel de copropriété, incluant ou non l'état descriptif de division, détermine la destination des parties tant privatives que communes, ainsi que les conditions de leur jouissance ; il fixe également, sous réserve des dispositions de la présente loi, les règles relatives à l'administration des parties communes.} >>
\end{quote}

\section{Rédaction du règlement de copropriété}

	\subsection{Les caractéristiques du règlement de copropriété}
	
		\subsubsection{Le règlement de copropriété est un document obligatoire}
		
			L'article 8 qui vient d'être rapporté figure au nombre des dispositions de la loi de 1965 auxquelles l'article 43 confère un caractère impératif. Il en résulte que le règlement de copropriété est aujourd'hui un document obligatoire, alors que sous l'empire de la loi de 1938, il n'était que facultatif, et à cette époque il existait des copropriétés sans règlement.
			
			Outre l'existence du règlement, le contenu de celui-ci présente aussi très largement un caractère impératif, car la plupart des règles de la loi concernant l'organisation et l'administration de la copropriété sont d'ordre public.
	
			Cependant, \textbf{l'existence d'un règlement ne constitue pas la condition nécessaire à l'application du statut de la copropriété à un immeuble collectif}\footnote{Cour d'Appel PARIS 23\degres Ch B 16 mai 2002 Loyers et Copropriété 2002 \no 265}.
			
			Dès que celui-ci est divisé en lots attribués à des personnes différentes, le statut de la copropriété s'applique et l'immeuble se trouve de plein droit soumis à toutes les dispositions de la loi de 1965 et du décret de 1967\footnote{Civ. 3ème, 3 octobre 1969, Informations Rapides de la Copropriété 1970 p.77; Civ. 3ème, 15 novembre 1989, Bull. Civ. III \no214, D.1990.195 note CAPOULADE et GIVERDON, Rep. Defrénois 1990, art. 34802 \no65 obs. H. SOULEAU}, quand bien même le règlement de copropriété n’aurait pas été rédigé.
			\begin{quote}
				\textbf{Civ. 3ème 15 novembre 1989} :
				
				« {\itshape La loi du 10 juill. 1965 régit tout immeuble bâti ou groupe d'immeubles bâtis dont la propriété est répartie entre plusieurs personnes, par lots comprenant chacun une partie privative et une quote-part de parties communes ;
				
				Le statut de la copropriété des immeubles bâtis s'applique de plein droit dès lors que sont remplies les seules conditions prévues par l'art. 1er, al. 1er, de la loi du 10 juill. 1965 ;
				
				Dès lors, viole cet article l'arrêt qui déclare qu'une disposition de cette loi (en l'occurrence l'art. 42, al. 1er) ne peut s'appliquer que dans les copropriétés organisées, ce qui implique l'existence d'un règlement de copropriété} »
			\end{quote}
			
			L'importance que revêt le règlement interdit qu'une copropriété en demeure trop longtemps dépourvue. C'est pourquoi en cas de carence ou d'impossibilité, l'établissement du règlement peut être requis à tout moment de l'autorité judiciaire\footnote{T.G.I. Brest 8 avril 1970, AJPI 1970, p.221 note BOUYEURE.}. 
			
			De même l’arrêt de cassation précité (Civ. 3ème, 15 novembre 1989) précise :
			\begin{quote}
					\itshape
					Viole aussi l'art. 14 de la loi du 10 juill. 1965, ensemble l'art. 3, al. 1er, du décret du 17 mars 1967, le même arrêt qui, pour refuser de commettre un expert afin de proposer un projet de règlement de copropriété aux parties et, à défaut d'accord, de recourir au juge, retient que, en matière d'organisation et de fonctionnement d'une copropriété, il n'appartient pas au juge de se substituer aux parties, s'agissant d'actes conventionnels, mais qu'il appartient aux copropriétaires réunis en une assemblée générale de décider, les tribunaux n'ayant vocation à trancher qu'en cas de désaccord ou de conflit mettant en échec l'application de la loi, alors que, à défaut d'accord entre les parties, le règlement de copropriété peut résulter d'un acte judiciaire constatant la division de l'immeuble dans les conditions fixées par la loi du 10 juill. 1965.
			\end{quote}
	
			Le juge peut donc être saisi par un plusieurs copropriétaires pour établir, voire compléter, le règlement de copropriété\footnote{TGI BOBIGNY, chambre 5 section 2, 2 juillet 2003}.
		
		\subsubsection{Le règlement de copropriété a une nature a la fois contractuelle et institutionnelle}
		
			Certains estiment qu'il s'agit d'un acte de nature contractuelle conformément aux termes de l'article 8 de la loi du 10 juillet 1965 qui emploie les termes de << règlement conventionnel >>.
			
			D'autres considèrent que le règlement a un caractère institutionnel, qu'il constitue un << acte-règlement >  parce que, dans certaines hypothèses, il est applicable à des copropriétaires qui n'y ont pas personnellement adhéré : cas du règlement voté à la majorité de l'article 26 qui s'impose aux minoritaires ou de celui établi par le juge au cours d'un partage judiciaire ou d'une impossibilité de réunir la majorité de l'article 26.
			
			Cet argumentation se trouve renforcée par l'arrêt du 15 novembre 1989 ayant déclaré le juge habilité à imposer un Règlement de Copropriété : dans une telle hypothèse on est fort éloigné de la convention !
			
			Malgré ces objections qui sont fondées sur les circonstances de l'élaboration du règlement, la jurisprudence adopte la qualification conventionnelle proposée par le législateur. On peut donc voir dans le règlement une convention qui s'apparente au contrat d'adhésion.
			
			C'est d'ailleurs cette nature de Contrat d'Adhésion qui a été retenue par un arrêt de la Cour d'Appel de PARIS du 21 décembre 1990\footnote{Cité in Revue de Droit Immobilier 1991.257} : << {\itshape Le Règlement de Copropriété est un contrat d'adhésion qui constitue la loi entre les parties}. >>
			
			Les conséquences d'une telle qualification sont nombreuses:
			\begin{itemize}
				\item L'interprétation du règlement relève, comme celle des contrats, du pouvoir souverain des juges du fond, sous réserve de la dénaturation de clauses claires et précises\footnote{Civ. 3ème, 28 février 1969, Bull. Civ. \no III \no190; Civ. 3ème 18 décembre 1973, Bull civ III \no 637 p. 464}.
				
				\item Les principes d'interprétation des contrats édictés par le Code civil aux articles 1156 et suivants sont applicables au règlement : ainsi a-t-il été jugé que lorsqu'une clause était douteuse, elle devait s'appliquer en faveur de celui qui avait adhéré au contrat (le
				copropriétaire) et contre le rédacteur du règlement, en vertu de l'article 1162 du Code civil.
				
				\item La responsabilité encourue par le copropriétaire en cas de violation du règlement est de nature contractuelle\footnote{ Civ. 3ème 18 janvier 1972, Bull. Civ. III \no39} : application de l'article 1143 du Code civil en ce qui concerne la démolition ou la remise en état.
				
				\item Exclusion des \textbf{actions possessoires} entre copropriétaires au cas de troubles émanant de l'un sur le lot de l'autre, en vertu de la règle selon laquelle de telles actions sont exclues entre contractants et que le préjudice résulte de la violation du règlement de copropriété qui est une convention\footnote{Civ. 3ème, 22 juin 1976, Bull. Civ. III \no274, Rep. Defrénois 1977, art.31350, obs. H.SOULEAU.}.
				
				\item Comme dans les contrats, une \textbf{clause pénale} peut être insérée dans le règlement de copropriété en prévision de la violation d'une de ses dispositions par un copropriétaire : Civ. 3ème, 30 octobre 1973, J.C.P. 1973 IV 402 ; ou une clause stipulant des intérêts de retard sur des appels de provision Paris 30 octobre 1979, D.1980, I.R. 240 obs. GIVERDON.
				
				\item Comme en matière conventionnelle civile, la \textbf{clause compromissoire }est prohibée (art. 2061 C.Civ.).
			\end{itemize}
	
	\subsection{Règlement préalable a la division de l’immeuble}
	
		Le règlement est, en règle générale, établi préalablement à la naissance de la copropriété, c'est-à-dire avant l'attribution à une personne d'un lot en propriété. La phase d'élaboration du règlement préalable présente des différences selon les circonstances dans lesquelles l'immeuble est mis en copropriété.
		
		\subsubsection{Vente en l'état futur d'achèvement}
		
			S'il s'agit d'un immeuble construit en vue d'être vendu par appartements, c'est le promoteur
			qui établit le règlement qu'il s'agisse d'un promoteur individuel ou d'une société de construction-vente régie par la loi du 16 juillet 1971. Ce règlement est annexé aux actes de vente, et l'acquéreur du lot vendu << adhère >> à ce règlement au moment de son acquisition.

		\subsubsection{Immeuble construit par une société d'attribution}
		
			Si l'immeuble est construit par une société d'attribution, le règlement est annexé ou intégré aux statuts de la société, de sorte que tout acquéreur de parts donnant vocation à l'attribution d'un lot en ait connaissance. Jusqu'à dissolution ou retrait d'un associé, le règlement ne vaut que comme règlement de jouissance, lequel est soumis dans une large mesure aux règles du statut de la copropriété. Dans ce cas, le règlement --- de jouissance puis de copropriété --- est établi par le rédacteur des statuts de la société d'attribution.
			
		\subsubsection{Vente d'un Immeuble existant après division}
		
			Si la mise en copropriété résulte de la division d'un immeuble déjà construit appartenant à un propriétaire unique, le règlement de copropriété est établi par les soins de ce dernier et il est également joint aux actes de vente.
			
			Dans ces trois premiers cas, l'élaboration du règlement émane d'une seule personne et c'est au moment de la passation des actes de vente que l'accord de volonté de l'acquéreur intervient : on retrouve le mécanisme du contrat d'adhésion.
		
		\subsubsection{Établissement lors du Partage de l'Immeuble}
		
			L'établissement préalable du règlement peut aussi résulter du partage d'un immeuble en indivision à la suite, par exemple, du décès d'un propriétaire unique. Lorsque le partage est amiable, le règlement de copropriété est établi à la suite d'un accord entre les héritiers copartageants. Il résulte d'une libre négociation et présente alors, dès sa phase d'élaboration, un caractère véritablement conventionnel.
			
			Au cas de désaccord entre les indivisaires, le partage est judiciaire et le règlement, généralement établi par le notaire qui liquide la succession, est imposé par la décision de justice. Disparaît alors toute intervention de la volonté des copartageants, futurs copropriétaires, et, par conséquent, toute dimension contractuelle.
	
	\subsection{Règlement postérieur a la naissance du syndicat des copropriétaires}
	
		On sait que sous l'empire de la loi de 1938, le règlement de copropriété n'était pas obligatoire. Il l'est devenu depuis 1965. Dès lors, dans les copropriétés anciennes qui ne comportaient pas de règlement, il a fallu, pour une mise en conformité, établir un règlement s'appliquant à une copropriété déjà constituée et ayant peut-être fonctionné depuis de longues années. Remarquons toutefois que cette situation est exceptionnelle.

		\subsubsection{Adoption par l’assemblée générale}
		
			C'est le syndicat, qui, d'après l'article 14 al.3, a compétence pour établir un règlement ultérieur.
			
			L'assemblée générale des copropriétaires devra procéder à cette opération à la majorité de l'article 26 b de la loi de 1965, c'est-à-dire à la majorité des membres du syndicat représentant les deux tiers des voix\footnote{On relèvera avec intérêt l’arrêt de rejet de la cour de cassation en date du 21 juin 2000 (\no 98-20-897) dans une affaire « chaufferie de la porte de Bâle » : sur consultation du Professeur Giverdon les propriétaires et copropriétaires concernés ont constaté que l’ensemble constituait un ensemble immobilier au sens de l’article 1 al 2 de la loi soumis au statut de la copropriété à défaut de convention contraire et ont en conséquence chargé le Conseil de Surveillance d’élaborer un Règlement de copropriété, sans en subordonner l’application à un vote de l'assemblée générale ; la cour de cassation rejette le pourvoi contre l’arrêt de la cour de Colmar ayant considéré que l'assemblée générale avait délégué la mise en oeuvre des modalités pratiques du Règlement de copropriété, sans subordonner la mise en application à un vote de l'assemblée générale en sorte que le Règlement de copropriété ainsi établi s’imposait à tous sans nécessité d’adoption en assemblée générale.}. Il en résulte que dans ce cas, le règlement peut être imposé par une majorité à une minorité. C'est pourquoi, l'on admet qu'un règlement voté dans de telles conditions ne peut porter atteinte aux << droits acquis >> préexistants de chacun des copropriétaires et ne peut régir que "la jouissance, l'usage et l'administration des parties communes" (art. 26 b).
		
		\subsubsection{Imposition par le juge}
		
			Étant donné le caractère obligatoire du règlement, l'opinion dominante est qu'il faut alors s'adresser au juge ; c'est le Tribunal de Grande Instance qui, après avis d'expert, procédera à la rédaction du règlement.
			
			Cette solution a été consacrée par la Cour de cassation dans l’arrêt du 15 novembre 1989\footnote{Civ. 3ème, 15 novembre 1989, Bull. Civ. III \no214, D.l990, 195, note CAPOULADE et GIVERDON, Rep. Defrénois 1990, art.34802 \no64 obs. H.SOULEAU.}, déjà cité, qui a décidé :
			\begin{quote}
				<< {\itshape qu'à défaut d'accord entre les parties, le règlement de copropriété peut résulter d'un acte judiciaire constatant la division de l'immeuble dans les conditions fixées par la loi du 10 juillet 1965}. >>
			\end{quote}
			
			La décision rendue paraît en conformité avec l'article 3 du décret de 1967 qui précise que le règlement peut résulter d'un << acte judiciaire >>.
			
			C’est sans doute l’avis de la cour de cassation dans un arrêt de rejet Civ.3ème 13 septembre 2005\footnote{Civ 3\degres, 13 sep 2005, AJDI jan 2006 p. 34, note Capoulade.}. A l’ origine la copropriété (trois copropriétaires) était dépourvue de Règlement de copropriété ; une décision de justice définitive avait homologué un règlement rédigé par un expert judiciairement commis, mais n’avait pas été publié dix ans après la décision. Un des copropriétaires avait demandé au juge des référés de désigner un notaire pour mettre en forme le règlement et le publier. Décision favorable du juge des référés, confirmée par la Cour d’Aix en Provence qui considérait la décision du juge des référés « opportune dans l’intérêt du syndicat ». Pourvoi en cassation des deux autres copropriétaires au visa des articles 14 de la loi et 13 du décret. La cour rejette le pourvoi en considérant que la motivation de la cour était justifiée.
	
\section{Contenu du règlement de copropriété}
	
	Les dispositions que peut contenir le règlement de copropriété revêtent tantôt un caractère obligatoire, tantôt un caractère facultatif.
	
	Mais la liberté des rédacteurs d'introduire des dispositions qui diffèrent du contenu de la loi est assez restreinte, car l'article 43 édicte que sont réputées non écrites toutes les clauses contraires aux dispositions des articles 6 a 37 et 42 de la loi. Or ces articles contiennent l'essentiel de l'organisation et de l'administration de la copropriété.
	
	\subsection{Clauses obligatoires}
	
		Le caractère obligatoire de certaines dispositions résulte de l'article 8 de la loi du 10 juillet 1965 et de l'article 1er du décret du 17 mars 1967.
		
		Aux termes l'article 8 de la loi duquel le Règlement de Copropriété :
		\begin{itemize}
			\item détermine la destination des parties tant privatives que communes et les conditions de leur jouissance ;
			\item fixe les règles relatives à l'administration des parties communes.
		\end{itemize}
		
		L'article 1\ier{} du Décret reprend ces deux premiers points et ajoute la nécessité pour le Règlement de Copropriété de répartir les charges entre les lots de copropriété :
		\begin{quote}
			<< {\itshape Le Règlement de Copropriété mentionné à l'article 8 de la loi du 10 juillet 1965 susvisé comporte les stipulations relatives aux objets visés par l'alinéa 1er dudit article ainsi que l'état de répartition des charges prévu au dernier alinéa de l'article 10 de ladite loi}. >>
		\end{quote}
	
		Reprenons successivement ces trois points.
		
		\subsubsection{La destination des parties tant privatives que communes et conditions de leur jouissance}
		
			\paragraph{La destination des parties communes}
			
				La destination qui doit en être déterminée par le règlement, vise leur affectation à tel ou tel usage : aire de stationnement de véhicules, terrain de jeux pour enfants, cour, espaces verts \etc
				
				Les conditions de jouissance résultent de leur affectation, mais demandent parfois à être précisées. Il en est ainsi:
				\begin{itemize}
					\item si certaines parties communes sont réservées à l'usage de certains lots seulement (terrasses, parkings, jardins) ;
					
					\item du mode d'utilisation de certains locaux communs : loge de concierge, salle de réunion, locaux techniques ;
					
					\item d'interdictions et de limitations diverses concernant les parties communes telles que restrictions relatives à l'apposition d'enseignes ou de plaques professionnelles (Civ. 3ème, 18 juin 1975, Gaz. Pal. l975 II Somm. 224) --- une clause interdisant purement et simplement la pose d’enseignes alors que les locaux peuvent être affectés à usage commercial ne serait pas justifiée par la destination de l’immeuble\footnote{Cour d'Appel Metz, 5 sep 2013, \no 11/01701 ; Loyers et Copropriété jan 2014 comm. \no 34}
					
					\item la réglementation de la circulation ou du stationnement sur les parties communes (Civ. 3ème, 11 décembre 1973, J.C.P. 1974 II 17659, note GUILLOT), l'interdiction d'étendre du linge aux fenêtres (Civ. 3ème, 2~ novembre 1973, J.C.P. 1974 II 17644
				\end{itemize}
	
			\paragraph{La destination des parties privatives}
				
				Les modalités d’usage des parties privatives du lot ne relèvent pas, en effet, du droit discrétionnaire du copropriétaire. C'est ainsi que le règlement précisera quelle est la destination des locaux principaux (usage commercial, professionnel ou d'habitation bourgeoise) et accessoires (caves, greniers, celliers, garages, chambres de service).
				
				Les conditions de jouissance des parties privatives se traduisent par des clauses interdisant certaines activités ou comportements : activités gênantes par le bruit, l'odeur, la chaleur ou les trépidations (blanchisserie, restaurant, cours de piano ou de chant), la modification des fenêtres parties privatives ou porte-fenêtre \etc
				
				Il conviendra cependant de faire la distinction entre ce que Me BOCCARA\footnote{J CL Construction Fasc. 112 – Concession Immobilière \no 11} a appelé la \textbf{destination première} des lots, c'est-à-dire leur destination générale (habitation, mixte habitation et profession libérale, profession libérale, bureaux, commerces, artisanat, industrie) et la \textbf{destination seconde} des lots qui concerne la branche particulière d’activités (blanchisserie, confiserie, etc). Cette distinction pourrait également se faire en ce sens que la destination du lot visée par l’article 8, c’est la destination première, alors que l’affectation du lot selon sa destination constituerait en fait son affectation (ou destination seconde).
				
				Les copropriétaires peuvent demander au syndicat des copropriétaires la libération de leur lot privatif illicitement occupé par un transformateur EDF. Cette occupation trouvait sa source dans une convention conclue entre EDF et le promoteur, à laquelle le syndicat des copropriétaires était étranger.
				
				En statuant ainsi, sans rechercher si la convention conclue entre le constructeur et EDF relative à l'installation d'un transformateur avait été transmise au syndicat, la cour d'appel n'a pas donné de base légale à sa décision\footnote{Cass. Civ 3e 7 juillet 2010}.
		
		\subsubsection{La destination de l'immeuble.}
		
			Cette notion joue en effet un rôle capital en matière de copropriété puisque comme l'édicte l'alinéa 2 de l'article 8 de la loi du 10 juillet 1965 :
			\begin{quote}
				<< {\itshape Le règlement de copropriété ne peut imposer aucune restriction aux droits des copropriétaires en dehors de celles qui seraient justifiées par la destination de l'immeuble telle qu'elle est définie aux actes, par ses caractères ou sa situation}. >>
			\end{quote}

			Or ces << actes >> visés par le texte consistent essentiellement dans le règlement de copropriété.
			
			En tout cas, il ne faut pas confondre destination du lot (voir ci-dessus) et destination de l’immeuble, quand bien même la destination des lots peut être un élément constitutif de la destination de l’immeuble.
			
			\paragraph{Une notion vague et complexe}\footnote{On consultera avec le plus grand intérêt la Revue de Droit Immobilier du 3\degres Trimestre 1995 p. 407 à 469 qui reproduit intégralement les interventions consacrées à la notion de Destination de l'Immeuble à l'occasion de la journée Henri Souleau (qui fût professeur à l'I.C.H. jusqu'à sa mort en août 1993).}
			
				La destination de l'immeuble au nom de laquelle des limitations peuvent être imposées aux droits des copropriétaires est une notion floue et complexe.
				
				Dans les travaux préparatoires de la loi, il en a été donné la définition suivante:
				\begin{quote}
					<< {\itshape L'ensemble des conditions en vue desquelles un copropriétaire a acheté son lot, compte tenu de divers éléments, notamment de l'ensemble des clauses des documents contractuels, des caractéristiques physiques et de la situation de l'immeuble, ainsi que de la situation sociale de ses occupants}. >>
				\end{quote}
			
				C'est en se référant à la destination de l'immeuble que l'on qualifiera tel immeuble de << résidence de luxe >>, << d'immeuble bourgeois >>, d'immeuble de type H.L.M. ou Logéco, de centre commercial, etc.
				
				En réalité, la destination de l’immeuble est un « standard juridique » ayant pour fonction d’être un \textbf{mécanisme régulateur}, et d'assurer la pérennité des caractéristiques essentielles de l'immeuble et une certaine possibilité d'évolution en fonction des progrès techniques, du changement des besoins ou des mœurs.
				
				De ce fait, la destination de l'immeuble est une notion de fait soumise à \textbf{l'appréciation souveraine des juges du fond}. C'est la notion à appliquer pour déterminer celle des caractéristiques de l'immeuble autour de laquelle toutes les autres vont s'ordonner.
			
			\paragraph{Les éléments constitutifs de la destination de l’immeuble}
			
			La destination de l'immeuble tient nécessairement compte de la destination des parties privatives.
			
			Elle comporte :
	
			\subparagraph{Des éléments objectifs afférents a l'immeuble}
			
				Il s’agit notamment de ses caractères architecturaux, son aspect, son caractère plus ou moins luxueux, son standing, sa situation : qualité de l'environnement, caractéristiques du quartier et du voisinage.
			
			\subparagraph{Des éléments contractuels provenant des actes tels que la possibilité ou non d'exercer une activité commerciale}\footnote{Cf. Mr SIZAIRE La détermination contractuelle de la destination de l'immeuble (RD imm. 17 (3), juill- sep 1995 p. 475. }
			
				Étant d'ailleurs précisé comme l'indique Monsieur SIZAIRE, que si le règlement peut définir la destination de l'immeuble, aux termes de l'article 8 al 2 de la loi, << il ne peut imposer aucune restriction aux droits des copropriétaires en dehors de celles qui seraient justifiées par la destination de l'immeuble. >> En sorte que la détermination de la destination de l'immeuble par le Règlement de Copropriété $\dots$ se trouve limitée par la destination de l'immeuble elle-même !
				
				Monsieur SIZAIRE ajoute : << N'y a t’il pas là une contradiction ? Pas vraiment, mais le règlement - plus précisément son auteur - ne peut faire ce qu'il veut. Il doit respecter une \textbf{certaine cohérence et le contrôle de cette cohérence appartient aux tribunaux}. >>
				
				Au titre de la destination de l'immeuble, le règlement de copropriété comportera une clause générale indiquant quel est le genre ou le type de l'immeuble en copropriété :
				\begin{itemize}
					\item Immeuble d'habitation exclusivement bourgeoise (qui interdit toute autre utilisation que l'habitation)
					
					\item Immeuble d'habitation bourgeoise (qui, outre l'habitation autorise l'exercice d'activités professionnelles compatibles avec la tranquillité des habitants : médecin, avocat, notaire, expert comptable).
					
					En revanche, est incompatible avec une telle clause l'exercice de toute industrie, de tout commerce ou de toute profession de nature à gêner les autres copropriétaires : mécanicien prothésiste gênant les voisins par l'émission d'odeurs nauséabondes (Paris, 10 décembre 1966, Gaz. Pal. 1, 208), laverie automatique (T.G.I. Paris 8ème ch. 15 avril 1969 A.J.P.I. 1970, 136, obs. CABANAC), atelier de réparation de cyclomoteurs (Civ. 3ème, 22 février 1984, Administrer 1984 p.43, obs. GUILLOT).
					
					Il peut arriver qu'en présence d'une clause d'habitation bourgeoise, le règlement indique une liste de professions dont l'exercice est admis. Une telle énumération est considérée, sauf stipulation contraire, comme énonciative et non limitative. Sont donc autorisées les professions qui ne causent			
					pas plus de troubles que celles expressément visées dans le règlement (Civ. 3ème, 13 novembre 1975, Bull. Civ. III \no332).
					
					\item Immeuble mixte : d'habitation bourgeoise, professionnel et commercial.
					
					Le Règlement de Copropriété peut également stipuler que les professions libérales ne pourront être exercées dans le lot que si elles sont l'accessoire de l'habitation ou encore que leur exercice ne pourra se faire que dans un nombre limité de pièces. Ces clauses doivent être considérées comme valables, le juge du fond exerçant un contrôle de la véracité de cette affectation (PARIS 19\degres Chambre 27 mai 1992; Loyers et Copropriété octobre 92 \no 399)
					
					\item Immeuble de bureaux (ce qui exclut l'habitation)
					
					\item Centre commercial
					
					\item Résidence services (ou unités retraite) pour habitants du troisième âge ; laquelle suppose, outre l'habitation la prestation de certains services médicaux, de restauration, d'animation ou de transport, dont il a été jugé à plusieurs reprises qu'ils étaient compatibles avec l'objet de la Copropriété.
				\end{itemize}
			
			\paragraph{La destination de l'immeuble, notion évolutive.}
		
				Bien souvent se pose la question de savoir si la destination de l'immeuble est figée à l'époque de l'établissement du Règlement de Copropriété, ou bien si au contraire elle peut présenter un caractère évolutif.
				
				Le problème se posera par exemple en cas d'évolution du quartier, ou encore en fonction du plus ou moins bon état d'entretien, ou même, en considération de la situation sociale des occupants.
				
				Deux décisions de la Cour de PARIS (PARIS 19\degres Chambre 27 mars 1992 et PARIS 23\degres Chambre 3 avril 1992, commentés à la Revue de Droit Immobilier 92.370) ont admis que la destination de l'immeuble pouvait présenter un caractère évolutif.
		
		\subsubsection{L'administration des parties communes}

			Le règlement de copropriété prévoit et organise le fonctionnement du syndicat et les pouvoirs du syndic.
			
			A cet égard, la liberté des rédacteurs du règlement est extrêmement restreinte car les dispositions de la loi sont sur ces questions presque toutes impératives : il y a donc sur ce point reproduction des textes légaux.
			
			Cette rubrique comprend les règles concernant les assemblées générales de copropriété (art.22 al.1 de la loi), les règles de convocation des copropriétaires, de tenue des séances de l'assemblée générale, des modalités de délégation du droit de vote. Il prévoit la périodicité des assemblées au moins annuelles.
			
			Il détermine, en se conformant à l'article 18 de la loi, les pouvoirs du syndic et ceux du conseil syndical.
			
			Il prévoit expressément que la gestion de l'immeuble pourra éventuellement être assurée par un syndicat coopératif.
			
			Toutes ces dispositions concernant l'administration des parties communes seront analysées à propos des différents organes qu'elles régissent.
		
		\subsubsection{La répartition des charges communes}
		
			Aux termes de l'article 10 de la loi,
			<< {\itshape le règlement fixe la quote-part afférente à chaque lot dans chacune des catégories de charges}. >>
			
			Les critères de répartition entre les différentes catégories de charges sont fixés par les alinéas 1 et 2 de l'article 10 et doivent figurer dans le règlement.
			
			\begin{enumerate}[label=alpha*)]
				\item Les charges relatives à la conservation, à l'entretien et à l'administration des parties communes que l'on appelle aussi charges générales communes se répartissent conformément << aux valeurs relatives des parties privatives comprises dans les lots, tel que ces valeurs résultent des dispositions de l'article 5. >>
				
				Ces charges communes générales sont donc proportionnelles aux tantièmes de copropriété ou aux quotes-parts de parties communes : il en est ainsi par exemple des honoraires du syndic.
	
				\item Les charges entraînées par les services collectifs ou les éléments d'équipement commun se répartissent en fonction de l'utilité que ces services ou équipements présentent à l'égard de chaque lot.
				
				Ainsi en est-il par exemple des charges d'ascenseur plus fortes pour un lot situé au sixième étage que pour celui qui se trouve au premier.
				
				Le règlement de copropriété doit comporter un état de répartition des charges qui fixe la quote-part afférente à chaque lot dans chacune des catégories de charges ou à défaut les bases selon lesquelles la répartition est faite pour une ou plusieurs catégories de charges (art.l al.3 D.17/3/67).
			\end{enumerate}
	
	\subsection{Clauses facultatives}
	
		\subsubsection{Clauses ayant pour objet d’écarter certaines dispositions du statut qui ne sont pas d’ordre public (à savoir les articles 1 à 5 et 38 à 41).}
		
			Il en est ainsi de :
			\begin{itemize}
				\item la répartition des éléments de l'immeuble entre parties privatives et parties communes (art.2 et 3) ;
				\item la détermination de la quote-part des parties communes : ces quotes-parts ne sont pas nécessairement calculées selon le mode de calcul de l'article 5 (consistance, superficies et situation des lots sans égard à leur utilisation), alors que ce texte est impératif pour le calcul de la quote-part des charges générales ; de la reconstruction de l'immeuble au cas de destruction.
			\end{itemize}
		
		\subsubsection{Clauses limitant les droits des copropriétaires sur leurs lots lorsqu'elles sont justifiées par la destination de l'immeuble}
		
			Remarquons que ces hypothèses sont assez rares et concernent généralement des \textbf{immeubles de grand standing}, comportant un nombre limité de copropriétaires et étant situés dans des quartiers résidentiels et calmes ou dans des espaces verts tranquilles. Afin de conserver cette qualité d'habitation, certaines restrictions peuvent être introduites concernant les droits des copropriétaires. Toutefois, la jurisprudence reconnaissant la validité de telles clauses ultra protectionnistes sont rares, les Tribunaux de façon générale montrent une attitude réservée à l'égard de ces clauses qu'ils assimilent le plus souvent à une atteinte inadmissible au libre droit de jouissance et de disposer de son lot que l'article 9 de la loi du 10 juillet 1965 reconnaît aux copropriétaires.
	
			\paragraph{Restrictions au droit de disposer de son lot}\footnote{Cf Henri Souleau : Droit de disposer d'un lot dans un immeuble en copropriété (Etudes Offertes à Jacques Flour publiées au Defrénois).}
			\begin{itemize}
				\item Clauses d'agrément subordonnant l'aliénation à une autorisation de l'assemblée.
				\item Clauses interdisant la vente séparée de chambres de service afin de ne pas multiplier le nombre des occupants (Civ. 3ème, 10 mars 1981, J.C.P. 1982 II 19765, note GUILLOT ; Rep. Defrénois 1981, art.32797, obs. H.SOULEAU; Paris, 23 ch. B, 19 juin 1985, D.1985, I.R. 425 obs. GIVERDON); pacte de préférence dans un immeuble familial.
				\item Clauses interdisant la division des lots afin de maintenir le standing de l'immeuble (Civ. 3ème, 9 mars 1982, Administrer, oct.1982 p.28 note GUILLOT; Paris 23è ch. B, 19 juin 1985, I.R.425, obs. GIVERDON).
			\end{itemize}
			
			\paragraph{Clauses restreignant les modalités de jouissance du lot}
			\begin{itemize}
				\item Clauses restreignant la liberté de louer en vue de préserver les conditions d'occupation d'origine ou d'interdire des modes d'occupation non-conformes au type d'habitat dans l'immeuble : interdiction de louer en meublé, de diviser les lots en vue de locations séparées\footnote{Paris 19 juin 1985 D.1985 I.R. 425, obs. GIVERDON}, interdiction de louer des chambres de service à des personnes étrangères à la copropriété\footnote{Paris 12 février 1976, D.1977 I.R.42; Paris 23\degres Chambre 4 juin 1997 déclare licite la clause interdisant de louer des chambres de service à des personnes étrangères, compte tenu du standing de l’immeuble (51av Georges Mandel).}.
				
				\item Clauses limitant ou interdisant l'exercice de certaines activités gênantes par le bruit ou l'odeur : par exemple, l'installation au rez-de-chaussée d'un immeuble de standing d'un commerce de bar, salon de thé ouvert jusqu'à deux heures du matin où l'on sert des aliments cuisinés\footnote{Civ. 3ème 14 janvier 1987, Gaz. Pal.1987 Pan.14}, ou installation dans l'immeuble d'une blanchisserie qui par le bruit, l'odeur et les émanations est de nature à incommoder les copropriétaires\footnote{Civ. 3ème 18 février 1987, Rev. Loyers 1987 p.221}.
				
				\item Clauses destinées à préserver l'harmonie de l'ensemble immobilier en interdisant certains aménagements de parties privatives (Paris 23\degres Chambre, 28/10/92 LOY ET COP jan 93 \no 32).
			\end{itemize}
		
		\subsubsection{Dispositions diverses}

		Telles que conditions de souscription de polices d'assurance, d'emprunts hypothécaires, élection de domicile en cas de litige, institution d'une procédure de conciliation amiable, clause pénale etc...
	
	\subsection{Clauses prohibées}
	
		\subsubsection{Les dispositions contraires à l'ordre public ou aux bonnes mœurs}
		
			\begin{itemize}
				\item Clause rendant un lot totalement inaliénable, contraire à la liberté du droit de disposition.
				
				\item Clause obligeant un copropriétaire à céder son lot, notamment à titre de sanction, contraire à l'article 546 du Code civil selon lequel << nul ne peut être contraint de céder sa propriété. >>
				
				\item Clause compromissoire interdite par l'article 1006 du Code de procédure civile dans les rapports civils.
				
				\item Clause interdisant la détention d'un animal familier dans les locaux d'habitation, contraire à l'article 10 1) de la loi du 10 juillet 1970 :
				\begin{quote}
					<< {\itshape Est réputée non écrite toute stipulation tendant à interdire la détention d'un animal dans un local d'habitation, dans la mesure où elle concerne un animal familier. Cette détention est toutefois subordonnée au fait que ledit animal ne cause aucun dégât à l'immeuble ni aucun trouble de jouissance aux occupants de celui-ci}. >>
				\end{quote}
				
				Reste à savoir si un perroquet (dont la détention est souvent interdite par le Règlement de Copropriété) est ou non un animal familier.
			\end{itemize}
		
		\subsubsection{Les clauses contraires aux dispositions d'ordre public de la loi du 10 juillet 1965 et du décret \no67-223 du 17 mars 1967}
		
			L'article 43 de la loi répute en effet non écrites les clauses contraires aux articles 6 à 37 et 42 de la loi et ceux du décret pris pour leur application. Il s'agit, à titre d'exemple :
			\begin{itemize}
				\item clause de répartition des charges communes non conforme aux critères posés par l'article 10 de la loi ;
				
				\item clause interdisant à un copropriétaire de déléguer son droit de vote aux assemblées générales à une personne étrangère à la copropriété (art.22) ;
				
				\item clause refusant d'instituer un conseil syndical (art.21) ;
				
				\item clause limitant les pouvoirs du syndic tels qu'ils résultent de l'article 18 ;
				
				\item clause excluant toute indemnisation pour les copropriétaires devant subir dans ses parties privatives des travaux sur parties communes, contrairement aux dispositions de l’article 9 de la loi du 10 juillet 1965\footnote{Paris 23\degres Ch B, 28 mai 2009 ; Loyers et Copropriété nov 2009 \no 269}.
			\end{itemize}
		
		\subsubsection{Les clauses restrictives non justifiées par la destination de l'immeuble}
		
			\paragraph{Clauses restrictives au droit d’usage ou au droit de disposer}
			
			On peut citer à cet égard toutes les clauses limitant le droit des copropriétaires d'user de jouir et de disposer de leurs lots si de telles restrictions ne sont pas justifiées par la sauvegarde de la destination de l'immeuble :
			\begin{itemize}
				\item o clauses de préférence en cas de vente d'un lot dans un immeuble de standing non exceptionnel ;
				
				\item clauses interdisant la division d’un lot si le standing de l’immeuble ne le justifie pas\footnote{Civ 3\degres 26 mai 1988 JCP N 89 PRAT p 9.Dans le même sens et avec les mêmes termes voir Civ 3\degres 5 juillet 1989, Bull. \no 154.} ;
				
				\item interdiction de la circulation ou du stationnement de véhicules à moteur qui, dans un immeuble où sont autorisées les activités commerciales, empêchent les commerçants d'être livrés\footnote{Paris 8è ch.B, 9 avril 1987, D.1987, I.R.130 (en l’espèce la copropriété avait étayé un plancher haut dans un local privatif rendant ce local inhabitable pendant un mois.} ;
				
				\item clause imposant au copropriétaire de faire ses travaux privatifs par un entrepreneur déterminé ou de confier la gestion de son lot au syndic de l'immeuble.
				
				\item clause interdisant l'accès des parties communes d'un bâtiment aux copropriétaires des autres bâtiments alors que ces bâtiments n'ont pas été constitués en parties communes spéciales\footnote{Civ. 3° 30 juin 1992 - Bull. III p 40 n° 230} ;
				
				\item clause interdisant la location en meublée qui ne peut être justifiée par la destination bourgeoise de l’immeuble, dès lors que le Règlement de copropriété autorise les professions libérales ; « l'exercice d'une telle activité entraîne des inconvénients similaires à ceux dénoncés par le syndicat des
				copropriétaires pour la location meublée de courte durée et l'existence de nuisances fautives des locataires n'est pas établie »\footnote{Civ. 3\degres 8 juin 2011 – \no 694 FSPB}.
			\end{itemize}
			
			Étant observé toutefois qu’il ne peut exister de réponse de principe : la destination de l'immeuble doit s'apprécier \emph{in concreto}, en fonction d'éléments extérieurs au règlement de copropriété tenant à la situation de l'immeuble, son environnement ou ses caractéristiques\footnote{Civ. 3\degres - 9-6-2010 \no 09-14.206 : Bull. civ. III \no 116}.
			
			\paragraph{Clauses restreignant la concurrence}
			
			Ces clauses sont considérées par la jurisprudence comme << étrangères >> à la destination de l'immeuble dès qu’elles n'ont pas pour but de préserver un intérêt collectif, mais qu'elles sont stipulées pour la protection d'intérêts particuliers comme celui du promoteur vendeur ou celui des commerçants déjà installés dans l'immeuble.
			
			Ces clauses sont donc illicites\footnote{Civ. 3ème 11 mars 1971 (2 arrêts), J.C.P.1971 II 16722, concl. PAUCOT, note GUILLOT; Civ. 3ème 15 octobre 1974 J.C.P.1974 II note GUILLOT, Rep. Defrénois 1975, art.30965 obs. H.SOULEAU; Civ. 3ème 29 mai 1979, Rep. Defrénois l979, art.32162, obs.H.SOULEAU}.
			
			Mais la Cour de cassation admet qu'insérées dans l'acte de vente les clauses de non concurrence sont valables si elles n'ont pas pour but de réaliser une fraude à la loi. Est donc licite la clause de l'acte de vente stipulant que l'acquéreur ne pourra exploiter dans le lot vendu qu'un commerce déterminé, le vendeur s'interdisant pour sa part de vendre un autre lot à un commerçant concurrent\footnote{Civ. 3ème 9 novembre 1982, Bull. Civ. III \no215, Rep. Defrénois 1583, art.33093, obs. H.SOULEAU.}.
			
			Enfin, sont considérées comme licites les clauses d'intangibilité des commerces figurant dans les règlements de copropriété des centres commerciaux éloignés des villes, au motif qu'il est alors conforme à la destination de l'immeuble de maintenir la diversité des commerces qui y sont exploités\footnote{Civ. 3ème, 25 novembre 1980, Bull. Civ. III \no184, Rep. Defrénois 1981, art.32608 obs. H.SOULEAU}. Mais ici encore, eu égard aux circonstances, le juge peut annuler la clause en considérant que la concurrence est plutôt de nature à dynamiser le centre commercial.
			
			\paragraph{Clauses de solidarité}

			Il s'agit de la clause selon laquelle en cas de vente d'un lot l'acquéreur sera solidairement tenu avec le vendeur et vis à vis du syndicat de toutes sommes afférentes au lot vendu et non acquittées au moment de la mutation, par exemple de l'arriéré des charges dues par le vendeur. Cette clause facilitant le recouvrement des charges était fort appréciée des syndics.
			
			Après une vive discussion doctrinale, la Cour de cassation a décidé qu'une telle clause insérée dans le règlement était illicite, dès lors que le syndicat dispose de l'hypothèque légale instituée par l'article 20 de la loi pour recouvrer les sommes restant dues par le vendeur d'un lot\footnote{Civ. 3ème ler juillet 1980, D.1981 I note GIVERDON et CAPOULADE, Rep. Defrénois 1981, art.32608, obs. H.SOULEAU}.
			
			Cependant, dans les rapports entre le cédant et le cessionnaire, il est licite de stipuler la solidarité entre vendeur et acheteur dans l'acte de vente.
			
			Toutefois, si la vente a lieu sur adjudication l'adjudicataire ne peut être tenu au paiement de l'arriéré des charges, alors même que cette obligation figurait en termes non équivoques dans le Cahier des Charges (Civ. 3\degres 6 mars 1991; Civ 3\degres 17 juin 1992; Civ 2\degres 2 décembre 1992. Pour ces trois arrêts voir IRC 1993.342) :
			\begin{quote}
				<< {\itshape En matière de saisie immobilière, le cahier des charges ne peut modifier directement ou indirectement l'ordre dans lequel le prix des biens du débiteur, qui constitue le gage commun des créanciers, doit être réparti entre eux}. >>
			\end{quote}
		
			Est réputée non écrite la clause du règlement de copropriété qui exclut toute indemnisation pour le copropriétaire devant subir, dans ses parties privatives, des travaux sur les parties communes. Cette clause ne peut priver le copropriétaire de son droit à indemnisation pour le préjudice subi du fait des travaux\footnote{CA Paris 28 mai 2009 JD 2009-377938}.
			
			La clause du règlement de copropriété faisant supporter une surprime d’assurance par le copropriétaire d’un lot dans lequel est exploitée une discothèque, doit être réputée non écrite : le paiement des primes d’assurance souscrites pour les parties communes et privatives de l’immeuble constitue une charge relative à la conservation, à l’entretien et l’administration des parties communes\footnote{Cass. Civ. 3e 17 mars 2010}.
			
			Les primes d’assurances souscrites dans l’intérêt de l’ensemble des copropriétaires d’un immeuble comprenant une galerie marchande constituent des charges générales à répartir entre tous les lots : la répartition des charges spéciales à la galerie marchande, ne peut s’appliquer aux primes d’assurance.
			
			Il en est de même des dépenses afférentes au responsable du service de sécurité de l’ensemble immobilier\footnote{Cass. Civ. 3e 4 juin 2009}.

			Doit être réputée non écrite par application de l’article 43 alinéa 1er de la loi, la clause du règlement de copropriété donnant tous pouvoirs au syndic pour régulariser à première demande d’une Société et à son profit une convention d’occupation précaire sur un local, partie commune, pour une durée maximum de 10 ans moyennant une redevance annuelle déterminable ou pour lui vendre dans ce même délai ce local, pour un prix ferme et définitif.
			
			Cette clause a pour effet de priver par avance l’assemblée générale des pouvoirs de disposition et d’administration sur les parties communes qu’elle tient des règles d’ordre public des articles 17, 26 et 24 de la loi\footnote{Cass. Civ. 3e 11 février 2009}.
			
			La clause du règlement de copropriété prévoyant que les appartements ne pourront être consacrés à la location meublée sans l’autorisation de l’assemblée générale, laquelle autorisation pourra être retirée par l’assemblée sans que celle-ci ait à motiver sa décision et sans que le propriétaire visé puisse prétendre à une indemnité, donne à l’assemblée générale le pouvoir discrétionnaire d’autoriser un copropriétaire à louer ses lots en meublés et de retirer à tout moment cette autorisation.
			
			Cette clause restreignant les droits des copropriétaires sur leurs parties privatives, non justifiée par la destination contractuelle de l’immeuble, est réputée non écrite\footnote{CA Paris 3 février 201,0 \no 09/00448, JD 2010-380757}.
			
			Si chaque copropriétaire est libre de subdiviser son lot sans l’autorisation de l’assemblée générale, dès lors que cette subdivision n’est pas contraire à la destination de l’immeuble, c’est à la condition que le règlement de copropriété ne comporte ni interdiction, ni restriction ou que celles-ci aient été jugées inopérantes\footnote{Cass. Civ. 3e 25 février 2010}.
			
			La clause du règlement de copropriété subordonnant la possibilité de diviser les lots à la nécessaire autorisation de l’assemblée générale et en interdisant la location en chambre meublée ainsi que l’exploitation d’une pension de famille, a pour finalité de conserver à l’immeuble son caractère résidentiel tenant compte de son environnement et de son standing, auquel porterait atteinte notamment la réduction des surfaces des appartements et l’augmentation corrélative du nombre des appartements et des occupants est parfaitement conforme à la destination de l’immeuble de grand standing\footnote{CA Versailles 16 juillet 2009, \no 09/03168, JD 2009-379397}.
	
\section{Modification et adaptation du règlement de copropriété}

	\subsection{Les règles permanentes concernant la modification du règlement de copropriété}
	
		La loi a prévu la possibilité de modifier le règlement de copropriété. Mais, étant donné que le règlement constitue la charte de la copropriété, il est nécessaire qu'il présente une stabilité suffisante et que sa modification soit soumise à des conditions strictes.
		
		C'est pourquoi deux conditions sont exigées, l'une concernant la majorité requise, l'autre l'objet de la modification.
		
		Bien évidemment la modification du Règlement de copropriété ne peut être valablement décidée, en application de l’article 17 « d’ordre public » que si cette modification est votée en assemblée générale : il ne peut être suppléé à un vote de l’assemblée générale par aucun autre moyen, même un demande faite au notaire par l’ensemble des copropriétaires\footnote{Civ. 3\degres Ch. 8 juin 2011 (10-18.220) – Arrêt \no 700 FSPB}.
		
		\subsubsection{Les modifications relevant de l’article 26 de la loi}
	
			La modification du règlement de copropriété ne peut être décidée qu'à la double majorité de l'article 26, c'est-à-dire à la majorité des copropriétaires représentant les deux tiers des voix.
		
		\subsubsection{Les modifications relevant de l’unanimité}
			
			L'assemblée \textbf{ne peut modifier, à quelque majorité que ce soit, les droits des copropriétaires sur leurs parties privatives}, en modifiant la destination de ces parties ou les modalités de leur jouissance.
			
			A ce titre, l'assemblée ne peut, sauf unanimité :
			\begin{itemize}
				\item modifier les quotes-parts de droits des copropriétaires sur les parties communes ou le nombre de voix attaché aux lots ;
				\item supprimer un service collectif (service de conciergerie par exemple)
				\item supprimer un élément d'équipement commun sans prévoir son remplacement (Boite aux lettres).
			\end{itemize}

		\subsubsection{Les modifications des charges}
		
			L'article 11 de la loi pose le principe selon lequel l'état de répartition des charges ne peut être modifié qu'à l'unanimité des copropriétaires $\dots$
			
			Sauf les exceptions des articles :
			\begin{itemize}
				\item 11 (en cas de travaux ou d'actes d'acquisition ou de dispositions, en cas d'aliénation séparée d'une ou plusieurs fractions d'un lot) ;
				\item 25 f (en cas de changement d'usage d'une partie privative).
			\end{itemize}
			
			Dans tous les autres cas, la modification ne peut intervenir qu'à l'unanimité ou par décision du juge, que ce soit en cas de lésion (article 12 de la loi) ou en cas de nullité de la répartition (article 43 de la loi).
			
			Les modifications apportées, par une décision d’assemblée générale, au règlement de copropriété concernant la répartition des charges sont inopposables à l’acquéreur si elles n’ont pas été publiées au fichier immobilier ou s’il n’est pas expressément mentionné dans l’acte d’achat de l’acquéreur qu’il a adhéré aux obligations qui en résultent\footnote{Cass. Civ. 3e 8 septembre 2009}.
			
		\subsubsection{L’objet des modifications}
		
			Sauf unanimité, la modification ne peut concerner que la jouissance, l'usage et l'administration des parties communes. En application de ces prescriptions, le syndicat peut :
			\begin{itemize}
				\item établir ou modifier les périodes de chauffage de l'immeuble : T.G.I. Seine 13 mars 1952, Gaz. Pal. 1952 1.372 ;
				
				\item décider d'aménager un parking dans la cour commune : Civ. 3ème, 19 décembre 1978, J.C.P. 1979 IV p.72, ou autoriser les détenteurs de voitures à les garer dans la cour commune - T.G.I. Paris 23 avril 1976, D.1976 I.R. 312 ;
				
				\item réglementer l'accès de l'immeuble\footnote{Civ. 3ème, 19 décembre 1978, D.1979 I.R. obs. GIVERDON} ;
				
				\item réglementer la convocation et la tenue des assemblées générales sous réserve de ne pas porter atteinte aux dispositions impératives de la loi ;
				
				\item établir un \emph{Règlement Intérieur}, sous réserve que ce Règlement Intérieur ne soit pas en contradiction avec les dispositions du Règlement de Copropriété ou n'impose pas de restrictions nouvelles aux modalités de jouissance de son lot par le copropriétaire --- la pratique révèle en effet que le plus souvent les copropriétaires de locaux d'habitation votent un Règlement Intérieur pour << canaliser >> les << débordements >> qu'ils imputent à l'occupant des locaux commerciaux. Si de tels débordements peuvent faire l'objet de sanctions au << coup par coup >> et aboutir à une éventuelle interdiction d'exercer, ils ne sauraient justifier le vote de disposition restreignant ses droits sur les parties communes ;
				
				\item autoriser la mise en place d'étalages extérieurs dans une galerie marchande, partie commune, pendant les heures d'ouverture des magasins\footnote{Civ. 3ème 9 juillet 1986, Bull. Civ. III \no106, Rep. Defrénois 1986 art.33825 obs . H . SOULEAU}.
			\end{itemize}
			
			Les modifications du règlement de copropriété ne sont opposables aux acquéreurs qu’à dater de leur publication au fichier immobilier\footnote{Cass. Civ 3e 22 septembre 2009}.
		
	\subsection{L’adaptation du règlement (art 24 f) de la loi de 1965)}
	
		Article 24 f) de la loi du 10 juillet 1965 (Modifié par la loi ALUR)
		Sont notamment adoptées (à la majorité de l’article 24) f) les adaptations du règlement de copropriété rendues nécessaires par les modifications législatives et réglementaires intervenues depuis son établissement. La publication de ces modifications du règlement de copropriété sera effectuée au droit fixe.
		
		\subsubsection{Principe}
		
			Il s’agissait initialement d’une disposition provisoire – inscrite dans l’article 49 de la loi - devant permettre une « mise en conformité » des règlements de copropriété notamment aux modifications législatives intervenues lors de la loi SRU du 13 décembre 2000.
			
			De prorogation en prorogation, ce dispositif transitoire a fini par devenir permanent, a fin d’inciter les Syndicat à mettre leurs règlement de copropriété en « conformité » avec les évolutions législatives les plus récentes, pour la parfaite information des copropriétaires.
			
			Toutefois ce texte s’avère assez peu ambitieux dans ses objectifs et finalement insuffisant pour « nettoyer » un règlement de copropriété de toutes les clauses douteuses ou réputées non écrites du fait de l’évolution de la Jurisprudence. En effet, ce dispositif est encadré strictement par le législateur.
			
			\paragraph{Il s’agit d’une adaptation « nécessaire »}
			
				Ce terme ne prête pas à discussion : adapter, c’est rendre conforme en sorte, dans le cas présent de faire disparaître les contradictions entre le texte du règlement de copropriété et celui de la loi (et du règlement d’administration public du 17 mars 1967, pris en application de la loi). En outre, l’adaptation doit être « nécessaire » $\dots$
			
			\paragraph{Aux modifications législatives}
		
				26 lois successives du 28 décembre 1966 au 28 décembre 2016, dont les lois de 1985 (dite Loi \nom{Bonnemaison}), de 1994 (Loi sur l’Habitat) et 2000 (loi SRU) sont les principales $\dots$ ont modifié la loi de 1965.
				
				A ces modifications législatives il convient d’ajouter les 23 modifications apportées ar décret au Décret du 17 mars 1967 jusqu’au 30 décembre 2015 ; il convient également d’ajouter les 6 Ordonnances ayant directement modifié la loi \no65-557 du 10 juillet 1965 dont l’Ordonnance du 10 février 2016 portant réforme des contrats, du régime général et de la preuve des contrats ; il faut également ajouter les deux arrêtés dont l’arrêté comptable ! En effet, tous ces textes concernent directement le statut de la copropriété et font partie du corpus législatif visé par l’article 24 f).
			
			\paragraph{Intervenues depuis l’établissement du règlement de copropriété}
			
				Les adaptations autorisées sont seulement les adaptations rendues nécessaires par les lois publiées depuis l’établissement du règlement de copropriété.
		
				La date à prendre en compte est donc celle de la rédaction initiale du règlement de copropriété et non la date de ses modificatifs successifs. Si un règlement de copropriété a été établi postérieurement à la loi de 1965 sans respecter pour autant les dispositions de cette loi, l’article 24 f) ne permet pas de le rendre conforme à la loi.
			
		\subsubsection{Les adaptations possibles au titre de l’article 24 f}
		
			\paragraph{En présence d’un texte littéralement contraire à la loi postérieure}
			
				Il est des cas où l’adaptation nécessaire est évidente : lorsque le texte du règlement de copropriété est littéralement contraire au texte de la loi ou à celui du décret de 1967 pris en application de la loi.
				
				Si un règlement de 1950 dit que les assemblées sont convoquées huit jours avant la date prévue pour la réunion, il s’agit d’une disposition qui doit être adaptée pour tenir compte du texte adopté en 1965 soit postérieurement à l’établissement du règlement de copropriété d’origine.
				
				Mais si la portée de l’article 24 f) est réduite simplement à l’obligation de recopier la loi sur la copropriété dès lors que le texte en a été modifié par un nouveau texte législatif ou réglementaire, cette disposition est sans grand intérêt.
			
			\paragraph{En présence de dispositions nouvelles}
			
				Depuis la publication de la loi de 1965 de nombreuses dispositions ont été ajoutées au texte législatif :
				\begin{itemize}
					\item le privilège du syndicat ;
					\item la procédure applicable aux copropriétés en difficulté ;
					\item la procédure de recouvrement des provisions ;
					\item \etc
				\end{itemize}
			
				Si le Règlement de Copropriété est antérieur à ces nouvelles dispositions, il convient de les insérer au titre des modifications de l’article 24 f)
			
			\paragraph{En présence d’un texte contraire a la jurisprudence prise en application de la loi publiée après l’établissement du règlement de copropriété}

				Par exemple nous avons vu le sort que les tribunaux réservent aux clauses de solidarité contenues dans tous les bons règlements de copropriété depuis les deux arrêts du 1er juillet 1980. Ne doit-on pas dès lors supprimer ces clauses par l’adaptation prévue à l’article 24 f) ?
				
				Les copropriétaires auront du mal à comprendre que l’on a confié à un technicien le soin de mettre le règlement de copropriété en conformité avec la loi $\dots$ mais que ce technicien a laissé des clauses qui ne peuvent pas s’appliquer !
				
				Pourtant, il convient de ne pas céder à la tentation et ce pour deux raisons essentielles :
				\begin{itemize}
					\item L’article 24 f) ne traite que des adaptations rendues nécessaires par les lois postérieures ; c’est donc aller bien au-delà du texte que tenir compte également de la jurisprudence.
					\item S’il est admis que la jurisprudence est source de droit, il n’en demeure pas moins qu’il n’existe pas en droit français d’arrêts de règlement. Chaque décision s’applique à une espèce déterminée, contrairement à la loi, la jurisprudence n’a pas vocation à l’universalité.
				\end{itemize}
			
			\paragraph{L’adaptation du règlement de copropriété et répartition des charges}
			
				Nous arrivons ici à la question essentielle : peut-on, sous couvert d’adaptation nécessaire du règlement de copropriété, modifier la répartition des charges au motif que la répartition figurant au règlement de copropriété n’est pas conforme aux nouvelles règles de droit ?
				
				Notons tout d’abord que ces règles n’ont pas changé depuis 1965. En conséquence, la question de l’aggiornamento des charges dans le règlement de copropriété ne peut concerner que les règlements de copropriété antérieurs à la loi du 10 juillet 1965.
				
				En revanche les répartitions de charges antérieures à 1965, sauf hasard heureux, ne faisaient, par définition, pas application des critères de l’article 5 de la loi de 1965. Dès lors que ces critères n’ont pas été appliqués, la publication de la loi du 10 juillet 1965 semble « rendre nécessaire » l’adaptation de ces charges.
				
				Pourtant plusieurs arrêts de cours d’appel avaient condamné une telle révision, pour un règlement rédigé sous l’empire de la loi de 1938 (CA Aix en provence 23 avril 2010 JD 2010-012830,) et dans le même sens BORDEAUX 26 FEVRIER 2010 JD 2010-003758 ; VERSAILLES 12 AVRIL 2010 JD 2010-011329

				\subparagraph{La question a fait l’objet d’un arret de la cour de cassation du 23 mai 2012}\footnote{civile 3\degres Chambre, 23 mai 2012, \no de pourvoi: 10-28619 ; Loyers et Copropriété sep 2012 \no 244}
			
				Le Règlement de copropriété avait été établi en février 1965, donc avant la loi du 10 juillet 1965 et dispensait le propriétaire du rez-de-chaussée de participer aux charges d’ascenseur alors que celui-ci desservait les sous-sols. L’assemblée générale en application de l'article 49 (aujourd’hui 24 f) de la loi du 10 juillet 1965, a décidé d'adapter le règlement de copropriété aux dispositions législatives en vigueur et avait voté en 2006 cette adaptation comportant une nouvelle répartition de charges d’ascenseur. Un copropriétaire opposant avait demandé au Tribunal l’annulation de ce vote en soutenant que la « répartition des charges de copropriété ne peut être modifiée qu'à l'unanimité des copropriétaires et ne peut relever de la procédure simplifiée prévue à l'article 49 de la loi du 10 juillet 1965 ».
				
				La Cour d’Appel de Pau avait rejeté le recours du copropriétaire.
				
				La cour de cassation rejette le pourvoi, au motif que la Cour d’Appel « a exactement retenu qu'il y avait lieu, en application de la loi dite SRU du 13 décembre 2000, de réexaminer, à la majorité de l'article 24 de la loi du 10 juillet 1965, cette disposition au regard de l'article 10 de la même loi qui dispose que les copropriétaires sont tenus de participer aux charges entraînées par le services collectifs et les éléments d'équipement communs en fonction de l'utilité qu'ils présentent à l'égard de chaque lot ».
				
				Bien que cet arrêt ne soit pas publié au Bulletin de la Cour de Cassation, il retient nettement la possibilité pour l’assemblée générale de modifier les charges en application de l’article 24 f) lorsque ces charges ont été établies avant 1965 … en tout cas s’agissant de charges afférentes aux éléments d’équipement commun dès lors que le Règlement de copropriété d’origine ne faisait pas application du critère de l’utilité. La solution pourrait être différente s’agissant des charges communes générales dès lors qu’il y aurait seulement un calcul non conforme et non pas la méconnaissance de l’un des critères légaux.
				
				Dans le même sens on citera un arrêt (de rejet) de la 3ème chambre civile du 8 avril 2014 \no de pourvoi: 13-11633, ayant validé une ventilation des charges d’un Règlement de copropriété antérieur à 1965 faite en tantièmes généraux, dès lors que « que par application de l'article 10 de la loi du 10 juillet 1965, d'ordre public, cette globalisation des charges était devenue impossible, une distinction devant être opérée entre les charges relatives à la conservation, l'entretien et l'administration des parties communes et celles entraînées par les services collectifs et les éléments d'équipement communs, la cour d'appel, qui a relevé, procédant à la recherche prétendument omise, que l'article 16 du nouveau règlement de copropriété appliquait cette distinction et reprenait par ailleurs l'énumération des charges générales prévues dans le règlement initial sans modifier la clé de répartition qui y était prévue. » 
	
\section{Effets du règlement de copropriété}
	
	\subsection{Les effets a l'égard des copropriétaires}
	
		\subsubsection{Effet Obligatoire à l’encontre de tout copropriétaire}
		
			Le règlement de copropriété a un effet obligatoire pour tous les copropriétaires. Il s'ensuit que les copropriétaires sont tenus d'exécuter les diverses charges et obligations que leur impose le règlement sans pouvoir s'en dégager par une décision unilatérale, quelle que soit la raison invoquée : répartition prétendument inéquitable des charges, non-usage des services communs, infractions commises par certains copropriétaires.
			
			Il en sera de même de toute décision d’assemblée générale modifiant le Règlement de Copropriété\footnote{Du moins tant qu’un tribunal n’en aura pas relevé la nullité ou l’inexistence.} : une telle modification est opposable de plein droit aux copropriétaires, c’est à dire à ceux qui possèdent un lot lors du vote de la modification du Règlement de Copropriété, qu’ils aient voté ou non en faveur de cette modification.
			
			Inversement, tout copropriétaire peut se prévaloir des droits qu'il tient du règlement : par exemple la fourniture des services collectifs promis par le règlement auxquels les autres copropriétaires auraient renoncé.
			
			Les clauses du règlement s'imposent à partir de son entrée en vigueur, c'est-à-dire à la date de sa publication. {\bfseries Le règlement, bien que non-publié, est cependant opposable aux copropriétaires initiaux qui y ont personnellement adhéré}. Les modifications qui lui sont apportées sont opposables de plein droit à ceux qui avaient la qualité de copropriétaires lors du vote de la modification ; le défaut de publication n’a pas pour effet de rendre le modificatif inopposable au syndicat des copropriétaires\footnote{Civ 3\degres, 19 novembre 2008 \no 06-12.567}.
		
		\subsubsection{Les clauses réputées non écrites}
		
			Il convient de noter que les copropriétaires ne sont tenus que des dispositions licites du règlement.
	
			Or, nous savons qu'aujourd'hui les clauses du règlement contraires aux articles impératifs de la loi de 1965 sont réputées non-écrites. La Cour de cassation décide que << les clauses réputées non écrites par l'article 43 de la loi du 10 juillet 1965 sont non-avenues par le seul effet de la loi >>\footnote{Civ. 3ème, ler avril 1987, J.C.P. 1987 IV 201, Administrer août-septembre 1987, p.52 note GUILLOT.}.
			
			Il semble donc que de telles clauses soient considérées comme inexistantes plutôt que nulles. Il y a des différences entre nullité et inexistence :
			\begin{itemize}
				\item la nullité doit être demandée dans un certain délai (délai de prescription) alors qu'aucune limitation de temps n'existe pour faire constater l'inexistence d'une clause contraire aux dispositions d'ordre public de la loi de 1965 : Civ. 3eme, 26 avril 1989, Bull. Civ. III \no93, Rep. Defrénois 1989 art.34633, obs. H.SOULEAU ;
				
				\item la nullité doit nécessairement être constatée par un Tribunal, ce qui n'est pas le cas de l'inexistence.
			\end{itemize}
		
		\subsubsection{La prohibition de l'Atteinte aux Droits Acquis.}
		
			Le règlement ne peut avoir d'effet rétroactif en ce sens qu'il ne peut porter atteinte à des droits acquis antérieurement à sa mise en application, notamment à l'exercice d'un commerce exploité précédemment.
			
			De la même façon, le règlement de copropriété ne pourra être modifié en touchant aux droits acquis par un copropriétaire depuis la mise en copropriété de l’immeuble.
			
			Cette notion de droits acquis est particulièrement difficile à cerner : nous pensons avec le Professeur ATIAS que le droit acquis n’est pas un droit régulièrement constitué à l’origine. Par exemple, il n’y a pas atteinte aux droits acquis lorsque les droits invoqués sont dans le règlement de copropriété. Un droit acquis en copropriété doit s’analyser comme résultant d’une décision d’assemblée générale ou d’un fait juridique irréguliers mais qui n’ont pas été remis en cause légalement : par exemple, le fait pour un copropriétaire d’affecter son lot pendant plus de dix ans à un usage prohibé par le règlement de copropriété.
			
			De plus il ne faut pas confondre un « droit acquis » avec un « droit précaire » ; ce dernier s’analysant en une simple tolérance qui peut toujours être remise en cause.
			
			Sur l’ensemble de la question, lire la remarquable étude de Monsieur Jean Marc Le Masson\footnote{Les droits acquis en copropriété, Administrer mai 2002 p 11 et s.}
	
	\subsection{Les effets a l'égard des tiers}
	
		\subsubsection{Les Ayants-Cause à titre universel.}
		
			Conformément aux principes du droit commun, les effets obligatoires du règlement s'étendent à tous les ayants-cause à titre universel des copropriétaires : héritiers, légataires universels et à titre universel.
			
			Il en résulte qu’un nouveau Règlement de Copropriété voté par l’assemblée générale s’impose aux héritiers du copropriétaire, ce alors même que ce nouveau règlement n’a pas été publiée au fichier immobilier\footnote{Civ 3\degres, 22 nov. 2000 Construction Urbanisme 2001 \no 29, note Sizaire.}.
			
			La même règle s’applique sans aucun doute aux modifications du Règlement de Copropriété votées en assemblée générale mais non encore publiées.
		
		\subsubsection{Les ayants-cause à titre particulier}\footnote{Ce sont les donataires, légataires à titre particulier ou acquéreurs.} 
		
			\paragraph{Opposabilité par la publication au fichier immobilier}
			
			L'article 13 de la loi du 10 juillet 1965 dispose :
			<< {\itshape le règlement de copropriété et les modifications qui peuvent lui être apportées ne sont opposables aux ayants-cause à titre particulier des copropriétaires qu'à dater de leur publication au fichier immobilier}\footnote{Le fichier Immobilier est un terme général qui vise à la fois la Conservation des Hypothèques (pour la France de l’Intérieur) et le Service du Livre Foncier qui existe en Alsace-Moselle.}. >>
			
			Les ayants-cause à titre particulier sont des personnes qui ont acquis un droit réel déterminé sur le lot, en pratique les acquéreurs du lot : acheteurs, donataires, légataires particuliers, coéchangistes \etc
			
			En contrepartie, la cession du lot a pour effet de libérer le cédant des obligations imposées par le règlement de copropriété. Ces obligations sont attachées au lot. Elles se transmettent avec lui. Ce sont des obligations "propter rem" (à cause de la chose et non à cause de la personne).
	
			L'acte constatant le transfert de propriété du lot doit mentionner que l'acquéreur a eu préalablement connaissance du règlement de copropriété ainsi que des actes qui l'ont modifié, s'ils ont déjà été publiés (art.4 al .ler du Décret du 17 mars 1967).
			
			Le non respect de cette mention à l'acte de transfert est sans conséquence dans les relations du syndicat et de l'acquéreur puisque le Règlement de Copropriété lui est opposable du seul fait de sa Publication (article 13 de la loi); par contre une telle omission peut entraîner la responsabilité du notaire rédacteur de l'acte de transfert.
		
		- Opposabilité par adhésion expresse au règlement dénoncé à l’acquéreur
		Par ailleurs, l'alinéa 3 de l'article 4 du Décret du 17 mars 1967 admet l'opposabilité du règlement même non publié aux ayants cause à titre particulier si ces derniers en ont eu effectivement connaissance et ont adhéré aux obligations qui en résultent:
		" Le Règlement de Copropriété, l'État descriptif de division et les actes qui les ont modifiés, même s'ils n'ont pas été publiés au fichier immobilier, s'imposent à l'acquéreur ou titulaire du droit s'il est expressément constaté aux actes visés au présent article qu'il en a eu préalablement connaissance et qu'il a adhéré aux obligations qui en résultent".
		Le texte exige que soient remplies à la fois les deux conditions précitées,
		- Connaissance effective,
		- Adhésion exprès,
		A défaut de l'une seulement de ces deux conditions, et notamment lorsque l'acte précisé au titre des déclarations du vendeur "qu'à la connaissance de ce dernier il n'y a pas eu de modification au Règlement de Copropriété", l'acquéreur ne peut se voir opposer ces modifications.
		Un arrêt de la cour de Cassation du 8 septembre 2009 rappelle cette évidence à propos d’une modification de répartition de charges adoptée par une assemblée générale non publiée353 :
		...Attendu, selon l'arrêt attaqué (Aix en Provence, 25 janvier 2008), que la société civile immobilière Michel Ange (la SCI) a acquis par acte notarié du 20 septembre 1999, de la société civile immobilière Détroit plusieurs lots de copropriété ; que l'acte se réfère à un cahier des charges et règlement de
		353 8 septembre 2009 Cass.3è civ. pourvoi \no 08-15146
		droit de la copropriété année 2019-2020
		284
		copropriété établi le 12 septembre 1950, à un état descriptif de division établi le 2 mars 1960, à un état modificatif du 20 avril 1967 et à un état descriptif complémentaire établi le 15 février 1980, tous ces actes ayant été publiés au bureau des hypothèques ; que la SCI a assigné le syndicat des copropriétaires pour que les appels de charges qui lui ont été adressés depuis son acquisition lui soient déclarés inopposables ;
		Attendu que pour rejeter la demande de la SCI, l'arrêt retient que l'assemblée générale extraordinaire des copropriétaires réunis le 5 mai 1998 a approuvé les modifications aux tableaux de répartition des charges avec une nouvelle numérotation des lots, dans les termes joints au procès-verbal de l'assemblée générale, portant notamment modification des millièmes de parties communes afférents audits lots, que la société civile immobilière
		Le Détroit ex propriétaire des lots acquis par la société civile immobilière Michel Ange avait été invitée à participer à ladite assemblée générale et n'a pas contesté les délibérations prises et que le premier juge a retenu à juste titre que les décisions prises par cette assemblée sont devenues définitives et ce d'autant que contractuellement la société civile immobilière Michel Ange est tenue par les obligations de son auteur ;
		Qu'en statuant ainsi, sans constater que les modifications du règlement de copropriété votées lors de l'assemblée générale du 5 mai 1998 avaient été publiées au fichier immobilier ou expressément mentionnées dans l'acte d'achat de la SCI avec adhésion aux obligations qui en résultent, la cour d'appel n'a pas donné de base légale à sa décision.
		Toutefois, le juge du fond a trop souvent tendance à déclarer opposable le modificatif dont l'acte fait mention sans qu'ait été recueilli l'accord exprès de l'acquéreur354. .
		Cette opposabilité permet au syndicat d'exiger à l'encontre de l'acquéreur la démolition d'ouvrages réalisés par le précédent copropriétaire en violation des droits de la Copropriété. L'acquéreur est tenu au lieu et place du vendeur !
		3. Les locataires
		En droit strict, le locataire d'un lot en copropriété est étranger au règlement qui, dès lors, ne devrait pas lui être opposable, sauf si une clause du bail lui en imposait le respect. En ce dernier cas, c'est par l'action contractuelle née du bail que le propriétaire-bailleur peut obliger le locataire à se conformer aux dispositions du règlement.
		Cette conception a paru exagérément juridique et la jurisprudence a très vite considéré que les copropriétaires soumis au règlement ne peuvent conférer aux tiers plus de droits qu'ils n'en ont eux-mêmes dans l'organisation collective et que par conséquent le locataire "ne dispose pas de droits plus étendus que ceux dont jouit son bailleur".
		354 sur la notion d'adhésion cf. PARIS 19\degres 17 mai 1991 RTDI 91.384
		droit de la copropriété année 2019-2020
		285
		Les locataires sont assimilés à de véritables ayants cause à titre particulier des copropriétaires et doivent, lorsqu'ils ont été informés des clauses du règlement, s'y conformer. Ils peuvent être directement poursuivis par le syndic en cas d'inobservation de ces clauses : par exemple, il les fera condamner à cesser d'encombrer les parties communes en y déposant des marchandises355, à la démolition en cas de travaux irrégulièrement effectués sur les parties communes356 ou en cas d'empiétement ou de stationnement abusif sur les parties communes357 .
		Pour que l'opposabilité du règlement au locataire puisse fonctionner sans encourir la moindre objection, la loi QUILLOT (22 juin 1982), puis la loi MEHAIGNERIE (23/12/1986) ont imposé au bailleur d’un local à usage d’habitation d'annexer au contrat de location des extraits du règlement de copropriété concernant les clauses déterminant la destination de l'immeuble, la jouissance et l'usage des parties privatives et communes ainsi que la quote-part afférente aux lots loués dans chacune des catégories de charges. Ces dispositions sont reprises dans la loi du 6 juillet 1989.
		En outre, le copropriétaire reste responsable des infractions de son locataire
	
	\subsection{Les sanctions du règlement de copropriété}
	
		Les sanctions de l'inobservation du règlement de copropriété sont les mêmes que celles qui sont applicables en cas d'inexécution d'un contrat : exécution en nature et dommages intérêts, ces deux sanctions étant d'ailleurs cumulables.
		l) Les personnes qui peuvent demander que l'infraction au règlement soit sanctionnée sont le syndicat par la voix de son syndic ainsi que tout copropriétaire justifiant d'un intérêt à agir.
		Le syndicat est recevable à agir parce qu’il doit veiller à la conservation de l'immeuble et à l'administration des parties communes. Il a donc qualité pour exiger l'exécution des obligations stipulées au règlement.
		Chaque copropriétaire a également cette qualité en tant que partie contractante au règlement violé.
		2) La Cour de cassation se montre extrêmement rigoureuse à l'encontre de ceux qui auraient méconnu les dispositions du règlement de copropriété. Elle les condamne, au besoin sous
		355 Civ. 3ème, 20 octobre 1981, Bull. Civ. III \no162
		356 Paris, 18 mars 1987, D.1987 I.R.99
		357 Civ. 3ème, 5 avril 1968, Bull. Civ. III \no158; Grenoble 13 octobre 1965, D.1966 168
		droit de la copropriété année 2019-2020
		286
		astreinte, à effacer toutes les traces de leur comportement illicite, c'est-à-dire à l'exécution en nature des obligations résultant du règlement, alors même que cette exécution serait extrêmement difficile (Civ. 3ème, 13 octobre 1981, Bull. Civ. III \no152), ou que le préjudice causé n'aurait été que minime (Civ. 3ème, 18 juillet 1972, Bull. Civ. III \no39). Il suffit que par des entreprises illicites le règlement ait été violé pour que la responsabilité contractuelle du contrevenant soit engagée : ni la faute du contrevenant, ni l'existence d'un préjudice n'ont à être prouvés.
		Cette exécution en nature peut se traduire par la démolition de constructions irrégulièrement édifiées, le rétablissement d'installations enlevées sans droit. De telles condamnations ont été prononcées au cas d'empiétement sur des parties communes ou d'enlèvement d'une marquise ou d'une enseigne : Grenoble, 11 octobre 1965, D.1966 168 note GIVORD. La condamnation sous astreinte peut aussi être prononcée au cas de stationnement interdit dans les parties communes (Paris, 20 octobre 1970, J.C.P. 1971 II 16765 note DESIRY), ou la location d'une chambre de service en infraction avec le règlement (surpeuplement).
		Rappelons cependant que ces condamnations ont été prononcées en application de l’article 1143 du code civil aux termes duquel : Article 1143 - « le créancier a le droit de demander que ce qui aurait été fait par contravention à l'engagement soit détruit ; et il peut se faire autoriser à le détruire aux dépens du débiteur, sans préjudice des dommages et intérêts s'il y a lieu »
		Cet article sera remplacé à compter du 1er octobre 2016 par les articles 1221 et 1222 ainsi rédigés :
		« Art. 1221. – Le créancier d’une obligation peut, après mise en demeure, en poursuivre l’exécution en nature sauf si cette exécution est impossible ou s’il existe une disproportion manifeste entre son coût pour le débiteur et son intérêt pour le créancier.
		« Art. 1222. – Après mise en demeure, le créancier peut aussi, dans un délai et à un coût raisonnables, faire exécuter lui–même l’obligation ou, sur autorisation préalable du juge, détruire ce qui a été fait en violation de celle–ci. Il peut demander au débiteur le remboursement des sommes engagées à cette fin.
		« Il peut aussi demander en justice que le débiteur avance les sommes nécessaires à cette exécution ou à cette destruction.
		Toutefois ces dispositions ne s’appliqueront qu’aux contrats conclus après le 1er octobre 2016 en sorte que nous risquons d’avoir une justice à deux vitesses : très sévère pour les infractions aux Règlements de copropriété antérieurs à cette date et plus souples pour les Règlement de copropriété postérieurs !
		3) Si la violation du règlement par un copropriétaire provoque un trouble à un autre dans la possession de son lot, l'action possessoire ne peut être intentée. En effet, la protection
		droit de la copropriété année 2019-2020
		287
		possessoire n'est pas accordée à celui qui demande la cessation d'un trouble provenant de l'inexécution d'un contrat : Civ. 3ème, 22 juin 1976, Rep. Defrénois 1977 art.31350 note H.SOULEAU; Civ. 3ème, 10 juin 1980, Rep. Defrénois 1981 art.32686, note H.SOULEAU.
		4) La procédure d’injonction de faire prévue par le Décret du 4 mars 1988 (articles 1425-1 à 1425-9) peut se concevoir en cas d'inexécution d'obligations résultant du Règlement de Copropriété. Aux termes de cette procédure, le demandeur (le syndicat des copropriétaires par exemple), sollicite du juge du Tribunal d'Instance, par voie de requête déposée au greffe, que soit délivrée injonction au défendeur (le copropriétaire en infraction), en cas de non respect du contrat, de faire ce qui doit être fait pour parvenir à l'exécution de l'obligation contractuelle. Le juge fixe l'objet de l'obligation, le délai et les conditions dans lesquelles l'obligation doit être exécutée ainsi que la date à laquelle il entendra les parties, notamment pour s'assurer que son injonction a été suivie d'effet. En fait la jurisprudence ne révèle pas d'exemple d'application de l'injonction de faire à la Copropriété, dans la mesure où l'injonction de faire doit rester dans les limites du taux de compétence du Tribunal d'Instance et alors que le plus souvent l'obligation mise à la charge du copropriétaire est d'un montant non défini.
		5) Le règlement de copropriété comme toute convention peut contenir une clause pénale, c'est-à-dire la fixation d'une somme forfaitaire dissuasive, conformément à l'article 1153 du Code civil, pour le cas d'inexécution.358
		Il peut prévoir aussi des intérêts de retard au cas où un copropriétaire ne paierait pas ses charges ou appels de fonds en temps utile.359 Mais la question est discutée de savoir si l'intérêt stipulé peut être supérieur au taux légal.360.
		La prescription de 10 ans de l’article 42 alinéa 1e de la loi du 10 juillet 1965 court du jour où la violation du règlement de copropriété (exercice d’une activité contraire au règlement de copropriété) a été commise361.
		358 Civ. 3ème, 30 octobre l973, J.C.P. 1973, IV 402.
		359 Paris 30 octobre 1979, D.1980 I.R.24, obs. GIVERDON
		360 pour l'affirmative, voir Paris 30 octobre 1979 précité
		361 CA Paris 2 juillet 2009, JD 2009-378837 et Cass. Civ. 3e 8 septembre 2009