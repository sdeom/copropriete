\chapter{Historique de la copropriété}

\section{Des prémices au Code Civil}
	\subsection{L’antiquité}
		Depuis longtemps les individus se sont groupés pour construire à frais communs un immeuble et s'en répartir la propriété ou pour diviser un immeuble existant. Cette propriété partagée entre plusieurs personnes constitue la Copropriété.
		
		Certains auteurs
		\footnote{Cuq, Etude sur les contrats de l'Epoque de la première dynastie babylonienne : Nouvelle revue historique droit français et étranger, juillet 1910, p. 458 et s.}
		font remonter la copropriété à la Première Dynastie babylonienne (soit au Deuxième Millénaire avant notre ère) et la propriété par étages semble avoir été connue dans tout l'Orient
		\footnote{Selon le Professeur MICHALOPOULOS (Rd Imm 17 (3), juill - sep 1995 p 409), les Phéniciens auraient développé dans les villes qu'ils ont créées, dont CARTHAGE, des règles juridiques ressemblant à notre copropriété. Carthage ayant été détruite \dots nous n’y avons trouvé aucune trace de copropriété verticale.}.
		
	\subsection{Les coutumiers français}
		En France le droit coutumier révèle que ce régime de copropriété existait dans certaines Régions.
		
		\begin{description}
			\item[La Coutume d'Auxerre de 1561\footnote{art. 116}] prévoyait ainsi :
				\begin{quote}
					Celui à qui appartient le bas est tenu de faire entretenir tout le tour du bas de la muraille, pans ou cloisons, tellement que le haut puisse porter dessus, et est tenu de faire le plancher dessus lui de poutres, solives ou torchis, et celui qui a le dessus est tenu en faire autant du haut \dots
				\end{quote}
			\item[La Coutume d'Orléans\footnote{Grand Coutumier général, Tome III, p. 600}] édictait une obligation d'entretien à frais communs :
				\begin{quote}
					Seront faits à frais communs et entretenus les pavés estans devant lesdites maisons
				\end{quote}
		\end{description}
		
		Mais il a fallu des circonstances particulières :
		\begin{itemize}
			\item tel l'Incendie de Rennes en 1720 et la décision de reconstruire en élargissant les voies et en réservant de nombreux espaces pour les places et les jardins,
			\item ou l'impossibilité de développer une ville enserrée dans ces fortifications comme Grenoble\footnote{C'est à Grenoble qu'apparaissent les conventions de copropriété appelées << règlements de copropriété >> (MICHALOPOULOS, op. cit.).} pour que l'on trouve les premières réalisations de constructions indivises entre plusieurs propriétaires.
		\end{itemize}
		
		\par Dans les deux cas le schéma de construction était le suivant : le propriétaire du rez-de-chaussée construisait les fondations et les gros murs de son niveau, les propriétaires des étages construisaient les planchers et gros murs de leurs étages, le propriétaire du dernier étage couvrait la maison. Pour l'entretien, chacun avait la charge de ce qu'il avait construit.
		
		Il s’agissait donc de superposition de propriétés plutôt que de copropriété.
		
	\subsection{L'article 664 du Code Civil}
		L'article 664 du Code Civil fut ajouté à la demande des cours d'Appel de Lyon et Grenoble. Cet article était ainsi rédigé :
		\begin{quote}
			Lorsque les différents étages d'une maison appartiennent à divers propriétaires, si les titres de propriété ne règlent pas le mode de réparations et reconstructions elles doivent être faites ainsi qu'il suit :
			\begin{itemize}
				\item  Les gros murs et le toit sont à la charge de tous les propriétaires, chacun en proportion de la valeur de l'étage qui lui appartient.
				\item  Le propriétaire de chaque étage fait le plancher sur lequel il marche.
				\item  Le propriétaire du premier étage fait l'escalier qui y conduit le propriétaire du second étage fait, à partir du premier, l'escalier qui conduit chez lui, et ainsi de suite.
			\end{itemize}
		\end{quote}
	
		On reconnaît l'inspiration de la coutume d'Auxerre, mais cet article ne traitait que du mode de répartition des réparations et constructions.
		
		La propriété appartenait à celui qui avait exécuté les travaux. Les Tribunaux estimèrent cependant qu'il existait une servitude d'indivision (c’est à dire une propriété partagée) pour les éléments séparatifs entre appartements, tels le plancher.
		
		Or si chaque étage était propriété privative, cet article ne prévoyait aucun mode de gestion pour les éléments communs. Il est vrai que l'article 664 Code Civil avait un caractère supplétif, donc ne s'appliquait qu'à défaut de dispositions contractuelles.
		
		Il appartenait donc aux propriétaires d'organiser conventionnellement la gestion de l'immeuble et les modalités de jouissance des parties communes, conformément aux dispositions de l’article 1134 du code civil.
		
		La jurisprudence sur l’article 664 du code civil fut très limitée au \siecle{xix}\footnote{Cf. La copropriété des immeubles bâtis dans la jurisprudence et la doctrine du \siecle{xix} Marc Ortolani ; Revue historique de droit français et étranger (1922-2014) Vol. 78, No. 2 (AVRIL-JUIN 2000), pp. 249-287}

\section{Le premier « statut » de la copropriété : la loi de 1938}

	\subsection{Évolution des idées et des faits au \siecle{xx}.}
		L'idée d'une indivision forcée entre les propriétaires d'un immeuble divisé par étages se fit jour au début du \siecle{xx}.
		
		Plusieurs facteurs ont permis le développement des immeubles en copropriété : destruction massive des immeubles pendant la première guerre mondiale, concentration des populations dans les villes, amenuisement des fortunes et alourdissement des charges fiscales qui incitent les propriétaires à vendre leurs immeubles par appartements $\dots$
		
		De plus, les techniques ont évolué en faisant intervenir des matériaux onéreux, tandis que les équipements des immeubles devenaient complexes.
		La constitution de sociétés civiles devint donc nécessaire pour réaliser des immeubles importants: les associés apportaient en effet les fonds nécessaires à ces réalisations.
		\begin{quote}
			Il est apparu que les choses ne pouvaient plus continuer à se passer en famille ou entre amis, comme à Grenoble où l'usage séculaire vient adoucir bien des difficultés. (Bull. de la Sté d'Etudes Législatives 1925-26, p. 167)
		\end{quote}
		Pourtant, à une époque où le droit affirmait que les sociétés devaient être constituées pour << partager des bénéfices >>, la constitution de sociétés destinées à réaliser des économies, et non à mettre en commun des bénéfices, était sujette à caution !
	
	\subsection{La loi du 28 juin 1938}
		Sans doute inspirée de la législation belge (loi du 8 juillet 1924 abrogeant l’article 664 du code civil), la loi de 1938 (abrogeant pour la France l’article 664 du code civil) comprenait deux parties : la première régissait les sociétés de construction, la seconde organisait le statut des immeubles en copropriété.
		
		Les différents propriétaires étaient obligatoirement groupés en un syndicat.
		
		La loi institue :
		\begin{itemize}
			\item  le « règlement de copropriété », objet d'une convention générale ou de l'engagement de chacun des intéressés et lui a conféré force obligatoire à l'égard des ayants cause à titre particulier des copropriétaires au moyen de sa transcription aux hypothèques ;
			\item  le syndic dont elle déterminait le mode de nomination, les pouvoirs et la rémunération.
		\end{itemize}
	
	\subsection{Les textes subséquents}
		Un \textbf{Décret loi du 29 novembre 1939} a renforcé le privilège du syndicat pour garantir le paiement des charges communes en précisant le rang et les effets de ce privilège immobilier et a créé un privilège mobilier sur les meubles meublants l'appartement du copropriétaire avec report de ce privilège sur les loyers lorsque l’appartement est donné en location.
		
		Ce privilège immobilier sera supprimé par le Décret portant réforme de la publicité foncière du 4 janvier 1955.
		
		Une \textbf{loi du 7 février 1953} a incidemment étendu le régime de la copropriété aux copropriétés horizontales.
		
		\begin{quote}
			Les dispositions de la loi du 28 juin 1938 sont étendues aux sociétés constituées ou à constituer, quelle qu'en soit la forme, ayant pour objet la construction, l'acquisition ou la gestion d'ensembles immobiliers à usage principal d'habitation, composés d'immeubles collectifs, de maisons individuelles et éventuellement des services communs y afférents et destinés à être attribués aux associés en propriété ou en jouissance $\dots$
		\end{quote}
		
		Ce texte servira de base à la rédaction de l'article 1er alinéa 2 de la loi de 1965.
		
		La \textbf{copropriété verticale} évoque un même corps de bâtiment dont les appartements sont superposés.
		La \textbf{copropriété horizontale} évoque plusieurs corps de bâtiments ou plusieurs maisons individuelles dans un même ensemble.
		
		Différents décrets de 1954 à 1959, destinés à protéger les associés des Sociétés Immobilières dites d'attribution évoqueront la copropriété des immeubles.
		
		Relevons enfin le Décret du 20 mai 1955 ayant pour objet la simplification des procédures administratives applicables à la construction d'ensembles immobiliers qui édicte qu'un cahier des charges annexé à la demande de permis de construire, précise les conditions dans lesquelles la gestion et l'entretien des ouvrages et aménagements d'intérêt collectif seront assurés par les copropriétaires ou par une association syndicale constituée à cet effet.

\section{La loi du 10 juillet 1965 et le décret du 17 mars 1967}
	Le projet de loi fut discuté pendant une année exactement, avant d’être signée le 10 juillet 1965 :
	\begin{itemize}
		\item 	à l'Assemblée Nationale sur le rapport de Monsieur \nom{Zimmermann} qui évoquait essentiellement le caractère impératif de ses dispositions en parlant à son sujet de contrat d'adhésion ;
		\item  au Sénat avec rapport de Monsieur \nom{Voyant} qui suggéra diverses modifications.
	\end{itemize}
	La loi ainsi votée fut complétée par un Règlement d'administration publique qui est en date du 17 mars 1967. La loi comportait 47 articles. Le Décret d'application comportait 65 articles, soit plus que la loi elle-même.

	\subsection{Les buts poursuivis par le legislateur}
		Les auteurs de la loi ont précisé dans l'exposé des motifs qu'ils avaient poursuivi cinq buts :
		\begin{enumerate}
			\item \textbf{Formuler des définitions claires et précises.}
			
			Il est de fait que la loi définit avec précision son objet (art. ler) et qu'elle donne une série de définitions (art. 3 et 4 sur les parties communes et privatives, par exemple).
			
			\item  \textbf{Faciliter la gestion collective de l'immeuble.}
			
			Le Règlement de Copropriété devient obligatoire.
			
			Le Syndicat des Copropriétaires a la personnalité civile.
			
			L'autorité du syndic est en principe accrue.
			
			Le Syndicat bénéficie d'un privilège immobilier et mobilier pour le recouvrement des charges.
			
			\item  \textbf{Garantir les droits des copropriétaires contre les stipulations contractuelles abusives.}
			
			Qu'il s'agisse des droits des copropriétaires sur les parties communes et privatives ou qu'il s'agisse de la répartition des charges de copropriété.
			
			\item  \textbf{Permettre des travaux d'amélioration de l'immeuble conformes a sa destination.}
			
			\item  \textbf{Permettre pour les grands ensembles une division de la copropriété unique d'origine en plusieurs entités nouvelles}.
			
			Les articles 27 et 28 de la loi permettent en effet la création de syndicats secondaires et l'éclatement en copropriétés séparées.
		\end{enumerate}
	
	\subsection{Les principes fondateurs de la loi du 10 juillet 1965}
	
		\subsubsection{L’adoption d’une conception dualiste de la copropriété}
		
			Le texte du 10 juillet 1965 a opté pour la théorie dualiste de la copropriété, plus cohérente avec la conception française du droit de propriété, alors qu'un régime unitaire aurait certainement simplifié les règles de gestion de la Copropriété.
			
			\paragraph{Le régime dualiste}
				On entend par cette expression la coexistence dans la personne du copropriétaire d’un droit de propriété classique sur les « parties privatives », c'est-à-dire le local acheté (<< cube d'air >> selon Mr \nom{Savatier} ou << Vase sans contour >> selon Mr \nom{Mazeaud}) et d’un droit d’une propriété indivise et forcée sur les parties communes.
				
				Cette théorie dualiste permet de sauvegarder la fiction d’une propriété pleine et entière du copropriétaire sur les « parties privatives » comprises dans son lot, dans lesquelles il est censé jouir des mêmes attributs qu’un propriétaire « exclusif » (\emph{usus}, \emph{abusus}, \emph{fructus}), tandis que ses droits sont fortement limités sur les « parties communes », affectés à l’usage et à l’utilité de tous, et dont le sort dépend des décisions collectives prises à l’Assemblée.
			
				Par contre dans le régime dualiste les inconvénients sont nombreux: les Tribunaux auront à choisir, cas par cas, entre l'intérêt général et l'intérêt du copropriétaire et les solutions adoptées seront fort différentes selon qu'ils privilégieront le copropriétaire sur la Collectivité ou inversement.
			
			\paragraph{Le régime unitaire}
				La conception unitaire reconnait au contraire une indivision générale organisée sur l’ensemble de l’immeuble : le copropriétaire ne possède qu'un double droit d'usage, sur une fraction déterminée de l'immeuble (ce que nous appelons les parties privatives) et sur les parties de l'immeuble affectées a la jouissance collective.
				
				Ainsi en Autriche la copropriété par appartements est une simple indivision avec droit réel de jouissance attaché à l'appartement. Conception unitaire également en Suisse (loi fédérale du 19 décembre 1963).
				
				Dans ce dernier pays, aucune distinction n’est faite entre parties communes et parties privatives : les copropriétaires sont propriétaires indivis des unes comme des autres et le règlement de copropriété ne leur attribue qu’un droit de jouissance et d’usage exclusifs sur certaines parties de l’immeuble. << Cette situation permet d’exclure de la copropriété le copropriétaire qui viole les obligations juridiques et morales découlant de l’organisation collective de la copropriété >>\footnote{Copropriété des Immeubles bâtis et ventes d’immeubles à construire au Grand Duché du Luxembourg (Elter et Schockwehler ) – Luxembourg 1978.}
				
				En Allemagne (loi du 15 mars 1951), la conception est plutôt dualiste (propriété privative d’une habitation jointe à une quote-part dans la propriété commune). Cependant, la majorité des propriétaires peut contraindre un copropriétaire qui manque gravement à ses obligations à aliéner son lot.
				
				On conçoit aisément les avantages du système unitaire : l’intérêt collectif primera sur le droit de propriété individuelle.
		
		\subsubsection{L’adoption d’un régime mixte, à la fois contractuel et institutionnel.}
			Le régime de la loi du 10 juillet 1965 est aussi un régime mixte puisqu’il est à la fois contractuel (le Règlement de Copropriété est un contrat) et Institutionnel (la loi impose l'essentiel des règles de fonctionnement.)
			
			Le professeur \nom{LARROUMET}, écrit \footnote{Cité par Madame \nom{Kischinewsky-Broquisse} in La Copropriété des Immeubles bâtis 4\ieme{} Edition, p. 14 \no 32} :
			\begin{quote}
				Le statut de la copropriété a essayé de ménager l'existence de deux principes nécessaires qui peuvent paraître paradoxaux : la liberté individuelle des propriétaires et l'obligation d'assurer une bonne administration commune.
			\end{quote}
			
			On peut effectivement se demander, en pratiquant régulièrement la copropriété et les copropriétaires si ces deux objectifs ne sont pas seulement paradoxaux, mais bien plutôt inconciliables !
		
		\subsubsection{La soumission des copropriétés à des dispositions d’ordre public}

			D'un texte de 10 articles dans la loi de 1938 on passait à 49 articles outre les 65 articles du R.A.P (\textbf{Règlement d'Administration Publique}).
			
			La multiplication de ces articles, le caractère d'\textbf{ordre public} de la plus grande partie des dispositions de la loi vont avoir pour conséquence de placer la copropriété sous la tutelle des Tribunaux.
			
			En effet, aux termes de l'article 43 de la loi du 10 juillet 1965 les articles 6 à 37 de la loi et les textes du Décret du 17 mars 1967 (dans leur rédaction actuelle) pris en application de ces articles 6 à 37 sont d'Ordre Public. Cela signifie que l'on ne peut y déroger, que ce soit dans le Règlement de Copropriété ou par décision d'assemblée générale.
			
			Le texte précise en effet :
			\begin{quote}
				Article 43 de la loi :\\
				\emph{Toutes clauses contraires aux dispositions des articles 6 à 37, 42 et 46 de la loi, et celles du règlement d'administration publique prises pour leur application sont réputées non écrites}.
			\end{quote}
		
			Ces clauses sont non seulement nulles, mais encore elles sont supposées ne pas exister du tout. La différence est importante dans la mesure où la nullité doit être constatée par le juge ; par contre il suffit de ne pas tenir compte de ce qui est réputé non écrit, sans avoir à faire préalablement constater l'inexistence par le juge.
			
			Nous verrons toutefois que la Jurisprudence considère qu’une clause, même contraire aux dispositions impératives de la loi doit recevoir application tant que le juge n’a pas constaté son inexistence.
		
		\subsubsection[Le principe d’unicité des « Copropriétés »]{Le principe d’unicité des « Copropriétés » (absence d’un statut légal adapté pour les « Grands Ensembles Immobiliers. »)}
		
			Si la loi de 1965 est un statut aisément applicable aux petites copropriétés ou aux copropriétés de moyenne importance, elle est par contre totalement inadaptée aux grands ensembles qui comportent un grand nombre de bâtiments.
			
			Lors de la discussion du projet de loi devant l'Assemblée Nationale, Monsieur \nom{Zimmermann} déclarait : << Votre rapporteur pense qu'un avenir prochain contraindra sans doute le législateur à donner aux grands ensembles un statut particulier >>. On attend encore l'adoption de ce statut particulier.
			
			Certes, la loi de 1965 a prévu les palliatifs indiqués plus haut : création de syndicats secondaires ou même scission de la Copropriété. Mais bien souvent les syndicats secondaires ne donnent que peu de solutions aux difficultés de gestion des ensembles et la scission est elle-même très difficile à mettre en œuvre.
			
			Certes, l'article 1er de la loi autorise le promoteur d'une opération immobilière portant sur un ensemble à choisir un statut différent de celui de la Copropriété, mais la réalité des faits permet de constater qu'il est rare de voir exercer cette faculté dans la pratique.
			
			C'est d'ailleurs sur cette question des grands ensembles\footnote{Bien que l'expression << Grands Ensembles >> soit fréquemment utilisée, il conviendrait de ne parler que d'Ensembles Immobiliers en citant les termes de l'article 1er alinéa 2 de la loi. En effet, le législateur ne connaît ni les Grands, ni les Petits Ensembles, il ne connaît que les Ensembles Immobiliers.} que se focalise les réformes à partir des années 1990. Toutefois, faute de trouver un système démocratique et efficace de gestion, le législateur aborde essentiellement ces grands ensembles par le prisme des copropriétés en difficultés.
	
	
	\section{Les adaptations de la loi 65-557 du 10 juillet 1965}
	
		\subsection{La phase de stabilité (1965--1985)}
			La loi du 10 juillet 1965 est restée relativement « stable » au cours des vingt premières années d’existence. Il s’agissait en effet d’un texte cohérent, auquel les lois successives ont apporté des modifications mineures mais non des bouleversements, ces adaptations étant destinées à adapter le statut aux différentes contraintes de la vie économique.
			
			Ex : Loi du 29 octobre 1974 relative aux économies d'énergie,\\
			Loi du 2 janvier 1979 relative aux droits grevant les lots d'un immeuble soumis au statut de la copropriété,\\
			Loi du 21 décembre 1984 sur la domiciliation des entreprises
		
		\subsection{Les lois correctives (1985--2000)}
		
			\subsubsection{Le diagnostic}
				De 1985 à 2000 apparaissent des lois « correctives » qui tirent le constat des dysfonctionnements de certaines copropriétés :
				\begin{itemize}
					\item 	absentéisme (notamment dans les grandes copropriétés), rendant impossible l’adoption de certaines décisions, notamment les modifications du règlement de copropriété et les travaux d’amélioration
					\item  abus « de pouvoir » de certains syndics, qui manipulent les fonds du syndicat des copropriétaires avec un contrôle très restreint du seul conseil syndical, au point que l’on déplore de véritables détournements de fonds
					\item  recouvrement des charges défaillant, car en cas de « faillite » du copropriétaire, le syndicat des copropriétaires n’est pas privilégié par rapport aux autres créanciers, notamment la banque $\dots$
					
					De plus, la plupart des copropriétés fonctionnent en appelant les charges $\dots$ une fois que les dépenses ont été faites (au réel), ce qui génère des a coups de trésorerie.
				\end{itemize}
				Dès les années 1990 apparaissent les « copropriétés en difficultés » (\emph{cf infra}) que l’on tente dans un premier temps de traiter par des mesures préventives, en améliorant la budgétisation des dépenses et le recouvrement des charges.
				Ainsi, la loi du 13 décembre 2000 dite SRU est présentée au Président de la façon suivante :
				\begin{quote}
					Les obstacles au bon fonctionnement du régime des copropriétés sont liés à la mauvaise information des copropriétaires à l'égard de l'état du bien qu'ils acquièrent, à la faiblesse des principes de gestion comptable, au peu d'empressement de certains copropriétaires à honorer leurs obligations financières, enfin, au développement de la précarité qui bloque toute décision d'entretien de l'immeuble et contribue
					non seulement à la dégradation de l'immeuble mais également à l'endettement du syndicat des copropriétaires.
				\end{quote}
				
			\subsubsection{Les lois correctives}
			
				\paragraph{La loi \nom{Bonnemaison} du 31 décembre 1985} : encadrement du syndic.
				
				\begin{itemize}
					\item Le Conseil Syndical est désormais obligatoire, mais les copropriétaires peuvent refuser son institution.
					
					\item  Le syndic doit nécessairement soumettre au vote de l'Assemblée Générale l'ouverture ou non d'un compte séparé ; mais si l'Assemblée n'exige pas ce compte séparé le syndic pourra conserver un compte unique.
					
					\item  Les copropriétaires peuvent prendre individuellement connaissance des comptes du syndic.
					
					\item  Assouplissement des conditions de représentation des copropriétaires aux assemblées générales : désormais, et quelles que soient les mentions du Règlement de Copropriété, toute personne pourra recevoir un mandat.
					
					La majorité de l'article 26 est abaissée des 3/4 des voix (et la moitié en nombre) est abaissée à la majorité des 2/3 des voix (et la moitié en nombre).
					
					\item  Abaissement de majorité pour certains travaux : économies d'énergie, quant à la mise en conformité des logements aux normes de salubrité, de sécurité et d'équipement et quant à la Sécurité des personnes et des biens. Les règles de majorité ont donc été assouplies pour ces décisions.
				\end{itemize}
			
				\paragraph{La loi relative à l'habitat du 21 juillet 1994} : amélioration du recouvrement des charges.
				
				L'obligation de participer aux charges est assortie d'un caractère réel, c'est à dire qu'elle pèse sur le lot et non plus seulement sur le copropriétaire. En sorte que l'acquéreur du lot se trouve de plein droit débiteur des charges.
				
				De la sorte le syndicat de copropriété, en cas de vente du lot sur saisie, se trouve payé par préférence aux autres créanciers inscrits pour les charges échues depuis moins de deux ans et en concurrence avec le prêteur de denier ou le vendeur pour deux autres années.
				
				\paragraph{La loi Solidarité Et Renouvellement Urbain (12 décembre 2000) et ses décrets d’application}.
				
				La loi SRU a modifié assez profondément le statut de la copropriété et les pratiques de gestion pour :
				\begin{itemize}
					\item 	réformer la comptabilité des Syndicat des Copropriétaires pour leur permettre une meilleure gestion de leur trésorerie --- passage aux provisions budgétaires, distinction claire du budget « ordinaire » et du budget « travaux », définition du plan comptable du Syndicat des Copropriétaires sur le modèle de la comptabilité d’entreprise, passage à la comptabilité d’engagement, obligation de présenter le budget et les comptes avec un comparatif de l’année précédente ;
					
					\item  obliger le syndic à ouvrir un compte bancaire séparé, sauf dispense votée à la majorité de l’article 25 ;
					
					\item  faciliter l’entretien du bâti---: travaux obligatoires à la majorité de l’article 24, pose des compteurs d’eau à la majorité de l’article 25, création du « carnet d’entretien » ;
					
					\item  renforcer les moyens de recouvrement des charges, pour prémunir la copropriété de l'absence de financement et sa conséquence, la dégradation du patrimoine --- création d’une procédure de recouvrement spécifique pour les appels budgétaires, devant le Juge des référés statuant au fond, mise à la charge du copropriétaire défaillant des frais de justice.
				\end{itemize}
			
				\subparagraph{Le Décret du 30 mai 2001} impose au syndic de tenir un carnet d’entretien pour chaque syndicat des copropriétaires, outil d’information permettant de connaitre le niveau d’entretien de l’immeuble, tant pour un futur acquéreur que pour les pouvoirs publics.
				
				\subparagraph{Le Décret du 27 mai 2004} :
				\begin{itemize}
					\item  définit les travaux hors budget ;
					
					\item  apporte enfin des définitions (clarification) qui faisaient défaut quant aux charges et provisions ;
					
					\item  permet l’entrée en vigueur de la réforme de la comptabilité, à partir du 1er janvier 2007
				\end{itemize}
				
				\subparagraph{Le Décret relatif aux comptes du syndicat des copropriétaires (14 mars 2005)} crée un plan comptable simplifié pour la copropriété, sauf pour les « petites copropriétés » (moins de 10 lots principaux dont le budget moyen annule est inférieur à 15.000 \euro) du respect de ce plan comptable. Les nouvelles règles comptables sont entrées en vigueur le 1er janvier 2007.
	
	\section{La phase d’instabilité (2000 - aujourd'hui)}
	
		\subsection{La création progressive d’un statut des copropriétés en difficulté}
		
			\subsubsection{Le diagnostic}
			
				Des modifications plus profondes interviennent à partir des années 2000, pour faire face à l’apparition de « copropriétés en difficulté », le plus souvent de grands ensembles urbains qui sont littéralement « en faillite » --- c’est-à-dire dans l’incapacité de faire face au payement des fournisseurs et à la réalisation des travaux nécessaires à la préservation du bâti en appelant des charges supportables par les copropriétaires, et dont le redressement va nécessiter d’importants investissements publics.
				
				A côté de logements sociaux ont été réalisés des immeubles du parc privé dont les acquéreurs, souvent issus d’une population fragile, ont été touchés de plein fouet par la crise économique traversée par la France dans les années 80 et 90.
				
				Ces habitants n’ont pas eu les fonds nécessaires, ni pour payer les emprunts souscrits pour leur acquisition, ni pour acquitter leurs charges de copropriété (principalement chauffage et eau fournis par des sociétés publiques ou parapubliques). De plus il apparaît souvent que les syndics successifs ont mal géré ces immeubles, laissant s’accumuler les impayés sans réagir.
				
				L’environnement s’est dégradé et les conditions locales d’habitation sont allées elles-mêmes en se dégradant (tags, ascenseurs en panne, chauffage collectif défaillant) ; les ventes ou mises en locations font apparaître un changement rapide de la population, et les nouveaux arrivants sont économiquement de plus en plus fragiles (cette fragilité est renforcée par une absence de mixité sociale, voire ethnique dans certains quartiers).
				
				Il y aurait près de \nombre{600 000} logements sur les 6 millions en copropriété qui relevant des dispositions relatives aux syndicats en difficulté ! Il s’agit d’ailleurs le plus souvent de « grandes copropriétés » situées en périphérie des grandes agglomérations, dans les « Quartiers » d’où sont parties les révoltes urbaines de 2005.
				
				Dans les années 2000, \textbf{la copropriété devient un enjeu majeur de la politique de la Ville}. Les pouvoirs publics regardent désormais cette partie du « Parc Privé » avec un oeil attentif pour deux raisons opposées.
				\begin{enumerate}
					\item - Les copropriétés « en difficulté » peuvent nécessiter une intervention publique pour arrêter une spirale de déqualification, et assurer le renouvellement urbain. La politique de « désengagement » total de ce secteur, qui a prévalu dans les années 70 et 80 n’est donc pas soutenable à long terme.
					
					\item Inversement, les copropriétés sont du renouvellement urbain, car des mesures apparemment peu « coercitives » ( par exemple, l’abaissement des majorités pour les travaux de rénovation énergétique, l’obligation de constituer une « épargne » destinée au travaux) peut accélérer de façon radicale le renouvellement des immeubles, sans qu’il soit nécessaire de recourir à des financements publics.
				\end{enumerate}
				
				Dès lors, toutes les lois inspirées par le Ministère du Logement comprendront un volet « Copropriété ».
				
			\subsubsection{Les principales réformes}
			
				Ce statut se construit au fil des réformes législatives, de façon totalement empirique: sont successivement introduites par les textes les dispositions « requises » pour faire face dans tel ou tel grand ensemble !

				\subparagraph{La loi relative à l'habitat du 21 juillet 1994} : création de l’administration provisoire des copropriétés en difficulté
				
				\subparagraph{La loi du 13 décembre 2000 dite SRU} : possibilité de « portage » des lots par la collectivité public, possibilité de procéder à la « scission » des grandes copropriétés pour revenir à des entités plus faciles à gérer et clarifier la domanialité public ou privée.
				
				\subparagraph{La loi du Urbanisme et Habitat du 2 juillet 2003 }: impose de nouvelles et lourdes contraintes pour assurer la sécurité des ascenseurs (loi SAE ou de « Sécurité des Ascenseurs Existants, échéance au 1er janvier 2010) ; facilite la réalisation de travaux d’accessibilité aux handicapés dans les immeubles en copropriété.
				
				\subparagraph{La loi \no 2003-710 du 1er août 2003 dite loi \nom{Borloo}} : possibilité pour le Maire de réaliser des travaux d’office en cas de « carence » du syndicat des copropriétaires (L129-1 CCH) ; expropriation du syndicat des copropriétaires pour carence, aide juridictionnelle accordée aux syndicats en difficulté placés sous administration provisoire pour le recouvrement des charges, faculté pour l’administrateur provisoire de l’article 29-1 de se faire assister par un tiers.
				
				\subparagraph{La loi \no 2006-872 du 13 juillet 2006, dite loi ENL (Engagement national pour le logement)} : création de sanctions pénales pour la mise en copropriété à usage d’habitation d’un immeuble dangereux, insalubre ou inhabitable en vue de créer des locaux d’habitation ; frais de recouvrement des charges mis à la charge du copropriétaire défaillant ; abaissement de la majorité requise pour décider de la fermeture de l’immeuble.
				
				\subparagraph{La loi \no 2009-323 du 25 mars 2009 dite << loi \nom{Boutin} >> ou loi MOLLE}
					\begin{itemize}
						\item Création d’un droit de préemption au profit du syndicat en cas de cession de parkings (pour permettre la « réappropriation » des nappes de parking en banlieue et lutter contre les « véhicules ventouses »).
						
						\item Procédure applicable aux copropriétés en « pré difficulté », mise sous «observation » par un « mandataire ad hoc » lorsqu’à la clôture des comptes, les impayés atteignent 25\% du budget prévisionnel et des dépenses pour travaux non compris dans le budget prévisionnel (décret du 20 avril 2010).
						
						\item Élargissement des pouvoirs de l’administrateur judiciaire
					\end{itemize}
				
				\subparagraph{La loi égalité et citoyenneté du 27 janvier 2017 } complète les dispositions sur les copropriétés en difficultés (renforcement des pouvoirs de l’administrateur).
				
		
		\subsection{Lutte contre le réchauffement climatique et transition énergétique}
		
		\subsubsection{Diagnostic}
			La loi ENE du 12 juillet 2010 fixe, en 257 articles, de nouvelles règles environnementales et de performance énergétique dans différents domaines (urbanisme, bâtiment, transports, climats…), afin de remplir les objectifs fixés par la loi Grenelle I (loi de programmation du 3 août 2009), notamment l’objectif général dit « facteur 4 », la division par 4 des émissions de gaz à effet de serre (GES), avec un horizon fixé à l’époque à 2020, repoussé depuis à 2050.
			
			Le secteur du bâtiment consomme actuellement 68,2 millions de tonnes d'équivalent pétrole, soit \pourcent{42,5} de l'énergie finale total.Il génère parallèlement 123 millions de tonnes de \dioxydeDeCarbone, soit \pourcent{23} des émissions nationales. L’énergie est consommée pour deux tiers dans les logements et pour un tiers dans le secteur tertiaire. Cette proportion reste sensiblement constante depuis vingt ans.
			
			Or le taux de renouvellement du parc existant est extrêmement faible : le taux de renouvellement des bâtiments anciens par des bâtiments neufs est inférieur à \pourcent{1} par an. Sans effort supplémentaire réalisé, ce faible taux, associé au rythme actuel des réhabilitations n'entraînerait un relèvement des performances énergétiques de la totalité des bâtiments construits avant 1975 que dans plus d'un siècle.
			
			Il faut donc prévoir un investissement massif en rénovation énergétique des bâtiments anciens : le coût global des travaux d'économie d'énergie réalisés dans le secteur du bâtiment serait de \nombre{1 000} milliards d’euros d’ici 2050\footnote{Le coût des rénovations énergétiques à réaliser pour parvenir au facteur 4 est de \montant{200} à \montant{400} du \metreCarre, selon la typologie du bâtiment, et les résidences principales représentent une surface d'environ 2,65 milliards de mètres carrés. Aujourd’hui, le coût moyen des rénovations énergétiques effectuées est de \prixSurface{125} seulement (renouvellement de chaudières, isolation des fenêtres)}.
			Les deux tiers de cet investissement concerneraient les ménages pour des montants qui seraient au moins de \montant{20 000} et pourraient dépasser les \montant{40 000}, à investir dans leurs logements en trois à quatre décennies.
			
			L'enjeu est important pour les copropriétés, car les charges de chauffage représentent environ \pourcent{70} du budget prévisionnel d'un syndicat de copropriétaires, et 4,7 millions de logements sur les 8 millions en copropriété sont chauffés collectivement
			
		\subsubsection{Les principales reformes}
		
			\paragraph{La loi \no 2010-788 du 12 juillet 2010 (ENE) dite Grenelle \textsc{ii}} est la première d’une longue série de textes ayant pour objectif de faciliter la réalisation des travaux d’économie d’énergie en copropriété, car il est plus aisé d’imposer cette rénovation énergétique dans des immeubles collectifs administrés par un professionnel que dans l’habitat individuel.
				
			Elle impose :
			\begin{enumerate}[label=\alph*)]
				\item la réalisation obligatoire d’un DPE ou d’un diagnostic énergétique dans les tous les immeubles équipés d’un dispositif commun de chauffage ou de refroidissement (L 134-4-1 du CCH), et dans les immeubles en copropriété à usage principal d’habitation de plus de 50 lots, dont le dépôt de la demande de permis de construire est antérieur au 1er janvier 2001, l’obligation de réaliser un véritable audit énergétique comprenant des préconisations de travaux ;
				
				\item l’obligation de soumettre à l’assemblée un plan de travaux d’économies d’énergie ou contrat de performance énergétique (CPE), à l’issue du DPE ;
				
				\item l’abaissement de la majorité requise pour les travaux d'économies d'énergie ou de réduction des émissions de gaz à effet de serre, soumis à l’article 25 avec un second vote possible en 25-1 (art.25 f de la Loi 65-557 du 10 juillet 1965)
				
				\item la possibilité d’imposer au copropriétaire, sur la partie privative de son lot des travaux de rénovation énergétique « d’intérêt collectif » sauf dans le cas où ce dernier est en mesure de produire la preuve de la réalisation de travaux équivalents dans les dix années précédentes (\emph{idem}) ;
				
				\item  la création d’un « droit à la prise » pour les installations l’alimentation des véhicules électriques (IRVE) : le copropriétaire peut réaliser ces travaux sans autorisation préalable d’assemblée générale, sauf pour le syndic à saisir le Tribunal pour s’y opposer (art. 24-5 de la Loi 65-557 du 10 juillet 1965 et L 111-6-4 du CCH).
			\end{enumerate}
			
			\paragraph{La loi \no 2015-992 du 17 aout 2015 sur la Transition Energétique et la Croissance Verte, dite loi TECV} votée 3 mois avant la COP 21, prévoit de nouvelles obligations destinées à accélérer la rénovation énergétique des copropriétés.
			
			\subparagraph{Obligation de rénovation énergétique avant 2025 pour les logements en F ou G}
			« Avant 2025, tous les bâtiments privés résidentiels dont la consommation en énergie primaire est supérieure à 330 kilowattheures d’énergie primaire par mètre carré et par an doivent avoir fait l’objet d’une rénovation énergétique. »
			
			Renforcement de l’obligation de pose de compteurs individuels de chaleur, avec sanction pénale( allant jusqu’à \montant{1 500} par an et par logement)
			
			\subparagraph{Obligation d’embarquer la performance énergétique dans les travaux d’entretien et de rénovation.}
			Dès parution du décret d’application, les copropriétés auront l’obligation (sauf cas exceptionnels) d’embarquer l’amélioration de performance énergétique en cas de travaux sur le clos et le couvert (ravalement de façade, réfection de toiture…). Les matériaux d’isolation mis en oeuvre « doivent permettre d’atteindre, en une ou plusieurs étapes », les performances d’un bâtiment basse consommation (étiquette A ou B). Le vote de ces travaux relèvera de l’article 24 de la loi du 10.07.65 (majorité relative)
			Cette disposition a fait l’objet d’un Décret d’application \no 2016-711 du 30 mai 2016 relatif aux travaux d'isolation en cas de travaux de ravalement de façade, de réfection de toiture ou d'aménagement de locaux en vue de les rendre habitables
			De même la loi comporte un renforcement de la réglementation en terme de performance énergétique des équipements dès lors que les équipements collectifs doivent être remplacés, réparés ou installés et notamment les chaudières collectives ou les équipements collectifs de refroidissement
			Possibilité de déroger aux règles d’urbanisme pour isoler thermiquement les copropriétés
			L’article 7 de la Loi de Transition Énergétique prévoit la possibilité de déroger aux règles des PLU (relatives à l’emprise au sol, à la hauteur, à l’implantation et à l’aspect extérieur des constructions) (Plan Locaux d’Urbanisme), pour les travaux suivants : Isolation thermique par l’extérieur en saillie des façades, isolation par surélévation des toitures, installation de protections solaires en saillie des façades (en attente du décret)
			Nouvelles règles sur le tiers financement
			Réservation de places de stationnement aménagées pour les véhicules électriques ou hybrides rechargeables.
			Obligation de pose de compteurs de chaleur : obligation renforcée, le syndic a l’obligation d’inscrire la question à l’ordre du jour et des sanctions financières sont prévues pour la 1ere fois (article L 241-9 du Code de l’Energie) :
			L’étendue de cette obligation a été précisée par le Décret \no  2016-710 du 30 mai 2016 relatif à la détermination individuelle de la quantité de chaleur consommée et à la répartition des frais de chauffage dans les immeubles collectifs
			Modernisation du carnet d’entretien de l’immeuble : il devra désormais être numérique et donner la part belle à la mesure de la performance énergétique. Pour commencer il deviendra obligatoire pour tous les logements neufs dont le permis de construire a été déposé après le 1er janvier 2017 et puis il sera progressivement étendu à l’ensemble du parc immobilier français.
			Ce texte a fait l’objet du Décret d’application \no  2016-1965 du 28 décembre 2016.
			
		\subsubsection{Incidences collaterales de l’acceleration des reformes « connexes »}
		
			\paragraph{La loi \no  2015-1776 du 28 décembre 2015 relative à l'adaptation de la société au vieillissement.}
			
			Cette loi modifie le titre IV bis de la loi du 10 juillet 1965 relatif aux résidences services en copropriété.
			Selon le nouvel article 41-1 de la loi du 10 juillet 1965 « Le règlement de copropriété peut étendre l'objet d'un syndicat de copropriétaires à la fourniture aux résidents de l'immeuble de services spécifiques dont les catégories sont précisées par décret et qui, du fait qu'ils bénéficient par nature à l'ensemble de ses résidents, ne peuvent être individualisés. Les services non individualisables sont fournis en exécution de conventions conclues avec des tiers. Les charges relatives à ces services sont réparties en application du premier alinéa de l'article 10.(…) »
			
			Le décret \no  2016-1737 du 14 décembre 2016 relatif aux résidences-services en copropriété définit limitativement les catégories de services non individualisables : accueil permanent, veille et surveillance, accès aux espaces de convivialité.
			
			\paragraph{Ordonnance \no  2016-131 du 10 février 2016 portant réforme du droit des contrats, du régime général et de la preuve des obligations}
			
			Le Règlement de copropriété est un contrat, quand bien même il est de nature institutionnel. Il est donc soumis aux règles du code civil Livre Troisième Titre III relatives aux Contrats et Obligations et notamment quant à l’effet des obligations ou encore de l’interprétation des conventions.
			
			L’Ordonnance du 10 février 2016 n’apporte pas de bouleversements aux dispositions légales d’origine et s’apparente davantage – sauf volonté contraire du texte - à une mise à jour au regard de la jurisprudence qui s’est forgée sur plus de deux siècles en application du code civil. Toutefois les innovations sont nombreuses, notamment la disparition de la notion de cause du contrat au profit du contenu du contrat, la non-rétroactivité de la clause suspensive, la procédure interrogatoire de l’article 1183 (si un contractant craint une action en annulation du contrat il peut intimer au co-contractant d’introduire son action dans le délai de six mois, à peine de forclusion)
		
			L’impact de cette réforme sera nécessairement limité en droit de la copropriété car :
			\begin{itemize}
				\item le texte entrera en vigueur le 1er octobre 2016 et tous les contrats (donc les règlements) antérieurs à cette date resteront soumis à «l‘ancien droit » (article 9 de l’Ordonnance du 10 février 2016) ;
				\item l’article 1105 nouveau précise que les règles générales des contrats s’appliquent sous réserve des règles particulières (\emph{specialia genralibus derogant}).
			\end{itemize}
			
			Cette observation n’est pas neutre s’agissant par exemple de la sanction aux infractions commises en violation du Règlement de copropriété.
			L’article 1143 actuel du code civil édicte que « le créancier est en droit de demander que ce qui a été fait par contravention à l’engagement soit détruit et l’on connaît la jurisprudence en application de ce texte qui veut que ces dispositions s’imposent au juge.
			Par exemple l’arrêt 3\ieme{} Chambre civile du 18 janvier 1972, Bull. civ. III, \no  30, p. 28 : « Violent l’article 1143 les juges du fond qui refusent d’ordonner la mise en état des lieux alors qu’ils constatent une infraction à une disposition du Règlement de copropriété »\footnote{
			Cette jurisprudence est d’application particulièrement stricte en matière de lotissement où le non respect du cahier des charges permet à tout co-loti de demander la démolition de l’ouvrage dès lors que cette démolition est possible, sans aucune référence à la disproportionnalité entre l’infraction au contrat et la sanction de démolition (civ. 3\ieme{} ch.,21 janvier 2016, \no  15-10566 – au Bulletin).
			}.
			Or, le nouvel Article 1221 est ainsi rédigé : « Le créancier d'une obligation peut, après mise en demeure, en poursuivre l'exécution en nature sauf si cette exécution est impossible ou s'il existe une disproportion manifeste entre son coût pour le débiteur et son intérêt pour le créancier ».
			$\dots$ sauf pour les juges à décider d’appliquer le nouveau texte, par exemple en matière d’interprétation des contrats. Il semble au demeurant que dès à présent la Cour de cassation tende à modérer sa rigueur comme en atteste par exemple un arrêt de la 3\ieme{} Chambre civile du 21 janvier 2016 (pourvoi \no  14-26085) qui a considéré que le juge après avoir annulé un contrat de construction d’une maison individuelle n’était pas tenu d’ordonner cette démolition dès lors que le bénéficiaire de la décision n’avait demandé que des Dommages et Intérêts en réparation de son préjudice tenant à l’annulation du contrat. En l’espèce le constructeur avait demandé au juge d’appel de dire que la résolution du Contrat ne permettait au maître de l’ouvrage que de demander la démolition et la restitution des fonds versés mais ne leur permettait pas de garder la maison et d’obtenir des Dommages et Intérêts.
			A ce sujet, rappelons que lorsque la démolition est demandée non pas devant le juge du fond, mais en référé, sur le fondement de l’article 809 du code de procédure civile, le juge des référés dispose d’un pouvoir d’appréciation (il peut prescrire la démolition ; il n’y est pas obligé).
			Par contre, et bien évidemment, le syndicat des copropriétaires ayant la personnalité morale peut passer tous les contrats conforme à son objet et à ce titre, à compter du 1er octobre 2016 il devra bien évidemment se soumettre au nouveau droit. A ce propos, relevons la nouvelle rédaction de l’article 1145 : « La capacité des personnes morales est limitée aux actes utiles à la réalisation de leur objet tel que défini par leurs statuts et aux actes qui leur sont accessoires, dans le respect des règles applicables à chacune d'entre elles. »
			
			\paragraph{Ordonnance \no  2016-301 du 14 mars 2016 : la refonte du code de la consommation}

			Dans un article liminaire le non-professionnel est défini comme « toute personne morale qui agit à des fins qui n’entrent pas dans le cadre de son activité commerciale, industrielle, artisanale, libérale ou agricole ».
			Le syndicat des copropriétaires est un « non professionnel » et donc soumis aux dispositions concernant « les non professionnels », telles que la faculté de dénoncer un contrat reconduit par tacite reconduction, en l’absence de dénonciation de la faculté de résiliation par le professionnel
			On trouve dans ce code, dans un livre dédié (Livre V – Pouvoirs d’enquête et suites données aux contrôles), les procédures et pouvoirs de la DGCCRF, notamment les dispositions des articles L 511-3 et suivants relatives aux Agents de la concurrence, de la consommation et de la répression des fraudes qui « sont habilités à rechercher et constater les infractions ou les manquements aux dispositions relatives aux honoraires du syndic et au contrat type ( art L 511-7 du Code cons)
		
	\section[La loi \textsc{alur} et ses suites]{La loi \textsc{alur} (loi \no  2014-366 du 24 mars 2014 pour l’accès au logement et a un urbanisme rénové) et ses suites}
	
		\subsection{Les principaux axes de la reformes}
			La réforme dite « loi \nom{Duflot} » du nom du ministre du logement, comme les textes importants de ces 30 dernières années est élaborés par le Ministère du logement et de l'urbanisme, n'a pas pour objectif unique la réforme du droit de la copropriété. Bien au contraire son objet essentiel est de régler les relations entre propriétaires et locataires pour « permettre l'accroissement de l'offre de logements ».
			
			La réforme du droit de la copropriété est introduite dans le titre II intitulé : « Lutter contre l’habitat indigne et les copropriétés dégradées », et largement inspiré du rapport de Dominique Braye, directeur de l’ANAH, intitulé « Prévenir et guérir les copropriétés en difficultés ». C’est donc avant tout une réforme axée, une nouvelle fois, sur les pathologies des copropriétés, devenues un enjeu majeur de la politique de la Ville. Il ne s’agit plus, toutefois, de rénover une nouvelle fois le régime des copropriétés en difficulté ( bien que la loi comporte des dispositions très innovantes sur le sujet), mais de réécrire l’ensemble de la loi pour en combattre les causes, qui résultent, selon le rapport BRAYE, d’une combinaison de facteurs institutionnels, économiques et sociaux.
			
			La loi repose donc sur ces trois piliers :
			\begin{itemize}
				\item \textbf{Institutionnel}, avec un contrôle accru du syndic, et un nouvel abaissement des seuils de majorités, notamment pour les travaux ;
				\item \textbf{Économique}, avec un élargissement et une protection accrue des ressources du syndicat ( compte séparé, fond de travaux), ainsi qu’une meilleure programmation des travaux ;
				\item \textbf{Social}, ce volet passant par la lutte contre les acquéreurs déstabilisateurs, l’amélioration de l’information de l’acquéreur et un meilleur recensement des copropriétés pour permettre l’action publique très en amont.
			\end{itemize}
			
			La réforme se caractérise par un double éclatement du statut de la copropriété.
			\begin{itemize}
				\item L’éclatement des dispositions s’accentue, entre la Loi du 10 juillet 1965 et le CCH, qui comporte désormais un livre VII consacré aux immeubles relevant du statut de la copropriété, qui traite de l’immatriculation des syndicats (titre I) – des dispositions	relatives à l’information des acquéreurs (titre II) de l’entretien, de la conservation et de la l’amélioration des immeubles (titre III).
				\item Plusieurs régimes juridiques distincts apparaissent en fonction du nombre de lots compris dans un syndicat de copropriétaires, des modalités d’occupation du syndicat de copropriété (logement seul, destination partielle ou totale d’habitation, syndicats de copropriétaires composés uniquement de personnes morales).
			\end{itemize}
			
			Cette réforme est sans doute la plus complète depuis la rédaction du texte d’origine, malheureusement elle conserve une approche éclatée, alors qu’une véritable « refonte » du statut serait sans doute nécessaire.
		
		\subsection{Les dispositions essentielles de la reforme}
		
			\subsubsection{Recensement, prévention et traitement des copropriétés en difficultés}
				\begin{enumerate}
					\item la création d'un registre national des copropriétés dont le but à terme est que tout syndicat de copropriété soit immatriculé et fournisse d'importants éléments d'information sur sa situation juridique et financière (entrée en vigueur progressive, à compter du 31 décembre 2016)
					\item la création d’une « fiche synthétique » pour chaque copropriété
					\item la réécriture des dispositions relatives aux copropriétés en pré-difficulté et en difficulté avec l'accentuation de la faculté pour les pouvoirs publics d'intervenir dans le parc privé. Le texte permet un véritable « rétablissement » des copropriétés en difficulté par une procédure de recensement des créanciers, d'établissement d'un plan d'apurement des dettes sur cinq ans avec l'effacement partiel de certaines dettes.
				\end{enumerate}
			
			\subsubsection{Nouvelles formalités préalables a la vente du lot}
				\begin{enumerate}
					\item Information sommaires dès l’annonce immobilière
					\item Information lors de la Promesse de vente : la remise du règlement de copropriété, de l’état descriptif de division, des procès verbaux d’assemblées générales et du « pré-état daté » subordonne le point de départ du délai de rétractation pour l’acquéreur.
					\item certains acquéreurs « déstabilisateurs » sont tous simplement frappés d’une interdiction d’acquérir (marchand de sommeil, copropriétaire débiteur déjà titulaire d’un lot dans la copropriété)
					\item Un diagnostic technique global remplace en outre le DPE.
				\end{enumerate}
			
			\subsubsection{Encadrement plus rigoureux du syndic}
				\begin{enumerate}
					\item Modification des règles professionnelles (Loi Hoguet) : nouvelles règles sur la délivrance de la carte professionnelle ( devenue la carte S délivrée par les CCI), sur la formation continue, la déontologie, et création du conseil national de la transaction et de la gestion immobilières et de la commission de contrôle des activités de transaction et de gestion immobilières
					\item Mise en concurrence obligatoire du syndic par le conseil syndical ( tous les ans selon la loi ALUR, puis tous les 3 ans selon la loi Macron)
					\item Principe de forfaitisation des honoraires et renvoi à un « contrat type » à définir par décret
					\item l'augmentation des tâches incombant au syndic, dont la mise en concurrence devient obligatoire, et la rémunération forfaitaire
					\item L’obligation, sans dérogation possible (sauf pour les petites copropriétés de moins de 10 lots et <15.000 euros de budget), pour le syndic de remettre les fonds reçoit sur un compte bancaire ouvert au nom du syndicat des copropriétaire.
					\item Encadrement de la fin de mandat ( préavis en cas de démission, règles relatives à l’empêchement), interdiction de siéger au conseil syndical faite aux collatéraux du syndic, modification des règles concernant les « conventions réglementée »
				\end{enumerate}
			
			\subsubsection{dispositions destinées a faciliter l’adoption des travaux}
				\begin{enumerate}
					\item Création du Diagnostic Technique Global, qui se substitue au DPE
					\item Institution d’un fonds de prévoyance obligatoire, intitulé « fonds de travaux » correspondant à 5 \% au moins du budget annuel, définitivement acquis au syndicat, qui doit être abondé tant que le plafond des travaux préconisés par le DTG n’est pas atteint
					\item Modification substantielle des règles de majorité aux assemblées générales. Les travaux d'amélioration relèveront désormais de la majorité de l'article 25 de la loi tandis que les travaux obligatoires relèveront de la majorité de l'article 24.
					\item Nouvelles dispositions destinées à favoriser la surélévation
				\end{enumerate}
			
			\subsubsection{Autres dispositions destinées a fluidifier le fonctionnement du syndicat}
			\begin{enumerate}
				\item Améliorations relatives à l’assemblée : convocation par tout copropriétaire en cas d’absence de syndic, convocation électronique ( entrée en vigueur différée jusqu’au décret)
				\item Amélioration de la transparence : extranet (à compter du 1er janvier 2016), nouvelles modalités de consultation des pièces comptable et d’information des résidents (entrée en vigueur suspendues aux décrets pour ces deux derniers points)
				\item la modification des règles de représentation des copropriétaires aux assemblées générales dans les grands ensembles (syndicats secondaires et ASL).
				\item la consécration de la division en volumes avec la faculté pour les copropriétés, sous certaines conditions, de faire une scission en volumes.
			\end{enumerate}
			
		\subsection{Les principaux décrets d’application de la loi \textsc{alur}}	
			La loi ALUR est entrée en vigueur pour partie de façon immédiate, pour partie de façon différée (entre le 1er juillet 2016 et le 31 décembre 2018 notamment) , compte tenu du temps nécessaire à la mise en place concrète de la réforme (nouveau contrat de syndic, compte bancaire séparé obligatoire, immatriculation des syndicats, fond de travaux).
			Mais surtout, la mise en oeuvre de nombreuses disposition est subordonnée à l’entrée en vigueur de plusieurs dizaines de décrets d’application, dont le calendrier de publication a pris beaucoup de retard\footnote{Sur l’ensemble du sujet voir Informations Rapides de la Copropriété \no  604, décembre 2014, Les impacts de la loi ALUR sur le régime de la copropriété ; Actes Pratiques et Ingéniérie Immobilière (revue du JurisClasseur) juin 2014 La Loi ALUR ; Administrer Juin et septembre 2014 (Commentaires Capoulade).}. 

			\paragraph{Décret \no 2014-843 du 25/07/2014} Composition et les modalités de constitution et de fonctionnement du conseil national de la transaction et de la gestion immobilières.
			
			\paragraph{Décret \no 2015-342 du 26/03/2015} Contrat type à respecter par le contrat de syndic (entrée en vigueur pour les mandats conclus à compter du 1er juillet 2015).
			
			\paragraph{Décret \no 2015-518 du 11/05/2015} Assurance obligatoire du syndicat des copropriétaires (fonctionnement du BECT, montant de la prime et de la franchise)
			
			\paragraph{Décret \no 2015-702 du 19/06/2015} Carte professionnelle : délivrance par le président de la chambre de commerce et d’industrie territoriale ou par le président de la chambre de commerce et d’industrie départementale d’Île-de-France.
			
			\paragraph{Décret \no 2015-999 du 17/08/2015} Modalités d’intervention et désignation du mandataire ad hoc (copropriétés en pré-difficulté), nouvelles règles relatives à la liquidation du syndicat.
			
			\paragraph{Décret \no 2015-1090 du 28/08/2015} Règles constituant le code de déontologie applicable aux personnes exerçant des activités d'entremise et de gestion des immeubles et fonds de commerce ( mais la commission de contrôle de la gestion immobilières n’est pas encore constituée).
			
			\paragraph{Décret \no 2015-1325 du 21/10/2015} relatif à la dématérialisation des notifications et des mises en demeure concernant les immeubles soumis au statut de la copropriété des immeubles bâtis.
			
			\paragraph{Décret \no 2015-1681 du 15/12/2015} Information des occupants de chaque immeuble de la copropriété des décisions prises par l’assemblée générale (le décret s'applique aux assemblées générales convoquées à compter du 1er avril 2016 : obligation d’un affichage destiné aux résidents, dans les parties communes, relatif aux décisions d’assemblée générale)
			
			\paragraph{Décret \no 2015-1907 du 30/12/2015} Modalités de consultation des justificatifs de charges ( au moins un jour ouvré, fixé par le syndic dans la convocation, avec faculté de prendre copie des pièces aux frais du copropriétaire).
			
			\paragraph{Décret \no 2016-173 du 18/02/ 2016} relatif à la formation continue des professionnels de l’immobilier : au moins 14 heures par an (dont 2 heures consacrées à la déontologie) et de 42 heures sur trois ans
			
			\paragraph{Décret \no 2016-1167 du 26 août 2016} relatif au registre national d'immatriculation des syndicats de copropriétaires Et Arrêté du 10 octobre 2016 sur immatriculation
			
		\subsection{Le detricotage de la loi \textsc{alur}}
			Certaines dispositions de la loi ALUR vont faire l’objet, en moins de 18 mois, de « correctifs »
			
			\paragraph{Loi \no 2014-1545 du 20/12/2014} Le texte revient sur l’obligation, prévue par la loi ALUR, de mentionner dans le certificat \nom{Carrez} (art.46 de la loi du 10.07.65) la superficie de la partie privative et la surface habitable, après avoir constaté que ces deux superficies $\dots$ sont identiques !
			
			\paragraph{Loi \no 2015-990 du 6 août 2015} pour la croissance, l'activité et l'égalité des chances économiques, dite Loi \nom{Macron}, modifiant les conditions de mise en concurrence du syndic (tous les 3 ans, avec une possibilité de dispense par l’assemblée de l’année précédant cette mise en concurrence obligatoire)

			\paragraph{Loi \no 2016-1321 du 7 octobre 2016} pour une République numérique habilite le gouvernement pour prendre par « ordonnance toute mesure relevant du domaine de la loi afin de favoriser la dématérialisation par le développement de l'envoi de documents par voie électronique, de l'usage de la signature électronique et de la lettre recommandée électronique dans les relations entre les personnes soumises à la loi \no  65-557 du 10 juillet 1965 fixant le statut de la copropriété des immeubles bâtis. »
	
	\section{La loi \no 2018-1021 du 23 novembre 2018 dite elan}
		\subsection{La loi \no 2018-1021 du 23 novembre 2018 dite elan}
			Le Gouvernement a fixé en 2018 plusieurs axes d'une politique de logement renouvelée, quitte à « bousculer » la loi ALUR :
			\begin{itemize}
				\item construire plus, mieux et moins cher pour provoquer un choc d'offres ;
				\item  accompagner l'évolution du secteur du logement social ;
				\item répondre aux besoins de chacun et favoriser la mixité sociale ;
				\item améliorer le cadre de vie.
			\end{itemize}
			Dans ce but il a rédigé un projet de loi « portant \textbf{évolution} du \textbf{logement}, de l’\textbf{aménagement} et du \textbf{numérique} » qui se veut une nouvelle approche de la politique du logement en France devant « conduire à davantage d’équilibre territorial et de justice sociale en faveur des plus fragiles, être un moteur durable de l’économie locale comme nationale et un vecteur d’innovation ».
			
			Ce projet a envisagé « des mesures en faveur de l’amélioration de la gouvernance des copropriétés pour remédier au constat d’un relatif vieillissement de la loi du 10 juillet 1965 ($\dots$) qui induit certaines rigidités dans la gouvernance et les modalités de décision et peut retarder par exemple la nécessaire rénovation énergétique des bâtiments »\footnote{Présentation du Projet de Loi, exposé des motifs}.
			Il convient d’observer que l’adoption de cette loi par le Parlement, après que l’Urgence ait été déclarée, a été précédée d’une « conférence de consensus sur le logement » en présence des principaux acteurs qui s’est déroulée au Sénat entre décembre 2017 et février 2018.
			Les dispositions finalement adoptées sur la loi de 1965 sont de deux ordres :
			\begin{enumerate}[label=\roman*)]
				\item En premier lieu, l’habilitation donnée au gouvernement pour légiférer par ordonnance (article 214 dans la Loi)
				\begin{itemize}
					\item dans le délai d’un an de la promulgation de la loi ELAN, à prendre par voie d’ordonnance les mesures visant à redéfinir le champ d’application de la loi de 1965 ainsi qu’à « Clarifier, moderniser, simplifier et adapter les règles d’organisation et de gouvernance de la copropriété, celles relatives à la prise de décision par le syndicat des copropriétaires ainsi que les droits et obligations des copropriétaires, du syndicat des copropriétaires, du conseil syndical et du syndic » ;
					\item dans un délai de deux ans, à procéder par voie d’ordonnance pour codifier la partie législative d’un code relatif à la copropriété ;
					\item  Un projet de loi de ratification devra être déposé devant le Parlement dans un délai de trois mois à compter de la publication de chaque ordonnance
				\end{itemize}
				\item  En second lieu, les articles 203 à 213, apportent dès à présent des modifications substantielles à différents articles de la loi de 65\footnote{La question est de savoir si ces nouvelles dispositions vont s’imposer au gouvernement dans le cadre de la refonte de la loi de 1965 ou si celui-ci sera libre de supprimer ou de modifier ces modifications. En principe le gouvernement n’est pas tenu de respecter ces nouvelles dispositions des articles 203 à 213, mais sa position serait inconfortable lorsqu’il demandera dans les trois mois suivant la promulgation de chaque ordonnance.}, ces modifications ayant été ajoutées par le Sénat au projet du gouvernement : le Sénat étant hostile dans un premier temps à la mise en œuvre de la procédure relative aux ordonnances gouvernementales. Certaines sont d’applications immédiates. Ces dispositions comprennent :
				\begin{itemize}
					\item la réécriture des premiers articles de la loi (intégration de la jurisprudence sur le lot de copropriété, le lot transitoire, les Parties communes spéciales, les Parties communes à jouissance privative, date d’entrée en vigueur du statut en cas de VEFA), avec une obligation de mise à jour des règlements de copropriété dans les 3 ans
					\item  des dispositions destinées à faciliter le recouvrement des charges (déchéance du terme) et l’utilisation du fond de travaux
					\item  des dispositions relatives à l’assemblée (délégations de vote --- nombre et subdélégation, interdictions étendues pour la prise de mandat, visioconférence et vote par correspondance, délai de notification des PV réduit à un mois) ;
					\item  la « remontée » à l’article 25 de la majorité pour les travaux de rénovation énergétique
					\item  des dispositions relatives à la numérisation et à l’accès à l’information : Sanction en cas de défaut de transmission des documents au conseil syndical, Contenu de l’extranet, Réception des coordonnées des locataires par le syndic, carnet numérique
					\item  des dispositions diverses : alignement de la prescription des actions en copropriété sur le droit commun (5 ans), intégration des colonnes montantes au domaine public, renforcement du dispositif de lutte contre les locations touristiques meublées), Individualisation des frais de chauffage et de refroidissement.
				\end{itemize}
			\end{enumerate}

		\subsection{Les decrets d’application de la loi \no  2018-1021 du 23 novembre 2018 dite elan}
			\paragraph{Décret \no 2019-502 du 23 mai 2019} relatif à la liste minimale des documents dématérialisés concernant la copropriété accessibles sur un espace sécurisé en ligne.
			
			L’Intranet obligatoire doit être sécurisé (code personnel), avec des documents téléchargeables et imprimables, actualisés au minimum une fois par an par le syndic, dans les trois mois précédant l'assemblée générale annuelle.
			
			Les documents mis en ligne doivent permettre à chaque copropriétaire fournir les documents nécessaires à une promesse de vente (règlement de copropriété, procès-verbal des 3 dernières années), d’accéder aux principaux contrats ( syndic, assurance) et au conseil syndical d’exercer son contrôle (balance des comptes, relevé du compte bancaire, assignation).
			
			\paragraph{Décret \no 2019-503 du 23 mai 2019} fixant le montant minimal de pénalités applicables au syndic en cas d’absence de communication des pièces au conseil syndical (15 euros).
			
			\paragraph{Décret \no  2019-650 du 27 juin 2019} portant diverses mesures relatives au fonctionnement des copropriétés et à l'accès des huissiers de justice aux parties communes d'immeubles.
			Ce décret précise les modalités de la « dématérialisation » des actes en copropriété :
			\begin{itemize}
				\item Notifications Electroniques
				\item Dématérialisations des annexes de l’assemblée générale
				\item Dématérialisation des appels de fonds
			\end{itemize}
			Le décret précise également que si un huissier doit accéder aux parties communes, le syndic doit lui fournir les codes ou le moyen d’accès dans les 5 jours de sa demande
		
		\subsection{L’ordonnance \no  2019-1101 du 30 octobre 2019 portant reforme du droit de la copropriete des immeubles batis}
			L’Ordonnance \no  2019-1101 du 30 octobre 2019 portant réforme du droit de la copropriété des immeubles bâtis (publiée au Jo du 31.10.2019, avec un rapport au Président de la République), entrera en vigueur le 30 juin 2020.
			Cette ordonnance a été adoptée après consultation des professionnels, des associations de consommateurs et des « corps constitués » (notaires, avocats $\dots$).
			Elle s’affiche clairement comme la réforme la plus ambitieuse du statut depuis 1965, en tout cas la première conçue comme réformant la totalité du statut, et consacrée uniquement à la copropriété.
			Elle ne constitue pas une révolution, mais plutôt une évolution du statut, destiné à le moderniser en en conservant les caractéristiques essentielles.

			Selon le Rapport, la genèse la réforme s’explique ainsi :
			\begin{quote}
				En 2015, le 50e anniversaire de la loi du 10 juillet 1965 a été l’occasion pour la doctrine et les praticiens de s’interroger sur les difficultés d’application du statut de la copropriété des immeubles bâtis.
				Il a alors été mis en lumière la nécessité de préserver ce système original, fondé sur des grands principes issus du droit des biens (propriété, indivision), du droit des personnes (personnalité morale du syndicat des copropriétaires, dotée d’un « patrimoine ») et du droit des obligations (paiement des charges), qui a inspiré de nombreux pays (Québec, Belgique, Algérie, Côte d’Ivoire, Haïti, etc.). Dans le même temps, ont également été soulignées les limites de ce statut, certes protecteur, mais n’offrant pas la possibilité d’une adaptation aux spécificités de chaque copropriété, qu’il s’agisse notamment de leur taille, de leur structure ou de leur destination. [$\dots$]
				A enfin été soulignée la nécessité d’une meilleure prise en compte par les copropriétaires de la dimension collective de la copropriété et de la nécessité de préserver leur patrimoine commun, en anticipant la réalisation de travaux indispensables à la conservation de leur immeuble.
			\end{quote}
			
			Aussi, les principales mesures de la Réforme sont-elles :
			\begin{itemize}
				\item La modernisation du statut par une redéfinition du champ d’application de la loi du 10 juillet 1965 au regard des caractéristiques, de la destination ou de la taille des immeubles ainsi que des règles applicables à ces copropriétés. Ainsi, le statut de la copropriété n’est plus obligatoire pour les immeubles autres que d’habitation, et il est en outre prévu un statut très dérogatoire pour les « petites copropriétés » (au plus cinq lots à usage de logements, de bureaux ou de commerces, ou lorsque le budget prévisionnel moyen du syndicat sur une période de trois exercices consécutifs est inférieur à 15 000 euros.)
				\item  une nouvelle simplification des règles de décisions en assemblée générale, avec un abaissement des majorités (même celle de l’article 26, avec un second vote possible),
				\item  la facilitation des travaux du syndicat des copropriétaires (notamment par la généralisation des « Travaux d’intérêt collectif ») ou à l’initiative d’un copropriétaire
				\item  un accroissement des pouvoirs du conseil syndical, avec des facultés de délégation beaucoup plus importantes
				\item  l’encadrement de la fin du mandat de syndic
				\item  la poursuite de l’effort de « codification à droit constant engagé par le Sénat avec la loi \no  2018-1021 du 23 novembre 2018 dite ELAN, avec quelques « retouches » des définitions qui y figuraient, ainsi que des définitions complémentaires
			\end{itemize}

	