\chapter{Le recouvrement des provisions et des charges}

SECTION I - LE RECOUVREMENT ACCELERE DES PROVISIONS ET DES CHARGES : ARTICLE 19-2 DE LA LOI
A. HISTORIQUE DE L'ARTICLE 19-2.
1. Innovation de la loi SRU
Dans le but d’accélérer le recouvrement des provisions, la loi SRU du 13 décembre 2000 a ajouté à la loi un article 19-2 permettant de saisir le « Président du Tribunal statuant comme en matière de référé » aux fins d’obtenir condamnation du copropriétaire débiteur au paiement des provisions votées par l’assemblée générale au titre du budget (article 14-1)
Le juge « statuant comme en matière de référé » désignait le Président du Tribunal statuant selon la procédure des référés (assignation à jour fixe et à bref délai), mais dont la décision était une décision au fond alors que le principe est qu’une décision rendue « en référé » se caractérise par son caractère provisoire, une partie pouvant introduire une procédure au fond et obtenir un jugement en sens contraire !
C’était donc une procédure efficace puisqu’abrégeant de façon très sensible les délais pour obtenir une condamnation au fond581.
Mais la portée de la décision était limitée aux seules sommes dues par le copropriétaire débiteur au titre du budget ordinaire, alors que bien évidemment les copropriétaires débiteurs doivent le plus souvent également des sommes au titre du budget travaux (I de l’article 14-2), et surtout des arriérés sur les charges (quote-part de dépenses dues par les copropriétaires sur les exercices antérieurs approuvés par l’assemblée générale).
Certes l’article 19-2 permettait au syndicat d’obtenir non seulement la condamnation du copropriétaire à sa quote-part des provisions échues mais instituait également une déchéance du terme rendant exigible immédiatement la totalité de la quote-part du copropriétaire sur l’exercice en cours. Mais encore fallait-il pour que cette mesure ait une quelconque efficacité que le juge soit saisi le plus tôt possible dans l’année
581 Actuellement une procédure ordinaire au fond prend au moins deux ans à Paris ou Nanterre.
droit de la copropriété année 2019-2020
475
(en tout cas avant le 1er octobre si l’année comptable commence au 1er janvier) date normale d’exigibilité des provisions sur les quatre trimestres !
Quand bien même il était alors possible d’introduire une procédure en référé (ou devant le Président du Tribunal statuant comme en matière de référé), le plus souvent le syndicat des copropriétaires s’il voulait mettre cette procédure en oeuvre, devait payer des frais et honoraires de procédure correspondant à un montant plus élevé que les « provisions sur budget » à récupérer, y compris avec la déchéance du terme.
C’est en vain que les plaideurs ont tenté d’étendre leur demande aux provisions votées pour les travaux ou au paiement des charges des exercices écoulés : même si certains juges de proximité582ont pu accepter ces demandes, les cours d’appel et la cour de cassation ont rappelé qu’e l’on ne pouvait agir sur le fondement de l’article 19-2 que pour les provisions votées sur le budget583.
2. Extension de l’article 19-2 au fonds de travaux par la loi ALUR
Art. 19-2 de la loi du 10 juillet 1965 modifié par la loi ALUR .
La loi ALUR, en créant le fonds de travaux obligatoire a, logiquement complété l’article 19-2 en ajoutant une phrase à la rédaction d’origine :
« Le présent article est applicable aux cotisations du fonds de travaux mentionné à l’article 14-2 »584.
Certes, il s’agit là d’une bonne intention mais dont la portée est microscopique lorsque l’on songe que le fonds de travaux appelé annuellement est presque toujours du minimum légal : 5 % du montant du budget !
Il n’y avait pas là matière à modifier la position antérieure des syndics : autant assigner en référé ordinaire pour avoir condamnation, certes provisionnelle, du copropriétaire défaillant pour la totalité des sommes exigibles : charges échues et provisions sur budget et travaux exigibles.
3. Généralisation de la procédure de l’article 19-2 par la loi ELAN
Pour répondre à la demande formulée par tous les praticiens et par le GRECCO, la loi a étendu la procédure
- Aux sommes devenues exigibles au titre des travaux votés (article 14-2)
582 Voire même la Cour d’Appel de Bastia dans un arrêt du 21 septembre 2011 \no 10/00504
583 CA Paris 8 octobre 2008, \no 08/02744, JD 2008-370462, 3\degres Ch. Civ. 20 juin 2012, Pourvoi \no 11-16307, Inédit
584 Disposition entrée en vigueur le 1er janvier 2017 : date de mise en vigueur du fond de travaux
droit de la copropriété année 2019-2020
476
- Aux sommes restant dues appelées au titre des exercices précédents approuvés,
par l’ajout de quelques mots au premier alinéa de l’article 19 qui est désormais rédigé comme suit :
ARTICLE 19-2 ALINEA 1ER
A défaut du versement à sa date d'exigibilité d'une provision due au titre de l'article 14-1 ou du I de l'article 14-2, et après mise en demeure restée infructueuse passé un délai de trente jours, les autres provisions non encore échues en application des mêmes articles 14-1 ou 14-2 ainsi que les sommes restant dues appelées au titre des exercices précédents après approbation des comptes deviennent immédiatement exigibles.
4. Exception de l’article 43 de la loi : budget de l’année n + 1 non voté par l’ag
L’article 43 du Décret de 1967 édicte que :
Le budget prévisionnel couvre un exercice comptable de douze mois. Il est voté avant le début de l'exercice qu'il concerne.
Toutefois, si le budget prévisionnel ne peut être voté qu'au cours de l'exercice comptable qu'il concerne, le syndic, préalablement autorisé par l'assemblée générale des copropriétaires, peut appeler successivement deux provisions trimestrielles, chacune égale au quart du budget prévisionnel précédemment voté. La procédure prévue à l'article 19-2 de la loi du 10 juillet 1965 ne s'applique pas à cette situation.
Bien qu’il s’agisse ici d’une provision sur budget, cette provision n’a fait l’objet d’aucun vote puisque l’assemblée générale de l’année précédente a simplement autorisé la reconduite du budget de l’année antérieure pour une période de six mois au maximum. Très logiquement ces provisions non votées ne peuvent faire l’objet de la procédure de l’article 19-2 de la loi ; ceci quand bien même la loi ELAN a étendu le bénéfice de cette procédure aux sommes dues au titre des exercices précédents approuvés.
PROCEDURE DE L’ARTICLE 19-2 DE LA LOI
ARTICLE 19-2 ALINEA 2 A 5
Le président du tribunal judiciaire statuant selon la procédure accélérée au fond, après avoir constaté, selon le cas, l'approbation par l'assemblée générale des copropriétaires du budget prévisionnel, des travaux ou des comptes annuels, ainsi que la défaillance du copropriétaire, condamne ce dernier au paiement des provisions ou sommes exigibles.
Le présent article est applicable aux cotisations du fonds de travaux mentionné à l'article 14-2.
droit de la copropriété année 2019-2020
477
Lorsque la mesure d'exécution porte sur une créance à exécution successive du débiteur du copropriétaire défaillant, notamment une créance de loyer ou d'indemnité d'occupation, cette mesure se poursuit jusqu'à l'extinction de la créance du syndicat résultant de l'ordonnance.
Si l'assemblée générale vote pour autoriser le syndic à agir en justice pour obtenir la saisie en vue de la vente d'un lot d'un copropriétaire débiteur vis-à-vis du syndicat, la voix de ce copropriétaire n'est pas prise en compte dans le décompte de la majorité et ce copropriétaire ne peut recevoir mandat pour représenter un autre copropriétaire en application de l'article 22.
UN ARTICLE « FOURRE TOUT »
On notera le caractère un peu « fourre-tout » de ce texte.
L’alinéa 5 est étranger à la procédure accélérée puisqu’elle se rapporte à la vente judiciaire du lot et au droit de vote du débiteur.
LA REFORME DE LA PROCEDURE : PROCEDURE ACCELEREE
On notera également qu’il n’est plus question de saisir le « Président du Tribunal statuant comme en matière de référé », mais le » président du tribunal judiciaire statuant selon la procédure accélérée au fond ».
Ce changement de terminologie résulte des dispositions de l’Ordonnance du 17 juillet 2019585 dans le cadre de la modification générale de la procédure devant le « Tribunal judiciaire ».
Concrètement le syndicat des copropriétaires devra impérativement avoir recours à un avocat pour introduire cette procédure, quel que soit le montant des sommes réclamées.
LA MISE EN DEMEURE PREALABLE
Que la demande soit faite pour avoir paiement de provisions article 14-1 ou 14-2 ou qu’elle soit faite pour avoir paiement des sommes restant dues sur les exercices antérieurs le syndic devra adresser une mise en demeure restée infructueuse pendant 30 jours.
S’agissant des sommes dues au titre des exercices antérieurs approuvés, il est probable que le syndic a déjà adressé une mise en demeure au copropriétaire. Compte tenu de la spécificité de la procédure de l’article 19-2 doit-il adresser une nouvelle mise en demeure visant les dispositions de cet article ? La réponse devrait être négative, sinon ce serait ajouter au texte qui exige seulement une mise en demeure et n’impose pas que celle-ci mentionne les dispositions de cet article.
585 Ordonnance \no 2019-738 du 17 juillet 2019 prise en application de l'article 28 de la loi \no 2019-222 du 23 mars 2019 de programmation 2018-2022 et de réforme pour la justice
droit de la copropriété année 2019-2020
478
Cette mise en demeure est faite par lettre recommandée ou par voie électronique si le copropriétaire a préalablement donné son accord pour ce faire en précisant qu’il accepte toutes les notifications par cette voie586
Le débiteur a 30 jours pour s’acquitter ; le délai commence à courir du lendemain de la première présentation de la lettre recommandée (article 64 du Décret de 67) ou du lendemain de la notification par voie électronique.587
RECOUVREMENT DES PROVISIONS ART 14-1, ART 14-2 I ET ART 14-2 II DECHEANCE DU TERME
30 jours après cette mise en demeure, et si le copropriétaire n'a toujours pas payé le terme exigible, c’est-à-dire le quart du budget, il devient alors débiteur de la totalité des trimestres de l'année budgétée.
Mais, cette déchéance du terme ne fait pas rentrer immédiatement l'argent dans les caisses du syndicat.
TENTATIVE PREALABLE OBLIGATOIRE DE RESOLUTION AMIABLE DU LITIGE
L’article 3 de la loi du 23 mars 2019 dite de programmation 2018-2022 et de réforme de pour la justice impose une tentative de conciliation, de médiation ou de convention de procédure participative préalable à la saisie de la juridiction lorsque la demande n’excède pas un montant défini par le Conseil d'Etat.
Le décret du 11 décembre 2019 réformant la procédure civile précise que cette tentative de résolution amiable du différend (article 750-1 CPC) est obligatoire pour toute demande en paiement n’excédant pas 5.000€
Toutefois l’alinéa 2 du nouvel article 750-1 du CPC énonce les situations dans lesquelles les parties peuvent être dispensées de cette tentative préalable de conciliation. Faute de disposition spécifique à la matière du recouvrement des charges, seule peut être invoquée « les situations dans lesquelles l’absence de tentative préalable de résolution amiable des différends est justifiée par un motif légitime ».
On peut ici se référer au Décret de 2015 obligeant à inscrire dans l’assignation en justice « les diligences entreprises en vue de parvenir à une résolution amiable du litige » qui précise les motifs légitimes de dispense, savoir : « motif tenant à l’urgence ou à la matière considérée, en particulier lorsqu’elle intéresse l’ordre public ».
586 Décret du 17 mars 1967, Art. 64-3 I - L'accord exprès du copropriétaire mentionné à l’article 42-1 de la loi du 10 juillet 1965 précise s’il porte sur les notifications, les mises en demeure ou les deux. Cet accord exprès peut ne porter que sur les modalités particulières de notification mentionnées à l’article 64-1.
587 Article 64-2 nouveau du Décret de 67 : « Le délai que les notifications et mises en demeure par voie électronique font courir a pour point de départ le lendemain de la transmission, par le prestataire de service de confiance qualifié, de l’avis électronique informant le destinataire d’un envoi électronique. »
droit de la copropriété année 2019-2020
479
Mais le Décret du 11 décembre ajoute que l’urgence doit être manifeste, ou tenir aux circonstances de l’espèce rendant impossible une telle tentative de conciliation.
Il est manifeste que la procédure de l’article 19-2 ne répond à aucune de ces exemptions.
Par contre reste un dernier motif possible : l’indisponibilité des conciliateurs de justice ; ce que l’on peut traduire par l’impossibilité d’obtenir une date de tentative de conciliation dans un délai raisonnable.
La médiation et la procédure participative sont des procédures onéreuses : la médiation est faite par un médiateur judiciaire et la procédure participative est un contrat judiciaire faisant nécessairement appel à des avocats.
La tentative de conciliation par un conciliateur de justice est une procédure gratuite devant une personne désignée par le premier président de la cour d’appel qui perçoit de l’Etat une indemnité annuelle de 928 €. Il n’est pas nécessaire de justifier d’une compétence juridique étendue pour devenir conciliateur de justice.
Il convient de s’adresser au greffe du Tribunal judiciaire en déposant un formulaire CERFA \no 15728*02 en précisant son identité, celle de son adversaire, avec une brève description du différend et en joignant les pièces. Le Greffe donnera le nom d’un ou plusieurs conciliateurs de justice à proximité du domicile du demandeur ou du défendeur.
Le conciliateur organisera la tentative de conciliation et sa mission ne peut normalement excéder trois mois …
Compte tenu du recours obligatoire à une telle tentative il est très probable que les conciliateurs ne pourront faire face à leurs missions dans des « délais raisonnables » …
Certes, pour échapper à cette obligation les avocats pourraient envisager des demandes de dommages-intérêts pour résistance abusive en sorte que le taux de 5.000 € (sommes dues et dommages intérêts) soit dépassé, mais rappelons que l’article 19-2 ne permet pas de solliciter des DI puisque son domaine d’application est limité par la loi aux provisions et charges !
SAISINE DU PRESIDENT DU TRIBUNAL SELON LA PROCEDURE ACCELEREE AU FOND
Le syndicat des copropriétaires – faute de conciliation - va saisir le Président du Tribunal par une assignation à jour fixe « selon la procédure accélérée au fond », en demandant la condamnation du copropriétaire à la totalité de sa quote-part sur l’année pour chaque catégorie de provision (provisions dues au titre du budget, des travaux ou du fonds de roulement de l’année en cours).
Le Président, après avoir :
droit de la copropriété année 2019-2020
480
- Constaté l’approbation de ces provisions (leur vote) par l’assemblée générale
- Constaté le défaut de paiement dans les 30 jours de la mise en demeure
Condamne le copropriétaire débiteur au paiement des provisions échues, ainsi qu’au paiement des provisions non encore exigibles : c’est la déchéance du terme.
Il s'agit là d'une sanction extrêmement forte. Au demeurant, lors de la discussion du projet devant la Commission de l'Assemblée Nationale, le rapporteur a fait valoir qu'il fallait user de cette nouvelle disposition avec précaution et qu'en réalité, l'arme n'avait été donnée au syndicat des copropriétaires qu'à l'effet de poursuivre les débiteurs de mauvaise foi, à l'exclusion des débiteurs de bonne foi, c’est-à-dire de ceux qui connaissaient des difficultés passagères.
RECOUVREMENT DES SOMMES RFESTANT DUES AU TITRE DES EXERCICES PRECEDENTS
La procédure débute par la mise en demeure et l’expiration du délai de 30 jours.
Le syndicat des copropriétaires va ensuite mettre en oeuvre la procédure de tentative de con ciliation si les charges dues représentent moins de 5.000 € et enfin saisir le Président du Tribunal par une assignation à jour fixe « selon la procédure accélérée au fond », en demandant la condamnation du copropriétaire au paiement des arriérés sur charges.
Le Président après avoir :
- Constaté l’approbation des charges (leur vote) par l’assemblée générale ou les assemblées générales précédentes
- Constaté le défaut de paiement dans les 30 jours de la mise en demeure
Condamne le copropriétaire débiteur au paiement de ces arriérés de charges
LE JUGE A L’OBLIGATION DE CONDAMNER
Dans sa rédaction antérieure à la loi ELAN le texte précisait : « le juge peut condamner ».
Le texte remanié par la loi ELAN affirme : « le juge condamne ce dernier (le copropriétaire) au paiement des provisions ou sommes exigibles ».
En sorte que si les deux conditions sont remplies (approbation des charges ou vote des provisions et défaut de paiement dans le délai de 30 jours) le juge a l’obligation de condamner le débiteur. Il s’agit bien évidemment d’un texte d’ordre public au sens de l’article 43 de la loi.
Par contre, pour tenir compte des circonstances propres au copropriétaire et sa bonne fois, lr Président pourra accorder des délais de paiement en application des dispositions de l’article 1343-5 du code civil : « Le juge peut, compte tenu de la situation du débiteur et en considération des besoins du créancier, reporter ou échelonner, dans la limite de deux années, le paiement des sommes dues. ». Le même article donnant faculté au juge de minorer le taux légal et de dire que les paiement s’imputeront d’abord sur le capital.
UNE PROCEDURE UNIQUE POUR LES PROVISIONS ET LES ARRIERRES DE CHARGES
droit de la copropriété année 2019-2020
481
Bien évidemment, si les vérifications portent sur des éléments distincts dans les deux hypothèses, le syndicat des copropriétaires ne fera qu’une seule procédure dès lors que le copropriétaire sera débiteur à la fois de provisions et de charges.
C’est le plus grand intérêt de la modification apportée par la loi ELAN.
Certes, toutes les sommes éventuellement dues par le copropriétaire ne peuvent faire l’objet de la procédure de l’article 19-2 : nous avons vu par exemple l’exclusion des provisions pour l’année n+ 1 ; tout comme sont exclues les sommes dues à d’autres titres que les charges approuvées et les provisions votées. Pour autant cette procédure simplifie considérablement le recouvrement de l’essentiel des sommes dont un copropriétaire peut être débiteur.
LES EFFETS DU JUGEMENT EN PROCEDURE ACCELERE AU FOND
Nous avons vu que même si la procédure s’apparente à une procédure de référé, il n’en demeure pas moins que la décision rendue est un jugement au fond et les mots « ordonnance » et « référé » disparaissent totalement de cette nouvelle désignation de « procédure accélérée au fond »588. Ceci quand bien même le texte issu de l’Ordonnance du 17 juillet 2019 fait encore référence à « l’ordonnance » (du juge) dans son 4ème alinéa589.
Par application des articles 514-1 à 514-6 du code de procédure civile, la décision du Président est exécutoire de droit à titre provisoire.
A défaut d’appel dans le délai de 15 jours590 de sa notification à partie, le jugement devient définitif. Par contre le débiteur pourra en faisant appel demander également la suspension de l’exécution provisoire.
Muni de cette décision, le syndic pourra alors procéder à l'exécution proprement dite, c’est-à-dire pratiquer une saisie attribution,
- par exemple, entre les mains du banquier du copropriétaire,
- ou encore entre les mains de son locataire si le lot est donné à bail.
Enfin, le 4ème alinéa de l’article 19-2 précise que cette mesure d'exécution se poursuivra jusqu'à l'extinction de la créance du syndicat telle que fixée par « « l'ordonnance ».
588 Le Décret \no 2019-1419 du 20 décembre 2019 crée un article 481-1 dans le Code de Procédure Civile qui détaille la procédure accélérée au fond.
589 On peut espérer que le texte de l’article 19-2 soit modifié dans le cadre de la ratification de l’Ordonnance ou de la codification.
590 Rappelons cependant que la décision est dite » en dernier ressort » pour les condamnations inférieures à 5.000 €, ce qui interdit d’en faire appel ; seul le recours en cassation étant alors possible.
droit de la copropriété année 2019-2020
482
SECTION II - LE RECOUVREMENT ORDINAIRE DES PROVISIONS ET CHARGES DUES.
La procédure spéciale de l’article 19-2 rend sans grand intérêt les procédures de droit commun dont disposent tous les créanciers pour obtenir paiement des sommes qui leur sont dues.
Mais bien évidemment l’article 19-2 s’ajoute aux procédures de droit commun toujours possibles.
Rappelons tout d’abord que les mesures conservatoires qui ont pu être prises ne confèrent pas de titre au syndicat des copropriétaires : pour obtenir paiement du copropriétaire défaillant, il pourra assigner, soit en le forme provisionnelle, soit devant le juge du fond.
A. LE TRIBUNAL COMPETENT
Le Tribunal localement compétent est celui du lieu de situation de l'immeuble (article 62 du décret de 1967).
Depuis le 1er janvier 2020, et en application de la loi du 23 mars 2019, il n’existe plus de compétence à raison du taux de la demande puisque quel que soit le montant réclamé, la procédure devra être introduite devant le Tribunal Judiciaire.
Les Tribunaux d’Instance ont disparu. A Paris, où existaient à la fois un TGI et 20 Tribunaux d’Instance, ces derniers ne seront plus que des dépendances du Tribunal Judiciaire. Dans les communes où existaient de tribunaux d’instance mais pas de Tribunal de grande instance, ces tribunaux deviennent de « tribunaux de proximité » avec des compétences propres (Décrets des 30 août et 18 septembre 2019).
Désormais le juge statue en dernier ressort si la demande est inférieure à 5.000 € et à charge d’appel au-dessus de cette somme.
Rappelons la nécessité d’une tentative de conciliation préalable : même en référé provision puisque d’une part le référé-provision peut être intenté en l’absence d’urgence et que d’autre part l’article 750-1 nouveau du CPC eige pour se dispenser de la tentative de conciliation que l’urgence soit manifeste !
B. LA DEMANDE DE CONDAMNATION PROVISIONNELLE
Bien évidemment les dispositions du code de procédure civil sur le « référé provision » reçoivent application.
droit de la copropriété année 2019-2020
483
L'obligation n'est pas sérieusement contestable lorsque le syndicat justifie que les sommes demandées le sont :
en exécution de décision d'assemblées générales définitives, c'est à dire non contestées dans le délai de deux mois par le copropriétaire débiteur.
ou en application de l'article 35 du décret qui précise les sommes dont le syndic peut exiger le versement : avance de trésorerie permanente, provisions votées par l'assemblée générale, remboursement des dépenses régulièrement engagées ou provisions trimestrielles, provisions spéciales pour l'exécution de travaux votés par l'assemblée générale.
C. LA PROCEDURE AU FOND.
Elle peut être intentée conformément aux dispositions du Code de Procédure Civile, soit sous forme d’une procédure d’injonction de payer, soit sous forme de citation ou d’assignation
1. Injonction de payer591.
La procédure d’injonction de payer prévue aux articles 1405 à 1424 du code de procédure civile, permet d’obtenir un titre exécutoire contre un débiteur lorsque sa dette est d’origine contractuelle ou résulte d’une obligation de caractère statutaire (article 1405 CPC).
Il s’agit d’une procédure maintes fois remaniée, qui relève depuis le 1er janvier 2020 de la compétence du tribunal judiciaire ou de ses chambres de proximité selon le montant des sommes demandées (plus ou moins de 5.000 €).
A compter du 1er janvier 2021 il s’agira d’une procédure totalement dématérialisée qui relèvera d’une juridiction unique ayant son siège à Strasbourg592.
En 2020 le requérant (donc le syndicat des copropriétaires) dépose une requête au greffe du Tribunal judiciaire (l’assistance d’un avocat n’est pas obligatoire) en remplissant un formulaire CERFA (\no 12948*06, qui demande d’énoncer l’identité du créancier et du débiteur, précise l’objet de la demande et le décompte des sommes demandées. La requête doit être accompagnée des pièces593.
591 Il existe même une procédure d’injonction européenne ; règlement (CE) \no 1896 / 2006 du Parlement européen et du Conseil du 12 décembre 2006) instituant une procédure européenne d'injonction de payer. qui permet d’agir contre un débiteur habitant un pays de l’Union Européenne (art. 1424-1 et s du code de procédure civile). 592 Loi de programmation et de réforme de la justice du 23 mars 2019
593 Par dérogation aux dispositions de l’article 1406 CPC, elle doit être présentée au greffe du Tribunal du domicile du défendeur (application de l’article 61-1 du Décret du 17 mars 1967).
droit de la copropriété année 2019-2020
484
Le magistrat, s’il estime la demande fondée en totalité ou partiellement délivre injonction au débiteur pour le principal, les intérêts, les frais accessoires et les dépens. Cette injonction est signifiée au débiteur qui peut payer, faire opposition par lettre recommandée ou par démarche physique au greffe de la juridiction compétente (auquel cas l’affaire suit les errements d’une procédure ordinaire au fond avec constitution d’avocat devant le Tribunal Judiciaire et le jugement se substituera à l’ordonnance) ou ne pas répondre à l’injonction du juge.
Dans ce dernier cas et à l’expiration du délai d’un mois, le syndicat des copropriétaires dispose, à son tour, d'un mois pour demander au juge d'apposer la formule exécutoire sur l'ordonnance. Celle-ci possède alors valeur de jugement.
Il convient de noter que si l’injonction de payer notifiée par huissier n’est pas notifiée à la personne du débiteur en ce cas ce dernier pourra faire opposition, même après le jugement et jusqu’à l’expiration du délai d’un mois suivant la première mesure d’exécution (saisie exécution) ; moralité : il convient de s’assurer que l’huissier notifie « à personne » !
On notera que la procédure d’injonction de payer peut être introduite sans tentative de conciliation préalable
2. Procédure ordinaire.
Se fait nécessairement par voie d’assignation.
A compter du 1er septembre 2020 l’assignation sera délivrée avec prise de date de première audience.
Mais la procédure devra être précédée d’une tentative préalable de résolution amiable du litige si la demande est formée pour un montant inférieur ou égale à 5.000 €
H. LA PROCEDURE SIMPLIFIEE DE RECOUVREMENT DES PETITES CREANCES
L’Ordonnance \no 2016-131 du 10 février 2016 portant réforme du droit des contrats, du régime général et de la preuve des obligations, Titre II, ayant modifié le code civil a créé une procédure simplifiée de recouvrement des petites créances dont les principes sont posés par le nouvel article L 125-1 du code des procédures civiles d’exécution :
Le Titre II du livre I er est complété par un chapitre V ainsi rédigé :
« CHAPITRE V « La procédure simplifiée de recouvrement des petites créances
« Art. L. 125-1. – Une procédure simplifiée de recouvrement des petites créances peut être mise en oeuvre par un huissier de justice à la demande du créancier pour le paiement d’une créance ayant une cause contractuelle ou résultant d’une obligation de caractère statutaire et inférieure à un montant défini par décret en Conseil d’Etat.
« Cette procédure se déroule dans un délai d’un mois à compter de l’envoi par l’huissier d’une lettre recommandée avec demande d’avis de réception invitant le débiteur à participer à
droit de la copropriété année 2019-2020
485
cette procédure. L’accord du débiteur, constaté par l’huissier de justice, suspend la prescription.
« L’huissier de justice qui a reçu l’accord du créancier et du débiteur sur le montant et les modalités du paiement délivre, sans autre formalité, un titre exécutoire.
« Les frais de toute nature qu’occasionne la procédure sont à la charge exclusive du créancier.
« Un décret en Conseil d’Etat fixe les modalités d’application du présent article, notamment les règles de prévention des conflits d’intérêts lors de la délivrance par l’huissier de justice d’un titre exécutoire ».
Concrètement cette procédure ne semble avoir eu aucun succès en matière de recouvrement de charges !
I. LES STATISTIQUES La Direction des affaires civiles et du sceau (DACS) du MINISTERE DE LA JUSTICE a publié le 20 février 2019 une étude sur le contentieux de la COPROPRIETE dont les conclusions sont ainsi rédigées : Cette étude indique notamment qu’entre 2007 et 2017, le nombre de contentieux de la copropriété portés devant les juridictions du premier degré a augmenté de 24%, passant de 33 600 à 41 700. Devant les cours d’appel, la hausse a été moins importante (+10%). Les contentieux concernant les demandes de paiement des charges représentent plus de deux tiers des contentieux de la copropriété. Près d’une demande en paiement sur quatre fait l’objet d’un abandon de procédure devant les juridictions de première instance, le plus souvent par un acte impliquant un accord des parties.
SECTION III - LES MODALITES PRATIQUES DU RECOUVREMENT DES CHARGES
A. LE DOSSIER DE RECOUVREMENT DES CHARGES.
Quelle que soit la juridiction saisie, les chances de succès dépendent essentiellement de la qualité du dossier soumis au juge.
Un bon dossier comprend nécessairement, selon le rappel qu’en a fait la Cour d’Appel de Paris594
Une copie du Règlement de Copropriété comportant l'état de répartition des charges, en sorte que le juge puisse contrôler l'imputation des charges par catégories.
594 Paris 23\degres Ch B, 30 sep 2004, IRC 2006 \no 515 p 13
droit de la copropriété année 2019-2020
486
Les procès-verbaux d' Assemblées Générales ayant approuvé les comptes et voté les budgets prévisionnels, avec justification que le copropriétaire débiteur, s'il était absent ou défaillant, a reçu la lettre recommandée d'envoi de ces procès-verbaux.
Les relevés des appels de fond (« avis » depuis le décret du 27 mai 2004) depuis la dernière date à laquelle le copropriétaire débiteur était à jour de ses charges. Chaque appel impayé doit être la suite logique du précédent en sorte que l'on voit apparaître systématiquement la reprise du décompte précédent et le montant du nouvel appel impayé.
Un tableau récapitulatif des dettes du copropriétaire couvrant les appels de fonds impayés avec référence à chacun de ces appels de fonds.
Les mises en demeure successives qui feront partir les intérêts légaux de l'article 36 du Décret pour chaque fraction impayée.
Un extrait de matrice cadastrale identifiant le propriétaire débiteur (en réalité au 1er janvier de l’année).
Dans la mesure où l'ensemble de ces documents seront remis au juge, la condamnation devrait être évidente. Si par contre les comptes sont mal présentés ou peu clairs, le juge rejettera la demande de paiement faite que ce soit sur l’article 19-2, en injonction de payer, à titre provisionnel ou au fond. Eventuellement il renverra à une expertise comptable dont le coût est souvent égal ou supérieur aux sommes à récupérer !
Pour échapper à la condamnation au paiement des charges, le débiteur invoquera divers moyens qui ont fait l’objet de nombreuses décisions de jurisprudence.
B. LES MOYENS DE DEFENSE DU COPROPRIETAIRE
Pour échapper à la condamnation au paiement des charges, le débiteur invoquera divers moyens qui ont fait l’objet de nombreuses décisions de jurisprudence.
1. 1. Le défaut d’approbation des comptes
L’un de ces moyens traditionnellement soulevés est le défaut d’approbation ou le défaut d’approbation régulière des comptes.
Certes, pour l’exercice en cours, la demande de condamnation ne peut porter par définition sur des charges approuvées. Par contre qu’en est-il pour les exercices antérieurs ?
droit de la copropriété année 2019-2020
487
Il convient ici de bien faire la distinction dans la défense du débiteur :
En cas de défaut de décision d’approbation des comptes, le juge ne pourra effectivement que rejeter la demande de condamnation présentée par le syndicat des copropriétaires595,
2. La contestation de l’assemblée générale ayant approuvé les comptes
Souvent le débiteur justifie avoir contesté l’assemblée générale ayant approuvé les comptes :
LA CONVOCATION DE L’ASSEMBLEE GENERALE ETAIT IRREGULIERE
Le débiteur affirme – et justifie même – avoir assigné en annulation de l’assemblée générale au motif que celle-ci a été convoquée irrégulièrement 596
LE DEBITEUR JUSTIFIE AVOIR DEMANDE JUDICIAIREMENT L’ANNULATION DE L’APPROBATION DES COMPTES.
Dans ces différentes hypothèses le juge ne doit pas rejeter la demande du syndicat mais faire application de la règle qui veut qu’une décision d’assemblée générale s’impose aux copropriétaires tant qu’elle n’est pas annulée. C’est ce qu’a rappelé la cour de cassation dans un arrêt du 27 juin 2001597
3. Les erreurs d’imputation ou les insuffisances de justification.
Par contre et bien évidemment, l’approbation des comptes ne vaut pas approbation de leur répartition598, en sorte que le copropriétaire peut toujours invoquer l’erreur d’imputation faite par le syndic au regard de la répartition stipulée au Règlement de copropriété ou résultant d’une décision d’assemblée générale :
- Charges de chauffage appelées en charges communes générales599
595 Civ. 3\degres Ch, 12 juin 1991 : JCP G 1991, IV, 317
596 CA Paris, 16 mai 2002 :JurisData \no 2002-178458
597 civ 3\degres ch. 27 juin 2001 Loyers et Copropriété oct 2001 \no 238
598 Article 45-1 du Décret \no 67-223 du 17 mars 1967 : « L'approbation des comptes du syndicat par l'assemblée générale ne constitue pas une approbation du compte individuel de chacun des copropriétaires »
599 CA Paris, 23e ch., 24 nov. 1989 : JurisData \no 1989-026109
droit de la copropriété année 2019-2020
488
- Absence de ventilation des charges entre les différents lots du copropriétaire
- Les charges demandées ont été payées600
C. L'EXECUTION DES CONDAMNATIONS.
1. La condamnation à astreinte.
Le Juge de l'Exécution peut assortir d'une astreinte une décision rendue par un autre juge si les circonstances en font apparaître la nécessité (article 33 de la loi du 9 juillet 1991).
Cette disposition peut être très efficace lorsque le débiteur se montre particulièrement peu empressé à exécuter une décision de condamnation.
2. Les saisies mobilières exécution.
Le syndicat pourra avec un titre pratiquer une saisie vente ou une saisie-attribution.
S'il avait obtenu préalablement une saisie conservatoire, le syndicat fera transformer la saisie conservatoire de créance en saisie attribution et la saisie conservatoire sur meubles en saisie-vente.
3. La saisie immobilière.
Qu'il ait préalablement inscrit l'hypothèque de l'article 19 de la loi ou non, qu'il ait obtenu ou non l'autorisation de prendre une sûreté, le syndicat pourra faire mettre en vente le lot du copropriétaire défaillant ou tout autre immeuble appartenant à ce copropriétaire, dès lors qu'il aura obtenu un titre définitif à son encontre.
Rappelons que désormais la procédure de l’article 19-2 permet d’obtenir un titre définitif601.
Cependant préalablement à la mise en vente d'un lot, le syndicat devra habiliter son syndic à pratiquer la saisie et la vente de ce lot, dans les termes de l'article 55 du Décret.
600 CA Paris, 23e ch., 3 févr. 1999 : Gaz. Pal. 17/18 sept. 1999, somm. p. 33
601 A compter du 1er juin 2020
droit de la copropriété année 2019-2020
489
Il convient tout d’abord de rappeler que la résolution – trop souvent - proposée aux copropriétaires de mandater a priori le syndic à l’effet de faire vendre le lot de tout copropriétaire débiteur n’a aucune efficacité juridique. L’habilitation ne peut être donnée qu’à l’encontre d’un débiteur dénommé, une fois obtenu un titre définitif à son encontre602
La raison de l’habilitation préalable du syndic est simple : en cas de vente judiciaire d'un immeuble (lot de copropriété ou autre), le créancier poursuivant est déclaré adjudicataire à défaut d'enchérisseur. Il y a donc un risque pour la Copropriété de faire -involontairement- l'acquisition du lot du débiteur. D'où le caractère solennel de la démarche qui impose une délibération préalable de l'Assemblée.
L’article 11-I du Décret du 17 mars 1967, relatif à la convocation de l’assemblée générale, a d’ailleurs été complété par le Décret du 20 avril 2010 qui ajoute un paragraphe 11 ainsi rédigé :
«Sont notifiés au plus tard en même temps que l'ordre du jour :
I. - Pour la validité de la décision (…) :
« 11\degres Les projets de résolution mentionnant, d’une part, la saisie immobilière d’un lot, d’autre part, le montant de la mise à prix, ainsi que le montant des sommes estimées définitivement perdues, lorsque l’assemblée générale est appelée à autoriser le syndic à poursuivre la saisie immobilière d’un lot ; »
Ces dispositions impliquent que deux résolutions, sinon trois (compte tenu de la rédaction de l’article 11 du décret), soient prises en assemblée générale préalablement à la mise en oeuvre de cette procédure :
- Une première résolution par laquelle l’assemblée générale, après rappel des diligences effectuées par le syndic pour obtenir paiement, définit « les sommes estimées définitivement perdues » conformément au Décret comptable du 14 mars 2005 (c'est-à-dire la totalité des sommes dues par le copropriétaire, que le syndicat des copropriétaires ait ou non obtenu un titre pour la totalité de ces sommes) et décide d’affecter ces sommes au compte 491 « Dépréciation des comptes copropriétaires » en sorte que cette somme fera l’objet d’une charge répartie entre tous les copropriétaires (en charges communes générales) au titre de l’exercice en cours à la date de la présente décision.
- Une deuxième résolution par laquelle l’assemblée générale, compte tenu des vaines poursuites engagées pour avoir paiement des sommes dues, décide de mettre en vente forcée le ou les lot(s) du copropriétaire débiteur.
- Une troisième résolution fixant la mise à prix de la vente du lot, en rappelant de préférence que le syndicat des copropriétaires a été préalablement informé par le syndic qu’en l’absence
602 Civ 3\degres ch 15 février 2006, AJDI 2006, 835, note Capoulade
droit de la copropriété année 2019-2020
490
d’enchère sur ce prix il sera déclaré adjudicataire en sa qualité de créancier poursuivant. Le prix pourra être fixé à un montant supérieur à la créance pour tenir compte des frais et des échéances impayées prévisibles jusqu’à la vente603.
Certes, la notion de « sommes estimées définitivement perdues » paraît inadaptée dans la mesure où la procédure de saisie aura justement pour objet de permettre le recouvrement de ces sommes, mais comme le précise M Laporte dans son ouvrage sur La nouvelle comptabilité de la copropriété604), si la vente permet de recouvrer tout ou partie des sommes dues il y aura lieu alors à « Reprise de dépréciations sur créances douteuses ».
Rappelons qu’en application de l’article 19-2 dernier alinéa de la loi du 10 juillet 1965
Si l'assemblée générale vote pour autoriser le syndic à agir en justice pour obtenir la saisie en vue de la vente d'un lot d'un copropriétaire débiteur vis-à-vis du syndicat, la voix de ce copropriétaire n'est pas prise en compte dans le décompte de la majorité et ce copropriétaire ne peut recevoir mandat pour représenter un autre copropriétaire en application de l'article 22
Dans les deux mois de la publication du commandement de payer valant saisie, le débiteur est assigné par le créancier à une audience d’orientation. Jusqu’à cette audience et le jour même de l’audience, le débiteur peut solliciter l’autorisation de vente amiable, contester la mise à prix, demander le report de l’adjudication pour causes graves et justifiées.
C’est à cette audience d’orientation que le juge de l’exécution autorisera la vente amiable ou la vente forcée. Lorsqu'il autorise la vente amiable, le juge s'assure qu'elle peut être conclue dans des conditions satisfaisantes compte tenu de la situation du bien, des conditions économiques du marché et des diligences éventuelles du débiteur (décret du 27.7.06 : art.49).
Le juge décide souverainement et se fonde sur les éléments de preuve apportés par le débiteur : attestations ou estimations de professionnels de l’immobilier (circulaire du 14.11.06 § 3.4.1.1).
Dans un arrêt du 10 septembre 2009, la Cour de cassation retient que le juge a pu déduire des éléments qui lui étaient soumis que le débiteur ne faisait état d’aucune perspective de vente de son bien et a donc valablement décidé de refuser la vente amiable.
603 Pierre REGNAULT et Denis TALON avocat spécialiste en saisie immobilière, dans un article publié à la Gazette du Palais 1998, 1081, proposaient de fixer la mise à prix entre le tiers et la moitié de la valeur prévisible du bien.
604 Editions Delmas, Deuxième Edition, 2011, \no 1925