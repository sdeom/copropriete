\chapter{La comptabilité du syndicat des copropriétaires}

Aux termes de l'article 18 de la loi du 10 juillet 1965 (disposition d'ordre public) le syndic de la copropriété
- tient la comptabilité du syndicat des copropriétaires,
- établit le budget prévisionnel,
- établit les comptes du syndicat et leurs annexes.
La comptabilité du syndicat des copropriétaires fait l'objet des articles 14-1 à 14-3 de la loi du 10 juillet 1965 issus de la loi S.R.U. du 13 décembre 2000 et des deux décrets d'application des 27 mai 2004 et 14 mars 2005.
Les articles de loi sont ainsi rédigés :
Art. 14-1 –Pour faire face aux dépenses courantes de maintenance, de fonctionnement et d'administration des parties communes et équipements communs de l'immeuble, le syndicat des copropriétaires vote, chaque année, un budget prévisionnel. L'assemblée générale des copropriétaires appelée à voter le budget prévisionnel est réunie dans un délai de 6 mois à compter du dernier jour de l'exercice comptable précédent.
Les copropriétaires versent au syndicat des provisions égales au quart du budget voté. Toutefois, l'assemblée générale peut fixer des modalités différentes.
La provision est exigible le premier jour de chaque trimestre ou le premier jour de la période fixée par l'assemblée générale.
Art. 14-2. Ne sont pas comprises dans le budget prévisionnel des dépenses pour travaux dont la liste sera fixée par décret en conseil d'état.
Les sommes afférentes à ses dépenses sont exigibles selon les modalités votées par l'assemblée générale.
Art. 14-3. Les comptes du syndicat comprenant le budget prévisionnel, les charges et produits de l’exercice, la situation de trésorerie, ainsi que les annexes au budget prévisionnel sont établis conformément à des règles comptables applicables spécifiques fixées par décret. Les comptes sont présentés avec comparatif des comptes de l’exercice précédent approuvé.
« Les charges et les produits du syndicat, prévus au plan comptable, sont enregistrés dès leur engagement juridique par le syndic indépendamment de leur règlement ou dès réception par lui des produits. L’engagement est soldé par le règlement. »
droit de la copropriété année 2019-2020
383
Ces nouveaux articles introduisent des règles de gestion comptable très strictes dans le statut de la copropriété qui en été jusqu’alors dépourvu (I). Ces règludgetes concernent l’approbation du budget (II) et des comptes (III)
SECTION I - HISTORIQUE DES REGLES COMPTABLES
A. LA LOI DU 10 JUILLET 1965
1. 1. L'absence de normes comptables dans la loi du 10 juillet 1965.
La comptabilité du syndicat des copropriétaires ne répondait à aucune obligation normative en sorte que si chaque syndic devait tenir la comptabilité des syndicats qu’il administrait, il le faisait à sa guise, selon ses propres recettes.
Les seules dispositions impératives dans le décret du 17 mars 1967 concernaient la reddition des comptes, puisqu'au terme de l'article 11 le syndic devait adresser aux copropriétaires avec la convocation à l'assemblée générale devant approuver les comptes et le budget :
- le compte des recettes et dépenses de l'exercice écoulé,
- l'état des dettes et créances,
- la situation de trésorerie,
- le budget prévisionnel.
La sixième recommandation de la commission relative à la copropriété avait tenté une certaine unification de présentation des comptes :
- recommandant au syndic l'utilisation d'appellations de catégories identiques pour tous les immeubles (charges communes générales, charges communes un groupe d'immeubles, charges de bâtiment, etc...),
- recommandant ensuite au syndic de faire une présentation identique des comptes d'un exercice sur l'autre et même de joindre à ses comptes une notice explicative,
- proposant des tableaux de présentation de l'état des dettes et des créances et de la situation de trésorerie.
droit de la copropriété année 2019-2020
384
2. 2. Comptabilité de trésorerie et comptabilité d'engagement.
Il existe essentiellement deux types de comptabilité :
- la comptabilité de trésorerie, selon laquelle les charges sont enregistrées lorsqu’elles sont décaissées et les recettes sont enregistrées lorsqu’elles sont encaissées. C'est le type même de la comptabilité familiale dite encore « comptabilité de ménagère ».
- La comptabilité d'engagement, selon laquelle les charges sont enregistrées le jour où elles font l'objet d'une obligation vis-à-vis du créancier, tandis que les produits sont enregistrés à la date d'acquisition de la créance. C'est le type même de la comptabilité commerciale.
Ni la loi du 10 juillet 1965 ni le décret du 17 mars 1967 n'imposaient aux syndics pour la tenue des comptes des copropriétés qu'ils administraient d'adopter l'un ou l'autre de ces deux types de comptabilité494. Étant simplement observés que les copropriétaires ne sont pas forcément commerçants ou polytechniciens et qu’il leur paraît plus facile de contrôler une comptabilité de ménagère qu’une comptabilité de type commercial.
La rédaction des articles de la loi et du décret faisaient plutôt référence à des notions tirées de la comptabilité de trésorerie plutôt que de la comptabilité d'engagement : ne serait-ce que l'obligation faite au syndic dans l'ancien article 11 du décret de joindre à la convocation le compte des « recettes » et le compte des « dépenses ». En sorte que le plus souvent, c'est effectivement une comptabilité de trésorerie qui était tenue par le syndic de la copropriété.
3. 3. Système de provision sur charges et système des charges échues.
A l’origine, les syndics de copropriété pouvaient opter entre deux systèmes opposés de comptabilité :
(1) Gestion en charges échues.
Selon ce système, le syndic paie au fur et à mesure les dépenses de la Copropriété. Trimestriellement, il demande aux copropriétaires de rembourser leurs quotes-parts de ces charges.
494 Étant observé que le syndic exerçant généralement sous forme commerciale doit tenir la comptabilité de son cabinet en la forme d'une comptabilité d'engagement.
droit de la copropriété année 2019-2020
385
Les copropriétaires ont un contrôle trimestriel des dépenses du syndicat. Les sommes appelées par le syndic sont les sommes dont les copropriétaires sont effectivement débiteurs après répartition dans les termes du Règlement de Copropriété.
En revanche, les charges trimestrielles varient en fonction des dépenses réalisées, et le syndic peut être amené à appeler à titre de charges des dépenses effectuées qui n'ont pas été préalablement décidées par l'Assemblée Générale.
(2) Gestion par provisions.
Dans ce type de système, le syndic appelle des provisions trimestrielles en quatre fractions d'un même montant, destinées à couvrir les dépenses d'un exercice annuel.
A l'issue de l'exercice le syndic compare le montant des provisions reçues et le montant des dépenses réellement réalisées. Cette comparaison permettra de constater si les provisions ont été supérieures aux dépenses ou si inversement les dépenses ont été supérieures aux provisions reçues.
En conséquence le syndic, après approbation des comptes par l'Assemblée, adressera aux copropriétaires un récapitulatif annuel aux termes duquel il sollicitera le complément nécessaire pour ajuster les dépenses et les appels de fonds... ou il créditera les copropriétaires des sommes appelées au-delà des dépenses réalisées.
Ce système présente l'avantage d'être parfaitement compris des copropriétaires; de plus il est sans surprise puisque dès le début de l'exercice chaque copropriétaire connaîtra le montant trimestriel des "provisions sur charges" qu'il devra adresser au syndic.
Il permet aux copropriétaires de recevoir un récapitulatif annuel des dépenses dont le contrôle sera facile.
Mais il présente aussi certains inconvénients :
- Le syndic n'a pas l'obligation d'adresser un récapitulatif des dépenses trimestrielles; en sorte que pendant toute une année le copropriétaire peut rester dans l'ignorance des dépenses réelles de la copropriété. Mais, en pratique les syndics qui pratiquaient le système des appels trimestriels envoyaient avec ces appels un récapitulatif des dépenses du trimestre.
- Les appels provisionnels sont appelés en tantièmes généraux et non au prorata de la participation de chacun dans les différentes dépenses de la Copropriété. En sorte qu'en fin d'exercice, certains copropriétaires auront trop payé et d'autres pas assez :
A titre d'exemple on évoquera la question des charges de chauffage : certains copropriétaires ne sont pas raccordés au chauffage central dont on sait qu'en copropriété il peut représenter 40 à 45 % de l'ensemble des charges. Dans le système des "provisions", le copropriétaire non raccordé se
droit de la copropriété année 2019-2020
386
verra appeler des charges tenant compte des dépenses de chauffage, en sorte qu'au moment de l'éclatement des dépenses, il apparaîtra comme ayant trop payé.
Mais il est vrai que rien n'interdit dans le système des appels provisionnels de répartir ces appels à proportion de la participation de chacun dans les différentes catégories de charges.
- Enfin, ce système ne permet pas de savoir instantanément les sommes dont un copropriétaire est débiteur envers le syndicat. Cette connaissance était cependant nécessaire lorsqu'en cas de mutation à titre onéreux le notaire interrogeait le syndic sur les sommes dues à la Copropriété par le copropriétaire vendeur.
B. LA LOI SRU DU 13 DECEMBRE 2000 ET SES TEXTES D'APPLICATION.
Dès 1985 l’idée avait été émise de créer une comptabilité spécifique pour les syndicats de copropriété (essentiellement en imposant un plan comptable aux syndicats). Mais la mise en oeuvre de cette idée parut trop compliquée pour être retenue à cette époque.
1. La loi du 13 décembre 2000.
La loi SRU va créer un véritable statut comptable du syndicat des copropriétaires en introduisant trois nouveaux articles dans la loi du 10 juillet 1965, ce sont les articles 14 – 1, 14 – 2 et 14 – 3.
Ces articles vont poser les principes suivants :
o une distinction doit être faite sans équivoque entre les dépenses sur budget (dépenses habituelles de fonctionnement et de maintenance courante de la copropriété) et les dépenses hors budget (correspondant essentiellement aux travaux et contrats exceptionnels) ;
o les provisions sur budget doivent être nécessairement appelées auprès des copropriétaires (normalement par trimestre) en fonction de la quote-part due par ces copropriétaires dans le budget voté ; la gestion en charges échues étant désormais interdite (article 14-1);
o les dépenses hors budget doivent faire l'objet d'un vote spécifique précisant la date d'exigibilité des provisions nécessaires à leur règlement (article 14-2);
o la comptabilité du syndicat des copropriétaires est une comptabilité d'engagement ;
o la comptabilité du syndicat est établie conformément à des règles comptables spécifiques qui seront fixées par décret (article 14-3).
Aux termes de la loi SRU (article 75 – III) l'article 14 – 3 relatif aux règles de comptabilité spécifique devaient entrer en vigueur le 1er janvier 2004.
droit de la copropriété année 2019-2020
387
Cette date d'entrée en vigueur a été retardée à plusieurs reprises :
o la loi du 2 juillet 2003 (Urbanisme et Habitat) a reporté l'application de cette disposition au 1er janvier 2005
o la loi du 18 janvier 2005 (Programmation pour la Cohésion Sociale) a reporté à son tour l'application de cette disposition au 1er janvier 2006
o la loi ENL du 13 juillet 2006, modifiant l’article 75 de la loi du 13 décembre 2000, a finalement imposé que les comptes du syndicat des copropriétaires soient tenus conformément aux normes édictées par le décret comptable « à partir du 1er exercice comptable commençant à compter du 1er janvier 2007 ».
En fait, ce retard était dû notamment495 à la discussion qui s'est instaurée sur l'utilité d'une comptabilité d'engagement pour les petites copropriétés.
Tant les syndicats professionnels de syndics (FNAIM, CNAB) que les associations de consommateurs (UNARC) ont souhaité en effet que les petites copropriétés soient dispensées d'une comptabilité d'engagement tenue en partie double. Les associations de consommateurs faisant valoir notamment que cette exigence était de nature à faire disparaître le syndic bénévole, tandis que les syndicats professionnels objectaient que le système mis en oeuvre constituait une véritable usine à gaz, de nature à augmenter les frais de gestion des copropriétés, sans véritable justification pour les copropriétaires.
Cette polémique devait être notamment entretenue par l'avis du Conseil National de la Comptabilité du 22 octobre 2002 qui invitait le gouvernement à définir une comptabilité simplifiée pour les petites copropriétés.
Si les services du ministre du logement devaient se montrer favorables à cette dualité de comptabilité distinguant entre les grandes et petites copropriétés, par contre la Chancellerie conservait une position hostile à toute dualité, estimant avec le président CAPOULADE qu’il n'existe qu'une seule copropriété pour toute la France496.
Finalement, à l’occasion du vote de la loi ENL du 13 juillet 2006, a été prévue une dispense de comptabilité en partie double des copropriétés comportant moins de dix lots à usage de logements, de bureaux ou de commerces et dont le budget annuel moyen sur trois ans est inférieur à 15.000 €.(art 92 de la loi ENL)
Quant à la dernière prorogation du délai d'entrée en vigueur des dispositions de l’article 14 – 3 de la loi du 10 juillet 1965 postérieurement 1er janvier 2006 , elle s'explique essentiellement par la tardivité de
495 La loi SRU devait être complétée sur ce point par deux décrets : le Décret modifiant le Décret du 17 mars 1967 pour adapter celui-ci aux dispositions de la loi SRU (décret devant être rédigé par la Chancellerie et approuvé par le Conseil d’Etat) qui ne sera finalement publié que le 27 mai 2004 et le Décret « comptable » (rédigé par le Ministère du Logement et devant être publié après avis du Conseil National de la Comptabilité).
496 La difficulté était surtout de déterminer le seuil à partir duquel une copropriété cessait d'être une petite copropriété pour devenir une grande copropriété.
droit de la copropriété année 2019-2020
388
publication du décret du 14 mars 2005 empêchant de mettre en place utilement la formation des professionnels et d'assurer la fiabilité des logiciels de gestion.
2. Le Décret du 27 mai 2004.
Le décret traite en premier lieu du budget prévisionnel en ce qui concerne son établissement et son contenu.
Il précise ensuite les dispositions relatives à l'ouverture du compte séparé.
Il donne par ailleurs des définitions précises de ce que l'on doit entendre s'agissant des provisions sur charges, des avances et des charges et redéfinit les sommes dont le syndic peut exiger le paiement auprès des copropriétaires.
Enfin, il redéfinit les documents qui doivent être annexés à la convocation de l'assemblée générale lorsque celle-ci est appelée à approuver les comptes et à voter le budget.
3. Le décret et l'arrêté du 14 mars 2005.
Il s'agit cette fois du décret comptable qui comporte 13 articles et 5 annexes correspondant ou documents de synthèse. Ce décret :
\begin{itemize}
	\item définit le champ d'application des règles comptables
	\item donne la définition des charges et des produits
	\item fixe la durée de l'exercice comptable du syndicat
	\item définit les pièces justificatives des enregistrements comptables
	\item présente l’organisation comptable des comptes de la copropriété
\end{itemize}
quant aux cinq annexes, elles sont ainsi définies :
\begin{itemize}
	\item annexe 1 : état financier qui comprend la situation financière et l'état des dettes et des créances.
	\item annexe 2 : compte de gestion générale et budget prévisionnel.
	\item annexe 3 : compte de gestion des opérations courantes du budget prévisionnel.
	\item annexe 4 : compte de gestion pour travaux et opérations exceptionnelles
	\item annexe 5 : état des travaux et dépenses exceptionnels votés et non clôturés en fin d'exercice.
\end{itemize}
droit de la copropriété année 2019-2020
389
Rappelons que pour la validité de l’approbation des comptes par l’assemblée générale, ces cinq annexes doivent être jointes à la convocation.
Mais leur compréhension par les copropriétaires est tellement problématique que le Décret \no 2010-391 du 20 avril 2010 exige que soit adressé de plus :
« 5 \no En vue de l’approbation des comptes par l’assemblée générale, le projet d’état individuel de répartition des comptes de chaque copropriétaire. »
SECTION II - LE BUDGET ET LES DEPENSES HORS BUDGET
A. LES DEPENSES SUR BUDGET (DEPENSES COURANTES).
1. CE QUE COMPRENNENT CES DEPENSES.
Les dépenses courantes, sont définies par la loi comme relatives à la maintenance, au fonctionnement et à l'administration des parties communes et des éléments d'équipement commun.
Comme par le passé, ces dépenses devront faire l’objet d’un budget adopté par l’assemblée Générale.
2. EPOQUE D’ADOPTION DU BUDGET.
CE QUE DIT LA LOI.
Pour que le dispositif nouvellement mis en place par le législateur soit efficace, il était nécessaire que le budget de l'année soit voté en début d'exercice et non pas dans le courant du troisième ou du quatrième trimestre de l'année.
L'idée était à l'origine que l'assemblée générale votant le budget devrait se réunir dans le courant du premier trimestre de l'exercice. Les syndics ont opposé la difficulté matérielle qui résulterait d'une telle exigence : soit toutes les assemblées générales annuelles se tenaient dans les trois premiers mois, soit il fallait prévoir au moins deux assemblées annuelles, l'une pour l'approbation du budget et l'autre pour l'adoption des questions courantes ou exceptionnelles de la copropriété.
Finalement, le parlement a donné un délai de six mois au syndic pour convoquer l'assemblée générale qui aura à adopter le budget :
L ’article 14-1 édicte en conséquence que : « L'assemblée générale des copropriétaires appelée à voter le budget prévisionnel est réunie dans un délai de six mois à compter du dernier jour de l'exercice comptable précédent ».
droit de la copropriété année 2019-2020
390
CE QUE DIT LE DECRET :
En réalité il apparaît inopportun de ne voter le budget qu’en cours d’exercice. C’est pourquoi, le décret, modifiant quelque peu la loi sur ce point pose les règles suivantes :
Article 43 (Modifié par Décret 2004-479 2004-05-27 art. 31, art. 32 JORF 4 juin 2004.)
« Le budget prévisionnel couvre un exercice comptable de douze mois. Il est voté avant le début de l'exercice qu'il concerne.
Toutefois, si le budget prévisionnel ne peut être voté qu'au cours de l'exercice comptable qu'il concerne, le syndic, préalablement autorisé par l'assemblée générale des copropriétaires, peut appeler successivement deux provisions trimestrielles, chacune égale au quart du budget prévisionnel précédemment voté. La procédure prévue à l'article 19-2 de la loi du 10 juillet 1965 ne s'applique pas à cette situation. »
« Le budget prévisionnel couvre un exercice comptable de douze mois », en sorte que bien évidemment les copropriétés peuvent prévoir un exercice comptable qui ne coïncide pas avec l’année civile : l’année peut commencer le 1er octobre par exemple.
En conséquence la pratique normale est de voter pendant l’année » n » le budget de l’année « n+1 », en sorte que le budget ne sera voté pendant l’année qu’il concerne que si l'assemblée générale n’a pu voter le budget avant le début de l’exercice.
A titre exceptionnel, si l'assemblée générale n’a pu adopter le budget avant le début de l’exercice qu’il concerne, le syndic pourra appeler jusqu’à deux fois ¼ du budget de l’année précédente… à condition toutefois que cette possibilité ait été donné au syndic par une assemblée générale préalable.
Comme on ne connaît jamais l’avenir, il serait bon que la résolution d’adoption du budget précise que « si l’assemblée générale ne peut être convoquée pour l’approbation de l’année n+2, le syndic pourra appeler successivement deux provisions trimestrielles, chacune égale au quart du budget prévisionnel présentement voté ».
Enfin, on peut penser qu’en cas de dérapage constaté au cours de l’année n, le syndic pourra inscrire à l’assemblée générale de l’exercice n+1 le vote d’un complément de budget. Mais il est vrai que le plan comptable ne comporte aucune ligne correspondant à un tel ajustement. D’où la question que posent certains experts comptables de la licéité d’un tel ajustement budgétaire.
droit de la copropriété année 2019-2020
391
3. EXIGIBILITE PAR QUART DU BUDGET.
article 14-1 al. 2 nouveau de la loi du 10 juillet 1965 :
« Les copropriétaires versent au syndicat des provisions égales au quart du budget voté. Toutefois, l'assemblée générale peut fixer des modalités différentes. »
Les dépenses courantes, telles que précédemment définies comme relatives à la maintenance, au fonctionnement et à l'administration des parties communes et des éléments d'équipement commun, font l'objet depuis le 1er janvier 2006 d'une comptabilité par provision.
De plus, le syndic ne peut plus demander aux copropriétaires le remboursement des dépenses acquittées ; il doit impérativement appeler les charges sur la base du budget prévisionnel.
Normalement, il y a quatre appels provisionnels par année. Mais la loi permet à l'assemblée générale de prévoir une périodicité différente. Cette décision est adoptée à la majorité de l'article 24 de la loi, puisqu’il n'est pas prévu de majorité particulière sur ce point.
On peut envisager que dans une petite copropriété l'assemblée générale prévoit deux appels provisionnels pour l'année.
Par contre, dans une copropriété importante, il peut être de l'intérêt de la personne morale du syndicat des copropriétaires de fixer des appels provisionnels mensuels. Auquel cas, le travail du syndic sera considérablement alourdi, sauf à prévoir en accord avec les copropriétaires un prélèvement mensuel automatique sur leur compte.
Article 14-1 alinéa 3 de la loi du 10 juillet 1965
« La provision est exigible le premier jour de chaque trimestre ou le premier jour de la période fixée par l'assemblée générale ».
Art 35-2 du Décret \no67-223 du 17 mars 1967 (Ajouté par le Décret \no 2004-479 du 27 mai 2004 – article 24)
Pour l'exécution du budget prévisionnel, le syndic adresse à chaque copropriétaire, par lettre simple, préalablement à la date d’exigibilité déterminée par la loi, un avis indiquant le montant de la provision exigible.
droit de la copropriété année 2019-2020
392
Si la provision est exigible comme il est dit à l’article 14-3 de la loi, il est nécessaire que le syndic adresse préalablement un avis d’exigibilité.
4. SANCTIONS AU DEFAUT DE PAIEMENT DE LA QUOTE-PART SUR BUDGET.
Cette sanction est édictée par l’article 19-2 de la loi qui prévoit une véritable déchéance de terme en sorte que le copropriétaire qui ne paie pas un appel trimestriel peut être condamné par le Président du TGI statuant comme en matière de référé à payer sa quote-part totale du budget annuel497.
B. LES DEPENSES POUR TRAVAUX : DEPENSES « HORS BUDGET ».
Article 14-2 de la Loi du 10 juillet 1965
« Ne sont pas comprises dans le budget prévisionnel les dépenses pour travaux dont la liste sera fixée par décret en Conseil d'Etat. Les sommes afférentes à ces dépenses sont exigibles selon les modalités votées par l'assemblée générale. »
1. Ce que comprennent les dépenses hors budget
Le « décret en Conseil d’Etat » du 27 mai 2004 n’a pas donné de liste à proprement parler, mais fixé des principes :
Art. 44. Les dépenses non comprises dans le budget prévisionnel sont celles afférentes :
1\no Aux travaux de conservation ou d'entretien de l'immeuble, autres que ceux de maintenance.
2\no Aux travaux portant sur les éléments d'équipement communs, autres que ceux de maintenance
3\no Aux travaux d'amélioration, tels que la transformation d'un ou de plusieurs éléments d'équipement existants, l'adjonction d'éléments nouveaux, l'aménagement de locaux affectés à l'usage commun ou la création de tels locaux, l’affouillement du sol et la surélévation de bâtiments ;
4\no Aux études techniques, tels que les diagnostics et consultations ;
5\no Et, d'une manière générale, aux travaux qui ne concourent pas à la maintenance et à l'administration des parties communes ou à la maintenance et au fonctionnement des équipements communs de l'immeuble
Bref, cet article ne permet pas de faire la distinction entre les travaux de maintenance courante et les autres travaux.
497 Cette question est traitée plus loin au titre des procédures de recouvrement.
droit de la copropriété année 2019-2020
393
2. L’exigibilité des appels de fonds pour travaux.
Bien évidemment, ces travaux sont « hors budget ». Par le passé nombreux étaient les syndics qui faisaient voter des travaux simplement en mentionnant leur coût provisionnel dans le budget. Cette Pratique était totalement contraire à la loi. L’’article 14-2 met les points sur les « i » !
L'assemblée générale votera spécialement chacun des travaux concernés et précisera en même temps les conditions de recouvrement des charges afférentes à ces travaux, c'est-à-dire qu’elle votera la date à laquelle les provisions sur travaux seront exigibles
L’article 35-2 du décret après avoir précisé dans le premier § que le syndic demandait le paiement de la quote-part du budget par lettre simple, ajoute dans un second paragraphe concernant les appels de fonds pour travaux « hors budget » :
« Pour les dépenses non comprises dans le budget prévisionnel, le syndic adresse à chaque copropriétaire, par lettre simple, préalablement à la date d'exigibilité déterminée par la décision d'assemblée générale, un avis indiquant le montant de la somme exigible et l'objet de la dépense ».
Comme par le passé, on admettra que « l’objet » de la dépense soit précisé d’un mot ou d’une simple expression « acompte sur ravalement » par exemple.
3. Le recouvrement des appels travaux se fait conformément au droit commun.
De plus, si le copropriétaire est bien débiteur des appels conformément à ce qui était décidé par l'assemblée générale pour le paiement des travaux, il ne peut pas cependant être condamné comme il est dit à l'article 19-1 de la loi.
Le syndic n'aura d'autre possibilité que de suivre la voie normale de la procédure à l'encontre du copropriétaire pour le paiement de sa quote-part des travaux votés en assemblée générale : il pourra demander la condamnation provisionnelle au juge des référés ou une condamnation au fond devant le tribunal.
SECTION III - LES COMPTES DU SYNDICAT.
Rappelons que ces règles sont fixées dans ler principe par l’article 14-3 de la loi du 10 juillet 1965 :
droit de la copropriété année 2019-2020
394
Art. 14-3 de la Loi du 10 juillet 1965 (Loi SRU du 13 décembre 2000, entre en vigueur le 1er janvier 2006)-
Les comptes du syndicat comprenant le budget prévisionnel, les charges et produits de l’exercice, la situation de trésorerie, ainsi que les annexes au budget prévisionnel sont établis conformément à des règles comptables applicables spécifiques fixées par décret. Les comptes sont présentés avec comparatif des comptes de l’exercice précédent approuvé ».
A. LES REGLES RELATIVES A L’ENREGISTREMENT DES ECRITURES
1. La comptabilité en partie double.
Article 1er de l’arrêté du 14 mars :
« les écritures sont passées selon le système dit « en partie double ». Dans ce système, tout mouvement ou variation enregistré dans la comptabilité est représenté par une écriture qui établit une équivalence entre ce qui est porté au débit et ce qui est porté au crédit des différents comptes affectés par cette écriture ».
2. La sincérité des écritures.
Aux termes de l’article 5 de l’arrêté du 14 mars 2005 :
« Les documents comptables sont tenus sans altération et sans blanc. Une écriture erronée est annulée par une écriture contraire.
« Une procédure de clôture destinée à figer la chronologie et à garantir l’intangibilité des enregistrements est mise en oeuvre à la date d’arrêté des comptes ».
En d’autres termes, une fois appuyé sur le bouton de validation de clôture de l’exercice il ne sera plus possible de modifier aucune écriture.
Les comptes 1 à 5 sont totalisés et font apparaître un solde qui sera le « report à nouveau » du début du nouvel exercice. Par contre les comptes 6 (charges par nature) et 7 (produits financiers) ne font pas l’objet d’un report à nouveau.
3. La comptabilité d’engagement.
droit de la copropriété année 2019-2020
395
L’article 14-3 de la loi précise que les charges et les produits sont enregistrés dès leur engagement juridique ; mais les articles 3 et 4 du décret définissent de façon plus libérale les notions d’engagement et de produit :
Article 3 du décret (dépenses sur budget) : Il faut comptabiliser les dépenses lorsque le syndicat a bénéficié de la fourniture ou du service
Article 4 du décret (dépenses hors budget) : il faut comptabiliser les dépenses au fur et à mesure de la réalisation des travaux ou de la fourniture des prestations
Article 3 du décret (produits): il faut enregistrer les sommes dues par les copropriétaires à la date d’exigibilité de ces sommes, que ce soit sur le budget ou sur les dépenses hors budget.
4. La comptabilité d’exercice.
L’article 5 al 1 du décret précise que la durée d’un exercice est normalement de 12 mois :
« L’exercice comptable du syndicat des copropriétaires couvre une période de douze mois. Les comptes sont arrêtés à la date de clôture de l’exercice. Pour le premier exercice, l’assemblée générale des copropriétaires fixe la date de clôture des comptes et la durée de cet exercice qui ne pourra excéder dix-huit mois ».
L’article 5 al 2 du décret autorise l'assemblée générale à modifier les dates de l’exercice annuel.
« La date de clôture de l’exercice pourra être modifiée sur décision motivée de l’assemblée générale des copropriétaires. Un délai minimum de cinq ans devra être respecté entre les deux décisions d’assemblées générales modifiant la date de clôture ».
Ce qui signifie que l'assemblée générale à la majorité de l’article 24 de la loi498 pourra par une décision motivée modifier non pas la durée de l’exercice (ce sera toujours un an) mais le point de départ de l’exercice : par exemple la résolution pourra être la suivante :
« L'assemblée générale pour faire coïncider l’exercice comptable avec le début de la campagne de chauffe décide que cet exercice comptable débutera le 1er octobre prochain pour s’achever au 30 septembre de l’année suivante ».
498 La loi étant muette sur ce point on peut se demander si le Règlement de copropriété ne peut pas prévoir une majorité différente. En tout cas dans le silence du Règlement de copropriété ce sera bien la majorité de l’article 24 qui s’appliquera.
droit de la copropriété année 2019-2020
396
De la même façon dans une résidence de vacances soumise au statut de la copropriété, on pourra motiver un exercice allant du 1er avril au 31 mars par la nécessité de tenir pendant la période estivale l'assemblée générale annuelle statuant sur les comptes et le budget.
Un nouveau changement des dates d’exercice ne pourra pas intervenir avant cinq années entières.
Bien évidemment chaque exercice est indépendant du précédent et du suivant, en sorte que si des dépenses sont échelonnées sur plusieurs exercices, elles devront apparaître dans un compte « solde en attente » reproduit à l’annexe V.
5. L’enregistrement de fait au fur et à mesure des engagements.
Cet enregistrement se fait dès l’engagement juridique par le syndic, indépendamment de leur règlement (article 2 al 2 du décret du 14 mars 2005). Sur ce fondement la Cour de Cassation a sanctionné un arrêt de Cour d'Appel ayant refusé l’annulation de la résolution d’approbation des comptes ne tenant pas compte d’une indemnité allouée au syndicat par le Tribunal et dont le règlement n’était pas intervenu au cours de cet exercice.499
6. L’enregistrement des écritures se fait TTC
L'article 3 de l'arrêté du 14 mars 2005 précise :
« les opérations sont enregistrées toutes taxes comprises dans les comptes dont l'intitulé correspond à leur nature. Le montant et le taux de taxe sont indiqués lorsque un ou plusieurs copropriétaires ont déclaré être assujettis à la TVA »
B. LE PLAN COMPTABLE DU SYNDICAT.
l'article 7 de l'arrêté du 14 mars 2005 donne la nomenclature des comptes.
1. Les cinq classes de comptes et leurs détails.
499 Civ. 3\no Ch. 5 février 2014, Pourvoi \no 12-19047, Au Bulletin.
droit de la copropriété année 2019-2020
397
Il y a cinq classes de comptes (les classes 1, et 3 à 7) :
CLASSE 1. Provisions, avances, subventions et emprunts
10 - Provisions et avances
102 - provisions pour travaux décidés 103 - avances 1031 - avances de trésorerie
1032 - avances travaux article 18 alinéa six
1033 - autres avances
12 - soldes en attente sur travaux et opérations exceptionnelles
13 - subventions
131 - subventions accordées en instance de versement
CLASSE 4. Compte des copropriétaires et des tiers
40 - fournisseurs 401 - factures parvenues
408 - factures non parvenues 409 - fournisseurs débiteurs
42 - personnel
421 - rémunérations dues
43 - sécurité sociale et autres organismes sociaux
431 - sécurité sociale
432 - autres organismes sociaux
44 - État et collectivités territoriales
441 - État et autres organismes -- subventions à recevoir 442 - État -impôts et versements assimilés
443 - collectivités territoriales -- aides
45 - collectivités des copropriétaires
450 - copropriétés individualisées 450-1 - copropriétaires -- budget prévisionnel
450-2 - copropriétaires -- travaux article 14 - 2
450-3 - copropriétaires -- avances
450-4 - copropriétaires -- emprunts
459 - copropriétaires -- créances douteuses
46 - débiteurs et créditeurs divers :
461 - débiteurs divers
droit de la copropriété année 2019-2020
398
462 - créditeurs divers
47 - compte d'attente :
471 - compte en attente d'imputation débiteur
472 - compte en attente d'imputation créditeur
48- comptes de régularisation :
486 - charges payées d'avance
487 - produits encaissés d'avance
49 - dépréciation des comptes de tiers :
491 - copropriétaires
492 - personnes autres que les copropriétaires
CLASSE 5 comptes financiers
50 - fonds placés
501 comptes à terme
502 - autre compte
51 -Banque, ou fonds disponibles en banque pour le syndicat
512 banques
514 - chèques postaux
53 - caisse
CLASSE 6 comptes de charges par nature
60 - achats de matières et fournitures :
601 - eau
602 - électricité
603 - chauffages, énergie et combustible
604 - achats produits d'entretien et petits équipements
605 - matériels
606 - fournitures
61 - services extérieurs :
611 - nettoyages des locaux
612 - locations immobilières
613 - locations mobilières
614 - contrats de maintenance
droit de la copropriété année 2019-2020
399
615 - entretien et petites réparations
616 - primes d'assurance
62 - frais d'administration et honoraires :
621 - rémunérations du syndic sur gestion copropriété
6211 - rémunérations du syndic
6212 - débours
6213 - frais postaux
622 - autres honoraires du syndic
6221 - honoraires travaux
6222 - prestations particulières
6223 - autres honoraires
623 - rémunérations de tiers intervenants
624 - frais du conseil syndical
63 - impôts -- taxes et versements assimilés :
632 - taxes de balayage
633 - taxes foncières
634 - autres impôts et taxes
64 - frais de personnel :
641 - salaires
642 - charges sociales et organismes sociaux
643 - taxes sur salaire
644 - autres (médecine du travail, mutuelle, etc.)
66 - charges financières des emprunts, agios ou autres :
661 - Remboursements d'annuité d'emprunts
662 - Autres charges financières et agios
67 - Charges pour travaux et opérations exceptionnelles :
671 - Travaux décidés par l'assemblée générale
672 - Travaux urgents
673 - Etudes techniques, diagnostic, consultation
677 - Pertes sur créances irrécouvrables
678 - Charges exceptionnelles
68 - Dotations aux dépréciations sur créances douteuses.
droit de la copropriété année 2019-2020
400
CLASSE 7 - Comptes de produits par nature
70 - Appels de fonds :
701 - Provisions sur opérations courantes
702 - Provisions sur travaux art. 14-2 703 - Avances
704 - remboursements d'annuités d'emprunt
71 - Autres produits:
711 - subventions
712 - emprunts
713 - indemnités d'assurance
714 - produits divers (dont intérêts légaux copro) 716 - produits financiers
718 - produits exceptionnels
78 - reprise de dépréciations sur créances douteuses
2. Il n'y a ni immobilisations ni stocks en copropriété.
On observera qu'il n'existe ni compte de la classe 2 (compte d'immobilisation) ni compte de la classe 3 (compte de stocks)., alors que ces comptes existent dans le plan comptable général. Il résulte de cette absence de comptes de ces deux classes que le syndicat des copropriétaires ne peut pratiquer aucune immobilisation et les stocks acquis en fin d'exercice sont considérés comme consommés au cours du même exercice (le fioul par exemple).
3. Des divisions de comptes sont possibles.
Le syndicat ne peut ouvrir aucun compte autre que ceux inclus dans la nomenclature. Par contre, rien n'interdit au syndicat de créer des subdivisions (électricité générale, électricité de l'ascenseur par exemple).
droit de la copropriété année 2019-2020
401
C. DOCUMENTS ET LES PIECES COMPTABLES.
1. Les pièces comptables
• Le livre journal
Le livre journal « enregistre chronologiquement les opérations ayant une incidence financière sur le fonctionnement du syndicat »,
• Le grand livre
Le grand livre des comptes « regroupe l'ensemble des comptes utilisés par le syndicat, opération par opération ».
Le grand livre de comptes regroupe les écritures chronologiques par catégories de comptes, en sorte que les totaux du grand livre pour un exercice déterminé correspondent aux totaux du livre journal pour le même exercice.
• La balance des comptes.
La balance des comptes synthétise le total des mouvements de chaque compte et le solde de chacun des comptes : pour une période donnée, les totaux de la balance générale en mouvement au débit et au crédit sont égaux entre eux et égaux au total des journaux relatifs à la même période.
L'article 2 de l'arrêté du 14 mars 2005 précise que le syndic édite deux balances générales des comptes :
- une balance éditée selon la nomenclature comptable,
- une balance éditée selon les clés de répartition des charges prévues par le règlement de copropriété.
2. Les archives comptables et leur transmission
Selon l’article 6 du décret du 14 mars 2005 :
« les pièces justificatives, document de base de toute écriture comptable, doivent être des originaux et comporter les références du syndicat (nom et adresse de l'immeuble). Elles doivent être datées et conservées par le syndic pendant 10 ans, sauf disposition contraire expresse »
Aux termes de l'arrêté du 14 mars 2005 :
droit de la copropriété année 2019-2020
402
« tout enregistrement comptable comporte un libellé permettant une identification de la pièce justificative qu'il appuie, notamment date et numéro de facture, actes et références supplément, période de l'appel de fonds et son objet.
La date à laquelle le paiement est intervenu peut être mentionné sur les factures, mémoires et situations »
L'article 33 du décret du 17 mars 1967 est consacré aux archives du syndic qui comprennent « les documents comptables du syndicat ».
Selon l'article 18 - 2 de la loi du 10 juillet 1965, en cas de changement de syndic, l’ancien syndic doit transmettre à son successeur dans le délai
- d'un mois de sa cessation de fonctions « la situation de trésorerie, la totalité des fonds immédiatement disponibles et l'ensemble des documents et archives du syndicat »
- et dans le délai complémentaire de deux mois (soit au total trois mois au plus après cessation des fonctions) l'ancien syndic doit remettre au nouveau syndic le solde des fonds disponibles après apurement des comptes et lui fournir l’état des comptes des copropriétaires ainsi que celui des comptes du syndicat ».
L'article 6 alinéa 2 du décret du 14 mars 2005 ajoute simplement :
"En cas de changement de syndic les documents comptables et les originaux des pièces justificatives sont transmis au successeur, le syndic sortant prenant ses propres dispositions afin de conserver les copies des pièces justificatives qu'il estime nécessaires pour la justification des opérations comptables qui lui incombaient ».
Le cumul de ces dispositions permet de conclure que le syndic doit transmettre les originaux à son successeur et qu'il lui incombe de garder les copies nécessaires à l'apurement des comptes.
3. La communication des comptes et de leur comparatif avec la convocation
Selon l’article 8 alinéa 5 du Décret du 14 mars 2005, développant l’article 14-3 de la loi :
« Les comptes de l’exercice clos sont à présenter pour leur approbation par les copropriétaires avec le budget voté correspondant à cet exercice et le comparatif des comptes approuvés de l’exercice précédent ».
Si les règles permettent un contrôle facilité des comptes de la copropriété, elles sont difficilement compréhensibles par le commun des copropriétaires pourtant appelés à approuver les comptes et le budget.
droit de la copropriété année 2019-2020
403
En fait le copropriétaire, s’il n’est membre du Conseil syndical, n’aura pas à examiner le détail des comptes. Pour autant il sera pleinement informé par l’annexion à la convocation de l'assemblée générale ayant à statuer sur les comptes des cinq annexes prévues par le décret du 14 mars 2005 :