\chapter*{Introduction}

\section*{Propos Liminaires}
	Le droit de la copropriété relève du droit des contrats.
	
	Le Code civil régit en effet les conventions conclues entre particuliers, en renvoyant essentiellement à la volonté des parties telle qu'exprimée dans leur accord. Il veille à la liberté et à la validité de l’échange des consentements, à la régularité de cet accord au regard de l’ordre public et pose la règle supérieure de la bonne foi des parties.
	
	Mais les temps modernes ont eu tendance à spécialiser le droit --- certains auteurs\footnote{Professeur Périnet Marquet Rapport au Congrès des Notaires de Deauville –-- mai 2003} parlent même de « balkanisation » du droit du contrat --- et peu à peu ce sont des pans entiers du contrat qui échappent au droit commun du Code civil pour faire l'objet de règles spécifiques de plus en plus complexes.
	
	En sorte que l'on peut penser que ces droits spéciaux sont en réalité des droits réglementaires et non plus des droits contractuels.
	
	L'extrême spécialisation de la réglementation présente un grand avantage : à quoi bon apprendre les principes généraux du droit dès lors qu'il suffira de connaître par coeur une réglementation dont le caractère impératif s'applique dans presque toutes les hypothèses rencontrées ?
	
	Le spécialiste sort sans aucun doute vainqueur de la confrontation entre la règle traditionnelle et la réglementation excessive : il ne suffit plus d'être juriste pour connaître la question ; il faut avant tout naviguer à l'aise dans cette réglementation.
	
	Mais la réglementation a toujours ses limites : elle ne peut tout prévoir et vient le moment où dans le silence de la loi et du décret nous devrons nous référer à la règle de droit traditionnelle.
	
	Le présent ouvrage s'adresse en priorité à des professionnels qui bien souvent connaissent –-- ou en tout cas appliquent --- tout ou partie de la réglementation ; il a pour but d'apporter à ces professionnels, comme à ceux qui découvrent la matière, non seulement une vue globale de la réglementation et l'acquisition de réflexes techniques, mais encore une connaissance des règles générales du droit et plus particulièrement de celles du Code civil en matière de contrats.
	
	En effet, une fois atteintes les limites de la réglementation, le Code civil demeure la référence supérieure dont la maîtrise permet d'atteindre véritablement le stade de la Spécialisation.

\section*{Lexique}

	\paragraph*{Copropriété}
	La copropriété est l'organisation d'un immeuble ou groupe d'immeubles dont la propriété est répartie en lots comportant chacun une partie privative et une quote part de partie commune. Cette forme existe dès l'instant qu'un immeuble est divisé entre 2 propriétaires et plus.
	
	\paragraph*{Lots}
	Chaque lot comprend une fraction d'immeuble appartenant au copropriétaire en pleine propriété (appartement, magasin, boutique, garage), dite « partie privative » et une quote-part des parties communes.
	Exemple : lot \no 1 : un appartement situé au rez-de-chaussée, à droite dans le hall d'entrée, comprenant, entrée, salon, salle à manger, chambre, cuisine, WC, salle de bains et les 75/1.000èmes des parties communes.
	
	\paragraph*{Parties privatives}
	Sont privatives les parties des bâtiments et des terrains réservées à l'usage exclusif d'un copropriétaire déterminé. Les parties privatives sont la propriété exclusive de chaque copropriétaire.
	Exemple : Les parquets et carrelages, les cloisons intérieures, les enduits des murs et plafonds, les peintures et tapisseries, les équipements et aménagements intérieurs (appareils sanitaires et de cuisine, placards).
	
	\paragraph*{Parties communes (ou choses communes)}
	Parties de l'immeuble qui sont en état d'indivision forcée entre tous les copropriétaires ou certains d'entre eux. Elles sont à l’usage ou à l’utilité de tous.
	Exemple : sols, cours, jardins, gros oeuvre des bâtiments, hall d'entrée, escaliers, toiture, etc...
	
	\paragraph*{Tantièmes}
	Les droits indivis de chaque copropriétaire sur les parties communes sont représentés par des tantièmes de propriété qui s'expriment en "1.000ème", "10.000ème", etc....Ces tantièmes sont calculés pour chaque lot en fonction de sa valeur relative, c’est-à-dire de sa consistance, de sa superficie et de sa situation sans qu'il soit tenu compte de son utilisation (article 5 de la loi qui n'est pas d'ordre public).
	Les tantièmes de propriété attribués à chaque lot déterminent le nombre de voix attribué à chaque copropriétaire ainsi que la quote part de charge commune générales.
	
	\paragraph*{Charges communes générales}
	Frais entraînés par la conservation, l'entretien et l'administration des parties communes.
	Ces frais ou charges sont en général répartis entre tous les copropriétaires au prorata de leurs tantièmes de copropriété et en tout cas, obligatoirement, en fonction de la consistance, de la superficie et de la situation de chaque lot (sans qu'il soit tenu compte de son utilisation) : Tantièmes de charges.
	
	\paragraph*{Charges relatives au services et éléments d’équipements}
	Frais entraînés par les services collectifs et les éléments d'équipement communs.
	Ces frais sont obligatoirement répartis entre les copropriétaires en fonction de l'utilité que les services et éléments présentent pour chaque lot.
	
	\paragraph*{Règlement de copropriété (autrefois dénommé « cahier des charges »)}
	Document obligatoire établi en principe au moment de la mise en copropriété de l'immeuble, et même avant :
	\begin{itemize}
		\item déterminant les droits et obligations des copropriétaires
		\item  fixant la répartition des charges,
		\item  définissant les modalités de fonctionnement du Syndicat.
	\end{itemize}
	
	\paragraph*{État descriptif de division}
	Document rendu obligatoire en matière de copropriété par le décret \no 59-89 du 7 janvier 1959, destiné à la publicité foncière, ayant pour objet l'identification de l'immeuble, sa division en lots numérotés, la description des lots avec l'indication de leurs tantièmes de copropriété. La désignation des lots est obligatoirement résumée dans un tableau récapitulatif.
	
	Exemple : lot 1 : appartement au rez-de-chaussée à gauche comprenant quatre pièces principales, une cuisine, une salle de bains.... et les 100/1.000èmes des parties communes générales.
	
	\paragraph*{Syndicat des copropriétaires}
	Collectivité qui regroupe tous les copropriétaires d’un immeuble l'ensemble des copropriétaires sans exception. Elle est dotée de la personnalité morale, ses organes de gestion sont l’Assemblée Générale, le Syndic et le Conseil Syndical.
	Le Syndicat des Copropriétaires est chargé de l'administration et de la conservation des parties communes.
	
	\paragraph*{Assemblées générales}
	Ce sont les réunions qui se tiennent au moins une fois par an, auxquelles assistent tous les copropriétaires et au cours desquelles, ils votent sur les questions intéressant l'immeuble, avec des voix proprotionnelles aux tantièmes de leur lot.
	
	Il n'y a pas d'assemblée générale extraordinaire, ni d'assemblée générale ordinaire, mais des assemblées générales aux cours desquelles certaines décisions devront être prises à la majorité simple, d'autres à la majorité absolue et même à la double majorité.
	
	\paragraph*{Syndic} 
	C'est le mandataire du syndicat des copropriétaires, élu en assemblée générale, pour administrer l'immeuble et veiller à sa conservation, ainsi qu’au respect du règlement de copropriété et des décisions d’assemblée générale. Il peut être bénévole ou professionnel.
	
	\paragraph*{Conseil syndical}
	C'est un groupe de copropriétaires élus par l'assemblée générale pour assister et contrôler le syndic.
	
	\paragraph*{Les majorités}
	Les décisions du Syndicat des Copropriétaires sont prises par les copropriétaires ayant des droits dans les parties communes concernées au cours de l’assemblée générale.
	Ces décisions sont votées nominativement en tenant compte du nombre de voix dont chaque copropriétaire dispose (ce nombre est fonction des tantièmes de propriété du lot) :
	\begin{description}
		\item[Majorité de l’article 24 de la loi (dite majorité relative)] Majorité des voix exprimées des présents ou représentés.
		\item[Majorité de l’article 25 de la loi (dite majorité absolue)] Majorité des voix des tous les copropriétaires de la loi.
		\item[Majorité de l’article 25-1 de la loi] Lorsque l'assemblée générale des copropriétaires n'a pas décidé à la majorité prévue à l'article précédent mais que le projet a recueilli au moins le tiers des voix de tous les copropriétaires composant le syndicat, la même assemblée peut décider à la majorité prévue à l'article 24 en procédant immédiatement à un second vote.
		\item[Majorité de l’article 26 de la loi (dite double majorité)] Majorité des 2/3 des voix de tous les copropriétaires représentant la majorité en nombre des copropriétaires. Cette majorité a été modifiée par l’Ordonnance du 30 octobre 2019
		\item[Majorité de l’article 26-1 de la loi] Lorsque l’assemblée générale n’a pas décidé à la majorité prévue au premier alinéa de l’article 26 mais que le projet a au moins recueilli l’approbation de la moitié des membres du syndicat des copropriétaires présents, représentés ou ayant voté par correspondance, représentant au moins le tiers des voix de tous les copropriétaires, la même assemblée se prononce à la majorité des voix de tous les copropriétaires en procédant immédiatement à un second vote.
		
		Exemple : si dans une copropriété existe 12 copropriétaires et que l’ensemble des lots représente 1.000/1.000èmes, la double majorité de l’article 26 est obtenue lorsque la résolution recueille le vote favorable de 7 (6 + 1) copropriétaires réunissant eux-mêmes 667 voix sur 1.000 (2/3 des voix).
	\end{description}

	Certaines décisions nécessitent l’unanimité des voix … donc des copropriétaires également.
	
\section*{Bibliographie}

\section*{Les acteurs en copropriété}

	\subsection*{La commission relative a la copropriété (CRC)}
	Par arrêté du 4 août 1987 le Garde des Sceaux a créé une commission relative à la copropriété, qui comprenait le représentant du directeur des affaires civiles et du sceau et celui du ministère de l'équipement, des transports et du logement, un Professeur de Droit, un Notaire et un Avocat, ainsi qu’à parité, des représentants d’association de propriétaires et d’associations représentant les administrateurs de bien.
	
	Depuis le 11 février 1994, M. CAPOULADE (Pierre), conseiller Honoraire à la Cour de cassation, était le président de la commission relative à la copropriété. M. PERINET MARQUET représentait la faculté.
	
	La Commission a rédigé des \textbf{Recommandations} sur toutes les questions relatives au statut de la copropriété et émis des \textbf{Avis} (non publiés) sur les questions qui lui étaient soumises par le Garde des Sceaux.
	
	Ces Recommandations n’ont pas valeur « normative », mais on ne connaît qu’un seul cas en jurisprudence où une Cour d’appel a refusé de faire application d’une Recommandation de la Commission.\footnote{CA Paris 1\iere Ch, 7 mai 2003, Administrer \no 359 octobre 2003, sur renvoi de cassation à propos de l’élection par vote bloqué du conseil syndical alors que le nombre de candidats était égal au nombre de postes à pourvoir.}
	
	A l’occasion de la loi ALUR, la commission a été supprimée au profit du CNTGI (dont les missions n’étaient pas cependant identiques\dots).
	
	\subsection*{Le CNTGI}
	Le CNTGI, Conseil National de la Transaction et de la Gestion Immobilières a été créé par la loi ALUR (article 24) du 24 mars 2014. Sa composition et son fonctionnement ont été modifiés par la loi du 27 janvier 2017 dite Egalité et Citoyenneté. Ces textes sont devenus caducs avec l’adoption de la loi ELAN dont l’article 53 d’origine (devenu l’article 151 de la Petite Loi) réforme profondément l’Institution.
	
	Le CNTGI doit être consulté « pour avis », notamment « sur les projets de textes législatifs ou réglementaires relatifs à la copropriété » et plus seulement sur les projets de loi relatifs à l’exercice des activités des professionnels de l’immobilier qui sont soumis à la loi Hoguet.
	
	La loi ELAN a retiré partiellement au CNTGI la fonction disciplinaire qui lui avait été donnée par la loi ALUR pour ne lui laisser, au moyen d’une Commission de contrôle des activités de transaction et de gestion immobilière, qu’un rôle d’instruction des « cas de pratiques abusives portées à la connaissance du conseil ».d’alerte auprès de la DGCCRF.
	
	Le financement de ce Conseil devait à l’origine être assuré par \textbf{le versement d’une cotisation professionnelle} par les personnes soumises à la réglementation Hoguet dont le montant aurait été fixé tous les trois ans par le Garde des Sceaux. Cette obligation de financement, très mal vécue par les professionnels, a été supprimée par la loi ELAN.
	
	\subsection*{Syndicats professionnels et associations}
	Au sein du CNTGI, quelle que soit la formule retenue par la loi ELAN, se trouvent un certain nombre de représentants des professionnels et des « consommateurs ».
	
	Les professionnels de la gestion en copropriété sont essentiellement groupés en deux syndicats professionnels qui ont pour but d’assurer la représentativité et l’influence de leur organisation professionnelle auprès des pouvoirs publics et de défendre leurs adhérents :
	\begin{enumerate}
		\item la FNAIM, numériquement le syndicat le plus important en nombre de membres (12.000 membres revendiqués) et qui consacre une grande partie de son activité aux agents immobiliers et transactionnaires ;
		\item l’Union des syndicats de l'immobilier (UNIS), regroupant également les agents immobiliers et gestionnaires mais dont l’activité principale est consacrée à la mise en valeur de la profession de syndic de copropriété et gestionnaire d’immeuble.
	\end{enumerate}
	
	Les associations dite de « consommateurs » sont essentiellement :
	\begin{enumerate}
		\item l’Association des Responsables de copropriété (ARC) qui adhère à l’UNARC (Union Nationale) et revendique l’adhésion de \nombre{14 000} syndicats de copropriétaires ; \footnote{Cette association met essentiellement en valeur la gestion des copropriétés par des syndics bénévoles (copropriétaires assurant la fonction de syndic) auxquels elle propose d’apporter ses compétences techniques, juridiques et comptables. Elle se veut un contre-pouvoir aux syndics professionnels.}
		\item l’Association Consommation, Logement et Cadre de Vie (CLCV) qui défend essentiellement les copropriétaires eux-mêmes et qui est très impliquée dans la protection de l’environnement ;
		\item  l’Union Nationale des propriétaires immobiliers (UNPI) forte de \nombre{250 000} adhérents ;
		\item  l’Association nationale de la copropriété et des copropriétaires (ANCC) historiquement dédiée aux syndicats coopératifs, mais dont la compétence a été élargie ;
		\item  la Fédération des syndicats coopératifs de copropriété (FSCC)
	\end{enumerate}
