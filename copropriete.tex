\documentclass[10pt,a4paper,twoside]{book}
\usepackage[utf8]{inputenc}
\usepackage[T1]{fontenc}
\usepackage{graphicx}
\usepackage{lmodern} % Police un peu plus jolie
\usepackage{eurosym}
\usepackage{chemist}
\usepackage{enumitem}
% à passer en dernier
\usepackage[french]{babel}
\usepackage{hyperref}

\author{Agnès \nom{Lebatteux}}
\title{Copropriété}
\date{2010}

\hypersetup{%
	pdfinfo={%
		Title={\@title}%
		%, Subject={}%
		%, Author={Samuel \nom{Déom}}%
		, Keywords={Copropriété, Master}%
	}%
	, colorlinks = true% colore, plutot qu'encadre, les liens hypertexte
	, linkcolor = black% colore les liens internes en noir
	, urlcolor = black% colore les liens externes en noir
	, breaklinks = true% autorise les liens à être étendus sur plusieurs lignes
}

\newcommand*{\siecle}[1]{\textsc{#1}\ieme{}~siècle}
\newcommand*{\nom}[1]{\textsc{#1}}
\newcommand*{\pourcent}[1]{\nombre{#1}~\%}
\newcommand*{\montant}[1]{\nombre{#1}~\euro}
\newcommand*{\metreCarre}{m\up{2}}
\newcommand*{\prixSurface}[1]{\montant{#1}/m\up{2}} 
\newcommand*{\dioxydeDeCarbone}{\chemform{CO_2}}
\newcommand*{\I}{\textsc{i}}
\newcommand*{\II}{\textsc{ii}}
\newcommand*{\III}{\textsc{iii}}
\newcommand*{\IV}{\textsc{iv}}
\newcommand*{\V}{\textsc{v}}
\newcommand*{\VI}{\textsc{vi}}
\newcommand{\CCH}{Code de la construction et de l'habitation}

\begin{document}
	\maketitle
	
	\part*{Préambule}
	
		« Un propriétaire, un bien ne sont jamais isolés ; toute
		propriété se heurte à d’autres propriétés, à d’autres libertés
		qui la limitent inévitablement" (C. Atias).
	
		\chapter*{Introduction}

\section*{Propos Liminaires}
	Le droit de la copropriété relève du droit des contrats.
	
	Le Code civil régit en effet les conventions conclues entre particuliers, en renvoyant essentiellement à la volonté des parties telle qu'exprimée dans leur accord. Il veille à la liberté et à la validité de l’échange des consentements, à la régularité de cet accord au regard de l’ordre public et pose la règle supérieure de la bonne foi des parties.
	
	Mais les temps modernes ont eu tendance à spécialiser le droit --- certains auteurs\footnote{Professeur Périnet Marquet Rapport au Congrès des Notaires de Deauville –-- mai 2003} parlent même de « balkanisation » du droit du contrat --- et peu à peu ce sont des pans entiers du contrat qui échappent au droit commun du Code civil pour faire l'objet de règles spécifiques de plus en plus complexes.
	
	En sorte que l'on peut penser que ces droits spéciaux sont en réalité des droits réglementaires et non plus des droits contractuels.
	
	L'extrême spécialisation de la réglementation présente un grand avantage : à quoi bon apprendre les principes généraux du droit dès lors qu'il suffira de connaître par coeur une réglementation dont le caractère impératif s'applique dans presque toutes les hypothèses rencontrées ?
	
	Le spécialiste sort sans aucun doute vainqueur de la confrontation entre la règle traditionnelle et la réglementation excessive : il ne suffit plus d'être juriste pour connaître la question ; il faut avant tout naviguer à l'aise dans cette réglementation.
	
	Mais la réglementation a toujours ses limites : elle ne peut tout prévoir et vient le moment où dans le silence de la loi et du décret nous devrons nous référer à la règle de droit traditionnelle.
	
	Le présent ouvrage s'adresse en priorité à des professionnels qui bien souvent connaissent –-- ou en tout cas appliquent --- tout ou partie de la réglementation ; il a pour but d'apporter à ces professionnels, comme à ceux qui découvrent la matière, non seulement une vue globale de la réglementation et l'acquisition de réflexes techniques, mais encore une connaissance des règles générales du droit et plus particulièrement de celles du Code civil en matière de contrats.
	
	En effet, une fois atteintes les limites de la réglementation, le Code civil demeure la référence supérieure dont la maîtrise permet d'atteindre véritablement le stade de la Spécialisation.

\section*{Lexique}

	\paragraph*{Copropriété}
	La copropriété est l'organisation d'un immeuble ou groupe d'immeubles dont la propriété est répartie en lots comportant chacun une partie privative et une quote part de partie commune. Cette forme existe dès l'instant qu'un immeuble est divisé entre 2 propriétaires et plus.
	
	\paragraph*{Lots}
	Chaque lot comprend une fraction d'immeuble appartenant au copropriétaire en pleine propriété (appartement, magasin, boutique, garage), dite « partie privative » et une quote-part des parties communes.
	Exemple : lot \no 1 : un appartement situé au rez-de-chaussée, à droite dans le hall d'entrée, comprenant, entrée, salon, salle à manger, chambre, cuisine, WC, salle de bains et les 75/1.000èmes des parties communes.
	
	\paragraph*{Parties privatives}
	Sont privatives les parties des bâtiments et des terrains réservées à l'usage exclusif d'un copropriétaire déterminé. Les parties privatives sont la propriété exclusive de chaque copropriétaire.
	Exemple : Les parquets et carrelages, les cloisons intérieures, les enduits des murs et plafonds, les peintures et tapisseries, les équipements et aménagements intérieurs (appareils sanitaires et de cuisine, placards).
	
	\paragraph*{Parties communes (ou choses communes)}
	Parties de l'immeuble qui sont en état d'indivision forcée entre tous les copropriétaires ou certains d'entre eux. Elles sont à l’usage ou à l’utilité de tous.
	Exemple : sols, cours, jardins, gros oeuvre des bâtiments, hall d'entrée, escaliers, toiture, etc...
	
	\paragraph*{Tantièmes}
	Les droits indivis de chaque copropriétaire sur les parties communes sont représentés par des tantièmes de propriété qui s'expriment en "1.000ème", "10.000ème", etc....Ces tantièmes sont calculés pour chaque lot en fonction de sa valeur relative, c’est-à-dire de sa consistance, de sa superficie et de sa situation sans qu'il soit tenu compte de son utilisation (article 5 de la loi qui n'est pas d'ordre public).
	Les tantièmes de propriété attribués à chaque lot déterminent le nombre de voix attribué à chaque copropriétaire ainsi que la quote part de charge commune générales.
	
	\paragraph*{Charges communes générales}
	Frais entraînés par la conservation, l'entretien et l'administration des parties communes.
	Ces frais ou charges sont en général répartis entre tous les copropriétaires au prorata de leurs tantièmes de copropriété et en tout cas, obligatoirement, en fonction de la consistance, de la superficie et de la situation de chaque lot (sans qu'il soit tenu compte de son utilisation) : Tantièmes de charges.
	
	\paragraph*{Charges relatives au services et éléments d’équipements}
	Frais entraînés par les services collectifs et les éléments d'équipement communs.
	Ces frais sont obligatoirement répartis entre les copropriétaires en fonction de l'utilité que les services et éléments présentent pour chaque lot.
	
	\paragraph*{Règlement de copropriété (autrefois dénommé « cahier des charges »)}
	Document obligatoire établi en principe au moment de la mise en copropriété de l'immeuble, et même avant :
	\begin{itemize}
		\item déterminant les droits et obligations des copropriétaires
		\item  fixant la répartition des charges,
		\item  définissant les modalités de fonctionnement du Syndicat.
	\end{itemize}
	
	\paragraph*{État descriptif de division}
	Document rendu obligatoire en matière de copropriété par le décret \no 59-89 du 7 janvier 1959, destiné à la publicité foncière, ayant pour objet l'identification de l'immeuble, sa division en lots numérotés, la description des lots avec l'indication de leurs tantièmes de copropriété. La désignation des lots est obligatoirement résumée dans un tableau récapitulatif.
	
	Exemple : lot 1 : appartement au rez-de-chaussée à gauche comprenant quatre pièces principales, une cuisine, une salle de bains.... et les 100/1.000èmes des parties communes générales.
	
	\paragraph*{Syndicat des copropriétaires}
	Collectivité qui regroupe tous les copropriétaires d’un immeuble l'ensemble des copropriétaires sans exception. Elle est dotée de la personnalité morale, ses organes de gestion sont l’Assemblée Générale, le Syndic et le Conseil Syndical.
	Le Syndicat des Copropriétaires est chargé de l'administration et de la conservation des parties communes.
	
	\paragraph*{Assemblées générales}
	Ce sont les réunions qui se tiennent au moins une fois par an, auxquelles assistent tous les copropriétaires et au cours desquelles, ils votent sur les questions intéressant l'immeuble, avec des voix proprotionnelles aux tantièmes de leur lot.
	
	Il n'y a pas d'assemblée générale extraordinaire, ni d'assemblée générale ordinaire, mais des assemblées générales aux cours desquelles certaines décisions devront être prises à la majorité simple, d'autres à la majorité absolue et même à la double majorité.
	
	\paragraph*{Syndic} 
	C'est le mandataire du syndicat des copropriétaires, élu en assemblée générale, pour administrer l'immeuble et veiller à sa conservation, ainsi qu’au respect du règlement de copropriété et des décisions d’assemblée générale. Il peut être bénévole ou professionnel.
	
	\paragraph*{Conseil syndical}
	C'est un groupe de copropriétaires élus par l'assemblée générale pour assister et contrôler le syndic.
	
	\paragraph*{Les majorités}
	Les décisions du Syndicat des Copropriétaires sont prises par les copropriétaires ayant des droits dans les parties communes concernées au cours de l’assemblée générale.
	Ces décisions sont votées nominativement en tenant compte du nombre de voix dont chaque copropriétaire dispose (ce nombre est fonction des tantièmes de propriété du lot) :
	\begin{description}
		\item[Majorité de l’article 24 de la loi (dite majorité relative)] Majorité des voix exprimées des présents ou représentés.
		\item[Majorité de l’article 25 de la loi (dite majorité absolue)] Majorité des voix des tous les copropriétaires de la loi.
		\item[Majorité de l’article 25-1 de la loi] Lorsque l'assemblée générale des copropriétaires n'a pas décidé à la majorité prévue à l'article précédent mais que le projet a recueilli au moins le tiers des voix de tous les copropriétaires composant le syndicat, la même assemblée peut décider à la majorité prévue à l'article 24 en procédant immédiatement à un second vote.
		\item[Majorité de l’article 26 de la loi (dite double majorité)] Majorité des 2/3 des voix de tous les copropriétaires représentant la majorité en nombre des copropriétaires. Cette majorité a été modifiée par l’Ordonnance du 30 octobre 2019
		\item[Majorité de l’article 26-1 de la loi] Lorsque l’assemblée générale n’a pas décidé à la majorité prévue au premier alinéa de l’article 26 mais que le projet a au moins recueilli l’approbation de la moitié des membres du syndicat des copropriétaires présents, représentés ou ayant voté par correspondance, représentant au moins le tiers des voix de tous les copropriétaires, la même assemblée se prononce à la majorité des voix de tous les copropriétaires en procédant immédiatement à un second vote.
		
		Exemple : si dans une copropriété existe 12 copropriétaires et que l’ensemble des lots représente 1.000/1.000èmes, la double majorité de l’article 26 est obtenue lorsque la résolution recueille le vote favorable de 7 (6 + 1) copropriétaires réunissant eux-mêmes 667 voix sur 1.000 (2/3 des voix).
	\end{description}

	Certaines décisions nécessitent l’unanimité des voix … donc des copropriétaires également.
	
\section*{Bibliographie}

\section*{Les acteurs en copropriété}

	\subsection*{La commission relative a la copropriété (CRC)}
	Par arrêté du 4 août 1987 le Garde des Sceaux a créé une commission relative à la copropriété, qui comprenait le représentant du directeur des affaires civiles et du sceau et celui du ministère de l'équipement, des transports et du logement, un Professeur de Droit, un Notaire et un Avocat, ainsi qu’à parité, des représentants d’association de propriétaires et d’associations représentant les administrateurs de bien.
	
	Depuis le 11 février 1994, M. CAPOULADE (Pierre), conseiller Honoraire à la Cour de cassation, était le président de la commission relative à la copropriété. M. PERINET MARQUET représentait la faculté.
	
	La Commission a rédigé des \textbf{Recommandations} sur toutes les questions relatives au statut de la copropriété et émis des \textbf{Avis} (non publiés) sur les questions qui lui étaient soumises par le Garde des Sceaux.
	
	Ces Recommandations n’ont pas valeur « normative », mais on ne connaît qu’un seul cas en jurisprudence où une Cour d’appel a refusé de faire application d’une Recommandation de la Commission.\footnote{CA Paris 1\iere Ch, 7 mai 2003, Administrer \no 359 octobre 2003, sur renvoi de cassation à propos de l’élection par vote bloqué du conseil syndical alors que le nombre de candidats était égal au nombre de postes à pourvoir.}
	
	A l’occasion de la loi ALUR, la commission a été supprimée au profit du CNTGI (dont les missions n’étaient pas cependant identiques\dots).
	
	\subsection*{Le CNTGI}
	Le CNTGI, Conseil National de la Transaction et de la Gestion Immobilières a été créé par la loi ALUR (article 24) du 24 mars 2014. Sa composition et son fonctionnement ont été modifiés par la loi du 27 janvier 2017 dite Egalité et Citoyenneté. Ces textes sont devenus caducs avec l’adoption de la loi ELAN dont l’article 53 d’origine (devenu l’article 151 de la Petite Loi) réforme profondément l’Institution.
	
	Le CNTGI doit être consulté « pour avis », notamment « sur les projets de textes législatifs ou réglementaires relatifs à la copropriété » et plus seulement sur les projets de loi relatifs à l’exercice des activités des professionnels de l’immobilier qui sont soumis à la loi Hoguet.
	
	La loi ELAN a retiré partiellement au CNTGI la fonction disciplinaire qui lui avait été donnée par la loi ALUR pour ne lui laisser, au moyen d’une Commission de contrôle des activités de transaction et de gestion immobilière, qu’un rôle d’instruction des « cas de pratiques abusives portées à la connaissance du conseil ».d’alerte auprès de la DGCCRF.
	
	Le financement de ce Conseil devait à l’origine être assuré par \textbf{le versement d’une cotisation professionnelle} par les personnes soumises à la réglementation Hoguet dont le montant aurait été fixé tous les trois ans par le Garde des Sceaux. Cette obligation de financement, très mal vécue par les professionnels, a été supprimée par la loi ELAN.
	
	\subsection*{Syndicats professionnels et associations}
	Au sein du CNTGI, quelle que soit la formule retenue par la loi ELAN, se trouvent un certain nombre de représentants des professionnels et des « consommateurs ».
	
	Les professionnels de la gestion en copropriété sont essentiellement groupés en deux syndicats professionnels qui ont pour but d’assurer la représentativité et l’influence de leur organisation professionnelle auprès des pouvoirs publics et de défendre leurs adhérents :
	\begin{enumerate}
		\item la FNAIM, numériquement le syndicat le plus important en nombre de membres (12.000 membres revendiqués) et qui consacre une grande partie de son activité aux agents immobiliers et transactionnaires ;
		\item l’Union des syndicats de l'immobilier (UNIS), regroupant également les agents immobiliers et gestionnaires mais dont l’activité principale est consacrée à la mise en valeur de la profession de syndic de copropriété et gestionnaire d’immeuble.
	\end{enumerate}
	
	Les associations dite de « consommateurs » sont essentiellement :
	\begin{enumerate}
		\item l’Association des Responsables de copropriété (ARC) qui adhère à l’UNARC (Union Nationale) et revendique l’adhésion de \nombre{14 000} syndicats de copropriétaires ; \footnote{Cette association met essentiellement en valeur la gestion des copropriétés par des syndics bénévoles (copropriétaires assurant la fonction de syndic) auxquels elle propose d’apporter ses compétences techniques, juridiques et comptables. Elle se veut un contre-pouvoir aux syndics professionnels.}
		\item l’Association Consommation, Logement et Cadre de Vie (CLCV) qui défend essentiellement les copropriétaires eux-mêmes et qui est très impliquée dans la protection de l’environnement ;
		\item  l’Union Nationale des propriétaires immobiliers (UNPI) forte de \nombre{250 000} adhérents ;
		\item  l’Association nationale de la copropriété et des copropriétaires (ANCC) historiquement dédiée aux syndicats coopératifs, mais dont la compétence a été élargie ;
		\item  la Fédération des syndicats coopératifs de copropriété (FSCC)
	\end{enumerate}

	
	\part{Le statut de la copropriété}
	
		\chapter{Historique de la copropriété}

\section{Des prémices au Code Civil}
	\subsection{L’antiquité}
		Depuis longtemps les individus se sont groupés pour construire à frais communs un immeuble et s'en répartir la propriété ou pour diviser un immeuble existant. Cette propriété partagée entre plusieurs personnes constitue la Copropriété.
		
		Certains auteurs
		\footnote{Cuq, Etude sur les contrats de l'Epoque de la première dynastie babylonienne : Nouvelle revue historique droit français et étranger, juillet 1910, p. 458 et s.}
		font remonter la copropriété à la Première Dynastie babylonienne (soit au Deuxième Millénaire avant notre ère) et la propriété par étages semble avoir été connue dans tout l'Orient
		\footnote{Selon le Professeur MICHALOPOULOS (Rd Imm 17 (3), juill - sep 1995 p 409), les Phéniciens auraient développé dans les villes qu'ils ont créées, dont CARTHAGE, des règles juridiques ressemblant à notre copropriété. Carthage ayant été détruite \dots nous n’y avons trouvé aucune trace de copropriété verticale.}.
		
	\subsection{Les coutumiers français}
		En France le droit coutumier révèle que ce régime de copropriété existait dans certaines Régions.
		
		\begin{description}
			\item[La Coutume d'Auxerre de 1561\footnote{art. 116}] prévoyait ainsi :
				\begin{quote}
					Celui à qui appartient le bas est tenu de faire entretenir tout le tour du bas de la muraille, pans ou cloisons, tellement que le haut puisse porter dessus, et est tenu de faire le plancher dessus lui de poutres, solives ou torchis, et celui qui a le dessus est tenu en faire autant du haut \dots
				\end{quote}
			\item[La Coutume d'Orléans\footnote{Grand Coutumier général, Tome III, p. 600}] édictait une obligation d'entretien à frais communs :
				\begin{quote}
					Seront faits à frais communs et entretenus les pavés estans devant lesdites maisons
				\end{quote}
		\end{description}
		
		Mais il a fallu des circonstances particulières :
		\begin{itemize}
			\item tel l'Incendie de Rennes en 1720 et la décision de reconstruire en élargissant les voies et en réservant de nombreux espaces pour les places et les jardins,
			\item ou l'impossibilité de développer une ville enserrée dans ces fortifications comme Grenoble\footnote{C'est à Grenoble qu'apparaissent les conventions de copropriété appelées << règlements de copropriété >> (MICHALOPOULOS, op. cit.).} pour que l'on trouve les premières réalisations de constructions indivises entre plusieurs propriétaires.
		\end{itemize}
		
		\par Dans les deux cas le schéma de construction était le suivant : le propriétaire du rez-de-chaussée construisait les fondations et les gros murs de son niveau, les propriétaires des étages construisaient les planchers et gros murs de leurs étages, le propriétaire du dernier étage couvrait la maison. Pour l'entretien, chacun avait la charge de ce qu'il avait construit.
		
		Il s’agissait donc de superposition de propriétés plutôt que de copropriété.
		
	\subsection{L'article 664 du Code Civil}
		L'article 664 du Code Civil fut ajouté à la demande des cours d'Appel de Lyon et Grenoble. Cet article était ainsi rédigé :
		\begin{quote}
			Lorsque les différents étages d'une maison appartiennent à divers propriétaires, si les titres de propriété ne règlent pas le mode de réparations et reconstructions elles doivent être faites ainsi qu'il suit :
			\begin{itemize}
				\item  Les gros murs et le toit sont à la charge de tous les propriétaires, chacun en proportion de la valeur de l'étage qui lui appartient.
				\item  Le propriétaire de chaque étage fait le plancher sur lequel il marche.
				\item  Le propriétaire du premier étage fait l'escalier qui y conduit le propriétaire du second étage fait, à partir du premier, l'escalier qui conduit chez lui, et ainsi de suite.
			\end{itemize}
		\end{quote}
	
		On reconnaît l'inspiration de la coutume d'Auxerre, mais cet article ne traitait que du mode de répartition des réparations et constructions.
		
		La propriété appartenait à celui qui avait exécuté les travaux. Les Tribunaux estimèrent cependant qu'il existait une servitude d'indivision (c’est à dire une propriété partagée) pour les éléments séparatifs entre appartements, tels le plancher.
		
		Or si chaque étage était propriété privative, cet article ne prévoyait aucun mode de gestion pour les éléments communs. Il est vrai que l'article 664 Code Civil avait un caractère supplétif, donc ne s'appliquait qu'à défaut de dispositions contractuelles.
		
		Il appartenait donc aux propriétaires d'organiser conventionnellement la gestion de l'immeuble et les modalités de jouissance des parties communes, conformément aux dispositions de l’article 1134 du code civil.
		
		La jurisprudence sur l’article 664 du code civil fut très limitée au \siecle{xix}\footnote{Cf. La copropriété des immeubles bâtis dans la jurisprudence et la doctrine du \siecle{xix} Marc Ortolani ; Revue historique de droit français et étranger (1922-2014) Vol. 78, No. 2 (AVRIL-JUIN 2000), pp. 249-287}

\section{Le premier « statut » de la copropriété : la loi de 1938}

	\subsection{Évolution des idées et des faits au \siecle{xx}.}
		L'idée d'une indivision forcée entre les propriétaires d'un immeuble divisé par étages se fit jour au début du \siecle{xx}.
		
		Plusieurs facteurs ont permis le développement des immeubles en copropriété : destruction massive des immeubles pendant la première guerre mondiale, concentration des populations dans les villes, amenuisement des fortunes et alourdissement des charges fiscales qui incitent les propriétaires à vendre leurs immeubles par appartements $\dots$
		
		De plus, les techniques ont évolué en faisant intervenir des matériaux onéreux, tandis que les équipements des immeubles devenaient complexes.
		La constitution de sociétés civiles devint donc nécessaire pour réaliser des immeubles importants: les associés apportaient en effet les fonds nécessaires à ces réalisations.
		\begin{quote}
			Il est apparu que les choses ne pouvaient plus continuer à se passer en famille ou entre amis, comme à Grenoble où l'usage séculaire vient adoucir bien des difficultés. (Bull. de la Sté d'Etudes Législatives 1925-26, p. 167)
		\end{quote}
		Pourtant, à une époque où le droit affirmait que les sociétés devaient être constituées pour << partager des bénéfices >>, la constitution de sociétés destinées à réaliser des économies, et non à mettre en commun des bénéfices, était sujette à caution !
	
	\subsection{La loi du 28 juin 1938}
		Sans doute inspirée de la législation belge (loi du 8 juillet 1924 abrogeant l’article 664 du code civil), la loi de 1938 (abrogeant pour la France l’article 664 du code civil) comprenait deux parties : la première régissait les sociétés de construction, la seconde organisait le statut des immeubles en copropriété.
		
		Les différents propriétaires étaient obligatoirement groupés en un syndicat.
		
		La loi institue :
		\begin{itemize}
			\item  le « règlement de copropriété », objet d'une convention générale ou de l'engagement de chacun des intéressés et lui a conféré force obligatoire à l'égard des ayants cause à titre particulier des copropriétaires au moyen de sa transcription aux hypothèques ;
			\item  le syndic dont elle déterminait le mode de nomination, les pouvoirs et la rémunération.
		\end{itemize}
	
	\subsection{Les textes subséquents}
		Un \textbf{Décret loi du 29 novembre 1939} a renforcé le privilège du syndicat pour garantir le paiement des charges communes en précisant le rang et les effets de ce privilège immobilier et a créé un privilège mobilier sur les meubles meublants l'appartement du copropriétaire avec report de ce privilège sur les loyers lorsque l’appartement est donné en location.
		
		Ce privilège immobilier sera supprimé par le Décret portant réforme de la publicité foncière du 4 janvier 1955.
		
		Une \textbf{loi du 7 février 1953} a incidemment étendu le régime de la copropriété aux copropriétés horizontales.
		
		\begin{quote}
			Les dispositions de la loi du 28 juin 1938 sont étendues aux sociétés constituées ou à constituer, quelle qu'en soit la forme, ayant pour objet la construction, l'acquisition ou la gestion d'ensembles immobiliers à usage principal d'habitation, composés d'immeubles collectifs, de maisons individuelles et éventuellement des services communs y afférents et destinés à être attribués aux associés en propriété ou en jouissance $\dots$
		\end{quote}
		
		Ce texte servira de base à la rédaction de l'article 1er alinéa 2 de la loi de 1965.
		
		La \textbf{copropriété verticale} évoque un même corps de bâtiment dont les appartements sont superposés.
		La \textbf{copropriété horizontale} évoque plusieurs corps de bâtiments ou plusieurs maisons individuelles dans un même ensemble.
		
		Différents décrets de 1954 à 1959, destinés à protéger les associés des Sociétés Immobilières dites d'attribution évoqueront la copropriété des immeubles.
		
		Relevons enfin le Décret du 20 mai 1955 ayant pour objet la simplification des procédures administratives applicables à la construction d'ensembles immobiliers qui édicte qu'un cahier des charges annexé à la demande de permis de construire, précise les conditions dans lesquelles la gestion et l'entretien des ouvrages et aménagements d'intérêt collectif seront assurés par les copropriétaires ou par une association syndicale constituée à cet effet.

\section{La loi du 10 juillet 1965 et le décret du 17 mars 1967}
	Le projet de loi fut discuté pendant une année exactement, avant d’être signée le 10 juillet 1965 :
	\begin{itemize}
		\item 	à l'Assemblée Nationale sur le rapport de Monsieur \nom{Zimmermann} qui évoquait essentiellement le caractère impératif de ses dispositions en parlant à son sujet de contrat d'adhésion ;
		\item  au Sénat avec rapport de Monsieur \nom{Voyant} qui suggéra diverses modifications.
	\end{itemize}
	La loi ainsi votée fut complétée par un Règlement d'administration publique qui est en date du 17 mars 1967. La loi comportait 47 articles. Le Décret d'application comportait 65 articles, soit plus que la loi elle-même.

	\subsection{Les buts poursuivis par le legislateur}
		Les auteurs de la loi ont précisé dans l'exposé des motifs qu'ils avaient poursuivi cinq buts :
		\begin{enumerate}
			\item \textbf{Formuler des définitions claires et précises.}
			
			Il est de fait que la loi définit avec précision son objet (art. ler) et qu'elle donne une série de définitions (art. 3 et 4 sur les parties communes et privatives, par exemple).
			
			\item  \textbf{Faciliter la gestion collective de l'immeuble.}
			
			Le Règlement de Copropriété devient obligatoire.
			
			Le Syndicat des Copropriétaires a la personnalité civile.
			
			L'autorité du syndic est en principe accrue.
			
			Le Syndicat bénéficie d'un privilège immobilier et mobilier pour le recouvrement des charges.
			
			\item  \textbf{Garantir les droits des copropriétaires contre les stipulations contractuelles abusives.}
			
			Qu'il s'agisse des droits des copropriétaires sur les parties communes et privatives ou qu'il s'agisse de la répartition des charges de copropriété.
			
			\item  \textbf{Permettre des travaux d'amélioration de l'immeuble conformes a sa destination.}
			
			\item  \textbf{Permettre pour les grands ensembles une division de la copropriété unique d'origine en plusieurs entités nouvelles}.
			
			Les articles 27 et 28 de la loi permettent en effet la création de syndicats secondaires et l'éclatement en copropriétés séparées.
		\end{enumerate}
	
	\subsection{Les principes fondateurs de la loi du 10 juillet 1965}
	
		\subsubsection{L’adoption d’une conception dualiste de la copropriété}
		
			Le texte du 10 juillet 1965 a opté pour la théorie dualiste de la copropriété, plus cohérente avec la conception française du droit de propriété, alors qu'un régime unitaire aurait certainement simplifié les règles de gestion de la Copropriété.
			
			\paragraph{Le régime dualiste}
				On entend par cette expression la coexistence dans la personne du copropriétaire d’un droit de propriété classique sur les « parties privatives », c'est-à-dire le local acheté (<< cube d'air >> selon Mr \nom{Savatier} ou << Vase sans contour >> selon Mr \nom{Mazeaud}) et d’un droit d’une propriété indivise et forcée sur les parties communes.
				
				Cette théorie dualiste permet de sauvegarder la fiction d’une propriété pleine et entière du copropriétaire sur les « parties privatives » comprises dans son lot, dans lesquelles il est censé jouir des mêmes attributs qu’un propriétaire « exclusif » (\emph{usus}, \emph{abusus}, \emph{fructus}), tandis que ses droits sont fortement limités sur les « parties communes », affectés à l’usage et à l’utilité de tous, et dont le sort dépend des décisions collectives prises à l’Assemblée.
			
				Par contre dans le régime dualiste les inconvénients sont nombreux: les Tribunaux auront à choisir, cas par cas, entre l'intérêt général et l'intérêt du copropriétaire et les solutions adoptées seront fort différentes selon qu'ils privilégieront le copropriétaire sur la Collectivité ou inversement.
			
			\paragraph{Le régime unitaire}
				La conception unitaire reconnait au contraire une indivision générale organisée sur l’ensemble de l’immeuble : le copropriétaire ne possède qu'un double droit d'usage, sur une fraction déterminée de l'immeuble (ce que nous appelons les parties privatives) et sur les parties de l'immeuble affectées a la jouissance collective.
				
				Ainsi en Autriche la copropriété par appartements est une simple indivision avec droit réel de jouissance attaché à l'appartement. Conception unitaire également en Suisse (loi fédérale du 19 décembre 1963).
				
				Dans ce dernier pays, aucune distinction n’est faite entre parties communes et parties privatives : les copropriétaires sont propriétaires indivis des unes comme des autres et le règlement de copropriété ne leur attribue qu’un droit de jouissance et d’usage exclusifs sur certaines parties de l’immeuble. << Cette situation permet d’exclure de la copropriété le copropriétaire qui viole les obligations juridiques et morales découlant de l’organisation collective de la copropriété >>\footnote{Copropriété des Immeubles bâtis et ventes d’immeubles à construire au Grand Duché du Luxembourg (Elter et Schockwehler ) – Luxembourg 1978.}
				
				En Allemagne (loi du 15 mars 1951), la conception est plutôt dualiste (propriété privative d’une habitation jointe à une quote-part dans la propriété commune). Cependant, la majorité des propriétaires peut contraindre un copropriétaire qui manque gravement à ses obligations à aliéner son lot.
				
				On conçoit aisément les avantages du système unitaire : l’intérêt collectif primera sur le droit de propriété individuelle.
		
		\subsubsection{L’adoption d’un régime mixte, à la fois contractuel et institutionnel.}
			Le régime de la loi du 10 juillet 1965 est aussi un régime mixte puisqu’il est à la fois contractuel (le Règlement de Copropriété est un contrat) et Institutionnel (la loi impose l'essentiel des règles de fonctionnement.)
			
			Le professeur \nom{LARROUMET}, écrit \footnote{Cité par Madame \nom{Kischinewsky-Broquisse} in La Copropriété des Immeubles bâtis 4\ieme{} Edition, p. 14 \no 32} :
			\begin{quote}
				Le statut de la copropriété a essayé de ménager l'existence de deux principes nécessaires qui peuvent paraître paradoxaux : la liberté individuelle des propriétaires et l'obligation d'assurer une bonne administration commune.
			\end{quote}
			
			On peut effectivement se demander, en pratiquant régulièrement la copropriété et les copropriétaires si ces deux objectifs ne sont pas seulement paradoxaux, mais bien plutôt inconciliables !
		
		\subsubsection{La soumission des copropriétés à des dispositions d’ordre public}

			D'un texte de 10 articles dans la loi de 1938 on passait à 49 articles outre les 65 articles du R.A.P (\textbf{Règlement d'Administration Publique}).
			
			La multiplication de ces articles, le caractère d'\textbf{ordre public} de la plus grande partie des dispositions de la loi vont avoir pour conséquence de placer la copropriété sous la tutelle des Tribunaux.
			
			En effet, aux termes de l'article 43 de la loi du 10 juillet 1965 les articles 6 à 37 de la loi et les textes du Décret du 17 mars 1967 (dans leur rédaction actuelle) pris en application de ces articles 6 à 37 sont d'Ordre Public. Cela signifie que l'on ne peut y déroger, que ce soit dans le Règlement de Copropriété ou par décision d'assemblée générale.
			
			Le texte précise en effet :
			\begin{quote}
				Article 43 de la loi :\\
				\emph{Toutes clauses contraires aux dispositions des articles 6 à 37, 42 et 46 de la loi, et celles du règlement d'administration publique prises pour leur application sont réputées non écrites}.
			\end{quote}
		
			Ces clauses sont non seulement nulles, mais encore elles sont supposées ne pas exister du tout. La différence est importante dans la mesure où la nullité doit être constatée par le juge ; par contre il suffit de ne pas tenir compte de ce qui est réputé non écrit, sans avoir à faire préalablement constater l'inexistence par le juge.
			
			Nous verrons toutefois que la Jurisprudence considère qu’une clause, même contraire aux dispositions impératives de la loi doit recevoir application tant que le juge n’a pas constaté son inexistence.
		
		\subsubsection[Le principe d’unicité des « Copropriétés »]{Le principe d’unicité des « Copropriétés » (absence d’un statut légal adapté pour les « Grands Ensembles Immobiliers. »)}
		
			Si la loi de 1965 est un statut aisément applicable aux petites copropriétés ou aux copropriétés de moyenne importance, elle est par contre totalement inadaptée aux grands ensembles qui comportent un grand nombre de bâtiments.
			
			Lors de la discussion du projet de loi devant l'Assemblée Nationale, Monsieur \nom{Zimmermann} déclarait : << Votre rapporteur pense qu'un avenir prochain contraindra sans doute le législateur à donner aux grands ensembles un statut particulier >>. On attend encore l'adoption de ce statut particulier.
			
			Certes, la loi de 1965 a prévu les palliatifs indiqués plus haut : création de syndicats secondaires ou même scission de la Copropriété. Mais bien souvent les syndicats secondaires ne donnent que peu de solutions aux difficultés de gestion des ensembles et la scission est elle-même très difficile à mettre en œuvre.
			
			Certes, l'article 1er de la loi autorise le promoteur d'une opération immobilière portant sur un ensemble à choisir un statut différent de celui de la Copropriété, mais la réalité des faits permet de constater qu'il est rare de voir exercer cette faculté dans la pratique.
			
			C'est d'ailleurs sur cette question des grands ensembles\footnote{Bien que l'expression << Grands Ensembles >> soit fréquemment utilisée, il conviendrait de ne parler que d'Ensembles Immobiliers en citant les termes de l'article 1er alinéa 2 de la loi. En effet, le législateur ne connaît ni les Grands, ni les Petits Ensembles, il ne connaît que les Ensembles Immobiliers.} que se focalise les réformes à partir des années 1990. Toutefois, faute de trouver un système démocratique et efficace de gestion, le législateur aborde essentiellement ces grands ensembles par le prisme des copropriétés en difficultés.
	
	
	\section{Les adaptations de la loi 65-557 du 10 juillet 1965}
	
		\subsection{La phase de stabilité (1965--1985)}
			La loi du 10 juillet 1965 est restée relativement « stable » au cours des vingt premières années d’existence. Il s’agissait en effet d’un texte cohérent, auquel les lois successives ont apporté des modifications mineures mais non des bouleversements, ces adaptations étant destinées à adapter le statut aux différentes contraintes de la vie économique.
			
			Ex : Loi du 29 octobre 1974 relative aux économies d'énergie,\\
			Loi du 2 janvier 1979 relative aux droits grevant les lots d'un immeuble soumis au statut de la copropriété,\\
			Loi du 21 décembre 1984 sur la domiciliation des entreprises
		
		\subsection{Les lois correctives (1985--2000)}
		
			\subsubsection{Le diagnostic}
				De 1985 à 2000 apparaissent des lois « correctives » qui tirent le constat des dysfonctionnements de certaines copropriétés :
				\begin{itemize}
					\item 	absentéisme (notamment dans les grandes copropriétés), rendant impossible l’adoption de certaines décisions, notamment les modifications du règlement de copropriété et les travaux d’amélioration
					\item  abus « de pouvoir » de certains syndics, qui manipulent les fonds du syndicat des copropriétaires avec un contrôle très restreint du seul conseil syndical, au point que l’on déplore de véritables détournements de fonds
					\item  recouvrement des charges défaillant, car en cas de « faillite » du copropriétaire, le syndicat des copropriétaires n’est pas privilégié par rapport aux autres créanciers, notamment la banque $\dots$
					
					De plus, la plupart des copropriétés fonctionnent en appelant les charges $\dots$ une fois que les dépenses ont été faites (au réel), ce qui génère des a coups de trésorerie.
				\end{itemize}
				Dès les années 1990 apparaissent les « copropriétés en difficultés » (\emph{cf infra}) que l’on tente dans un premier temps de traiter par des mesures préventives, en améliorant la budgétisation des dépenses et le recouvrement des charges.
				Ainsi, la loi du 13 décembre 2000 dite SRU est présentée au Président de la façon suivante :
				\begin{quote}
					Les obstacles au bon fonctionnement du régime des copropriétés sont liés à la mauvaise information des copropriétaires à l'égard de l'état du bien qu'ils acquièrent, à la faiblesse des principes de gestion comptable, au peu d'empressement de certains copropriétaires à honorer leurs obligations financières, enfin, au développement de la précarité qui bloque toute décision d'entretien de l'immeuble et contribue
					non seulement à la dégradation de l'immeuble mais également à l'endettement du syndicat des copropriétaires.
				\end{quote}
				
			\subsubsection{Les lois correctives}
			
				\paragraph{La loi \nom{Bonnemaison} du 31 décembre 1985} : encadrement du syndic.
				
				\begin{itemize}
					\item Le Conseil Syndical est désormais obligatoire, mais les copropriétaires peuvent refuser son institution.
					
					\item  Le syndic doit nécessairement soumettre au vote de l'Assemblée Générale l'ouverture ou non d'un compte séparé ; mais si l'Assemblée n'exige pas ce compte séparé le syndic pourra conserver un compte unique.
					
					\item  Les copropriétaires peuvent prendre individuellement connaissance des comptes du syndic.
					
					\item  Assouplissement des conditions de représentation des copropriétaires aux assemblées générales : désormais, et quelles que soient les mentions du Règlement de Copropriété, toute personne pourra recevoir un mandat.
					
					La majorité de l'article 26 est abaissée des 3/4 des voix (et la moitié en nombre) est abaissée à la majorité des 2/3 des voix (et la moitié en nombre).
					
					\item  Abaissement de majorité pour certains travaux : économies d'énergie, quant à la mise en conformité des logements aux normes de salubrité, de sécurité et d'équipement et quant à la Sécurité des personnes et des biens. Les règles de majorité ont donc été assouplies pour ces décisions.
				\end{itemize}
			
				\paragraph{La loi relative à l'habitat du 21 juillet 1994} : amélioration du recouvrement des charges.
				
				L'obligation de participer aux charges est assortie d'un caractère réel, c'est à dire qu'elle pèse sur le lot et non plus seulement sur le copropriétaire. En sorte que l'acquéreur du lot se trouve de plein droit débiteur des charges.
				
				De la sorte le syndicat de copropriété, en cas de vente du lot sur saisie, se trouve payé par préférence aux autres créanciers inscrits pour les charges échues depuis moins de deux ans et en concurrence avec le prêteur de denier ou le vendeur pour deux autres années.
				
				\paragraph{La loi Solidarité Et Renouvellement Urbain (12 décembre 2000) et ses décrets d’application}.
				
				La loi SRU a modifié assez profondément le statut de la copropriété et les pratiques de gestion pour :
				\begin{itemize}
					\item 	réformer la comptabilité des Syndicat des Copropriétaires pour leur permettre une meilleure gestion de leur trésorerie --- passage aux provisions budgétaires, distinction claire du budget « ordinaire » et du budget « travaux », définition du plan comptable du Syndicat des Copropriétaires sur le modèle de la comptabilité d’entreprise, passage à la comptabilité d’engagement, obligation de présenter le budget et les comptes avec un comparatif de l’année précédente ;
					
					\item  obliger le syndic à ouvrir un compte bancaire séparé, sauf dispense votée à la majorité de l’article 25 ;
					
					\item  faciliter l’entretien du bâti---: travaux obligatoires à la majorité de l’article 24, pose des compteurs d’eau à la majorité de l’article 25, création du « carnet d’entretien » ;
					
					\item  renforcer les moyens de recouvrement des charges, pour prémunir la copropriété de l'absence de financement et sa conséquence, la dégradation du patrimoine --- création d’une procédure de recouvrement spécifique pour les appels budgétaires, devant le Juge des référés statuant au fond, mise à la charge du copropriétaire défaillant des frais de justice.
				\end{itemize}
			
				\subparagraph{Le Décret du 30 mai 2001} impose au syndic de tenir un carnet d’entretien pour chaque syndicat des copropriétaires, outil d’information permettant de connaitre le niveau d’entretien de l’immeuble, tant pour un futur acquéreur que pour les pouvoirs publics.
				
				\subparagraph{Le Décret du 27 mai 2004} :
				\begin{itemize}
					\item  définit les travaux hors budget ;
					
					\item  apporte enfin des définitions (clarification) qui faisaient défaut quant aux charges et provisions ;
					
					\item  permet l’entrée en vigueur de la réforme de la comptabilité, à partir du 1er janvier 2007
				\end{itemize}
				
				\subparagraph{Le Décret relatif aux comptes du syndicat des copropriétaires (14 mars 2005)} crée un plan comptable simplifié pour la copropriété, sauf pour les « petites copropriétés » (moins de 10 lots principaux dont le budget moyen annule est inférieur à 15.000 \euro) du respect de ce plan comptable. Les nouvelles règles comptables sont entrées en vigueur le 1er janvier 2007.
	
	\section{La phase d’instabilité (2000 - aujourd'hui)}
	
		\subsection{La création progressive d’un statut des copropriétés en difficulté}
		
			\subsubsection{Le diagnostic}
			
				Des modifications plus profondes interviennent à partir des années 2000, pour faire face à l’apparition de « copropriétés en difficulté », le plus souvent de grands ensembles urbains qui sont littéralement « en faillite » --- c’est-à-dire dans l’incapacité de faire face au payement des fournisseurs et à la réalisation des travaux nécessaires à la préservation du bâti en appelant des charges supportables par les copropriétaires, et dont le redressement va nécessiter d’importants investissements publics.
				
				A côté de logements sociaux ont été réalisés des immeubles du parc privé dont les acquéreurs, souvent issus d’une population fragile, ont été touchés de plein fouet par la crise économique traversée par la France dans les années 80 et 90.
				
				Ces habitants n’ont pas eu les fonds nécessaires, ni pour payer les emprunts souscrits pour leur acquisition, ni pour acquitter leurs charges de copropriété (principalement chauffage et eau fournis par des sociétés publiques ou parapubliques). De plus il apparaît souvent que les syndics successifs ont mal géré ces immeubles, laissant s’accumuler les impayés sans réagir.
				
				L’environnement s’est dégradé et les conditions locales d’habitation sont allées elles-mêmes en se dégradant (tags, ascenseurs en panne, chauffage collectif défaillant) ; les ventes ou mises en locations font apparaître un changement rapide de la population, et les nouveaux arrivants sont économiquement de plus en plus fragiles (cette fragilité est renforcée par une absence de mixité sociale, voire ethnique dans certains quartiers).
				
				Il y aurait près de \nombre{600 000} logements sur les 6 millions en copropriété qui relevant des dispositions relatives aux syndicats en difficulté ! Il s’agit d’ailleurs le plus souvent de « grandes copropriétés » situées en périphérie des grandes agglomérations, dans les « Quartiers » d’où sont parties les révoltes urbaines de 2005.
				
				Dans les années 2000, \textbf{la copropriété devient un enjeu majeur de la politique de la Ville}. Les pouvoirs publics regardent désormais cette partie du « Parc Privé » avec un oeil attentif pour deux raisons opposées.
				\begin{enumerate}
					\item - Les copropriétés « en difficulté » peuvent nécessiter une intervention publique pour arrêter une spirale de déqualification, et assurer le renouvellement urbain. La politique de « désengagement » total de ce secteur, qui a prévalu dans les années 70 et 80 n’est donc pas soutenable à long terme.
					
					\item Inversement, les copropriétés sont du renouvellement urbain, car des mesures apparemment peu « coercitives » ( par exemple, l’abaissement des majorités pour les travaux de rénovation énergétique, l’obligation de constituer une « épargne » destinée au travaux) peut accélérer de façon radicale le renouvellement des immeubles, sans qu’il soit nécessaire de recourir à des financements publics.
				\end{enumerate}
				
				Dès lors, toutes les lois inspirées par le Ministère du Logement comprendront un volet « Copropriété ».
				
			\subsubsection{Les principales réformes}
			
				Ce statut se construit au fil des réformes législatives, de façon totalement empirique: sont successivement introduites par les textes les dispositions « requises » pour faire face dans tel ou tel grand ensemble !

				\subparagraph{La loi relative à l'habitat du 21 juillet 1994} : création de l’administration provisoire des copropriétés en difficulté
				
				\subparagraph{La loi du 13 décembre 2000 dite SRU} : possibilité de « portage » des lots par la collectivité public, possibilité de procéder à la « scission » des grandes copropriétés pour revenir à des entités plus faciles à gérer et clarifier la domanialité public ou privée.
				
				\subparagraph{La loi du Urbanisme et Habitat du 2 juillet 2003 }: impose de nouvelles et lourdes contraintes pour assurer la sécurité des ascenseurs (loi SAE ou de « Sécurité des Ascenseurs Existants, échéance au 1er janvier 2010) ; facilite la réalisation de travaux d’accessibilité aux handicapés dans les immeubles en copropriété.
				
				\subparagraph{La loi \no 2003-710 du 1er août 2003 dite loi \nom{Borloo}} : possibilité pour le Maire de réaliser des travaux d’office en cas de « carence » du syndicat des copropriétaires (L129-1 CCH) ; expropriation du syndicat des copropriétaires pour carence, aide juridictionnelle accordée aux syndicats en difficulté placés sous administration provisoire pour le recouvrement des charges, faculté pour l’administrateur provisoire de l’article 29-1 de se faire assister par un tiers.
				
				\subparagraph{La loi \no 2006-872 du 13 juillet 2006, dite loi ENL (Engagement national pour le logement)} : création de sanctions pénales pour la mise en copropriété à usage d’habitation d’un immeuble dangereux, insalubre ou inhabitable en vue de créer des locaux d’habitation ; frais de recouvrement des charges mis à la charge du copropriétaire défaillant ; abaissement de la majorité requise pour décider de la fermeture de l’immeuble.
				
				\subparagraph{La loi \no 2009-323 du 25 mars 2009 dite << loi \nom{Boutin} >> ou loi MOLLE}
					\begin{itemize}
						\item Création d’un droit de préemption au profit du syndicat en cas de cession de parkings (pour permettre la « réappropriation » des nappes de parking en banlieue et lutter contre les « véhicules ventouses »).
						
						\item Procédure applicable aux copropriétés en « pré difficulté », mise sous «observation » par un « mandataire ad hoc » lorsqu’à la clôture des comptes, les impayés atteignent 25\% du budget prévisionnel et des dépenses pour travaux non compris dans le budget prévisionnel (décret du 20 avril 2010).
						
						\item Élargissement des pouvoirs de l’administrateur judiciaire
					\end{itemize}
				
				\subparagraph{La loi égalité et citoyenneté du 27 janvier 2017 } complète les dispositions sur les copropriétés en difficultés (renforcement des pouvoirs de l’administrateur).
				
		
		\subsection{Lutte contre le réchauffement climatique et transition énergétique}
		
		\subsubsection{Diagnostic}
			La loi ENE du 12 juillet 2010 fixe, en 257 articles, de nouvelles règles environnementales et de performance énergétique dans différents domaines (urbanisme, bâtiment, transports, climats…), afin de remplir les objectifs fixés par la loi Grenelle I (loi de programmation du 3 août 2009), notamment l’objectif général dit « facteur 4 », la division par 4 des émissions de gaz à effet de serre (GES), avec un horizon fixé à l’époque à 2020, repoussé depuis à 2050.
			
			Le secteur du bâtiment consomme actuellement 68,2 millions de tonnes d'équivalent pétrole, soit \pourcent{42,5} de l'énergie finale total.Il génère parallèlement 123 millions de tonnes de \dioxydeDeCarbone, soit \pourcent{23} des émissions nationales. L’énergie est consommée pour deux tiers dans les logements et pour un tiers dans le secteur tertiaire. Cette proportion reste sensiblement constante depuis vingt ans.
			
			Or le taux de renouvellement du parc existant est extrêmement faible : le taux de renouvellement des bâtiments anciens par des bâtiments neufs est inférieur à \pourcent{1} par an. Sans effort supplémentaire réalisé, ce faible taux, associé au rythme actuel des réhabilitations n'entraînerait un relèvement des performances énergétiques de la totalité des bâtiments construits avant 1975 que dans plus d'un siècle.
			
			Il faut donc prévoir un investissement massif en rénovation énergétique des bâtiments anciens : le coût global des travaux d'économie d'énergie réalisés dans le secteur du bâtiment serait de \nombre{1 000} milliards d’euros d’ici 2050\footnote{Le coût des rénovations énergétiques à réaliser pour parvenir au facteur 4 est de \montant{200} à \montant{400} du \metreCarre, selon la typologie du bâtiment, et les résidences principales représentent une surface d'environ 2,65 milliards de mètres carrés. Aujourd’hui, le coût moyen des rénovations énergétiques effectuées est de \prixSurface{125} seulement (renouvellement de chaudières, isolation des fenêtres)}.
			Les deux tiers de cet investissement concerneraient les ménages pour des montants qui seraient au moins de \montant{20 000} et pourraient dépasser les \montant{40 000}, à investir dans leurs logements en trois à quatre décennies.
			
			L'enjeu est important pour les copropriétés, car les charges de chauffage représentent environ \pourcent{70} du budget prévisionnel d'un syndicat de copropriétaires, et 4,7 millions de logements sur les 8 millions en copropriété sont chauffés collectivement
			
		\subsubsection{Les principales reformes}
		
			\paragraph{La loi \no 2010-788 du 12 juillet 2010 (ENE) dite Grenelle \textsc{ii}} est la première d’une longue série de textes ayant pour objectif de faciliter la réalisation des travaux d’économie d’énergie en copropriété, car il est plus aisé d’imposer cette rénovation énergétique dans des immeubles collectifs administrés par un professionnel que dans l’habitat individuel.
				
			Elle impose :
			\begin{enumerate}[label=\alph*)]
				\item la réalisation obligatoire d’un DPE ou d’un diagnostic énergétique dans les tous les immeubles équipés d’un dispositif commun de chauffage ou de refroidissement (L 134-4-1 du CCH), et dans les immeubles en copropriété à usage principal d’habitation de plus de 50 lots, dont le dépôt de la demande de permis de construire est antérieur au 1er janvier 2001, l’obligation de réaliser un véritable audit énergétique comprenant des préconisations de travaux ;
				
				\item l’obligation de soumettre à l’assemblée un plan de travaux d’économies d’énergie ou contrat de performance énergétique (CPE), à l’issue du DPE ;
				
				\item l’abaissement de la majorité requise pour les travaux d'économies d'énergie ou de réduction des émissions de gaz à effet de serre, soumis à l’article 25 avec un second vote possible en 25-1 (art.25 f de la Loi 65-557 du 10 juillet 1965)
				
				\item la possibilité d’imposer au copropriétaire, sur la partie privative de son lot des travaux de rénovation énergétique « d’intérêt collectif » sauf dans le cas où ce dernier est en mesure de produire la preuve de la réalisation de travaux équivalents dans les dix années précédentes (\emph{idem}) ;
				
				\item  la création d’un « droit à la prise » pour les installations l’alimentation des véhicules électriques (IRVE) : le copropriétaire peut réaliser ces travaux sans autorisation préalable d’assemblée générale, sauf pour le syndic à saisir le Tribunal pour s’y opposer (art. 24-5 de la Loi 65-557 du 10 juillet 1965 et L 111-6-4 du CCH).
			\end{enumerate}
			
			\paragraph{La loi \no 2015-992 du 17 aout 2015 sur la Transition Energétique et la Croissance Verte, dite loi TECV} votée 3 mois avant la COP 21, prévoit de nouvelles obligations destinées à accélérer la rénovation énergétique des copropriétés.
			
			\subparagraph{Obligation de rénovation énergétique avant 2025 pour les logements en F ou G}
			« Avant 2025, tous les bâtiments privés résidentiels dont la consommation en énergie primaire est supérieure à 330 kilowattheures d’énergie primaire par mètre carré et par an doivent avoir fait l’objet d’une rénovation énergétique. »
			
			Renforcement de l’obligation de pose de compteurs individuels de chaleur, avec sanction pénale( allant jusqu’à \montant{1 500} par an et par logement)
			
			\subparagraph{Obligation d’embarquer la performance énergétique dans les travaux d’entretien et de rénovation.}
			Dès parution du décret d’application, les copropriétés auront l’obligation (sauf cas exceptionnels) d’embarquer l’amélioration de performance énergétique en cas de travaux sur le clos et le couvert (ravalement de façade, réfection de toiture…). Les matériaux d’isolation mis en oeuvre « doivent permettre d’atteindre, en une ou plusieurs étapes », les performances d’un bâtiment basse consommation (étiquette A ou B). Le vote de ces travaux relèvera de l’article 24 de la loi du 10.07.65 (majorité relative)
			Cette disposition a fait l’objet d’un Décret d’application \no 2016-711 du 30 mai 2016 relatif aux travaux d'isolation en cas de travaux de ravalement de façade, de réfection de toiture ou d'aménagement de locaux en vue de les rendre habitables
			De même la loi comporte un renforcement de la réglementation en terme de performance énergétique des équipements dès lors que les équipements collectifs doivent être remplacés, réparés ou installés et notamment les chaudières collectives ou les équipements collectifs de refroidissement
			Possibilité de déroger aux règles d’urbanisme pour isoler thermiquement les copropriétés
			L’article 7 de la Loi de Transition Énergétique prévoit la possibilité de déroger aux règles des PLU (relatives à l’emprise au sol, à la hauteur, à l’implantation et à l’aspect extérieur des constructions) (Plan Locaux d’Urbanisme), pour les travaux suivants : Isolation thermique par l’extérieur en saillie des façades, isolation par surélévation des toitures, installation de protections solaires en saillie des façades (en attente du décret)
			Nouvelles règles sur le tiers financement
			Réservation de places de stationnement aménagées pour les véhicules électriques ou hybrides rechargeables.
			Obligation de pose de compteurs de chaleur : obligation renforcée, le syndic a l’obligation d’inscrire la question à l’ordre du jour et des sanctions financières sont prévues pour la 1ere fois (article L 241-9 du Code de l’Energie) :
			L’étendue de cette obligation a été précisée par le Décret \no  2016-710 du 30 mai 2016 relatif à la détermination individuelle de la quantité de chaleur consommée et à la répartition des frais de chauffage dans les immeubles collectifs
			Modernisation du carnet d’entretien de l’immeuble : il devra désormais être numérique et donner la part belle à la mesure de la performance énergétique. Pour commencer il deviendra obligatoire pour tous les logements neufs dont le permis de construire a été déposé après le 1er janvier 2017 et puis il sera progressivement étendu à l’ensemble du parc immobilier français.
			Ce texte a fait l’objet du Décret d’application \no  2016-1965 du 28 décembre 2016.
			
		\subsubsection{Incidences collaterales de l’acceleration des reformes « connexes »}
		
			\paragraph{La loi \no  2015-1776 du 28 décembre 2015 relative à l'adaptation de la société au vieillissement.}
			
			Cette loi modifie le titre IV bis de la loi du 10 juillet 1965 relatif aux résidences services en copropriété.
			Selon le nouvel article 41-1 de la loi du 10 juillet 1965 « Le règlement de copropriété peut étendre l'objet d'un syndicat de copropriétaires à la fourniture aux résidents de l'immeuble de services spécifiques dont les catégories sont précisées par décret et qui, du fait qu'ils bénéficient par nature à l'ensemble de ses résidents, ne peuvent être individualisés. Les services non individualisables sont fournis en exécution de conventions conclues avec des tiers. Les charges relatives à ces services sont réparties en application du premier alinéa de l'article 10.(…) »
			
			Le décret \no  2016-1737 du 14 décembre 2016 relatif aux résidences-services en copropriété définit limitativement les catégories de services non individualisables : accueil permanent, veille et surveillance, accès aux espaces de convivialité.
			
			\paragraph{Ordonnance \no  2016-131 du 10 février 2016 portant réforme du droit des contrats, du régime général et de la preuve des obligations}
			
			Le Règlement de copropriété est un contrat, quand bien même il est de nature institutionnel. Il est donc soumis aux règles du code civil Livre Troisième Titre III relatives aux Contrats et Obligations et notamment quant à l’effet des obligations ou encore de l’interprétation des conventions.
			
			L’Ordonnance du 10 février 2016 n’apporte pas de bouleversements aux dispositions légales d’origine et s’apparente davantage – sauf volonté contraire du texte - à une mise à jour au regard de la jurisprudence qui s’est forgée sur plus de deux siècles en application du code civil. Toutefois les innovations sont nombreuses, notamment la disparition de la notion de cause du contrat au profit du contenu du contrat, la non-rétroactivité de la clause suspensive, la procédure interrogatoire de l’article 1183 (si un contractant craint une action en annulation du contrat il peut intimer au co-contractant d’introduire son action dans le délai de six mois, à peine de forclusion)
		
			L’impact de cette réforme sera nécessairement limité en droit de la copropriété car :
			\begin{itemize}
				\item le texte entrera en vigueur le 1er octobre 2016 et tous les contrats (donc les règlements) antérieurs à cette date resteront soumis à «l‘ancien droit » (article 9 de l’Ordonnance du 10 février 2016) ;
				\item l’article 1105 nouveau précise que les règles générales des contrats s’appliquent sous réserve des règles particulières (\emph{specialia genralibus derogant}).
			\end{itemize}
			
			Cette observation n’est pas neutre s’agissant par exemple de la sanction aux infractions commises en violation du Règlement de copropriété.
			L’article 1143 actuel du code civil édicte que « le créancier est en droit de demander que ce qui a été fait par contravention à l’engagement soit détruit et l’on connaît la jurisprudence en application de ce texte qui veut que ces dispositions s’imposent au juge.
			Par exemple l’arrêt 3\ieme{} Chambre civile du 18 janvier 1972, Bull. civ. III, \no  30, p. 28 : « Violent l’article 1143 les juges du fond qui refusent d’ordonner la mise en état des lieux alors qu’ils constatent une infraction à une disposition du Règlement de copropriété »\footnote{
			Cette jurisprudence est d’application particulièrement stricte en matière de lotissement où le non respect du cahier des charges permet à tout co-loti de demander la démolition de l’ouvrage dès lors que cette démolition est possible, sans aucune référence à la disproportionnalité entre l’infraction au contrat et la sanction de démolition (civ. 3\ieme{} ch.,21 janvier 2016, \no  15-10566 – au Bulletin).
			}.
			Or, le nouvel Article 1221 est ainsi rédigé : « Le créancier d'une obligation peut, après mise en demeure, en poursuivre l'exécution en nature sauf si cette exécution est impossible ou s'il existe une disproportion manifeste entre son coût pour le débiteur et son intérêt pour le créancier ».
			$\dots$ sauf pour les juges à décider d’appliquer le nouveau texte, par exemple en matière d’interprétation des contrats. Il semble au demeurant que dès à présent la Cour de cassation tende à modérer sa rigueur comme en atteste par exemple un arrêt de la 3\ieme{} Chambre civile du 21 janvier 2016 (pourvoi \no  14-26085) qui a considéré que le juge après avoir annulé un contrat de construction d’une maison individuelle n’était pas tenu d’ordonner cette démolition dès lors que le bénéficiaire de la décision n’avait demandé que des Dommages et Intérêts en réparation de son préjudice tenant à l’annulation du contrat. En l’espèce le constructeur avait demandé au juge d’appel de dire que la résolution du Contrat ne permettait au maître de l’ouvrage que de demander la démolition et la restitution des fonds versés mais ne leur permettait pas de garder la maison et d’obtenir des Dommages et Intérêts.
			A ce sujet, rappelons que lorsque la démolition est demandée non pas devant le juge du fond, mais en référé, sur le fondement de l’article 809 du code de procédure civile, le juge des référés dispose d’un pouvoir d’appréciation (il peut prescrire la démolition ; il n’y est pas obligé).
			Par contre, et bien évidemment, le syndicat des copropriétaires ayant la personnalité morale peut passer tous les contrats conforme à son objet et à ce titre, à compter du 1er octobre 2016 il devra bien évidemment se soumettre au nouveau droit. A ce propos, relevons la nouvelle rédaction de l’article 1145 : « La capacité des personnes morales est limitée aux actes utiles à la réalisation de leur objet tel que défini par leurs statuts et aux actes qui leur sont accessoires, dans le respect des règles applicables à chacune d'entre elles. »
			
			\paragraph{Ordonnance \no  2016-301 du 14 mars 2016 : la refonte du code de la consommation}

			Dans un article liminaire le non-professionnel est défini comme « toute personne morale qui agit à des fins qui n’entrent pas dans le cadre de son activité commerciale, industrielle, artisanale, libérale ou agricole ».
			Le syndicat des copropriétaires est un « non professionnel » et donc soumis aux dispositions concernant « les non professionnels », telles que la faculté de dénoncer un contrat reconduit par tacite reconduction, en l’absence de dénonciation de la faculté de résiliation par le professionnel
			On trouve dans ce code, dans un livre dédié (Livre V – Pouvoirs d’enquête et suites données aux contrôles), les procédures et pouvoirs de la DGCCRF, notamment les dispositions des articles L 511-3 et suivants relatives aux Agents de la concurrence, de la consommation et de la répression des fraudes qui « sont habilités à rechercher et constater les infractions ou les manquements aux dispositions relatives aux honoraires du syndic et au contrat type ( art L 511-7 du Code cons)
		
	\section[La loi \textsc{alur} et ses suites]{La loi \textsc{alur} (loi \no  2014-366 du 24 mars 2014 pour l’accès au logement et a un urbanisme rénové) et ses suites}
	
		\subsection{Les principaux axes de la reformes}
			La réforme dite « loi \nom{Duflot} » du nom du ministre du logement, comme les textes importants de ces 30 dernières années est élaborés par le Ministère du logement et de l'urbanisme, n'a pas pour objectif unique la réforme du droit de la copropriété. Bien au contraire son objet essentiel est de régler les relations entre propriétaires et locataires pour « permettre l'accroissement de l'offre de logements ».
			
			La réforme du droit de la copropriété est introduite dans le titre II intitulé : « Lutter contre l’habitat indigne et les copropriétés dégradées », et largement inspiré du rapport de Dominique Braye, directeur de l’ANAH, intitulé « Prévenir et guérir les copropriétés en difficultés ». C’est donc avant tout une réforme axée, une nouvelle fois, sur les pathologies des copropriétés, devenues un enjeu majeur de la politique de la Ville. Il ne s’agit plus, toutefois, de rénover une nouvelle fois le régime des copropriétés en difficulté ( bien que la loi comporte des dispositions très innovantes sur le sujet), mais de réécrire l’ensemble de la loi pour en combattre les causes, qui résultent, selon le rapport BRAYE, d’une combinaison de facteurs institutionnels, économiques et sociaux.
			
			La loi repose donc sur ces trois piliers :
			\begin{itemize}
				\item \textbf{Institutionnel}, avec un contrôle accru du syndic, et un nouvel abaissement des seuils de majorités, notamment pour les travaux ;
				\item \textbf{Économique}, avec un élargissement et une protection accrue des ressources du syndicat ( compte séparé, fond de travaux), ainsi qu’une meilleure programmation des travaux ;
				\item \textbf{Social}, ce volet passant par la lutte contre les acquéreurs déstabilisateurs, l’amélioration de l’information de l’acquéreur et un meilleur recensement des copropriétés pour permettre l’action publique très en amont.
			\end{itemize}
			
			La réforme se caractérise par un double éclatement du statut de la copropriété.
			\begin{itemize}
				\item L’éclatement des dispositions s’accentue, entre la Loi du 10 juillet 1965 et le CCH, qui comporte désormais un livre VII consacré aux immeubles relevant du statut de la copropriété, qui traite de l’immatriculation des syndicats (titre I) – des dispositions	relatives à l’information des acquéreurs (titre II) de l’entretien, de la conservation et de la l’amélioration des immeubles (titre III).
				\item Plusieurs régimes juridiques distincts apparaissent en fonction du nombre de lots compris dans un syndicat de copropriétaires, des modalités d’occupation du syndicat de copropriété (logement seul, destination partielle ou totale d’habitation, syndicats de copropriétaires composés uniquement de personnes morales).
			\end{itemize}
			
			Cette réforme est sans doute la plus complète depuis la rédaction du texte d’origine, malheureusement elle conserve une approche éclatée, alors qu’une véritable « refonte » du statut serait sans doute nécessaire.
		
		\subsection{Les dispositions essentielles de la reforme}
		
			\subsubsection{Recensement, prévention et traitement des copropriétés en difficultés}
				\begin{enumerate}
					\item la création d'un registre national des copropriétés dont le but à terme est que tout syndicat de copropriété soit immatriculé et fournisse d'importants éléments d'information sur sa situation juridique et financière (entrée en vigueur progressive, à compter du 31 décembre 2016)
					\item la création d’une « fiche synthétique » pour chaque copropriété
					\item la réécriture des dispositions relatives aux copropriétés en pré-difficulté et en difficulté avec l'accentuation de la faculté pour les pouvoirs publics d'intervenir dans le parc privé. Le texte permet un véritable « rétablissement » des copropriétés en difficulté par une procédure de recensement des créanciers, d'établissement d'un plan d'apurement des dettes sur cinq ans avec l'effacement partiel de certaines dettes.
				\end{enumerate}
			
			\subsubsection{Nouvelles formalités préalables a la vente du lot}
				\begin{enumerate}
					\item Information sommaires dès l’annonce immobilière
					\item Information lors de la Promesse de vente : la remise du règlement de copropriété, de l’état descriptif de division, des procès verbaux d’assemblées générales et du « pré-état daté » subordonne le point de départ du délai de rétractation pour l’acquéreur.
					\item certains acquéreurs « déstabilisateurs » sont tous simplement frappés d’une interdiction d’acquérir (marchand de sommeil, copropriétaire débiteur déjà titulaire d’un lot dans la copropriété)
					\item Un diagnostic technique global remplace en outre le DPE.
				\end{enumerate}
			
			\subsubsection{Encadrement plus rigoureux du syndic}
				\begin{enumerate}
					\item Modification des règles professionnelles (Loi Hoguet) : nouvelles règles sur la délivrance de la carte professionnelle ( devenue la carte S délivrée par les CCI), sur la formation continue, la déontologie, et création du conseil national de la transaction et de la gestion immobilières et de la commission de contrôle des activités de transaction et de gestion immobilières
					\item Mise en concurrence obligatoire du syndic par le conseil syndical ( tous les ans selon la loi ALUR, puis tous les 3 ans selon la loi Macron)
					\item Principe de forfaitisation des honoraires et renvoi à un « contrat type » à définir par décret
					\item l'augmentation des tâches incombant au syndic, dont la mise en concurrence devient obligatoire, et la rémunération forfaitaire
					\item L’obligation, sans dérogation possible (sauf pour les petites copropriétés de moins de 10 lots et <15.000 euros de budget), pour le syndic de remettre les fonds reçoit sur un compte bancaire ouvert au nom du syndicat des copropriétaire.
					\item Encadrement de la fin de mandat ( préavis en cas de démission, règles relatives à l’empêchement), interdiction de siéger au conseil syndical faite aux collatéraux du syndic, modification des règles concernant les « conventions réglementée »
				\end{enumerate}
			
			\subsubsection{dispositions destinées a faciliter l’adoption des travaux}
				\begin{enumerate}
					\item Création du Diagnostic Technique Global, qui se substitue au DPE
					\item Institution d’un fonds de prévoyance obligatoire, intitulé « fonds de travaux » correspondant à 5 \% au moins du budget annuel, définitivement acquis au syndicat, qui doit être abondé tant que le plafond des travaux préconisés par le DTG n’est pas atteint
					\item Modification substantielle des règles de majorité aux assemblées générales. Les travaux d'amélioration relèveront désormais de la majorité de l'article 25 de la loi tandis que les travaux obligatoires relèveront de la majorité de l'article 24.
					\item Nouvelles dispositions destinées à favoriser la surélévation
				\end{enumerate}
			
			\subsubsection{Autres dispositions destinées a fluidifier le fonctionnement du syndicat}
			\begin{enumerate}
				\item Améliorations relatives à l’assemblée : convocation par tout copropriétaire en cas d’absence de syndic, convocation électronique ( entrée en vigueur différée jusqu’au décret)
				\item Amélioration de la transparence : extranet (à compter du 1er janvier 2016), nouvelles modalités de consultation des pièces comptable et d’information des résidents (entrée en vigueur suspendues aux décrets pour ces deux derniers points)
				\item la modification des règles de représentation des copropriétaires aux assemblées générales dans les grands ensembles (syndicats secondaires et ASL).
				\item la consécration de la division en volumes avec la faculté pour les copropriétés, sous certaines conditions, de faire une scission en volumes.
			\end{enumerate}
			
		\subsection{Les principaux décrets d’application de la loi \textsc{alur}}	
			La loi ALUR est entrée en vigueur pour partie de façon immédiate, pour partie de façon différée (entre le 1er juillet 2016 et le 31 décembre 2018 notamment) , compte tenu du temps nécessaire à la mise en place concrète de la réforme (nouveau contrat de syndic, compte bancaire séparé obligatoire, immatriculation des syndicats, fond de travaux).
			Mais surtout, la mise en oeuvre de nombreuses disposition est subordonnée à l’entrée en vigueur de plusieurs dizaines de décrets d’application, dont le calendrier de publication a pris beaucoup de retard\footnote{Sur l’ensemble du sujet voir Informations Rapides de la Copropriété \no  604, décembre 2014, Les impacts de la loi ALUR sur le régime de la copropriété ; Actes Pratiques et Ingéniérie Immobilière (revue du JurisClasseur) juin 2014 La Loi ALUR ; Administrer Juin et septembre 2014 (Commentaires Capoulade).}. 

			\paragraph{Décret \no 2014-843 du 25/07/2014} Composition et les modalités de constitution et de fonctionnement du conseil national de la transaction et de la gestion immobilières.
			
			\paragraph{Décret \no 2015-342 du 26/03/2015} Contrat type à respecter par le contrat de syndic (entrée en vigueur pour les mandats conclus à compter du 1er juillet 2015).
			
			\paragraph{Décret \no 2015-518 du 11/05/2015} Assurance obligatoire du syndicat des copropriétaires (fonctionnement du BECT, montant de la prime et de la franchise)
			
			\paragraph{Décret \no 2015-702 du 19/06/2015} Carte professionnelle : délivrance par le président de la chambre de commerce et d’industrie territoriale ou par le président de la chambre de commerce et d’industrie départementale d’Île-de-France.
			
			\paragraph{Décret \no 2015-999 du 17/08/2015} Modalités d’intervention et désignation du mandataire ad hoc (copropriétés en pré-difficulté), nouvelles règles relatives à la liquidation du syndicat.
			
			\paragraph{Décret \no 2015-1090 du 28/08/2015} Règles constituant le code de déontologie applicable aux personnes exerçant des activités d'entremise et de gestion des immeubles et fonds de commerce ( mais la commission de contrôle de la gestion immobilières n’est pas encore constituée).
			
			\paragraph{Décret \no 2015-1325 du 21/10/2015} relatif à la dématérialisation des notifications et des mises en demeure concernant les immeubles soumis au statut de la copropriété des immeubles bâtis.
			
			\paragraph{Décret \no 2015-1681 du 15/12/2015} Information des occupants de chaque immeuble de la copropriété des décisions prises par l’assemblée générale (le décret s'applique aux assemblées générales convoquées à compter du 1er avril 2016 : obligation d’un affichage destiné aux résidents, dans les parties communes, relatif aux décisions d’assemblée générale)
			
			\paragraph{Décret \no 2015-1907 du 30/12/2015} Modalités de consultation des justificatifs de charges ( au moins un jour ouvré, fixé par le syndic dans la convocation, avec faculté de prendre copie des pièces aux frais du copropriétaire).
			
			\paragraph{Décret \no 2016-173 du 18/02/ 2016} relatif à la formation continue des professionnels de l’immobilier : au moins 14 heures par an (dont 2 heures consacrées à la déontologie) et de 42 heures sur trois ans
			
			\paragraph{Décret \no 2016-1167 du 26 août 2016} relatif au registre national d'immatriculation des syndicats de copropriétaires Et Arrêté du 10 octobre 2016 sur immatriculation
			
		\subsection{Le detricotage de la loi \textsc{alur}}
			Certaines dispositions de la loi ALUR vont faire l’objet, en moins de 18 mois, de « correctifs »
			
			\paragraph{Loi \no 2014-1545 du 20/12/2014} Le texte revient sur l’obligation, prévue par la loi ALUR, de mentionner dans le certificat \nom{Carrez} (art.46 de la loi du 10.07.65) la superficie de la partie privative et la surface habitable, après avoir constaté que ces deux superficies $\dots$ sont identiques !
			
			\paragraph{Loi \no 2015-990 du 6 août 2015} pour la croissance, l'activité et l'égalité des chances économiques, dite Loi \nom{Macron}, modifiant les conditions de mise en concurrence du syndic (tous les 3 ans, avec une possibilité de dispense par l’assemblée de l’année précédant cette mise en concurrence obligatoire)

			\paragraph{Loi \no 2016-1321 du 7 octobre 2016} pour une République numérique habilite le gouvernement pour prendre par « ordonnance toute mesure relevant du domaine de la loi afin de favoriser la dématérialisation par le développement de l'envoi de documents par voie électronique, de l'usage de la signature électronique et de la lettre recommandée électronique dans les relations entre les personnes soumises à la loi \no  65-557 du 10 juillet 1965 fixant le statut de la copropriété des immeubles bâtis. »
	
	\section{La loi \no 2018-1021 du 23 novembre 2018 dite elan}
		\subsection{La loi \no 2018-1021 du 23 novembre 2018 dite elan}
			Le Gouvernement a fixé en 2018 plusieurs axes d'une politique de logement renouvelée, quitte à « bousculer » la loi ALUR :
			\begin{itemize}
				\item construire plus, mieux et moins cher pour provoquer un choc d'offres ;
				\item  accompagner l'évolution du secteur du logement social ;
				\item répondre aux besoins de chacun et favoriser la mixité sociale ;
				\item améliorer le cadre de vie.
			\end{itemize}
			Dans ce but il a rédigé un projet de loi « portant \textbf{évolution} du \textbf{logement}, de l’\textbf{aménagement} et du \textbf{numérique} » qui se veut une nouvelle approche de la politique du logement en France devant « conduire à davantage d’équilibre territorial et de justice sociale en faveur des plus fragiles, être un moteur durable de l’économie locale comme nationale et un vecteur d’innovation ».
			
			Ce projet a envisagé « des mesures en faveur de l’amélioration de la gouvernance des copropriétés pour remédier au constat d’un relatif vieillissement de la loi du 10 juillet 1965 ($\dots$) qui induit certaines rigidités dans la gouvernance et les modalités de décision et peut retarder par exemple la nécessaire rénovation énergétique des bâtiments »\footnote{Présentation du Projet de Loi, exposé des motifs}.
			Il convient d’observer que l’adoption de cette loi par le Parlement, après que l’Urgence ait été déclarée, a été précédée d’une « conférence de consensus sur le logement » en présence des principaux acteurs qui s’est déroulée au Sénat entre décembre 2017 et février 2018.
			Les dispositions finalement adoptées sur la loi de 1965 sont de deux ordres :
			\begin{enumerate}[label=\roman*)]
				\item En premier lieu, l’habilitation donnée au gouvernement pour légiférer par ordonnance (article 214 dans la Loi)
				\begin{itemize}
					\item dans le délai d’un an de la promulgation de la loi ELAN, à prendre par voie d’ordonnance les mesures visant à redéfinir le champ d’application de la loi de 1965 ainsi qu’à « Clarifier, moderniser, simplifier et adapter les règles d’organisation et de gouvernance de la copropriété, celles relatives à la prise de décision par le syndicat des copropriétaires ainsi que les droits et obligations des copropriétaires, du syndicat des copropriétaires, du conseil syndical et du syndic » ;
					\item dans un délai de deux ans, à procéder par voie d’ordonnance pour codifier la partie législative d’un code relatif à la copropriété ;
					\item  Un projet de loi de ratification devra être déposé devant le Parlement dans un délai de trois mois à compter de la publication de chaque ordonnance
				\end{itemize}
				\item  En second lieu, les articles 203 à 213, apportent dès à présent des modifications substantielles à différents articles de la loi de 65\footnote{La question est de savoir si ces nouvelles dispositions vont s’imposer au gouvernement dans le cadre de la refonte de la loi de 1965 ou si celui-ci sera libre de supprimer ou de modifier ces modifications. En principe le gouvernement n’est pas tenu de respecter ces nouvelles dispositions des articles 203 à 213, mais sa position serait inconfortable lorsqu’il demandera dans les trois mois suivant la promulgation de chaque ordonnance.}, ces modifications ayant été ajoutées par le Sénat au projet du gouvernement : le Sénat étant hostile dans un premier temps à la mise en œuvre de la procédure relative aux ordonnances gouvernementales. Certaines sont d’applications immédiates. Ces dispositions comprennent :
				\begin{itemize}
					\item la réécriture des premiers articles de la loi (intégration de la jurisprudence sur le lot de copropriété, le lot transitoire, les Parties communes spéciales, les Parties communes à jouissance privative, date d’entrée en vigueur du statut en cas de VEFA), avec une obligation de mise à jour des règlements de copropriété dans les 3 ans
					\item  des dispositions destinées à faciliter le recouvrement des charges (déchéance du terme) et l’utilisation du fond de travaux
					\item  des dispositions relatives à l’assemblée (délégations de vote --- nombre et subdélégation, interdictions étendues pour la prise de mandat, visioconférence et vote par correspondance, délai de notification des PV réduit à un mois) ;
					\item  la « remontée » à l’article 25 de la majorité pour les travaux de rénovation énergétique
					\item  des dispositions relatives à la numérisation et à l’accès à l’information : Sanction en cas de défaut de transmission des documents au conseil syndical, Contenu de l’extranet, Réception des coordonnées des locataires par le syndic, carnet numérique
					\item  des dispositions diverses : alignement de la prescription des actions en copropriété sur le droit commun (5 ans), intégration des colonnes montantes au domaine public, renforcement du dispositif de lutte contre les locations touristiques meublées), Individualisation des frais de chauffage et de refroidissement.
				\end{itemize}
			\end{enumerate}

		\subsection{Les decrets d’application de la loi \no  2018-1021 du 23 novembre 2018 dite elan}
			\paragraph{Décret \no 2019-502 du 23 mai 2019} relatif à la liste minimale des documents dématérialisés concernant la copropriété accessibles sur un espace sécurisé en ligne.
			
			L’Intranet obligatoire doit être sécurisé (code personnel), avec des documents téléchargeables et imprimables, actualisés au minimum une fois par an par le syndic, dans les trois mois précédant l'assemblée générale annuelle.
			
			Les documents mis en ligne doivent permettre à chaque copropriétaire fournir les documents nécessaires à une promesse de vente (règlement de copropriété, procès-verbal des 3 dernières années), d’accéder aux principaux contrats ( syndic, assurance) et au conseil syndical d’exercer son contrôle (balance des comptes, relevé du compte bancaire, assignation).
			
			\paragraph{Décret \no 2019-503 du 23 mai 2019} fixant le montant minimal de pénalités applicables au syndic en cas d’absence de communication des pièces au conseil syndical (15 euros).
			
			\paragraph{Décret \no  2019-650 du 27 juin 2019} portant diverses mesures relatives au fonctionnement des copropriétés et à l'accès des huissiers de justice aux parties communes d'immeubles.
			Ce décret précise les modalités de la « dématérialisation » des actes en copropriété :
			\begin{itemize}
				\item Notifications Electroniques
				\item Dématérialisations des annexes de l’assemblée générale
				\item Dématérialisation des appels de fonds
			\end{itemize}
			Le décret précise également que si un huissier doit accéder aux parties communes, le syndic doit lui fournir les codes ou le moyen d’accès dans les 5 jours de sa demande
		
		\subsection{L’ordonnance \no  2019-1101 du 30 octobre 2019 portant reforme du droit de la copropriete des immeubles batis}
			L’Ordonnance \no  2019-1101 du 30 octobre 2019 portant réforme du droit de la copropriété des immeubles bâtis (publiée au Jo du 31.10.2019, avec un rapport au Président de la République), entrera en vigueur le 30 juin 2020.
			Cette ordonnance a été adoptée après consultation des professionnels, des associations de consommateurs et des « corps constitués » (notaires, avocats $\dots$).
			Elle s’affiche clairement comme la réforme la plus ambitieuse du statut depuis 1965, en tout cas la première conçue comme réformant la totalité du statut, et consacrée uniquement à la copropriété.
			Elle ne constitue pas une révolution, mais plutôt une évolution du statut, destiné à le moderniser en en conservant les caractéristiques essentielles.

			Selon le Rapport, la genèse la réforme s’explique ainsi :
			\begin{quote}
				En 2015, le 50e anniversaire de la loi du 10 juillet 1965 a été l’occasion pour la doctrine et les praticiens de s’interroger sur les difficultés d’application du statut de la copropriété des immeubles bâtis.
				Il a alors été mis en lumière la nécessité de préserver ce système original, fondé sur des grands principes issus du droit des biens (propriété, indivision), du droit des personnes (personnalité morale du syndicat des copropriétaires, dotée d’un « patrimoine ») et du droit des obligations (paiement des charges), qui a inspiré de nombreux pays (Québec, Belgique, Algérie, Côte d’Ivoire, Haïti, etc.). Dans le même temps, ont également été soulignées les limites de ce statut, certes protecteur, mais n’offrant pas la possibilité d’une adaptation aux spécificités de chaque copropriété, qu’il s’agisse notamment de leur taille, de leur structure ou de leur destination. [$\dots$]
				A enfin été soulignée la nécessité d’une meilleure prise en compte par les copropriétaires de la dimension collective de la copropriété et de la nécessité de préserver leur patrimoine commun, en anticipant la réalisation de travaux indispensables à la conservation de leur immeuble.
			\end{quote}
			
			Aussi, les principales mesures de la Réforme sont-elles :
			\begin{itemize}
				\item La modernisation du statut par une redéfinition du champ d’application de la loi du 10 juillet 1965 au regard des caractéristiques, de la destination ou de la taille des immeubles ainsi que des règles applicables à ces copropriétés. Ainsi, le statut de la copropriété n’est plus obligatoire pour les immeubles autres que d’habitation, et il est en outre prévu un statut très dérogatoire pour les « petites copropriétés » (au plus cinq lots à usage de logements, de bureaux ou de commerces, ou lorsque le budget prévisionnel moyen du syndicat sur une période de trois exercices consécutifs est inférieur à 15 000 euros.)
				\item  une nouvelle simplification des règles de décisions en assemblée générale, avec un abaissement des majorités (même celle de l’article 26, avec un second vote possible),
				\item  la facilitation des travaux du syndicat des copropriétaires (notamment par la généralisation des « Travaux d’intérêt collectif ») ou à l’initiative d’un copropriétaire
				\item  un accroissement des pouvoirs du conseil syndical, avec des facultés de délégation beaucoup plus importantes
				\item  l’encadrement de la fin du mandat de syndic
				\item  la poursuite de l’effort de « codification à droit constant engagé par le Sénat avec la loi \no  2018-1021 du 23 novembre 2018 dite ELAN, avec quelques « retouches » des définitions qui y figuraient, ainsi que des définitions complémentaires
			\end{itemize}

	
		\chapter{Les immeubles régis par la copropriété}

\section*{Introduction}
	
	\subsection{Quel est le parc de logements en copropriété ?}
	
	Selon l’enquête INSEE de 2013, \pourcent{28,1} des logements du parc métropolitain sont en copropriété. Le reste
	relève du parc social (\pourcent{11,8}) ou du secteur libre en mono-propriété (\pourcent{60,1}). La quasi-totalité des
	logements en copropriété sont des appartements (\pourcent{94,3}), et près de la moitié sont des résidences
	principales occupées par leur propriétaire. Les autres sont occupés à titre de résidence principale par des
	locataires, ou sont des résidences secondaires ou encore des logements vacants. Dans l’habitat individuel,
	très peu sont en location : ils sont en quasi-totalité occupés par leur propriétaire, à titre de résidence
	principale ou secondaire.
	
	Le parc est relativement ancien : près des deux tiers des copropriétés dans le collectif ont au moins un
	appartement construit avant 1970 (contre un logement sur deux dans l’ensemble du parc collectif), dont
	la moitié ont été bâtis avant 1914.
	
	Si l’on s’en tient à la région Ile de France l’habitat se répartissait ainsi en 2007 :
	\begin{itemize}
		\item La monopropriété qui représente aujourd’hui \pourcent{26} de l’habitat (en nette diminution).
		\item La maison individuelle qui représente aujourd’hui \pourcent{28,5} de l’habitat (en progression)
		\item La copropriété qui a fortement progressé ces dix dernières années et qui représentait en 2007
	\pourcent{45,1} de l’habitat.
	\end{itemize}
	
	En sorte qu’au total il y avait en 2007 plus de \nombre{2 400 000} logements en copropriété pour la seule région Ile-
	de-France\footnote{Il semble qu’au total il existe actuellement plus de 8 millions de logements soumis au statut de la copropriété sur un total de 34	millions de logements environ, soit 1/4 environ soumis au statut de la copropriété (cf. INSEE Le parc de logements en France au 1er janvier	2014).}
	
	Le registre des copropriétés a permis d’affiner ces chiffres. Au 3ème trimestre 2019, les
	statistiques (pour les copropriétés recensées) sont les suivantes :
	\begin{figure}[h]
		\centering
		\includegraphics[width=0.7\linewidth]{images/nombreDeCoproprietesParLots}
		\caption[Nombre de copropriétés par lots]{}
		\label{fig:nombredecoproprietesparlots}
	\end{figure}
	
	\begin{figure}[h]
		\centering
		\includegraphics[width=0.7\linewidth]{images/tailleDesCoproprietes}
		\caption[Taille des copropriétés]{Taille des copropriétés (les copropriétés de moins de 10 lots sont sans doute sous représentées)}
		\label{fig:tailledescoproprietes}
	\end{figure}
	
	\subsection{Quels sont les immeubles régis par le statut de la copropriété ?}
	
		\subsubsection{Définition du champ d’application par l’article 1 de la Loi 65-557 du 10 juillet 1965 modifie par l’ordonnance du 30 octobre 2019}
		Les immeubles auxquels s’applique le statut de la copropriété sont désignés à l'article 1er de la loi du 10
		juillet 1965, ainsi rédigé après l’Ordonnance du 30 octobre 2019 :
		\begin{quote}
			« \textsc{i} - La présente loi régit tout immeuble bâti ou groupe d’immeubles bâtis à usage total ou partiel
			d’habitation dont la propriété est répartie par lots entre plusieurs personnes
			Le lot de copropriété comporte obligatoirement une partie privative et une quote-part de parties
			communes, lesquelles sont indissociables. [$\dots$]
			
			« \textsc{ii} - A défaut de convention y dérogeant expressément et mettant en place une organisation dotée de
			la personnalité morale et suffisamment structurée pour assurer la gestion de leurs éléments et services
			communs, la présente loi est également applicable :
			
			« 1\degre{} A tout immeuble ou groupe d’immeubles bâtis à destination totale autre que d’habitation dont la
			propriété est répartie par lots entre plusieurs personnes ;
			
			« 2\degre{} A tout ensemble immobilier qui, outre des terrains, des volumes, des aménagements et des services
			communs, comporte des parcelles ou des volumes, bâtis ou non, faisant l’objet de droits de propriété
			privatifs.
			
			« Pour les immeubles, groupes d’immeubles et ensembles immobiliers mentionnés aux deux alinéas cidessus
			et déjà régis par la présente loi, la convention mentionnée au premier alinéa du présent II est
			adoptée par l’assemblée générale à l’unanimité des voix de tous les copropriétaires composant le
			syndicat. »
		\end{quote}
	
		\subsubsection{Le double champs d’application du texte : Obligatoire ou supplétif}
			Avant l’Ordonnance du 30 octobre 2018, l’article 1 n’était expressément visé parmi les
			dispositions « d’ordre public » du texte.
			
			Cependant, du fait de sa rédaction, les auteurs et la jurisprudence considéraient que l'on ne
			pouvait déroger au statut de la Copropriété dès lors que la propriété de l'immeuble ou du groupe
			d'immeubles est répartie entre plusieurs personnes\footnote{
			Civ 3\degre{} 15 nov. 1989 : Bull. civ. III \no 213 p. 117; D. 1990. J. 195 note Giverdon et Capoulade
			Un immeuble avait fait l'objet d'une donation-partage et à cette occasion avait été établi un État descriptif de division créant plusieurs lots ensuite vendus à des personnes différentes. Mais cet État descriptif de division ne fixait pas de quotes-parts de parties communes attribuées à chaque lot. Un propriétaire ayant réalisé des travaux sur parties communes prétendait que la loi sur la copropriété ne s'appliquait pas du fait que s'il existait des lots, ceux-ci ne comportaient pas de quote-part de parties communes. La cour de cassation répond : << le statut des immeubles bâtis s'applique de plein droit dès que sont remplies les seules conditions prévues à l'article 1er, alinéa 1er, de la loi du 10 juillet 1965 >>.
			Également Civ. 3\degre{} Ch. 30 juin 1998, JCP G. 1998, IV, 2961	
			}, et ce alors même qu'aucun Règlement de Copropriété n'a été établi.
			
			L’ordonnance du 30 octobre 2018 consacre ce caractère d’ordre public, car l’article 1 fait
			désormais partie des textes énumérés par l’article 43 de la Loi 65-557 du 10 juillet 1965, et toute
			clause contraire est « réputée non écrite ».
			
			Toutefois, le champ d’application « obligatoire » est cantonné par le grand (\I) de l’article 1 :
			\begin{itemize}
				\item aux « immeubles ou groupes d’immeuble bâtis dont la propriété est
				repartie par lots comprenant chacun une partie privative et une quote part
				de parties communes, les deux étant indissociables » (il existe une
				indivision forcée sur les parties communes) ;
				\item\textbf{ si ces immeubles sont partiellement à usage d’habitation}, ce qui est une
				nouveauté issue de l’ordonnance
			\end{itemize}
		
			Inversement, le statut peut être écarté par une convention expresse mettant en place une
			personne morale suffisamment structurée (\II) :
			\begin{itemize}
				\item \textbf{pour un ensemble immobilier} (parcelles ou volumes distinct) dans lequel il
				n’y a pas de propriété indivise mais des terrains, volumes, équipements ou
				services « communs », c’est-à-dire d’intérêt collectif
				\item \textbf{même dans un Immeuble ou groupe d’immeuble comportant des parties
				communes en indivision}, si les lots sont tous à usage autre que d’habitation
			\end{itemize}
		
		\subsubsection{Les éléments indifférents à l’application du statut}
		
			\paragraph{L’absence d’organisation}
			
			L’absence d’assemblée générale ou de syndic (absence d’organisation de la copropriété) ne sont pas des
			conditions d’application du statut de la copropriété\footnote{Cass. Civ. 3e 14 décembre 2010 – Juris Data \no 2011-000224}. D’ailleurs, les premières immatriculations de	copropriété révèlent que \pourcent{20} des copropriétés seraient dépourvues de syndic, et ce chiffre est sans doute sous- estimé car l’immatriculation se fait alors au fil des ventes par les notaires
			
			\paragraph{L’absence d’immatriculation}
			
			De même, l’absence d’immatriculation de l’immeuble au Registre des Copropriétés est sans conséquence
			sur l’application du statut (article 1-1 de la Loi 65-557 du 10 juillet 1965 issue de la loi \no 2018-1021 du 23
			novembre 2018 dite ELAN, dernier alinéa).
			
			\paragraph{L’absence de règlement de copropriété}
			
			Lorsqu’un ensemble immobilier fait l’objet d’un état descriptif de division mais qu’il n’y a pas de règlement
			de copropriété, cet immeuble est soumis à la loi du 10 juillet 1965\footnote{Cass. Civ. 3e 1er décembre 2009 pourvoi: 08-22102}. En ce cas, l’état descriptif de division, quelle que soit sa date, approuvé ou non, revêt un caractère contractuel entre colotis (résultant de l’acte de vente) et ses clauses engagent les colotis entre eux pour toutes les stipulations qui y sont contenues\footnote{Cass. Civ. 3e 12 janvier 2011 pourvoi: 09-13822}.
			La vente de lot de copropriété en l’absence de règlement de copropriété et d’état descriptif de division
			n’est pas nulle pour indétermination de l’objet (1109 Code Civil) dès lors que les lots étaient individualisés
			et qu'il n'en résultait aucune confusion avec les lots de l'autre copropriétaire, bien que le règlement de
			copropriété soit obligatoire et doit être soumis à l’acquéreur\footnote{Cass. Civ. 3e 17 novembre 2010}.
	
	\subsection*{Plan}
		Pour déterminer le champ d’application du statut de la copropriété, il faut par conséquent distinguer :
		\begin{itemize}
			\item La copropriété des régimes voisins, dans lesquels il n’existe pas de division de
			l’immeuble entre plusieurs propriétaires ayant des droits concurrents sur les parties
			communes % (section I)
			\item La copropriété horizontale, la copropriété verticale et la construction en volume
			%(section II)
			\item Pour ce qui concerne les ensembles comprenant plusieurs bâtiments, les « groupes
			d’immeubles bâtis », pour lesquels le statut de la copropriété s’applique
			obligatoirement, des « ensembles immobiliers » pour lesquels une organisation
			différente peut être mise en place %(section III)
			\item Enfin, il sera examiné la comptabilité entre le statut de la copropriété et les
			servitudes %(section IV)
		\end{itemize}

\section[Champ d’application impératif]{Champ d’application impératif (article \II) : immeuble ou groupe d’immeuble affecte a l’habitation}
	Depuis l’Ordonnance du 30 octobre 2019, l’article \I{} repose sur une double opposition
	\begin{itemize}
		\item l’Immeuble ou le groupe d’immeuble par opposition à « l’ensemble immobilier »
		\item et l’affectation à usage total ou partiel d’habitation par opposition à l’Immeuble ou le
		groupe d’immeuble à « destination totale autre que d’habitation »
	\end{itemize}
	
	Le champ d’application impératif du statut est désormais le suivant :
	\begin{quote}
		« \I{} - La présente loi régit tout immeuble bâti ou groupe d’immeubles bâtis à usage total ou partiel
		d’habitation dont la propriété est répartie par lots entre plusieurs personnes.
		Le lot de copropriété comporte obligatoirement une partie privative et une quote-part de parties
		communes, lesquelles sont indissociables. »
	\end{quote}
	
	\subsection{L’immeuble ou « groupe d’immeuble bâtis » caractérisé par l’homogénéité du sol}
	
		\subsubsection{Critère de distinction : homogénéité du sol}
		
			Qu'est-ce qui différencie le groupe d'immeubles bâtis où le statut sur la copropriété s'applique
			nécessairement, de l'ensemble immobilier où le statut de la copropriété ne s'applique qu'à défaut de
			conventions contraires ?
			
			Selon l’article 1, c’est l’existence de lots « comportant obligatoirement une partie privative et une
			quote-part de parties communes, lesquelles sont indissociables » qui justifie l’application
			impérative du statut, en d’autres termes la situation d’indivision forcée et perpétuelle dans
			laquelle se trouve les copropriétaires sur les parties communes qui détermine l’application du
			statut impératif.
			
			Cette situation correspond naturellement à un immeuble unique divisé en lots, mais également –-- c’est
			l’hypothèse du « groupe d’immeubles bâtis » aussi appelé « copropriété horizontale » --- à une parcelle
			cadastrale unique sur laquelle sont édifiés plusieurs bâtiments ou pavillon, tant que tous copropriétaires
			ont à la fois des « parties privatives » (appartement ou maison) et des droits indivis dans le sol (partie
			commune).
			
			Par conséquent, le groupe d’immeuble bâti relevant impérativement du statut de la copropriété se
			caractérise par l’homogénéité su sol, par opposition à l’hétérogénéité de la propriété du sol dans un
			ensemble immobilier\footnote{
			Givord, Giverdon, Capoulade, La Copropriété, Ed. Dalloz 2018 p. 85 ; Atias, Guide de la Copropriété bâtie, Edilex, 6\degre{} Edition, p. 35 ;
			Lafond Roux, Code de la Copropriété Ed. 2018 p. 11, J. Cabanac, Les ensembles immobiliers et le nouveau statut de la copropriété : Inf.
			rap. copr. mai 1966, p. 66. – Les ensembles immobiliers et la loi du 10 juillet 1965 : Gaz. Pal. 1966, 1, doctr. p. 117.– B. Leclercq, Les
			ensembles immobiliers : Rapport au 73e Congrès des notaires de France, Strasbourg 1976, p. 403 et s. – C. Lebatteux et J. Barnier-
			Sztabowvicz, Les ensembles immobiliers et l'adoption de l'organisation différente : Administrer juin 1994, p. 9 et s. – P. Capoulade :
			Copropriété et structures foncières dans la jurisprudence de la Cour de cassation : Administrer janv. 1996, p. 4 et s. – P. Capoulade et
			Cl. Giverdon, Propos sur les ensembles immobiliers : RD imm. 1997, p. 161 et s..
			} . Il y a homogénéité lorsque tous les propriétaires ont des droits réels sur
			l'ensemble du terrain servant d'assiette aux immeubles. Comme l’a écrit Pierre CAPOULADE\footnote{Administrer janvier 2001 \no 329 p. 37 et suivantes.} :
			\begin{quote}
				« Les copropriétaires possèdent des droits indivis sur l’ensemble du sol. Ils disposent sur celui-ci d’une
				quote-part numérique entrant, d’une manière indissociable avec les parties privatives, dans la composition
				du lot de copropriété ».
			\end{quote}
	
		\subsubsection{Exemples :}
			\begin{enumerate}
				\item Groupe d’immeuble bâti sur une seule parcelle comportant plusieurs Bâtiments d’habitation ou
				mixte (avec plusieurs lots), et des maisons individuelles chacune constitutive d’un lot, mais le sol
				est indivis
				\begin{center}
					\includegraphics[width=0.7\linewidth]{images/assietteCopro}
				\end{center}
				
				\item  Plan à rez-de-chaussée d’une copropriété (en jaune pâle : parties communes :
				sol et hall), avec deux Bâtiments constitués en lot
				\begin{center}
					\includegraphics[width=0.7\linewidth]{images/planRdcCopro}
				\end{center}
				
				
				\item Plan de coupe d’une copropriété « verticale » classique
				\begin{center}
					\includegraphics[width=0.7\linewidth]{images/planCoupeCopro}
				\end{center}
				
			\end{enumerate}

		\subsubsection{Conséquences}
			Lorsque la propriété du sol est homogène (appartenant indivisément à l’ensemble des copropriétaires), il
			existe un syndicat des copropriétaires de plein droit. Certes il sera possible de créer des syndicats
			secondaires pour chacun des bâtiments mais il n’y aura toujours qu’un seul syndicat « principal ».
			Ce principe a amené la Cour de Cassation à préciser qu’en cas de division d’un lot donnant vocation à la
			construction d’un bâtiment, la division de ce lot ne peut avoir pour conséquence de créer une seconde
			copropriété sur un terrain homogène\footnote{
			3\degre{} civ. 18 janv. 2018, \no 16-26072 au Bulletin et sur le site de la cour de cassation.
			} quand bien même ce lot correspondrait-il à un bâtiment séparé.
			
			La Cour d'Aix en Provence\footnote{
			16 avril 1992, Résidence l'Esplanade, Loyers et Copropriété 1993.272 et RTDI 93 p 115
			} a également retenu ce critère du régime du sol : retenant que l'immeuble
			présentait une structure homogène, l’arrêt a refusé d'assimiler l'Esplanade à un ensemble immobilier et
			a jugé en conséquence que seul le statut de la Copropriété devait recevoir application. En sorte que sur la
			demande d'un copropriétaire, les décisions de l'association syndicale libre mise en place par le promoteur
			de la Résidence lui ont été déclarées inopposables.
			
	\subsection{L’affectation a « usage total ou partiel d’habitation », nouveau critère du champ d’application impératif}

		Jusqu’à l’ordonnance du 30 octobre 2019, le statut de la copropriété s’appliquait à toutes sortes
		d’immeubles, quel que soit leur destination, dès lors que l’on se trouve en présence du régime homogène
		de la propriété du sol (chaque copropriétaire ayant des droits indivis sur la totalité du sol).
		
		En sorte que le statut s’applique bien évidemment aux immeubles d’habitation comme aux immeubles
		mixtes (habitation, professionnel, bureaux, commerces $\dots$) qui constituent la majorité en nombre
		d’immeubles soumis au statut de la copropriété. Mais le même statut s’applique également
		obligatoirement aux immeubles à usage exclusif de bureaux ou de commerces, et ce quel que soit le mode
		d’exploitation de l’immeuble dès lors que celui-ci se trouve divisé par lots.
		
		L’Ordonnance du 30 octobre 2019 revient aux origines du droit de la copropriété et aux intentions du
		législateur : que ce soit dans l’article 664 du Code civil de 1804 ou dans la loi du 28 juin 1938, les textes
		envisageaient la copropriété sous son aspect « habitation » : rappelons-nous en effet que le statut de la
		copropriété issu de la loi du 28 juin 1938 constituait le titre \II{} de cette loi, dans le titre 1\ier{} régissait les
		sociétés d’attribution qui avaient pour objet l’acquisition d’un appartement sur plan.
		
		La loi de 1965 a eu pour volonté principale d’établir un équilibre dans les droits et devoirs des
		copropriétaires à titre individuel d’une part et à titre collectif d’autre part ; implicitement cet équilibre
		concerne essentiellement les immeubles d’habitation : les travaux préparatoires de la loi de 1965 faisaient
		état de l’avenir « de l’habitat urbain ». Depuis lors s’est développée une conscience consumériste à laquelle
		l’idée du logement n’est pas étrangère. Enfin l’étude d’impact du projet de loi ELAN s’interrogeait
		justement sur la possibilité de différencier le statut selon le type d’occupation de l’immeuble.
		
		On peut s’interroger sur le point de savoir si le statut protecteur de la copropriété doit effectivement
		profiter à des immeubles exclusivement consacrés à des activités professionnelles : qu’il s’agisse de
		commerces ou d’immeubles de bureaux réalisés par des investisseurs.
		
		Le gouvernement a décidé de franchir le pas en modifiant l’alinéa 1er de la loi du 10 juillet 1965 par l’ajout
		des mots « \textit{à usage total ou partiel d’habitation} », en sorte qu’\textit{a contrario }le statut de la copropriété ne s’applique pas de plein droit aux immeubles dont la propriété est répartie par lots entre plusieurs
		personnes dès lors qu’aucun de ces lots n’a vocation à être affecté à l’habitation.
		
		Notons que pourra également échapper au statut de la copropriété un immeuble composé exclusivement
		de parkings !
		
		Cette réduction du champ d’application obligatoire du statut est tellement importante que l’Ordonnance
		a jugé utile de la réaffirmer dans l’alinéa 2 de l’article 1er de la loi en faisant état de la possibilité
		d’échapper au statut de la copropriété lorsqu’il s’agit « \textit{de tout immeuble ou groupes d’immeubles bâtis, à
		destination totale autre que l’habitation} ».
		Ainsi, le statut de la copropriété « obligatoire » est écarté, \textbf{alors même que l’immeuble est divisé par lots
		et que le sol est homogène, dès lors qu’il n’y a aucun lot à usage d’habitation}.
	
		Toutefois, l’affectation d’un sol lot à usage d’habitation (loge de gardien par exemple), aura pour
		conséquence de faire retomber l’immeuble en copropriété.

\section{Le champ d’application supplétif du statut : les ensembles immobiliers et les immeubles a destination autre que d'habitation}
	Le champ d’application du statut « par subsidiarité » (à défaut d’organisation contraire), ou supplétif est
	désormais défini, en application de l’article 1-\II{} de la Loi 65-557 du 10 juillet 1965 dans sa rédaction issue
	de l’Ordonnance du 30 octobre 2019
	\begin{quote}
		« \II. - A défaut de convention y dérogeant expressément et mettant en place une organisation dotée de
		la personnalité morale et suffisamment structurée pour assurer la gestion de leurs éléments et services
		communs, la présente loi est également applicable :
		« 1\degre{} A tout immeuble ou groupe d’immeubles bâtis à destination totale autre que d’habitation dont la
		propriété est répartie par lots entre plusieurs personnes ;
		« 2\degre{} A tout ensemble immobilier qui, outre des terrains, des volumes, des aménagements et des services
		communs, comporte des parcelles ou des volumes, bâtis ou non, faisant l’objet de droits de propriété
		privatifs.
		« Pour les immeubles, groupes d’immeubles et ensembles immobiliers mentionnés aux deux alinéas ci dessus
		et déjà régis par la présente loi, la convention mentionnée au premier alinéa du présent \II{} est
		adoptée par l’assemblée générale à l’unanimité des voix de tous les copropriétaires composant le
		syndicat. »
	\end{quote}
	
	\subsection{Les ensembles immobiliers hétérogènes}
	
		Qu'est ce qu'un ensemble immobilier ? C'est un ensemble dans lequel il existe à la fois :
		\begin{itemize}
			\item Des biens ou services « communs »: terrains, des volumes, des aménagements et des services
			communs, tels par exemple des allées et voies de desserte, un local social, des terrains de jeux
			et de sport, une piscine, des tennis, un gardiennage avec une maison de gardien, etc.
			\item Des parcelles ou volumes, bâtis ou non, faisant l’objet de droits de propriété « privatifs » ou
			« divis » (le sol de ces parcelles n’est pas en indivision forcée).
		\end{itemize}
		
		\subsubsection{L’ensemble immobilier caractérisé par « l’hétérogénéité du sol »}
			Il y a hétérogénéité lorsque l'ensemble immobilier comporte les terrains attribués à différentes personnes
			(le foncier est éclaté), aucune ne pouvant se prévaloir de droits réels (indivis) sur l'ensemble des terrains.
			Mais entre ces différentes parcelles existent des terrains communs, ou des éléments fédérateurs (indivis
			ou non)
		
			M \nom{SIZAIRE} : « à côté d'un terrain et d'éléments communs ou bien superposés à ceux-ci, il existe des propriétés ou des	copropriétés particulières : l'ensemble immobilier est juridiquement hétérogène »\footnote{
			Journées d'étude du CNEIL 22/23 nov. 1965, p. 52}
			
			M \nom{VIGNERON} : « l’ensemble immobilier tient sa spécificité du fait que les sols d’assiette font l’objet de modes d’appropriation différents, en propriété ou en copropriété selon les parcelles incluses dans cet ensemble et autonomes les unes par rapport aux autres »\footnote{
			(G. Vigneron- JurisClasseur Construction - Urbanisme > Fasc. 90-20 : STATUT DE LA COPROPRIÉTÉ. – Champ
			d'application du statut, \no 35}
			
			\paragraph{Exemples}
			
			\subparagraph{Lotissement} : toutes les parcelles font l’objet d’un droit de propriété exclusif, mais il existe des terrains, voiries, équipements d’intérêt collectifs dont la propriété est confiée à une ASL (voie = parcelle cadastrale)
			\begin{center}
				\includegraphics[width=0.7\linewidth]{images/lotissement}
			\end{center}
			
			
			\subparagraph{Voie privée} : Voie laissée en indivision avec de part et d’autres des parcelles distinctes, ou encore faisant l’objet d’une propriété de chaque riverain au droit de sa façade, jusqu’à la moitié de la voie.
			\begin{center}
				\includegraphics[width=0.7\linewidth]{images/voiePrivee}
			\end{center}
			
		
			\subparagraph{Jurisprudence antérieure à l’ordonnance}
			\begin{description}
				\item[Civ. 3ème 17 février 1999\footnote{Cour de cassation Chambre civile 3 17 Février 1999 \no 97-14.368 Bull. Association foncière urbaine libre Grand Ecran c/ associationsyndicale Italie-Vandrezanne}] : l’ensemble immobilier susceptible de faire l’objet d’une organisation différente résulte du seul fait que certains copropriétaires avaient des droits réels exclusifs sur certaines parcelles du terrain faisant ressortir l’hétérogénéité du régime juridique des fractions de l’ensemble en question.
				
				\item[Civ. 15 décembre 1993\footnote{Civ 15 décembre 1993, pourvoi: 91-12645 , au bulletin, Recueil Dalloz 1994, Somm. p. 205 (Domaine des Clausonnes)}]	: << \textit{Un lotissement comportant, selon les dispositions de l'art. R. 315-1 c. urb., division du sol en propriété ou en jouissance, privant les allotis de droits concurrents sur l'ensemble du terrain, une cour d'appel, qui constate, par motifs non critiqués, qu'un arrêté préfectoral a autorisé le lotissement et approuvé le cahier des charges et qu'une association syndicale a été constituée, d'où il résulte que l'application de la loi \no 65-557 du 10 juill. 1965 se trouve exclue, n'a pas à procéder à une recherche que ses constatations rendaient inopérante et qui n'était pas demandée} >>.
			
				\item[Paris 23\degre{} Ch 29 octobre 1997\footnote{Paris 23\degre{} Ch 29 octobre 1997 Tour Cantate Loyers et Copropriété 1998 \no 20}] Constitue un ensemble immobilier la juxtaposition d’immeubles en copropriété et d’immeubles	appartenant à une société détentrice de droits réels exclusifs sur une partie du terrain.
			\end{description}
		
		\subsubsection{L’ensemble immobilier dit « complexe » (en volumes)}
		
			\paragraph{Origine}
			
				La loi sur la copropriété ne s'applique pas nécessairement à des locaux superposés ou imbriqués, sans
				création corrélative de parties communes.
				En effet, il existe une présomption posée par l'article 552 du Code Civil, selon laquelle << la propriété du sol
				emporte la propriété du dessus et du dessous >> ; mais la preuve contraire peut être rapportée : c'est le cas
				assez fréquent d'une cave qui déborde sur le terrain voisin. Cette cave << débordante >> ne sera pas en
				copropriété avec l'immeuble dans le sol duquel elle se trouve.\footnote{
				Civ 1\degre{}, 2 avril 1962. B. \no 66, p. 54; cf également Civ 3\degre{} 11 mai 1994 RD Imm 1994, 486 pour une terrasse qui constitue
				le sol d'un immeuble et la couverture d'un autre immeuble.}
				
				La généralisation de ce mécanisme a permis la création de construction dites en « volume » (ou Ensembles
				Immobiliers Complexes) qui échappent également au statut de la copropriété.
				
				<< \emph{La volonté de limiter le gaspillage de l'espace se concrétisa avec l'apparition d'un nouveau parti
				architectural et urbanistique : << \emph{l'ensemble immobilier complexe (E.I.C.)} >>, ouvrage formant un tout,
				techniquement indivisible, où se juxtaposent, se superposent, s'imbriquent, s'articulent des volumes de
				destinations variées (habitations, activités, équipements, circulation, etc) et où l'utilisation systématique
				du tréfonds fait disparaître la notion même de sol naturel. L'objectif consiste à aménager non plus des
				terrains, mais l'espace urbain dans ses trois dimensions} >>(\nom{Walet} et \nom{Chambelland})\footnote{
				Walet et Chambelland in La Construction en Volumes (Masson Editeur, 1989)}
				
			\paragraph{Définition}
				Selon MM \nom{Walet} et \nom{Chambelland}, l'E.I.C. sera caractérisé :
				\begin{quote}
					<< 1\degre{} Par la juxtaposition et la superposition à l'intérieur \emph{d'une même structure technique},
					de volumes dont chacun \emph{abrite une fonction spécifique} et reçoit un statut \emph{approprié à
					celle-ci} ;
					<< 2\degre{} par son absence de parties communes entre ces volumes, \emph{absence due à
					l'hétérogénéité juridique de ces derniers} ;
					<< 3\degre{} par son appropriation, sa construction, sa gestion par \emph{deux ou plusieurs maîtres
					d'ouvrage} dont souvent l'un d'eux \emph{relève du droit public} ;
					<< 4\degre{} par ses contraintes de construction et de gestion entraînées à la fois par la structure
					technique et la pluralité de maîtrises de l'ouvrage et de destinations;
					<< 5\degre{} par l'intervention d'un spécialiste, promoteur ou aménageur. >>
				\end{quote}
			
				\begin{figure}[h]
					\centering
					\includegraphics[width=0.7\linewidth]{images/volumetrieCoupe}
					\caption[Exemple : Urbanisme sur Dalle (Nanterre)]{}
					\label{fig:volumetriecoupe}
				\end{figure}
				)
				Le plus souvent on trouve à l’origine de cette conception :
				\begin{itemize}
					\item une dalle servant de base à l'aménagement d'une zone --- cette dalle sera un ouvrage public;
					\item  plusieurs niveaux techniques et à usage de parkings sous la dalle ;
					\item des ouvrages ou parties d'ouvrages à vocation publique également (service de mairie
					ou Tribunal de commerce par exemple à Nanterre) sur la dalle ; 
					\item  à côté de ces ouvrages à caractère public ou parfois même au-dessus on trouvera des commerces
					et des habitations.
				\end{itemize}
				
				Chaque volume sera défini en altimétrie par rapport au Nivellement Général de la France\footnote{
				Une altitude est un écart de hauteur par rapport à un niveau de référence, le niveau moyen des mers. En France, l’altitude 0 correspond au niveau moyen de la Méditerranée enregistré au marégraphe de Marseille (dans le Vieux Port) de 1885 à 1897.
				Depuis ce lieu les géomètres ont parcouru l’Hexagone pour installer des repères d’altitude par la méthode du nivellement	géométrique précis à quelques millimètres. Il y a actuellement 450.000 repères sur un réseau de \nombre{300 000} kms de long.
				}(NGF) en sorte que la propriété du volume sera comprise entre deux cotes N.G.F.
				
				A l'intérieur du volume pourront être créés plusieurs niveaux qui seront eux-mêmes soumis à un statut
				prédéterminé, par exemple le volume A sera une propriété unique (Un Commerce au niveau O) et le
				volume B une copropriété (habitations du niveau 1 au niveau 5).
				
				S'il n'existe pas de parties communes, il existe cependant une certaine imbrication entre les différents
				volumes. Ces imbrications feront l'objet de servitudes entre les volumes (servitude d'appui, d'accrochage,
				de vue, etc ...).
				
				Il n’y a pas de copropriété s’il n’y a pas de parties indivises entre les copropriétaires (parties communes)\footnote{
				Cass. Civ 3e 8 septembre 2010
				}. Ces volumes seront le plus souvent regroupés en une Association Syndicale de Propriétaires qui gérera
				l'Ensemble Immobilier.
				
			\paragraph{Volumétrie par « affectation »}
				On peut s'interroger sur la légalité d'un découpage de bâtiments qui ne correspond pas à une réalité
				physique. Mais on imagine mal les tribunaux remettre en cause un nombre aujourd'hui important
				d'opérations immobilières d'envergure\footnote{
				Sur la question Cf. Sizaire - Division en volumes et Copropriété des immeubles bâtis JCP 88.1.3367.
				}
				
				Étant toutefois observé l'abus manifeste de certains promoteurs qui pour mieux vendre certains locaux en
				les faisant échapper aux contraintes de la loi de 1965 n'hésitent pas à créer des Divisions en volumes
				purement artificiels qui comprennent deux parties : un volume commercial au rez de chaussée et un
				volume habitation pour les étages au-dessus des commerces.
				
				La loi ALUR a consacré l’organisation des immeubles en volumes… dans l’hypothèse de la scission d’une
				copropriété initiale devant donner naissance à plusieurs propriétés distinctes. L’article 28 de la loi sur la
				scission comporte désormais deux nouveaux paragraphes qui sont ainsi rédigés :
				\begin{quote}
					\textsc{iv}. – La procédure (de scission) prévue au présent article peut également être employée pour la
					division en volumes d’un ensemble immobilier complexe comportant soit plusieurs bâtiments
					distincts sur dalle, soit plusieurs entités homogènes affectées à des usages différents pour autant
					que chacune de ces entités permettent une gestion autonome. Si le représentant de l'État dans le
					département ne se prononce dans les deux mois, son avis est réputé favorable.
					
					Elle ne peut en aucun cas être employée pour la division en volumes d’un bâtiment unique. ($\dots$)
				\end{quote}
			
				Toutefois il convient d’observer qu’en excluant la division en volumes d’un bâtiment unique, le législateur
				semble considérer que la Volumétrie doit correspondre à une « autonomie de gestion ».
				
			\paragraph{Consécration par l’ordonnance du 30 octobre 2019}
				
				L’ordonnance du 30 octobre 2019 a consacré la possibilité de constituer un Ensemble Immobilier
				complexe, puisque l’article 1- \II{} vise explicitement l’hypothèse de l’ensemble immobilier composé non de
				« parcelles » mais de « volumes ».
				
				Contrairement à ce qui avait été pressenti, l’ordonnance ne restreint pas les cas dans lesquels une mise
				en Volume est envisageable, par exemple en exigeant une autonomie structurelle ou une superposition
				des domaines publics et des domaines privées. Elle se contente de renforcer les exigences concernant
				l’organisation à mettre en place pour éviter à ces ensembles de se trouver « sans gouvernail », mais ces
				exigences ne sont pas de nature à éviter certains abus ( par exemple, la prise en charge de la toiture
				exclusivement par le volume supérieur, qui se trouve être celui d’habitation).
				
				Aussi, l’ordonnance pourrait-elle avoir un effet accélérateur sur la constitution d’ensembles immobiliers
				en Volumes :
				\begin{itemize}
					\item  elle supprime le contrôle du Préfet (prévu dans la loi \no 2014-366 DU 24 MARS 2014 DITE ALUR)
					en cas de scission en Volumes
					\item  elle encourage le découpage d’un ensemble Immobilier complexe en volumes « par
					affectation », dès lors que le Volume « commerce » en pied d’immeuble, même s’il est lui-même
					divisé en plusieurs lots comportant des parties communes indivises, pourra échapper
					complètement au statut de la copropriété
				\end{itemize}
		
		\subsubsection{Les éléments fédérateurs}
			Pour être qualifié comme tel, l’ensemble immobilier doit cependant nécessairement comprendre un ou
			des éléments fédérateurs, constitués par « \emph{des terrains, des volumes, des aménagements et des services communs} » tels que, par exemple, des parcelles indivises, une impasse commune\footnote{3ème Civ., 11 février 2009, Bull. civ., III, \no 34, pourvoi 08-10109}, une chaufferie
			centrale\footnote{3ème Civ., 21 juin 2000, pourvoi \no 98-20897}, etc.
			Ces « choses communes » peuvent être en indivision forcée, attribuées de façon divise à chaque coloti (
			voie privée propriété de chaque coloti jusqu’à la moitié de la voie) ou leur propriété peut être transférée
			à l’organisme de gestion ( à l’ASL en lotissement)
			La question qui peut se poser est celle de savoir si des aménagements communs, en l’absence de terrain
			commun suffisent à constituer un ensemble immobilier : en effet l’énumération de l’article 1 – II de la loi
			issu de l’ordonnance (et avant de dernier alinéa de l’article 1) et reliée par la conjonction « et », et non ou
			Cela signifie-t’il qu’il faut « en plus » de terrains ou volumes communs, des aménagements ou des
			équipements communs ? En l’état de la jurisprudence il semble qu’il suffit d’avoir un élément commun
			pour constituer un ensemble immobilier\footnote{
			(Civ. 3\degre{} Ch. 21 juin 2000, Pourvoi \no 98-20897) – Solution critiquée par Atias ; op. cit. p. 35
			}	(par exemple une chaufferie commune).
	
	\subsection{Les ensembles immobiliers à destination autre que d’habitation (ordonnance du 30 octobre 2019)}
		Relèvent également du champ supplétif d’application de la copropropriété « tout immeuble ou groupe
		d’immeubles bâtis à destination totale autre que d’habitation dont la propriété est répartie par lots entre
		plusieurs personne »
		
		Il faut noter que les termes du \I{} et du \II{} ne sont pas totalement transposable : le \I{} (champ d’application
		obligatoire) parle des immeubles « à usage total ou partiel d’habitation », tandis que le \II{} (champ
		d’application supplétif) parle des immeubles « à destination autre que d’habitation ».
		
		Or, en principe, la destination résulte du règlement de copropriété, et elle est intangible, tandis que l’usage
		d’un lot est une question de fait, un lot pouvant librement changer d’usage (ou d’affectation), tant que ce
		changement demeure compatible avec la destination de l’immeuble.
		
		En réalité, ces deux conditions sont sans doute cumulatives. Pour qu’un immeuble puisse être soumis à un
		statut autre que la copropriété :
		\begin{itemize}
			\item il faut qu’à sa conception, donc dans le « cahier des charges » constitutif, il se trouve
				intégralement destiné à un usage autre que d’habitation ;
			\item il faut en outre que cette destination ait été respectée concrètement, car si l’immeuble devient
				à usage partiel ou total d’habitation, il retombera sous le coup du statut de la copropriété.
		\end{itemize}
	
	\subsection{L’exigence d’une organisation différente « suffisamment structurée » et « dotée de la personnalité morale » (ordonnance du 30 octobre 2019)}
		Bien que l’ordonnance ait pour l’essentiel consacré l’évolution jurisprudentielle en cours, elle crée pour
		ces immeubles soumis à un statut « alternatif » à la copropriété un cadre beaucoup plus rigide que celui
		de l’ancien article I dernier alinéa qui se contentait d’appliquer le statut aux ensembles immobiliers
		hétérogènes « à défaut de convention contraire créant une organisation différente ».
		
		Désormais, le statut redeviendra applicable « à défaut de \emph{convention y dérogeant expressément} et mettant
		en place une organisation dotée de la \emph{personnalité morale} et \emph{suffisamment structurée pour assurer}, la
		gestion de leurs éléments et services communs, la présente loi est également applicable ($\dots$) ».
		
		\subsubsection{Exigence d’une dérogation explicite}
			L’Ordonnance substitue au mot « convention contraire » les mots « convention y dérogeant
			expressément ». Ce faisant le texte reprend purement et simplement les mots qui avaient été retenus
			dans l’arrêt de la cour de cassation du 19 septembre 2012.
			
			Cette modification est heureuse : la convention à intervenir n’est pas nécessairement contraire aux
			dispositions de la loi de 1965 ; on peut concevoir en effet la mise en place d’une ASL qui outre les quelques
			dispositions impératives de l’Ordonnance du 1er juillet 2004, reprendrait pour l’essentiel les dispositions
			de la Loi 65-557 du 10 juillet 1965, par exemple quant à la répartition des charges.
		
		\subsubsection{Exigence de la création d’une personne morale}
		
			Cette précision est une codification de la jurisprudence récente de la Cour de Cassation. En effet, certaines
			voies privées, ou certains lotissements (avant 1943) sont dotés d’un cahier des charges, ou d’une
			convention d’indivision ou de servitude, indiquant les modalités de répartition des charges, voire de prise
			de décision, mais dépourvus de la personnalité morale, si bien qu’ils constituent des groupements de fait\footnote{cf. civ. 3\degre{} Ch. 31 mars 1993 \no 90-10143.}.
			
			Cette situation aboutit immanquablement à une impasse : dépourvu des attributs de la personnalité
			morale, un tel groupement ne peut ni régulariser de contrat, ni agir en justice pour préserver les
			« éléments fédérateurs », ni poursuivre judiciairement l’un de ses membres en recouvrement des charges.
			
			Aussi la cour de cassation\footnote{Civ. 3\degre{} Ch. 19 sep 2012, Pourvoi \no 11-13679 11-13789, au Bulletin} dans un arrêt du 19 septembre 2012 rendu à propos d’immeubles divisés en
			volumes, avait-elle déjà considéré que le statut ne pouvait être écarté en l’absence de personnalité
			juridique assurant l’entretien de ces éléments d’équipement:
			\begin{quote}
				« Attendu que, pour débouter la SCI [\emph{de sa demande tendant à l’application du statut de la
				copropriété}], l'arrêt [$\dots$] relève que l'état descriptif de division stipule que l'ensemble immobilier ne
				sera pas régi par la loi du 10 juillet 1965 et qu'à cette fin, l'acte identifie des volumes immobiliers de
				pleine propriété dans le cadre du régime du droit de superficie, et énonce l'ensemble des servitudes
				issues de l'imbrication de ces volumes qui permettent leur coexistence ainsi que l'attribution [de]
				\nombre{3 026}/\nombre{10 000}\iemes{} des charges générales au lot \no 4, retient que l'état descriptif de division constitue,
				relativement à ce lot, la convention contraire visée à l'article 1er, alinéa 2, de la loi du 10 juillet 1965 ;
				Qu'en statuant ainsi, sans constater la création d'une organisation différente, au sens de la loi, pour
				la gestion des éléments communs de l'ensemble immobilier, la cour d'appel a violé le texte susvisé »
			\end{quote}
		
		\subsubsection{Une personnalité morale « suffisamment structurée » -- les 3 types de	« structures » communément envisageables}
		
		On ignore pour quelles raisons l’Ordonnance a cru devoir exiger que cette personnalité morale soit
		suffisamment structurée.
		
		Si l’on prend cet exemple de l’association syndicale libre l’Ordonnance du 1er juillet 2004 édicte en son
		article 7 que « les statuts fixent son objet, son siège et ses règles de fonctionnement, précise ses modalités
		de financement et le mode de ($\dots$) recouvrement des cotisations ». Aux termes de l’article 9 de la même
		ordonnance du 1er juillet 2004 : l’association « est administrée par un syndicat composé de membres élus
		($\dots$) » et « le syndicat règle par ses délibérations les affaires de l’association ».
		
		Il est vrai cependant que les statuts peuvent être mal rédigés ou contradictoires, mais en ce cas on voit
		mal le juge requalifier l’association syndicale en syndicat de copropriété puisqu’il doit simplement
		appliquer les dispositions du droit des obligations en ce qui concerne l’interprétation des contrats : article
		1188 et suivants modifiés par l’Ordonnance \no 2016-131 du 10 février 2016.
		
		De plus, il a été jugé à maintes reprises qu'un statut d'organisation différente est exclusif du statut de la
		Copropriété : les charges seront réparties conformément aux statuts qui pourront retenir une répartition
		différente de celle organisée par la loi de 1965 (sauf dans l'hypothèse de la Société d'Attribution non
		encore dissoute) : la loi du 10 juillet 1965 est étrangère au fonctionnement d'une ASL régie par
		l’ordonnance du 1er juillet 2004 relative aux associations de propriétaires (ayant remplacé la loi du 21 juin
		1865 abrogée).
		
		Pour autant on peut cependant envisager l’hypothèse où les statuts seraient muets sur des points
		essentiels comme par exemple la majorité applicable aux décisions ou l’absence de répartition des
		cotisations entre les immeubles membres de l’association syndicale libre.
		
		En cette hypothèse et dans le premier cas (absence de mention sur les majorités applicables) le juge devra t-il faire application de la règle de l’unanimité ou pourrait-il requalifier le statut de l’ensemble immobilier
		pour le soumettre au droit de la copropriété ? Dans le second cas (absence de répartition des cotisations),
		on pourrait concevoir effectivement que le juge considère qu’il y a lieu d’appliquer le statut de la
		copropriété. En ce cas il y aurait substitution pure et simple du régime de la copropriété aux statuts de
		l’association syndicale libre.
		
		Face à cette exigence, il sera préférable, de retenir, pour administrer l’immeuble, une des formes
		communément admises pour la gestion des ensembles immobiliers.
		
		\paragraph{L'association de propriétaires de l’ordonnance du 1er juillet 2004 et le décret du 8 mai 2006}
			Ces associations de propriétaires sont en fait les anciennes associations syndicales de la loi du 21 juin 1865.
			L’association de propriétaires peut être libre ou autorisée. La forme moderne de ces Associations est
			l’Association Foncière Urbaine Libre (AFUL), prévue par le Code de l’Urbanisme pour gérer les lotissements
			(art Article L322-9-1 du Code de l’Urbanisme).
			
			La contrainte majeure concernant ces Associations Syndicales est qu’il faut recueillir le consentement
			individuel de tous les membres situés dans son périmètre. L’ASL n’est donc, le plus souvent, que constituée
			\emph{ab initio}. Sinon, l’unanimité est requise.
			
			Ces Associations Syndicales sont administrées par un « Syndicat » élu par l'Assemblée des syndicataires et
			représenté par un Président (salarié ou élu) qui est l'agent d'exécution du Syndicat. Lorsqu’elles
			comprennent des copropriétés, celles-ci peuvent être représentées à l’assemblée générale par le
			président du conseil syndical (ASL) ou le syndic (AFUL), préalablement habilité par l’assemblée générale
			de copropriété.
			
			L’Ordonnance du 1er juillet 2004 et son décret d’application imposent un contenu minimum aux statuts,
			mais ne comprennent pratiquement aucune disposition d’ordre public, ce qui laisse une grande latitude
			dans l’organisation de l’ensemble immobilier.
	
			Toutefois on peut s’inquiéter de cette trop grande liberté s’agissant notamment de la participation aux
			charges qui peut être particulièrement déséquilibrée : par exemple dans un centre commercial où
			l’opérateur pourra avoir tendance à favoriser les « locomotives » - dont les murs peuvent appartenir à une
			ou plusieurs de ses filiales - au détriment des simples « wagons » que sont les boutiques installées le long
			du mail commercial !
			
			De la même façon, les propriétaires minoritaires ne pourront en cas de décision majoritaire contraire à
			leurs intérêts invoquer l’atteinte à la destination de l’immeuble, garde-fou essentiel dans l’application du
			statut de la copropriété.
			
			Il est vrai que le même risque peut exister dans le cas d’ensembles immobiliers à vocation mixte
			(habitation, commerce, bureau). Toutefois le risque est alors limité puisqu’il ne portera que sur les
			éléments fédérateurs à plusieurs immeubles, alors que s’agissant d’un même immeuble à usage exclusif
			de bureaux et/ou de commerces, ce statut (organisation dotée de la personnalité morale et suffisamment
			structurée) s’appliquera à toutes les parties communes.
		
		\paragraph{La Société Immobilière de Gestion et d'Entretien}
		
			Souvent appelée société de location, cette société loue des immeubles ou les met à disposition de ses
			associés. Ce type de société est régi par le droit commun (article 1844 et s. du code civil).
		
		\paragraph{L'Union de syndicats de copropriétaires}
		
			C’est une organisation différente prévue à l’article 29 de la loi de 1965 et qui fait l’objet des articles 63 à
			63-4 du Décret de 1967.
			
			Par exemple, l'ensemble réalisé à la Plagne en Savoie a adopté ce schéma juridique. Notons cependant que
			pour qu'il y ait Union de Syndicats, il faut par application de la loi du 10 juillet 1965 sur la copropriété que
			cette Union comprenne au moins un syndicat de copropriété, peu importe le statut juridique des autres
			immeubles. De plus, cette structure n’est pas juridiquement stable : sauf pour les Unions issues d’une
			scission volumétrique, il ne peut être imposé à l’un des membres de l’union l’interdiction de s’en retirer.
			
			Il n’en reste pas moins que l’Union est une bonne solution pour « recouvrir » d’une personnalité juridique
			une convention de gestion qui en était jusqu’alors dépourvue, car l’adhésion de la copropriété sera votée
			à la majorité de l’article 25 (l’unanimité n’est donc pas requise).
	
	\subsection{Le passage du statut impératif au statut supplétif (et inversement)}
	
		\subsubsection{Modalité de la « sortie » du régime de la copropriété (Ord. 30 oct. 2019)}
		
			L’ordonnance prévoit les modalités selon lesquelles un immeuble déjà existant, pour lequel l’application
			du statut n’est pas ou plus obligatoire, mais qui en relèverait actuellement, pourrait se soustraire au statut
			impératif : il est exigé une décision prise à l’unanimité des copropriétaires.
		
			Bien que le dernier alinéa vise les « deux alinéas ci –dessus », donc tant l’hypothèse d’un immeuble
			pouvant échapper au statut du fait de sa destination autre que d’habitation, que celle d’un ensemble
			immobilier :
			\begin{itemize}
				\item  pour l’immeuble à destination autre que d’habitation, il est normal qu’un changement des
				règles du jeu aussi fondamental relève de l’unanimité des membres du syndicat des
				copropriétaires
				\item  toutefois, pour les ensembles immobiliers hétérogènes, soumis par « défaut » au statut de la
				copropriété, la solution peut s’avérer en contradiction avec les dispositions concernant les
				Unions de Syndicat. En effet, la constitution d’une Union devrait permettre d’échapper au
				statut de la Loi 65-557 du 10 juillet 1965, et ne requière qu’une décision à l’article 25. A moins
				que le gouvernement n’ait considéré que les Unions relevaient encore du statut ?
			\end{itemize}
			
			Par ailleurs, l’adoption de cette convention n’est pas nécessairement incompatible avec le maintien d’une
			indivision forcée sur les parties communes. Pour un immeuble de bureau, par exemple, il pourrait être
			constituée (à l’unanimité) une ASL pour l’administration des parties communes, chacun demeurant
			propriétaire de « lots » comportant une quote part indivise de parties communes. En effet, la division en
			Volumes a posteriori d’un tel immeuble serait sans doute d’une complexité inextricable.
		
		\subsubsection{Exemple de requalification : application du statut de la copropriété aux voies privées dépourvues d’organisation collective}
			
			On peut s'interroger sur la possibilité de faire appliquer la loi aux immeubles riverains d'une voie privée
			lorsque cette voie ne bénéficie pas d'un statut conventionnel (A.S.L.) ou forcé (loi sur l'assainissement de
			1912, aujourd’hui abrogée).\footnote{
			Cf D. \nom{SIZAIRE} Gazette du Palais 14 juillet 1985 p. 4 : Le statut de la copropriété des immeubles bâtis et la gestion des voies privées, cours ou jardins}
			
			Un premier arrêt de cassation, concernant le Passage de Briare\footnote{
			Civ.3ème 11 octobre 2000, \no 99-10039, Administrer janvier 2001 \no 329, comm Capoulade
			}, avait consacré ce principe, étant précisé que la propriété du sol était commune à l’ensemble des immeubles riverains. La cour d’appel avait écarté	l’application de la loi sur la copropriété puisque cette voie était bordée de propriétés distinctes. L’arrêt est cassé car « cette situation n’exclut pas de plein droit l’application du statut de la copropriété pour l’entretien, la gestion et l’administration de la voie. »
			
			La même solution a été retenue sans équivoque dans un arrêt de cassation de la 3ème Chambre de la Cour
			de Cassation du 11 février 2009\footnote{
			Cour de cassation, chambre civile 3, 11 février 2009, \no de pourvoi: 08-10109 Publié au bulletin Cassation Recueil Dalloz,
			\no 8, 26 février 2009, Actualité jurisprudentielle, p.496-497, note Yves Rouquet (“Etablissement d’enseignement et propriété
			commerciale”). Voir également la revue Loyers et copropriété, \no 4, avril 2009, commentaire \no 99, p.23-24, note Guy Vigneron
			(“Ensemble immobilier”).
			}	: la cour d’appel avait considéré que « la copropriété pure et simple
			appliquée à un ensemble immobilier n’est pas sans inconvénient et qu’il existe d’autres modes
			d’organisation différente ». La cassation se fait au visa de l’article 1er alinéa 2 (devenu l’article I-2\degre{}) de la
			loi et avec l’attendu de principe : « Attendu qu'à défaut de convention contraire créant une organisation
			différente, la présente loi est également applicable aux ensembles immobiliers qui, outre des terrains, des
			aménagements et des services communs, comportent des parcelles, bâties ou non, faisant l'objet de droits
			de propriété privatifs ».

\section{Copropriété et autres formes d’appropriation des biens}
	
	\subsection{Copropriété et monopropriété}
	
		La loi sur la copropriété ne s'applique pas lorsqu'une même personne, une même famille ou une même
		société est propriétaire de la totalité de l'immeuble (Cf. les immeubles de rapport du début du siècle).
		
		Mais dès que cette personne (pour payer le ravalement ou pour mettre l'immeuble aux normes de
		salubrité et de confort, par exemple) vend un seul appartement sur l'ensemble de ceux qui composent le
		bâtiment, le statut sur la copropriété reçoit application : par le fait de cette vente la propriété est répartie
		entre plusieurs personnes"\footnote{Civ 1ère 19 janvier 1960. Bull. \no 35 p 28}.
		
		\paragraph{Sur l’apparition de la copropriété}
		Dès lors que la propriété de l'immeuble est répartie entre plusieurs personnes, par lots comportant chacun
		une partie privative constituée d'une maison et une quote-part de parties communes, il y a lieu d'en
		déduire que le statut de la copropriété était applicable à la date où l'immeuble avait comporté deux lots
		bâtis appartenant à deux personnes différentes\footnote{Cass. Civ 3e 12 janvier 2011 Pourvoi \no 09-13822}.
		L’acte de partage qui répartit les lots de l’immeuble entre les copartageants, condition d’application du
		statut de la copropriété immobilière, marque la naissance de plein droit du syndicat de copropriété.\footnote{
		Cass. Civ. 15 mars 2011	}%(cf. Chap III)
	
		\paragraph{Sur la dissolution de la copropriété par réunion de tous les lots en une seule main}
		Il résulte de cette exigence de « division » de l’immeuble entre plusieurs propriétaires qu’une copropriété
		peut disparaître dans l'hypothèse inverse : lorsque tous les lots deviennent la propriété d'une seule
		personne (achat, héritage,...)\footnote{
		Cf. sur ce point l'article du Coneiller GUILLOT Administrer avril 1979 : La disparition du Syndicat des Copropriétaires.
		}, comme prévu désormais par l’article 46-1 de la Loi 65-557 du 10 juillet
		1965 (article 39 de l’ordonnance du 30.10.2019)% –cf infra chapitre III
	
	\subsection{La copropriété et les sociétés de construction}
	
		\subsubsection{Les sociétés d’Attributions (art 212 à 212-17 du CCH)}
			
			Les sociétés d'attribution sont les anciennes sociétés de la loi de 1938 Chapitre 1er, appelées ensuite
			Sociétés du Titre \II{} de la loi du 16 juillet 1971 (abrogée) qui relèvent des dispositions du chapitre \II{} du Titre
			1\ier{} – Statut des sociétés de construction du CCH, articles 212-1 à 212-17.
			
			Ces sociétés sont constituées par des promoteurs en vue de vendre l'immeuble, mais les capitaux
			nécessaires à la construction sont recueillis auprès des futurs propriétaires. Ces promoteurs vont
			constituer entre eux une société dont l'objet sera la construction de l'immeuble en vue de son attribution
			par fractions divises aux associés qui souscriront les parts des promoteurs (Méthode dite de Paris ayant
			succédé à la méthode dite de Grenoble).
			
			Les acquéreurs de parts ne deviennent pas propriétaires des appartements, mais associés : les parts
			acquises donnant vocation à l'attribution en jouissance de l'appartement jusqu'à dissolution de la société,
			C'est donc la dissolution ou le retrait de l'associé qui transfère à celui-ci la propriété de son appartement.
			
			C'est la Société qui est propriétaire de l'immeuble et non pas les associés, l'associé peut seulement donner ses parts en nantissement ; par contre il ne peut pas consentir d’hypothèque sur un immeuble dont il n’est pas propriétaire. Certes, la société pourrait consentir une hypothèque ; mais elle ne peut le faire au profit d’un associé sans perdre le régime fiscal de la << transparence >>. Ceci explique qu’il est plus difficile d’obtenir un crédit pour acquérir des parts de société d’attribution que pour acquérir un appartement ; d’où la liquidation de ces sociétés lorsque l’immeuble est achevé et les comptes de construction approuvés.
	
			\subsubsection*{Zoom}
			
			Le régime de ces sociétés se rapproche de celui de la copropriété : l'article L 212-2 édicte en effet que
			lorsque l'immeuble est destiné à passer en copropriété, doivent exister un règlement de copropriété et un
			état descriptif de division "avant tout commencement des travaux" ou "avant toute entrée en jouissance".
			
			Les règles de répartition des charges sont identiques aux règles posées par la loi sur la copropriété et la
			révision judiciaire de ces charges est possible dans les mêmes conditions.
			
			Cette identité n'est cependant pas complète : si le règlement de jouissance est un règlement de
			copropriété avant la lettre, la gestion de la Société même après achèvement de l'immeuble demeure régie
			par les dispositions de ses statuts et les majorités applicables sont celles des statuts, non celles prévues
			par la loi de 1965\footnote{PARIS 12 janvier 1983 R.T.D.I. 1983 p 258 et 261} : l’article L 212-1 précise au demeurant : « L’objet de ces sociétés comprend la gestion
			et l’entretien des immeubles jusqu’à la mise en place d’un statut différent ».
			Notamment, la modification de la répartition des charges entre associés doit intervenir dans les conditions
			prévues par les statuts, non en fonction de la loi du 10 juillet 1965, inapplicables au fonctionnement d’une
			société d’attribution\footnote{Civ 3\degre{} 12 fév 1997 Loyers et Copropriété mai 1997 \no 149}.
			Le passage en copropriété résultera de la première « sortie » d’un associé de la société. Il demandera
			« l’attribution des parts aux lots »), car il se formera alors une copropriété à deux : l’attributaire et la SCIA.
			Il peut aussi résulter de la dissolution de la SCIA (cf. Chapitre 3)
		
		\subsubsection{Les sociétés d’habitat participatif (art l 200-1 et s. du cch)}
	
		Il s’agit d’une création de la loi ALUR insérée dans le Livre \II{} du CCH : « TITRE PRÉLIMINAIRE : « LES
		SOCIÉTÉS D'HABITAT PARTICIPATIF » (articles L. 200-1 et suivants)
		\begin{quote}
			« L'habitat participatif est une démarche citoyenne qui permet à des personnes physiques de s'associer,
			le cas échéant avec des personnes morales, afin de participer à la définition et à la conception de leurs
			logements et des espaces destinés à un usage commun, de construire ou d'acquérir un ou plusieurs
			immeubles destinés à leur habitation et, le cas échéant, d'assurer la gestion ultérieure des immeubles
			construits ou acquis.
			« En partenariat avec les différents acteurs agissant en faveur de l'amélioration et de la réhabilitation du
			parc de logements existant public ou privé et dans le respect des politiques menées aux niveaux national
			et local, l'habitat participatif favorise la construction et la mise à disposition de logements, ainsi que la
			mise en valeur d'espaces collectifs dans une logique de partage et de solidarité entre habitants ».
		\end{quote}
		
		\paragraph{Les principales caractéristiques de ces sociétés}
		
	\begin{enumerate}[label=\arabic*)]
		\item elles ont pour vocation de construire puis de loger des associés personnes physiques dont les parts
		donneront vocation à l'attribution en propriété ou en jouissance de leur résidence principale.
		
		\item ces sociétés ont également pour objet de gérer l'immeuble une fois construit ou acquis ; cette gestion
		portant notamment sur les parties communes et les " espaces partagés" ; en d'autres termes ces sociétés
		ont également pour objet de " contribuer au développement de la vie collective" étant précisé que ces
		sociétés pourront sous certaines conditions offrir des services non seulement à leurs membres mais
		également à des tiers.
		
		\item  la responsabilité des associés est limitée à leur apport dans le capital social.
		En réalité ces sociétés auront pour " promoteurs" des personnes morales à vocation sociale (sociétés HLM,
		sociétés d'économie mixte ...) qui détiendront jusqu'à \pourcent{30} du capital social et se verront attribuer les
		logements correspondant à leur participation dans le capital social.
		
		\item Ces sociétés seront de deux types distincts :
		\begin{enumerate}[label=\roman*)]
			\item Sociétés coopératives d'habitants
			
			Les sociétés coopératives d'habitants ne donneront à leurs associés que la jouissance des logements, en
			sorte que l'immeuble restera la propriété de la société coopérative, donc l'immeuble aura un propriétaire
			unique et le statut de la copropriété ne s'appliquera pas.
			
			\item Sociétés d'autopromotion
			
			Les sociétés d'autopromotion décideront dès l'approbation de leurs statuts de choisir entre deux vocations
			distincts : soit l'attribution en jouissance des logements soit l'attribution en jouissance puis en propriété
			des logements.
			
			Si l’attribution en jouissance est prévue, l’associé pourra quand même se retirer de la société mais son
			retrait n'entraînera pas la disparition de ses parts qu'il pourra céder à un successeur choisi par lui et agréé
			par la société ou en cas de refus de ce successeur par la société, choisi par cette dernière.
		
			Si l’attribution en propriété est prévue, tout associé peut se retirer de la société une fois l’immeuble
			construit et les comptes de construction approuvés. En ce cas le retrait entraîne annulation des parts
			correspondantes et l'immeuble se trouve soumis au statut de la copropriété puisque appartenant à
			plusieurs propriétaires.
			
			Mais dans l'un et l'autre cas, l'assemblée générale statuant à la double majorité des deux tiers des voix et
			des deux tiers des associés pourra prononcer la dissolution de la société ; le partage pouvant alors
			entraîner attribution des fractions d'immeubles aux associés.
		\end{enumerate}
	\end{enumerate}
	
	\subsection{Copropriété et démembrement de propriété}
	
		\subsubsection{Location Accession de la loi du 12 juillet 1984}
			
			Même si l'article 32 de cette loi édicte que la signature d'un contrat de location est assimilée à une
			mutation, en réalité, la Copropriété ne naîtra à la vie civile que du jour où le locataire-accédant sera devenu
			réellement propriétaire au terme du contrat. La disposition de l'article 32 signifie simplement que si
			l'immeuble est soumis au régime de la Copropriété, le locataire-accédant aura pour partie les droits d'un
			copropriétaire (par exemple celui de participer et de voter sur certaines questions aux Assemblées
			Générales).
		
		\subsubsection{La copropriété et la multipropriété (Loi du 6 janvier 1986)}
	
			Le terme de multipropriété n'est pas exclusif : on rencontre en effet la propriété à temps partagé ou encore
			la propriété spatio-temporelle $\dots$ sans évoquer son appellation franglaise de \emph{société de time-sharing}.
			Le << multipropriétaire" >> est tout $\dots$ sauf copropriétaire. Il est essentiellement un acquéreur des parts ou
			actions d'une << société d'attribution d'immeubles à temps partagé >> donc sans attribution en propriété.
			C'est donc une société de type particulier.
	
			Dans ce système l'acquéreur achète les parts qui lui donneront vocation à la jouissance d'une fraction de
			l'immeuble pendant une période déterminée : huit jours à Noël ou quinze jours en été, par exemple.
			
			La loi de 1965 ne s'applique pas davantage à cette catégorie particulière de société ; cependant, compte
			tenu de l'essor de ce type d'habitat de loisir, le législateur, par une loi du 6 janvier 1986, leur a donné un
			statut spécial qui par nombre d'aspects s'inspire, en matière de gestion de l'immeuble de la société et de
			répartition des charges, du régime de la loi du 10 juillet 1965 :
			
			Par exemple le gérant de la société est nécessairement désigné (ou révoqué) par décision des associés
			représentant plus de la moitié des parts sociales (dans une Copropriété le syndic est désigné par les
			copropriétaires représentant plus de la moitié des voix de l'ensemble immobilier). De même doit exister
			un \textbf{conseil de surveillance} qui donne son avis aux dirigeants sociaux ou à l'assemblée générale sur toutes
			les questions concernant la société pour lesquelles il est consulté ou dont il se saisit lui-même.
			
			De plus les décisions votées en Assemblée Générale doivent être prises selon les cas à la majorité des parts
			sociales, à la majorité de plus de la moitié des parts sociales ou à la majorité des deux-tiers des voix des
			associés; système des trois majorités tiré du statut de la Copropriété.
			
			S'agissant de la répartition des charges, il distingue entre les charges générales (réparties
			proportionnellement au nombre de parts détenues par les associés dans le capital - article 9) et les charges
			entraînées par les services collectifs et éléments d'équipement (réparties en fonction de la situation, de
			la consistance du local et de la période de jouissance).
			
			Mais cette répartition tient compte de l'utilisation effective des lots, (les charges de ces services collectifs
			et équipements collectifs ne sont pas dues si l'associé n'occupe pas son lot) ce qui est antinomique du
			statut de la copropriété qui ne tient compte que de l'utilité des services et équipements collectifs pour le
			lot (et non de l'utilisation des lots).
		
		\subsubsection{Le BRS (Bail réel solidaire)}
		
			La loi \no 2014-366 du 24 mars 2014 dite loi ALUR a créé, à son article 164, les organismes de foncier
			solidaire (OFS). Il s’agit d’organismes sans but lucratif qui ont pour objet d’acquérir et de gérer des terrains,
			bâtis ou non, en vue de réaliser des logements et des équipements collectifs, destinés à la location ou à
			l’accession à la propriété, à usage d’habitation principale ou à usage mixte professionnel et d’habitation
			principale. Cet article a été codifié à l’article L. 329-1 du Code de l’urbanisme.
			
			L’OFS constitue ainsi un nouvel acteur foncier dont l’objet est notamment d’affecter durablement du
			foncier, bâti ou non bâti, dont ils restent propriétaires, à la construction ou à la gestion de logements en
			accession à la propriété ou en location pour des ménages sous plafond de ressources.
			Pour ce faire, le législateur a prévu un nouveau dispositif visant à dissocier les propriétés du sol et du bâti
			à travers un bail de longue durée générateur de droits réels, dont la durée est reconduite à chaque
			mutation : le bail réel solidaire (BRS).
			
			L’article L. 255-1 du Code de la construction et de l’habitation en présente les caractéristiques essentielles.
			La mise sous bail réel solidaire est une faculté réservée aux OFS. Le bail réel solidaire, d’une durée comprise
			entre dix-huit et quatre-vingt-dix-neuf ans, permet de consentir des droits réels immobiliers portant sur
			des logements en vue soit de la location, soit de l’accession à la propriété de logements. Le bail réel
			solidaire peut avoir pour objet la construction ou la réhabilitation de logements. Il peut également
			concerner une construction existante, ne nécessitant pas de travaux. Le preneur doit verser une redevance
			au bailleur en vue de la location ou de l’accession à la propriété de logements à prix modéré.
			
			La principale innovation du bail réel solidaire repose sur son caractère rechargeable. En cela, à chaque
			cession des droits réels afférents au logement par le preneur, le cessionnaire conclut un nouveau bail réel
			avec l’OFS et voit la durée de ce nouveau bail de plein droit prorogée lorsqu’un agrément est délivré
			(article L. 255-12 du Code de la construction et de l’habitation). Ainsi, tout nouveau preneur bénéficie de
			droits réels immobiliers pour une durée égale à celle prévue dans le contrat initial.
			Le BRS peut être compatible avec le statut de la copropriété :
			- le « droit au bail » est un droit réel qui peut servir d’assiette à une copropriété (comme un bail
			emphytéotique) –art L. 255-7 du Code de la construction et de l’habitation
			- on pourrait même envisager que le BRS porte sur un ou plusieurs lots de copropriété
		
		\subsubsection{L’Usufruit Locatif Social (USL -- issu de la loi \no 2006-872 du 13 juillet 2006 portant engagement national pour le logement)}
		
			Inversement, une copropriété peut être consituée uniquement, pendant un certain temps, de la nuepropriété
			des lots, tandis que l’usufruit de tous les lots est réuni en une seule main
			C’est le principe de l’Usufruit Locatif Social qui repose repose sur le principe du démembrement
			temporaire de propriété sur une période de 15 à 20 ans. Le copropriétaire acquiert la nue-propriété d’un
			bien à un prix décoté, tandis que son usufruit est cédé à un bailleur institutionnel. La pleine propriété se
			reconstitue sans formalités ni frais au terme du contrat. L’investisseur peut dès lors vendre, louer ou
			occuper son bien.
			
			Le nu-propriétaire quant à lui ne perçoit aucun loyer mais il bénéficie d’un régime fiscal favorable et le
			bailleur social lui garantit la libération du bien et sa remise en état à l’échéance de la convention.
			De règles dérogatoires au régime permettent l’administration du bien pendant la phase de
			démembrement, car la copropriété est pratiquement « suspendue » pour la gestion courante. Cependant,
			les droits du nu propriétaire sont préservés pour les décisions qui engagent l’avenir (travaux..)

	\subsection{Le bornage au sein d’une copropriete}
	
		Les propriétaires de différents lots ne peuvent intenter d’action en bornage de la partie privative de leurs
		lots au sein de la copropriété « car la propriété de l’entier immeuble demeure commune »\footnote{
		Cour d'Appel Nancy, 7 janvier 2016, \no 15/00252 – JurisData \no 2016-000162, Loyers et Copropriété 2016 \no 107
		}.
		C’est ce qui est dit par la Cour de Cassation\footnote{arrêt de rejet du 19 novembre 2015, \no de pourvoi: 14-25403, Publié au bulletin
		}, qui approuve la cour d’Appel d’avoir jugé que :
		« L'action en bornage ne peut pas être intenté dans une même copropriété, que se soit pour délimiter des
		parcelles affecter à la jouissance privative de deux copropriétaires distincts, ou des parcelles
		conventionnellement exclues des parties communes et attribuées privativement, ou des parties privatives
		dont la délimitation résulte du règlement de copropriété et de l'état descriptif de division ».
		\chapter[Naissance et disparition du Syndicat]{Naissance et disparition du syndicat des copropriétaires}

\section[Naissance et immatriculation]{Naissance et immatriculation du syndicat des copropriétaires}

	\subsection{Naissance de plein droit lorsque sont réunies les conditions de l’article 1\ier{} et immatriculation}

		\subsubsection{La constitution du syndicat des copropriétaires de plein droit}
		
			Le syndicat des copropriétaires est une des rares personnes morales dépourvues d’acte de naissance ou de constitution. En effet, le Syndicat des Copropriétaires existe dès lors que les conditions prévues par l’article 1\ier{} sont réunies, puisque le statut de la copropriété est d’application impérative.
			
			De ce fait, le législateur n’a pas voulu conditionner la naissance du Syndicat à une quelconque formalité --- il n’existe pas de déclaration préalable à la mise en copropriété contrairement à ce qui s’impose dans certains pays --- de peur de permettre à celui qui devrait accomplir cette formalité --- le promoteur notamment, en cas de construction de l’immeuble, de différer le plus longtemps possible cette formalité afin d’en éviter les contraintes.
			
			Dès qu’il nait à la vie civile, le syndicat des copropriétaires a la personnalité civile, ceci sans aucune forme particulière.
			
			Il suffit donc que l’immeuble soit divisé par lots appartenant à plusieurs personnes pour que le statut s’applique, ceci quand bien même il n’a pas été établi de Règlement de copropriété, d’état descriptif de division et quand bien même aucun syndic n’a été élu :
			
			Cette naissance de plein droit a bien évidemment des effets importants : par exemple la nécessité de s’adresser au syndicat des copropriétaires pour avoir réparation des parties communes ou remboursement des sommes avancées pour le compte du syndicat.\footnote{
			Civ. 3\degre{} Ch. 11 janvier 2012, \no 10-24413, au Bulletin ; D 2012, 219 note Y. Rouquet. En l’espèce l’un des deux copropriétaires avait réalisé des travaux conservatoires sur le sol commun et assigné l’autre copropriétaire pour avoir remboursement de sa quote-part : irrecevabilité de la demande qui aurait dû être engagée contre le syndicat des copropriétaires}
		
		\subsubsection{L’immatriculation obligatoire de certaines copropriétés}
		
			Le fait que les syndicats existent de plein droit dès que se trouvent remplies les conditions posées à l’article 1\ier{} de la loi a permis de constater une méconnaissance des copropriétés par les pouvoirs publics. Certes leur nombre est à peu près connu par l’enquête réalisée tous les cinq ans par l’INSEE et le Fichier des Logements par Commune (FILCOM) recoupé avec les données fiscales donne une idée approximative de ce nombre.
			
			Cette approximation n’avait pas grande importance à l’époque où les copropriétés relevaient totalement du droit privé, si ce n’est parfois la difficulté d’identifier le représentant du syndicat des copropriétaires.
			
			Mais depuis 1994 les autorités publiques sont amenées à participer au redressement des copropriétés en difficulté. Il est donc nécessaire de connaître beaucoup plus précisément non seulement le nombre de copropriétés mais également l’état de ces copropriétés. Suivant la proposition du rapport BRAYE, la loi ALUR a adopté l’obligation d’immatriculation des copropriétés sur un Registre spécifique. Ce sont les nouveaux articles L 711-1 à L 711-7 du \CCH, complété par le décret \no 2016-1167 du 26 août 2016 relatif au registre national d'immatriculation des syndicats de copropriétaires.
			
			\paragraph{Qui tient le Registre des Syndicats de Copropriété ?}
			
				Le registre des syndicats de copropriétaires est tenu par un établissement public de l'État (le Teneur du Registre). Par arrêté en date du 26 octobre 2016, l’ANAH a été désignée en cette qualité à compter du 1\ier{} novembre 2016.
			
			\paragraph{Le Registre des Copropriétés est un Registre dématérialisé}
			
				Le dépôt du dossier et les modifications qui y sont apportées sont dématérialisées ; la déclaration est donc faite par le $\dots$ « Télédéclarant ».
				
			\paragraph{Quels sont les syndicats concernés ?}
			
				Uniquement les syndicats dont les immeubles sont à usage total ou partiel d’habitation (au moins un lot à usage d’habitation).
			
			\paragraph{A qui incombe l’obligation d’immatriculation ?}
				
				La réponse est sans équivoque : Pour les immeubles à construire ou mis en copropriété, cette obligation incombe au notaire qui reçoit l’EDD et le RC\footnote{
					Immatriculation des syndicats de copropriétaires : le rôle du notaire ; Jacques Lafond, JCPN \no 38 – 23 Septembre 2016 ts
				} ; pour les immeubles déjà existants, il incombe au syndic d’effectuer cette immatriculation et de transmettre les actualisations nécessaires. A défaut il s’expose à une amende (pouvant aller jusqu’à \montant{20} par lot) assortie d’une astreinte recouvrées par le Teneur du Registre (art. L 711-6 CCH). Si un acte est reçu par un notaire sur un immeuble non immatriculé, le notaire
				procèdera à l’immatriculation et le syndic paiera l’amende ! Bien évidemment le syndic ne pourra pas se faire rembourser cette amende par le syndicat des copropriétaires. De plus le syndicat non immatriculé ou dont les données ne sont pas à jour ne pourra percevoir de subventions de l’Etat.
				
			\paragraph{Le contenu de l’immatriculation.}
			
				Les informations à porter au Registre sont fixées par Décret en Conseil d’Etat ; il s’agit du Décret \no 2016-1167 du 26 août 2016.
				
				Elles sont de deux catégories (art. L 711-1 CCH) :
				\begin{itemize}
					\item  d’une part les informations permettant d'identifier le syndicat, de préciser son mode de gestion et de connaître les caractéristiques financières et techniques de la copropriété et de son bâti, notamment le nom, l'adresse et la date de création du syndicat ainsi que, le cas échéant, le nom du syndic et le nombre et la nature des lots
					
					\item d’autre part les informations financières, les informations sur une mise sous administration judiciaire, la mise en oeuvre d’une procédure de carence ou de sauvegarde.
							
					Mais les informations « financières » seront limitées si l’immeuble comporte moins de dix lots à usage de logements, de bureaux ou de commerces, dont le budget prévisionnel moyen sur une période de trois ans consécutifs est inférieur à \montant{15 000}.
				\end{itemize}
			
			\paragraph{Les modalités de télédéclaration}
		
				La télédéclaration se fait en plusieurs temps.
				
				\subparagraph{La création d’un compte de télédéclarant.}
				Soit le télédéclarant est un « professionnel » : notaire, syndic, mandataire ad hoc, administrateur judiciaire ; auquel cas il aura un compte général auquel il « rattachera » la copropriété concernée.
				Soit le déclarant est un syndic bénévole, auquel cas il remplira directement le formulaire de sa copropriété.
				
				\subparagraph{La déclaration de la copropriété concernée.}
				Le télédéclarant fournit au Teneur de Registre tous les documents et renseignements prévus par l’arrêté du 26 octobre 2016, Annexe 1 à 7 (certaines annexes concernant les renseignements à fournir chaque année et celles à fournir en cas de changement de syndic, tant par le syndic sortant que par le nouveau syndic).
			
				\subparagraph{L’immatriculation du syndicat des copropriétaires}
				Une fois complétés les renseignements demandés, l’ANAH fournit au Télédéclarant un \no d’im\-ma\-triculation qui sera unique pour le syndicat des copropriétaires tout au long de son existence.
				
				\subparagraph{Conservation des données pendant 5 ans}
								L’article R 711-15 prévoit que les données fournies sont conservées pendant 5 ans
			
			\paragraph{La mise à jour des informations}
			
				Doivent être portées au Registre :
				\begin{itemize}
					\item toute modification dans les données du Registre ;
					\item à l’issue de chaque exercice comptable les données financières actualisées.
				\end{itemize}
			
				C’est ainsi qu’en cas de changement de syndic, le syndic sortant doit informer le Teneur de Registre de la cessation de ses fonctions et son successeur doit à son tour informer le Teneur de Registre de ce qu’il est le nouveau représentant légal du syndicat des copropriétaires.
				
				Précisons que dores et déjà il est possible d’accéder au site du Teneur de Registre et de télécharger les formulaires à compléter sur le site \url{http://www.registre-coproprietes.gouv.fr/} .
			
			\paragraph{Qui peut accéder aux informations portées au Registre ?}
			
				\begin{enumerate}
					\item \textbf{Le notaire}, qui doit porter le numéro d'immatriculation de la copropriété dans les actes de vente.
					\item \textbf{Les copropriétaires} ont un droit d'accès aux données relatives au syndicat dont ils font partie et peuvent solliciter le syndic aux fins de rectification des données erronées.
					\item \textbf{L'État} et ses services ainsi que ses opérateurs, les EPCI compétents en matière d'habitat, les départements et les régions obtiennent du teneur du registre communication des informations du répertoire relatives à chaque copropriété située sur leur territoire.
					\item \textbf{Des tiers}, selon des conditions précisées par décret en Conseil d'État pris après avis de la Commission nationale de l'informatique et des libertés (CNIL).
				\end{enumerate}
			
			\paragraph{Date d’entrée en vigueur de ces nouvelles dispositions.}
				\begin{enumerate}[label=\alph*)]
					\item Tous les immeubles à usage total ou partiel d’habitation dès à présent soumis au statut de la copropriété doivent être immatriculés avant le 31 décembre 2018.
					\item Pour les immeubles neufs ou mis en copropriété : l’immatriculation devra se faire automatiquement à compter du 1er décembre 2016.
				\end{enumerate}
		
	\subsection{Naissance a la date de la division de l’immeuble par lots}
	
		Ne sont soumis de plein droit au statut de la copropriété que tous « les immeubles bâtis dont la propriété est répartie entre plusieurs personnes $\dots$ ».
		
		Par conséquent, le Syndicat des Copropriétaires prend naissance à la plus tardive des deux dates suivantes :
		\begin{itemize}
			\item division de l’immeuble par lots comprenant chacun une quote part de parties communes et une partie privative, division qui se traduit en principe par la rédaction d’un Règlement de copropriété et d’un État descriptif de division ;
			\item existence d’un immeuble bâti, pour les immeubles vendus en VEFA.
		\end{itemize}
		
		\subsubsection{Vente par appartements d'un immeuble ancien déjà construit}
		
			L’article 1\ier{}-1 (rédaction de la loi ELAN) consacre la jurisprudence existante et édicte :
			\begin{quote}
				« En cas de mise en copropriété d’un immeuble bâti existant, l’ensemble du statut s’applique à compter du premier transfert de propriété d’un lot ».
			\end{quote}
			
			C'est le cas de l'immeuble de rapport vendu par appartements par son propriétaire.
			
			Celui-ci va procéder ou plutôt faire procéder par un ou plusieurs hommes de l'art (géomètre, expert, notaire, avocat) à l'établissement d'un Règlement de Copropriété, et d’un état descriptif de division, puis vendre un par un les lots issus de cette division.
	
			Le syndicat des copropriétaires va naître dès la première vente entre l'ancien propriétaire qui sera propriétaires de tous les lots à l'exclusion d'un seul et l'acquéreur de ce lot : en effet dès cette première vente la propriété de l'immeuble est répartie entre plusieurs personnes.
			En pratique on relève très souvent des modifications au Règlement de Copropriété et parfois des contradictions qui tiennent au fait que l'acquéreur a étudié le Règlement de Copropriété qui lui est soumis par le vendeur avant la signature de l'acte authentique de vente et tente d'obtenir divers avantages que le vendeur, désireux d'en finir pour réaliser la vente, accepte de prendre en compte.
		
		\subsubsection{Dissolution de la Société d'Attribution ou retrait d'associé (Sociétés du Titre \II{} de la loi du 16 juillet 1971)}
		
			Ces sociétés constituées pour la construction ou l'acquisition d'un immeuble en vue de sa division par fractions destinées à être attribuées aux associés en propriété ont donc pour vocation d'être dissoutes pour laisser place au régime de la Copropriété.
			
			L'immeuble construit, les associés réunis en Assemblée Générale Extraordinaire approuvent les comptes de construction et décident de la liquidation de la Société. Ils désignent un liquidateur qui aura notamment pour mission d'établir un projet de partage que signeront (ou pourront contester) individuellement les associés. Les Associés se réuniront pour approuver les comptes de liquidation. Il ne restera plus alors qu'à signer l'Acte de Partage. Par cette signature de l'acte de partage chaque associé se voit attribuer la propriété du lot dont il n'avait jusque lors que la jouissance.
			
			Toutefois les associés peuvent préférer rester en société nonobstant l'achèvement de l'immeuble et l'approbation des comptes de construction. Dans cette hypothèse cependant tout associé peut demander à se retirer de la Société et à se voir attribuer la propriété de son appartement : il exerce la faculté de retrait anticipé. Le retrait est simplement constaté par acte authentique signé par le retrayant et le représentant légal de la société.
			
			L'associé ne peut demander son retrait que s'il remplit trois conditions :
			\begin{itemize}
				\item être à jour de tous les appels de fonds nécessaires à la réalisation de l'objet social (la jurisprudence ne prend en considération que les appels de fonds nécessaires à la construction de l’immeuble et non les appels de fonds relatifs à la gestion de l’immeuble une fois construit).
				\item justifier que l'achèvement de l'immeuble social et sa conformité avec l’État descriptif de division ont été constatés par l'Assemblée Générale des Associés.
				\item justifier que l'Assemblée Générale des Associés a décidé des comptes définitifs de l'opération.
			\end{itemize}
			
			Toutefois l'associé peut demander au Tribunal de Grande Instance de faire ces constatations et mettre en œuvre la procédure de retrait malgré le refus ou la passivité des organes de direction de la Société.
			
			Dans les deux cas évoqués (dissolution ou retrait anticipé), la loi sur la copropriété s'appliquera à compter de la signature de l'acte authentique :
			\begin{itemize}
				\item en cas de liquidation il n'existe plus que des copropriétaires.
				\item en cas de retrait la copropriété fait coexister la société et le retrayant ; c'est alors une Copropriété à deux personnes.
			\end{itemize}
		
			Relevons toutefois un intéressant arrêt de la Cour d'Appel de Versailles du 27 juin 1991\footnote{Rev. Dr. Imm 1991, 3\degre{} T, p. 337} qui a affirmé que la Copropriété naissait le jour de l'Assemblée ayant approuvé les comptes de liquidation. C'est en effet cette Assemblée qui procède à l'attribution des lots conformément aux statuts et à l’État descriptif de division. Mais cette décision demeure isolée. L’opinion généralement admise étant que la copropriété naît au moment de l’attribution à chaque associé de sa part divise\footnote{Cf. Givors, Giverdon, Capoulade, La Copropriété Dalloz Ed 2018 \no 111.31}.
		
		\subsubsection{Partage en nature de l'immeuble ou licitation des lots}
		
			Lorsqu'un immeuble échoit à plusieurs héritiers, ceux-ci se trouvent placés sous le régime de l'indivision régie par les articles 812 et suivants du Code Civil. Tant que l'immeuble demeure en indivision, il appartient à une même famille et n'est donc pas soumis au statut de la Copropriété : il n’y a pas attribution des parts aux lots, et chacun est propriétaire d’une quote part virtuelle de l’immeuble.
			
			Cependant, << Nul n'est tenu de rester dans l'indivision >> en sorte que tout indivisaire est en droit de demander à sortir de l'indivision et à se faire attribuer, après partage de l'immeuble, la quote-part d'immeuble à laquelle il a droit.
			
			Préalablement au partage de l'immeuble les héritiers feront établir par le notaire un Règlement de Copropriété et un état descriptif de division.
			
			Si les héritiers ou co-indivisaires sont d'accord sur l'attribution des lots l'immeuble sera partagé en nature. En cas de désaccord des héritiers, les lots seront vendus sur licitation, les héritiers pouvant bien entendu se porter acquéreurs de lots.\footnote{Toulouse 18 Jan. 88 - juris Data \no 043735}
			
			Par contre, dès la première attribution (signature de l'acte de partage ou licitation) l'immeuble entre dans le champ d'application de la loi du 10 juillet 1965.

			Relevons que le nouvel article 1\ier{}-1 de la loi n’envisage comme point de départ du statut de la copropriété dans un immeuble existant que l’hypothèse du premier transfert de propriété du lot, alors qu’il eut été plus exhaustif d’ajouter « ou de la première attribution d’un lot ».
		
		\subsubsection{Vente d’un lot de S.C.I. d’attribution sur poursuites judiciaires.}
		
			Hypothèse plus originale où la Copropriété est imposée par voie de justice : la vente sur poursuites judiciaires d'une partie du patrimoine d'une Société d'Attribution. On peut envisager effectivement le cas où un créancier d'une société d'attribution ayant inscrit une hypothèque conventionnelle ou judiciaire sur l'immeuble propriété de la Société poursuit la vente judiciaire d'une partie de l'immeuble seulement (la valeur de cette partie d'immeuble étant suffisante pour l'indemniser de sa créance). Dans cette hypothèse l'adjudicataire sera copropriétaire de l'immeuble avec la Société saisie.
		
		\subsubsection{Surélévation d'un immeuble appartenant à un propriétaire}
		
			Dans cette hypothèse, le propriétaire d'un immeuble construit va céder à un tiers le droit de surélever son immeuble en y construisant un ou plusieurs étages supplémentaires. Une convention de copropriété devra être établie avec rédaction d'un Règlement de Copropriété et d'un État descriptif de division.
			
			Le statut de la copropriété s’appliquera dès la cession du lot « droit de surélever ».
		
		\subsubsection{Location vente ou location attribution}
		
			Si la location-vente est fréquente appliquée à un lot d'un immeuble déjà soumis au droit de la Copropriété, il arrive également que l'immeuble soit réalisé par une société qui soumettra l'ensemble au régime de la location-vente. Hypothèse qui se rencontre essentiellement dans le domaine de la Coopérative H.L.M.
			
			La société d'H.L.M. demeurera propriétaire jusqu'à ce que soit réalisée la promesse de vente. Mais dès l'origine un état descriptif de division et un Règlement de Copropriété auront été établis. Rappelons que la location accession est régie par les dispositions de la loi du 17 juillet 1984 qui emporte différentes conséquences sur le statut de la Copropriété, par exemple participation du locataire acquéreur aux Assemblées Générales de copropriété et même au Conseil Syndical.
			
	\subsection{Naissance a la date de l’achèvement de l’immeuble}
	
		\subsubsection{L’article 1\ier{}-1 de la loi du 10 juillet 1965}
	
			Pour la réalisation d’un immeuble à construire, les promoteurs peuvent constituer entre eux une société, de vente d'immeuble (art. L 210-1 et s. CCH pour les sociétés civile) , dont l'objet est l'acquisition et la
			vente d'appartements. Les fonds nécessaires à la construction étant réunis, partie en fonds propres des promoteurs et partie grâce aux prêts promoteurs consentis par les banques.
			
			Le candidat acquéreur n'est pas associé à la Société de construction mais achète directement son local qui lui sera livré soit achevé soit en l'état futur d'achèvement,
			
			Si l'immeuble est vendu << achevé >>, la Copropriété prend naissance dès la première vente.
			
			Si l'immeuble est vendu en l'état futur d'achèvement la Copropriété prend naissance dès achèvement de l'immeuble.
			C’est ce qui est expressément mentionné dans l’article 1\ier{}–1 nouveau de la loi du 10 juillet 1965 qui est ainsi rédigé :
			« Pour les immeubles à construire, le fonctionnement de la copropriété découlant de la personnalité morale du syndicat de copropriétaires prend effet lors de la livraison du premier lot ».
			Ce faisant le législateur de 2018 n’a fait que consacrer légalement une jurisprudence relativement constante\footnote{Civ 3\degre{} 20 déc 1976, JCP 1978, II, 18800, Obs. Guillot.}, quand bien même certaines décisions de cours d’appel avaient cru devoir affirmer que le statut aurait été applicable dès la première vente en l’état futur\footnote{Paris 23\degre{} Ch. 22 sep 1995, Loyers et Copropriété 1996 \no 80 ; Versailles 22 mai 1984, Rev. Dr. imm. 1984 p. 351 ; Aix 16 avril 1992 Loyers et Copropriété 1993 \no 275.}.
	
		\subsubsection{Construction d'un bâtiment unique.}
		
			S'agissant de vente en l'état futur d'achèvement le décret d'application de la loi du 3 janvier donne une définition de cet achèvement :
			\begin{quote}
			Art 261-1 C.C.H.\\
			<< L'immeuble ($\dots$) est réputé achevé ($\dots$) lorsque sont exécutés les ouvrages et sont installés les éléments d'équipement qui sont indispensables à l'utilisation, conformément à sa destination, de l'immeuble faisant l'objet du contrat. Pour l'appréciation de cet achèvement les défauts de conformité avec les prévisions du contrat ne sont pas pris en considération lorsqu'ils n'ont pas un caractère substantiel, ni les malfaçons qui ne rendent pas les ouvrages ou éléments ci-dessus précisés impropres à leur utilisation >>.
			\end{quote}

			Cet achèvement n’est donc pas subordonné à la livraison de tous les lots, ou encore à l’obtention du certificat de conformité.
		
		\subsubsection{L’immeuble vendu en l’état futur d’inachèvement}
		
			La loi ELAN a modifié le \CCH sur la vente en l’état futur d’achèvement, en ajoutant à l’article L 261-15 qui traite du contrat préliminaire à la VEFA un \II{} ainsi rédigé :
			\begin{quote}
				« \II{}. – Le contrat préliminaire peut prévoir qu’en cas de conclusion de la vente, l’acquéreur se réserve l’exécution de travaux de finition ou d’installation d’équipements qu’il se procure par lui-même. Le contrat comporte alors une clause en caractères très apparents stipulant que l’acquéreur accepte la charge, le coût et les responsabilités qui résultent de ces travaux, qu’il réalise après la livraison de l’immeuble. ($\dots$)
				
				Un décret en Conseil d’État précise les conditions d’application du présent \II{}, notamment la nature des travaux dont l’acquéreur peut se réserver l’exécution »
			\end{quote}
			
			Bien évidemment ce texte n’entrera en application qu’après publication du Décret.
			
			Pour autant il est ici question de travaux d’achèvement et non de parachèvement en sorte que les travaux que l’acquéreur se réserve peuvent porter sur des ouvrages ou éléments d’équipement indispensables à l’utilisation du lot (équipements de la salle de bains par exemple).
			
			Dans cette hypothèse faudra-t’il attendre que l’acquéreur ait réalisé ces travaux ou éléments d’équipement indispensables pour faire application du statut ?
			
			Pour des raisons pratiques et de texte, la réponse nous paraît devoir être négative et la date à retenir pour l’application du statut devrait être la livraison du premier lot $\dots$ même non achevé.
		
		\subsubsection{Pluralité de bâtiments}
		
			Il est très délicat en revanche de déterminer l'achèvement d'un ensemble de bâtiments livrés successivement (Problème des tranches successives de construction) et il est en conséquence difficile de déterminer la date à laquelle le statut de la Copropriété va recevoir application.
			
			Or, la loi s'applique à << tout immeuble bâti ou groupe d'immeubles bâtis >>.
			
			Ce texte est susceptible de deux interprétations :
			\begin{itemize}
				\item soit l'on considère que la loi n'est applicable qu'à l'achèvement du groupe d'immeubles --- lorsque le groupe est construit ;
				\item soit l'on considère que la loi s'applique dès achèvement du premier bâtiment --- dès qu'il y a un immeuble bâti.
			\end{itemize}
		
			Dans un arrêt du 4 décembre 1979\footnote{3\degre{} Ch Civ. B \no 218 p. 171} la Cour de Cassation a admis implicitement que la loi devait recevoir application dès achèvement du premier bâtiment.
			Alors que les autres bâtiments n'étaient pas construits, ce premier bâtiment achevé s'était constitué en syndicat secondaire. Une telle constitution d'un syndicat secondaire n'était possible que dans l'hypothèse où la loi de 1965 recevait application. C'est ce qu'a décidé la Cour de Cassation.
			
			Cette solution n'est plus guère discutée aujourd'hui : en fait elle constitue une réponse pratique à un problème difficile. La solution contraire aurait créé un << vide juridique >> : quel aurait été en effet le régime applicable au bâtiment achevé qui, par suite des retards souvent apportés à la réalisation des travaux de construction par << tranches successives >>, serait resté seul achevé pendant plusieurs mois, voire plusieurs années ?
			
			Au demeurant la rédaction du nouvel article 1\ier{}-1 qui fait application du statut dès la livraison du premier lot ne fait pas de distinction entre la VEFA passée dans le cadre d’un bâtiment unique ou dans le cas d’une pluralité de bâtiments.
		
			La jurisprudence précitée devrait continuer de s’appliquer : le statut s’applique dès la livraison du premier lot dans le premier bâtiment ; pour les bâtiments non construits, le vendeur sera considéré comme propriétaire des lots correspondant à ces bâtiments, mais ne participera qu’aux charges générales.
		
		\subsubsection{Le problème du lot transitoire}
			
			\paragraph{Historique}
			
				Étant admis que la loi reçoit application dès achèvement du premier bâtiment, doit-elle être appliquée au seul bâtiment achevé ou à l'ensemble des lots, que ceux-ci soient construits ou non ?
				
				Ce d'autant que la technique le plus souvent appliquée dans la Construction par << Tranches Successives >> est celle dite du lot transitoire\footnote{Autrement appelé selon Mr \nom{CAPOULADE} dans son étude pour le Rapport de la Cour de Cassation de 1994 : << lot d'attente >>, << lot par anticipation >> ou << lot parthénogénique >>}.
				
				\begin{quote}
					En pratique le Promoteur de l'ensemble constitue des lots intermédiaires, généralement un lot par bâtiment à construire. Par exemple si cinq bâtiments sont prévus, l’État descriptif de division comprendra cinq lots intermédiaires 100, 200, 300, 400 et 500.

					Chacun de ces lots étant défini non par l'addition d'une partie privative et de tantièmes généraux de copropriété, mais par un droit de construire un bâtiment qui comprendra un nombre de tantièmes généraux de la Copropriété, correspondant à l’ensemble des tantièmes du futur bâtiment.
				
					Lors de l'édification du premier bâtiment le lot 100 est << éclaté >> en autant de lots qu'il y aura de locaux principaux et accessoires dans ce bâtiment, de telle sorte que le lot 100 sera annulé et remplacé par exemple par les lots 101 à 152. Avec l'édification du deuxième bâtiment le lot 200 sera remplacé par les lots 201 à 260, etc.
				\end{quote}
				
				Ce lot transitoire est-il véritablement un lot de copropriété alors qu'il ne comprend pas de parties privatives ? La doctrine avait en effet douté de l'applicabilité de la loi à ce type de lot qui ne comporte pas de véritable "partie privative", ou plutôt, dont la partie privative ne présente pas de consistance matérielle.\footnote{Giverdon et Bouyeure, ADMINISTRER déc 81, p. 7.}
				
				La Cour de Cassation a finalement admis l’existence des lots transitoires, notamment par un important arrêt du 15 novembre 1989\footnote{Bulletin \no 213, p. 117; D 90. J. 216, note Giverdon et Capoulade} à propos de la saisie immobilière de lots transitoires
				\begin{quote}
					<< Retenant que chacun des lots saisis, placé par l'auteur de la division sous le régime de la copropriété, comprenait selon l'état descriptif, le droit exclusif d'utiliser une surface déterminée du sol pour y édifier des constructions conformément à un permis de construire délivré..., ainsi qu'une quote-part de la propriété du sol et des parties communes, la cour d'appel a exactement décidé... que le lot privatif du débiteur constituait un immeuble par nature pouvant faire l'objet d'une saisie immobilière >>.
				\end{quote}
				Et surtout par un arrêt SCI SALENGRO du 3 juin 1991\footnote{3\degre{} Ch. Civ. du 3 juin 1991 (JCP 92, \I{}, 3560)} aux termes duquel le lot transitoire ou intermédiaire est un véritable lot de copropriété, en sorte que dès l'achèvement du premier bâtiment le régime de la Copropriété va s'appliquer à l'ensemble immobilier, parties bâties et non bâties et que la SCI promotrice, en sa qualité de propriétaire des lots transitoires participera aux Assemblées Générales de la Copropriété.
				
				On retiendra également un arrêt du 14 novembre 1991 par lequel la cour de cassation a rejeté un pourvoi à l'encontre d'un arrêt par lequel la cour d'appel
				\begin{quote}
					« Retenant à bon droit que les lots dits << transitoires >> ne sont pas assujettis à un régime particulier, en a exactement déduit$\dots$ que la SCI était copropriétaire au sens de la loi du 10 juillet 1965 »
				\end{quote}.
	
			\paragraph{Le nouvel article 1\ier{} alinéas 2 et 3} 

				La loi ELAN a tenu compte de cette jurisprudence pour ajouter un nouvel alinéa à l’article 1er qui est ainsi rédigé :
				\begin{quote}
					« Ce lot peut être un lot transitoire. Il est alors formé d’une partie privative constituée d’un droit de construire précisément défini quant aux constructions qu’il permet de réaliser sur une surface déterminée du sol, et d’une quote-part de parties communes correspondante. (art. 59 bis D)
					
					« La création et la consistance du lot transitoire sont stipulées dans le règlement de copropriété. »
				\end{quote}
			
			\paragraph{La construction future doit être précisément définie}
	
				Ce nouvel article est très important.
				
				La jurisprudence ayant validé le lot transitoire acceptait parfaitement que celui-ci ne soit en aucune façon défini. Par exemple, un arrêt du 3 novembre 2016\footnote{Civ. 3\degre{} Ch. 16 novembe 2016, Pourvoi \no 15-14895 15-15113, Inédit} a considéré qu’était parfaitement valable le lot appartenant à une société , constitué du droit de construire sur une partie du terrain, sans aucune précision quant à la future construction alors que le demandeur au pourvoi avait fait valoir que : « le droit de construire mentionné dans la description d'un lot de copropriété ne peut en constituer la partie privative qu'à la condition d'être concrètement défini dans sa consistance et son implantation » . De façon lapidaire la cour de cassation a simplement répondu : « a cour d'appel a retenu, à bon droit, que le lot de la société constituait un lot privatif composé pour sa partie privative du droit exclusif d'utiliser le sol pour édifier une construction et d'une quote-part de parties communes et en a exactement déduit que la société était titulaire d'un lot transitoire.
				
				Jurisprudence d’autant plus favorable au titulaire du lot que le droit à construire sur le lot transitoire n'est pas soumis aux règles d'autorisation de la copropriété de l’article 25 dès lors qu'en vertu du règlement de copropriété son titulaire bénéficie du droit de construite « tous bâtiments et constructions »\footnote{Cass. Civ. 3e 8 juin 2011 ; \no de pourvoi: 10-20276 Publié au bulletin Revue des Loyers juillet 2011 page 322, note Jean-Marc \nom{Roux}}. De même il a été jugé que le promoteur, qui a édifié sur le lot transitoire un bâtiment à usage de garage et qui n'a fait qu'user d'un droit reconnu par le règlement de copropriété, n'était pas tenu de solliciter pour construire l'autorisation de l'assemblée générale dans la mesure où la nature et l’affectation étaient définies\footnote{Cass. Civ. 3e 4 novembre 2010}.
				
				Cette jurisprudence avait un effet déstabilisateur à l’intérieur de la copropriété du fait de la totale liberté dont bénéficiait le titulaire du lot transitoire.
				Le nouvel article 1er alinéa 2 réagit contre cette jurisprudence en imposant justement que le lot transitoire soit précisément défini :
				\begin{itemize}
					\item quant à sa surface au sol ;
					\item quant à la construction à réaliser ;
					\item quant aux quotes-parts attribuées à ce lot transitoire.
				\end{itemize}
				
				De plus, ce lot transitoire doit non seulement figurer comme tel dans l’état descriptif de division mais il doit également être mentionné et décrit dans le règlement de copropriété.\footnote{Nous verrons ultérieurement que l’état descriptif de division ne se confond pas avec le règlement de copropriété.}
			
			\paragraph{Dispositions transitoires prévues par la loi ELAN}
	
				En principe ces nouvelles dispositions ne devraient s’appliquer qu’aux lots transitoires créés postérieurement à la publication de la loi ELAN, en application du principe de non-rétroactivité des lois.
				
				La loi ELAN en décide autrement en imposant des dispositions transitoires ainsi rédigées :
				\begin{quote}
					– Les syndicats des copropriétaires disposent d’un délai de trois ans à compter de la promulgation de la présente loi pour mettre, le cas échéant, leur règlement de copropriété en conformité avec les dispositions relatives au lot transitoire de l’article 1er de la loi \no 65-557 du 10 juillet 1965 fixant le statut de la copropriété des immeubles bâtis.
					
					À cette fin et si nécessaire, le syndic inscrit à l’ordre du jour de chaque assemblée générale des copropriétaires organisée dans ce délai de trois ans la question de la mise en conformité du règlement de copropriété. La décision de mise en conformité du règlement de copropriété est prise à la majorité des voix exprimées des copropriétaires présents ou représentés.
				\end{quote}
				
				Ce texte posera certainement des difficultés d’application : Sera-t’il seulement applicable lorsque l’état descriptif de division comportera un lot transitoire non défini ou pourra-t’il être recevoir application dans l’hypothèse où le droit de construire figurera seulement dans le règlement de copropriété.
				
				En tout cas dès la première assemblée générale suivant la publication de la loi ELAN les syndics devront vérifier si les états descriptif de division comprennent des lots transitoires non précisément définis ni quant à la surface au sol ni quant à la construction projetée et en ce cas proposer à l’assemblée générale de modifier le règlement de copropriété pour y inscrire la définition du lot transitoire.
				
				Comment se fera cette définition ? Il semble difficile d’imposer au titulaire du lot une description du bâtiment à construire.
				
				A quelle majorité l’assemblée générale se prononcera sur cette modification du règlement de copropriété qui ne concerne ni l’administration de l’immeuble ni l’usage des parties communes , seules hypothèses de modifications du règlement de copropriété à la double majorité de l’article 26 de la loi ? Doit-on considérer que cette adaptation du règlement de copropriété relève de la majorité de l’article 24 f) de la loi ?
				Ce droit de construire sera-t’il caduc si l’assemblée générale n’a pas défini le lot transitoire dans le délai de 3 ans de la publication de la loi ?

			\paragraph{Jurisprudence complémentaire sur le lot transitoire}
			
				La jurisprudence, tirant les conséquences de l’’existence du lot provisoire, condamne les clauses du règlement de copropriété qui ne font participer aux assemblées générales et aux charges que les propriétaires des lots construits\footnote{Civ 3\degre{} 30 juin 1998 - Administrer janvier 1999 p. 62, note \nom{CAPOULADE}.} : le titulaire d’un lot transitoire doit obligatoirement être convoqué aux assemblées générales (à défaut il pourra en demander l’annulation pendant dix ans), et doit participer aux charges communes générales (assurances et frais de gestion notamment) $\dots$
				
				La modification d’un droit de construire attaché à un lot dit << transitoire >> pour l’étendre à des activités annexes et/ou complémentaires relève de la double majorité de l’article 26 de la loi du 10 juillet 1965 et non de l’unanimité\footnote{Cass. Civ. 3e 30 novembre 2010 30 novembre 2010 \no de pourvoi: 09-72386, non publié}.
	
\section{Disparition du syndicat des coproprietaires}

	Nous venons de voir précédemment dans quelles conditions un immeuble qui appartenait à une seule personne se trouve du fait de sa division entre plusieurs propriétaires soumis au statut de la copropriété.
	
	L’hypothèse inverse peut se rencontrer : un immeuble divisé en plusieurs lots appartenant à des copropriétaires différents va devenir la propriété d’une seule personne\footnote{Il ne faut pas confondre cette hypothèse avec la situation qui résulte de l’application des dispositions de l’article 28 de la loi sur la copropriété en cas de retrait d’un ou plusieurs lots de la copropriété d’origine. En effet aux termes de cette loi, la scission entraîne la disparition du syndicat des copropriétaires d’origine. Cette question sera traitée ultérieurement.}.
	
	La disparition du syndicat peut également résulter de son $\dots$ annulation par le juge !
	
	\subsection{La disparition du syndicat pour l’avenir}
	
		\subsubsection{La dissolution de plein droit du syndicat des copropriétaires, et sa survie pour les besoins de la liquidation}
		
			La réunion de tous les lots entre les mains d’un même propriétaire entraîne, de plein droit, la disparition du Syndicat\footnote{Cass. Civ. 3e 28 janvier 2009 pourvoi: 06-19650 Publié au bulletin} du fait que disparaît l’une : « la réunion de tous les lots entre les mains d'un même propriétaire avait entraîné de plein droit la disparition de la copropriété ».

			C’est pourquoi la cour de cassation\footnote{Civ 3ème Ch 4 juillet 2007, pourvoi: 06-11015 Publié au bulletin, Loyers et Copropriété 2007 \no 204} dans un arrêt du 4 juillet 2007 pose le principe selon lequel la réunion de tous les lots entraîne la disparition de la copropriété.
			
			Ce principe étant posé, reste à en découvrir les conséquences.
			
			\vskip
			
			Aux termes de l’article 1844–8 du code civil la société civile se survie pour les besoins de sa liquidation. Aucune disposition spécifique n’a été prévue dans la loi du 10 juillet 1965. Un syndicat des copropriétaires n’est pas une société ! Dès lors peut-on soutenir que le syndicat des copropriétaires se survie pour les besoins de sa liquidation ?
			
			\vskip
			
			Une réponse positive à cette question a été donnée par la cour de cassation\footnote{Civ 3ème Ch 5 déc 2007, pourvoi: 07-11188 07-11204 Publié au bulletin Loyers et Copropriété 2008 \no 43} dans un arrêt du 5 décembre 2007 dont les enseignements sont les suivants :
			\begin{itemize}
				\item la personnalité morale du syndicat subsiste pour les besoins de sa liquidation,
				\item la liquidation du syndicat des copropriétaires ne peut se faire sous le régime du droit de la copropriété - le liquidateur peut être désigné à l’unanimité des anciens propriétaires
				\item le Syndicat peut également être représenté par un mandataire ad hoc dans les procédures en cours, à la demande de tout intéressé.\footnote{Cour de cassation chambre civile 3 27 avril 2017 \no de pourvoi: 16-11278 Non publié au bulletin Jurisdata}
			\end{itemize}
			La demande était fondée sur l’article 1844-8 code civil. La cour de cassation ne vise pas cet arrêt : cela signifie-t’il qu’elle écarte effectivement les règles de liquidation des sociétés civiles ?
			
			\begin{quote}
				(la cour d’appel ayant) retenu à bon droit qu'en cas de réunion de tous les lots entre les mains d'une même personne, aucune disposition de la loi du 10 juillet 1965 n'avait vocation à régir la liquidation de la copropriété et que sa personnalité morale subsistait pour les besoins de sa liquidation, et constaté que l'assemblée générale tenue entre tous les anciens copropriétaires après la vente des lots avait désigné à l'unanimité M. Z$\dots$ aux fonctions de liquidateur amiable, la cour d'appel en a exactement déduit, abstraction faite d'un motif surabondant relatif à la représentation de chaque copropriétaire par le liquidateur, que la fin de non-recevoir soulevée par M. X$\dots$ devait être rejetée
			\end{quote}
			
		\subsubsection{Disparition d’une copropriété horizontale}
	
			Dans une copropriété horizontale les propriétaires de pavillons souhaitent mettre fin à la copropriété.
		
			Si cette demande est consécutive à la cession des VRD à la Commune, les copropriétaires décideront simplement de se partager le terrain d’assiette de la copropriété, en sorte que chaque propriétaire
			possèdera en pleine propriété son pavillon et le terrain d’assiette de la construction (et éventuellement le jardin).
			
			Il ne s’agira pas d’une scission de la copropriété, mais d’un partage pur et simple qui peut s’effectuer à la suite d’une décision unanime de tous les copropriétaires décidant de s’attribuer mutuellement le sol suivie d’un acte notarié de partage auquel comparaîtront tous les copropriétaires. Cette décision sera complétée par la désignation d’un liquidateur du syndicat des copropriétaires.
			
			S’il subsiste des VRD communs, l’assemblée générale pourra décider de les apporter en propriété à une ASL qu’elle constituera concomitamment\footnote{Cf. La revue fiscale du patrimoine \no 7-8, Juillet 2012, form. 7 par Jacques Lafond}. Mais attention lorsque les biens des copropriétaires sont grevés d’hypothèques.
	
		\subsubsection{Le statut de la copropriété ne s’applique pas à la liquidation du syndicat}
		
			Dès lors que la loi du 10 juillet 1965 ne s’applique plus à l’immeuble dont la propriété n’appartient qu’à une seule personne, on voit mal pour quelle raison le statut de la copropriété survivrait à la disparition du syndicat des copropriétaires.
			
			C’est la position qu’adopte la cour de cassation dans l’arrêt précité du 5 décembre 2007 :
			\begin{quote}
				« (la cour d’Appel) ayant retenu à bon droit à qu’en cas de réunion de tous les lots entre les mains d’une même personne, aucune disposition de la loi du 10 juillet 65 n’avait vocation à régir la liquidation de la copropriété $\dots$ »
			\end{quote}
			
			Dès lors en cas de dissolution du syndicat des copropriétaires, il appartient :
			\begin{itemize}
				\item - soit à l’assemblée générale du syndicat des copropriétaires avant la cession du dernier lot à une seule et même personne ;
				\item soit aux anciens copropriétaires agissant à l’unanimité ;
				\item soit à tout intéressé --- et principalement au propriétaire unique --- une fois la dissolution de plein droit réalisée ;		
			\end{itemize}
			de désigner ou de faire désigner par le juge statuant sur requête un liquidateur au syndicat des copropriétaires.
			
			Ce liquidateur pouvant bien évidemment être désigné avant dissolution du syndicat des copropriétaires par l’assemblée générale des copropriétaires sous condition suspensive de la réunion de tous les lots entre les mains d’une seule et même personne. Il peut être également dit par l’assemblée que le liquidateur ne prendra ses fonctions qu’au jour où l’ensemble des lots aura été réuni entre les mains d’une seule personne.

			Ce liquidateur se verra conférer les pouvoirs habituels du liquidateur d’une personne morale.
		
		\subsubsection{Les conséquences de la liquidation du Syndicat des Copropriétaires}
		
			De questions complémentaires doivent être résolues :
			\begin{itemize}
				\item qui doit être convoqué par le liquidateur pour l’approbation de ses comptes de liquidation ?
				\item les droits et actions du syndicat des copropriétaires sont-ils transmis de plein droit à l’acquéreur unique de tous les lots ?
			\end{itemize}
			Ici encore nous ne disposions d’aucune jurisprudence sur la question. Certes, l’article 28 de la loi, en cas de scission d’une copropriété, affirme bien que celle-ci emporte dissolution du syndicat d’origine.
			
			\paragraph{Qui est concerné par les opérations de liquidation ?}
			
			Le liquidateur devra convoquer en assemblée l’ensemble des personnes qui étaient copropriétaires au jour de la dissolution de la copropriété. C’est entre ces personnes qu’il fait les comptes et c’est à elles qu’il demande le complément nécessaire à l’apurement des dettes du syndicat en cours de liquidation.
			
			\paragraph{Les droits et obligations des copropriétaires sont-ils transmis à l’acquéreur unique ?}
			
				\subparagraph{Principe général}
				
				Contrairement à une opinion qui avait été émise par certains auteurs, l’acquéreur unique ne se trouve pas aux droits et obligations du syndicat des copropriétaires.
			
				Certes, il convient de citer un arrêt de la cour de cassation\footnote{Civ 3ème 12 septembre 200, Loyers et Copropriété 2007 \no 228. Bull Civ.} en date du 12 septembre 2007 qui reconnaît à l’acquéreur unique le bénéfice de l’action à l’encontre de l’assureur dommages ouvrage appartenant précédemment au syndicat des copropriétaires.
		
				Mais cet arrêt ne signifie en aucune façon que l’acquéreur se trouve au droit du syndicat des copropriétaires : il reprend purement et simplement le principe relatif à l’assurance dommages ouvrage
				qui veut que le l’action bénéficie aux propriétaires successifs de l’immeuble pendant la durée de la garantie décennale.
				
				Dans un arrêt de cassation\footnote{Civ. 3ème 2 octobre 2013, \no de pourvoi: 12-17098, non publié au Bulletin ; Loyers et Copropriété jan 2014 \no 27} du 2 octobre 2013, la Cour de Cassation, au visa de l’article 1165 du code civil et 14 de la loi de 1965, rappelle : «qu’en l'absence de clause expresse, la société bénéficiaire de l'apport, venue aux droits de la société apporteuse, n'était pas tenue de plein droit des obligations personnelles du syndicat ». La cour de cassation prend sa décision sur le fondement de l’art. 1165 du code civil aux termes duquel les convenbtions n’ont d’effet qu’entre les parties.
				
				\subparagraph{Le nouvel article 28 de la loi}
				
				Il est dérogé au principe que nous venons d’évoquer par le nouvel article 28 de la loi en matière de scission de copropriété. En effet cet article comprend désormais trois nouveaux paragraphes ainsi rédigés :
				\begin{quote}
					« La répartition des créances et des dettes est effectuée selon les principes suivants :
				
					« 1\degre{} Les créances du syndicat initial sur les copropriétaires anciens et actuels et les hypothèques du syndicat initial sur les lots des copropriétaires sont transférées de plein droit aux syndicats issus de la division auquel le lot est rattaché, en application du 3\degre{} de l’article 1251 du code civil ;
				
					« 2\degre{} Les dettes du syndicat initial sont réparties entre les syndicats issus de la division à hauteur du montant des créances du syndicat initial sur les copropriétaires transférées aux syndicats issus de la division. » ;
				\end{quote}
				
				On peut dès lors considérer que ce nouveau texte est une exception au principe général qui ne s’applique qu’à l’hypothèse de la scission d’une copropriété.
				
				Il est regrettable que nous ayons désormais deux régimes distincts de liquidation d’un syndicat des copropriétaires selon les causes de cette disparition !
				
				Ce nouveau texte légal est au demeurant surprenant puisqu’il fait « application » de l’article 1251 3\degre{} (ancien) du code civil selon lequel la subrogation a lieu « au profit de celui qui, tenu pour d’autres au paiement de la dette, avait intérêt de l’acquitter » : certes désormais le syndicat issu de la scission est bien tenu pour le syndicat d’origine en liquidation, des dettes de ce dernier, mais la transmission des créances de l’ancien syndicat sur les copropriétaires a lieu au profit du nouveau syndicat avant que ce dernier se soit acquitté des dettes de l’ancien syndicat.

	\subsection{La disparition du syndicat du fait de son annulation par le juge}
	
		Cette question concerne l’annulation d’un syndicat secondaire irrégulièrement créé.
		
		Dans une affaire PIGUET c/ Syndicat Le Consul\footnote{Civ. 3\degre{} 20 mai 2009 pourvoi: 07-22051 08-10043 08-10495 Publié au bulletin Loyers et Copropriété sep 2009 \no 218}, le Règlement de copropriété avait prévu l’existence d’un syndicat principal et de plusieurs syndicats secondaires alors que les conditions matérielles de séparation des bâtiments n’étaient pas réunies (cf. Troisième Partie – Le Syndicat secondaire).
		
		Quarante ans après l’entrée en application du Règlement de copropriété les consorts PIGUET ont demandé au juge d’annuler la clause du Règlement de copropriété prévoyant la constitution des syndicats secondaires. La Cour d’Appel annule la clause du Règlement de copropriété mais refuse de déclarer inexistants les syndicats secondaires : « Le syndicat secondaire institué par une clause réputée non écrite du Règlement de copropriété est censé n’avoir jamais existé ».
		
		La Cour d’Appel refuse de prononcer cette annulation affirmant que l’annulation du syndicat secondaire doit être prononcée pour l’avenir seulement (à compter de la décision annulant les syndicats secondaires). Pourvoi des consorts Piguet : en statuant ainsi la cour d’appel a violé les dispositions de l’article 43 de la loi. La Cour de Cassation rejette le pourvoi :
		\begin{quote}
			« Attendu que la Cour d’appel retient à bon droit que, même s’ils ont été institués par une clause du Règlement de copropriété ultérieurement réputée non écrite, les syndicats secondaires n’en ont pas moins acquis dès leur constitution et jusqu’à la décision ordonnant leur suppression, une personnalité juridique opposable aux tiers »
		\end{quote}.
		
		Si les termes de cet arrêt ne sont guère convaincants sur le plan des principes, la solution n’en est pas moins heureuse au plan pratique.
	
	\subsection{La question de la mise en redressement judiciaire ou liquidation du syndicat}
	
		\subsubsection{Discussion antérieure à la loi du 21 juillet 1994}
		
			Le problème se pose lorsque le syndicat est débiteur d'un tiers, par exemple d'un entrepreneur pour des travaux réalisés dans l'immeuble. Les copropriétaires ne paient pas leur quote-part de charges relative à ces travaux. Comment le créancier peut-il recouvrer sa créance ?

			La pratique a imaginé dans ce cas que le créancier assigne le syndicat des copropriétaires en liquidation judiciaire. En effet la loi sur la faillite de 1967, comme la loi du 25 janvier 1985 sur le redressement et la liquidation judiciaires des entreprises ont édicté que ce régime s'appliquait tant aux sociétés commerciales qu'aux sociétés civiles, associations ou syndicats.
			
			Cependant la doctrine admettait difficilement qu'un syndicat de copropriétaires qui n'est normalement propriétaire d'aucun bien (puisque les parties privatives comme les parties communes sont la propriété des copropriétaires), puisse être soumis à un régime de $\dots$ la liquidation de biens.
			
			De même un syndicat peut-il réellement se trouver en état de cessation de paiements dès lors que les débiteurs véritables sont les copropriétaires eux-mêmes ?
			
			De plus, admettre une procédure de redressement se concevait parfaitement dans la mesure où elle pouvait aboutir à un Plan de Redressement. Mais dans le cas contraire, le redressement se trouvait transformé en liquidation; d'où disparition de la personne morale.
			
			Or, peut-on faire disparaître un syndicat de copropriétaires autrement que par la réunion de toutes les parties d’immeubles entre les mains d'une seule personne ? Liquider un syndicat de Copropriété, c'est en définitive donner naissance à un nouveau syndicat remplaçant de plein droit celui qui existait antérieurement, sans pour autant que soient réglées les difficultés à l'origine de la liquidation de biens du premier syndicat.
			
			Par contre, refuser la procédure légale de redressement ou de liquidation d'un syndicat paraissait difficile dès lors que la loi s'appliquait à toutes les personnes morales sans distinction. De plus, il est vrai que la procédure de redressement présentait de sérieux avantages, surtout pour les syndicats en graves difficultés financières : on a d'ailleurs vu des syndics d'importantes résidences banlieusardes à Sarcelles ou près de Marseille, déposer le bilan de leur Copropriété.
			
			La loi du 21 juillet 1994 modifiant le statut de la Copropriété a tranché la question\footnote{Étant précisé que cet article était à l’origine l’article 29-4 de la loi et qu’il est devenu l’article 29-6 lors de la modification apportée par la loi SRU, puis 29-14 après publication de la loi ALUR !} :
			\begin{quote}
				ARTICLE 29-14 DE LA LOI : « Les procédures prévues au livre VI du code de commerce ne sont pas applicables aux syndicats de copropriétaires ».
			\end{quote}
			
			Par la même loi du 21 juillet 1994, le Parlement a doté les Copropriétés d'une arme nouvelle qui s'apparente à la procédure de l'Administrateur ad hoc prévue par la loi du 1er mars 1984 pour les Entreprises en Difficulté et que nous retrouverons ultérieurement lorsque nous étudierons l'administration judiciaire des syndicats de Copropriété.

			Précisons que la loi ALUR a modifié très sensiblement la procédure l’administration des copropriétés en difficulté la rapprochant de plus en plus de la procédure de redressement judiciaire des sociétés ; elle a elle-même été remaniée par la loi du 27 janvier 2017 dite Égalité et Citoyenneté.
			
		\subsubsection{L’action des créanciers contre les copropriétaires}
		
			Les créanciers –-- tant que le syndicat des copropriétaires n’est pas soumis à la procédure des articles 29-1 et suivants de la loi\footnote{Dispositions qui font l’objet de la dernière partie du cours (Cf. Poly 2)} --- peuvent bien évidemment agir contre le syndicat des copropriétaires pour obtenir sa condamnation au paiement de leur créance.
			
			Notamment ces créanciers disposent de la faculté de mettre en oeuvre les dispositions de l’article 29-1 A aux fins de désignation d’un mandataire ad hoc à copropriété en pré-difficulté\footnote{Cf. infra Chapitre XII, Section I}
			Le syndicat des copropriétaires n’a pas de patrimoine et ne peut pas être mis en en règlement judiciaire ou en liquidation de biens.
			
			Certes, le syndic devra recouvrer auprès des copropriétaires les sommes auquel il aura été condamné et nous avons vu qu’en ce cas ces condamnations étaient charges communes générales réparties entre tous les copropriétaires au prorata de leurs tantièmes généraux de copropriété, quelle que soit la cause de la condamnation au profit du tiers créancier.
			
			Si le syndic omet de recouvrer les sommes dues il néglige d’exercer les droits et actions des copropriétaires et nous verrons qu’en ce cas un administrateur judiciaire pourra être désigné à la copropriété, à la requête du créancier sur le fondement de l’article 24 f) du décret du 17 mars 1967\footnote{Civ 3\degre{} Ch 16 sep 2003, Loyers et Copropriété, comm 244}. Pour autant la question s’est rapidement posée de savoir si les créanciers du syndicat des copropriétaires ne disposaient pas d’une action directe à l’encontre des copropriétaires eux-mêmes.
			
			La cour de cassation devait répondre positivement à cette question en décidant qu’un créancier peut toujours agir contre les copropriétaires pour recouvrer sa créance (Civ 3\degre{} 30 oct 84, JCP 85 G, IV, 12; Civ 3\degre{} 7 nov 1990, Loy. et Cop. 1991 \no 41),
			La question posée a été de connaître le fondement juridique de cette action : action directe ou action oblique de l’article 1166 du code civil ?
			La cour de Paris dans un arrêt du 12 janvier 2000\footnote{Loyers et copr., 2000, comm. \no 156, note G. Vigneron} y a vu une application de l’action oblique : « Le créancier est fondé à exercer par la voie de l'action oblique, aux lieu et place du syndicat défaillant, l'action en recouvrement des quote-parts des charges communes contre chacun des copropriétaires pour obtenir le paiement des sommes qui lui sont dues ».
			
			La Cour de cassation a validé cette interprétation\footnote{3\degre{} Civ 26 oct 2005, pourvoi: 04-16664 Publié au bulletin ; Loyers et Copropriété 2006 \no 21} en écartant expressément l’action directe :
			\begin{quote}
				« Le syndicat des copropriétaires étant une personne morale de droit privé dont le patrimoine est distinct de celui de ses membres, le créancier dudit syndicat dispose d’une action oblique et non d’une action directe à l’égard des copropriétaires en paiement des sommes qui lui sont dues »
			\end{quote}
			
			Cette solution n’est pas idéale dans la mesure où les fonds obtenus par la voie de l’action oblique ne reviennent pas directement au créancier mais au syndicat des copropriétaires, à charge pour celui-ci de régler le créancier.
			
			Bien évidemment le créancier ne pourra pas demander à un copropriétaire de payer la totalité de sa créance puisqu’il n’y a pas de solidarité entre copropriétaires, il ne pourra demander paiement que de sa créance à hauteur de la quote-part du copropriétaire dans les parties communes générales et dans la mesure bien évidemment où ce copropriétaire n’a pas répondu aux appels de fonds du syndic destinés à régler le créancier : l’action oblique ne peut être mise en oeuvre que si le copropriétaire est effectivement débiteur du syndicat. En sorte que les copropriétaires peuvent s’exonérer en faisant la preuve qu’ils ont réglé cette quote-part entre les mains du syndic de la copropriété.
			
			La Cour de Paris, sur un appel de référé a d’ailleurs estimé que le créancier (en l’occurrence le chauffagiste) justifiait d’un intérêt légitime pour exiger du syndic la communication des noms et adresses des copropriétaires débiteurs sans que le syndic puisse utilement opposer que les coordonnées des copropriétaires pouvaient être obtenues au cadastre et à la conservation des hypothèques (documents qui ne permettent pas de savoir si les copropriétaires sont à jour ou non envers le syndicat) ni le secret professionnel\footnote{Paris 14\degre{} Chambre 28 avril 2006, Loyers et Copropriété 2006 \no 209}.
	
	\subsection{La disparition du syndicat en cas d’expropriation de l’immeuble pour carence et en cas d’expropriation des parties communes}
	
		Nous verrons à propos des copropriétés en difficulté que la loi \nom{Borloo} a institué une procédure de carence qui permet d’exproprier les copropriétaires lorsque ceux-ci n’entretiennent pas leur immeuble compromettant de la sorte la santé et la sécurité des occupants.
	
		Bien évidemment l’expropriation porte sur la totalité de l’immeuble, parties communes et parties privatives. Ce qui a pour conséquence de faire disparaître le syndicat des copropriétaires, tous les lots étant réunis entre les mains du bénéficiaire de l’expropriation (la Commune ou l’Opérateur choisi par la Commune).

		Cette procédure a été considérablement modifiée par la loi ALUR, article 72 qui, à titre expérimental, a créé une avant dernière étape avant l’expropriation totale de l’immeuble : l’expropriation des parties communes.
	
		Ainsi, la loi ALUR (article 72)\footnote{Numérotation de la Petite Loi (article 37 d’origine)} a institué, dans le cadre des remèdes à l’état de carence des copropriétés un article L 615-10 au CCH dont le I est ainsi rédigé :
		\begin{quote}
			Art. L. 615-10. CCH\\
		– I. – Par dérogation à l’article 6 de la loi \no 65-557 du 10 juillet 1965 fixant le statut de la copropriété des immeubles bâtis, une possibilité d’expropriation des parties communes est instaurée à titre expérimental et pour une durée de dix ans à compter de la promulgation de la loi \no du pour l’accès au logement et un urbanisme rénové. Dans ce cas, l’article L. 13-10 du code de l’expropriation pour cause d’utilité publique est applicable
		\end{quote}
		
		Ce texte signifie que pendant les dix années à venir, le Maire ou l’EPCI pourra lancer une procédure en expropriation des parties communes de l’immeuble.
		
		Le texte prévoit l’expropriation complète ou l’expropriation « de l’ensemble des parties communes ». Dans le dernier cas ces parties communes seront entretenues ou cédées à un Opérateur choisi par la Commune ou l’EPCI.
		
		A l’issue de cette procédure d’expropriation de l’ensemble des parties communes l’immeuble ne comportera plus que des propriétés privatives : celles d’origine restant appartenir aux copropriétaires d’origine et les anciennes parties communes devenues propriété de la Ville ou de l’Opérateur acquéreur de ces parties communes.
		
		La disparition du caractère commun de ces parties expropriées est d’autant plus évidente que le texte précise bien que ces parties d’immeuble devenues propriété de l’Opérateur vont se trouver grevées de servitudes au profit des locaux privatifs.
		
		Or, pour qu’il y ait copropriété, il faut qu’existent des parties communes dans l’immeuble.
		En sorte que la mise en œuvre de cette procédure entraîne la disparition du syndicat des copropriétaires.
		
		Le législateur était parfaitement conscient de cette situation puisque le même article L 615-10
		comprend un \VI{} ainsi rédigé :
		\begin{quote}
			\VI{}. – Après avis favorable de la commune ou de l’établissement public de coopération intercommunale compétent en matière d’habitat à l’origine de l’expérimentation et des propriétaires des biens privatifs, l’immeuble peut faire l’objet d’une nouvelle mise en copropriété à la demande de l’opérateur.
		\end{quote}
	
		Ceci explique d’ailleurs que la même loi ALUR précise au \VI{} de L. 615-6 du \CCH :
		\begin{quote}
			\VI{}. – Le cas échéant, dans l’ordonnance prononçant l’état de carence, le président du tribunal de grande instance désigne un administrateur provisoire mentionné à l’article 29-1 de la loi \no 65-557 du 10 juillet 1965 précitée pour préparer la liquidation des dettes de la copropriété et assurer les interventions urgentes de mise en sécurité.
			
			Sans préjudice des dispositions des articles L. 615-7 à L. 615-10 du présent code, la personnalité morale du syndicat subsiste après expropriation pour les besoins de la liquidation des dettes jusqu’à ce que le président du tribunal de grande instance mette fin à la mission de l’administrateur provisoire.
		\end{quote}
		
		C’est donc l’administrateur provisoire qui sera en ce cas liquidateur de plein droit du syndicat des copropriétaires.
	
	\subsection{Disparition par réunion des lots en une seule main}
	
		Il résulte de cette exigence de « division » de l’immeuble entre plusieurs propriétaires qu’une copropriété peut disparaître dans l'hypothèse inverse : lorsque tous les lots deviennent la propriété d'une seule personne (achat, héritage, $\dots$)\footnote{Cf. sur ce point l'article du Coneiller GUILLOT Administrer avril 1979 : La disparition du Syndicat des Copropriétaires.}, comme prévu désormais par l’article 46-1 de la Loi 65-557 du 10 juillet 1965 (article 39 de l’ordonnance du 30.10.2019).
		
		L’article 39 de l’ordonnance rétablit l’article 46-1 de la loi du 10 juillet 1965 afin de consacrer la jurisprudence de la Cour de cassation sur la réunion de tous les lots en une seule main :
		\begin{itemize}
			\item cette réunion entraîne la disparition de plein droit de la copropriété sans qu’une décision de l’assemblée générale ou du juge ne soit nécessaire pour le constater ;
			\item dans ce cas, le syndicat des copropriétaires, personne morale, subsiste uniquement pour les besoins de sa liquidation, sur le modèle de l’article 1844-8 du code civil en droit des société ;
			\item  le syndic est habilité à procéder aux opérations de liquidation et qu’à défaut, un mandataire ad hoc peut être désigné judiciairement pour représenter le syndicat en liquidation.
		\end{itemize}

		De même, un partage avec attribution divise peut entraîner la disparition du syndicat --- hypothèse réalisée par exemple à Roissy-en-Brie où plusieurs syndicats de copropriété horizontaux ont été purement et simplement dissous après que la Commune ait pris en charge l'ensemble des Voies et Réseaux Divers. Dans ce cas, il subsiste une pluralité de propriétaires, mais chacun est propriétaire, de façon exclusive de sa parcelle : il n’y a plus de parties communes indivises.
		\chapter{Parties communes et parties privatives}

	L'article 664 de Code Civil ne réglait que le mode de réparation et de reconstruction des parties de
	l'immeuble en copropriété.
	
	L'article 5 de la loi de 1938 faisait la distinction entre les choses communes et les choses << affectées à
	l'usage exclusif >> d'un copropriétaire.
	
	Il permettait de distinguer les parties communes des parties privatives en fonction de l'usage qui en était
	fait :
	\begin{itemize}
		\item usage exclusif = parties privatives
		\item usage commun = parties communes
	\end{itemize}
	
	Ce texte d'autre part prévoyait une présomption de communauté pour tout ce qui n'était pas privatif.
	
	La loi du 10 juillet 1965 apporte des définitions plus précises des parties communes et privatives et donne
	une énumération des parties communes qui révèle le souci du législateur de faciliter la rédaction des
	règlements de copropriété.
	
	L'article 2 donne une définition des parties privatives plus précise que celle de 1938, mais c’est toujours
	l'usage exclusif qui détermine le caractère privatif.
	\begin{quote}
		Sont privatives les parties des bâtiments et des terrains réservées à l'usage exclusif d'un copropriétaire déterminé.
	\end{quote}

	L'article 3 donne une définition plus large des parties communes que celle donnée par la loi de 1938 :
	\begin{quote}
		sont communes les parties de bâtiments et de terrains affectées à l'usage ou à l'utilité de tous les
		copropriétaires ou de plusieurs d'entre eux. 
	\end{quote}

	Ce n'est donc plus l'usage seulement qui détermine le caractère commun, mais également l'utilité commune.
	
	Ces dispositions ne sont pas d’Ordre Public (elles ne sont pas comprises dans l'énumération de l'article 43
	de la loi). Cela signifie que l'on peut y déroger soit dans le Règlement de Copropriété soit par décision
	d'assemblée générale. Ce caractère supplétif est renforcé par les termes mêmes de l'article 3 de la loi qui,
	avant de donner une liste de parties communes, précise :
	\begin{quote}
		dans le silence ou la contradiction des titres,	sont réputées parties communes \dots
	\end{quote}
	
	Cependant la liberté des auteurs de Règlement de Copropriété se trouve limitée par la nature des choses :
	il est de l'essence même de certaines parties d'immeubles d'être communes à plusieurs lots ( la toiture,
	le gros œuvre, les fondations de l'immeuble). De même l'affectation de certaines parties de l'immeuble
	implique leur classement en parties communes, comme par exemple le sol de la cour donnant accès à
	l'immeuble depuis le porche sur rue.
	
	La loi \no 2018-1021 du 23 novembre 2018 dite ELAN a, en consacrant une jurisprudence antérieure de la
	cour de Cassation :
	\begin{enumerate}
		\item complété la liste des parties figurant dans l’article 3;
		\item consacré les « parties communes spéciales », c’est-à-dire des parties
		communes entre certains copropriétaires seulement (article 6-2) ;
		\item  consacré la possibilité de procéder à un démembrement de la
		propriété des parties communes, en créant des « parties communes
		à jouissance exclusive >> (article 6-3).
	\end{enumerate}

\section{Les parties privatives}

	\subsection{Definition des parties privatives : parties a l’usage exclusif d’un	coproprietaire}

		L'article 2 dispose :
		\begin{quote}
			Sont privatives les parties des bâtiments et des terrains réservées à l'usage exclusif d'un copropriétaire déterminé
		\end{quote}
		C'est donc l'usage exclusif qui détermine le caractère privatif. L'usage exclusif, c'est celui qui est
		incompatible avec l'usage d'un autre copropriétaire : le copropriétaire a sur ses parties privatives un
		véritable monopole.
		
		Selon l'article 2 peuvent être privatives non seulement des parties de bâtiments mais également des
		parties de terrains : ainsi, le Règlement de Copropriété peut classer dans les parties privatives des jardins,
		terrasses ou cours d'immeuble.
		
		Le plus souvent ces terrasses ou jardins ne sont pas classés en parties privatives, mais en parties communes
		à jouissance exclusive. Cette jouissance exclusive, même si elle interdit la jouissance concurrente des
		autres copropriétaires n'a pas pour effet de déclasser ces parties communes en parties privatives.
	
	\subsection{Qualification usuelle des parties privatives}
		La loi ne donne aucune énumération des parties privatives, même à titre subsidiaire. C'est donc le
		Règlement de copropriété qui détermine les parties privatives.
		
		En cas d'ambiguïté du Règlement de Copropriété, lorsque ses clauses sont obscures ou contradictoires ou
		encore lorsqu'il n'existe pas de Règlement de Copropriété les Tribunaux qualifient habituellement de
		parties privatives :
		\begin{itemize}
			\item les parquets, revêtements de sol, y compris ceux des balcons et terrasses, mais pas le gros œuvre de ces balcons et terrasses\footnote{TGI Paris 17 mars 1975 et TGI Paris 10 mars 1973} ;
			\item les enduits intérieurs des murs et plafonds, les corniches, moulures, lambris, revêtements
			intérieurs ;
			\item les cloisons intérieures non porteuses ;
			\item les fenêtres et porte-fenêtre avec leur bâti, vitres, volets, persiennes, barres d'appui de
			fenêtres et balcons, les bannes ou stores ;
			\item les menuiseries intérieures (placards, portes y compris la porte palière) ;
			\item les devantures et vitrines des magasins et boutiques ;
			\item les installations de cuisine et sanitaires (éviers, baignoires, lavabos), les canalisations et
			branchements des appareils jusqu'aux canalisations communes ;
			\item les installations de chauffage et de production d'eau chaude individuelle ;
			\item les installations d'eau, d'électricité et de gaz jusqu'au branchement.
		\end{itemize}

\section{Les parties communes}

	\subsection{Definition des parties communes}
	
		\subsubsection{Définition des parties communes : parties affectées à l’usage ou à l’utilité de tous}
		
			L’article 3 de la loi du 10 juillet 1965 dispose « sont parties communes les parties des bâtiments et des
			terrains affectées à l'usage ou à l'utilité de tous les copropriétaires ou de plusieurs d'entre eux ».En
			pratique, presque tout le bâti relève des « parties communes » : gros murs, fondations, planchers, toiture
			et sol…
			Ces parties communes se caractérisent non seulement par une « utilité » commune, mais par la propriété
			indivise des copropriétaires sur celles-ci : le droit du copropriétaire sur ces parties communes se traduit
			par la « quote- indivise » ou les « tantièmes » qui sont attachés à son lot (\emph{lot \no 1 : un appartement et les	125/ \nombre{1 000}\ieme des parties communes})
			
		\subsubsection{L’énumération de l’article 3 et son caractère supplétif}
		
			L'article 3 de la loi, deuxième alinéa comporte une liste d’éléments de bâtis « réputés parties communes »
			« Dans le silence ou la contradiction des titres sont réputées parties communes... ». Cette énumération
			n'est donc que supplétive de la volonté des parties, c'est à dire de l'auteur du règlement de Copropriété.
	
			Le Règlement de Copropriété peut donc prévoir de classer en parties privatives des parties d'immeuble
			qui dans le silence ou la contradiction du Règlement de Copropriété auraient été placées en parties
			communes.
			Il en est ainsi par exemple de l'étanchéité et du gros oeuvre d'une terrasse que le
			Règlement de Copropriété classe expressément en partie privative alors que dans le
			silence contractuel la terrasse est classée en partie commune dans son gros oeuvre et son
			étanchéité\footnote{Civ 3\degre{} Ch 13 nov 1975, pourvoi: 74-11703 ; Bull Civ \no 331 p 251}.
			Il en sera de même d'un gros mur que le Règlement de Copropriété aura classé en partie
			privative, en sorte que si ce gros mur est mitoyen, en cas de travaux à réaliser sur ce mur,
			le voisin dirigera son action contre le copropriétaire titulaire du lot comprenant ce mur et
			non contre le syndicat des copropriétaires\footnote{LYON 30 Auch 1973, JCP 74 II 17715}.
		
		\subsubsection{L’absence de partie commune par « subsidiarité »}
		
			Les Tribunaux ont considéré pendant de nombreuses années que toute partie de bâtiment qui n'est pas
			affectée à l'usage exclusif d'un copropriétaire est une partie commune. C’était notamment la position de
			la Cour de Cassation dans un arrêt du 16 janvier 1979\footnote{Administrer 1979 \no 94}.
			Cependant la Cour de Cassation est revenue sur cette jurisprudence à propos d'une terrasse classée partie
			commune par le juge du fond, car non affectée à un copropriétaire par le Règlement de Copropriété :
			l’arrêt a été cassé au motif que le juge du fond aurait dû rechercher à l'usage et à l'utilité de qui cette
			partie de l'immeuble était réservée\footnote{Civ 3\degre{} Ch – 14 février 1990 - pourvoi: 88-17781 Inf. Rapides de la Copropriété. 1990 p 281}.
			Un arrêt du 6 mai 1993\footnote{PARIS 14 décembre 1990 - D. 91 IR 15} de la Cour de PARIS se référant à l'arrêt de Cassation du 14 février 1990 éclaire
			bien cette jurisprudence : "Un comble qui ne peut servir qu'à assurer la réparation de la toiture de
			l'immeuble, incontestablement partie commune, constitue lui-même une partie commune, peu important
			qu'il existe une trappe d'accès dans le lot privatif du copropriétaire défendeur, dès lors qu'un accès est
			effectivement nécessaire, étant observé de surcroît que cette trappe permet d'accéder à l'ensemble du
			comble et non seulement à la partie située au-dessus de son appartement".
			La jurisprudence la plus récente sur la question\footnote{Cass. 3e civ., 30 mai 1995, pourvoi: 93-16347 somJCP N 1995 / \no 26 / p. 981. et Dalloz 1994, Somm p.203 Les conditions à remplir pour que le s combles de l'immeuble constituent des parties privatives. La pose	de chiens-assis dans les combles peut-elle être assimilée à une surélévation ? par Jean-Robert Bouyeure} admet qu’il n’y a pas de parties communes par	subsidiarité et qu’il convient à chaque fois de rechercher si la partie d’immeuble concernée est à usage exclusif\footnote{Cf. l’intéressant article de Me Hocquard in IRC 2006 \no 515 p 29} d’un copropriétaire (auquel cas elle est bien une partie privative) ou si elle est à l’usage de tous ou de plusieurs (auquel cas elle est une partie commune).
	
	\subsection{Enumeration des parties communes (art 3 alinea 2 de la loi)}
	
		\subsubsection{Le sol}
		
		Le sol de la Copropriété est présumé commun ; mais comme dit précédemment il pourrait être classé
		partie privative. Ce dernier cas se présente parfois dans le domaine des copropriétés horizontales.
		Cependant, l’Administration refuse de délivrer le permis de construire dans le cadre de copropriétés
		horizontales dès lors que le sol est partie privative ou à jouissance privative. Elle considère en effet
		que dans ces deux hypothèses il y a réalisation d’un lotissement.
		Il arrive parfois que le sol soit qualifié de << partie commune à jouissance privative >>, ce qui ne lui fait pas
		perdre son caractère de partie commune\footnote{Cf infra le << problème des parties communes à jouissance privative >>.}.
		
		\subsubsection{Les cours}
		
		Les cours posent souvent problème du fait du stationnement des véhicules. Un copropriétaire ne peut
		prétendre encombrer les parties communes et à ce titre ne peut laisser stationner un véhicule dans la cour
		de l'immeuble. Mais les copropriétaires peuvent décider en assemblée générale d'autoriser le
		stationnement des véhicules.
		
		La Cour de PARIS a estimé illicite la décision d'assemblée générale concédant un droit d'usage
		exclusif sur le sol de la cour pour y faire stationner des véhicules, dès lors que ce droit d'usage exclusif
		était consenti à certains copropriétaires seulement\footnote{PARIS 23\degre{} 7 mai 1993 - Loyers et Copropriété Août Sep 93 \no 316}.
		
		\subsubsection{Les parcs et jardins, les voies d'accès}
		
		\subsubsection{Le gros œuvre des bâtiments}
		
		\subsubsection{Les murs}
		
		Sont présumés communs les murs qui constituent le gros œuvre de la construction (gros murs ou murs de
		refend, les poteaux, poutres et poutrelles verticales ou horizontales, métalliques ou en maçonnerie).
		
		Le plus souvent les revêtements muraux comme les revêtements de sol sont placés dans les parties
		privatives. Toutefois, dans le silence du Règlement de copropriété, s’agissant d’un flocage « amiante »,
		destiné à assurer la protection de l’immeuble contre l’incendie, même si le flocage est situé au plancher
		haut de locaux privatifs, il doit être considéré comme partie commune\footnote{Cass. 3e civ., 7 mai 2003 : JurisData \no 2003-018913 ; JCP G 2003, p. 1869 ; Loyers et copr. 2003, comm.}. En sorte que le désamiantage
		devra être pris en charge par le syndicat des copropriétaires.
		
		\subsubsection{Les planchers}
		
		Il s’agit du gros œuvre du plancher : poutres, solives, hourdis. Il convient d'y assimiler les voûtes des caves.
		
		\subsubsection{Le toit, les terrasses}
		
		Le toit étant partie commune, son entretien est à la charge du Syndicat. Mais on peut imaginer que le
		Règlement de Copropriété classe en partie privative le toit d'un bâtiment constitué d'un seul lot de
		copropriété, ou la verrière surplombant un appartement\footnote{CA Lyon 30 novembre 2011 à propos d’un « ciel vitré » : Les ciels vitrés, ayant pour utilité de faire pénétrer la lumière dans les appartements du dernier étage, constituent des parties privatives au même titre que les fenêtres lesquelles sont qualifiées comme telles par le règlement de copropriété, même s'ils sont incorporés dans le pan de la toiture, dès lors que leur utilité principale est d'éclairer des appartements privés et non de couvrir l'immeuble.}. Peut également se poser la question des
		« Vélux » fenêtres de toit : alors même que le Règlement de copropriété les qualifie de parties privatives
		la Cour de Paris a considéré que le syndicat des copropriétaires devait répondre des infiltrations d’eau
		ayant leur siège à la jonction des velux parties privatives et de la toiture commune\footnote{Paris, 23\degre{} Chambre A, 25 septembre 2002, 92 rue Rochechouart, Jurisdata 2002-188829 ; cf. également, dans le silence du Règlement de copropriété, la qualification des velux en parties communes par la Cour de Pau, en ce qu’ils assurent non seulement l’éclairage d’un lot mais également le couvert de l’immeuble (Pau, 1\degre{} Chambre, 15 février 2013, JurisData 2013-01064).}.
		
		En ce qui concerne la terrasse d'un bâtiment quand bien même cette terrasse est classée en partie
		privative, seul le revêtement est partie privative. Par contre le dispositif d'étanchéité et la couche
		d'étanchéité sont des parties communes, de même que le gros oeuvre de la terrasse et les maçonneries
		sous-jacentes\footnote{Civ. 16 nov. 1976 D 77 IR 62}.
		Les structures métalliques de la terrasse lorsqu’elles sont ornements extérieurs des façades doivent
		également être placées dans les parties communes\footnote{Paris 23\degre{} Ch A 11 déc 1996; Loyers et Copropriété 1997 \no 119}.
		
		\subsubsection{Les balcons et bow-windows}
		
		Une jurisprudence abondante a classé les balcons en parties privatives au motif que ces balcons sont à
		l'usage exclusif du lot dont ils dépendent, mais la jurisprudence la plus récente a tendance à classer ces
		balcons en parties communes. Par exemple un balcon qui est le prolongement extérieur de la dalle de
		l'étage sera très certainement classé en partie commune quant à son gros oeuvre. De même un balcon,
		pièce rapportée en façade dont il constitue l'ornementation contribue à l'harmonie générale et à ce titre
		sera également classé en partie commune.
		
		Mais bien évidemment ce classement par le juge du fond ne se fait que dans le silence ou la contradiction
		des titres. Le plus souvent le Règlement de Copropriété opère une distinction entre le gros œuvre du
		balcon (effectivement classé en partie commune) et les balustrades et garde-corps (classés en parties
		privatives).
		
		Les bow-windows semblent devoir être placés en parties privatives pour leurs parties vitrées et en parties
		communes pour la dormant métallique qui concourt à l'harmonie, et qui, non-manœuvrable constitue
		partie de la façade de l'immeuble\footnote{Par assimilation avec les châssis vitrés aménagés dans les boxes d’un immeuble, cf. 3e civ., 4 juill. 1990 : Loyers et copr. 1990, comm. 399 ; D. 1991, p. 76}.
		
		\subsubsection{Les escaliers}
		
		Les escaliers sont présumés parties communes comme faisant partie du gros œuvre des bâtiments. Il en
		est de même des marches et contremarches.
		
		A défaut de disposition expresse du Règlement de Copropriété, l'escalier est une partie commune à tous
		les copropriétaires, en sorte que tous doivent y participer $\dots$ et tous pourront l'emprunter.
		
		On ne saurait en effet, ni dans le Règlement de Copropriété, ni par décision d' assemblée générale, limiter
		de façon discriminatoire les droits d'usage d'un copropriétaire sur les parties communes au motif qu'il n'a
		pas l'utilité réelle de ces parties communes (par ex. par la pose d’un code à la porte d'accès à l'escalier de
		l'immeuble en refusant la clé aux copropriétaires du rez de chaussée).
		
		\subsubsection{Canalisations traversant les locaux privés}
		
		\paragraph{La présomption de parties communes ne s'applique qu'aux canalisations qui
		traversent (sans s’y arrêter) les locaux privatifs}.\footnote{Cf Les canalisations en copropriété, Me Guilhem GIL, I.R.C. \no 595 – jan/fev 2014 p. 17}
	
		En effet << ces canalisations, quand bien même elles sont situées pour parties à l'intérieur de lots privatifs,
		sont, par leur fonction propre, rattachées à l'élément commun qu'elles prolongent, et non au local privatif
		qu'elles traversent, sans servir à l'usage du propriétaire du lot>>\footnote{CA PARIS 7\degre{} Ch du 3 mars 1993}.
		D'où une controverse pour les canalisations qui desservent les parties privatives sans en ressortir. Il
		incombe aux rédacteurs de Règlement de Copropriété d'être particulièrement clairs sur ce point et de
		qualifier de privatives ou communes ces canalisations.
		
		Mais encore une fois, le juge ne pourra aller contre une classification précise du Règlement de Copropriété.
		Par exemple à propos de canalisations classées parties privatives par le Règlement de Copropriété dans
		leur partie qui traverse les lots privatifs et ce alors même que ces canalisations n'ont aucune utilité pour
		ces lots.
		
		\paragraph{Canalisations encastrées dans le plancher commun}
		
		Assez souvent se pose la question des canalisations ne desservant qu’un lot privatif encastrées
		dans le plancher partie commune.
		Si le Règlement de Copropriété précise que les canalisations ne desservant qu’un seul lot sont
		privatives, il ne fait aucun doute qu’encastrées ou non dans le gros oeuvre, ces canalisations
		demeurent privatives.
		
		Si le Règlement de copropriété exclue des parties privatives les canalisations qui ne se trouvent pas
		à l'intérieur des lots qu'elles desservent, la Cour d’Appel en déduit exactement que ces canalisations
		encastrées dans le plancher partie commune se trouvant à l’extérieur des parties privatives, constitue
		des parties communes\footnote{Civ. 3\degre{} Ch. 26 janvier 2017, pourvoi \no 15-2825, non publié}.
		
		Si le Règlement de Copropriété reste muet sur la qualification de ces canalisations, il convient à
		notre sens d’appliquer les dispositions de l’article 2 de la loi et de les qualifier de privatives ;
		cependant on citera un arrêt du 1er juillet 2003 qui a reproché à la cour d’appel dans une telle
		hypothèse de n’avoir pas recherché si la canalisation encastrée dans le plancher partie commune
		n'était pas elle-même une partie commune\footnote{Civ 3\degre{} Ch 1er juillet 2000 », pourvoi \no 01-03430, inédit.}.
		
		\subsubsection{Coffres, gaines et têtes de cheminées}
		
		Ces coffres, gaines et têtes de cheminées sont la plus souvent incorporés dans les gros murs.
		
		\subsubsection{Locaux et services communs}
		
		Il s’agit de la loge du concierge, le local social, le local a bicyclettes, \emph{etc.}
		
		Il existe une jurisprudence abondante sur le caractère privatif ou commun de la loge de concierge dont le
		promoteur fait très souvent un lot privatif : par le passé certains arrêts ont décidé en ce cas que même
		affectée de tantièmes de copropriété, la loge pouvait être classée partie commune, puisqu’elle était
		d’intérêt collectif.
		
		Aujourd’hui la solution inverse est affirmée avec vigueur\footnote{CA Paris 23\degre{} Ch. A, 28 mai 1997 Recueil Dalloz 1997, Sommaires commentés p. 325} :
		\begin{quote}
			<< La loge du concierge, qui comprend d'une manière indissociable des parties privatives et une
			quote-part correspondante des parties communes, ne peut être classée comme partie
			commune sans perdre sa nature et son indépendance ; En raison de cette incompatibilité, entre
			la notion de lot et la qualification de partie commune, il y a lieu de juger qu'il existe une erreur
			de rédaction évidente dans le règlement de copropriété classant la loge de gardien parmi les
			parties communes, cette erreur pouvant faire l'objet d'une correction par décision de
			l'assemblée générale des copropriétaires de l'immeuble statuant aux conditions de majorité de
			l'art. 26 de la loi \no 65-557 du 10 juill. 1965 >>
		\end{quote}
	
		Il est donc incontestable que si la loge de concierge est constituée en partie privative dotée de tantièmes
		généraux, le juge ne peut requalifier cette loge de partie commune\footnote{Civ 3\degre{} 19 nov 1986; Civ 3\degre{} 4 jan 1989; Civ 27 juin 1990 , Rev Droit Imm. 90, 526}, même si le règlement de copropriété
		fait état d’une loge dans les parties communes\footnote{Paris 23\degre{} Chambre 28 mai 1997 Loyers et Copropriété 1997 \no 297, qui précise que l’assemblée générale peut corriger le règlement de copropriété sur ce point à la majorité de l’article 26 de la loi.}.
		
		Pour autant, si le Règlement de Copropriété prévoit qu'il existera dans l'immeuble une loge de concierge,
		le local même classé en partie privative peut-il être considéré comme une partie privative à affectation
		déterminée et « perpétuelle »?
		
		Une réponse négative a été apportée par la Cour de Cassation dans un arrêt du 4 novembre 2004\footnote{Civ - Troisième chambre Arrêt \no 1101 du 4 novembre 2004}, motif
		pris que le règlement de copropriété ne peut porter atteinte au droit de disposition de son lot par un
		copropriétaire.
		
		En l’espèce le promoteur, tombé depuis lors en liquidation de biens, avait conservé des lots utilisés
		par le syndicat des copropriétaires comme conciergeries dans la plus grande copropriété de France
		(La Rouvière à Marseille). Après la vente de ces lots à la barre du tribunal, l’adjudicataire avait initié
		une procédure d’expulsion de ces locaux à l’encontre du syndicat des copropriétaires. Pour refuser
		l’expulsion, la cour d’appel avait retenu : « que l’affectation des lots est justifiée par la destination
		de l’immeuble, que l’article 55 du règlement de copropriété selon lequel << le concierge habitera
		obligatoirement au rez-de-chaussée dans des locaux spécialement affectés à cet effet >> est licite et
		opposable à l’acquéreur, que les caractéristiques de l’immeuble imposent la présence à demeure de
		concierges, que les locaux du rez-de-chaussée ont été immédiatement affectés en conciergeries par
		le promoteur-constructeur et équipés d’un matériel spécifique, qu’il les a loués verbalement au
		syndicat des copropriétaires et que l’attention de l’adjudicataire a été attirée par le cahier des
		charges sur leur usage impératif ».
		
		Cassation au visa des articles 544 du Code civil, et de l’article 9 de la loi du 10 juillet 1965 au motif
		lapidaire : Qu’en statuant ainsi, alors que cette stipulation du règlement de copropriété ne pouvait
		avoir pour effet d'instituer de restriction aux droits de copropriétaires sur leur lot, la cour d'appel a
		violé les textes susvisés.
		
		En revanche le Promoteur-Vendeur pourra être condamné pour publicité mensongère, et le cas échéant à
		livrer une loge à la Copropriété s’il a pris cet engagement envers les acquéreurs.
		L'analyse conjuguée de l'acte de vente (descriptif annexé à une vente en l'état futur d'achèvement, par
		exemple) et du Règlement de Copropriété permettra de déterminer les obligations du vendeur vis-à-vis
		de ses acquéreurs. Bien souvent le Descriptif lui-même est muet sur certains locaux collectifs ou éléments
		d'équipement alors que le Règlement de Copropriété traitera de ces parties de l'immeuble.
		De plus, et alors même que le Promoteur Vendeur a souscrit des engagements envers les acquéreurs à
		titre individuel et non envers la personne morale du syndicat des copropriétaires, les Tribunaux ont déclaré
		le syndicat des copropriétaires recevable à agir en justice pour avoir condamnation du Promoteur Vendeur
		à lui livrer ces parties d'immeubles\footnote{Civ. 3\degre{} 19 fèv 1980, Bull Civ III \no 41 p 29}.
		
		\subsubsection{Les passages et corridors}
		
		Présumés parties communes, ils sont souvent placés en parties communes spéciales aux bâtiments dont
		ils dépendent.
		
		\subsubsection{Les parties communes par incorporation}
		
		La loi ELAN a ajouté un point à cette énumération : est également réputé parties communes : « tout
		élément incorporé dans les parties communes. » Cette présomption peut être particulièrement utile pour
		les réseaux de chauffage au sol noyés dans la dalle partie commune, mais également aux éventuels bowwindows,
		rampe d’accès PMR, voire ascenseur réalisé par un copropriétaire, si le règlement de copropriété
		ne précise rien à ce sujet.
		
		\subsection{Les parties communes « speciales »}
		
		\subsubsection{Définition}
		
		L'article 3 ébauche une distinction entre les parties communes à tous les copropriétaires et les parties
		communes à "plusieurs d'entre eux". Effectivement le Règlement de Copropriété peut distinguer les
		parties qui sont communes à tous les copropriétaires (parties communes « générales ») et les parties qui
		sont spéciales à certains d'entre eux (parties communes « spéciales »),
		Cette distinction a été consacrée dans l’article 6-2 de la loi \no 65-557 du 10 juillet 1965 dans sa rédaction
		issue de la loi loi \no 2018-1021 du 23 novembre 2018 dite ELAN (rectifiée par l’ordonnance)
		Article 6-2 Créé par LOI \no2018-1021 du 23 novembre 2018 - art. 209 (V) -Ordonnance \no2019-1101 du 30
		octobre 2019 - art. 5.
		
		Les parties communes spéciales sont celles affectées à l'usage et ou à l'utilité de plusieurs
		copropriétaires. Elles sont la propriété indivise de ces derniers.
		La création de parties communes spéciales est indissociable de l'établissement de charges spéciales à
		chacune d'entre elles.
		
		Les décisions afférentes aux seules parties communes spéciales peuvent être prises soit au cours d'une
		assemblée spéciale, soit au cours de l'assemblée générale de tous les copropriétaires. Seuls prennent
		part au vote les copropriétaires à l'usage et ou à l'utilité desquels sont affectées ces parties communes.
		
		En outre, l’article 6-4 dispose désormais
		
		Article 6-4 Créé par LOI \no2018-1021 du 23 novembre 2018 - art. 209 (V)
		L'existence des parties communes spéciales et de celles à jouissance privative est subordonnée à leur
		mention expresse dans le règlement de copropriété.
		
		Lorsque des parties communes sont instituées (par exemple par Bâtiment), le copropriétaire a à la fois des
		droits indivis dans les « parties communes générales » (en général le sol, les espaces verts et les voiries)
		et des droits indivis spécifiques dans son Bâtiment. En revanche, il ne détient aucun droit dans le Bâtiment
		voisin.
		Ex : lot \no1 : un appartement au Rez-de-chaussée du Bâtiment B et les 125/10.000 du sol et les 200/1000
		des parties communes du Bâtiment A
		
		\subsubsection{Conséquences de la création de parties communes spéciales}
		
		Il résulte de la création de parties communes spéciales à certains copropriétaires :
		\begin{itemize}
			\item Une spécialisation des charges : il avait déjà été jugé par la Cour de cassation que « la
			création dans le règlement de copropriété de parties communes spéciales avait pour
			corollaire l’instauration de charges spéciales et déduit que les charges afférentes à la
			réfection de l’étanchéité de la terrasse devaient être réparties entre les propriétaires de ce
			bloc » ; le principe est consacré par l’article 6-2 (auparavant : 3è Chambre civile, 6 juin 2007
			(Bull. \no 98)
			\item  Une spécialisation des droits de vote en assemblée générale (ou une assemblée spéciale):
			seuls les copropriétaires ayant des tantièmes dans les parties communes spéciales peuvent
			prendre des décisions à leur sujet, puisque les tantièmes de propriété déterminent les
			droits de vote. L’article 6-3 consacre même la faculté d’une assemblée spéciale, incertaine
			jusqu’alors (en ce sens avant la loi \no 2018-1021 du 23 novembre 2018 dite ELAN CA Paris
			23è chambre B 29 juin 2000, loyers et copropriété février 2001 comm. \no53 ; CA Paris 23è
			chambre section A 15 mai 2002, AJDI septembre 2002, page 618) ; à la condition toutefois
			que la décision ne concerne que les parties communes spéciales ( le ravalement, par
			exemple, peut aussi concerner l’aspect extérieur de l’ensemble immobilier ; la vente d’une
			partie commune spéciale implique ensuite un modificatif au règlement de copropriété qui
			concerne tous les copropriétaires - CA Paris 23è chambre, 4 septembre 2008, Administrer
			mai 2009 page 49)
			\item  Une limitation des droits des copropriétaires : lorsqu'une partie commune est qualifiée de
			particulière à certains copropriétaires, les autres copropriétaires de l'immeuble n'ont
			aucun droit de propriété indivise sur cette partie. Aussi le prix de vente des parties
			communes spéciales doit-il être réparti entre les copropriétaires, ou l’accès au Bâtiment
			peut être interdit aux copropriétaires n’ayant pas de droit dans celui-ci.
		\end{itemize}
		
		\subsubsection{Adaptation des règlements de copropriété}
		
		La loi \no 2018-1021 du 23 novembre 2018 dite ELAN consacre l’existence des parties communes spéciales,
		mais impose leur mention dans le règlement de copropriété (elles apparaissaient parfois seulement dans
		l’état descriptif de division) et impose également la création explicite d’une clef de charge constitutive.
		
		Il peut donc être nécessaire d’adapter le règlement de copropriété existant.
		
		Cette adaptation a été prévue par la loi ELAN dans son article 209 – \II (disposition transitoire)
		
		Les syndicats des copropriétaires disposent d’un délai de trois ans à compter de la promulgation de la
		présente loi pour mettre, le cas échéant, leur règlement de copropriété en conformité avec les dispositions
		de l’article 6-4 de la loi \no 65-557 du 10 juillet 1965 fixant le statut de la copropriété des immeubles bâtis.
		
		A cette fin, le syndic inscrit à l’ordre du jour de chaque assemblée générale des copropriétaires la question
		de la mise en conformité du règlement de copropriété. La décision de mise en conformité du règlement
		de copropriété est prise à la majorité des voix exprimées des copropriétaires présents ou représentés.

\section{Droits accessoires aux parties communes}

	Un certain nombre de droits sont définis comme des droits « accessoires » aux parties communes, c’està-
	dire attaché au sol ou au gros oeuvre du Bâtiment. Par application du principe selon lequel « l’accessoire
	suit le principal », ces droits sont eux même réputés parties communes. L’article 3 établit une liste de ces
	droits « réputés parties communes », mais elle n’est pas limitative (A) Cependant, tout comme les parties
	communes « matérielles » de l’immeuble ces droits peuvent être « privatisés », c’est-à-dire réservés à
	l’usage exclusif d’un copropriétaire, soit par intégration dans la partie privative du lot (B), soit par
	convention (C)
	
	\subsection{Definition des droits accessoires aux parties communes}
	
		\subsubsection{Enumération des droits accessoires (art 3 alinéa 3)}
		
			L'article 3 al 3 de la loi donne la liste des « droits accessoires » aux parties communes, donc des droits
			immatériels réputés propriété indivise à tous. Seul le Syndicat, peut donc en disposer, moyennant une
			contrepartie financière, à la majorité de l’article 26 de la loi sous réserve que cette cession des droits ne
			porte pas atteinte aux modalités de jouissance des autres lots\footnote{Versailles 29 nov 2004, Administrer \no 375 mars 2005.} ou à la destination de l’immeuble.
			
			Cette énumération a été complétée par la loi \no 2018-1021 du 23 novembre 2018 dite ELAN
			Article 3- modifié loi \no 2018-1021 du 23 novembre 2018 dite ELAN
			" Sont réputés droits accessoires aux parties communes dans le silence ou la contradiction des titres
			- le droit de surélever un bâtiment affecté à l'usage commun ou comportant plusieurs locaux qui
			constituent des parties privatives différentes, ou d'en affouiller le sol ;
			- le droit d'édifier des bâtiments nouveaux dans des cours, parcs ou jardins constituant des parties
			communes ;
			- le droit d'affouiller de tels cours, parcs ou jardins ;
			- le droit de mitoyenneté afférent aux parties communes.
			« – le droit d’affichage sur les parties communes ; (art. 59 bis F loi ELAN)
			« – le droit de construire afférent aux parties communes. » ; (art. 59 bis F loi ELAN)
		
		\subsubsection{Sur le caractère non limitatif des droits accessoires énumérés par l’article 3}
		
			La jurisprudence a ajouté à cette liste d’autres droits accessoires aux parties communes, que le syndicat
			peut céder à un copropriétaire à la majorité de l’article 26 de la loi, affirmant que cette liste n’est pas
			limitative\footnote{3\degre{} Chambre civile 24 mai 2006. \no 05-14.038. au Bulletin - C.A. Paris, 27 janvier 2005} :
			\begin{quote}
				« La liste des droits accessoires aux parties communes définis par l'article 3 de la loi du 10 juillet 1965 n'est
			pas limitative.
			\end{quote}
		
			Dès lors, une cour d'appel qui, interprétant souverainement les stipulations d'un règlement de copropriété,
			retient que la faculté que s'y réserve un propriétaire de clore une terrasse dont il a la jouissance privative
			constitue un droit accessoire, en déduit exactement, par application de l'article 37 de cette loi, que cette
			faculté est devenue caduque si ce droit n'a pas été exercé dans les dix années qui suivent la convention ».
			
			C’est ainsi que la jurisprudence antérieure à la loi ELAN avait qualifié le droit « administratif » de construire
			de droit accessoire aux parties communes- à l’époque où il existait encore une limitation de la superficie
			constructible à la parcelle par le Coefficient d’Occupation des sols\footnote{Civ 3\degre{} 10 janvier 2001 (D. 2002 p. 1521, note Giverdon) : le droit de construire sur le sol commun résultant des règles	d’urbanisme constitue un droit accessoire aux parties communes. » et , 3\degre{} Ch. Civ. 1er oct 2013, \no pourvoi : 12-21785, non publié. }. En d’autres termes, lorsqu’un
			copropriétaire entend déposer une demande de permis de construire, il utilise les droits de construire
			propriété du syndicat, à moins que ce droit lui ait été donné par le Règlement de Copropriété. Il ne pourra
			le faire que dans la mesure où le syndicat les lui aura expressément cédés. Il en était ainsi dans l’hypothèse
			d’un changement d’utilisation d’un lot dès lors que ce changement avait une influence sur le COS de
			l’immeuble, ceci quand bien même le Règlement de Copropriété autoriserait-il toute affectation du lot\footnote{114}.
			
			Cette jurisprudence a été consacrée par la loi \no 2018-1021 du 23 novembre 2018 dite ELAN qui inclus
			désormais dans les droits accessoires aux parties communes « le droit de construire afférent aux parties
			communes ». Ce droit ne se confond pas avec le droit de surélever un Bâtiment ou d’en construire un
			nouveau.
			
			Cependant, la disparition du COS limite la portée de cette disposition, puisque, désormais, le droit de
			construire sur la parcelle est limitée essentiellement par le gabarit ou le prospect. Ainsi, désormais, la
			réalisation d’une mezzanine est sans conséquence sur la constructibilité du Bâtiment, et ne nécessite plus
			de cession préalable du droit de construire 115. En revanche, la surélévation d’un Bâtiment exclusivement
			privatif, qui n’est pas en principe un droit « accessoire » appartenant au syndicat des copropriétaires116 (cf
			la rédaction de l’art 3 al 3) peut porter atteinte à la constructibilité via les règles de gabarit $\dots$
	
	\subsection[Conventions de l'article 37]{Les conventions relatives aux droits accessoires sur parties	communes (art. 37 Et 37-1 de la loi \no 65-557 du 10 juillet 1965 )}
	
	L’article 37 de la loi a été ainsi rédigé dans la loi \no 65-557 du 10 juillet 1965
	Toute convention par laquelle un propriétaire ou un tiers se réserve l'exercice de l'un des
	droits accessoires visés à l'article 3, autre que le droit de mitoyenneté, devient caduque si
	ce droit n'a pas été exercé dans les dix années qui suivent ladite convention.
	
	Si la convention est antérieure à la promulgation de la présente loi, le délai de dix ans court
	de ladite promulgation.
	
	Avant l'expiration de ce délai, le syndicat peut, statuant à la majorité prévue à l'article 25,
	s'opposer à l'exercice de ce droit, sauf à en indemniser le titulaire dans le cas où ce dernier
	justifie que la réserve du droit comportait une contrepartie à sa charge.
	
	Toute convention postérieure à la promulgation de la présente loi, et comportant réserve
	de l'un des droits visés ci-dessus, doit indiquer, à peine de nullité, l'importance et la
	consistance des locaux à construire, et la modification que leur exécution entraînerait dans
	les droits et charges des copropriétaires.
	
	On voit ici la méfiance du législateur à l'égard des propriétaires d'immeubles qui veulent vendre
	l'immeuble par lots en conservant certains droits dans cet immeuble. Rédacteurs du Règlement de
	Copropriété auquel ils feront adhérer leurs acquéreurs, ils prévoient de se réserver des droits exorbitants
	tels le droit d'affouiller le sol ou de surélever l'immeuble.
	114 Plus récemment, pour la réalisation d’une nouvelle construction
	115 Cf. IRC … 2018, par Cédric Jobelot
	116 Versailles 7 novembre 1983 . Rev. Loyers 1985, p 386

	Or de telles conventions sont inscrites dans les Règlements de Copropriété dans l'intérêt du bénéficiaire
	exclusivement et en conséquence n'étaient aucunement justifiées au regard de l'objet du syndicat des
	copropriétaires. C’est pourquoi le législateur en a limité la portée, au point de procéder, sous certaines
	conditions à une véritable << expropriation privée >>\footnote{L'expression est de \nom{Cabanac}.}.
	Seul le droit de mitoyenneté échappait à cette censure.
	
	\subsubsection{L’article 37 prévoit tout d’abord une limitation de la validité de ces conventions à 10	ans}
	
	Depuis le 10 juillet 1975 toutes les conventions antérieures au 10 juillet 1965 par lesquelles un
	copropriétaire ou un tiers se réservait l'exercice d'un des accessoires sont caduques.
	Pour les règlements postérieurs à la loi du 10 juillet 1965, la caducité est également encourue si le droit
	réservé à un copropriétaire par le règlement de copropriété n’a pas été mis en oeuvre dans le délai de 10
	ans suivant la rédaction du règlement de copropriété.
	
	\subsubsection{Mentions obligatoires}
	
	 La convention doit indiquer, à peine de nullité :
	- l'importance des locaux à construire
	- la consistance des locaux à construire
	- les tantièmes de copropriété et de charges affectés aux lots ont construire.
	
	En d'autres termes la Convention ne sera valable que dans la mesure où son bénéficiaire aura permis aux
	autres copropriétaires, soit lors de leur acquisition dans l'immeuble, soit lors du vote en assemblée
	générale, de savoir avec précision la nature et même l'aspect des constructions qui risquent d'être un jour
	réalisées et de connaître les répercussions que cette construction aura dans les droits et charges de la
	copropriété.
	
	Par contre, pour être valable, la convention ne doit pas obligatoirement préciser la superficie des locaux
	qui seront réalisés. Malgré l'opinion contraire émise par MM LAFOND et STEMMER, on ne saurait ajouter
	au texte une condition qui n'y est pas précisée (Cour de Versailles 23 avril 1992) De la même façon il ne
	semble pas qu'il soit nécessaire de préciser l'affectation projetée pour ces locaux, affectation et
	consistance étant deux choses distinctes.
	
	\subsubsection{Le droit d'opposition du syndicat.}

	Le syndicat peut à la majorité de l'art. 25 (501/l000\degre{}) s'opposer à l'exercice de ce droit. Si le syndicat exerce
	ce droit d'opposition, il n'a pas à motiver sa décision.
	En cas d'exercice du droit d'opposition, le copropriétaire bénéficiaire de ce droit ne sera pas indemnisé, à
	moins de justifier que la réserve de droit inscrite à son profit comportait une contrepartie à sa charge.
	
	\subsubsection{Disparition des conventions de l’article 37 sur les « droits de construire » à compter de la loi \no 2018-1021 du 23 novembre 2018 dite ELAN}
	
	La très grande fragilité des conventions de l’article 37 depuis la loi \no 65-557 du 10 juillet 1965, ont peu à
	peu fait tomber celles-ci en désuétude, en tous cas pour ce qui concerne les conventions relatives aux
	« droits de construire » (affouiller, surélever, édifier), au profit d’une autre technique notariale, celle du
	lot « transitoire ». La loi \no 2018-1021 du 23 novembre 2018 dite ELAN acte ces évolutions et interdit,
	pour l’avenir, les conventions portant sur ces droits (article 37-1 de la loi \no 65-557 du 10 juillet 1965)
	« Art. 37-1. – Par dérogation à l’article 37, les droits de construire,d’affouiller et de
	surélever ne peuvent faire l’objet d'une convention par laquelle un propriétaire ou un tiers
	se les réserverait. Ces droits peuvent toutefois constituer la partie privative d’un lot
	transitoire. (art. 59 bis F)
	II (nouveau). – Les conventions par lesquelles un tiers ou un copropriétaire s’est réservé,
	dans les conditions prévues à l’article 37 de la loi \no 65-557 du 10 juillet 1965 fixant le statut
	de la copropriété des immeubles bâtis, dans sa rédaction antérieure à la publication de la
	présente loi, l’exercice d’un droit de construire, d’affouiller ou de surélever, demeurent
	valables.
	
	\paragraph{Les conventions antérieures à la loi ELAN}
	Celles-ci demeurent valables si elles respectent les dispositions de l’article 37 de la loi, donc si le droit de
	construire, d’affouiller ou de surélever est parfaitement défini, dès lors que ce droit a été constitué depuis
	moins de dix ans et n’a pas été remis en cause par une assemblée générale. Compte tenu de la caducité
	de l’article 37, toutes ces conventions auront néanmoins disparu au plus tard en 2028.
	
	\paragraph{Les conventions postérieures à la loi ELAN}
	Celles-ci sont désormais impossibles si elles portent sur un droit de construire, d’affouiller ou de surélever
	l’immeuble. Ce qui signifie qu’elles demeurent valables si elles portent sur un autre doit accessoire.
	Cette disposition concerne manifestement le droit de mitoyenneté afférent aux parties communes (prévu
	dans la loi de 1965) que le droit d’affichage qualifié de droit accessoire par la loi ELAN.
	Mais cette disposition concerne-t-elle également le droit administratif de construire ? La réponse semble
	devoir être négative de par la généralité des termes employés : « le droit de construire » pouvant être
	aussi bien le droit d’édifier de nouveaux bâtiments que le droit administratif de construire.

	\subsection{L’integration de droits reputes accessoires aux parties communes a des lots privatifs}
	
		Cette faculté est désormais expressément prévue par l’article 37-1 de la loi \no 65-557 du 10 juillet 1965
		Issu de la loi ELAN:
		Article 37-1 Créé par LOI \no2018-1021 du 23 novembre 2018 - art. 208 (V)
		Par dérogation à l'article 37, les droits de construire, d'affouiller et de surélever ne peuvent faire l'objet
		d'une convention par laquelle un propriétaire ou un tiers se les réserverait. Ces droits peuvent toutefois
		constituer la partie privative d'un lot transitoire.
		Ce faisant la loi ELAN entérine une pratique notariale constante, validée par la jurisprudence.
		
		\subsubsection{Fondement de la pratique notariale}
		
			L’ article 3 précise que << Sont réputés droits accessoires aux parties communes $\dots$ >> les droits de construire
			(surélever, affouiller, édifier), de mitoyenneté et d’affichage. Il établit un présomption réfragable, et il
			n’est pas d’ordre public.
			
			Il est donc tout à fait possible pour le rédacteur du règlement de copropriété de considérer que les droits
			ainsi décrits (notamment le droit de construire au sens large) peuvent être placés par le règlement de
			copropriété en droits accessoires à des parties privatives et non pas en droits accessoires à des parties
			communes.
			
			Aussi, pour éviter la rigueur des dispositions de l'article 37, la pratique notariale a souvent placé les droits
			« accessoires » dans la constitution même du lot privatif; quant à la jurisprudence elle a parfois qualifié
			ces droits de << droits accessoires à des parties privatives >> alors même que ces droits n'ont pas été
			expressément placés comme constitutifs de lots de copropriété.
			
			L’assemblée générale peut également céder un tel droit, après l’avoir constitué en un « lot transitoire ».
		
		\subsubsection{Consécration par la jurisprudence}
		
			Mieux encore qu’un droit accessoire aux parties privatives, le droit de construire est en effet constitutif
			d’une partie privative au sens de l’article 1er de la loi du 10 juillet 1965 (Civ. 3ème 30 juin 1998 ; Loyers et
			Copropriété 1998 \no 254 ou encore Civ 3ème 30 sep 1998 ; Loyers et Copropriété 1999 \no 25).
			Exemple : Lot \no 1 ; droit d'édifier un bâtiment de x étages sur le terrain délimité par les lettres ABCD
			sur le plan annexé ainsi que x x/1000 èmes des parties communes générales de l'immeuble.
			Il est possible également d’intégrer ce droit dans la définition du lot privatif figurant à l’état descriptif de
			division, en modulant en fonction les tantièmes affectés au lot.
			Exemple : Lot \no X, "Un appartement avec terrasse et droit d'édifier un étage supplémentaire sur
			cette terrasse".
		
			En ce cas, le copropriétaire n’est pas tenu de demander au syndicat la cession du droit de construire
			comme dans l’hypothèse précédente. Il devra en revanche, à l’époque où il entendra entreprendre les
			travaux, obtenir l’autorisation de l’assemblée si la construction est susceptible d’affecter les parties
			communes ou l’aspect extérieur de l’immeuble, en application de l’article 25 b de la loi.
			Une décision de la 3ème Chambre Civile de la Cour de Cassation en date du 8 juin 2011 a même admis que
			le promoteur n’était pas soumis aux règles d’autorisation de la copropriété « puisqu’en vertu du règlement
			de copropriété, son titulaire bénéficiait du droit d’édification de tous bâtiments et constructions ».\footnote{
			Une société d’HLM avait édifié une copropriété par tranches successives, se réservant dans le règlement de copropriété et dans	l’état descriptif de division, le « droit d’édification de tous bâtiments et constructions », sous la forme d’un lot transitoire assorti	de tantièmes de parties communes.
			}
			On notera cependant un arrêt de la Cour d’Appel de Paris en date du 21 janvier 2015 RG \no 13/03562
			rendu sur un lot simplement défini comme « la propriété exclusive et particulière d'un droit à construire
			sur la partie de terrain cadastrée (xx) avec droit à la jouissance exclusive et particulière dudit terrain » qui
			considère que : « la consistance de ce droit n'étant pas clairement définie dans le règlement de
			copropriété, sa mise en oeuvre nécessitait une autorisation de l'assemblée générale des copropriétaires »
			(majorité de l’article 25-II-b).
		
		\subsubsection{Le « Lot transitoire » consacré et encadré par la loi ELAN}
		
			La loi \no 2018-1021 du 23 novembre 2018 dite ELAN, tout en consacrant la technique du lot transitoire, a
			tâché d’en encadré la pratique, dans un effort assez semblable à celui de l’article 37 en 1965, en réaction
			à cette jurisprudence libérale.
		
			Dans le cadre de la mise en œuvre de son droit à construire institué en lot transitoire, la Société d’HLM a supprimé un certain
			nombre d’espaces et aménagement (voie de circulation goudronnée et ses équipements, un rond-point) que les demandeurs à la
			procédure considéraient partie communes ou parties à usage commun.
			
			A ce titre, ces derniers soutenaient qu’une autorisation de l’assemblée générale aurait été nécessaire et sollicitaient le
			rétablissement desdits aménagements.
			
			La Cour de Cassation, tout comme la Cour d’Appel a néanmoins jugé :
			\begin{itemize}
				\item D’une part, que le lot transitoire considéré ne faisait état d’aucune partie commune générale ou spéciale.
				\item  D’autre part, que les aménagements supprimés n’étaient que des aménagements provisoires dont les copropriétaires
				n’avaient qu’un droit de jouissance temporaire puisque lesdits équipements n’avaient pas la qualité de parties
				communes et étaient appelés à disparaître en fonction de la poursuite du programme de construction.
				\item  Qu’en conséquence, le droit à construire sur le lot transitoire n’était pas soumis aux règles d’autorisation de la
				copropriété.
			\end{itemize}
			
			L’arrêt considéré ne se prononce pas sur l’implication des constructions litigieuses sur les parties communes ou l’aspect extérieur de
			l’immeuble, mais uniquement sur la suppression d’aménagements existants.
		
			Or, en ce qui concerne de tels aménagements, la Cour d’appel puis la Cour de Cassation, ne les ont pas retenus en tant que parties
			communes ou éléments d’équipement communs mais en tant qu’aménagements provisoires dont les copropriétaires n’avaient qu’un
			droit de jouissance temporaire.
			
			Article 1 alinéa 3-issu de la loi \no 2018-1021 du 23 novembre 2018 dite ELAN
			Ce lot peut être un lot transitoire. Il est alors formé d'une partie privative constituée d'un droit de
			construire précisément défini quant aux constructions qu'il permet de réaliser sur une surface
			déterminée du sol, et d'une quote-part de parties communes correspondante. La création et la
			consistance du lot transitoire sont stipulées dans le règlement de copropriété.
			
			Article 37-1 Créé par LOI \no2018-1021 du 23 novembre 2018 - art. 208 (V)
			Par dérogation à l'article 37, les droits de construire, d'affouiller et de surélever ne peuvent faire l'objet
			d'une convention par laquelle un propriétaire ou un tiers se les réserverait. Ces droits peuvent toutefois
			constituer la partie privative d'un lot transitoire.
			
			Ainsi :
			\begin{itemize}
				\item La technique du lot transitoire est consacrée et devient la seule technique possible
				\item  Ce droit est encadré : le droit de construire doit être précisément défini quant aux constructions
				qu'il permet de réaliser et d'une quote-part de parties communes correspondante. La référence au
				sol, qui figurait dans le texte de la loi ELAN, a été vivement condamnée par la pratique et par la
				doctrine \footnote{Cf. notamment l’article du Professeur POUMAREDE à la Revue de Droit Immobilier, Dalloz, 2019, 44} puisqu’ainsi rédigé le texte ne permettait plus de créer de lots transitoires de
				surélévation pourtant relativement fréquents en pratique à une époque où la disparition de terrains
				à bâtir conduit le même législateur à favoriser la surélévation des immeubles. C’est pourquoi
				l’Ordonnance du 30 octobre 2019 rectifie le tir en supprimant les mots malvenus « sur une surface
				déterminée du sol ».
				\item  Ce lot transitoire doit être mentionné dans le règlement de copropriété. Or, la plupart du temps, il
				ne figurait que dans l’état descriptif de division
				\item  L’article 206 \II de la loi \no 2018-1021 du 23 novembre 2018 dite ELAN fait obligation au syndicat de
				copropriété de mettre, dans un délai de 3 ans à compter de la promulgation de la loi, les règlements
				de copropriété en conformité avec les dispositions relatives aux lots transitoires, donc l’obligation
				faite à l’assemblée générale de définir la consistance du lot transitoire figurant dans l’état descriptif
				de division de l’immeuble existant. Toutefois, ce texte ne dit rien sur les conséquences de l’absence
				de mise à jour ( inexistence du droit comme contraire à l’article 1 ? Mais l’article n’est pas d’ordre
				public $\dots$), ou la compétence du Juge en cas de refus d’adaptation.
			\end{itemize}

\section{Parties communes a jouissance privative}\footnote{Sur la question lire l’excellent article de Jean-Marc Roux in Loyers et Copropriété d’octobre 2004 Et 9.}

	Bien que le législateur n’ait aucunement évoqué cette possibilité, la pratique (c’est à dire les rédacteurs
	des règlements de copropriété) a imaginé de placer une partie d’immeuble en partie commune, mais d’en
	donner la jouissance privative à un lot déterminé.

	C’est ainsi que le sol, un jardin, une cour, une terrasse pourront être en tout ou en partie classés en
	<< parties communes à jouissance privative >>.
	
	On s’interroge sur la nature juridique d’un droit de jouissance sur partie commune. A l’évidence il ne s’agit
	pas d’une servitude : nous ne sommes pas en présence de propriétés distinctes : le titulaire d’un droit de
	jouissance exclusive n’est pas propriétaire, quand bien même ce droit de jouissance doit être l’accessoire
	d’un lot.
	
	Les auteurs se sont interrogés sur la régularité et la qualification d’un tel droit de jouissance privatif. Mais
	la jurisprudence a très tôt reconnu la validité de cette institution. C’est la jurisprudence qui a élaboré
	progressivement le régime de ces « droits de jouissance ».
	
	La loi \no 2018-1021 du 23 novembre 2018 dite ELAN a consacré cette institution :
	\begin{quote}
		Art. 6‑3 de la loi \no 65-557 du 10 juillet 1965 \\
		Les parties communes à jouissance privative sont les parties communes affectées à l’usage et
		à l’utilité exclusifs d’un lot. Elles appartiennent indivisément à tous les copropriétaires. Le droit
		de jouissance privative est nécessairement accessoire au lot de copropriété auquel il est
		attaché. Il ne peut en aucun cas constituer la partie privative d’un lot.
	\end{quote}
	
	\subsection{Le droit de jouissance exclusive est un droit reel}
	
		\subsubsection{Le droit de jouissance est transmissible aux propriétaires successifs du lot}
		
			Si ce droit de jouissance est accordé à un copropriétaire et non attaché à un lot, il ne s’agit plus alors d’un
			droit réel, mais d’un droit personnel qui disparaît avec le décès de son bénéficiaire ou la cession du lot à
			un tiers.
			
			Au contraire, lorsque le « droit de jouissance exclusif » attaché au lot, suit celui-ci entre les mains de tout
			ayant droit du copropriétaire du lot : constituant un droit réel, il est transmissible à tous les acquéreurs
			successifs\footnote{Civ 3\degre{} 17 juin 1997, Loyers et Copropriété 1997 \no 296}, et ce même si son titulaire y renonce, tant que cette renonciation ne se traduit pas par un
			modificatif au règlement de copropriété\footnote{Civ.3ème 21 juin 2006}.
			Cette création d’un nouveau démembrement de la propriété, non prévu par le Code Civil, n’était pas
			évident, car la liste des droits réels prévus par le code civil a longtemps été considérée comme limitative.
			Cependant, le premier arrêt MAISON DE LA POESIE\footnote{Note 14 Cass. 3e civ., 31 oct. 2012, \no 11-16.304 : JurisData \no 2012-024285 ; Bull. civ. 2012, III, \no 159 ; JCP G 2012, \no 52, 1400,
				note F.-X. Testu ; Defrénois 2013, p. 13, obs. L. Tranchant ; LPA 16 janv. 2013, p. 11, note F.-X.Agostini. – V. sur cette décision, H. Périnet-
				Marquet, La liberté de création des droits réels est consacrée : Constr.-urb. 2013, Repère 1 . – V. précédemment Cass. 3e civ., 23 mai
				2012, \no 11-13.202 : JurisData \no 2012-010886 ; Bull. civ. 2012, III, \no 84 ; LPA 24 oct. 2012, p. 12, note J.-F. Barbieri. – V. sur cette
				décision, F. Danos, Perpétuité, droits réels sur la chose d'autrui et droit de superficie : Defrénois 2012, art. 40637, p. 106 ; ZALEWSKISICARD  Vivien 03/10/2016} en date du 31 octobre 2012, a admis qu’ « un
			propriétaire peut consentir, sous réserve des règles d'ordre public, un droit réel conférant le bénéfice d'une
			jouissance spéciale de son bien ». Il s’agissait précisément d’un droit de jouissance exclusif, cet arrêt
			consacrant le fait que la liste des droits réels n’était pas limitative.
		
		\subsubsection{Le droit de jouissance est perpétuel}
		
			Si le droit de jouissance exclusif est un démembrement de la propriété des parties communes, il devrait,
			en principe, ne pas avoir un caractère perpétuel : le droit de propriété a vocation a se reconstituer
			pleinement à terme, comme prévu par l’article 619 (usufruit) ou 625 ( droit d’usage et d’habitation).
			Pour autant, les droits de jouissance exclusifs sur les parties communes ne sont jamais assortis d’une
			durée. Il est vrai que ce droit s’exerce lui-même sur des parties communes en indivision « perpétuelle »,
			ce qui est déjà une anomalie au regard des règles du Code Civil. Cette question de la perpétuité du droit
			de jouissance exclusif a donné lieu à plusieurs arrêts récents :
			
			\begin{itemize}
				\item \item Dans un arrêt du 28 janvier 2015, la cour de cassation\footnote{3\degre{} Ch Civ 28 janvier 2015 (14-10013) Sur le site de la Cour de Cassation} affirme que : « lorsque le propriétaire
				consent un droit réel, conférant le bénéfice d’une jouissance spéciale de son bien, ce droit, s’il n’est pas limité
				dans le temps par la volonté des parties, ne peut être perpétuel et s’éteint dans les conditions prévues par
				les articles 619 et 625 du code civil » (l’article 619 est relatif à l’usufruit et l’article 625 au droit d’usage
				et d’habitation. Cette formule, très générale, semblait pouvoir s’appliquer au droit de jouissance
				privative sur une partie commune. Dans le cas d’espèce, le syndicat des copropriétaires avait
				consenti à l’EDF (aujourd’hui ERDF) un droit d’usage sur un lot composé d’un transformateur sans
				limitation de durée, et ce droit a été considéré comme éteint au bout de 30 ans.
				
				\item Dans un second arrêt MAISON DE LA POESIE en date du 8 septembre 2016\footnote{Propriété - Maison de Poésie II : combien de temps dure la perpétuité en France ? - Note sous arrêt par Julien Laurent 07/11/2016	La Semaine Juridique - Edition générale}, la cour de Cassation	a précisé que la stipulation par laquelle un droit réel de jouissance spéciale est consenti pour la	durée d'une Fondation, ne rend pas ce droit perpétuel, lui permettant d'échapper au terme
				trentenaire que fixent les articles 619 et 625 du Code civil. En d’autres termes, elle considère que
				si le droit été stipulé comme perpétuel, il aurait été soumis à extinction au bout de 30 ans$\dots$ Même
				si, en pratique, il suffit de proroger la durée de vie de la personne morale pour que ce droit soit
				effectivement perpétuel. Ainsi, dans cet arrêt, la Cour de cassation consacre la « quasi perpétuité »
				de ce démembrement de propriété, mais refuse d’aller jusqu’à la perpétuité. Ce positionnement
				est manifestement en décalage avec les arrêts rendus en matière de droit de jouissance sur parties
				privatives en copropriété.
				
				\item Dans un arrêt du 7 juin 2018, publié au Bulletin\footnote{civ. 3\degre{} Ch. 7 Juin 2018, \no 17-17.240, P+B+R+I , Semaine Juridique Edition Générale \no 36, 3 Septembre 2018, 893,	Commentaire Hugues Périnet-Marquet : « Est perpétuel un droit réel attaché à un lot de copropriété conférant le bénéfice	d'une jouissance spéciale d'un autre lot ; la cour d'appel a retenu que les droits litigieux, qui avaient été établis en faveur}, la troisième chambre civile a consacré la	perpétuité du droit réel attaché à un lot de copropriété, ceci quand bien même en l’espèce il ne s’agissait pas du droit de jouissance consenti à un lot sur une partie commune, mais du droit de
				jouissance consenti à l’ensemble des copropriétaires successifs sur une piscine dépendant d’un lot
				privatif En effet, avec le Professeur Périnet-Marquet, on peut penser que : « l’arrêt laisse entendre
				clairement que, dans le cadre des copropriétés, les droits conférés soit sur des lots soit, sans doute,
				sur des parties communes, présentent un caractère perpétuel qui ne peut être remis en cause ».
				Pour lever toute équivoque il eut été bon que la loi ELAN, en créant l’article 6-3 dans la loi de 65 relatif aux
				parties communes à jouissance privative, précisa qu’il s’agissait d’un droit réel perpétuel, mais la mention
				n’y figure pas.
			\end{itemize}
		
		\subsubsection{Une jouissance privative peut faire l’objet d’une prescription acquisitive trentenaire}
		
			Un copropriétaire qui manifeste sans équivoque son intention de se comporter en seul bénéficiaire de la
			jouissance d’une partie commune, en interdisant l’accès aux autres copropriétaires, étant allés jusqu’à
			fermer à clé l’accès à cette cour (Civ. 3\degre{} 24 octobre 2007, pourvoi \no 06-19260)
			« Un droit de jouissance privatif sur des parties communes est un droit réel et perpétuel qui peut
			s’acquérir par usucapion »\footnote{Civ. 3\degre{} 24 octobre 2007, pourvoi \no 06-19260}
			
			On notera également – mais sur un fondement plus traditionnel - l’hypothèse où le syndicat acquiert la
			prescription d’une cave appartenant à un propriétaire voisin dont il a eu la jouissance pendant plus de 30
			ans\footnote{Paris 23\degre{} Ch A 7 mai 2002 Loyers et Copropriété 2002 \no 270}.
			
			Cependant la jurisprudence est particulièrement exigeante pour accepter la prescription acquisitive
			(jouissance trentenaire paisible, publique, continue, non équivoque et à titre de propriétaire
			conformément à l’article 2230 du code civil).
			
			De plus, il est de jurisprudence constante qu’une simple tolérance à l’occupation d’une partie commune
			(autorisée par l’assemblée générale) ne donne pas vocation à son bénéficiaire d’en acquérir la propriété
			ou la jouissance exclusive par usucapion, car nul ne peut prescrire contre son titre\footnote{3e civ., 6 mai 2014, \no 13-16.790, F-D (JCPN du 6 juin 2014) ; 3\degre{} civ. 6 septembre 2018 \no pourvoi: 17-22180 Non publié au bulletin Rejet}.
		
		\subsubsection{}
		
			En tant que droit réel accessoire aux parties communes le droit de jouissance exclusive peut également être concédé par décision d’assemblée générale :
			
			des autres lots de copropriété et constituaient une charge imposée à certains lots, pour l'usage et l'utilité des autres lots
			appartenant à d'autres propriétaires, étaient des droits réels \emph{sui generis} trouvant leur source dans le règlement de
			copropriété et que les parties avaient ainsi exprimé leur volonté de créer des droits et obligations attachés aux lots des
			copropriétaires ; il en résulte que ces droits sont perpétuels
		
			La décision sera alors prise à la double majorité de l’article 26 de la loi\footnote{PARIS 23\degre{} Ch A 21 déc 1994 D 97 IR 244}. Cette cession ne devant pas	cependant être confondue, ni avec la cession de la propriété de la partie commune, ni avec une jouissance	précaire et révocable accordée personnellement à un copropriétaire déterminé.
		
		\subsubsection{Ce droit de jouissance ne peut être remis en cause en cas de non usage}
		
			Les copropriétaires ne peuvent imposer au lot bénéficiaire la perte de ce de droit de jouissance, même
			s’il n’est plus mis en œuvre depuis 30 ans – en l’espèce, la station essence n’utilisait plus le droit de
			jouissance sur la cour\footnote{Civ 3\degre{} 4 mars 1992, Rép. Déf. 1992-1140, obs Henri Souleau ; D 1992-2-386 Ch. Atias.}
	
	\subsection{Les attributs du droit de jouissance}
	
		\subsubsection{Ce droit emporte la jouissance exclusive de la partie commune}
		
			Le copropriétaire qui a la jouissance exclusive d'une partie du sol est en droit de s'opposer au passage des
			occupants de l'immeuble par cette partie commune à jouissance privative, ce quand bien même le passage
			est demandé pour les besoins de l'entretien de l'immeuble.
			
			En l'espèce, "le titulaire d'un droit à jouissance exclusive du jardin, est en droit de s'opposer, sans
			commettre d'abus, au passage quotidien des conteneurs à ordures ménagères, dont le transit
			pouvait, par un aménagement approprié, s'effectuer par les parties communes de l'immeuble non
			affectées à un droit de jouissance privative".\footnote{PARIS 14 décembre 1990 - D. 91 IR 15}
			
			C’est ainsi que la Cour de Cassation a jugé\footnote{Civ 3 19 déc 1990, D 91 IR p 15} que : 
			\begin{quote}
				Le droit de jouissance, affecté d'une quote-part
				des parties communes correspondant aux charges que son titulaire supporte pour la conservation
				et l'entretien de la cour, n'est pas assimilable à un droit de propriété et ne donne pas à son titulaire
				la possibilité de transformer en local clos privatif une cour rangée par le règlement de copropriété
				dans les parties communes.
			\end{quote}
		
			Le copropriétaire titulaire d’un droit de jouissance exclusif a qualité à agir pour faire cesser un
			empiétement sur la partie commune sur laquelle s’exerce son droit de jouissance exclusif\footnote{(Cass. 3e civ., 15 décembre 2016, \no 15-22.583, F-D : JurisData \no 2016-028801 ; Loyers et copr.
			2016, comm. 55, note A. Lebatteux)}
			\begin{quote}
				Sur le moyen unique, ci-après annexé :
			
				Attendu, selon l'arrêt attaqué (Pau, 29 mai 2015), que la SCI S… et M. X..., copropriétaires de lots
				dans le bâtiment A d'un immeuble en copropriété, ont obtenu, par délibération de l'assemblée
				générale des copropriétaires de la résidence C$\dots$ (le syndicat des copropriétaires) du 12 décembre
				2005, réitérée le 19 décembre 2008, l'autorisation d'affouiller le sol d'un terrain affecté à la
				jouissance exclusive du bâtiment A pour y construire une piscine ; que M. Y$\dots$, copropriétaire de deux
				lots dans un autre bâtiment de cet immeuble, se plaignant de l'empiétement de la piscine sur le
				jardin affecté à son usage privatif, a assigné la SCI S…, M. X$\dots$ et le syndicat des copropriétaires en
				annulation de la délibération de l'assemblée générale du 19 décembre 2008 et démolition de la
				piscine par les deux premiers ;
				
				Si un droit de jouissance exclusive sur des parties communes n'est pas un droit de propriété, le
				titulaire de ce droit réel et perpétuel a qualité et intérêt à assurer la défense en justice, sur le
				fondement de l'article 15 de la loi du 10 juillet 1965 $\dots$
			\end{quote}
		
		\subsubsection{Le droit de jouissance n’est pas un droit de propriété, en sorte que la partie communes	reste indivise entre tous les copropriétaires}
		
			\begin{itemize}
				\item Le droit de jouissance n’emporte pas requalification de la partie commune, ce n’est qu’un droit de
				« superficie\footnote{Civ 3\degre{} 29 oct 1973 Bull Civ III \no 552} ».
				
				La question pouvait se poser compte tenu de la rédaction de l’article 2 : « sont parties privatives
				les parties des bâtiments et des terrains réservés à l’usage exclusif d’un copropriétaire ». Sur ces
				dispositions un copropriétaire dont le lot est constitué d’une maison et de la jouissance privative
				d’un terrain, affirme qu’il a en fait la propriété du terrain correspondant. La cour de cassation
				répond :
				\begin{quote}
					qu’ayant, par motifs propres et adoptés, relevé que les lots des copropriétaires étaient
					composés du droit à la jouissance exclusive et privative d’une parcelle de terrain sur
					lesquels est implantée chaque maison et la propriété privative des constructions ainsi que
					de millièmes de parties communes, la cour d’appel a retenu, à bon droit et sans
					dénaturation, que seul un droit réel de jouissance était conféré aux copropriétaires et que
					le sol était une partie commune\footnote{Civ. 3ème Ch. 2 octobre 2013, \no pourvoi 12-17084 – au Bulletin}.
				\end{quote}
				
				Ce principe est rappelé dans l’article 6-3 de la loi de 1965 issu de la loi loi \no 2018-1021 du 23
				novembre 2018 dite ELAN :
				[Les parties communes indivises] appartiennent indivisément à tous les copropriétaires.
				
				\item  Le droit de jouissance ne confère pas le droit de construire.
				
				La jouissance privative d’une partie commune n’entraîne en aucun cas un doit de construire, lequel
				demeure un accessoire des parties communes\footnote{Civ 3\degre{} 22 juillet 1987 D 87 IR p 187}. De la même façon le bénéficiaire ne peut affouiller le sol
				par exemple pour y creuser une piscine.
				
				Il est même douteux que le copropriétaire bénéficiaire de ce droit de jouissance ait la faculté de clore la
				partie dont il a la jouissance\footnote{Civ 3\degre{} 22 juillet 1987 D 87 IR p 187}.
			
				\item  Le droit de jouissance ne permet pas le retrait de la copropriété.
				Un tel droit de jouissance n’étant pas un droit de propriété ne permet pas à son titulaire de retirer le lot
				de la copropriété pour cette partie de jouissance\footnote{Civ 3\degre{} 29 janvier 1997, Administrer juin 1997 p. 42 note Capoulade} quand bien même serait-elle assortie du droit de
				construire.
			\end{itemize}
		
		\subsubsection{Le droit de jouissance s’exerce dans le respect de la destination de la partie commune}
		
			Le droit de jouissance exclusif sur une partie commune ne modifie pas la destination de cette partie
			commune ; il doit donc être conforme aux dispositions de l’acte qui l’a institué\footnote{Civ. 3\degre{}, 20 mars 2002, Loyers et Copropriété 2002, \no 159 ; Civ. 3\degre{} 26 mai 2006 ; Juris-Data \no 2006-302912 ; Civ. 3\degre{}, 8 nov 2006,Administrer jan 2007 p. 59 ; Versailles 30 jan 2012, Loyers et Copr 2012 \no 152}.
		
			Notons qu’il n’est pas toujours facile de définir ce qui relève de la jouissance privative d’une partie
			commune : par exemple lorsque le jardin est à jouissance privative … l’arbre est-il privatif ? La réponse
			semble devoir être négative\footnote{PARIS 23\degre{} Ch B 11 avril 2002 Loyers et Copropriété 2002 \no 268}. Il a été même été jugé que la décision de l'assemblée générale d’effectuer
			des plantations dans un jardin à jouissance privative en remplacement de végétaux malades ne portait pas
			atteinte aux modalités de jouissance de son lot par le copropriétaire\footnote{Administrer \no 375 mars 2005}.
		
		\subsubsection{Le droit de jouissance peut générer une obligation spécifique d’entretien}
		
			En principe, le copropriétaire titulaire du droit de jouissance privative doit assurer l’entretien du
			revêtement superficiel, et même sa remise en état à la suite de travaux engagés par le Syndicat des
			Copropriétaires sur l’étanchéité partie commune :
			\begin{itemize}
				\item les travaux rendus nécessaires par l'état de la terrasse relèvent du syndicat pour le grosoeuvre
				et l'étanchéité de l'ouvrage, et du bénéficiaire du droit d'usage exclusif pour le surplus
				(Cass. 3e civ., 18 déc. 1996 : Juris-Data \no 1996-005007 ; Loyers et copr. 1997, comm. 90. –
				CA Paris, 20 juin 2001 : Juris-Data \no 2001-146857).
				\item le syndicat décideur des travaux est fondé à réclamer au copropriétaire le coût des
				dépenses engagées pour la remise en état du revêtement superficiel, de même que pour la
				remise en place des autres installations privatives réalisées sur la terrasse (Cass. 3e civ.,
				30 avr. 2002 : Juris-Data \no 2002-014270 ; Administrer nov. 2002, p. 36. – CA Aix-en-
				Provence, 1re ch. civ., 6 mai 1997 : D. 1998, somm. 122. – CA Versailles, 20 déc. 1990 :
				Administrer mars 1991, p. 64).
				\item Les frais de pose de carrelage des terrasses doivent, après la réfection de leur étanchéité, être
				supportés par les seuls copropriétaires qui en ont la jouissance exclusive et l'obligation
				d'entretien (CA Aix-en-Provence, 6 mai 1997 : D. 1998, somm. p. 122. – CA Reims, 27 sept.
				2004 : JurisData \no 2004-266119 ; JCP G 2005, IV, 2402).
				\item doit être qualifiée de partie commune à usage exclusif d'un copropriétaire le toit-terrasse
				d'un immeuble. Une distinction doit cependant être faite entre le revêtement superficiel et
				tous ses accessoires, partie privative et l'ossature de l'immeuble, y compris l'étanchéité qui y
				est incorporée qui est une partie commune. Lorsque des travaux d'étanchéité sont effectués
				sur un toit-terrasse, partie commune à jouissance exclusive, les frais de dépose et de repose
				des aménagements privatifs mis en place, en l'absence de disposition contraire du règlement
				de copropriété, doivent rester à la charge du copropriétaire bénéficiaire du droit de
				jouissance exclusive. (Cour d'appel Bastia, 11 Juin 2008, \no 06/01099)
				Le copropriétaire ne peut pas refuser l’accès au Syndicat des Copropriétaires pour effectuer les travaux
				sur les parties communes dont il a la jouissance, en revanche le syndic doit respecter les modalités prévues
				par l’article 9 de la loi du 10 juillet 1965 (cf infra, poly 2 : les travaux)
			\end{itemize}
		
			Ce principe est consacré par l’ordonnance du 30 octobre 2019 qui complète l’article 6-3 par un dernier
			alinéa
			\begin{quote}
				Le règlement de copropriété précise, le cas échéant, les charges que le titulaire de ce droit de
				jouissance privative supporte.
			\end{quote}
	
	\subsection{Le droit de jouissance d’une partie commune est necessairement l’accessoire d’un lot comportant des parties privatives}
	
		\subsubsection{Le droit de jouissance est indivisible par rapport au lot auquel il est attribué}\footnote{Civ 3\degre{} 8 juillet 1992 RDI 1992 p 364}
		
			Le lot ne doit pas être constitué du seul droit de jouissance : le droit de jouissance est présenté
			comme le complément d’un lot ; par exemple : << lot \no 1 appartement au rez-de- chaussée et
			jouissance exclusive d’une cour >>.
			
			En effet, un lot de copropriété comporte nécessairement une partie privative et une quote-part de parties
			communes ; or, un lot ne comportant qu’un droit de jouissance ne comporte pas de partie privative ; selon
			l’expression de M \nom{CAPOULADE} : « Il n’en représente qu’une dépendance, un complément ou un
			accessoire ».
			
			Cette question est à l’origine d’une procédure complexe dans une affaire Fournier c/ Syndicat Pauline
			Borghèse, dans laquelle un copropriétaire détenait un lot composé « du droit de jouissance du jardin et
			de 48/10.000 ème de propriété du sol et des parties communes\footnote{Civ.3ème 6 novembre 2002 puis Civ3ème 1er Mars 2006} >>.
			La cour de cassation dans un arrêt du 6 novembre 2002 constate qu’un lot de jouissance $\dots$ ne peut être
			un lot de copropriété ; elle casse. La cour de renvoi (Paris 23 juin 2004) en déduit que le lot a
			disparu. Cassation à nouveau le 1er mars 2006. L’arrêt ne nie pas l’existence du droit de jouissance
			mais considère que n’étant pas attaché à un lot, les tantièmes affectés au lot doivent être retranchés
			des tantièmes de propriété. Le Président \nom{Capoulade}\footnote{Administrer février 2003 p. 45} et M \nom{Vigneron}\footnote{Loyers et Copropriété mai 2006 \no 112} en déduisent que ces
			tantièmes n’ont d’existence qu’en tant que tantièmes de charges (Civ. 3\degre{} 4 mai 1995, Bull. civ. \no
			113).
			
			Un arrêt de la 3ème Chambre de la Cour de Cassation\footnote{Civ. 3\degre{} Arrêt \no 546 Syndicat des Copropriétaire Les Rotondes}, en date du 6 juin 2007 semble apporter un point
			final à cette discussion :
			\begin{quote}
				« Un droit de jouissance exclusif sur des parties communes n’est pas un droit de propriété et ne peut	constituer la partie privative d’un lot ».
			\end{quote}
			
			Cette décision est d’autant plus importante qu’il ne s’agissait pas en l’espèce d’un lot de jouissance du sol
			mais d’un « lot de jouissance exclusive et particulière d’emplacement de stationnement ». On peut penser
			que si l’auteur de l’état descriptif de division avait simplement qualifié le lot « d’emplacement de
			parking », nous aurions été en présence d’un véritable lot comportant une partie privative et non un
			simple lot de jouissance.
			
			Pour autant un lot constitué à la fois d’un droit de jouissance exclusive et d’un droit de construire, même
			non défini dans sa consistance future, n’est pas un « lot de jouissance » mais un « lot transitoire », c’est-à-
			dire un lot comme les autres comportant bien une partie privative\footnote{Civ 3\degre{} - Arrêt \no 699 du 8 juin 2011 (10-20.276) Bulletin numérique des arrêts publiés}
			\begin{quote}
				De la même façon, ce droit de jouissance ne peut être aliéné séparément du lot auquel	il est attaché\footnote{Paris 23\degre{} Ch 16 fév 2006, SCI ALTAIR c/ Syndicat 34 rue Guynemer}, même pour être rattaché à un autre lot.
			\end{quote}
			
			Toutefois la cour de cassation dans un arrêt du 17 décembre 2013 a admis implicitement que le droit de
			jouissance appartenant à un propriétaire peut être partagé entre deux lots, avec l’accord de l’assemblée
			générale du syndicat des copropriétaires\footnote{Cass. 3e civ., 17 déc. 2013, \no 12-23.670, FS-P+B}.
			Jurisprudence sans doute caduque avec l’entrée en vigueur de la loi ELAN puisque le nouvel article 6-3 de
			la loi du 10 juillet 1965 édicte que « Le droit de jouissance privative est nécessairement accessoire au lot
			de copropriété auquel il est attaché ».
		
		\subsubsection{Un « lot de jouissance » ne peut être doté de tantièmes de copropriété}
		
			Ce principe figure désormais dans l’article 6-3 issu de la loi loi \no 2018-1021 du 23 novembre 2018 dite
			ELAN : « Le droit de jouissance privative est nécessairement accessoire au lot de copropriété auquel il est
			attaché. Il ne peut en aucun cas constituer la partie privative d’un lot. »
		
			Un droit de jouissance exclusif sur les parties communes n'étant pas un droit de propriété ne peut
			constituer la partie privative d'un lot de copropriété. En conséquence un copropriétaire est fondé à
			demander au tribunal qu’il annule les tantièmes de ce lot\footnote{Civ. 3\degre{} 4 nov 2014, Pourvoi \no 13-22243, non publié au Bulletin}.
			\begin{quote}
				Vu les articles 1 alinéa 2 et 2 de la loi du 10 juillet 1965 ;
				 
				Attendu que pour rejeter la demande de M. X... tendant à l'annulation des tantièmes de copropriété afférents à
				l'espace dit « lot 19 », la cour d'appel retient que le lot \no 19 ne comporte qu'un jardin et une cour sans aucune
				édification, de sorte qu'il est improprement désigné comme lot de parties privatives et constitue en réalité un droit
				de jouissance exclusif sur des parties communes, mais qu'il est désormais admis que le droit de jouissance exclusif
				d'un copropriétaire soit affecté d'une quote-part des parties communes correspondant aux charges que son titulaire
				doit supporter sans pour autant être assimilé à un droit de propriété ;
				
				Qu'en statuant ainsi, alors qu'elle avait constaté que le lot \no 19 était constitué d'un droit de jouissance exclusif sur		des parties communes et d'une quote-part de parties communes dans la propriété du sol, la cour d'appel n'a pas tiré	les conséquences légales de ses constatations
			\end{quote}
			
			Cet arrêt rend extrêmement compliquée la cession du droit de jouissance par le syndicat des
			copropriétaires à un copropriétaire : en effet, pour réaliser cette cession, il est traditionnellement procédé
			à la création d’un lot » droit de jouissance privative de la terrasse », qui subsiste un instant de raison, puis
			qui est réuni au lot principal. Cette technique semble condamnée si l’on ne peut assortir des tantièmes un
			lot de jouissance $\dots$
		
		\subsubsection{Quel est le sort d’un « droit de jouissance » qui ne peut être rattaché à un lot ?}
		
			Par ailleurs, la question qui se pose inévitablement est le sort des « lots de jouissance » séparées des lots
			principaux, tels que les parkings, qui ont été constitué au fil des années. Sont-ils, du fait de leur irrégularité,
			dépourvu de toute existence et dispensés de toute participation aux charges ? La Cour de cassation a
			répondu de façon pour le moins ambiguë dans un arrêt du 2 décembre 2009, 08-20.310, Publié au bulletin,
			concernant des lots de jouissance de stationnement :
			
			\begin{quote}
				Attendu que le syndicat fait grief à l'arrêt de refuser de constater l'inexistence du droit de jouissance exclusive de M.	X$\dots$ sur les emplacements de stationnement, alors, selon le moyen, qu'un droit de jouissance exclusive sur une partie		commune n'est pas un droit de propriété et ne peut constituer la partie privative d'un lot ; qu'il en résulte que, privé		de cause et d'objet, ce droit disparaît avec le lot le constituant ; qu'en refusant dès lors de constater l'inexistence du	droit de jouissance de M. X$\dots$ sur les emplacements de parking, la cour d'appel a violé l'article 1er de la loi du 10 juillet	1965 ;
				
				Mais attendu que la cour d'appel a exactement retenu que si le seul droit de jouissance exclusif sur un ou plusieurs
				emplacements de stationnement ne conférait pas la qualité de copropriétaire, son titulaire bénéficiait néanmoins
				d'un droit réel et perpétuel et qu'il n'y avait pas lieu de constater que le droit de jouissance exclusif de M. X... sur	ces emplacements avait disparu ;
				
				D'où il suit que le moyen n'est pas fondé ;
			
				Sur le second moyen $\dots$
				
				Attendu qu'ayant relevé que selon les stipulations du règlement de copropriété les bénéficiaires de droit de
				jouissance exclusif sur les emplacements de stationnement n'étaient redevables que des frais d'entretien et de
				réparation de ces emplacements, la cour d'appel a retenu à bon droit, sans dénaturation, que la délibération \no 2 de
				l'assemblée générale du 4 juin 1998 mettant à la charge de ses bénéficiaires une quote part des charges communes,
				alors qu'ils n'avaient pas la qualité de copropriétaires, devait être annulée.
			\end{quote}
			
			Ainsi, la Cour de cassation quelque peu embarrassée reconnaît avoir créé un « \nom{ojni} » (objet juridique non
			identifié) : le droit de stationnement est un droit de jouissance exclusif qui doit être reconnu, alors même
			qu’il ne peut être constitutif d’un lot, au titre duquel le bénéficiaire peut être redevable de frais d’entretien
			et de réparation, mais cette obligation ne peut être traduite en quote-parts de charges communes
			générales ! Le titulaire d’un tel lot se trouve donc dans la situation d’un tiers à la copropriété qui devrait à
			celle-ci une redevance pour la jouissance dont il bénéficie, les modalités de calcul de cette redevance
			n’étant toutefois pas précisées\footnote{Doctrine sur le sujet : Loyers et Copropriété \no 10, Octobre 2015, dossier 6 Droit de jouissance exclusif et copropriété : une histoire	tourmentée Etude par Jacques LAFOND docteur en droit - avocat honoraire au barreau de Paris ; D. SIzaire. ss Cass. 3e civ., 1er mars 2006 \no 04-18.547 : JurisData \no 2006-032514 ; Constr.-Urb. 2006, comm. 139 ; V. C. Calfayan, La notion de partie privative à l'épreuve du droit de jouissance exclusive sur une partie commune : JCP N 2007, \no 1273. – Y. Stemmer, note ss Cass. 3e civ., 24 oct. 2007, \no 06-19.260 : JurisData \no 2007-041007 ; JCP N 2007, 1328. – C. Atias, Jouissance réelle ne vaut pas propriété virtuelle : D. 2007, p. 2358. ; S. Lelièvre et S. Chaix-Bryan, Le droit de jouissance exclusive d'une partie commune : la fin d'un questionnement ? : Defrénois 2007, art. 38637, p. 1173.G. Gil, Les « lots » constitués par la jouissance exclusive d'une partie commune : le point de la situation :	Administrer juin 2013, p. 7 et s. – V. aussi J.-R. Bouyeure, Réflexion sur les conséquences de la nullité des lots dont la partie privative est constituée par un droit de jouissance exclusive sur une partie commune : Administrer oct. 2007, p. 70. – Pour une recension des diverses solutions adoptées, par la jurisprudence, V. V. Matet, La pratique notariale du droit de jouissance exclusif sur les parties communes :	l'histoire d'un inventeur dépassé par sa créature : Bull. Cridon Bordeaux-Toulouse, févr. 2010, 610.B. Kan-Balivet, La nature juridique du droit de jouissance exclusive sur les parties communes : Defrénois 2008, art. 38825, p. 1765. – R. Boffa, La nature juridique du droit de	jouissance exclusive sur les parties communes : LPA 10 nov. 2010, p. 3.}.
	
	\subsection{La consecration dans la loi \no 65-557 du 10 juillet 1965 (article 6-3 et 6-4)}
	
		\subsubsection{La loi \no 2018-1021 du 23 novembre 2018 dite ELAN}
		
			La loi ELAN du 23 novembre 2018 consacre le droit de jouissance privatif sur parties communes, mais
			l’encadre dans les nouveaux articles 6-3 et 6-4 de la Loi 65-557 du 10 juillet 1965 :
			\begin{quote}
				Art. 6‑3. – Les parties communes à jouissance privative sont les parties communes affectées
				à l’usage et à l’utilité exclusifs d’un lot. Elles appartiennent indivisément à tous les
				copropriétaires. Le droit de jouissance privative est nécessairement accessoire au lot de
				copropriété auquel il est attaché. Il ne peut en aucun cas constituer la partie privative d’un
				lot.
			
				Art. 6‑4. – L’existence des parties communes spéciales et de celles à jouissance privative
				est subordonnée à leur mention expresse dans le règlement de copropriété.
			\end{quote}
			
			L’article 6-3 est une consécration de la jurisprudence antérieure, mais a le mérite de stabiliser une situation
			juridique devenue incertaine avec les arrêts « Maison de la Poésie ».
			En revanche, l’article 6-4, en imposant que le droit de jouissance privatif figure non seulement dans la
			description du lot (donc dans l’état descriptif de division) mais également dans le règlement de copropriété
			risque de poser problème, si les modalités de cette intégration ne sont pas précisées.
			En effet, il est prévu dans les dispositions transitoires de la loi ELAN
			II (nouveau). – Les syndicats des copropriétaires disposent d’un délai de trois ans à compter de la
			promulgation de la présente loi pour mettre, le cas échéant, leur règlement de copropriété en conformité
			avec les dispositions de l’article 6-4 de la loi \no 65-557 du 10 juillet 1965 fixant le statut de la copropriété
			des immeubles bâtis.
			
			À cette fin, le syndic inscrit à l’ordre du jour de chaque assemblée générale des copropriétaires la question
			de la mise en conformité du règlement de copropriété. La décision de mise en conformité du règlement de
			copropriété est prise à la majorité des voix exprimées des copropriétaires présents ou représentés.
			Mais que se passera-t-il si le syndicat des copropriétaires refuse cette intégration ? … La décision de
			l’assemblée générale serait-elle valable, mais à charge dans ce cas pour le syndicat des copropriétaires
			d’indemniser le titulaire du droit de jouissance ? Le copropriétaire pourrait demander son annulation pour
			atteinte aux modalités de jouissance de son lot, mais rien n’est prévu pour permettre au magistrat de se
			substituer en ce cas à l’assemblée générale $\dots$
			
			Enfin, les titulaires de droit de jouissance devront être particulièrement attentifs à la régularisation de
			cette situation, sous peine de voir leur droit frappé, au 24 novembre 2021, de caducité.
		
		\subsubsection{L’ordonnance du 30 octobre 2019}
		
			L’Ordonnance du 30 octobre 2019 ajoute un alinéa au nouvel article 6–3 aux termes duquel : « le règlement
			de copropriété précise le cas échéant les charges que le titulaire de ce droit de jouissance privative
			supporte».
			
			Rappelons que par arrêt du 27 mars 2008 , la Cour de cassation a approuvé chaleureusement la cour
			d’appel d’avoir énoncé « exactement » qu’un « droit de jouissance exclusive d’un copropriétaire pouvait
			être affecté d’une quote-part de parties communes correspondant aux charges que son titulaire devait
			supporter sans pour autant être assimilé à un droit de propriété ».
			
			En pratique, lorsque le géomètre définit le lot en y mentionnant la jouissance privative (par exemple un
			appartement de 5 pièces avec jouissance exclusive d’une terrasse) il affecte le lot d’un nombre de
			tantièmes de propriété (et de charges communes) correspondant à ce droit de jouissance privative en
			sorte qu’il n’y a pas lieu à ajouter dans le règlement de copropriété des charges complémentaires à
			supporter par le bénéficiaire de cette jouissance privative.
			
			Par contre, cette nouvelle disposition présente un intérêt lorsqu’un copropriétaire se voit céder une
			jouissance privative par l’assemblée générale : le bout de jardin devant son appartement par exemple.
			Dans cette hypothèse il n’y a pas lieu à création d’un nouveau lot e (qui serait d’ailleurs un lot de
			jouissance) mais l’assemblée générale peut parfaitement prévoir que ce droit de jouissance est assorti de
			X tantièmes des charges communes générales. Au besoin l’assemblée générale pourra adopter un état de
			répartition des charges modificatif majorant la quote-part de charges du lot du copropriétaire attributaire
			de ce droit de jouissance qui, rappelons-le, est et demeure nécessairement l’accessoire d’un lot de
			copropriété.
			
			Cette modification de la répartition des charges sera votée à la même majorité que la décision de cession
			de la jouissance privative, par application de l’article 11 de la loi du 10 juillet 1965 : lorsque des (…) actes
			d'acquisition ou de disposition sont décidés par l'assemblée générale statuant à la majorité exigée par la
			loi, la modification de la répartition des charges ainsi rendue nécessaire peut être décidée par l'assemblée
			générale statuant à la même majorité.
			
			Par ailleurs, cet article peut également se lire comme consacrant la faculté de prévoir, dans le règlement
			de copropriété, que les charges d’entretien de la partie superficielle ( de la toiture, du jardin) incomberont
			au bénéficiaire du lot, comme l’a admis la jurisprudence.

\section{Les parties mitoyennes}

	L'article 7 de la loi dispose :
	\begin{quote}
		Les cloisons ou murs, séparant des parties privatives et non compris dans le gros œuvre, sont présumés mitoyens entre les locaux qu'ils séparent.
	\end{quote}
	
	Par conséquent il existe, à côté des parties communes et des parties privatives, des parties mitoyennes.
	Bien que l'article 43 de la loi incorpore les dispositions de l'article 7 de la loi dans les textes d'Ordre Public,
	la rédaction de ce dernier article autorise l'auteur du Règlement de Copropriété à classer ces cloisons ou
	murs dans les parties communes : la présomption disparaît en présence de la preuve contraire.
	Pour que cette présomption de mitoyenneté s'applique, encore faut-il qu’il s’agisse de cloison ou de mur
	non porteur et non de gros œuvre.
	
	Le texte ne vise que les cloisons et murs non porteurs séparant des parties privatives entre elles. Par contre
	il ne vise pas les cloisons et murs non porteurs situés entre une partie privative et une partie commune. Il
	n’y a pas à ce jour de jurisprudence qui donne la qualification à retenir pour ce dernier type de séparation.
	Dans le silence du Règlement de Copropriété ces cloisons et murs non porteurs séparant un lot des parties
	communes pourront être qualifiés selon le cas de parties communes ou de parties privatives ; la porte
	palière étant pour sa part une partie privative, puisqu’à l’usage exclusif du copropriétaire.

\section{Les servitudes}

	\subsection{Servitude de vue}

		Un arrêt de cassation\footnote{Civ. 3\degre{} 2 décembre 1980, JCP 1981 Ed N. II. 266, note Stemmer} a considéré qu'il ne pouvait y avoir de servitude de vue à l'intérieur d'une
		Copropriété, fut elle horizontale.\footnote{Cf également : Civ 3\degre{} 18 déc 1991 JCP N 92 IV \no 247}
		
		Civ 3ème Chambre 19 juillet 1995 : \\
		Cassation d’un arrêt de la Cour d’Appel ayant retenu l’existence d’une servitude de vue alors que
		dans le régime de la copropriété des immeubles bâtis, les lots ne sont séparés par aucune ligne
		divisoire et que la totalité du sol est partie commune, et que les juges du fond ne relèvent pas si la
		vue directe s'exerçait sur un fonds distinct et indépendant.
	
	\subsection{Servitude de passage}
		
		Il peut paraître utile d'imposer à un lot privatif une servitude de passage soit au profit d'un autre lot, soit
		au profit des parties communes. Si l'on oblige le propriétaire à laisser le droit de passage, cette obligation
		disparaîtra avec le propriétaire obligé, par exemple lorsqu'il vendra son lot. Si par contre on créée une
		servitude de passage, celle-ci se perpétuera quel que soit le propriétaire du lot. La finalité dans l'intérêt de
		la Copropriété est de perpétuer cette obligation de passage au détriment d'un lot et non pas d'une
		personne.
		
		Aussi nombre de Règlements de Copropriétés comportent ce type de clause ou encore on voit des
		Assemblées Générales qui cèdent à un copropriétaire les combles d'un immeuble avec une servitude de
		passage pour l'entretien de la toiture.
		
		La question s'est posée assez rapidement de la validité de ces servitudes à l'intérieur d'une même
		copropriété : en effet, qui dit copropriété dit essentiellement existence d'un « héritage » unique puisque
		nous n'avons qu'une seule propriété partagée entre plusieurs lots, chaque lot comportant une partie
		privative et une quote-part de parties communes.
		
		\subsubsection{La jurisprudence antérieure au 30 juin 2004.}
		
			La réponse donnée a été simple, sinon simpliste : une servitude ne pouvant exister qu'entre deux fonds
			distincts, il ne peut en être créée entre parties communes et lots privatifs. Pas plus qu'il ne peut être créé
			de telles servitudes sur une partie privative au profit d'une autre partie privative, car les parties communes
			comprises dans le lot sont indivises.
			
			Ce principe appliqué en matière de copropriété a été affirmé de par la Cour de Cassation à partir de 1984,
			malgré de vives critiques doctrinales154. Cette jurisprudence est restée constante pendant près de 20 ans\footnote{Civ 3ème 10 janvier 1984 D. 85, 335 puis Civ 3\degre{} 11 jan 1989 RDI 89 p 243, Civ 3\degre{} 6 mars 1991; 51 Civ 3\degre{} 18 juin 1997 Somm. JCP N 1997 \no 47 p. 1430 (passage d’un lot sur un autre).}.
			
			\begin{quote}
				(Il y a) incompatibilité entre la division d'un immeuble en lots de copropriété et la création d'une servitude			sur une partie commune au profit d'un lot.
			\end{quote}
			
			Les opposants à la jurisprudence\footnote{J.-L. Aubert, Rev. Administrer, mai 1993, p. 11, et notes sous Cass. 3e civ., 10 janv. 1984, D. 1985.335, et 30 juin 1992,		D.1993.156 ; P. Capoulade et C. Giverdon, RD imm. 1991.247, 374 ; ibid. 1992.535 ; C. Giverdon, Rev. Huissiers 1993.1333		; E.-J. Guillot, Rev. Administrer, juin 1993, p. 28 ; J. Lafond, JCP éd. N 1990. Prat. 1503, spéc. p. 451 ; Ann. Loyers 1986.115,		obs. R. Martin ; H. Souleau, note sous Cass. 3e civ., 6 mars 1991, D.1991.355 ; Defrénois1992.1140 ; F. Zenati, RTD civ.		1990.} précitée ont fait valoir avec le Professeur Aubert que la règle de l'article	667 implique qu'il y a dualité de propriétaires, en sorte que s'il ne peut y avoir servitude lorsqu'il y a un seul propriétaire (elle serait d'ailleurs inutile), par contre la servitude peut exister lorsque les deux fonds	n'appartiennent pas << entièrement et exclusivement >> au même propriétaire. Or, en Copropriété il y a	nécessairement pluralité de propriétaires puisque si chaque copropriétaire est propriétaire indivis des
			parties communes, il est par contre seul propriétaire de ses parties privatives.
		
		\subsubsection{L’arrêt du 30 juin 2004 de la 3ème chambre civile de la cour de cassation.}
		
			Cette décision\footnote{Juris-Data \no 2004-024376 ; H. Périnet Marquet, Droit des biens : JCP G 2004, \no 43, I, 171, spéc. p. 1905 ; Constr.-urb. 2004,	comm. 161, note D. Sizaire ; Loyers et copr. 2004, comm. 196, note G. Vigneron ; Defrénois 2005, \no 10, p. 861, note G. Daublon et	B. Gelot} constitue un véritable revirement de jurisprudence puisqu’elle admet la constitution d’une servitude de passage entre deux lots :
			\begin{quote}
				Mais attendu que le titulaire d'un lot de copropriété disposant d'une propriété exclusive sur
				la partie privative de son lot et d'une propriété indivise sur la quote part de partie commune
				attachée à ce lot, la division d'un immeuble en lots de copropriété n'est pas incompatible
				avec l'établissement de servitudes entre les parties privatives de deux lots, ces héritages
				appartenant à des propriétaires distincts.
			\end{quote}
			
			On notera la précision des termes de la cour de cassation : celle-ci ne valide la servitude qu’à la condition
			qu’elle soit établie entre deux lots privatifs. Effectivement les parties communes ne constituent pas des
			« héritages » distincts. La cour de cassation a confirmé son revirement ultérieurement, à propos d’une
			servitude de passage de canalisations\footnote{Cass. 3e civ., 13 sept. 2005, pourvoi \no 04-15.742}.
			
			Des copropriétaires qui ne peuvent accéder à leurs lots que par une cour privative attachée au lot
			appartenant à un autre copropriétaire sont enclavés et peuvent donc réclamer un passage suffisant pour
			assurer la desserte de leur fonds alors que le règlement de copropriété est muet . Il y a une servitude
			lorsque celle-ci est nécessaire à l’exploitation normale du lot\footnote{Cass. Civ. 3e 19 janvier 2010 pourvoi: 09-12522}.
			\begin{quote}
				Attendu qu'ayant constaté que, pour accéder au lot privatif \no 17 leur appartenant, les époux X...
				n'avaient d'autre possibilité que de passer par la cour \no 2 qui était une partie privative attachée
				au lot \no 12 appartenant à M. Y..., la cour d'appel, qui a souverainement relevé que le lot \no 17
				était enclavé, en a déduit exactement que les époux X... étaient fondés à réclamer sur la cour n'appartenant à leur voisin M. Y... un passage suffisant pour assurer la desserte de leur fonds
			\end{quote}


	\subsection{Le nouvel article 6-1 a de la loi du 10 juillet 1965}
	
		L’Ordonnance du 30 octobre 2019 a créé un article 6-1-A (de la loi \no 65-557 du 10 juillet 1965), aux termes
		duquel :
		\begin{quote}
			« Aucune servitude ne peut être établie sur une partie commune au profit d’un lot ».
		\end{quote}
		Ce faisant, le gouvernement a voulu inscrire dans le marbre de la loi une jurisprudence parfaitement
		établie.
		
		Ce principe appliqué en matière de copropriété a été affirmé par la Cour de Cassation à partir de 1984\footnote{Civ 3ème 10 janvier 1984 D. 85, 335 puis Civ 3\degre{} 11 jan 1989 RDI 89 p 243, Civ 3\degre{} 6 mars 1991} :
		\begin{quote}
			(Il y a) incompatibilité entre la division d'un immeuble en lots de copropriété et la création d'une servitude
			sur une partie commune au profit d'un lot\footnote{Certes, l’article L 615-10 du CCH qui prévoit – à titre expérimental et pour dix ans – l’expropriation des parties communes au profit d’un opérateur désigné par la commune ou l’EPCI, édicte que : « Au  moment de l'établissement du contrat de concession ou de la prise de possession par l'opérateur, l'état		descriptif de division de l'immeuble est mis à jour ou établi s'il n'existe pas. Aux biens privatifs mentionnés dans l'état de division est attachée une servitude des biens d'intérêt collectif. Les propriétaires de ces biens privatifs sont tenus de respecter un règlement d'usage établi par l'opérateur ». Mais cette expropriation des parties communes fait disparaître la copropriété !}.
		\end{quote}
		
		Il est toutefois regrettable que le gouvernement ne soit pas allé jusqu’à reproduire intégralement la
		jurisprudence relative aux servitudes à l’intérieur d’un immeuble en copropriété, et n’ait pas précisé que
		si aucune servitude ne peut être établie sur une partie commune au profit d’un lot, par contre il peut être
		créé des servitudes entre les parties privatives de deux lots, conformément à la jurisprudence issue de
		l’arrêt du 30 juin 2004 de la 3ème chambre civile de la cour de cassation\footnote{Juris-Data \no 2004-024376 ; H. Périnet Marquet, Droit des biens : JCP G 2004, \no 43, I, 171, spéc. p. 1905 ; Constr.-urb. 2004, comm. 161, note D. Sizaire ; Loyers et copr. 2004, comm. 196, note G. Vigneron ; Defrénois 2005, \no 10, p. 861, note G. Daublon et B. Gelot,} :
		\begin{quote}
			« Mais attendu que le titulaire d'un lot de copropriété disposant d'une propriété exclusive sur la partie
			privative de son lot et d'une propriété indivise sur la quote-part de partie commune attachée à ce lot, la
			division d'un immeuble en lots de copropriété n'est pas incompatible avec l'établissement de servitudes
			entre les parties privatives de deux lots, ces héritages appartenant à des propriétaires distincts ».
		\end{quote}
		
		Cette rédaction ne remettrait d’ailleurs pas en cause la règle selon laquelle il ne peut y avoir de servitude
		de vue au sein d’une copropriété horizontale puisque ces servitudes ne sont pas établies entre les parties
		privatives de deux lots mais sont établies sur le fonds voisin (article 673 Code civil) comme le rappelait la
		Cour de cassation dans un arrêt du 19 juillet 1995\footnote{civile 3e chambre, 19 juillet 1995–Administrer novembre 1996, page 57, commentaire \nom{CAPOULADE}}. Or il n’y a pas de fonds voisin puisque le terrain est
		bien celui d’une même copropriété.

	
\section{Le jeu de la prescription acquisitive}

	Une fraction d’immeuble peut-elle faire l’objet d’une prescription acquisitive, et être ainsi « requalifiée »
	de partie commune à privative (ou inversement) ?
	
	La question concerne en réalité l’affectation privative de parties communes ou à l’inverse l’affectation
	commune d’une partie privative.
	
	La Cour de cassation donne une réponse positive dans les deux cas, tout en se montrant très exigeante sur
	la preuve de la possession trentenaire :
	\begin{itemize}
		\item - usucapion trentenaire d’une partie commune par un	copropriétaire\footnote{Civ. 3 e , 26 mai 1993, \no 91-11.185, RDI 1993. 411, obs. P. Capoulade et C. Giverdon} ou prescription abrégée par juste titre\footnote{Civ. 3e , 30 avr. 2003, \no 01-15.078, D. 2003. 2047, obs. B. Mallet-Bricout} ;
		\item  usucapion trentenaire d’une partie privative par le syndicat des	copropriétaires\footnote{Civ. 3e , 8 oct. 2015, FS-P+B, \no 14-16.071; Dalloz Actualité, note Le Rudelier}
	\end{itemize}.
	
	Concrètement, comment faire reconnaître un droit de propriété privative sur une partie commune par
	usucapion ?
	
	Pour pouvoir publier son titre, il faut justifier au Service de publicité Foncière d’un « effet relatif » sur
	l’origine du lot. Aussi en pratique bien souvent le « bénéficiaire » demande à l’assemblée générale
	d’approuver la création d’un lot nouveau à partir des parties communes correspondant à la fraction
	accaparée depuis plus de 30 ans et la cession de ce nouveau lot à titre gratuit ou pour 1 \euro symbolique.
	
	Cette procédure présente deux inconvénients :
	\begin{itemize}
		\item d’une part elle ne correspond pas à la réalité puisqu’il n’y a pas de cession ;
		\item d’autre part elle renvoie aux dispositions de l’article 26 a) de la loi et à la double majorité.
	\end{itemize}
	
	Pour sa part Jacques Lafond au Jurisclasseur Copropriété F 91-40 mutations concernant les parties
	communes propose une solution différente.
	\begin{enumerate}
		\item  Le « bénéficiaire » fait établir un acte de notoriété par notaire. Certes un tel acte ne vaut pas preuve
		de la propriété, mais il a pour mérite, outre la participation de témoins, de viser et d’annexer
		éventuellement tous les éléments de preuve de la prescription acquisitive.
		\item Le « bénéficiaire » fait établir un modificatif à l’état descriptif de division avec création d’un nouveau
		lot privatif.
		\item Le « bénéficiaire » demande l’inscription à l’ordre du jour de la prochaine assemblée générale de
		résolutions aux fins :
		\begin{itemize}
			\item de constater le jeu de la prescription au profit du copropriétaire concerné ;
			\item  d'approuver le projet de modification du règlement de copropriété et de l'état descriptif de division.
		\end{itemize}
	\end{enumerate}
	
	S’agissant de la constatation d’un droit et non de la cession d’une partie commune, faute de majorité
	spécifique dans la loi \no 65-557 du 10 juillet 1965 la question sera posée à la majorité de l’article 24. En cas
	de refus de l’assemblée générale le copropriétaire pourra demander au juge de reconnaître son droit et
	pour modifier l’état descriptif (en application de l’article 3 du Décret de 67).
	
	Il faut noter que la prescription acquisitive peut jouer dans le sens inverse. En effet, la Cour de cassation a
	récemment indiqué (Cour de cassation 3e chambre civile 8 Octobre 2015) « qu'aucune disposition ne
	s'oppose à ce qu'un syndicat de copropriétaires acquière par prescription la propriété d'un lot », alors
	même que le pourvoi avait soutenu qu’il ne pouvait être de l’objet du syndicat des copropriétaires de
	porter atteinte aux droits privatives d’un copropriétaire. Pour autant, il risque d’être difficile de démontrer
	la possession non équivoque du syndicat des copropriétaires : quel sera l’acte de possession matérielle
	indiquant que le syndicat des copropriétaires entend utiliser le lot privatif de telle façon qu’il doit être
	considéré comme à l’utilité de tous ?
		\chapter{La cession du lot}

Comme tout bien susceptible d'appropriation privée, le lot peut faire l'objet d'actes de disposition et
notamment de cession : vente, échange, donation.

Ce droit est d'ailleurs consacré par l'article 9 de la loi du 10 juillet 1965 qui dispose :
\begin{quote}
	Chaque copropriétaire dispose des parties privatives comprises dans son lot.
\end{quote}

Ce texte comporte une légère inexactitude car la cession ne porte pas seulement sur les parties privatives
mais aussi sur la quote-part de parties communes y afférentes, c'est-à-dire sur la totalité du lot.

Le droit de céder s'exerce en principe librement. Cependant, des restrictions particulières au statut de la
copropriété viennent le limiter, si bien que la liberté de vendre un lot n'est pas aussi étendue que celle
d'aliéner tout autre bien immobilier. Par ailleurs, la cession du lot comporte des formalités spécifiques.

Enfin, les effets d'une telle cession suscitent des difficultés particulières qu'il conviendra d'examiner.

\section{La liberté de céder son lot}

	Si la liberté de céder son lot constitue le principe, des restrictions de plus en plus nombreuses sont venues
	la limiter, faisant ainsi apparaître la différence irréductible existant entre la situation d'un copropriétaire
	et celle d'un propriétaire.
	
	\subsection{Liberté de cession et restrictions issues du règlement de copropriété}
	
		Pendant longtemps, le caractère absolu de ce principe a été affirmé par la jurisprudence.
		
		Le fondement textuel d'une telle position était le suivant : l'article 9 al. 1\ier{} de la loi, dans une première
		proposition reconnaît au copropriétaire le droit de disposer sans restriction du lot alors que la deuxième
		phrase de ce texte soumet le droit d'en user ou d'en jouir à la double condition de ne porter atteinte ni
		aux droits des autres copropriétaires, ni à la destination de l'immeuble.
		
		La troisième chambre civile de la Cour de cassation a sur ce fondement décidé dans un arrêt très remarqué
		en date du 17 juillet 1972 que « la notion de destination de l'immeuble ne concerne que l'usage et la
		jouissance de l'immeuble ; elle ne saurait justifier une restriction au droit de disposer librement des
		lots\footnote{D.1972, 727, note E.F.; J.C.P. 1972, II, 17241, note E.J. GUILLOT, Rep. DEFRENOIS 1973 art.30293 \no 11, obs. H.	SOULEAU}. »
		
		Il résultait de cette jurisprudence que toute clause ou décision limitant le droit de disposer était par nature
		illicite. Mais, cette conception exagérément littérale (art.9 de la loi), et absolue s'est révélée artificielle et
		quelque fois nuisible dans les hypothèses où la limitation au droit de disposer se justifiait par la sauvegarde
		de la disposition de l'immeuble.
		
		On retrouve ici la notion de destination de l’immeuble : une atteinte au droit de libre cession peut être
		justifiée, dès lors que la destination de l’immeuble le justifie.
		
		C'est pourquoi, après avoir affirmé le 17 juillet 1972\footnote{Cf. \emph{supra}} que cette destination était une notion étrangère à
		la disposition des lots, la Cour de cassation a depuis un arrêt en date du 10 mars 1981 admis la validité
		d'une clause interdisant de vendre des chambres de service séparément des appartements dès lors qu'une
		telle interdiction se trouvait justifiée par la destination de l'immeuble.
		
		En l’espèce il s'agissait d'une copropriété de taille réduite, la destination de cet immeuble ne pouvait pas
		permettre une subdivision indéfinie des parties privatives; l'absence de lots accessoires tels que chambres
		de service, garages ou débarras entraînerait un encombrement des parties communes et gênerait
		l'utilisation normale de l'immeuble\footnote{Civ. 3ème 10 mars 1981, Bull. Civ. III \no52, Rep. DEFRENOIS 1981, art.32797, obs. H.SOULEAU}
		
		Il convient donc de distinguer entre les clauses licites au droit de disposer et les clauses illicites.
		
		\subsubsection{Restrictions licites au droit de disposer}
		
			Aujourd'hui sont donc reconnues valables les clauses du règlement de copropriété restreignant la liberté
			de disposer lorsqu'elles se justifient par la sauvegarde de la destination de l'immeuble.
			
			Si cette condition est remplie, sont donc considérées comme licites :
			\begin{description}
				\item[les clauses d'inaliénabilité temporaire] dans un immeuble comportant deux
				appartements construits pour couples amis, la clause interdisant la cession pendant un
				délai raisonnable pour permettre aux uns de tenter d'acquérir la part des autres est
				valable\footnote{exemple donné par CH. ATIAS, La Copropriété des Immeubles Bâtis, Sirey 1989,				\no63)} ;
				
				\item[les clauses interdisant la division des lots] dans les immeubles où la multiplication des
				unités familiales altérerait le caractère de l'habitat et serait incompatible avec les
				équipements et installations existants qui ont été conçus pour un nombre déterminé
				d'utilisateurs ;
				
				\item[les clauses de préemption ou pactes de préférence] conférant au copropriétaire originaire
				un droit de priorité pour l'acquisition d'un lot cédé par un membre du syndicat, ce n'est
				qu'au cas de refus des titulaires de ce droit que le vendeur recouvre la liberté de vendre à
				qui il veut ;
				
				\item[les clauses d'exclusivité] portant sur des locaux accessoires tels que caves, greniers,
				chambres de service ou garages interdisant de vendre séparément ces biens à d'autres
				personnes que des copropriétaires ;
				
				\item[les clauses d'indivisibilité] interdisant de vendre des locaux accessoires séparément de
				l'appartement principal\footnote{Cf. Civ. 3\ieme{}, 10 mars 1981 précité}.
			\end{description}
			
			Bien entendu, si ces clauses restrictives ne se justifient pas par la nécessité de la
			sauvegarde de la destination de l'immeuble, elles doivent être annulées.
			
		\subsubsection{Restrictions illicites au droit de disposer}
		
			En revanche, il existe des clauses qui par nature sont illicites et que ne peut valider le
			recours à la destination de l'immeuble : ce sont celles qui auraient pour résultat de rendre
			le copropriétaire prisonnier de son lot, c'est-à-dire mis dans l'impossibilité de vendre. Il
			s'agit :
			
			\begin{description}
				\item[des clauses d'inaliénabilité perpétuelle] ;
	
				\item[des clauses de préemption ou de préférence] portant sur les locaux principaux --- si
				aucun des copropriétaires ne désire acheter, le cédant est totalement privé du
				droit de disposer\footnote{CA Toulouse, 10 jan 2011 (Loyers et Copropriété Juin 2011 \no 187) ; Civ. 3\ieme{} 29 mai 1979 ; JCPN 1979, II, p. 237, note Lafond)} ;
				
				\item[des clauses d'agrément] qui imposent au copropriétaire cédant l'assentiment du
				syndicat sur la personne de l'acquéreur --- si l'assemblée refuse systématiquement
				tous les candidats qui lui sont présentés par le cédant, celui-ci se trouve dans
				l'impossibilité de vendre ;
				
				\item[de la clause faisant obligation au copropriétaire d’un lot de le louer] au syndicat
				pour être affecté au logement du gardien de l’immeuble\footnote{Paris 23\ieme{} Ch. 19 fév. 1997 Loyers et Copropriété 1997 \no 186}.
			\end{description}
	
	\subsection[Les droits de préemption]{Les droits de préemption : atteinte a la liberté de céder par substitution d’acquéreur}
		
		\subsubsection{Le droit de préemption au profit du locataire}\footnote{Sur l’ensemble des droits de préemption urbains et au profit des locataires, consulter la remarquable étude de M \nom{Casteran} et Mme \nom{Lambret-Borderie} au JCP, Ed N. 2012 – Études \no 1237}
		
			\paragraph{Droit de préemption du locataire ou occupant de bonne foi en cas de division de l’immeuble en vue de la vente par lots (Loi \no 1351 du 31 décembre 1975 modifiée en dernier lieu par la loi \no 526 du 22 juin 1982).}
			
				Ce texte a pour origine les abus de marchands de biens vis-à-vis des locataires de la loi de 48. Il s’applique
				lorsque le propriétaire unique d’un immeuble établit un État descriptif de division par lots et met en vente
				ces lots.
				
				\begin{quote}
					« Préalablement à la conclusion de toute vente d'un ou plusieurs locaux à usage d'habitation ou à usage
					mixte d'habitation et professionnel, consécutive à la division initiale ou à la subdivision de tout ou partie d'un immeuble par lots, le bailleur doit, à peine de nullité de la vente, faire connaître par lettre
					recommandée avec demande d'avis de réception, à chacun des locataires ou occupants de bonne foi, l'indication du prix et des conditions de la vente projetée pour le local qu'il occupe. Cette notification vaut offre de vente au profit de son destinataire ».
				\end{quote}
				
				\subparagraph{Condition de mise en œuvre du droit de préemption}
				
				Pour que ce texte s’applique trois conditions doivent être réunies :
				\begin{itemize}
					\item il doit y avoir vente\footnote{Il n’y a pas vente en cas de partage (même avec soulte), de cession de parts ou d’échange, ni a fortiori si la cession est faite à titre gratuit.} portant sur des locaux à usage d'habitation\footnote{Le droit existe en cas de vente d’un local accessoire à usage d’habitation (chambre de service) et même pour un garage accessoire d’un appartement en location.} ou à usage mixte d'habitation et professionnelle ;
					\item la vente doit être consécutive à la division\footnote{La publication de l'État descriptif de division préalablement à la vente n’est pas obligatoire : hypothèse de la promesse synallagmatique de vente avant publication.} de l’immeuble par lots de copropriété\footnote{Ce qui implique que l’immeuble soit collectif.} ;
					\item il doit s’agir de la première vente après division de l’immeuble\footnote{La Cour de Cassation considère que s’il existe deux bâtiments et que l’un de ces bâtiments est vendu « en bloc », le droit de préemption ne bénéficie pas aux locataires de ce bâtiment. 5 juillet 1995 (Bull. civ. 1995, III, \no 173 ; JCP G 1995, IV, 2199).}.
				\end{itemize}
				
				Ce droit de préemption n’est pas limité à la première vente d’un lot de l’immeuble divisé mais s’applique
				à la première vente de chacun des lots issus de la division de l’immeuble, en sorte que le droit de
				préemption peut s’exercer dans l’immeuble sur une longue période de temps. Il s’applique sur la vente
				d’un ou plusieurs lots (vente en bloc), mais le prix doit alors être ventilé par lot.
				
				\subparagraph{Bénéficiaire du droit de préemption}
				
				Le bénéficiaire est le locataire ou l’occupant de bonne foi (titulaire d’un bail « loi de 48 » bénéficiant du
				droit au maintien dans les lieux). Le bénéficiaire doit occuper effectivement les lieux.
				
				En cas de pluralité de titulaires chacun d’eux (époux par exemple) bénéficie de la procédure de
				préemption.
				
				\subparagraph{Procédure de préemption}
				
				La notification doit être faite par lettre recommandée avec demande d'avis de réception préalablement à
				la conclusion de la vente. Par cette notification le vendeur offre au locataire ou occupant de bonne foi la
				faculté d’acquérir le local qu’il occupe.
				
				L’offre précise le prix et les conditions de la vente et, à peine de nullité, reproduit les différents textes
				l'article 10, \I{} de la loi du 31 décembre 1975 qui développe le droit de préemption.
				
				Le locataire dispose d’un délai de deux mois pour accepter l’offre d’acquérir. La vente devra être réalisée
				à son profit dans le délai de deux mois de son acceptation (quatre mois si le locataire déclare recourir à un
				prêt).
				
				\subparagraph{Sanction en cas de non respect du droit de préemption}
				
				En cas de non-respect de la procédure, comme en cas de notification irrégulière, la nullité de la vente est
				encourue\footnote{Cass. 3e civ., 11 juin 1997 : Loyers et copr. 1997, comm. 251, obs. B. Vial-Pédroletti}. Elle doit être demandée par le locataire dans le délai de cinq ans de l’article 1304 du Code	civil.
				
				Par contre le locataire ne bénéficie pas d’un droit de substitution.
				
				Toutefois si la vente se fait pour un prix plus avantageux que le prix offert, il n’y a pas lieu à annulation de la vente mais en ce cas, le locataire bénéficiera d’un nouveau droit de préemption.
			
			\paragraph{Le droit de préemption du locataire en cas de congé pour	vendre (loi du 6 juillet 1989, article 15 \II)}
			
				\par Ce droit de préemption bénéficie au locataire « habitation » qui se voit notifié en fin de bail un congé pour vendre. Ce droit a été modifié en dernier lieu par la loi ALUR.
				
				Le bailleur n’a aucune obligation de donner congé pour vendre : il peut parfaitement vendre directement
				son bien occupé. Auquel cas le bail se poursuivra avec l’acquéreur $\dots$ qui pourra par exemple attendre
				l’issue du bail pour donner congé pour habiter lui-même, donc sans droit de préemption pour le locataire.
				
				Cependant, s’il étend vendre libre, le lot doit être offert en priorité au locataire qui occupe les lieux .
				
				\subparagraph{Conditions de mise en œuvre}
				
				\begin{itemize}
					\item Existence d’un bail soumis à la loi du 6 juillet 1989 (locaux d’habitation principale ou à usage mixte professionnel et d'habitation).
					
					Le bail doit être en cours, qu’il s’agisse d’un bail reconduit ou renouvelé. Si le bail a été résilié ou
					annulé, le locataire --– devenu simple occupant --– ne peut prétendre au bénéfice du droit de
					préemption ; il en ira de même si le locataire a préalablement donné congé.
					
					\item  La cession échappe au droit de préemption du locataire en cas de cession intervenant entre parents
					jusqu'au troisième degré inclus, sous la condition que l'acquéreur occupe le logement pendant une
					durée qui ne peut être inférieure à deux ans à compter de l'expiration du délai de préavis.
					
					\item  Le droit de préemption du locataire est écarté en cas de préemption par la commune ou vente à un
					OPHLM.
				\end{itemize}
				
				\subparagraph{Procédure de préemption}
				
				Le congé est délivré par lettre recommandée avec AR ou acte d’huissier, ou depuis la loi ALUR par a remise
				en main propre contre récépissé ou émargement, six mois au moins avant le terme du terme du bail\footnote{Si ce délai n’est pas respecté, le congé est nul.}.
				
				Le congé doit être motivé, comporter l’offre de vente (prix et modalités de paiement, commission de l’agent
				immobilier, conditions de la vente avec diagnostics) et reproduire les dispositions de l’article 15 alinéa 1 à
				5 de la loi de 1989, c'est-à-dire les dispositions qui exposent le mécanisme de ce droit de préemption.

				Par contre le vendeur n’a pas l’obligation de joindre le Règlement de copropriété au congé si la copropriété
				préexiste. Le vendeur n’a pas davantage l’obligation de joindre le mesurage « Carrez » à ce stade de la
				vente.
				
				Le congé vaut offre de vente, laquelle est « valable pendant les deux premiers mois du délai de préavis” >>.
				Elle doit donc être maintenue pendant toute la durée de ce délai. Aucune rétractation n'est possible de la
				part du bailleur.
				
				Si le locataire accepte l’offre, la vente est parfaite il ne peut pas se rétracter et le bailleur ne peut pas
				davantage révoquer son offre\footnote{Par contre le locataire disposera du délai de sept jours de la remise de l’acte authentique pour se rétracter (\article{L}{271-1} du \CCH).}. Son droit d’occupation est prorogé jusqu’à la réalisation de la vente, qui
				doit intervenir dans un délai préfix de deux mois à compter de l’acceptation (délai porté à 4 mois si le
				locataire fait connaître au bailleur on intention de souscrire un prêt).
				
				\subparagraph{Sanction}
				
				Si le congé pour vendre ne respecte pas les dispositions légales, il est nul : par exemple si le congé n’est
				pas motivé ou si le bailleur n’a pas l’intention réel de vendre. Pour autant le bailleur n’a pas à justifier qu’il a d’ores et déjà trouvé un autre acquéreur.
				
				Si le congé a été donné frauduleusement (prix artificiellement gonflé, local conservé par le bailleur puis
				reloué à un loyer supérieur) ou sans respect préjudiciable des obligations légales, le locataire peut obtenir
				des Dommages Intérêts. En outre, la loi ALUR \no 2014-366 du 24 mars 2014 a instauré une sanction pénale.
				
				L'article 15-\IV{} nouveau de la loi \no 89-462 du 6 juillet 1989 prévoit que :
				\begin{quote}
					Le fait pour un bailleur de délivrer un congé justifié frauduleusement par sa décision de reprendre ou de
					vendre le logement est puni d'une amende pénale dont le montant ne peut être supérieur à \montant{6 000} pour une personne physique et à \montant{30 000} pour une personne morale.
				\end{quote}
				
				\subparagraph{Droit de préemption subsidiaire} Si le propriétaire décide de vendre à un tiers à des conditions ou à un
				prix plus avantageux, ce prix ou ces conditions doivent être notifiés au locataire soit par le bailleur, soit
				par le notaire (si le bailleur n'y a pas préalablement procédé). Cette notification est imposée à peine de
				nullité de la vente. Effectuée soit à l'adresse indiquée par le locataire au bailleur, soit à l'adresse des locaux
				dont la location avait été consentie, elle vaut offre de vente au profit du locataire et est valable pendant
				une durée d'un mois à compter de sa réception.
			
			\paragraph{Le droit de préemption du locataire ou occupant de bonne foi en cas de « vente à la	découpe » (loi \nom{Aurillac} du 13 juin 2006)}
			
			\par
			Cette loi trouve son origine dans les ventes d’importants patrimoines réalisées par les grands investisseurs
			(Établissements Financiers et Assureurs, notamment) d’immeubles entiers, sans division préalable de
			ceux-ci en lots de copropriété : il est plus rapide de vendre un immeuble entier plutôt que des lots de
			copropriété.
			
			Ces « ventes en bloc » dites encore « ventes à la découpe » ont été dénoncées médiatiquement :
			reproche étant fait aux acquéreurs (parfois fonds spéculatifs américains) de maltraiter les locataires en
			proposant l’acquisition des lots à des prix prohibitifs et les mêmes media ont réclamé une loi de protection
			des locataires. La loi a été modifiée par la loi ALUR
			181 

			\subparagraph{A) CONDITIONS DE MISE EN OEUVRE}.
			Cette loi crée au profit des locataires à usage d'habitation ou à usage mixte d'habitation et professionnel,
			un nouveau droit de préemption en cas de vente de l’immeuble « dans sa totalité et en une seule fois ».
			C’est l’hypothèse de la « vente en bloc »
			Ce droit de préemption ne peut jouer que si l’immeuble comporte plus de CINQ logements, en ce compris
			les logements non donnés à bail (seuil abaissé de 10 à 5 logements par la loi ALUR)
			Pour bénéficier du droit de préemption l’occupant doit justifier d’un bail en cours (d’habitation ou
			d’habitation et professionnel) à la date de signature de l’acte de vente ou être titulaire d’un droit au
			maintien dans les lieux (loi de 48).
			B) PROCEDURE DE PREEMPTION
			Ce droit de préemption sera proposé avant la vente de l’immeuble : le vendeur fait connaître par lettre
			recommandée A.R. à ses locataires le prix et les conditions de la vente de l’immeuble dans son ensemble
			et du logement en particulier. A cette offre le propriétaire joint le diagnostic technique (constat de l'état
			apparent de la solidité du clos et du couvert et de celui de l'état des conduites et canalisations collectives)
			et une copie du projet de Règlement de copropriété.
			Le locataire dispose d’un délai de quatre mois pour accepter cette offre d’acquérir et la vente se fera dans
			les deux mois qui suivent.
			Si le locataire ne préempte pas dans le délai de 4 mois le vendeur pourra passer la vente « en bloc ».
			Si un locataire préempte les autres lots seront alors vendus « en copropriété ». Si ces lots sont vendus
			moins cher qu’au prix proposé précédemment, le locataire pourra se substituer à l’acquéreur au prix de
			l’acte de vente (après que cette vente lui ait été notifié le locataire disposera d’un délai d’un mois
			seulement pour se substituer à l’acquéreur du lot).
			C) LA PROROGATION DES BAUX EN COURS PENDANT SIX ANS.
			Toutefois le vendeur peut échapper à cette obligation de mise en oeuvre du droit de préemption s’il
			s’oblige“à proroger les contrats de bail à usage d'habitation en cours à la date de la conclusion de la vente
			afin de permettre à chaque locataire ou occupant de bonne foi de disposer du logement qu'il occupe pour
			une durée de six ans à compter de la signature de l'acte authentique de vente qui contiendra la liste des
			locataires concernés par un engagement de prorogation de bail”. Cet engagement figurera dans l’acte de
			vente et sera transmis de plein droit à son acquéreur.
			En sorte que si la vente intervient alors que le bail du locataire doit s’achever dans les deux ans, ce locataire
			bénéficiera au total d’un bail de huit ans !
			Le bailleur a le plus grand intérêt à notifier par LAR son engagement de prorogation afin que son droit à
			délivrer congé après ces 6 années ne soit pas contesté par le locataire.
			D) SANCTION EN CAS DE NON RESPECT DU DROIT DE PREEMPTION OU DE PROROGATION
			DES BAUX.
			droit de la copropriété année 2018-2019
			166
			L’absence de notification dans l’un ou l’autre cas définis précédemment entraîne la nullité de la vente qui
			pourra être demandée par le locataire en place.
		
		\subsubsection{Le droit de préemption urbain}
		
			Le droit de préemption urbain permet à la commune d'acquérir prioritairement un bien foncier ou
			immobilier lorsque celui-ci est sur le point d'être vendu, afin de lui permettre de réaliser ses projets
			d’aménagement.
			
			Sont exclus du droit de préemption les successions, les donations portant sur des immeubles ou droits
			sociaux (SCI) entre parents jusqu’au 6ème degré ou entre personnes ayant des liens issus d’un mariage ou
			d’un pacs, le partage, les immeubles faisant l'objet d'un contrat de vente d'immeubles à construire, les
			donations.
			
			La commune ne peut en principe exercer son droit que sur les biens immobiliers dont la construction est
			achevée depuis au moins 4 ans (date de la DAACT) qui font l'objet d'une cession volontaire ou forcée à titre
			onéreux (vente, échange, apport en société $\dots$).
			
			L'\article{L}{211-4} du Code de l'urbanisme exclut en outre du champ d'application du droit de préemption
			urbain l’aliénation d'un ou plusieurs lots constitués soit par un seul local à usage d'habitation, à usage
			professionnel ou à usage professionnel et d'habitation, soit par un tel local et ses locaux accessoires, soit
			par un ou plusieurs locaux accessoires d'un tel local, compris dans un bâtiment soumis […$\dots$] au régime de
			la copropriété depuis plus de dix ans (date de publication du règlement de copropriété).
			
			En revanche, le DPU est applicable :
			\begin{itemize}
				\item en cas de vente d’un lot de copropriété « habitation » d’un immeuble achevé depuis plus de
				4 ans, et dont le règlement de copropriété a été publié il y a moins de 10 ans (donc à un
				immeuble neuf soumis au régime après \VEFA{} entre la 5\ieme{} et la 10\ieme{} année après dépôt du
				règlement de copropriété) ;
				\item  en cas de vente d’un lot affecté à une activité commerciale ;
				\item  en cas de délibération motivée de la Commune ou de l’EPCI de créer un droit de préemption
				dit « renforcé » ;
				\item  dans les ZAD, c'est-à-dire les Zones d’Aménagement Différé, secteur où une collectivité locale,
				un établissement public y ayant vocation ou une Société d'économie mixte (SEM) titulaire
				d'une convention d'aménagement dispose, pour une durée de 14 ans, d'un droit de
				préemption sur toutes les ventes et cessions à titre onéreux de biens immobiliers ou de droits
				sociaux.
			\end{itemize}
			
		\subsubsection{Le droit de préemption au profit des autres copropriétaires}
			
			Il s’agit ici d’un texte ajouté à la loi sur la Copropriété par et qui constitue L’ article 8-1 de la Loi \no 65-557 du 10 juillet 1965, introduite par la loi MOLLE du 25 mars 2009, ouvre la possibilité d’un droit de
			préemption au profit des autres copropriétaires sur les lots de stationnement.
			
			\begin{quote}
				« Le règlement de copropriété des immeubles dont le permis de construire a été délivré conformément à
				un plan local d'urbanisme ou d'autres documents d'urbanisme imposant la réalisation d'aires de
				stationnement peut prévoir une clause attribuant un droit de priorité aux copropriétaires à l'occasion de la
				vente de lots exclusivement à usage de stationnement au sein de la copropriété.
			
				Dans ce cas, le vendeur doit, préalablement à la conclusion de toute vente d'un ou plusieurs lots à usage
				de stationnement, faire connaître au syndic par lettre recommandée avec demande d'avis de réception son
				intention de vendre, en indiquant le prix et les conditions de la vente.
				
				Cette information est transmise sans délai à chaque copropriétaire par le syndic par lettre recommandée
				avec demande d'avis de réception, aux frais du vendeur. Elle vaut offre de vente pendant une durée de deux
				mois à compter de sa notification ».
			\end{quote}
		
			\paragraph{Origine de ce texte}
			
			\par Aux termes de l'article 8 de la loi du 10 juillet 1965 les clauses restrictives du règlement de copropriété ne
			sont admises que si elles sont justifiées par la destination de l'immeuble, alors que l'article 9 pose le
			principe selon lequel chaque copropriétaire dispose librement de son lot. En application de ces dispositions
			légales la jurisprudence est le plus souvent hostile aux clauses restrictives au droit de disposer librement
			de leur lot par les copropriétaires. Elle considère principalement que le droit de préemption donné aux
			copropriétaires est étranger à l'objet même du syndicat des copropriétaires.
			
			Or revendre un parking à un tiers étranger au syndicat de copropriété alors que le permis de construire a
			imposé la réalisation d'un certain nombre de places de stationnement constitue un véritable
			détournement des autorisations obtenues.
			
			\paragraph{Le mécanisme mis en place}
			
			\par Ce mécanisme se ne s'applique que si le copropriétaire vend séparément son parking des autres lots qu'il
			peut posséder. Il devra adresser une lettre recommandée au syndic pour l'informer de ce qu'il vend son
			emplacement de stationnement et du prix auquel il se propose de réaliser cette vente.
			
			Le syndic devra alors, à son tour, adresser « sans délai » une lettre recommandée à tous les
			copropriétaires, « aux frais du vendeur » ( toutefois, cet envoi ne doit pas donner à perception
			d’honoraire, n’étant pas prévu dans le contrat type).
			
			Dans les copropriétés importantes (où les mutations de parkings seuls sont les plus fréquentes) cette
			procédure aura un effet dissuasif sur le copropriétaire compte tenu des frais et éventuellement honoraires
			qu'elle va engendrer pour lui.
			
			Une fois « cette notification faite par le syndic, le copropriétaire devra attendre l'expiration d'un délai de
			deux mois pour savoir si son parking a été préempté ou s’il récupère la libre disposition de celui-ci.
	
			Toutefois, la loi MOLLE prévoit simplement la possibilité d’insérer une telle clause dans le règlement de
			copropriété. Il faut donc faire une modification du règlement de copropriété qui, en principe, requière
			l’unanimité (modification des droits du copropriétaire sur son lot)\footnote{en ce sens : réponse ministérielle, Question \no \nombre{120 883}, publiée au JO le 3 janvier 2012, page 86}.
		
		\subsubsection{Le droit de préemption en cas de vente par le syndicat de copropriété du droit de surélever (art. 35 de la loi du 10 juillet 1965)}
	
			\begin{quote}
				\textbf{Article 35 de la loi alinéa premier}\newline
				La surélévation ou la construction de bâtiments aux fins de créer de nouveaux locaux à usage privatif ne
				peut être réalisée par les soins du syndicat que si la décision en est prise à la majorité prévue à l’article 26.
			\end{quote}
			
			La décision d’aliéner aux mêmes fins le droit de surélever un bâtiment existant exige la majorité prévue à
			l’article 26, et, si l’immeuble comprend plusieurs bâtiments, la confirmation par une assemblée spéciale
			des copropriétaires des lots composant le bâtiment à surélever, statuant à la majorité indiquée ci-dessus.
			
			Les copropriétaires de l'étage supérieur du bâtiment surélevé bénéficient d'un droit de priorité à l'occasion
			de la vente par le syndicat des locaux privatifs créés. Préalablement à la conclusion de toute vente d'un ou
			plusieurs lots, le syndic notifie à chaque copropriétaire de l'étage supérieur du bâtiment surélevé l'intention
			du syndicat de vendre, en indiquant le prix et les conditions de la vente. Cette notification vaut offre de
			vente pendant une durée de deux mois à compter de sa notification.
			
			Les copropriétaires de l'étage supérieur du bâtiment à surélever bénéficient du même droit de priorité à
			l'occasion de la cession par le syndicat de son droit de surélévation. << Ce droit de priorité s'exerce dans les
			mêmes conditions que celles prévues au quatrième alinéa >>.
			
			\bigskip
			Avant la loi ALUR, les propriétaires du dernier étage disposaient d’un droit de véto en cas de surélévation.
			Ce droit a été transformé en simple droit de « priorité », purgé par le syndic (et non le notaire !), qui doit
			notifier à ces propriétaires « le prix et les conditions de la vente ». Ce n’est qu’une fois ce « droit de
			priorité » purgé que le droit peut être offert à la vente aux tiers.
			
			Toutefois, le texte ne précise ni comment le syndic fixe le prix de vente avant la tenue de l’assemblée
			générale, ni comment traiter l’éventuel conflit entre différents propriétaires du dernier étage, ni si l’offre
			doit comporter un prix de vente « global » ou divisé.
	
	\subsection{L’interdiction d’acquérir}
	
		Sous couvert d’un chapitre relatif à la lutte contre « les acquéreurs déstabilisateurs en copropriété », la loi
		ALUR a introduit une double restriction au droit d’acquérir un lot. L’objectif de ces dispositions est
		d’éviter, dans une copropriété en difficulté notamment, les acquisitions massives de lots par des
		marchands de sommeil, qui diviserons les lots pour louer , sans pour autant payer leurs charges.
		Le dispositif a été complété et renforcé par la loi ELAN.
		
		\subsubsection{Lutte contre les marchands de sommeil}
		
			La loi ALUR a ajouté au Code Pénal des peines complémentaires à l’encontre de personnes condamnées
			pour avoir soumis « une personne, dont la vulnérabilité ou l'état de dépendance sont apparents ou connus
			de l'auteur, à des conditions de travail ou d'hébergement incompatibles avec la dignité humaine ». La
			sanction principale est puni de cinq ans d'emprisonnement et de 150 000 euros d'amende. » ( infraction
			défini à l’article L.~225-14).
			
			Cette peine complémentaire sanctionne précisément les pratiques des marchands de sommeil :
			\begin{itemize}
				\item Le fait de soumettre une personne, dont la vulnérabilité ou l'état de dépendance sont apparents
				ou connus de l'auteur, à des conditions de travail ou d'hébergement incompatibles avec la
				dignité humaine est puni de cinq ans d'emprisonnement et de 150 000 euros d'amende (article
				225-14 du Code Pénal).
				
				\item  Le fait de mettre à disposition aux fins d'habitation, à titre gratuit ou onéreux des caves, sous-sols,
				combles, pièces dépourvues d'ouverture sur l'extérieur et autres locaux par nature impropres à
				l'habitation et de ne pas déférer à l’injonction préfectorale de faire cesser la situation (\article{L}{1337-4} du	Code de la santé publique).
				
				\item  Le refus délibéré et sans motif légitime de réaliser des travaux prescrit par le Maire pour faire
				cesser la situation d'insécurité constatée par la commission de sécurité dans un établissement
				recevant du public à usage total ou partiel d'hébergement (\article{L}{123-3} du \CCH).
			
				\item  Le refus délibéré et sans motif légitime, constaté après mise en demeure, d'exécuter les travaux
				prescrits par le Maire par suite d’un arrêté de péril ou d’insalubrité (\article{L}{511-6} du \CCH)
			\end{itemize}
			
			Selon l’actuel \article{L}{225-26} du Code pénal, doivent être obligatoirement prononcées les peines
			complémentaires suivantes (sauf décision motivée).
			\begin{quote}
				1\degre{} La confiscation de tout ou partie de leurs biens, quelle qu'en soit la nature, meubles ou immeubles, divis
				ou indivis, ayant servi à commettre l'infraction. Lorsque les biens immeubles qui appartenaient à la
				personne condamnée au moment de la commission de l'infraction ont fait l'objet d'une expropriation pour
				cause d'utilité publique, le montant de la confiscation en valeur prévue au neuvième alinéa de l'article
				131-21 est égal à celui de l'indemnité d'expropriation ;
				
				2\degre{} L'interdiction pour une durée de dix ans au plus d'acheter un bien immobilier à usage d'habitation ou
				un fonds de commerce d'un établissement recevant du public à usage total ou partiel d'hébergement ou
				d'être usufruitier d'un tel bien ou fonds de commerce. Cette interdiction porte sur l'acquisition ou
				l'usufruit d'un bien ou d'un fonds de commerce soit à titre personnel, soit en tant qu'associé ou mandataire
				social de la société civile immobilière ou en nom collectif se portant acquéreur ou usufruitier, soit sous
				forme de parts immobilières ; cette interdiction ne porte toutefois pas sur l'acquisition ou l'usufruit d'un
				bien immobilier à usage d'habitation à des fins d'occupation à titre personnel ;
				
				3\degre{} La confiscation de tout ou partie des biens leur appartenant ou, sous réserve des droits du propriétaire
				de bonne foi, dont elles ont la libre disposition, quelle qu'en soit la nature, meubles ou immeubles, divis
				ou indivis.
			\end{quote}
			
			Il est en conséquence inséré au sein du Code des procédures civiles d'exécution, un \article{L}{322-7-1} : la
			personne qui est condamnée à l'une des peines complémentaires précitées ne peut se porter enchérisseur
			pendant la durée de cette peine pour l'acquisition d'un bien immobilier à usage d'habitation ou d'un fonds
			de commerce d'un établissement recevant du public à usage total ou partiel d'hébergement, sauf dans le
			cas d'une acquisition pour une occupation à titre personnel.
			
			Le Code de la Construction et de l'Habitation comprend également un \article{L}{551-1} qui impose au notaire
			qui établit l’acte de vente de vérifier si l’acquéreur a fait l’objet d’une condamnation prévue par l’\article{L}{225-19}. Si tel est le cas, il ne pourra recevoir l’acte de vente, et la promesse sera résiliée aux torts de
			l’acquéreur.

			Le notaire doit interroger l'Association pour le développement du service notarial placée sous le contrôle
			du Conseil supérieur du notariat, qui demande consultation du bulletin \no 2 du casier judiciaire de
			l'acquéreur au casier judiciaire national automatisé.
			
			Enfin, depuis la loi ELAN, le syndic doit signaler au procureur de la République les faits qui sont susceptibles
			de constituer une des infractions prévues aux articles 225-14 du Code pénal, \article{L}{1337-4} du Code de la santé
			publique et \article{L}{123-3}, L. 511-6 et L. 521-4 du Code de la construction et de l'habitation (L. \no 65-557, 10 juill.
			1965, art. 18-1-1.), à l’exception des syndics non professionnels. La disposition a été introduite dans la Loi
			du 10 juillet 1965, mais également dans la loi HOGUET (L. \no 70-9, 2 janv. 1970, JO 4 janv., art. 8-2-1, nouv.),
			elle est donc contrôlée par le CNTGI et la DGCCRF.
			
			Les personnes qui se livrent à des activités d'entremise et de gestion des immeubles et fonds de commerce
			doivent effectuer ce même signalement
		
		\subsubsection{Interdiction d’acquérir un nouveau lot concernant un copropriétaire débiteur de charges}
		
			L’article 55 de la loi ALUR, ajoutant un paragraphe \II{} à l’article 20 de la loi du 10 juillet 1965, a été introduit
			dans le but d’interdire aux copropriétaires « déstabilisateurs » d’acheter de nouveaux lots dans la
			copropriété –-- en pratique est visé tout copropriétaire présentant un risque identifiable de ne pas acquitter
			ses charges.
			
			Préalablement à la signature de la vente, le notaire doit indiquer au syndic le nom du futur acheteur, ainsi
			que celui de son partenaire pacsé ou de son époux, ou encore de tous ses associés et mandataires sociaux
			s’il s’agit d’une SCI ou d’une SNC, afin de savoir si l’une de ces personnes est déjà copropriétaire dans
			l’immeuble et s’il est à jour du paiement de ses charges.
			
			Le syndic dispose d’un mois pour répondre. S’il indique l’une de ces personnes a fait l’objet d’une mise en
			demeure de payer restée infructueuse plus de 45 jours, le notaire doit informer les parties de
			l’impossibilité de régulariser la vente. Le candidat acquéreur (ou son conjoint, pacsé $\dots$) devra s’acquitter
			des sommes dues dans les 30 jours de la notification par le notaire de son refus d’instrumenter ; seul un
			certificat du syndic attestant de cet acquit permettra au notaire de régulariser la vente. A défaut, la
			promesse de vente sera résiliée, aux torts de l’acquéreur, lui faisant ainsi perdre la somme versée à titre
			d’indemnité d’immobilisation ou de dépôt de garantie.
			
			Le dispositif, peu efficace contre les marchands de sommeil, est en revanche redoutable contre les
			copropriétaires « mauvais payeurs » qui souhaitent acheter un lot annexe dans la copropriété

\section{La cession des parties communes}

	\subsection{Interdiction de cession de la quote-part de partie commune séparément des parties privatives (article 6 de la loi du 10 juillet 1965)}
	
		\begin{quote}
			<< Les parties communes et les droits qui leur sont accessoires ne peuvent faire l'objet, séparément des
			parties privatives, d'une action en partage ni d'une licitation forcée >>.
		\end{quote}
		
		Le copropriétaire n'a de droit de cession que sur son lot, considéré en tant qu'une entité indivisible
		composé des parties privatives et de la quote-part de parties communes.
		
		On peut cependant s'interroger sur les droits accessoires aux parties privatives.
		\begin{itemize}
			\item Lorsque de tels droits sont constitués en lots privatifs, leur aliénation ne posera aucun
			problème puisque le lot comportera à la fois un droit purement privatif (celui de construire
			par exemple) et une quote part de parties communes (les tantièmes dont ce lot est affecté).
			
			\item Lorsque le lot privatif comportera en même temps un volume privatif, la jouissance d'une
			partie de l'immeuble (jardin) et le droit accessoire proprement dit : le copropriétaire pourra
			dans le respect de l'article 6 de la loi diviser son lot en deux nouveaux lots en répartissant
			les tantièmes existants entre ces deux lots, le premier comprenant le volume privatif et le
			second la jouissance du jardin avec droit de construire.
			
			\item En revanche lorsque ces droits accessoires ne sont pas constitués en lot, mais se
			<< découvrent >> à la lecture des dispositions du Règlement de Copropriété, ou dans la
			description du lot, le copropriétaire ne sans enfreindre les dispositions de l'article 6 de la
			loi, ne céder que ces droits accessoires en conservant pour lui le volume seul constitué en
			lot privatif doté de quotes-parts de parties communes.
		\end{itemize}
		
		\textbf{Exception :} l’\article{L}{615-10} du \CCH{} (article 72 de la loi ALUR)
		\begin{quote}
			– \I. – Par dérogation à l’article 6 de la loi \no 65-557 du 10 juillet 1965 fixant le statut de la copropriété des
			immeubles bâtis, une possibilité d’expropriation des parties communes est instaurée à titre expérimental
			et pour une durée de dix ans à compter de la promulgation de la loi \no du pour l’accès au logement
			et un urbanisme rénové. Dans ce cas, l’article L. 13-10 du code de l’expropriation pour cause d’utilité
			publique est applicable
		\end{quote}
		
		Cette procédure qui concerne le dispositif législatif mis en place pour parer à la carence des
		syndicats de copropriété fait l’objet de commentaires au Chapitre du cours relatif à la disparition
		d’un syndicat des copropriétaires et du dernier Chapitre du cours (Poly 2) relatif aux copropriétés
		en difficulté.
	
	\subsection{Interdiction de cession des parties communes matérielles du syndicat}
	
		Quant aux parties communes matérielles, il va de soi qu'un copropriétaire ne peut en disposer
		personnellement, par exemple vendre des locaux de services communs.
		
		Ce droit de disposer de parties communes déterminées appartient au syndicat statuant à la majorité de
		l'article 26 (majorité des copropriétaires représentant les deux tiers des voix), étant précisé que s'il s'agit
		de parties communes dont la conservation est nécessaire au respect de la destination de l'immeuble,
		l'unanimité est requise (article 26 dernier alinéa).
		
		La vente de parties communes, à la supposer parfaite, apporte une modification au règlement de
		copropriété qui n’est pas opposable aux acquéreurs qu’à dater de sa publication au fichier immobilier
		Le transfert de propriété de peut s’opérer qu’à compter de la création de nouveaux lots de copropriété et
		l’attribution au lot créé de tantièmes de parties communes.
		
		La décision qui arrête le principe de la vente n’est qu’une décision préparatoire qui n’opère pas de
		transfert de propriété, celle-ci requérant une nouvelle décision d’assemblée statuant sur la vente et sur
		toutes les modifications du règlement de copropriété et de l’état descriptif de division en résultant\footnote{CA Paris 2 juillet 2009 JD 09-378843}.
	
	\subsection{L’expropriation du syndicat}
	
		Quatre types d’expropriation peuvent frapper le Syndicat
		\begin{itemize}
			\item  l'expropriation pour cause d’utilité publique ;
			\item  l'expropriation pour carence (\article{L}{615-6} du \CCH, dite Loi \nom{Borloo})
			\item  l’expropriation en cas de péril imminent ( art. , dite loi Vivien)
			\item  l’expropriation des parties communes à titre expérimental, pour les copropriétés
			<< irrémédiablement dégradées >>
		\end{itemize}
		
		\subsubsection{L'expropriation pour cause d’utilité publique}
		
		La procédure d'expropriation permet à une collectivité territoriale de s'approprier des biens immobiliers
		privés, afin de réaliser un projet d'aménagement dans un but d'utilité publique. Une opération
		d'expropriation ne peut être légalement déclarée d'utilité publique que si les atteintes à la propriété privé,
		le coût financier et éventuellement les inconvénients d'ordre social qu'elle comporte ne sont pas excessifs
		eu égard à l'intérêt qu'elle présente.
		
		La procédure d'expropriation se décompose en deux phases.
		\begin{enumerate}
			\item \textbf{La phase administrative} dont la finalité est la déclaration d'utilité publique du projet prononcé par arrêté
			préfectoral (enquête d'utilité publique) et la détermination des parcelles à exproprier définies par un
			arrêté préfectoral de cessibilité (enquête parcellaire) ; elle aboutit à deux arrêté préfectoraux, l’arrêté de
			déclaration d’utilité publique et l’arrêté de cessibilité. Chacune de ces décisions est susceptible de recours
			devant le Juge administratif dans le délai de deux mois suivant la notification de la décision.
			
			\item \textbf{La phase judiciaire}, qui correspond à la procédure de transfert de propriété des biens et d'indemnisation
			des propriétaires. Cette procédure est instruite par le juge de l'expropriation dès la transmission du dossier
			administratif finalisé par le préfet au juge de l'expropriation. Dans un délai qui ne peut excéder 6 mois à
			compter de la date de l'arrêté de cessibilité, et si l'acquisition des parcelles n'a pas pu se faire à l'amiable,
			l'expropriant saisit le préfet aux fins de transmettre le dossier au juge de l'expropriation (au greffe du
			tribunal de grande instance), afin que celui-ci prononce l'ordonnance d'expropriation.
		\end{enumerate}
		
		C'est en effet le préfet, exclusivement, qui saisit le juge de l'expropriation sur demande de l'expropriant.
	
		Le principal effet de l'ordonnance d'expropriation est de transférer à l'expropriant la propriété de
		l'immeuble exproprié. Mais la prise de possession est subordonnée au fait que l'indemnité d'expropriation
		ait été payée ou consignée.
		
		Cette expropriation peut porter sur une partie commune (cas le plus fréquent, expropriation d’une partie
		du terrain), auquel cas seul le Syndicat est poursuivi dans le cadre de la procédure d’expropriation, et
		l’indemnité sera répartie entre tous les copropriétaires au prorata des tantièmes. Cependant
		l'indemnisation du syndicat des copropriétaires pour l'expropriation de parties communes n'exclut pas
		nécessairement celle de chaque copropriétaire pour la dévalorisation de la partie privative de son lot\footnote{Cour de cassation, 3e Chambre civ., 11 octobre 2006 (\no de pourvoi: 05-16.037)}.
		
		Si l’expropriation touche aussi une partie privative, ou une partie commune à jouissance privative ( par
		exemple un lot parking), l’expropriation devra être poursuivie contre le Syndicat et le copropriétaire à titre
		individuel\footnote{voir sur ce sujet le mémoire d’anne Gazeau: Le statut de la copropriété dans la procédure
			D’expropriation https://dumas.ccsd.cnrs.fr/dumas-01701078}.
		
		L’expropriation de la partie commune a pour conséquence le retrait de la partie de terrain expropriée (au
		delà de la ligne divisoire, qui doit figurer dans l’enquête parcellaire), donc une scission de la copropriété
		(art 16-2 de la loi du 10 juillet 1965 issu de la loi \no96-987 du 14 novembre 1996). Elle nécessite donc, en
		principe, une adaptation du règlement consécutive à la scission.
		
		\subsubsection{L’expropriation du Syndicat en état de carence (loi \no 2003-710 du 1er août 2003 d'orientation et de programmation pour la ville et la rénovation urbaine, dite « loi Borloo » : article L.615-6 et L.~615-7 du \CCH)}
		
		L’état de carence peut être prononcé par le président du tribunal de grande instance, suite au rapport de
		l’expert désigné par ses soins, lorsqu' en raison de graves difficultés financières et de gestion un
		propriétaire, un syndicat des copropriétaires est dans l’impossibilité d’assurer la réalisation des travaux
		nécessaire à la conservation des immeubles et à la sécurité des occupants.
		
		Lorsque l’état de carence du ou des immeubles a été déclaré (par ordonnance du Président du TGI rendue
		au vue du rapport de l’expert et les parties entendues), l’expropriation est poursuivie, dans les conditions
		fixées par le code de l’expropriation pour cause d’utilité publique, au bénéfice de la commune ou de
		l’établissement public de coopération intercommunale compétent en matière de logement pour la mise
		en oeuvre d’actions ou d’opérations de rénovation urbaine ou de politique locale de l’habitat. Cette
		expropriation porte sur les parties communes et sur les lots privatifs.
		
		Le cas échéant, dans l'ordonnance prononçant l'état de carence, le président du tribunal de grande
		instance désigne un administrateur provisoire mentionné à l'article 29-1 de la loi \no 65-557 du 10 juillet
		1965 précitée pour préparer la liquidation des dettes de la copropriété et assurer les interventions
		urgentes de mise en sécurité.

		Ces dispositions ont été renforcées par la loi ALUR\footnote{Voir Poly 2 – Les Copropriétés en difficulté}, et par la Loi ELAN.
		Ainsi, depuis la loi ELAN :
		\begin{itemize}
			\item l’agence nationale de l’habitat (Anah) doit encourager et faciliter l’exécution d’opérations de
			résorption d’une copropriété dont l’état de carence a été déclaré conformément à l’\article{L}{615-6} du Code de la construction et de l’habitation (CCH, art. L. 321-1, \I, mod.) ;
			\item la procédure d’expropriation doit être menée de façon contradictoire à l’encontre du Syndicat,
			mais aussi de chacun des copropriétaires, ce qui n’était pas le cas jusqu’à la loi ELAN (le rapport
			de l’expert doit leur être notifié, et ils sont nécessairement assignés individuellement dans le cadre
			de la procédure).
		\end{itemize}
		
		\subsubsection{L’expropriation des copropriétés à usage d’habitation insalubres ou dangereux (Dite expropriation << Loi \nom{Vivien} >> L 1331-25 et L 1331-28 du CCH, article L 511-2 du CCH)}
		
		Cette expropriation concerne visés les immeubles déclarés insalubres à titre irrémédiable en
		application des articles L. 1331-25 et L. 1331-28 du code de la santé publique, ou frappés d’un arrêté de
		péril et interdits définitivement à l’habitation, en application de l’\article{L}{511-2} du \CCH. Ce même article
		précise que peuvent aussi être expropriés selon ce mode dérogatoire des immeubles qui ne sont eux-mêmes
		ni insalubres, ni impropres à l'habitation, lorsque leur expropriation est indispensable à la
		démolition des immeubles insalubres ou d'immeubles menaçant ruine, ainsi que des terrains où sont situés
		les immeubles déclarés insalubres ou menaçant ruine lorsque leur acquisition est nécessaire à la
		résorption de l'habitat insalubre.
		
		La DUP peut être signée par le préfet sur la seule base des arrêtés d'insalubrité irrémédiable ou d'un arrêté
		de péril ayant ordonné la démolition du bâtiment ou prononcé une interdiction définitive d'habiter.
		
		La DUP désigne le bénéficiaire de l'expropriation, mentionne obligatoirement les offres de relogement
		faites aux occupants (y compris les propriétaires). Elle est notifiée au Syndicat et aux copropriétaires.
		
		La même DUP déclare cessibles les terrains et immeubles visés dans l'arrêté (cette cession peut intervenir
		au profit d’un acquéreur privé) et fixe le montant des indemnités provisionnelles dues aux propriétaires
		(et titulaires de baux commerciaux) qui ne peuvent être inférieures à l'évaluation des Domaines.
		
		L’indemnité d’expropriation est fixée à la valeur dite de << récupération foncière >>, c’est à dire à la valeur du
		terrain nu, déduction faite des travaux de démolition. L’indemnité est réduite du montant des frais de
		relogement des occupants assuré, lorsque le propriétaire n'y a pas procédé. Seuls les occupants de bonne
		fois, depuis au moins 2 ans avant la notification de l'arrêté d’insalubrité ou de péril, ont droit à une
		indemnité fixée selon le droit commun (ou si l’immeuble n’est exproprié que par voie de conséquence
		dans le cadre d’une opération de RHI, sans être lui-même insalubre ou frappé de péril).
		
		\subsubsection{L’expropriation des parties communes a titre expérimental}

		C’est un dispositif << révolutionnaire >> introduit par la loi ALUR dans l’\article{L}{615-10} du \CCH, qui déroge
		explicitement à l’article 6, puisqu'il aboutit précisément à la dissociation des parties communes et
		privatives. Le but est de réduire le coût du portage par l’opérateur en cas de réhabilitation d’une
		copropriété en difficulté ne nécessitant pas l’expulsion des occupants. Il aboutit à la disparition du Syndicat
		(voir le chapitre I).
		
		\begin{quote}
			Article L 615-10 CCH (nouveau)
			<< \I. – Par dérogation à l’article 6 de la loi \no 65-557 du 10 juillet 1965 fixant le statut de
			la copropriété des immeubles bâtis, une possibilité d’expropriation des parties
			communes est instaurée à titre expérimental et pour une durée de dix ans à compter de
			la promulgation de la loi \no du pour l’accès au logement et un urbanisme
			rénové. Dans ce cas, l’article L. 13-10 du code de l’expropriation pour cause d’utilité
			publique est applicable
			
			\medskip
			« \II. – Lorsque le projet mentionné au \V{} de l’article L. 615-6 du présent code prévoit
			l’expropriation de l’ensemble des parties communes, la commune ou l’établissement
			public de coopération intercommunale compétent en matière d’habitat peut confier
			l’entretien de ces biens d’intérêt collectif à un opérateur ou désigner un opérateur au
			profit duquel l’expropriation est poursuivie.
			
			« Au moment de l’établissement du contrat de concession ou de la prise de
			possession par l’opérateur, l’état descriptif de division de l’immeuble est mis à jour ou
			établi s’il n’existe pas. Aux biens privatifs mentionnés dans l’état de division est attachée
			une servitude des biens d’intérêt collectif. Les propriétaires de ces biens privatifs sont
			tenus de respecter un règlement d’usage établi par l’opérateur.
			
			« En contrepartie de cette servitude, les propriétaires sont tenus de verser à
			l’opérateur une redevance mensuelle proportionnelle à la superficie de leurs parties
			privatives. Cette redevance, dont les modalités de révision sont prévues par décret,
			permet à l’opérateur de couvrir les dépenses nécessaires à l’entretien, à l’amélioration
			et à la conservation de parties communes de l’immeuble et des équipements communs.
			
			« Pour les propriétaires occupants, cette redevance ouvre droit aux allocations de
			logement prévues aux articles L. 542-1 à L. 542-9 et L. 831-1 à L. 835-7 du code de la
			sécurité sociale.
			
			\medskip
			« \III. – L’opérateur est chargé d’entretenir et de veiller à la conservation des biens
			d’intérêt collectif. Il est responsable des dommages causés aux propriétaires de parties
			privatives ou aux tiers par le vice de construction ou le défaut d’entretien des biens
			d’intérêt collectif, sans préjudice de toutes actions récursoires.
			
			« Il réalise un diagnostic technique des parties communes, établit un plan
			pluriannuel de travaux actualisé tous les trois ans et provisionne, dans sa comptabilité,
			des sommes en prévision de la réalisation des travaux.
			
			\medskip
			« \IV. – Le droit de préemption urbain renforcé prévu à l’article L. 211-4 du code de
			l’urbanisme peut lui être délégué.
			
			\medskip
			« \V. – Dans le cadre de l’expérimentation prévue au présent article, en cas de déséquilibre
			financier important, l’opérateur peut demander à la commune ou à l’établissement
			public de coopération intercommunale compétent en matière d’habitat à l’origine de
			l’expérimentation de procéder à l’expropriation totale de l’immeuble. Un nouveau
			projet d’appropriation publique doit alors être approuvé dans les conditions prévues
			au V de l’article L. 615-6 du présent code. La procédure est poursuivie dans les
			conditions prévues à l’article L. 615-7.
			
			\medskip
			« \VI. – Après avis favorable de la commune ou de l’établissement public de coopération
			intercommunale compétent en matière d’habitat à l’origine de l’expérimentation et des
			propriétaires des biens privatifs, l’immeuble peut faire l’objet d’une nouvelle mise en
			copropriété à la demande de l’opérateur. Les propriétaires versent alors une indemnité
			au propriétaire de ces biens d’intérêt collectif équivalente à la valeur initiale
			d’acquisition des parties communes ayant initialement fait l’objet de l’expropriation,
			majorée du coût des travaux réalisés, de laquelle est déduit le montant total des
			redevances versées à l’opérateur. Cette indemnité est répartie selon la quote-part des
			parties communes attribuée à chaque lot dans le projet de règlement de copropriété.
		\end{quote}
		
		C’est donc une loi à caractère temporaire (par contre l’expropriation pratiquée est définitive). Seule est
		prévue au \VI, la possibilité de recréer une copropriété à la demande de l’opérateur lorsque les travaux de
		remise en état des parties communes auront été réalisés ; auquel cas les copropriétaires,
		proportionnellement à leurs tantièmes verseront à l’opérateur « une indemnité ( ?) » correspondant au
		coût d’achat des parties communes par l’Opérateur, majoré du coût des travaux réalisés par l’Opérateur
		et diminué le montant des redevances payés par les propriétaires à l’Opérateur pour la conservation de
		ces (ex)parties communes.
		
		\par Le texte précise les modalités d’application et les conséquences quant à la gestion de ces parties
		communes expropriées.
		\begin{itemize}
			\item  La procédure normale d’expropriation est mise en œuvre (avec bien évidemment une DUP).
			
			\item  La Commune (ou l’EPCI) confie l’entretien des parties communes expropriées à un Opérateur qui
			établira une servitude au profit des parties privatives sur les parties communes expropriées et
			rédigera un règlement d’usage. En contrepartie de cette servitude les copropriétaires devront
			payer une « redevance » à l’opérateur qui couvrira non seulement les dépenses d’entretien et de
			conservation de ces parties communes mais également les travaux d’amélioration de ces parties
			communes et de leurs équipements.
			
			\item  Enfin le texte précise que cette expropriation partielle (ne portant que sur les parties communes)
			peut être le premier pas vers une expropriation totale de l’immeuble (donc des parties privatives)
			si l’Opérateur ne parvient pas à équilibrer la gestion de ces parties communes.
			
			En fait ce texte est l’aboutissement d’une longue réflexion du Ministère du Logement qui avait suscité les
			plus grandes réserves de la part des juristes mais qui avait été reprise dans le rapport \nom{Braye}. Sa
			constitutionnalité est douteuse (cette procédure n’aboutit-elle pas, en fait, à l’expropriation du lot?), et le
			décret d’application reste en attente.
		\end{itemize}
	
\section[Lors de la promesse de vente]{Les informations et vérifications accomplies lors de la promesse de vente du lot de copropriété}

	Les formalités destinées à la bonne information de l’acquéreur --- auparavant limitées à la transmission du
	règlement de copropriété --- se sont multipliées ces dernières années : obligation de faire figurer la
	superficie de la partie privative du lot vendu, diagnostic technique obligatoire concernant les parties
	privatives et les parties communes $\dots$
	La loi ALUR a considérablement renforcé cette obligation
	d’information de l’acquéreur, et la place désormais au stade de la promesse de vente : un ensemble
	d’information concernant la situation technique, juridique, et financière de la copropriété doit être
	transmise dès la promesse de vente, et ce n’est qu’à compter de la remise de l’ensemble des éléments
	d’informations que courra le délai de rétractation de l’acquéreur.
	
	Simultanément, la loi ALUR a obligé le notaire à effectuer certaines vérifications avant la vente concernant
	l’acquéreur lui-même, afin de lutter contre les acquéreurs « déstabilisateurs » (cf supra, section II C)
	
	\subsection{Les informations a communiquer a l’acquéreur lors de la promesse de vente}
	
		La liste des informations à communiquer à l’acquéreur résulte de la loi ALUR du 24 août 2014, amendée
		par l’ordonnance numéro 2015-1075 du 27 août 2015 dite de « simplification » qui a simplifié le mode de
		communication de ces informations et a introduit certaines dérogations. Ces obligations figurent dans
		l’\article{L}{721-2} et l’\article{L}{721-3} du Code de la Construction et de l’Habitation.
		
		Les documents et informations visés à l’\article{L}{721-2} ne doivent plus être « annexés » à la promesse de
		vente, ils doivent désormais être remis à l’acquéreur « au plus tard à la date de signature de la
		promesse ».
		
		Ils peuvent être remis à l'acquéreur en amont du compromis de vente, par tous moyens, y compris par un
		procédé dématérialisé », c’est-à-dire par voie électronique « sous réserve de l’acceptation expresse par
		l’acquéreur ». L’acquéreur doit alors attester les avoir reçus.
		
		Si les documents n’ont pas été joints à la promesse de vente, ils doivent être joints à l’acte authentique de
		vente.
		
		Le délai de rétractation, initialement de sept jours, porté à dix jours par la loi \nom{Macron} du 6 août 2015.
		(\article{L}{271-1} modifié du \CCH) ne court qu’à compter du lendemain de la communication de ces
		documents (excepté pour la notice d’information). En d’autres termes, l’omission de l’un de ces
		documents permet à l’acquéreur de se dégager de la promesse de vente, jusqu’au jour de la signature de
		la vente – voire dans les 10 jours suivant cette vente-, sans indemnité, en faisant jouer son droit de
		rétractation.
		
		\subsubsection{Documents concernant l’organisation de la copropriété}
			
			Doivent être communiqués à ce titre
			
			\paragraph{La fiche synthétique de la copropriété}
			
			\par Le syndic est tenu de réaliser une fiche synthétique de la copropriété regroupant les données
			financières et techniques essentielles relatives à la copropriété et à son bâti (art 8-2 modifié de
			la Loi \no 65-557 du 10 juillet 1965), sauf pour les copropriétés autres que d’habitation.
			
			Cette fiche doit être réalisée dans des délais qui varient en fonction de la taille de la copropriété :
			\begin{itemize}
				\item plus de 200 lots de copropriété, à partir du 31 décembre 2016 ;
				\item plus de 50 lots et jusqu'à 200 lots de copropriété, à partir du 31 décembre 2017 ;
				\item jusqu'à 50 lots de copropriété, à partir du 31 décembre 2018.
			\end{itemize}
			
			Cette fiche doit être mise à disposition des copropriétaires qui en font la demande par tous
			moyens, et remise lors de la promesse de vente. Elle peut extraite du registre national des
			copropriétés si la copropriété est déjà immatriculée sur ce registre. Elle doit comporter la date de
			délivrance et la signature du syndic si c’est lui qui l’établit. Elle doit être mise à jour tous les ans,
			dans le délai de 2 mois suivant la notification du procès-verbal de l'assemblée générale au cours
			de laquelle les comptes de l'exercice clos ont été approuvés.
			
			La fiche synthétique est le seul document dont l’absence n’a pas pour effet de reporter le point
			de départ du délai de rétractation. En revanche, Le défaut de réalisation de la fiche synthétique
			est un motif de révocation du syndic. Les contrats de syndic prévoient obligatoirement une
			pénalité financière forfaitaire automatique à l'encontre du syndic chaque fois que celui-ci ne met
			pas la fiche synthétique à disposition d'un copropriétaire dans un délai de quinze jours à compter
			de la demande. Cette pénalité est déduite de la rémunération du syndic lors du dernier appel de
			charges de l'exercice.
			
			Le contenu de la fiche synthétique a été fixé par le Décret \no 2016-1822 du 21 décembre 2016.
			\begin{itemize}
				\item Identification de la copropriété, du syndic ou de l'administrateur provisoire.
					\newline La fiche synthétique doit mentionner :
					\begin{itemize}
						\item le nom d'usage, s'il y a lieu, et l'adresse(s) du syndicat de copropriétaires ;
						\item l'adresse(s) du ou des immeubles (si différente de celle du syndicat) ;
						\item le numéro d'immatriculation de la copropriété et la date de sa dernière mise à jour ;
						\item la date d'établissement du règlement de copropriété et le numéro identifiant d'établissement
						(SIRET) du syndicat ;
						\item le nom, prénom et adresse du représentant légal de la copropriété (syndic ou administrateur
						provisoire) et le numéro identifiant d'établissement (SIRET) du représentant légal ;
						\item le cadre d'intervention du représentant légal (mandat de syndic ou mission d'administration
						provisoire).
					\end{itemize}
			
				\item Organisation juridique de la copropriété
					\newline La fiche synthétique doit mentionner :
					\begin{itemize}
						\item la nature du syndicat (principal/secondaire/coopératif) ou résidence-services ;
						\item s'il s'agit d'un syndicat secondaire, numéro d'immatriculation au registre national des
						copropriétés du syndicat principal du syndicat de copropriétaires.
					\end{itemize}
				
				\item Caractéristiques techniques de la copropriété
					\newline La fiche synthétique doit mentionner :
					\begin{itemize}
						\item le nombre total de lots inscrit dans le règlement de copropriété ;
						\item le nombre total de lots à usage d'habitation, de commerces et de bureaux inscrit dans le
						règlement de copropriété ;
						\item le nombre de bâtiments ;
						\item la période de construction des bâtiments.
					\end{itemize}
				
				\item  Équipements de la copropriété
					La fiche synthétique doit mentionner :
					\begin{itemize}
						\item le type de chauffage et, pour un chauffage collectif (partiel ou total) non urbain le type d'énergie utilisée ;
						\item le nombre d'ascenseurs.
					\end{itemize}
				
				\item  Caractéristiques financières de la copropriété
					\begin{itemize}
						\item les dates de début et de fin de l'exercice comptable et la date de l'assemblée générale ayant approuvé les comptes ;
						\item le montant des charges pour opérations courantes ;
						\item le montant des charges pour travaux et opérations exceptionnelles ;
						\item le montant des dettes fournisseurs, rémunérations et autres ;
						\item le montant des impayés ;
						\item le nombre de copropriétaires débiteurs dont la dette dépasse \montant{300} ;
						\item le montant du fonds de travaux.
					\end{itemize}
			\end{itemize}
			
			Toutefois, les syndicats comportant moins de 10 lots à usage de logements, de bureaux ou de commerces,
			dont le budget prévisionnel moyen sur une période de 3 exercices consécutifs est inférieur à \montant{15 000}, ne
			sont pas tenus de fournir le nombre de copropriétaires débiteurs et le montant des impayés.
			
			\paragraph{L’état descriptif de division et le règlement de copropriété et ses modificatifs}
			Doivent être remis
			\begin{itemize}
				\item le règlement de copropriété et l’état descriptif de division ;
				\item ainsi que tous les actes modificatifs publiés même s’ils ne concernent pas directement les lots
			vendus.
			\end{itemize}
			
			Le notaire devra donc veiller à dénoncer d’éventuels modificatifs de l’état descriptif de division dont le
			syndic pourrait ne pas avoir connaissance (état descriptif de division modificatif consécutif à une division
			de lot, par ex.), en levant une fiche d’immeuble.

			Il devra également veiller à interroger le syndic sur les éventuels modificatifs votés et qui ne seraient pas
			publiés.
			
			\paragraph{Les procès-verbaux des 3 dernières assemblées générales}
			Le vendeur doit communiquer les procès-verbaux des trois dernières années, « sauf s’il n’a pas été en
			mesure de les obtenir du syndic ».
			
			\paragraph{La notice d’information générale sur les droits et obligations du copropriétaire}
			En attente d’un arrêté.
			
		\subsubsection{Les documents concernant la situation technique de l’immeuble}
		
			\paragraph{Le carnet d’entretien de l’immeuble}
			
			\par Le carnet d'entretien est un document obligatoire, qui recense toutes les informations
			permettant le suivi des travaux et des contrats de maintenance concernant l'immeuble et ses
			équipements.
			
			\subparagraph{Informations obligatoires}
			Le carnet d'entretien mentionne :
			\begin{itemize}
				\item l'adresse de l'immeuble,
				\item l'identité du syndic en exercice,
				\item les références des contrats d'assurance souscrits par le syndicat des copropriétaires, avec leurs
				dates d'échéance,
				\item l'année de réalisation des gros travaux (ravalement de façade, réfection de toiture, remplacement
				de chaudière, d'ascenseur ou de canalisations par exemple), et ceux apparaissant nécessaires au
				vu du diagnostic technique global (DTG),
				\item l'identité des entreprises qui ont réalisé ces travaux,
				\item la référence des contrats d'assurance dommage-ouvrage dont la garantie est en cours,
				\item s'ils existent, les références des contrats d'entretien et de maintenance des équipements
				communs (ascenseur, chaudière...) avec leurs dates d'échéance, ainsi que l’échéancier du
				programme pluriannuel de travaux décidé en assemblée générale.
			\end{itemize}
				
			\subparagraph{Informations complémentaires}
			
			\par Le carnet d'entretien doit également mentionner toutes les informations complémentaires que
			les copropriétaires décident d'y faire figurer lors d'un vote en assemblée générale à la majorité
			simple.

			Il peut notamment s'agir d'informations relatives :
			\begin{itemize}
				\item à la construction de l'immeuble ;
				\item aux études techniques réalisées.
			\end{itemize}
		
			La loi \no 2018-1021 du 23 novembre 2018 dite ELAN a généralisé le carnet d’entretien, en l’imposant pour
			tout logement, donc également pour la partie privative du lot. Toutefois, cette obligation généralisée ne
			concerne que les logements neufs (permis postérieur au 1\ier{} janvier 2020), ou les mutations de lots dans
			les immeubles antérieures à compter de 2025.
			
			Ce carnet d’entretien est dématérialisé, sa transmission concernant les parties communes incombe au
			syndic. Toutefois, l’entrée en vigueur du texte est subordonnée à un décret.
			
			Ainsi, l’\article{L}{111-10-5} du \CCH{} dispose désormais :
			\begin{quote}
				\I{} - Il est créé pour tout logement un carnet numérique d'information, de suivi et d'entretien de
				ce logement.
				
				Ce carnet permet de connaître l'état du logement et du bâtiment, lorsque le logement est soumis au
				statut de la copropriété, ainsi que le fonctionnement de leurs équipements et d'accompagner
				l'amélioration progressive de leur performance environnementale. Ce carnet permet l'accompagnement
				et le suivi de l'amélioration de la performance énergétique et environnementale du bâtiment et du
				logement pour toute la durée de vie de celui-ci.
				
				Les éléments contenus dans le carnet n'ont qu'une valeur informative.
				
				Le carnet numérique d'information, de suivi et d'entretien est un service en ligne sécurisé qui regroupe les
				informations visant à améliorer l'information des propriétaires, des acquéreurs et des occupants des
				logements.
				
				L'opérateur de ce service le déclare auprès de l'autorité administrative et assure la possibilité de
				récupérer les informations et la portabilité du carnet numérique sans frais de gestion supplémentaires.
				
				Le carnet numérique intègre le dossier de diagnostic technique mentionné à l'article L. 271-4 et, lorsque
				le logement est soumis au statut de la copropriété, les documents mentionnés à l'article L. 721-2
				
				Le carnet numérique d'information, de suivi et d'entretien du logement est obligatoire pour toute
				construction neuve dont le permis de construire est déposé à compter du 1er janvier 2020 et pour tous les
				logements et immeubles existants faisant l'objet d'une mutation à compter du 1er janvier 2025.
				
				Le carnet numérique d'information, de suivi et d'entretien du logement est établi et mis à jour :
				\newline 1\degre{} Pour les constructions neuves, par le maître de l'ouvrage qui renseigne le carnet numérique
				d'information, de suivi et d'entretien et est tenu de le transmettre à son acquéreur à la livraison du
				logement ;
				\newline 2\degre{} Pour les logements existants, par le propriétaire du logement. Le syndicat des copropriétaires
				transmet au propriétaire les informations relatives aux parties communes.
				
				Le carnet est transféré à l'acquéreur du logement au plus tard lors de la signature de l'acte de
				mutation.
			\end{quote}
			
			\paragraph{Le DPE ou le diagnostic technique global de l'article L. 731-1 s’il a été établi DPE ou DTG ?}
			
			\par S’il est réalisé (il n’est pas encore obligatoire dans toutes les copropriété, voir chapitre TRAVAUX) le
			diagnostic technique global doit être communiqué pour faire partir le délai de rétractation. L’obligation de
			communiquer le plan pluriannuel de travaux normalement établi à la suite.
			
			Le DTG comprend :
			\begin{quote}
				\emph{1\degre{} Une analyse de l'état apparent des parties communes et des équipements communs de l'immeuble ;
				\newline 2\degre{} Un état de la situation du syndicat des copropriétaires au regard des obligations légales et
				réglementaires au titre de la construction et de l'habitation ;
				\newline 3\degre{} Une analyse des améliorations possibles de la gestion technique et patrimoniale de l'immeuble ;
				\newline4\degre{} Un diagnostic de performance énergétique de l'immeuble tel que prévu aux articles L. 134-3 ou L. 134-4-1 du présent code. L'audit énergétique prévu au même article L. 134-4-1 satisfait cette obligation\footnote{
					Article \articleCodifie{L}{134-3} : En cas de vente de tout ou partie d'un immeuble bâti, le diagnostic de
					performance énergétique est communiqué à l'acquéreur dans les conditions et selon les modalités
					prévues aux articles \articleCodifie{L}{271-4} à \articleCodifie{L}{271-6}. Lorsque l'immeuble est offert à la vente ou à la location, le	propriétaire tient le diagnostic de performance énergétique à la disposition de tout candidat acquéreur ou locataire.
				}.
				\newline \medskip
				Il fait apparaître une évaluation sommaire du coût et une liste des travaux nécessaires à la conservation
				de l'immeuble, en précisant notamment ceux qui devraient être menés dans les dix prochaines années.}
			\end{quote}
			
			A défaut, c’est le DPE qui doit être transmis à l’acquéreur. Celui doit avoir été établi dans toutes les
			copropriété depuis le 31 décembre 2017. Il est inclus dans le DTG s’il en a été établi un.
			
			Établi par un diagnostiqueur professionnel, le DPE indique la quantité annuelle d’énergie consommée ou
			estimée pour une utilisation standardisée du bâtiment, ainsi qu’une classification du bâtiment en fonction
			de la quantité d’émission de gaz à effet de serre, le tout afin de connaitre sa performance énergétique. Il
			est accompagné de recommandations du diagnostiqueur destinées à améliorer la performance
			énergétique du bâtiment (\article{L}{134-1} du \CCH). Sa durée de validité est de dix ans.
			
			En cas de vente ou de location de tout ou partie d’un immeuble bâti, le vendeur ou le bailleur a donc
			l’obligation d’annexer à la promesse de vente, à l’acte authentique de vente ou au contrat de bail, un
			diagnostic de performance énergétique (inclus ou non dans le DTG), sauf exceptions prévues par les textes
			(articles L.271-4 du CCH pour la vente et L.134-3-1 du CCH pour la location).
			
			\subparagraph{Quelle est la Valeur du DPE ? (inclus ou non dans le DTG)}
			
			La loi ELAN modifie les articles \articleCodifie{L}{271-4} et \articleCodifie{L}{134-3-1} du \CCH{} afin que
			les informations contenues dans le diagnostic de performance énergétique à compter du 1er janvier 2021
			soient opposables aux vendeurs et aux bailleurs.
			
			Jusqu’alors, le DPE n’avait qu’une valeur informative, si bien que l’acquéreur ne pouvait, en principe, se
			prévaloir à l’encontre du vendeur ou du bailleur des informations qu’il contient. L’acquéreur ou le locataire
			peut en revanche se retourner contre le diagnostiqueur afin d’engager sa responsabilité délictuelle.
			Cependant, les nouveaux articles \articleCodifie{L}{271-4} et \articleCodifie{L}{134-3-1} du \CCH{}, issus de loi ELAN, suppriment le caractère informatif du DPE et rendent ses informations opposables au vendeur et au bailleur.
			
			Autrement dit, le vendeur ou le bailleur engagera sa responsabilité contractuelle envers l’acquéreur ou le
			locataire en cas d’information erronée figurant dans le DPE, à la condition que ladite information erronée
			leur cause effectivement un préjudice pouvant résulter, par exemple, de la perte de chance d’acquérir à
			un prix moindre ou de négocier à la baisse le montant des loyers.
			
			En revanche, les nouveaux articles \articleCodifie{L}{271-4} et \articleCodifie{L}{134-3-1} du \CCH{} prévoient que les recommandations du	diagnostiqueur accompagnant le DPE conserveront un caractère informatif et ne seront pas opposables.
			
			La loi fixe l’entrée en vigueur de ces nouvelles dispositions au 1er janvier 2021 « AFIN DE LAISSER LE
			TEMPS NÉCESSAIRE AU PLAN DE FIABILISATION DES DIAGNOSTICS ENGAGE PAR LE GOUVERNEMENT DE
			PRODUIRE TOUS SES EFFETS » (Rapp. Commission mixte paritaire, 2017-2018, art. 55 bis C). Par
			conséquent, seront opposables les informations contenues dans les DPE établis à compter de cette date.
			
		\subsubsection{Les informations financières}
		
			Les principales informations financières concernant la situation du vendeur et de la copropriété sont
			désormais transmises lors de la promesse de vente –-- au point que les praticiens parlent d’un « pré état
			daté », même si ce terme ne figure nulle part.
			
			Doivent être communiqués :
			\begin{itemize}
				\item  le montant des charges courantes du budget prévisionnel et des charges hors budget prévisionnel
				payées par le copropriétaire vendeur au titre des deux exercices comptables précédant la vente ;
				\item  l'état global des impayés de charges au sein du syndicat et de la dette vis-à-vis des fournisseurs ;
				\item  lorsque le syndicat des copropriétaires dispose d'un fonds de travaux, le montant de la part du
				fonds de travaux rattachée au lot principal vendu et le montant de la dernière cotisation au fonds
				versée par le copropriétaire vendeur au titre de son lot ;
				\item  les sommes qui seront dues au syndicat par l'acquéreur (mais non les sommes restant dues par le
				vendeur, cette obligation a été supprimée en 2015, ce montant étant de toutes façons prélevé sur
				le prix de vente, sans solidarité de l’acquéreur).
			\end{itemize}
			
			Les informations financières concernant le syndicat de copropriété doivent être communiquées à la date
			du dernier arrêté des comptes.
			
		\subsubsection{Formalités simplifiées pour certaines ventes}
			
			\paragraph{Acquéreur déjà copropriétaire dans l'immeuble :}
			N’ont pas à être communiques
			\begin{itemize}
				\item  le règlement de copropriété, l'état descriptif de division,
				\item  les procès-verbaux d'assemblées générales,
				\item  le carnet d'entretien de l'immeuble.
			\end{itemize}
			Seules les informations financières doivent donc lui être fournies.
			
			\paragraph{Lot annexe (parking, cave, grenier, débarras, placard, remise, garage ou cellier)}
			Seules les informations financières de la copropriété ainsi que le règlement de copropriété doivent être
			maintenant fournis à l'acquéreur.
		
	\subsection{Les vérifications imposées pour la lutte contre les acquéreurs déstabilisateurs}
		
		En raison des limitations imposées au droit d’acquérir par la loi ALUR, deux vérifications s’imposent au
		notaire dès le stade de la promesse de vente, ou, en l’absence d’avant contrat, lors de la vente.
		\begin{itemize}
			\item Vérification de l’absence de dettes de charges du candidat acquéreur (art 20 \II{} de la Loi du 10
			juillet 1965), par interrogation du syndic qui, dans le mois suivant le courrier du notaire, doit
			indiquer si le futur acquéreur a déjà un lot dans la copropriété, et si tel est le cas, qu’il n’a pas fait
			l’objet d’une mise en demeure de payer ses charges restée infructueuse plus de 45 jours
			
			\item Vérification, au casier judiciaire, de l’absence de condamnation du candidat acquéreur, à
			l’interdiction de l’article 5\degre{} bis de l’article 225-19 du Code pénal (marchands de sommeil)
			Cf. Supra, Section II C
		\end{itemize}
		
	\subsection[Loi << \nom{Carrez} >>]{Les dispositions légales destinées a faire connaitre la surface réelle des lots vendus. (loi 18 décembre 1996 améliorant la protection des acquéreurs de lots de copropriété dite loi « carrez », modifiant l’article 46 de la loi du 10 juillet 2010)}
		
		L’idée maîtresse de cette loi était la nécessité de protéger l’acquéreur d’un lot immobilier, là où les
		dispositions du code civil sont insuffisantes.
		
		L'article 1583 du code civil dispose que la vente est parfaite s'il y a accord sur la chose et sur le prix. Le
		vendeur est tenu de délivrer la contenance telle que portée au contrat (article 1616 du code civil).
		
		Sauf le cas de vente << à la mesure >>, une marge d'un vingtième en plus ou en moins est tolérée. Au-delà,
		une action en réduction de prix est ouverte à l'acquéreur et une action en augmentation de prix est
		ouverte au vendeur (article 1619 du code civil) ; dans cette dernière hypothèse cependant l'acquéreur a
		la ressource de se désister du contrat en remboursant le prix et les frais (article 1620 du code civil). Ces
		actions sont prescrites par un délai d'un an.
		
		Enfin, les parties ont toujours la liberté de stipuler une clause de non garantie, expressément prévue à
		l'article 1619 précité du code civil.
		
		Jusqu’à l’entrée en vigueur de cette loi, la superficie de l'immeuble vendu n'était pratiquement jamais
		mentionnée dans l'acte définitif. Quand bien même la superficie était mentionnée dans l'acte authentique,
		l'acquéreur se heurtait à la clause de non-garantie, devenue systématique : << l'immeuble est vendu dans
		son état actuel, sans garantie de la contenance indiquée, la différence avec celle réelle, même supérieure
		à 1/20\ieme{} devant faire le profit ou la perte de l'acquéreur >>. Cette clause est considérée comme valable par
		une jurisprudence constante.
		
		Cette absence de garantie paraît tout à fait anormale dans les grandes villes où les appartements sont
		vendus à un prix calculé par référence à la valeur au \metreCarre. Iniquité d'autant plus flagrante, qu'à l'opposé, les
		acquéreurs d'appartements neufs sont privilégiés. En effet, les articles L. 26 1 -11 et suivants du code de
		la construction et de l'habitation applicables à la vente en l'état futur d'achèvement dans le secteur
		protégé (habitation ou habitation et professionnel) prévoient la mention de la superficie de l'immeuble
		vendu et dans le contrat préliminaire, et dans le contrat définitif.
		
		Le législateur a donc choisi d’instaurer une protection spécifique de l’acquéreur d’un lot en copropriété
		quant à la surface vendue, et a inséré cette disposition dans la loi sur la copropriété\footnote{La loi ALUR avait ajouté l’obligation d’indiquer en outre la « surface habitable » du lot. Cette ineptie législative a été supprimée par	la loi du \no 2014-1545,du 20 déc. 2014.}.
		
		\paragraph{Article 46 de la loi du 10 juillet 1965}
		
		\begin{quote}
			{\emph Toute promesse unilatérale de vente ou d'achat, tout contrat réalisant ou constatant la vente d'un lot ou
			d'une fraction de lot mentionne la superficie de la partie privative de ce lot ou de cette fraction de lot. La
			nullité de l'acte peut être invoquée sur le fondement de l'absence de toute mention de superficie.
			
			Cette superficie est définie par le décret en Conseil d'État prévu à l'article 47.
			
			Les dispositions du premier alinéa ci-dessus ne sont pas applicables aux caves, garages, emplacements de
			stationnement ni aux lots ou fractions de lots d'une superficie inférieure à un seuil fixé par le décret en
			Conseil d'État prévu à l'article 47.
			
			Le bénéficiaire en cas de promesse de vente, le promettant en cas de promesse d'achat ou l'acquéreur peut
			intenter l'action en nullité, au plus tard à l'expiration d'un délai d'un mois à compter de l'acte authentique
			constatant la réalisation de la vente.
			
			La signature de l’acte authentique constatant la réalisation de la vente mentionnant la superficie de la
			partie privative du lot ou de la fraction de lot entraîne la déchéance du droit à engager ou à poursuivre une
			action en nullité de la promesse ou du contrat qui l’a précédé, fondée sur l’absence de mention de cette
			superficie.
			
			Si la superficie est supérieure à celle exprimée dans l’acte, l’excédent de mesure ne donne lieu à aucun
			supplément de prix.
			
			Si la superficie est inférieure de plus d’un vingtième à celle exprimée dans l’acte, le vendeur, à la demande
			de l’acquéreur, supporte une diminution du prix proportionnelle à la moindre mesure.
			
			L’action en diminution du prix doit être intentée par l’acquéreur dans un délai d’un an à compter de l’acte
			authentique constatant la réalisation de la vente, à peine de déchéance.}
		\end{quote}
	
		La loi utilise le terme de << superficie >> et non celui de << surface >> (par référence à la surface habitable du code
		de la construction et de l'habitation) ou de << contenance >> (par référence à la garantie de contenance du
		code civil) pour harmoniser la loi nouvelle avec les autres dispositions de la loi sur la copropriété.
		
		Cette superficie est définie par le Décret qui a modifié le Décret du 17 mars 1967, articles 4-1 à 4-3,
		s’inspirant largement de la définition du \CCH{} à propos de la SHON.
			\begin{itemize}
				\item La superficie des planchers des locaux clos et couverts après déduction des surfaces occupées par
					les murs, cloisons, marches et cages d'escaliers, gaines, embrasures de portes et de fenêtres.
				\item Il n'est pas tenu compte des planchers des parties des locaux d'une hauteur inférieure à 1,80 mètre.
				\item Les lots ou fractions de lots d'une superficie inférieure à 8 mètres carrés n’étant pas pris en
					compte pour le calcul de la superficie (art. 4-2 du décret).
			\end{itemize}
		
		\subsubsection{Champ d’application : les actes concernés}
		
			\paragraph{La loi ne s’applique qu’aux lots de copropriété et ne s’applique pas en \VEFA{}}
			
			\par Il a été jugé que si par suite de la réunion de tous les lots entre les mains d’une seule personne, il n’y a plus
			de copropriété et dès lors il n’y a plus obligation, lors de la revente de procéder à un mesurage des anciens
			lots au titre de la loi Carrez\footnote{Civ. 3ème Ch. 28 janvier 2009, \no 06-19650}.
			
			Par ailleurs, après nombre de discussions et arrêts contradictoires des cours d’appel, la cour de cassation
			considère que les dispositions de la loi Carrez ne s’appliquent pas à la vente en \VEFA\footnote{3\ieme{} Ch. 11 janvier 2012, \no 10-22.924 au Bulletin}.
			
			Par contre, et à défaut de clause de non garantie insérée à l’acte de vente, les dispositions des articles
			1619 et s. sur la vente à la mesure et la faculté de réduction de prix dans le délai d’un an du jour du contrat
			s’applique à la \VEFA, sous réserve que le point de départ de ce délai de déchéance est la date de remise
			des clés\footnote{3\ieme{} Ch. Civ. 24 nov 1999 JCP 2000, I, 237 ; 3\ieme{} Ch. Civ. 8 oct 2013, Pourvoi \no 12-23275, non publié}
			
			\paragraph{Obligation de mentionner la « superficie » dans l’avant contrat et dans l’acte de vente}
			
			\par Le législateur a tenu à imposer cette mention le plus en amont possible, car elle est de nature à influencer
			le consentement définitif (obligation précontractuelle de renseignement).
			
			La superficie doit être mentionnée dans les avant-contrats. L'article 46 vise expressément les << promesses
			unilatérales de vente ou d'achat >>.
			
			La superficie doit être également mentionnée dans << tout contrat réalisant ou constatant la vente d'un lot >>.
			
			A ce titre, la superficie aurait sans doute dû être mentionnée dans le congé valant offre de vente dans le
			cadre du droit de préemption reconnu au locataire par la loi du 6 juillet 1989, complétée par la loi du 21
			juillet 1994. Toutefois la loi SRU a dispensé le bailleur de cette obligation $\dots$ à titre rétroactif :
			\begin{quote}
				Cf. Art. 15 de la loi du 6 juillet 1989 modifié
			
				« Les dispositions de l'article 46 de la loi no 65-557 du 10 juillet 1965 fixant le statut de la copropriété des	immeubles bâtis ne sont pas applicables au congé fondé sur la décision de vendre le logement. »
			\end{quote}
	
			La même dispense ne s’applique pas à l'offre de vente au locataire en cas de première vente après division
			ou subdivision de l'immeuble, prévue par l'article 10 de la loi du 31 décembre 1975 alors même que la
			vente sera parfaite du seul fait de l'acceptation du locataire
			
			\paragraph{Les modalités de la remise du « certificat carrez »}
			
			\par La mention de la superficie se fait par la remise par le notaire « contre émargement ou récépissé, une copie
			simple de l'acte signé ou un certificat reproduisant la clause de l'acte mentionnant la superficie de la partie
			privative du lot ou de la fraction du lot vendu, ainsi qu'une copie des dispositions de l'article 46 de la loi du
			10 juillet 1965 lorsque ces dispositions ne sont pas reprises intégralement dans l'acte ou le certificat. »
			(art.4-3 D)
			Le vendeur est libre de procéder comme il l'entend au métrage de son lot : il peut le mesurer lui-même.
			En pratique, la surface est « offerte » par l’agent immobilier chargé de la vente du lot ou le propriétaire
			fait le plus souvent appel à un géomètre expert\footnote{
				La cour de cassation (3\ieme Civ. - 21 juin 2006) a précisé que Le mesurage de la superficie de la partie privative d'un lot de copropriété en application de l'article 46 de la loi du 10 juillet 1965, qui est une prestation topographique n'ayant pas pour objet la délimitation des	propriétés, ne relève pas de la compétence exclusive des géomètres experts
			}, un métreur vérificateur ou à un architecte.
			
		\subsubsection{Les lots concernés par l'obligation de métrage}
			
			\paragraph{Exclusion des lots ou fraction de lots de moins de \surface{8}}
			
			\par La loi vise la << vente d'un lot ou d'une fraction de lot >>. En pratique la fraction de lot désigne le lot lui même $\dots$
			
			Un arrêt\footnote{3\ieme{} Ch Civ. 28 janvier 2015 \no 13-26035, au Bulletin} du 28 janvier 2015 apporte une réponse sans ambiguïté à une question à portée générale :
			qu’est-ce qu’un lot ou une fraction de lot inférieure à \surface{8} ? Doit-on désigner par là une partie du lot située à l’extérieur du lot ou toute partie du lot inférieure à \surface{8} ? En l’espèce le lot était désigné comme comportant deux loggias (l’une de \surface{6,27} et l’autre de \surface{6,69}). Ces loggias incluses dans le lot étaient	donc fermées. L’acquéreur affirme que ces loggias étant chacune d’une superficie inférieure à \surface{8}	auraient dû être exclues du calcul de la surface vendue. La cour de cassation répond que « \emph{la cour d’appel ayant constaté que les deux loggias étaient incluses dans le lot et qu’elles étaient closes et habitables en a déduit à bon droit que ces loggias devaient être prises en compte pour le calcul de la superficie des parties	privatives vendues} ».
			
			La superficie du lot doit être indiquée quelle que soit la nature de celui-ci : lot d'habitation, lot
			professionnel ou commercial ou même lot industriel ou $\dots$ transitoire.
			
			\paragraph{Exclusion des lots accessoires}
			
			L'obligation de métrage est exclue pour les lots dits << accessoires >>, c'est-à-dire, selon les termes mêmes
			de la loi, les lots désignés comme : << caves, garages, emplacements de stationnement >> ou les lots ou
			fractions de lots d'une superficie inférieure à 8 \metreCarre{} » (la liste est limitative).

			Cette dispense de métrage s'applique aussi bien lorsque le vendeur cède une cave à titre exclusif, que
			lorsqu'il cède une cave à titre accessoire à son appartement.
			
			Toutefois, le lot « cave » a pu être transformé par son occupant et devenir de la sorte un complément du
			lot principal. En ce cas, et contrairement à ce qu’a pu juger par le passé la Cour de Paris, la superficie de la
			cave ne doit pas être écartée, mais au contraire comprise dans la loi Carrez. C’est ce que décide la cour de
			cassation\footnote{Civ. 3\ieme{} Ch. 2 octobre 2013, \no de pourvoi: 12-21918, au Bulletin} dans un arrêt du 2 octobre 2013 :
			\begin{quote}
				\emph{Ayant exactement retenu que pour l'application de l'article 46 de la loi du 10 juillet 1965, il y avait
				lieu de prendre en compte le bien tel qu'il se présentait matériellement au moment de la vente, la
				cour d'appel, qui, procédant à la recherche prétendument omise, a souverainement estimé que le
				local situé au sous-sol, annexe de la pièce du rez-de-chaussée à laquelle il était directement relié,
				n'était plus une cave comme l'énonçaient le règlement de copropriété et l'acte de vente mais avait
				été aménagé et transformé en réserve, et qui n'était pas tenue de répondre à un moyen inopérant
				relatif au caractère inondable de ce sous-sol, en a déduit à bon droit que ce local devait être pris
				en compte pour le calcul de la superficie des parties privatives vendues.}
			\end{quote}
			
			Il en ira de même si le copropriétaire a transformé ses caves en bureaux, ceci quand bien même aucune
			autorisation de l’Administration n’a été sollicitée\footnote{civ. 3\ieme{} Ch. 7 février 2012, Pourvoi \no 11-11.608, F-D}.
			
			Mais l’inverse n’est pas toujours juste : en l’espèce le propriétaire d’un local d’habitation transforme celui-ci
			en garage pour sa voiture de collection $\dots$ La Cour de Cassation\footnote{Arrêt précité du 7 février 2012} retient, nonobstant cette
			transformation, que la surface du lot doit être incluse dans la superficie de la loi \nom{Carrez} ) au double motif
			:
			\begin{itemize}
				\item de première part que le lot est toujours défini comme un local d'habitation dans le règlement de
				copropriété ;
				\item de seconde part que l'acquéreur peut toujours et à tout moment réaffecter ce lot à usage
				d'habitation.
			\end{itemize}
			
			\paragraph{Les surfaces à prendre en compte}
			
			Doit être mesurée « la superficie privative des planchers des locaux clos et couverts après déduction des
			surfaces occupées par les murs, cloisons, marches et cages d'escaliers, gaines, embrasures de portes et de
			fenêtres ».
			
			Deux hypothèses doivent être envisagées :
			\begin{itemize}
				\item .. Lorsque le lot a été modifié dans sa consistance sans annexion de parties communes (création
				d’une mezzanine par exemple\footnote{
					civ., 3e ch., 13 avril 2005, \no 03-21004 et 03-21015 ; Contra voir cependant les arrêts de la CA de Paris du 20 septembre 2007 et 5
					juin 2012 (RG 10/24914) Sten Mickael X… c/ SARL LOFT, inédit qui considère que la mezzanine n’avait pas d’existence juridique faute
					d’autorisation administrative et d’autorisation de l’assemblée générale).
				}, ou encore transformation d’un abris non clos en local clos et couvert\footnote{civ. 3\ieme{} Ch. 6 septembre 2011 – \no 994 F-D}, il y a lieu de prendre en compte le bien tel qu'il se présentait matériellement au	moment de la vente.
				
				\item Le lot a été modifié par incorporation de parties communes, sans modification de l’état descriptif
				de division : en ce cas il n’y a pas à tenir compte des surfaces irrégulièrement annexées.
			\end{itemize}
			
		\subsubsection{Les sanctions}
			
			\paragraph{La vente ou la promesse de vente est nulle, faute de mention de la superficie}
			
				\subparagraph{S’agissant de la promesse de vente.}
				
				L’absence de mention de la superficie du lot objet de la promesse de vente est sanctionnée par la nullité
				de la promesse de vente, sauf régularisation lors de la signature de l’acte authentique de vente :
				\begin{quote}
					On ne peut rattraper l’omission de la superficie dans la promesse de vente par la signature postérieure
					d’un acte complémentaire « indissociable » de la promesse de vente\footnote{civ. 3\ieme{} Ch. 14 mars 2019, \no 18-10214 – JCP N 2019, \no 13, act. 338.}.
				\end{quote}
				En sorte que l’acquéreur pourra refuser de régulariser l’acte authentique.
				
				\subparagraph{S’agissant de l’acte authentique de vente.}
				
				\par En premier lieu et aux termes de l’alinéa 5 de l’article 46 :
				\begin{quote}
					« \emph{La signature de l'acte authentique constatant la réalisation de la vente et mentionnant la
				superficie entraîne la déchéance du droit à engager ou à poursuivre l'action en nullité de la promesse
				ou du contrat qui l'a précédé, fondée sur l'absence de mention de superficie} »
				\end{quote}
				
				En sorte que si l’acquéreur, nonobstant l’omission de la superficie dans la promesse de vente, accepte de
				signer l’acte authentique avec mention de la superficie, il perd le droit d’agir en annulation.
				
				\bigskip
				En second lieu, si la mention est omise dans l’acte authentique, l’acquéreur pourra demander au juge
				d’annuler la vente (ceci quand bien même la superficie aurait-elle figuré dans la promesse de vente).
				
				Le délai d'action est d'un mois au plus à compter de << l'acte authentique constatant la réalisation de la
				vente >>, ceci afin de limiter le contentieux et la durée de fragilité du contrat. Ce délai est certainement un
				délai préfixe, seulement susceptible d'interruption par voie d’assignation en justice (articles 2242 à 2250
				du code civil).
				
				Le but de cette règle est de limiter le contentieux en faisant jouer au notaire un rôle de << filtre >> : il ne
				laissera pas passer un acte sans mention de superficie.
			
			\paragraph{Sanction de l'erreur sur la superficie dans l'acte authentique : l'action en	réduction de prix}
			
				\begin{quote}
					<< \emph{Si la superficie est inférieure de plus de 1/20\ieme{} à celle exprimée dans l'acte, le vendeur, à
					la demande de l'acquéreur, supporte une diminution de prix proportionnelle à la moindre	mesure.} »
				\end{quote}
				
				Il n'existe pas de sanction si la superficie réelle est supérieure à celle mentionnée dans l'acte authentique,
				contrairement aux dispositions du code civil.
				
				L'action en réduction de prix n'est ouverte que si la différence entre la superficie mentionnée et la
				superficie réelle est supérieure à 1/20\ieme, marge de tolérance reprise du code civil. En effet, il est souhaitable	que l'action ne soit ouverte que dans les cas où l'acquéreur supporte une perte financière substantielle. D'autre part, cette tolérance permet un métrage par le vendeur lui-même.
				
				Pour déterminer les conditions d'ouverture de l'action, il faut calculer cette différence en nature et non
				en valeur, contrairement au code civil\footnote{
					Soit un appartement acheté \montant{300 000} pour une superficie de 100 \metreCarre{} et en conséquence un prix du \metreCarre{} de \montant{30 000}. Si la superficie
					réelle est seulement de 92 \metreCarre, donc inférieure de plus de \pourcent{5} à la surface mentionnée à l’acte de vente, le vendeur sera débiteur de 100-
					92=8\metreCarre{} x \prixSurface{30 000} = \montant{240 000}
				}.
				
				Le délai d'action est d'un an à compter de l'acte authentique constatant la réalisation de la vente, à peine
				de déchéance. C'est une transposition des solutions de droit commun. Il s'agit donc là encore d'un délai
				préfix.
				
				L’action est ouverte à l’acquéreur quand bien même aurait-il acquis en sachant que la surface portée à
				l’acte était supérieure à la surface Carrez : la bonne ou la mauvaise foi n’a pas à être recherchée dans une
				action en réduction du prix\footnote{Cour de Cass. Ch. civ 3 - 10 décembre 2015 Numéro de pourvoi 14-13.832 Non publié au Bulletin (\no 1403 FS-PB)}.
				
				La cour de cassation impose au juge du fond de caractériser la nature des surfaces déduites, en application
				de l'art. 4-1 du décret \no 67-223 du 17 mars 1967 ; il ne peut donc se contenter de la certification de
				superficie établie par le cabinet d'expertise faisant apparaître que la superficie au sens de l'art. 46 telle
				que définie par l'art. 1er du décret \no 97-532 du 23 mai 1997 est inférieure de plus d'un vingtième à celle
				exprimée dans l'acte\footnote{Civ 3\ieme{}, 7 nov 2001 – Dalloz 2003 somm p. 1332}.
			
			\paragraph{Exclusion de l’action en garantie des vices cachés}
			
			Quand bien même les parties communes extérieures aux lots vendus ne doivent pas entrer dans le calcul
			de la loi Carrez, il arrive que le « technicien » tienne compte des surfaces « absorbées » dans son calcul.
			L’acquéreur peut-il alors prétendre avoir un titre sur ces parties communes ainsi privatisées ? La cour de
			cassation\footnote{Civ. 3\ieme{} Ch. 8 oct. 2013, \no de pourvoi: 12-19854 , non publié} donne une réponse dans les relations entre vendeur et acquéreur, l’acquéreur s’étant
			retourné vers le vendeur sur la base du vice caché au motif que celui-ci lui avait dissimulé que les surfaces
			absorbées (paliers et escalier) n’étaient pas privatives : « Attendu qu'ayant, par motifs adoptés, retenu
			que le fait que le rez-de-chaussée, les paliers, et les escaliers intérieurs constituaient des parties
			communes n'avait pas été caché à M. X... puisque l'acte authentique ne les mentionnait pas
			comme lots vendus, la cour d'appel procédant à la recherche prétendument omise a légalement
			justifié sa décision de rejeter la demande fondée sur la garantie des vices cachés »
			
			D) SORT DES FRAIS ACCESSOIRES.
			
			L’acquéreur a supporté un prix supérieur au prix qu’il aurait dû payer. Il a en conséquence payé des droits
			et émoluments supérieurs à ce qu’il aurait dû payer dès lors que ces droits et émoluments ont pour
			assiette le prix de vente du lot. Peut-il demander au vendeur le remboursement de ces frais
			complémentaires ? La cour de cassation répond négativement\footnote{
				Civ.3ème - 22 novembre 2006 : « Viole les dispositions de l’article 46, alinéa 7, de la loi du 10 juillet 1965 une cour d’appel qui, sur ce fondement, condamne le vendeur d’un appartement dans un immeuble en copropriété à payer à l’acquéreur le montant de frais afférents au surplus indu du prix de vente >>
			} :
			
		\subsubsection{Le recours en garantie contre les professionnels}
			
			L’action de la loi Carrez est une action en restitution du prix. Dès lors, même si l’erreur de calcul est
			imputable à un professionnel, le vendeur ne saurait lui demander garantie des sommes qu’il a dû
			rembourser à l’acquéreur.
			
			« La restitution à laquelle le vendeur est condamné à la suite de la diminution du prix prévue par l’article
			46, alinéa 7, de loi du 10 juillet 1965, résultant de la délivrance d’une moindre mesure par rapport à la
			superficie convenue, ne constitue pas un préjudice indemnisable. Elle ne peut, dès lors, donner lieu à
			garantie de la part du professionnel de mesurage »\footnote{3e Civ. - 25 octobre 2006}.
			
			La diminution du prix de vente dû à la méconnaissance des dispositions de la loi Carrez ne constitue pas
			un préjudice indemnisable par le notaire.
			Cependant, celui-ci peut être tenu à indemniser l’acquéreur en tant que débiteur subsidiaire lorsque le
			vendeur est insolvable\footnote{Cass. Civ. 1e 25 mars 2010}.
			S’agissant du technicien auquel avait été simplement confié la mission de déterminer la superficie privative
			du lot objet de la vente, en l’occurrence une société spécialisée en diagnostics, on citera un arrêt\footnote{Civ. 3\ieme{} Ch. 18 septembre 2013 \no 12-24077 inédit} du 18 septembre 2013 qui casse un arrêt de la cour de Bordeaux ayant condamné ce technicien à indemniser le
			vendeur pour avoir pris en compte la superficie de locaux d'un immeuble attenant ne faisant pas partie de
			la copropriété : « la société X n'était pas tenue, dans le cadre de la mission de mesurage qui lui avait été
			confiée, de procéder à l'analyse juridique du lot objet de la vente, la cour d'appel a violé les textes
			susvisés ».
			
			A) UNE PREMIERE EVOLUTION JURISPRUDENTIELLE : VERS L’INDEMNISATION D’UNE PERTE DE	CHANCE
			
			Un arrêt de la cour de cassation\footnote{Civ 3\ieme{} Ch 2 juillet 2014, pourvoi: 12-26619, Non publié} du 2 juillet 2014, bien que non publié au Bulletin, paraît apporter
			un assouplissement à sa position antérieure quant à l’indemnisation du vendeur tenu à restitution
			du « surplus du prix » par suite d’une erreur de calcul de surface par le professionnel :
			Mais attendu qu'ayant retenu que si la preuve d'une faute de la société François C... lors des opérations
			de calcul de la superficie des appartements vendus se trouvait rapportée, il n'existait aucun lien de
			causalité direct entre cette faute et le préjudice invoqué par les consorts Y...- X..., la cour d'appel, qui n'a
			pas modifié l'objet du litige et qui a statué sur les dernières conclusions déposées en retenant que les
			consorts Y...- X... demandaient l'indemnisation d'une « perte de surface » et non pas d'une perte de
			chance, a pu en déduire, abstraction faite d'un motif erroné mais surabondant tenant au fait que la perte
			de surface alléguée avait pour cause exclusive le défaut d'exercice de l'action prévue par l'article 46 de
			la loi du 10 juillet 1965, que la demande en dommages-intérêts des consorts Y...- X... à l'égard de la
			société François C... devait être rejetée ;
			Mais il est vrai, dans le cas analysé, que d’une part l’action n’avait pas été introduite par le vendeur mais
			par l’acquéreur, et que d’autre part l’acquéreur n’avait pas agi en restitution de prix contre le vendeur
			dans le délai d’un an.
			
			B) LA CONSECRATION PAR LA COUR DE CASSATION DU DROIT A INDEMNITE POUR PERTE DE
			CHANCE..
			
			Un arrêt\footnote{chambre civile 3 , 28 janvier 2015 , \no de pourvoi: 13-27397, Publié au bulletin} du 28 janvier 2015, publié au Bulletin, nous apprend qu’en définitive tout est question de
			rédaction de la demande.
			En l’espèce le vendeur n’a pas demandé condamnation du professionnel (le métreur) à restitution de la
			différence de prix … mais à Dommages et Intérêts pour perte de chance de vendre le bien au même prix !
			La Cour d’Appel ayant fait droit à la demande du vendeur sur cette base de la perte de chance, ce dernier
			régularise un pourvoi … rejeté par la cour de cassation !
			« Mais attendu qu'ayant retenu, à bon droit, que, si la restitution, à laquelle le vendeur est tenu en
			vertu de la loi à la suite de la diminution du prix résultant d'une moindre mesure par rapport à la
			superficie convenue, ne constitue pas, par elle-même, un préjudice indemnisable permettant une
			action en garantie, le vendeur peut se prévaloir à l'encontre du mesureur ayant réalisé un mesurage
			erroné, d'une perte de chance de vendre son bien au même prix pour une surface moindre, la cour
			d'appel a souverainement apprécié l'étendue du préjudice subi par Mme X.. »
			
			Certes les Tribunaux considèrent habituellement qu’une perte de chance ne peut permettre une
			indemnisation totale (sauf à considérer que la « chance » était quasiment certaine). Mais récupérer –-- ne
			serait-ce que \pourcent{50} --- de la différence de prix ce n’est déjà pas trop mal $\dots$ et en fait sous couvert d’une qualification juridique distincte, nous sommes en présence d’un véritable revirement de jurisprudence.
	
\section{Les formalités lors de cession du lot}

	Il convient que le syndic soit informé du changement du titulaire du lot afin qu'il puisse s'adresser à
	l'acquéreur en tant que copropriétaire dans les différents actes de la vie de la copropriété. Par ailleurs, le
	syndic devra être avisé de la vente pour tenter de récupérer sur le prix les sommes restant dues au syndicat
	par le vendeur.
	
	Enfin, cédant et cessionnaire doivent être renseignés exactement sur les dépenses déjà engagées par le
	syndicat, mais qui ne constituent pas encore des créances liquides et exigibles : cette information
	permettra d'établir un juste prix.
	A. NOTIFICATION AU SYNDIC DE TOUT TRANSFERT DE PROPRIETE D'UN LOT OU D'UNE
	FRACTION DE LOT (ARTICLE 6 AL 1 ET 3 DU DECRET DU 17 MARS 1967).
	Le syndic doit impérativement être informé de tout mutation (à titre onéreux ou gratuit, particulier ou
	universel) afin qu’il connaisse l'identité du nouveau copropriétaire, le convoque aux assemblées et appelle
	les charges auprès de lui. C’est la fonction de la « notification » de l’article 6 du décret.
	" Tout transfert de propriété d'un lot ou d'une fraction de lot, toute constitution sur ces derniers d'un droit
	d'usufruit, de nue propriété, d'usage ou d'habitation, tout transfert de l'un de ces droits est notifié, sans
	délai, au syndic, soit par les parties, soit par le notaire qui établit l'acte, soit par l'avocat ou soit par l’avoué
	qui a obtenu la décision judiciaire, acte ou décision qui suivant le cas, réalise, atteste, constate ce transfert
	ou cette constitution.
	Cette notification doit être faite indépendamment de l'avis de mutation prévu à l'article 20 de la loi du 10
	juillet 1965 modifiée ".
	1. Contenu (art. 6 alinéa 2) :
	droit de la copropriété année 2018-2019
	194
	"Cette notification comporte la désignation du lot ou de la fraction de lot ainsi que l'indication des nom,
	prénom, domicile réel ou élu de l'acquéreur et, le cas échéant, du mandataire commun prévu à l'article 23
	de la loi."
	Elle est faite sous forme de lettre recommandée avec avis de réception ou par télécopie avec récépissé
	depuis le 1er avril 2007(s’agissant d’une notification, il y a lieu d’rappliquer les dispositions de l’article 64
	du Décret).
	S'agissant du "transfert de propriété", c'est en principe l'acte sous seing privé qui a cet effet translatif,
	c'est-à-dire qu'il faudrait notifier le compromis209 En pratique cependant, la notification n'est souvent faite
	qu'après la passation de l'acte authentique : les actes sous seing privé sont le plus souvent assortis de
	conditions suspensives qui ne seront définitivement levées qu’au jour de la signature de l’acte
	authentique.
	2. Conséquences de la notification.
	Seule la notification d’une mutation opérée selon le formalisme prévu par l’article 6 du décret de 1967
	rend cette mutation opposable au syndicat des copropriétaires quand bien même le syndic aurait eu
	connaissance de la vente par d’autres moyens210
	Le syndic pourra mettre à jour la liste des copropriétaires qui devra désormais comprendre l'acquéreur
	tant pour la gestion documentaire et comptable que pour les convocations aux assemblées.
	Art. 32. – Le syndic établit et tient à jour une liste de tous les copropriétaires avec l'indication des lots qui
	leur appartiennent, ainsi que de tous les titulaires des droits visés à l'article 6 ci-dessus ; il mentionne leur
	état civil ainsi que leur domicile réel élu.
	A défaut de notification de la vente, le vendeur est toujours copropriétaire au regard du syndicat des
	copropriétaires, en sorte qu’il sera valablement convoqué aux assemblées générales et tenu au paiement
	des charges.
	B. L’AVIS DE MUTATION DONNE AU SYNDIC EN VUE DE FORMER OPPOSITION AU
	VERSEMENT DES FONDS (ARTICLE 20 DE LA LOI DU 10 JUILLET 1965)
	209 Aix, 4ème ch., 23 mars 1983, Bull. Aix Janvier 1983 \no30 p.46.
	210 Civ.3ème 22 mars 2000, Civ.3ème 26 sept 2007
	droit de la copropriété année 2019-2020
	195
	"Lors de la mutation à titre onéreux d'un lot, et si le vendeur n'a pas présenté au notaire un certificat du
	syndic ayant moins d'un mois de date, attestant qu'il est libre de toute obligation à l'égard du syndicat, avis
	de la mutation doit être donné par le notaire au syndic de l'immeuble, par lettre recommandée avec avis
	de réception. Avant l'expiration d'un délai de quinze jours à compter de la réception de cet avis, le syndic
	peut former, au domicile élu, par acte extrajudiciaire, opposition au versement des fonds, dans la limite ciaprès
	pour obtenir le paiement des sommes restant dues par l'ancien propriétaire. Cette opposition
	contient élection de domicile dans le ressort du tribunal de grande instance de la situation de l'immeuble.
	Et, à peine de nullité, énonce le montant et les causes de la créance. Les effets de l'opposition sont limités
	au montant ainsi énoncé.
	Tout paiement ou transfert amiable ou judiciaire du prix opéré en violation des dispositions de l'alinéa
	précédent est inopposable au syndic ayant régulièrement fait opposition.
	L'opposition régulière vaut au profit du syndicat mise en oeuvre du privilège mentionné à l'article 19-1 ".
	L'avis de mutation prévu par l'article 20 de la loi a une finalité très précise. Il doit permettre au syndic,
	avant que le prix de vente ne soit versé au vendeur, de paralyser un tel versement afin d'obtenir paiement
	des sommes restant dues au syndicat par le cédant.
	Il ne concerne donc que les mutations à titre onéreux, par opposition à la notification de l’article 6 du
	décret.
	1. Le processus de l'article 20 de la loi.
	Cet avis et l'opposition éventuelle qu'il peut susciter s'inscrivent dans un processus comportant plusieurs
	phases et alternatives.
	1. Le vendeur doit présenter au notaire un certificat ayant moins d'un mois de date, attestant
	qu'il est libre de toute obligation à l'égard du syndicat. Si tel est le cas, l'acte de vente est
	passé sans autres formalités. En pratique, c'est le notaire qui demande le certificat au
	syndic211.
	211 Cour d’Appel de PARIS – 27 mars 2002 : cet arrêt a rappelé que l’avis doit être envoyé au syndic avant la signature
	de l’acte de vente et en tout état de cause avant la remise du prix au vendeur au risque de remettre des sommes indues
	au vendeur débiteur de la copropriété.
	droit de la copropriété année 2018-2019
	196
	2. A défaut pour le vendeur de pouvoir présenter ce certificat, avis de mutation doit être
	donné au syndic, par lettre recommandée à la diligence du notaire (le texte antérieur à la
	réforme du 21 juillet 1994 disait "à la requête de l'acquéreur").
	3. La réception de cet avis fait courir un délai de quinze jours (au paravent il n'était que de
	huit jours, délai très court qui explique la modification légale), au cours duquel le syndic
	peut, par acte d'huissier, former opposition au versement des fonds pour obtenir le
	paiement des sommes liquides et exigibles restant dues par l'ancien propriétaire. Cette
	opposition devra mentionner, à peine de nullité, le montant et les causes de la créance.
	2. Effets de l’opposition
	Initialement, cette opposition rendait indisponible la totalité du prix de vente entre les mains de
	l'acquéreur ou, plus précisément, du notaire ; si bien que tout paiement était en principe inopposable au
	syndicat. La loi du 21 juillet 1994 limite le "blocage" du prix au seul montant de l'opposition pratiquée par
	le syndic.
	Depuis cette même loi, l'opposition régulièrement pratiquée confère un véritable privilège au Syndicat des
	Copropriétaires, privilège plaçant le syndicat en tête de tous les créanciers sur les immeubles, juste après
	les salariés et les frais de justice qui bénéficient d'un privilège général (sur les meubles et immeubles)..
	L'article 5-1 du Décret d'Application de la loi exige que l'opposition du syndic énonce de manière précise
	le montant et les causes de la créance, en distinguant les sommes dues selon les différents rangs de sûreté
	dont le syndicat est susceptible de bénéficier :
	- Créances du syndicat afférentes aux charges et travaux de l'année courante et des deux
	dernières années échues.
	- Créances du syndicat afférentes aux deux années antérieures aux deux dernières années
	échues.
	- Créances de toute nature du syndicat garanties par une hypothèque légale et non comprises
	dans les créances privilégiées ci-dessus. (En pratique cela devrait concerner les hypothèques
	légales prises pour la période remontant à plus de quatre ans).
	- Créances de toute nature non comprises dans les créances ci-dessus.
	3. Auteur de l’opposition
	L’article 5-1 du Décret précise :
	" Si le lot fait l'objet d'une vente devant le tribunal sur licitation ou sur saisie immobilière, l'avis de mutation
	prévu à l'article 20 est donné au syndic par le notaire ou l'avocat du demandeur ou du créancier
	poursuivant; s'il fait l'objet d'une expropriation pour cause d'utilité publique ou de l'exercice d'un droit de
	préemption, l'avis de mutation est donné au syndic par l'autorité expropriante, le titulaire du droit de
	droit de la copropriété année 2019-2020
	197
	préemption ou le notaire. Si l'acte est reçu en la forme administrative, l'avis de mutation est donné au
	syndic par l'autorité qui authentifie la convention
	C. LES INFORMATIONS A FOURNIR A L’ACHETEUR LORS DE LA CESSION
	De nouvelles formalités sont imposées lors de la signature de l’acte authentique de vente : la dénonciation
	du règlement de copropriété pour son opposabilité à l’acquéreur (I), la mise à jour des informations
	financières via l’état daté, permettant de faire les comptes entre le syndicat de copropriété et le vendeur
	(2).
	1. Dénonciation du règlement de copropriété et de l'état descriptif de division
	à l'acquéreur
	Si, comme il est de règle, le règlement de copropriété et l'état descriptif de division ont fait l'objet d'une
	publication au fichier immobilier, leur opposabilité à l'égard du nouveau propriétaire est automatique à
	dater de cette publication.
	L'article 4 du décret impose néanmoins pour plus de sûreté que l'acte mentionne expressément que
	l'acquéreur a eu préalablement connaissance du règlement, de ses modificatifs ainsi que de l'état
	descriptif et les actes qui l'ont modifié. Mais l'omission de cette mention n'est assortie d'aucune sanction
	puisque l'opposabilité de ces documents résulte de la publication. Simplement, si l'acquéreur subit un
	préjudice du fait de cette opposabilité, il peut mettre en jeu la responsabilité du rédacteur de l'acte (agent
	immobilier, notaire, avocat) qui a omis la mention exigée212
	Si le règlement et l'état descriptif n'ont pas été publiés, ils ne s'imposeront à l'acquéreur que s'il est
	expressément constaté dans l'acte de cession qu'il en a eu préalablement connaissance et qu'il a adhéré
	aux obligations qui en résultent (art.4 al.3 D.17 mars 1967).
	Si cette double mention n'est pas portée à l'acte, on doit alors considérer que les documents en question
	ne sont pas opposables à l'acquéreur213 .
	Dans une certaine mesure, l’existence de l’état descriptif de division est une condition nécessaire à la
	détermination de l’objet de la vente (identification de l’appartement et des quote parts de parties
	communes), donc à la validité même de la vente. Toutefois, la jurisprudence louvoie sur le sujet :
	212 GIVORD et GIVERDON \no193
	213 GIVORD et GIVERDON \no193 in fine
	droit de la copropriété année 2018-2019
	198
	La vente d’un lot de copropriété n’est pas réalisée, en l’absence de la détermination de la quote-part de
	parties communes afférente au lot qui constituait un élément essentiel de la convention : en l’absence de
	détermination suffisante de l’objet de la vente, celle-ci n’est pas parfaite214.
	La vente (de combles) est parfaite entre les parties dès lors que l’objet de la vente est déterminable, même
	en l’absence de décision de l’assemblée générale devenue définitive approuvant l’état descriptif de
	division créant le nouveau lot et lui attribuant des millièmes de parties communes215.
	2. L’état daté (art. 5 du décret \no67-223 du 17 mars 1967, en sa rédaction issue
	du décret du 27 mars 2004)
	Avant la réforme du 21 juillet 1994, l'application de cet article se confondait avec la mise en oeuvre des
	dispositions de l'article 20 de la loi (avis de mutation à titre onéreux). Désormais les choses sont
	parfaitement distinctes : l'article 5 nouveau du Décret s'applique dans tous les cas de mutation ou de
	constitution de droit réel sur le lot; qu'il y ait vente ou non.
	Avant l'établissement d'un acte réalisant ou constatant le transfert de propriété d'un lot ou d'une fraction
	de lot, ou la constitution sur ces derniers d'un droit réel, le syndic doit adresser au notaire chargé de
	recevoir l'acte, à la demande de ce dernier ou à celle du copropriétaire, un état daté qui en vue de
	l'information des parties comporte trois parties, avec des objets distincts :
	- Première partie : Déterminer les dettes du copropriétaire vendeur envers le syndicat.
	- Deuxième partie : Déterminer les dettes du syndicat envers le copropriétaire vendeur.
	- Troisième partie : Déterminer les sommes qui devront incomber à l’acquéreur.
	a. 1ère partie : les dettes du copropriétaire envers le Syndicat
	1\degre{} Dans la première partie, le syndic indique, d'une manière même approximative et sous réserve de
	l'apurement des comptes, les sommes pouvant rester dues, pour le lot considéré, au syndicat par le
	copropriétaire cédant, au titre :
	a) Des provisions exigibles du budget prévisionnel ;
	b) Des provisions exigibles des dépenses non comprises dans le budget prévisionnel ;
	c) Des charges impayées sur les exercices antérieurs ;
	d) Des sommes mentionnées à l'article 33 de la loi du 10 juillet 1965 ;
	e) Des avances exigibles.
	Ces indications sont communiquées par le syndic au notaire ou au propriétaire cédant, à charge pour eux
	de les porter à la connaissance, le cas échéant, des créanciers inscrits.
	214 Cass. Civ. 3e 11 février 2009
	215 Cass. Civ. 3e 10 septembre 2008
	droit de la copropriété année 2019-2020
	199
	b. 2ème partie : les dettes du Syndicat des Copropriétaires envers le
	copropriétaire vendeur
	2\degre{} Dans la deuxième partie, le syndic indique, d'une manière même approximative et sous réserve de
	l'apurement des comptes, les sommes dont le syndicat pourrait être débiteur, pour le lot considéré, à
	l'égard du copropriétaire cédant, au titre :
	a) Des avances mentionnées à l'article 45-1 ;
	b) Des provisions du budget prévisionnel pour les périodes postérieures à la période en cours et rendues
	exigibles en raison de la déchéance du terme prévue par l'article 19-2 de la loi du 10 juillet 1965.
	c. Troisième partie : les sommes qui devront incomber à l’acquéreur
	3\degre{} Dans la troisième partie, le syndic indique les sommes qui devraient incomber au nouveau copropriétaire,
	pour le lot considéré, au titre :
	a) De la reconstitution des avances mentionnées à l'article 45-1 et ce d'une manière même approximative
	b) Des provisions non encore exigibles du budget prévisionnel ;
	c) Des provisions non encore exigibles dans les dépenses non comprises dans le budget prévisionnel.
	Dans une annexe à la troisième partie de l'état daté, le syndic indique la somme correspondant, pour les
	deux exercices précédents, à la quote-part afférente au lot considéré dans le budget prévisionnel et dans
	le total des dépenses hors budget prévisionnel. Il mentionne, s'il y a lieu, l'objet et l'état des procédures en
	cours dans lesquelles le syndicat est partie.
	Cet état daté fait l’objet d’un formulaire qui a été rédigé en concertation avec le conseil supérieur du
	Notariat et les différents syndicats représentatifs de la profession de syndic.
	Ainsi informées, les parties pourront prendre en compte ces éléments pour évaluer le prix de vente du lot,
	et éventuellement décider -dans leurs relations - ce que l'acquéreur remboursera au vendeur ou ce que le
	vendeur s'engage à supporter (par exemple au titre de travaux non encore exécutés et dont les charges
	ne sont pas encore exigibles).
	Etant bien entendu que les conventions entre vendeur et acquéreur ne sont aucunement opposables à la
	copropriété qui a l’obligation de faire application des dispositions des articles 6-2 et 6-3 quant à la
	répartition des dettes et créances entre vendeur et acquéreur.
	Ainsi, le Décret organise en réalité la bonne application d'un principe général du droit contractuel en
	général et plus particulièrement du droit de la vente d'immeuble : le vendeur d'un immeuble - donc d'un
	droit de la copropriété année 2018-2019
	200
	lot de copropriété - doit informer son acquéreur, ne serait-ce que parce que l'acquéreur doit bénéficier
	d'une information loyale et complète pour donner un accord en connaissance de cause216
	D. LES BASES IMMOBILIERES DU NOTARIAT
	Le Décret \no 2013-803, complété par deux arrêtés du 30 septembre 2016 oblige les notaires, à compter du
	1er janvier 2017, à alimenter deux bases de données distinctes tenues par le Conseil Supérieur du Notariat
	:
	- Une base des avant-contrats
	- Une base des actes authentiques
	Les renseignements ainsi transmis (dans les 60 jours de leur signature pour les actes authentiques et dans
	les 30 jours de leur signature ou de leur remise au notaire pour les avant-contrats) permettent au C.S.N.
	d’établir des tableaux de résultats statistiques pour l’ensemble des mutations portant sur une période
	d’un ou plusieurs trimestres consécutifs, relevées dans un cadre territorial de référence (unité urbaine
	d’un arrondissement de la Ville de Paris, par exemple) portant sur au moins une vingtaine de biens. Ces
	données statistiques devant être transmises à toute personne en faisant la demande : ces renseignements
	précisant le prix et les caractéristiques essentielles de chaque bien217.

\section{Les effets de la cession}

	Il ne sera pas fait état ici des effets généraux de la vente : le vendeur est tenu des obligations de délivrance
	et de garantie, l'acheteur de celle de payer le prix.
	Mais la vente a aussi des effets concernant le syndicat : d’une part, il modifie la composition de celui-ci (I)
	et d’autre part l’acquéreur devient débiteur des charges de copropriété à compter de la notification de la
	mutation (II)
	A. EFFETS QUANT A LA COMPOSITION DU SYNDICAT.
	L'acquéreur se substitue au vendeur en tant que copropriétaire. Il devient titulaire des droits du cédant en
	tant que membre du syndicat : participation aux assemblées générales, droit d'en contester les décisions,
	etc...
	Plusieurs questions se posent à l’occasion de la cession d’un lot :
	216 cf. L'intéressante étude Madame Muriel Fabre-Magnan intitulé : De l'Obligation d'information dans les Contrats aux éditions L.G.D.J.
	1992.
	217 Cf. Rapport \no 2621 sur le proje- de loi de modernisation des professions judiciaires et juridiques réglementées : « La diffusion d’une
	information pertinente sur l’évolution du marché immobilier est de nature à la fluidifier et donc, à favoriser l’accès à la propriété de nos
	concitoyens » JCPN 2016 \no 40, act. 1072
	droit de la copropriété année 2019-2020
	201
	1. Si le cédant n'a pas informé le syndic de la cession
	Bien évidemment, s'agissant d'une mutation à titre onéreux, le mécanisme de l'article 20 permet de penser
	qu'il n'y aura pas de difficulté à ce que le syndic soit immédiatement informé de la cession. Par contre dans
	tous les autres cas, le syndic n'est informé que par la mise en oeuvre des dispositions de l'article 6 du
	Décret.
	Or, ces dispositions sont parfois ignorées par les personnes chargées d'établir les actes constatant la
	mutation : en cas de succession, trop fréquemment le notaire oublie d’informer le syndic; de même en cas
	de démembrement de la propriété du lot (nue-propriété et usufruit; vente en crédit-bail ...).
	En ce cas il a été jugé à maintes reprises que le syndic convoquait régulièrement à l'Assemblée l'ancien
	propriétaire du lot218 : c’est en effet la notification de la mutation qui, selon l’expression de MM LAFOND
	et STEMMER “ confère à l’acquéreur la qualité de copropriétaire à l’égard tant du syndicat que des autres
	copropriétaires ”219
	Toutefois, la Cour de Paris220 a estimé que la notification de l’article 6 du Décret n’ayant pas la même
	finalité que l’article 20 de la loi, le syndic doit tenir compte d’une mutation portée à sa connaissance par
	lettre simple et non par lettre recommandée alors qu’il a bien reçu cette lettre simple.
	2. Si la cession est notifiée entre la convocation de l’assemblée générale et la
	tenue de celle-ci
	Si le syndic n'a pas été avisé de la mutation, celle-ci est inopposable au syndicat. Le syndic peut alors
	valablement continuer à s'adresser à l'ancien copropriétaire pour la convocation aux assemblées générales
	ou le recouvrement des charges221.
	Si la notification de la vente lui est faite plus de quinze jours avant la tenue de l'assemblée générale le
	syndic devra convoquer l’acquéreur, même si le vendeur avait été précédemment convoqué.
	218 Civ 3\ieme{} 6 nov 1991, Administrer juillet 1992 p. 32
	219 Cour d’Appel de PARIS – 28 mars 2002 : la cour a rappelé que le critère de l’opposabilité du transfert du lot au syndic
	n’était pas la publication de l’acte de transfert mais sa notification au syndic.
	220 Paris 23\ieme{} Chambre 13 mars 1991 Dalloz 1992, Somm p 132.
	221 Civ 3\ieme{}, 21 juin 1995 et Versailles 22 mars 2004, Loyers et Copropriété oct 2004 \no 170 : « Le syndic convoque
	valablement le vendeur à l'assemblée générale dès lors qu’à la date d’envoi de la convocation il n’avait pas reçu la
	notification du transfert de propriété ».
	droit de la copropriété année 2018-2019
	202
	3. La contestation de l’assemblée générale par l’acquéreur.
	C'est le propriétaire qui doit exercer l'action en contestation.
	LE CEDANT N’A PLUS QUALITE POUR DEMANDER L’ANNULATION D’UNE ASSEMBLEE GENERALE.
	Cependant, le vendeur peut conserver intérêt à contester l'Assemblée Générale. En conséquence, s'il était
	encore copropriétaire lors de l'Assemblée, et sous réserve qu'il justifie d'un intérêt à agir, il pourra
	contester les décisions de l'Assemblée Générale alors même qu'il aura vendu son lot depuis la tenue de
	l'assemblée.
	LE CESSIONNAIRE DU LOT NE PEUT AGIR EN NULLITE D'UNE DELIBERATION D'ASSEMBLEE VOTEE
	ALORS QU’IL N’AVAIT PAS ENCORE ACQUIS LA PROPRIETE DU LOT.
	L'acquéreur ne peut contester une décision d'Assemblée tenue avant que la cession soit devenue effective,
	même s'il représentait son vendeur lors de cette assemblée. Cependant, doit être considéré comme
	valablement poursuivie par l’acquéreur l’action intentée par le vendeur en contestation d'une délibération
	de l'assemblée générale alors qu’avait été précisé à l’acte d’acquisition qu'il reprendrait la procédure222
	LE CESSIONNAIRE PEUT SOLLICITER L’ANNULATION DE L’ASSEMBLEE GENERALE SI LA
	NOTIFICATION DE LA MUTATION EST INTERVENUE AVANT SA TENUE
	Si le transfert a été notifié au syndic plus de trois semaines avant la tenue de l'assemblée générale, et que
	le syndic n’a pas re-convoqué l’acquéreur, celui-ci pourra demander l’annulation pour défaut de
	convocation223.
	Si l'acquéreur n'a pas été convoqué, sans qu'il y ait eu faute du syndic, il pourra contester l'Assemblée pour
	tout motif autre que l'absence de convocation régulière.
	Si le procès verbal de l’Assemblée n’est pas notifié à l’acquéreur, celui-ci pourra contester l’assemblée
	générale pendant un délai de 10 ans !
	B. EFFETS DE LA VENTE SUR LE PAIEMENT DES CHARGES.
	222 Cour d'appel de Paris, 19e ch. B, 12 octobre 1995 Recueil Dalloz 1996, Somm. p. 91
	223 Civ 3\ieme{} 22 juin 1994, JCP N 1994, II, p. 336.
	droit de la copropriété année 2019-2020
	203
	Un point essentiel est de déterminer quelles sont les charges et dettes qui restent dues par le cédant et
	celles qui incombent au cessionnaire.
	1. Suspension des effets de la vente à la notification du transfert de propriété.
	En premier lieu, comme rappelé précédemment, tant que la notification de la vente (article 6 du Décret)
	n’a pas été faite au syndic, le vendeur demeure débiteur des charges de copropriété224 :
	« Tant que la notification au syndic du transfert de propriété d'un lot ou d'une fraction de lot n'a pas été
	opérée en application de l'art. 6 du décret \no 67-223 du 17 mars 1967, le transfert de propriété est
	inopposable au syndic qui peut valablement recouvrer auprès du vendeur les charges dues par ce dernier
	sans tenir compte de la vente intervenue ».
	D'après les textes (art.5 du décret de 1967), le cédant est débiteur des charges et sommes devenues
	liquides et exigibles avant qu'il ne perde sa qualité de propriétaire. Le cessionnaire sera tenu de celles qui
	sont devenues liquides et exigibles après qu'il ait acquis la qualité de copropriétaire.
	A l'égard du syndicat, le moment à prendre en considération est celui de la notification du transfert au
	syndic (notification dont il a été dit précédemment qu’elle est exigée par les dispositions de l’article 6 du
	décret de 1967).
	La Cour de Cassation a affirmé que le syndicat des copropriétaires, qui oppose à l'acquéreur
	l'inopposabilité du transfert de propriété intervenu à défaut de notification de la mutation, ne peut lui
	réclamer le paiement des charges de copropriété225
	2. Exigibilité des charges au regard de la jurisprudence antérieure à la loi SRU
	Les décisions rendues avant l’intervention de la loi SRU posaient le principe selon lequel ce ne sont pas les
	décisions votées qui rendent la créance exigible : l’exigibilité résulte de l’appel de fonds; en sorte que le
	vendeur n’était pas tenu vis à vis de la copropriété des travaux votés tant que le syndic n’a pas appelé les
	provisions correspondant au coût de ces travaux 226
	Par conséquent, le débiteur des charges de copropriété est celui qui est propriétaire au moment où l'appel
	de fonds est lancé par l'assemblée
	224 PARIS, 23\ieme{} Chambre B 23 septembre 1994 Dalloz 1996 Somm p 159.
	225 civile 3\ieme{}, 8 juillet 2015 \no de pourvoi: 14-12995 - publié au bulletin - Cassation
	226 PARIS 8\ieme{} Chambre 6 fév 1997, Dalloz 1997 IR p 63.
	droit de la copropriété année 2018-2019
	204
	Il s'en suit que si les appels de fonds sont “ lancés ” avant la cession, les sommes appelées restent dues
	par le cédant, bien que les travaux ne soient exécutés qu'après. Le syndic pourra former sur ces sommes
	opposition sur le prix de vente en vertu de l'article 20 de la loi.
	3. Exigibilité des appels de charges postérieurement à la loi SRU
	Les textes combinés de la loi SRU et du Décret du 27 mai 2004 permettent aujourd’hui de répondre sans
	la moindre hésitation à l’imputabilité des appels de charges entre vendeur et acquéreur.
	Article 6-2 du Décret du 17 mars 1967 (rédaction du 27 mai 2004) :
	1\degre{} Le paiement de la provision exigible du budget prévisionnel, en application du troisième alinéa de l'article
	14-1 de la loi du 10 juillet 1965, incombe au vendeur ;
	2\degre{} Le paiement des provisions des dépenses non comprises dans le budget prévisionnel incombe à celui,
	vendeur ou acquéreur, qui est copropriétaire au moment de l'exigibilité ;
	3\degre{} Le trop ou moins perçu sur provisions, révélé par l'approbation des comptes, est porté au crédit ou au
	débit du compte de celui qui est copropriétaire lors de l'approbation des comptes.
	Certes, ce texte ne précise pas la date d’exigibilité dans chaque cas, alors que le principe demeure que le
	syndic ne peut exiger que les sommes … exigibles. Reprenons cependant successivement ces trois
	dispositions :
	LA PROVISION SUR BUDGET.(ART 14-1 DE LA LOI)
	D’après cet article, la provision sur charge est « exigible le premier jour de chaque trimestre ou le premier
	jour de la période fixée par l'assemblée générale »
	Pour savoir qui est débiteur il suffit donc de savoir qui est copropriétaire au premier jour du trimestre ou
	à la date d’exigibilité fixée par l'assemblée générale.
	LES TRAVAUX HORS BUDGET.(ART.14-2 DE LA LOI)
	D’après cet article, les dépenses sur travaux, qui ne peuvent être comprises dans le budget prévisionnel
	sont « exigibles selon les modalités votées par l'assemblée générale. »
	droit de la copropriété année 2019-2020
	205
	Si l’assemblée générale vote des travaux, elle doit impérativement fixer elle-même la date d’exigibilité
	des appels de fonds correspondant à ces travaux.
	Certes se pose la question de savoir qui sera débiteur si l’assemblée générale ne respectant pas les
	dispositions légales oublie de voter sur les modalités d’exigibilité des travaux.
	Deux hypothèses sont envisageables :
	- soit on en revient à la jurisprudence actuelle et on considère qu’est débiteur celui
	qui est copropriétaire au jour où l’appel de fonds est lancé,
	- soit on considère que faute par l’assemblée générale d’avoir voté les modalités
	d’exigibilité des fonds travaux … ceux-ci ne sont pas exigibles. Sanction
	particulièrement rigoureuse pour le syndic qui sera dans l’impossibilité de faire
	exécuter les décisions d’assemblée générale.
	L’APUREMENT DES COMPTES.
	Le décret précise que dans les relations entre le syndicat des copropriétaires et les copropriétaires il n’y a
	pas lieu à faire un compte prorata : la règle est simple, c’est celui qui est copropriétaire à la date de
	l’approbation des comptes (donc lors de l’assemblée générale approuvant les comptes) qui fait son affaire
	personnelle du trop versé en application des appels provisionnels ou de l’insuffisance de ces appels
	provisionnels.
	ABSENCE DE CONSEQUENCE DES APPELS DE FONDS POUR LE FONDS DE TRAVAUX
	La loi ALUR oblige la quasi-totalité des copropriétés comportant un ou plusieurs logements à créer un
	« fonds de travaux » qui se traduira par des appels décidés en assemblée générale227.
	mais il n’y aura pas lieu à modification du Décret du fait de la création du fonds de travaux puisque celuici
	est attaché au lot et reste acquis au syndicat des copropriétaires en cas de vote du lot.
	4. Les Conventions entre les parties concernant les charges.
	Les règles qui viennent d'être exposées ne concernent que les rapports des parties avec le syndicat.
	227 Voir infra le II de l’article 14-2 de la loi après modification par la loi ALUR.
	droit de la copropriété année 2018-2019
	206
	Mais il est possible, dans les rapports du vendeur avec l'acheteur, que la répartition soit aménagée
	différemment, c'est-à-dire que la contribution finale de chacun soit fixée librement par accord contractuel.
	Ainsi, les parties peuvent-elles convenir que les créances du syndicat exigibles lors de la mutation seront
	prises en charge par l'acquéreur. Cette clause sera obligatoire entre les parties mais inopposable au
	syndicat, à l'égard duquel le vendeur restera seul débiteur.
	Il s’agit là de l’application d’une règle de droit :
	Article 1165 du Code Civil : “ Les conventions n’ont effet qu’entre les parties contractantes; elles ne nuisent point
	au tiers, et elles ne lui profitent que dans le cas prévu par l’article 1121 (stipulation pour autrui) ”.
	Le décret du 27 mai 2004 renforce cette règle puisqu’il édicte :
	Article 6-3 :
	Toute convention contraire aux dispositions de l'article 6-2 n'a d'effet qu'entre les parties à la mutation
	à titre onéreux.
	Il est vrai que le syndic aura intérêt à se rapprocher le plus possible des conventions des parties lorsque
	celles-ci lui permettront d’obtenir paiement au moment de la mutation, surtout, lorsque la convention
	stipule que les travaux votés restent à la charge du vendeur aux termes du contrat, alors que l’appel de
	fonds correspondant n’est pas encore exigible.
	En ce cas le syndic en demandera le paiement au notaire qui doit recevoir les fonds et on voit mal pour
	quelles raisons le vendeur s’y opposerait. Pour autant, le syndic ne pourra pas faire l’opposition de l’article
	20 de la loi alors que les fonds ne sont pas exigibles.
	C. EFFETS DES DECISIONS JUDICIAIRES SUR LA VENTE DU LOT
	1. Les conséquences de l’annulation de la vente
	On relève sur ce sujet un arrêt de la cour de Colmar en date du 8 novembre 1996228 :
	228 Recueil Dalloz 1997, Sommaires commentés p. 245
	droit de la copropriété année 2019-2020
	207
	En cas de résolution de la vente d'un lot de copropriété, le vendeur est censé être resté
	propriétaire de l'appartement en cause ;
	En l'état d'un jugement prévoyant que le transfert de propriété au profit des bénéficiaires de
	la promesse de vente initiale interviendrait soit par la signature d'un acte notarié, soit par la
	publication du jugement qui en tiendrait lieu, le propriétaire reste tenu au paiement des
	charges de copropriété jusqu'à la notification informant le syndic du transfert de propriété de
	l'appartement.
	2. Les conséquences de l’annulation des travaux
	Il arrive que, soit à la suite d’une décision de justice, soit à la suite d’un vote en A.G. la décision de travaux
	soit annulée et les appels de fonds remboursés. A notre sens, le syndic rembourse le “ copropriétaire ”,
	c’est à dire le propriétaire du lot au moment de l’annulation. Mais, à notre connaissance, la question n’a
	pas été jugée.
	Symétriquement, si un nouvel appel de fonds est voté en « substitution » du premier, annulé, et n’est pas
	contesté, le nouveau copropriétaire en sera redevable.229 :
	3. Affectation des remboursements obtenus par le syndicat des
	copropriétaires après la vente du lot.
	Fréquemment le syndicat des copropriétaires, tout en engageant une procédure contre l’assureur ou
	contre le responsable décide de réaliser les travaux et fait un appel de fonds auprès des copropriétaires.
	Ultérieurement le syndicat des copropriétaires obtiendra condamnation de cet assureur ou du
	responsable au remboursement des sommes dont il a fait l’avance. Ce remboursement pourra être effectif
	alors que le copropriétaire ayant fait l’avance des travaux a cédé son lot à un acquéreur. Le syndicat des
	copropriétaires devra répartir les sommes reçues entre les copropriétaires : doit-il faire ce
	remboursement à celui qui avait fait l’avance provisionnel, à celui qui était copropriétaire lors de la
	réception des fonds ou à celui qui est copropriétaire à la date d’approbation des comptes de l’exercice au
	cours duquel le remboursement est intervenu (les fonds ont pu être reçus en janvier 2011 et les comptes
	ne seront approuvés qu’en juin 2012 (or entre temps l’acquéreur a peut être revendu).
	La réponse est apportée par un arrêt de cassation230 du 19 décembre 2012 qui casse l’arrêt de la cour
	ayant condamné le syndicat des copropriétaires à rembourser le copropriétaire ayant fait l’avance des
	fonds : « Qu’en statuant ainsi, alors que le trop perçu sur provisions qui apparaît après la mutation à titre
	onéreux de lots de copropriété est porté au crédit de celui qui est copropriétaire lors de l’approbation des
	comptes, la cour d’appel a violé le texte susvisé ».
	229 3\ieme{} Chambre civile 6 octobre 1999, Administrer jan 2000 p 30, note Capoulade.
	230 3\ieme{} Chambre civile 19 décembre 2012, \no 11-17178 (Cité Rivern c/ Itraco) au Bulletin
	droit de la copropriété année 2018-2019
	208
	On peut sans doute s’étonner de voir la cour de cassation décider que si le syndicat des
	copropriétaires obtient remboursement par un tiers des travaux votés, les fonds reçus des copropriétaires
	pour la réalisation de ces travaux deviennent des « trop perçus », mais il convient d’en déduire purement
	et simplement que les fonds reçus d’un tiers par le syndicat doivent être attribués à celui qui est
	copropriétaire lors de l’approbation des comptes comportant ces « produits ». D’où un conseil important :
	le syndic ne doit pas distribuer les fonds lorsqu’il les perçoit ; il ne pourra les imputer (donc les distribuer)
	qu’après approbation des comptes et à la date de celle-ci !
		\chapter{La location du lot}

Il faut envisager successivement le droit de louer, c'est-à-dire la mesure dans laquelle le copropriétaire a la faculté de donner son lot en location, puis les effets produits par cette location.

\section{Le droit de louer}

	\subsection{La liberté de louer son lot}
	
		Le droit de louer, c'est-à-dire de retirer les fruits de son lot, est une prérogative appartenant à tout propriétaire. L'opération de location ne peut directement ou indirectement être interdite ni par le règlement de copropriété, ni par une décision d'assemblée générale.
		
		Ainsi, est considérée comme nulle la clause subordonnant la conclusion du bail à l'autorisation du conseil syndical\footnote{Lyon, 22 janvier 1969, A.J.P.I. 1969. 418, note BOUYEURE}. Il en serait incontestablement de même d'une clause exigeant l'accord du syndic préalablement à toute location.
		
		De même, sont réputées non-écrites les limitations à la liberté de louer qui ne sont pas justifiées par la sauvegarde de la destination de l'immeuble.
		
		Par exemple, l'interdiction générale de toute location sera presque toujours réputée non-écrite car il faudrait des circonstances tout-à-fait exceptionnelles pour qu'une telle stipulation puisse trouver sa justification dans une référence à la destination de l'immeuble.
	
	\subsection{Les restrictions licites pouvant figurer dans le règlement de copropriété}
	
	En revanche, cette destination de l'immeuble peut rendre licites certaines limitations :
	\begin{itemize}
		\item clause qui, dans un immeuble résidentiel, stipule que les lots ne pourront en aucun cas être divisés en vue de la location\footnote{Paris, 19 juin 1985, I.R. 452, obs. GIVERDON} ;
		\item clause qui, dans un immeuble de grand standing, interdit la location des chambres de service à des personnes étrangères à la copropriété\footnote{Paris, 12 février 1976, D.1977 I.R. 42} ;
		\item clause interdisant les locations meublées dans un immeuble d’habitation. Ce type de location à caractère commercial multiplie les allées et venues dans l'immeuble et sont source d'insécurité.
	\end{itemize}
	
\section{Le contrat de bail}
	
	\subsection{L’information du locataire sur la destination de l’immeuble}
	
		Outre les conditions de validité tenant au droit commun, le contrat de bail doit se conformer à la destination de l'immeuble en copropriété : ainsi dans un immeuble à usage d'habitation exclusive, il est interdit de louer à des fins professionnelles ou commerciales. Le copropriétaire qui agirait ainsi devrait indemniser les autres membres du syndicat de leur préjudice, mais si les parties avaient conclu en connaissance de cause, le bail demeurerait valable dans les rapports entre elles (bailleur / locataire).
		
		Cependant, la responsabilité du bailleur envers son locataire pourrait être engagée si le règlement comportait des contraintes dommageables pour lui et dont il aurait fallu qu'il fût informé.
		
		De plus, la nullité du contrat de location pourrait être envisagée si les stipulations du règlement de copropriété faisaient obstacle à une activité essentielle pour le locataire et formellement prévue par le contrat : logement loué à usage pour partie professionnel alors que le règlement contient une clause d'habitation exclusivement bourgeoise\footnote{AUBERT et BIHR, La location d'habitation, 1990, \no67 }.
		
		Afin de faciliter les actions directes du Syndicat contre le locataire du lot, et sans doute un meilleur contrôle des activités exercées (lesquelles doivent être conforme à la destination de l’immeuble), la loi \no 2018-1021 du 23 novembre 2018 dite ELAN avait prévu la communication au syndic des coordonnées de son locataire par le bailleur « avec l’accord de celui-ci ». Toutefois, la disposition a été censurée par le Conseil constitutionnel, comme cavalier parlementaire. Le syndic ne peut donc obtenir du propriétaire, ni la copie du bail ,ni l’identité du locataire, sauf si une clause du règlement de copropriété le prévoit.
	
	\subsection{L’obligation de communication du règlement de copropriété a la charge du bailleur}

		L'article 3 de la loi du 6 juillet 1989 (dite loi \nom{Méhaignerie}) dispose\footnote{La loi ALUR a réécrit l’article 3 de la loi du 6 juillet 1989, mais cette phrase a été conservée dans sa rédaction d’origine.} :
		\begin{quote}
			<< {\itshape lorsque l'immeuble est soumis au statut de la copropriété, le copropriétaire bailleur est tenu de communiquer au locataire les extraits du règlement de copropriété concernant la destination de l'immeuble, la jouissance et l'usage des parties privatives et communes et précisant la quote-part afférente au lot loué dans chacune des catégories de charges,
			
			\lips
			
			Le bailleur ne peut pas se prévaloir de la violation du présent article. } >>
		\end{quote}
		Ces extraits peuvent être indifféremment annexés au contrat de bail ou remis dans les mains du locataire. Dans ce cas, il est souhaitable que le contrat de bail mentionne que cette communication a eu lieu. Le bailleur pourra aussi demander à son locataire un reçu signé indiquant que les extraits ont été remis.
		
		Les sanctions du défaut de communication ne sont pas précisées par la loi; si ce n'est l'interdiction pour le bailleur de se prévaloir de la violation de ce texte.
		
		Il faut d'abord souligner que la sanction ne consiste pas dans l'inopposabilité du règlement au locataire. Ce dernier s'y trouve, on le sait, automatiquement soumis\footnote{Cf. Supra les effets du règlement de copropriété}.
	
	\subsection{La question des locations de courte durée}
	
		L’article 16 de la loi ALUR\footnote{C’est le seul texte concernant le droit de la copropriété qui a été remis en cause par les Députés devant le Conseil Constitutionnel qui y voient une atteinte à la propriété de l’immeuble pour le cas où l’assemblée générale refuserait son accord.} impose un certain nombre de contraintes nouvelles à la location dite « de courte durée » qui concerne le plus souvent les appartements ou studios meublés joliment situés dans les villes touristiques que les propriétaires, par l’intermédiaire du net ou d’agences spécialisées, mettent à la disposition de touristes pour une durée qui varie le plus souvent d’une nuit à une semaine.
		Cet article précise dans un premier temps que cette mise en location constitue un changement d’usage au sens de l’article L 631-7 du CCH.
		
		Il précise en deuxième lieu que cette pratique peut être soumise à l’autorisation du Conseil Municipal ; ladite autorisation fixant les modalités de cette mise en location (article 631-7-1 A nouveau du CCH), sauf dans le cas où cette mise à disposition porte sur le local qui constitue l’habitation principale du loueur.
		
		Enfin, il convient de signaler que la loi ALUR, article 19, adoptée par le Parlement avait prévu, si le local mis à disposition pour location de courte durée est un lot de copropriété que le syndicat pourrait décider que cette mise à disposition nécessiterait son accord préalable. Mais cette disposition a été censurée par le Conseil Constitutionnel : que ce texte « a ainsi, dans des conditions contraires à l'article 2 de la Déclaration de 1789, permis à l'assemblée générale des copropriétaires de porter une atteinte disproportionnée aux droits de chacun des copropriétaires ».
		
		Rappelons également les dispositions de l’article L 324-1-1 du Code du Tourisme :
		\begin{quote}
			I.-Toute personne qui offre à la location un meublé de tourisme, que celui-ci soit classé ou non au sens du présent code, doit en avoir préalablement fait la déclaration auprès du maire de la commune où est situé le meublé.
			
			Cette déclaration préalable n'est pas obligatoire lorsque le local à usage d'habitation constitue la résidence principale du loueur, au sens de l'article 2 de la loi \no 89-462 du 6 juillet 1989 tendant à améliorer les rapports locatifs et portant modification de la loi \no 86-1290 du 23 décembre 1986.
			
			II.-Dans les communes où le changement d'usage des locaux destinés à l'habitation est soumis à autorisation préalable au sens des articles L. 631-7 et L. 631-9 du code de la construction et de l'habitation une délibération du conseil municipal peut décider de soumettre à une déclaration préalable soumise à enregistrement auprès de la commune toute location pour de courtes durées d'un local meublé en faveur d'une clientèle de passage qui n'y élit pas domicile.
			
			Lorsqu'elle est mise en œuvre, cette déclaration soumise à enregistrement se substitue à la déclaration mentionnée au I du présent article.
			
			Un téléservice permet d'effectuer la déclaration. La déclaration peut également être faite par tout autre moyen de dépôt prévu par la délibération susmentionnée.
			
			Dès réception, la déclaration donne lieu à la délivrance sans délai par la commune d'un accusé-réception comprenant un numéro de déclaration.
			
			Un décret détermine les informations qui peuvent être exigées pour l'enregistrement.
		\end{quote}
		
		Et les articles D 324-1 et D 324-1-1 du Code de Tourisme modifiés pat le Décret \no 2017-678 du 28 avril 2017 sur les modalités de la Déclaration en mairie. Ce décret définit les données exigibles par les communes en cas de déclaration obligatoire des loueurs en meublé touristique.
		
		Sur la question on pourra se reporter utilement à la Réponse Ministérielle \no 97541 du 21 mars 2017, p. 2459 (JCP N – \no 14-15 – 7 avril 2017 p. 18) qui constitue une synthèse appréciable des difficultés engendrées par la location touristique de courte durée.
		
		Lorsque les communes (Paris, Lyon, Marseille) exigent une déclaration pour changement d'affectation, on note une tendance des juges du fond admettre que, si la location meublée touristique correspond aux conditions définies par la commune, il ne s'agit plus d'un local à destination d'habitation. Cela peut également poser des difficultés en termes de conformité de l'affectation du lot au regard de la destination de l'immeuble tel que défini par le règlement de copropriété.
		
		La cour de cassation a évolué dans le même sens. En effet, elle a d’abord considéré que l’exercice de cette activité n’était pas incompatible avec une destination de l’immeuble dite « simplement bourgeoise », c’est-à-dire tolérant l’exercice d’activités libérales.

		Un arrêt (non publié) de la 3ème chambre civile est revenue sur ce principe en 2018,assimilant plutôt cette activité à une affectation commerciale :
		\begin{quote}
			\textbf{Civ. 3ème, 8 mars 2018 – \no 14-15864 – non publié}
			
			« {\itshape En présence d’un règlement de copropriété autorisant seulement un usage mixte habitation-professionnel avec clause obligeant les bailleurs à aviser le syndic de l’existence d’un bail, le juge du fond peut souverainement décider que l’installation d’occupants pour de très brèves – ou plus longues - périodes dans des « hôtels studios meublés » avec prestations de services est contraire à la destination résidentielle de l’immeuble} »
		\end{quote}
	
\section{Les effets de la location}
	
	\subsection{Effets de droit commun du bail}
	
		Le bailleur a l'obligation de procurer la jouissance du bien loué, de garantir le locataire contre les troubles et d'entretenir le bien.
		
		Le locataire est tenu de payer les loyers et les charges récupérables, d'user normalement du bien loué et de procéder aux réparations locatives.
		
		En ce qui concerne la récupération des charges dues à la suite d'un bail de locaux d'habitation ou mixtes, il faut souligner qu'elle ne peut être exigée que sur justification et en contrepartie :
		\begin{itemize}
			\item des services rendus liés à l'usage des différents éléments de la chose louée;
			\item des dépenses d'entretien courant et des menues réparations sur les éléments d'usage commun dans la chose louée;
			\item du droit au bail et des impositions qui correspondent à des services dont le locataire profite directement (art.23 L.6 juillet 1989).
		\end{itemize}
		La liste de ces charges est fixée par décret en Conseil d'Etat : il s’agit du Décret du 9 novembre 1982 “ pris en application de l’article L. 442-3 du Code de Construction et de l’Habitation. Par conséquent, toutes les charges de copropriété ne sont pas nécessairement répercutables sur le locataire.
	
	\subsection{Rôle des locataires au sein de la copropriété}

		\subsubsection{Le locataire n’est pas substitué au propriétaire du lot}
		
			En principe, le locataire n'est pas membre du syndicat. Il n'a aucun lien de droit avec celui-ci.C'est pourquoi il ne participe pas en principe à la vie de la copropriété :
			\begin{itemize}
				\item Il n'est pas membre de l'assemblée générale; il ne peut y participer en son nom ni voter les délibérations qui lui sont soumises. Il ne peut davantage être membre du Conseil Syndical\footnote{ L’Avant Projet de loi de 1997 prévoyait cependant cette faculté pour les locataires.}.
				\item Il n'a pas qualité pour demander l'inscription d'une question à l'ordre du jour, ni pour agir en nullité de décisions votées par l'assemblée générale.
				\item Il ne peut demander l'autorisation d'exécuter des travaux sur les parties communes\footnote{PARIS 23\degres, 4/06:91 - Loyers et Copropriété octobre 1991}.
				\item Il ne peut faire tierce-opposition aux décisions de l’A.G., car il ne peut avoir plus de droits que le copropriétaire\footnote{Civ. 3\degres 18/11/92, Loyers et Copropriété oct. 1991}.
			\end{itemize}
		
			Corrélativement, il n'est pas redevable à l'égard du syndicat du paiement des charges communes qui restent dues par le bailleur en tant que copropriétaire. Le syndic ne peut donc l'assigner directement en paiement des charges, même celles qui font partie des charges récupérables.
			
			De même, le locataire n'est pas responsable des tiers (le médecin pour ses patients qui stationnent abusivement sur les parties communes\footnote{Civ. 3\degres 12/06/91, Bull Civ III \no 171}).
		
		\subsubsection{La représentation des locataires au sein des instances du Syndicat des Copropriétaires}
		
		Il a été institué certaines structures permettant aux locataires d'être informés du fonctionnement de la copropriété.
		
		Ces règles figurent aujourd'hui à l'article 44 dernier alinéa de la loi du 23 décembre 1986 (dans sa rédaction issue de la loi du 6 juillet 1989) :
		\begin{itemize}
			\item Il peut être constituée dans chaque immeuble ou groupe d'immeubles une association de locataires regroupant au moins 10\% des locataires ou des associations siégeant à la Commission Nationale de Concertation.	
			\item Ces associations désignent au syndic par lettre recommandée avec accusé de réception le nom de trois au plus de leurs représentants choisis parmi les locataires de l'immeuble.
			\item Ces représentants ont accès aux différents documents concernant la détermination et l'évolution des charges locatives.
			\item A la demande de ceux-ci, le syndic les consulte chaque semestre sur les différents aspects de la gestion de l'immeuble.
			\item Ils peuvent assister aux assemblées générales et formuler des observations sur les questions inscrites à l'ordre du jour. Mais ils n'ont pas le droit de vote.
			\item Le syndic informe les représentants de l'association par lettre recommandée avec demande d'accusé de réception de la date, de l'heure, du lieu et de l'ordre du jour de l'assemblée générale.
			\item Enfin, dans chaque bâtiment d'habitation, un panneau d'affichage doit être mis à la disposition des associations pour leurs communications portant sur le logement et l'habitat, dans un lieu de passage des locataires.
		\end{itemize}
	
		Cette loi a été complétée par une loi du 13 décembre 2000 (SRU) qui ajoute aux associations de locataires représentant ces derniers vis-à-vis du syndicat des copropriétaires les groupements de locataires affiliés à une organisation siégeant à la Commission Nationale de Concertation qui se voient reconnaître les mêmes droits que l’association de locataire\footnote{MM Lafond et Stemmer considèrent cependant que la rédaction de la loi de 2000 ne permet pas à ces groupements de locataires de participer aux assemblées générales (Cf op. cité p. 742)}.
		
		Toutefois, le locataire à titre individuel ne possède aucun de ces droits qui sont nécessairement exercés par un représentant d'association (représentant qui peut être lui-même locataire de l'immeuble concerné) ou membre d’un groupement de locataires.
	
	\subsection{Les obligations du locataire vis-a-vis du syndicat et leurs sanctions}
	
		\subsubsection{Opposabilité du règlement de copropriété}
		
			Le locataire vit dans l'immeuble et il n'est pas possible de ne pas tenir compte de ce fait.
			
			C'est pourquoi on sait que le règlement de copropriété est opposable de plein droit au locataire ainsi que les décisions du syndicat. (civ.3ème 14 avril 2010, Administrer août 2010, p40)
		
			De la même manière, le locataire est tenu de laisser exécuter dans les parties privatives louées les travaux régulièrement décidés par l'assemblée en vertu des articles 25 e), g), h), i), 26 et 30 de la loi du 10 juillet 1965.
			
			Il s'agit des travaux visés à l'article 9 de la loi : travaux légalement ou réglementairement obligatoires, travaux d'économie d'énergie, de mise aux normes de salubrité et de sécurité ou d'équipement, travaux d'accessibilité aux personnes handicapées, travaux de fermeture destinés à assurer la sécurité de l'immeuble, travaux d'amélioration de l'immeuble.
		
		\subsubsection{Action Oblique contre le locataire.}
		
			Le copropriétaire bailleur est responsable du non respect par son locataire des obligations résultant du Règlement de Copropriété. Souvent ce copropriétaire, pris entre deux feux, aura tendance à "protéger" son locataire contre les exigences du syndicat, et ce quand bien même le locataire a un comportement fautif.
			
			Le syndicat des copropriétaires n'a pas de lien de droit avec le locataire.
			
			Certes il peut demander sa condamnation sous astreinte à faire cesser les troubles. Mais le syndicat souhaiterait faire cesser définitivement la gêne ressentie par les copropriétaires, et en quelque sorte, se substituer au copropriétaire bailleur pour faire résilier le bail.
			
			Or, aux termes de l'article 1166 du Code Civil : << les créanciers peuvent exercer tous les droits et actions de leur débiteur, à l'exception de ceux qui sont exclusivement attachés à leur personne. >>
			
			En d'autres termes, le créancier peut agir directement contre le débiteur de son débiteur lorsque son débiteur n'exerce pas lui-même les actions qu'il possède contre ce débiteur\footnote{A est créancier de B qui est lui-même créancier de C. Si B n'agit pas contre C pour obtenir paiement, A intentera directement l'action en paiement contre C !}.
			
			Dans notre domaine, si le locataire ne respecte pas les dispositions du Règlement de Copropriété, il commet une faute contractuelle qui autorise le copropriétaire bailleur à l'assigner en résiliation judiciaire du contrat de bail.
		
			Avant la réforme du Code Civil, La Cour de Cassation avait admis que le syndicat des copropriétaires pouvait invoquer les dispositions de l'article 1166 du Code Civil et assigner directement le locataire en résiliation du bail \footnote{Civ. 3\degres 20 octobre 1981, Bull. III \no 162 p. 117; Civ. 3\degres 14 novembre 1985, JCP 86 IV 39.}.
			
			A l’époque, l’article 1166 du code civil disposait :
			\begin{quote}
				\textbf{Art. 1166.}- Néanmoins, les créanciers peuvent exercer tous les droits et actions de leur débiteur, à l’exception de ceux qui sont exclusivement attachés à la personne
			\end{quote}
			
			La question du maintien de cette jurisprudence après la réforme du code Civil se pose, car les conditions de l’action oblique ont été redéfinies et sont plus strictes
				
			\begin{quote}
				\textbf{Art. 1341-1.}- Lorsque la carence du débiteur dans l’exercice de ses droits et actions à caractère patrimonial compromet les droits de son créancier, celui-ci peut les exercer pour le compte de son débiteur, à l’exception de ceux qui sont exclusivement rattachés à sa personne. 
			\end{quote}»
			
			Or l’action en résiliation du bail pour violation du règlement de copropriété n’a pas réellement un « caractère patrimonial ». Pour autant, cette action a été admise récemment par la cour d’appel de Lyon :
			
			\begin{quote}
				\textbf{Cour d'appel Lyon Chambre civile 1, section B 14 Novembre 2017 \no 15/08882 R.G : 15/08882 Association MUTATION MUTUELLE REPUBLIQUE DEMOCRATIQUE DU CONGO NGO}
				
				\textit{Le syndicat des copropriétaires subissant les nuisances liées à l’exploitation, dans les locaux commerciaux loués, d’une activité d'établissement de nuit de nature à incommoder les copropriétaires, en violation des clauses du règlement de copropriété, est recevable et bien fondé à demander la résiliation du bail et par l’action oblique, outre des dommages et intérêts contre le copropriétaire bailleur.}
			\end{quote}
			
			Cependant l’action oblique ne peut être intentée par le syndicat qu’à condition de démontrer l’inertie du bailleur
			\begin{quote}
				\textbf{Civ.3ème 20 décembre 1994}
				
				\textit{Justifie sa décision, au regard de l'art. 1165 c. civ., la cour d'appel qui, sur l'action d'un syndicat de copropriétaires, ordonne l'expulsion d'un locataire qui exerce dans les lieux loués une activité nuisant à la tranquillité des copropriétaires, dès lors qu'elle relève que la carence du bailleur est une condition de recevabilité de l'action exercée par voie oblique, et que la mise en demeure du bailleur n'a pu mettre fin à cette contravention aux clauses du bail et au règlement de copropriété, le syndicat des copropriétaires agissant dans les seuls droits du copropriétaire-bailleur en poursuivant la résiliation du bail et l'expulsion du locataire.}
			\end{quote}
		
			Les même dispositions de l’article 1166 ont été appliquées par la 6ème chambre de la Cour d’appel de Paris alors que des occupants sans droit ni titre s'étaient installés dans un logement au décès de son propriétaire ; le syndicat des copropriétaires est déclaré recevable à agir, par voie oblique, en expulsion\footnote{Paris 6e ch. C 20 juin 2000 Recueil Dalloz 2001, Somm p. 351 voir aussi CA paris 31 mars 2001, AJDI 2001, p 806 ; CA paris 24 sept 2003, Jd \no223282}.
	
	\subsection{Les relations entre syndicat, locataire et bailleur}
	
		\subsubsection{Responsabilité du Syndicat vis-à-vis du locataire}
		
			Dans le domaine de la responsabilité civile, le syndicat peut être tenu de réparer les dommages causés aux locataires par le vice de construction ou le défaut d'entretien des parties communes (art.14 al.3 de la loi du 10 juillet 1965).
			
			Par ailleurs, les locataires peuvent agir sur le fondement de la responsabilité du fait des choses (art.1384 al 1. du Code civil), ce qui dispense le locataire de faire la preuve du vice de construction ou du défaut d’entretien de l’immeuble; sauf toutefois dans l’hypothèse de l’effondrement (total ou partiel) de l’immeuble, auquel cas l’action du locataire doit avoir pour fondement les dispositions de l’article 1386 du code civil, ce qui implique, comme sur le fondement de l’article 14 de la loi de 1965, de démontrer que cet effondrement est consécutif à un défaut d’entretien ou à un vice de construction.
		
		\subsubsection{Responsabilité du bailleur vis-à-vis du syndicat et des autres copropriétaires}
		
			A l'égard du syndicat, c'est le bailleur qui garde la qualité de copropriétaire avec toutes les prérogatives et charges qui y sont attachées. Participation à l'assemblée, droit de vote, droit de contester les décisions, obligation de payer les charges communes (quitte à se faire rembourser par le locataire des charges récupérables).
			
			Au cas de troubles ou de dommages causés par le preneur, le copropriétaire bailleur sera tenu pour responsable et disposera d'un recours contre le locataire fautif. ( PARIS, 20 janvier 1983, D 83 IR 335; CIV 3\degres 18 déc 1991, Loyers et Copropriété février 1992 \no 84).
			
			Le dommage qui porte atteinte aux parties communes peut aussi bien provoquer l'action du syndicat que celle d'un copropriétaire, dans la limite cependant du préjudice effectivement subi par ce dernier\footnote{Civ. 3\degres 12 mai 1993, Loyers et Copropriété Juillet 1993 \no 280}.
		
		\subsubsection{Recours entre Copropriété, Bailleur, Locataire}

			Un arrêt de la cour d’appel de Paris 23\degres Chambre A du 21 novembre 2000 (Loyers et Copropriété avril 2002 \no 102) permet de bien préciser les responsabilités respectives du syndicat des copropriétaires, du propriétaire bailleur et du locataire.
			L'arrêt apporte les précisions suivantes:
			\begin{itemize}
				\item Le contrat de location concerne exclusivement les rapports entre le bailleur et le preneur: il ne peut avoir aucune incidence dans la détermination des responsabilités encourues en raison de dommages mettant en cause la copropriété (Cass. 3 e ci v., 4 janv. 1991 : JCP N 199 1, /1, p. 269. - 16 juin 1993: Rev. Loyers 1994, p. 192. - 20 nov 1996 : Loyers et copr 1997, comm. no 127).
				\item Par voie de conséquence, le copropriétaire-bailleur demeure seul responsable vis-à-vis du syndicat ou de l'un de ses membres des troubles imputables au comportement de son locataire, qui doit lui-même se soumettre aux obligations inscrites dans le règlement de copropriété.
				\item Lorsque des dommages supportés par le copropriétaire ou son locataire sont dus à un vice de construction ou au défaut d'entretien des parties communes (L. art. 14), leur réparation incombe directement au syndicat (Cass. 3' civ, 3 juil. 1991 : Loyers et copr. 1991, comm. no 353. - le, avr. 1999 : Loyers et copr. 1999, comm. no 168).
				\item Si le locataire d'un lot est victime de troubles du voisinage provoqués par un autre copropriétaire ou locataire, la responsabilité de leur auteur doit être recherchée sur le fondement soit de la responsabilité contractuelle inhérente au bail (contre le bailleur), soit de la responsabilité quasi-délictuelle instituée aux articles 1382 et 1383 du Code civil (contre le voisin responsable).
				Cela étant, le copropriétaire-bailleur jugé responsable à l'égard du syndicat peut exercer une action récursoire à l'encontre de son locataire en application des stipulations du bail.
			\end{itemize}
	
\section{Interventions du locataire dans la vie de la copropriété}
	
	\subsection{La location accession}
	
		Rappelons qu'il s'agit d'une catégorie particulière de location avec promesse de vente régie par les dispositions de la loi du 12 juillet 1984.
		Aux termes de cette loi, le locataire-accédant est assimilé au copropriétaire dès lors que l'immeuble est déjà soumis au statut de la loi de 1965. En sorte que le locataire-accédant aux termes de l'article 32 de la loi << est subrogé dans les droits et obligations du vendeur >>.
	
		En conséquence :
		\begin{enumerate}
			\item Il participe aux assemblées générales et vote sur toutes questions autres que celles réservées au bailleur par la loi et qui concernent :
			\begin{itemize}
				\item les travaux qui sont à la charge du bailleur (sur éléments porteurs et concourant à la stabilité ou à la solidité des bâtiments, aux éléments d’équipement intégrés à eux, et aux éléments qui assurent le clos, le couvert et l’étanchéité, sauf parties mobiles) ;
				\item les actes de disposition et les travaux d’amélioration.
			\end{itemize}
			\item Il participe aux charges afférentes à l'entretien et aux réparations de l'immeuble (Toutefois le vendeur est garant des charges dues par le locataire-accédant).
			\item Il peut être membre du conseil syndical de l’immeuble.
		\end{enumerate}
	
	\subsection{Le bail réel et solidaire}
	
		Le bail réel est solidaire est un démembrement du droit de propriété, plus proche de la division entre nue-propriété et usufruit que d'un bail véritable. Il a pour objet la dissociation du foncier et de la construction. Le preneur est titulaire d'un véritable droit réel, pouvant faire l'objet d'une mutation.
		
		La loi \no 2018-1021 du 23 novembre 2018 a complété ce dispositif et précisé qu’il pouvait porter sur un lot de copropriété.
		
		C'est pourquoi l’article L 255-7-1 assimile le preneur d’un tel bail à un copropriétaire :
		\begin{quote}
			\textbf{Art. L. 255-7-1.}-Pour l'application de la loi \no 65-557 du 10 juillet 1965 fixant statut de la copropriété des immeubles bâtis, la signature d'un bail réel solidaire est assimilée à une mutation et le preneur est subrogé dans les droits et obligations du bailleur, sous réserve des dispositions suivantes :
			
			1\degres Le preneur dispose du droit de vote pour toutes les décisions de l'assemblée générale des copropriétaires, à l'exception de décisions prises en application des d et n de l'article 25 et des a et b de l'article 26 de la même loi ou de décisions concernant la modification du règlement de copropriété, dans la mesure où il concerne les spécificités du bail réel solidaire. Le bailleur exerce également les actions qui ont pour objet de contester les décisions pour lesquelles il dispose du droit de vote. Aucune charge ne peut être appelée auprès du bailleur y compris pour des frais afférents aux décisions prises par lui ou pour son compte
			
			2\degres Chacune des deux parties peut assister à l'assemblée générale des copropriétaires et y formuler toutes observations sur les questions pour lesquelles elle ne dispose pas du droit de vote.
		\end{quote}
		On notera que cette disposition a pour effet de faire payer par le preneur les travaux d’amélioration (art 25 n), alors qu’il n’aura pas le droit de vote sur ceux-ci.
		\chapter{L'usage du lot}
		\chapter{Le règlement de copropriété}

L'article 664 du Code civil n'avait prévu aucune réglementation dans les rapports entre les copropriétaires puisqu'il ne traitait que du partage des charges d'entretien de l'immeuble.

Aujourd'hui, l'organisation de la copropriété est prévue dans un acte juridique fondamental : le règlement de copropriété.

Cet acte, qui constitue la charte de la copropriété, détermine les prérogatives de chaque copropriétaire, tant sur les parties privatives que sur les parties communes ainsi que les règles d'organisation et d'administration de la copropriété de l'immeuble.

Le texte essentiel qui régit le règlement de copropriété est l'article 8 de la loi du 10 juillet 1965 aux termes duquel
\begin{quote}
	<< {\itshape Un règlement conventionnel de copropriété, incluant ou non l'état descriptif de division, détermine la destination des parties tant privatives que communes, ainsi que les conditions de leur jouissance ; il fixe également, sous réserve des dispositions de la présente loi, les règles relatives à l'administration des parties communes.} >>
\end{quote}

\section{Rédaction du règlement de copropriété}

	\subsection{Les caractéristiques du règlement de copropriété}
	
		\subsubsection{Le règlement de copropriété est un document obligatoire}
		
			L'article 8 qui vient d'être rapporté figure au nombre des dispositions de la loi de 1965 auxquelles l'article 43 confère un caractère impératif. Il en résulte que le règlement de copropriété est aujourd'hui un document obligatoire, alors que sous l'empire de la loi de 1938, il n'était que facultatif, et à cette époque il existait des copropriétés sans règlement.
			
			Outre l'existence du règlement, le contenu de celui-ci présente aussi très largement un caractère impératif, car la plupart des règles de la loi concernant l'organisation et l'administration de la copropriété sont d'ordre public.
	
			Cependant, \textbf{l'existence d'un règlement ne constitue pas la condition nécessaire à l'application du statut de la copropriété à un immeuble collectif}\footnote{Cour d'Appel PARIS 23\degres Ch B 16 mai 2002 Loyers et Copropriété 2002 \no 265}.
			
			Dès que celui-ci est divisé en lots attribués à des personnes différentes, le statut de la copropriété s'applique et l'immeuble se trouve de plein droit soumis à toutes les dispositions de la loi de 1965 et du décret de 1967\footnote{Civ. 3ème, 3 octobre 1969, Informations Rapides de la Copropriété 1970 p.77; Civ. 3ème, 15 novembre 1989, Bull. Civ. III \no214, D.1990.195 note CAPOULADE et GIVERDON, Rep. Defrénois 1990, art. 34802 \no65 obs. H. SOULEAU}, quand bien même le règlement de copropriété n’aurait pas été rédigé.
			\begin{quote}
				\textbf{Civ. 3ème 15 novembre 1989} :
				
				« {\itshape La loi du 10 juill. 1965 régit tout immeuble bâti ou groupe d'immeubles bâtis dont la propriété est répartie entre plusieurs personnes, par lots comprenant chacun une partie privative et une quote-part de parties communes ;
				
				Le statut de la copropriété des immeubles bâtis s'applique de plein droit dès lors que sont remplies les seules conditions prévues par l'art. 1er, al. 1er, de la loi du 10 juill. 1965 ;
				
				Dès lors, viole cet article l'arrêt qui déclare qu'une disposition de cette loi (en l'occurrence l'art. 42, al. 1er) ne peut s'appliquer que dans les copropriétés organisées, ce qui implique l'existence d'un règlement de copropriété} »
			\end{quote}
			
			L'importance que revêt le règlement interdit qu'une copropriété en demeure trop longtemps dépourvue. C'est pourquoi en cas de carence ou d'impossibilité, l'établissement du règlement peut être requis à tout moment de l'autorité judiciaire\footnote{T.G.I. Brest 8 avril 1970, AJPI 1970, p.221 note BOUYEURE.}. 
			
			De même l’arrêt de cassation précité (Civ. 3ème, 15 novembre 1989) précise :
			\begin{quote}
					\itshape
					Viole aussi l'art. 14 de la loi du 10 juill. 1965, ensemble l'art. 3, al. 1er, du décret du 17 mars 1967, le même arrêt qui, pour refuser de commettre un expert afin de proposer un projet de règlement de copropriété aux parties et, à défaut d'accord, de recourir au juge, retient que, en matière d'organisation et de fonctionnement d'une copropriété, il n'appartient pas au juge de se substituer aux parties, s'agissant d'actes conventionnels, mais qu'il appartient aux copropriétaires réunis en une assemblée générale de décider, les tribunaux n'ayant vocation à trancher qu'en cas de désaccord ou de conflit mettant en échec l'application de la loi, alors que, à défaut d'accord entre les parties, le règlement de copropriété peut résulter d'un acte judiciaire constatant la division de l'immeuble dans les conditions fixées par la loi du 10 juill. 1965.
			\end{quote}
	
			Le juge peut donc être saisi par un plusieurs copropriétaires pour établir, voire compléter, le règlement de copropriété\footnote{TGI BOBIGNY, chambre 5 section 2, 2 juillet 2003}.
		
		\subsubsection{Le règlement de copropriété a une nature a la fois contractuelle et institutionnelle}
		
			Certains estiment qu'il s'agit d'un acte de nature contractuelle conformément aux termes de l'article 8 de la loi du 10 juillet 1965 qui emploie les termes de << règlement conventionnel >>.
			
			D'autres considèrent que le règlement a un caractère institutionnel, qu'il constitue un << acte-règlement >  parce que, dans certaines hypothèses, il est applicable à des copropriétaires qui n'y ont pas personnellement adhéré : cas du règlement voté à la majorité de l'article 26 qui s'impose aux minoritaires ou de celui établi par le juge au cours d'un partage judiciaire ou d'une impossibilité de réunir la majorité de l'article 26.
			
			Cet argumentation se trouve renforcée par l'arrêt du 15 novembre 1989 ayant déclaré le juge habilité à imposer un Règlement de Copropriété : dans une telle hypothèse on est fort éloigné de la convention !
			
			Malgré ces objections qui sont fondées sur les circonstances de l'élaboration du règlement, la jurisprudence adopte la qualification conventionnelle proposée par le législateur. On peut donc voir dans le règlement une convention qui s'apparente au contrat d'adhésion.
			
			C'est d'ailleurs cette nature de Contrat d'Adhésion qui a été retenue par un arrêt de la Cour d'Appel de PARIS du 21 décembre 1990\footnote{Cité in Revue de Droit Immobilier 1991.257} : << {\itshape Le Règlement de Copropriété est un contrat d'adhésion qui constitue la loi entre les parties}. >>
			
			Les conséquences d'une telle qualification sont nombreuses:
			\begin{itemize}
				\item L'interprétation du règlement relève, comme celle des contrats, du pouvoir souverain des juges du fond, sous réserve de la dénaturation de clauses claires et précises\footnote{Civ. 3ème, 28 février 1969, Bull. Civ. \no III \no190; Civ. 3ème 18 décembre 1973, Bull civ III \no 637 p. 464}.
				
				\item Les principes d'interprétation des contrats édictés par le Code civil aux articles 1156 et suivants sont applicables au règlement : ainsi a-t-il été jugé que lorsqu'une clause était douteuse, elle devait s'appliquer en faveur de celui qui avait adhéré au contrat (le
				copropriétaire) et contre le rédacteur du règlement, en vertu de l'article 1162 du Code civil.
				
				\item La responsabilité encourue par le copropriétaire en cas de violation du règlement est de nature contractuelle\footnote{ Civ. 3ème 18 janvier 1972, Bull. Civ. III \no39} : application de l'article 1143 du Code civil en ce qui concerne la démolition ou la remise en état.
				
				\item Exclusion des \textbf{actions possessoires} entre copropriétaires au cas de troubles émanant de l'un sur le lot de l'autre, en vertu de la règle selon laquelle de telles actions sont exclues entre contractants et que le préjudice résulte de la violation du règlement de copropriété qui est une convention\footnote{Civ. 3ème, 22 juin 1976, Bull. Civ. III \no274, Rep. Defrénois 1977, art.31350, obs. H.SOULEAU.}.
				
				\item Comme dans les contrats, une \textbf{clause pénale} peut être insérée dans le règlement de copropriété en prévision de la violation d'une de ses dispositions par un copropriétaire : Civ. 3ème, 30 octobre 1973, J.C.P. 1973 IV 402 ; ou une clause stipulant des intérêts de retard sur des appels de provision Paris 30 octobre 1979, D.1980, I.R. 240 obs. GIVERDON.
				
				\item Comme en matière conventionnelle civile, la \textbf{clause compromissoire }est prohibée (art. 2061 C.Civ.).
			\end{itemize}
	
	\subsection{Règlement préalable a la division de l’immeuble}
	
		Le règlement est, en règle générale, établi préalablement à la naissance de la copropriété, c'est-à-dire avant l'attribution à une personne d'un lot en propriété. La phase d'élaboration du règlement préalable présente des différences selon les circonstances dans lesquelles l'immeuble est mis en copropriété.
		
		\subsubsection{Vente en l'état futur d'achèvement}
		
			S'il s'agit d'un immeuble construit en vue d'être vendu par appartements, c'est le promoteur
			qui établit le règlement qu'il s'agisse d'un promoteur individuel ou d'une société de construction-vente régie par la loi du 16 juillet 1971. Ce règlement est annexé aux actes de vente, et l'acquéreur du lot vendu << adhère >> à ce règlement au moment de son acquisition.

		\subsubsection{Immeuble construit par une société d'attribution}
		
			Si l'immeuble est construit par une société d'attribution, le règlement est annexé ou intégré aux statuts de la société, de sorte que tout acquéreur de parts donnant vocation à l'attribution d'un lot en ait connaissance. Jusqu'à dissolution ou retrait d'un associé, le règlement ne vaut que comme règlement de jouissance, lequel est soumis dans une large mesure aux règles du statut de la copropriété. Dans ce cas, le règlement --- de jouissance puis de copropriété --- est établi par le rédacteur des statuts de la société d'attribution.
			
		\subsubsection{Vente d'un Immeuble existant après division}
		
			Si la mise en copropriété résulte de la division d'un immeuble déjà construit appartenant à un propriétaire unique, le règlement de copropriété est établi par les soins de ce dernier et il est également joint aux actes de vente.
			
			Dans ces trois premiers cas, l'élaboration du règlement émane d'une seule personne et c'est au moment de la passation des actes de vente que l'accord de volonté de l'acquéreur intervient : on retrouve le mécanisme du contrat d'adhésion.
		
		\subsubsection{Établissement lors du Partage de l'Immeuble}
		
			L'établissement préalable du règlement peut aussi résulter du partage d'un immeuble en indivision à la suite, par exemple, du décès d'un propriétaire unique. Lorsque le partage est amiable, le règlement de copropriété est établi à la suite d'un accord entre les héritiers copartageants. Il résulte d'une libre négociation et présente alors, dès sa phase d'élaboration, un caractère véritablement conventionnel.
			
			Au cas de désaccord entre les indivisaires, le partage est judiciaire et le règlement, généralement établi par le notaire qui liquide la succession, est imposé par la décision de justice. Disparaît alors toute intervention de la volonté des copartageants, futurs copropriétaires, et, par conséquent, toute dimension contractuelle.
	
	\subsection{Règlement postérieur a la naissance du syndicat des copropriétaires}
	
		On sait que sous l'empire de la loi de 1938, le règlement de copropriété n'était pas obligatoire. Il l'est devenu depuis 1965. Dès lors, dans les copropriétés anciennes qui ne comportaient pas de règlement, il a fallu, pour une mise en conformité, établir un règlement s'appliquant à une copropriété déjà constituée et ayant peut-être fonctionné depuis de longues années. Remarquons toutefois que cette situation est exceptionnelle.

		\subsubsection{Adoption par l’assemblée générale}
		
			C'est le syndicat, qui, d'après l'article 14 al.3, a compétence pour établir un règlement ultérieur.
			
			L'assemblée générale des copropriétaires devra procéder à cette opération à la majorité de l'article 26 b de la loi de 1965, c'est-à-dire à la majorité des membres du syndicat représentant les deux tiers des voix\footnote{On relèvera avec intérêt l’arrêt de rejet de la cour de cassation en date du 21 juin 2000 (\no 98-20-897) dans une affaire « chaufferie de la porte de Bâle » : sur consultation du Professeur Giverdon les propriétaires et copropriétaires concernés ont constaté que l’ensemble constituait un ensemble immobilier au sens de l’article 1 al 2 de la loi soumis au statut de la copropriété à défaut de convention contraire et ont en conséquence chargé le Conseil de Surveillance d’élaborer un Règlement de copropriété, sans en subordonner l’application à un vote de l'assemblée générale ; la cour de cassation rejette le pourvoi contre l’arrêt de la cour de Colmar ayant considéré que l'assemblée générale avait délégué la mise en oeuvre des modalités pratiques du Règlement de copropriété, sans subordonner la mise en application à un vote de l'assemblée générale en sorte que le Règlement de copropriété ainsi établi s’imposait à tous sans nécessité d’adoption en assemblée générale.}. Il en résulte que dans ce cas, le règlement peut être imposé par une majorité à une minorité. C'est pourquoi, l'on admet qu'un règlement voté dans de telles conditions ne peut porter atteinte aux << droits acquis >> préexistants de chacun des copropriétaires et ne peut régir que "la jouissance, l'usage et l'administration des parties communes" (art. 26 b).
		
		\subsubsection{Imposition par le juge}
		
			Étant donné le caractère obligatoire du règlement, l'opinion dominante est qu'il faut alors s'adresser au juge ; c'est le Tribunal de Grande Instance qui, après avis d'expert, procédera à la rédaction du règlement.
			
			Cette solution a été consacrée par la Cour de cassation dans l’arrêt du 15 novembre 1989\footnote{Civ. 3ème, 15 novembre 1989, Bull. Civ. III \no214, D.l990, 195, note CAPOULADE et GIVERDON, Rep. Defrénois 1990, art.34802 \no64 obs. H.SOULEAU.}, déjà cité, qui a décidé :
			\begin{quote}
				<< {\itshape qu'à défaut d'accord entre les parties, le règlement de copropriété peut résulter d'un acte judiciaire constatant la division de l'immeuble dans les conditions fixées par la loi du 10 juillet 1965}. >>
			\end{quote}
			
			La décision rendue paraît en conformité avec l'article 3 du décret de 1967 qui précise que le règlement peut résulter d'un << acte judiciaire >>.
			
			C’est sans doute l’avis de la cour de cassation dans un arrêt de rejet Civ.3ème 13 septembre 2005\footnote{Civ 3\degres, 13 sep 2005, AJDI jan 2006 p. 34, note Capoulade.}. A l’ origine la copropriété (trois copropriétaires) était dépourvue de Règlement de copropriété ; une décision de justice définitive avait homologué un règlement rédigé par un expert judiciairement commis, mais n’avait pas été publié dix ans après la décision. Un des copropriétaires avait demandé au juge des référés de désigner un notaire pour mettre en forme le règlement et le publier. Décision favorable du juge des référés, confirmée par la Cour d’Aix en Provence qui considérait la décision du juge des référés « opportune dans l’intérêt du syndicat ». Pourvoi en cassation des deux autres copropriétaires au visa des articles 14 de la loi et 13 du décret. La cour rejette le pourvoi en considérant que la motivation de la cour était justifiée.
	
\section{Contenu du règlement de copropriété}
	
	Les dispositions que peut contenir le règlement de copropriété revêtent tantôt un caractère obligatoire, tantôt un caractère facultatif.
	
	Mais la liberté des rédacteurs d'introduire des dispositions qui diffèrent du contenu de la loi est assez restreinte, car l'article 43 édicte que sont réputées non écrites toutes les clauses contraires aux dispositions des articles 6 a 37 et 42 de la loi. Or ces articles contiennent l'essentiel de l'organisation et de l'administration de la copropriété.
	
	\subsection{Clauses obligatoires}
	
		Le caractère obligatoire de certaines dispositions résulte de l'article 8 de la loi du 10 juillet 1965 et de l'article 1er du décret du 17 mars 1967.
		
		Aux termes l'article 8 de la loi duquel le Règlement de Copropriété :
		\begin{itemize}
			\item détermine la destination des parties tant privatives que communes et les conditions de leur jouissance ;
			\item fixe les règles relatives à l'administration des parties communes.
		\end{itemize}
		
		L'article 1\ier{} du Décret reprend ces deux premiers points et ajoute la nécessité pour le Règlement de Copropriété de répartir les charges entre les lots de copropriété :
		\begin{quote}
			<< {\itshape Le Règlement de Copropriété mentionné à l'article 8 de la loi du 10 juillet 1965 susvisé comporte les stipulations relatives aux objets visés par l'alinéa 1er dudit article ainsi que l'état de répartition des charges prévu au dernier alinéa de l'article 10 de ladite loi}. >>
		\end{quote}
	
		Reprenons successivement ces trois points.
		
		\subsubsection{La destination des parties tant privatives que communes et conditions de leur jouissance}
		
			\paragraph{La destination des parties communes}
			
				La destination qui doit en être déterminée par le règlement, vise leur affectation à tel ou tel usage : aire de stationnement de véhicules, terrain de jeux pour enfants, cour, espaces verts \etc
				
				Les conditions de jouissance résultent de leur affectation, mais demandent parfois à être précisées. Il en est ainsi:
				\begin{itemize}
					\item si certaines parties communes sont réservées à l'usage de certains lots seulement (terrasses, parkings, jardins) ;
					
					\item du mode d'utilisation de certains locaux communs : loge de concierge, salle de réunion, locaux techniques ;
					
					\item d'interdictions et de limitations diverses concernant les parties communes telles que restrictions relatives à l'apposition d'enseignes ou de plaques professionnelles (Civ. 3ème, 18 juin 1975, Gaz. Pal. l975 II Somm. 224) --- une clause interdisant purement et simplement la pose d’enseignes alors que les locaux peuvent être affectés à usage commercial ne serait pas justifiée par la destination de l’immeuble\footnote{Cour d'Appel Metz, 5 sep 2013, \no 11/01701 ; Loyers et Copropriété jan 2014 comm. \no 34}
					
					\item la réglementation de la circulation ou du stationnement sur les parties communes (Civ. 3ème, 11 décembre 1973, J.C.P. 1974 II 17659, note GUILLOT), l'interdiction d'étendre du linge aux fenêtres (Civ. 3ème, 2~ novembre 1973, J.C.P. 1974 II 17644
				\end{itemize}
	
			\paragraph{La destination des parties privatives}
				
				Les modalités d’usage des parties privatives du lot ne relèvent pas, en effet, du droit discrétionnaire du copropriétaire. C'est ainsi que le règlement précisera quelle est la destination des locaux principaux (usage commercial, professionnel ou d'habitation bourgeoise) et accessoires (caves, greniers, celliers, garages, chambres de service).
				
				Les conditions de jouissance des parties privatives se traduisent par des clauses interdisant certaines activités ou comportements : activités gênantes par le bruit, l'odeur, la chaleur ou les trépidations (blanchisserie, restaurant, cours de piano ou de chant), la modification des fenêtres parties privatives ou porte-fenêtre \etc
				
				Il conviendra cependant de faire la distinction entre ce que Me BOCCARA\footnote{J CL Construction Fasc. 112 – Concession Immobilière \no 11} a appelé la \textbf{destination première} des lots, c'est-à-dire leur destination générale (habitation, mixte habitation et profession libérale, profession libérale, bureaux, commerces, artisanat, industrie) et la \textbf{destination seconde} des lots qui concerne la branche particulière d’activités (blanchisserie, confiserie, etc). Cette distinction pourrait également se faire en ce sens que la destination du lot visée par l’article 8, c’est la destination première, alors que l’affectation du lot selon sa destination constituerait en fait son affectation (ou destination seconde).
				
				Les copropriétaires peuvent demander au syndicat des copropriétaires la libération de leur lot privatif illicitement occupé par un transformateur EDF. Cette occupation trouvait sa source dans une convention conclue entre EDF et le promoteur, à laquelle le syndicat des copropriétaires était étranger.
				
				En statuant ainsi, sans rechercher si la convention conclue entre le constructeur et EDF relative à l'installation d'un transformateur avait été transmise au syndicat, la cour d'appel n'a pas donné de base légale à sa décision\footnote{Cass. Civ 3e 7 juillet 2010}.
		
		\subsubsection{La destination de l'immeuble.}
		
			Cette notion joue en effet un rôle capital en matière de copropriété puisque comme l'édicte l'alinéa 2 de l'article 8 de la loi du 10 juillet 1965 :
			\begin{quote}
				<< {\itshape Le règlement de copropriété ne peut imposer aucune restriction aux droits des copropriétaires en dehors de celles qui seraient justifiées par la destination de l'immeuble telle qu'elle est définie aux actes, par ses caractères ou sa situation}. >>
			\end{quote}

			Or ces << actes >> visés par le texte consistent essentiellement dans le règlement de copropriété.
			
			En tout cas, il ne faut pas confondre destination du lot (voir ci-dessus) et destination de l’immeuble, quand bien même la destination des lots peut être un élément constitutif de la destination de l’immeuble.
			
			\paragraph{Une notion vague et complexe}\footnote{On consultera avec le plus grand intérêt la Revue de Droit Immobilier du 3\degres Trimestre 1995 p. 407 à 469 qui reproduit intégralement les interventions consacrées à la notion de Destination de l'Immeuble à l'occasion de la journée Henri Souleau (qui fût professeur à l'I.C.H. jusqu'à sa mort en août 1993).}
			
				La destination de l'immeuble au nom de laquelle des limitations peuvent être imposées aux droits des copropriétaires est une notion floue et complexe.
				
				Dans les travaux préparatoires de la loi, il en a été donné la définition suivante:
				\begin{quote}
					<< {\itshape L'ensemble des conditions en vue desquelles un copropriétaire a acheté son lot, compte tenu de divers éléments, notamment de l'ensemble des clauses des documents contractuels, des caractéristiques physiques et de la situation de l'immeuble, ainsi que de la situation sociale de ses occupants}. >>
				\end{quote}
			
				C'est en se référant à la destination de l'immeuble que l'on qualifiera tel immeuble de << résidence de luxe >>, << d'immeuble bourgeois >>, d'immeuble de type H.L.M. ou Logéco, de centre commercial, etc.
				
				En réalité, la destination de l’immeuble est un « standard juridique » ayant pour fonction d’être un \textbf{mécanisme régulateur}, et d'assurer la pérennité des caractéristiques essentielles de l'immeuble et une certaine possibilité d'évolution en fonction des progrès techniques, du changement des besoins ou des mœurs.
				
				De ce fait, la destination de l'immeuble est une notion de fait soumise à \textbf{l'appréciation souveraine des juges du fond}. C'est la notion à appliquer pour déterminer celle des caractéristiques de l'immeuble autour de laquelle toutes les autres vont s'ordonner.
			
			\paragraph{Les éléments constitutifs de la destination de l’immeuble}
			
			La destination de l'immeuble tient nécessairement compte de la destination des parties privatives.
			
			Elle comporte :
	
			\subparagraph{Des éléments objectifs afférents a l'immeuble}
			
				Il s’agit notamment de ses caractères architecturaux, son aspect, son caractère plus ou moins luxueux, son standing, sa situation : qualité de l'environnement, caractéristiques du quartier et du voisinage.
			
			\subparagraph{Des éléments contractuels provenant des actes tels que la possibilité ou non d'exercer une activité commerciale}\footnote{Cf. Mr SIZAIRE La détermination contractuelle de la destination de l'immeuble (RD imm. 17 (3), juill- sep 1995 p. 475. }
			
				Étant d'ailleurs précisé comme l'indique Monsieur SIZAIRE, que si le règlement peut définir la destination de l'immeuble, aux termes de l'article 8 al 2 de la loi, << il ne peut imposer aucune restriction aux droits des copropriétaires en dehors de celles qui seraient justifiées par la destination de l'immeuble. >> En sorte que la détermination de la destination de l'immeuble par le Règlement de Copropriété $\dots$ se trouve limitée par la destination de l'immeuble elle-même !
				
				Monsieur SIZAIRE ajoute : << N'y a t’il pas là une contradiction ? Pas vraiment, mais le règlement - plus précisément son auteur - ne peut faire ce qu'il veut. Il doit respecter une \textbf{certaine cohérence et le contrôle de cette cohérence appartient aux tribunaux}. >>
				
				Au titre de la destination de l'immeuble, le règlement de copropriété comportera une clause générale indiquant quel est le genre ou le type de l'immeuble en copropriété :
				\begin{itemize}
					\item Immeuble d'habitation exclusivement bourgeoise (qui interdit toute autre utilisation que l'habitation)
					
					\item Immeuble d'habitation bourgeoise (qui, outre l'habitation autorise l'exercice d'activités professionnelles compatibles avec la tranquillité des habitants : médecin, avocat, notaire, expert comptable).
					
					En revanche, est incompatible avec une telle clause l'exercice de toute industrie, de tout commerce ou de toute profession de nature à gêner les autres copropriétaires : mécanicien prothésiste gênant les voisins par l'émission d'odeurs nauséabondes (Paris, 10 décembre 1966, Gaz. Pal. 1, 208), laverie automatique (T.G.I. Paris 8ème ch. 15 avril 1969 A.J.P.I. 1970, 136, obs. CABANAC), atelier de réparation de cyclomoteurs (Civ. 3ème, 22 février 1984, Administrer 1984 p.43, obs. GUILLOT).
					
					Il peut arriver qu'en présence d'une clause d'habitation bourgeoise, le règlement indique une liste de professions dont l'exercice est admis. Une telle énumération est considérée, sauf stipulation contraire, comme énonciative et non limitative. Sont donc autorisées les professions qui ne causent			
					pas plus de troubles que celles expressément visées dans le règlement (Civ. 3ème, 13 novembre 1975, Bull. Civ. III \no332).
					
					\item Immeuble mixte : d'habitation bourgeoise, professionnel et commercial.
					
					Le Règlement de Copropriété peut également stipuler que les professions libérales ne pourront être exercées dans le lot que si elles sont l'accessoire de l'habitation ou encore que leur exercice ne pourra se faire que dans un nombre limité de pièces. Ces clauses doivent être considérées comme valables, le juge du fond exerçant un contrôle de la véracité de cette affectation (PARIS 19\degres Chambre 27 mai 1992; Loyers et Copropriété octobre 92 \no 399)
					
					\item Immeuble de bureaux (ce qui exclut l'habitation)
					
					\item Centre commercial
					
					\item Résidence services (ou unités retraite) pour habitants du troisième âge ; laquelle suppose, outre l'habitation la prestation de certains services médicaux, de restauration, d'animation ou de transport, dont il a été jugé à plusieurs reprises qu'ils étaient compatibles avec l'objet de la Copropriété.
				\end{itemize}
			
			\paragraph{La destination de l'immeuble, notion évolutive.}
		
				Bien souvent se pose la question de savoir si la destination de l'immeuble est figée à l'époque de l'établissement du Règlement de Copropriété, ou bien si au contraire elle peut présenter un caractère évolutif.
				
				Le problème se posera par exemple en cas d'évolution du quartier, ou encore en fonction du plus ou moins bon état d'entretien, ou même, en considération de la situation sociale des occupants.
				
				Deux décisions de la Cour de PARIS (PARIS 19\degres Chambre 27 mars 1992 et PARIS 23\degres Chambre 3 avril 1992, commentés à la Revue de Droit Immobilier 92.370) ont admis que la destination de l'immeuble pouvait présenter un caractère évolutif.
		
		\subsubsection{L'administration des parties communes}

			Le règlement de copropriété prévoit et organise le fonctionnement du syndicat et les pouvoirs du syndic.
			
			A cet égard, la liberté des rédacteurs du règlement est extrêmement restreinte car les dispositions de la loi sont sur ces questions presque toutes impératives : il y a donc sur ce point reproduction des textes légaux.
			
			Cette rubrique comprend les règles concernant les assemblées générales de copropriété (art.22 al.1 de la loi), les règles de convocation des copropriétaires, de tenue des séances de l'assemblée générale, des modalités de délégation du droit de vote. Il prévoit la périodicité des assemblées au moins annuelles.
			
			Il détermine, en se conformant à l'article 18 de la loi, les pouvoirs du syndic et ceux du conseil syndical.
			
			Il prévoit expressément que la gestion de l'immeuble pourra éventuellement être assurée par un syndicat coopératif.
			
			Toutes ces dispositions concernant l'administration des parties communes seront analysées à propos des différents organes qu'elles régissent.
		
		\subsubsection{La répartition des charges communes}
		
			Aux termes de l'article 10 de la loi,
			<< {\itshape le règlement fixe la quote-part afférente à chaque lot dans chacune des catégories de charges}. >>
			
			Les critères de répartition entre les différentes catégories de charges sont fixés par les alinéas 1 et 2 de l'article 10 et doivent figurer dans le règlement.
			
			\begin{enumerate}[label=alpha*)]
				\item Les charges relatives à la conservation, à l'entretien et à l'administration des parties communes que l'on appelle aussi charges générales communes se répartissent conformément << aux valeurs relatives des parties privatives comprises dans les lots, tel que ces valeurs résultent des dispositions de l'article 5. >>
				
				Ces charges communes générales sont donc proportionnelles aux tantièmes de copropriété ou aux quotes-parts de parties communes : il en est ainsi par exemple des honoraires du syndic.
	
				\item Les charges entraînées par les services collectifs ou les éléments d'équipement commun se répartissent en fonction de l'utilité que ces services ou équipements présentent à l'égard de chaque lot.
				
				Ainsi en est-il par exemple des charges d'ascenseur plus fortes pour un lot situé au sixième étage que pour celui qui se trouve au premier.
				
				Le règlement de copropriété doit comporter un état de répartition des charges qui fixe la quote-part afférente à chaque lot dans chacune des catégories de charges ou à défaut les bases selon lesquelles la répartition est faite pour une ou plusieurs catégories de charges (art.l al.3 D.17/3/67).
			\end{enumerate}
	
	\subsection{Clauses facultatives}
	
		\subsubsection{Clauses ayant pour objet d’écarter certaines dispositions du statut qui ne sont pas d’ordre public (à savoir les articles 1 à 5 et 38 à 41).}
		
			Il en est ainsi de :
			\begin{itemize}
				\item la répartition des éléments de l'immeuble entre parties privatives et parties communes (art.2 et 3) ;
				\item la détermination de la quote-part des parties communes : ces quotes-parts ne sont pas nécessairement calculées selon le mode de calcul de l'article 5 (consistance, superficies et situation des lots sans égard à leur utilisation), alors que ce texte est impératif pour le calcul de la quote-part des charges générales ; de la reconstruction de l'immeuble au cas de destruction.
			\end{itemize}
		
		\subsubsection{Clauses limitant les droits des copropriétaires sur leurs lots lorsqu'elles sont justifiées par la destination de l'immeuble}
		
			Remarquons que ces hypothèses sont assez rares et concernent généralement des \textbf{immeubles de grand standing}, comportant un nombre limité de copropriétaires et étant situés dans des quartiers résidentiels et calmes ou dans des espaces verts tranquilles. Afin de conserver cette qualité d'habitation, certaines restrictions peuvent être introduites concernant les droits des copropriétaires. Toutefois, la jurisprudence reconnaissant la validité de telles clauses ultra protectionnistes sont rares, les Tribunaux de façon générale montrent une attitude réservée à l'égard de ces clauses qu'ils assimilent le plus souvent à une atteinte inadmissible au libre droit de jouissance et de disposer de son lot que l'article 9 de la loi du 10 juillet 1965 reconnaît aux copropriétaires.
	
			\paragraph{Restrictions au droit de disposer de son lot}\footnote{Cf Henri Souleau : Droit de disposer d'un lot dans un immeuble en copropriété (Etudes Offertes à Jacques Flour publiées au Defrénois).}
			\begin{itemize}
				\item Clauses d'agrément subordonnant l'aliénation à une autorisation de l'assemblée.
				\item Clauses interdisant la vente séparée de chambres de service afin de ne pas multiplier le nombre des occupants (Civ. 3ème, 10 mars 1981, J.C.P. 1982 II 19765, note GUILLOT ; Rep. Defrénois 1981, art.32797, obs. H.SOULEAU; Paris, 23 ch. B, 19 juin 1985, D.1985, I.R. 425 obs. GIVERDON); pacte de préférence dans un immeuble familial.
				\item Clauses interdisant la division des lots afin de maintenir le standing de l'immeuble (Civ. 3ème, 9 mars 1982, Administrer, oct.1982 p.28 note GUILLOT; Paris 23è ch. B, 19 juin 1985, I.R.425, obs. GIVERDON).
			\end{itemize}
			
			\paragraph{Clauses restreignant les modalités de jouissance du lot}
			\begin{itemize}
				\item Clauses restreignant la liberté de louer en vue de préserver les conditions d'occupation d'origine ou d'interdire des modes d'occupation non-conformes au type d'habitat dans l'immeuble : interdiction de louer en meublé, de diviser les lots en vue de locations séparées\footnote{Paris 19 juin 1985 D.1985 I.R. 425, obs. GIVERDON}, interdiction de louer des chambres de service à des personnes étrangères à la copropriété\footnote{Paris 12 février 1976, D.1977 I.R.42; Paris 23\degres Chambre 4 juin 1997 déclare licite la clause interdisant de louer des chambres de service à des personnes étrangères, compte tenu du standing de l’immeuble (51av Georges Mandel).}.
				
				\item Clauses limitant ou interdisant l'exercice de certaines activités gênantes par le bruit ou l'odeur : par exemple, l'installation au rez-de-chaussée d'un immeuble de standing d'un commerce de bar, salon de thé ouvert jusqu'à deux heures du matin où l'on sert des aliments cuisinés\footnote{Civ. 3ème 14 janvier 1987, Gaz. Pal.1987 Pan.14}, ou installation dans l'immeuble d'une blanchisserie qui par le bruit, l'odeur et les émanations est de nature à incommoder les copropriétaires\footnote{Civ. 3ème 18 février 1987, Rev. Loyers 1987 p.221}.
				
				\item Clauses destinées à préserver l'harmonie de l'ensemble immobilier en interdisant certains aménagements de parties privatives (Paris 23\degres Chambre, 28/10/92 LOY ET COP jan 93 \no 32).
			\end{itemize}
		
		\subsubsection{Dispositions diverses}

		Telles que conditions de souscription de polices d'assurance, d'emprunts hypothécaires, élection de domicile en cas de litige, institution d'une procédure de conciliation amiable, clause pénale etc...
	
	\subsection{Clauses prohibées}
	
		\subsubsection{Les dispositions contraires à l'ordre public ou aux bonnes mœurs}
		
			\begin{itemize}
				\item Clause rendant un lot totalement inaliénable, contraire à la liberté du droit de disposition.
				
				\item Clause obligeant un copropriétaire à céder son lot, notamment à titre de sanction, contraire à l'article 546 du Code civil selon lequel << nul ne peut être contraint de céder sa propriété. >>
				
				\item Clause compromissoire interdite par l'article 1006 du Code de procédure civile dans les rapports civils.
				
				\item Clause interdisant la détention d'un animal familier dans les locaux d'habitation, contraire à l'article 10 1) de la loi du 10 juillet 1970 :
				\begin{quote}
					<< {\itshape Est réputée non écrite toute stipulation tendant à interdire la détention d'un animal dans un local d'habitation, dans la mesure où elle concerne un animal familier. Cette détention est toutefois subordonnée au fait que ledit animal ne cause aucun dégât à l'immeuble ni aucun trouble de jouissance aux occupants de celui-ci}. >>
				\end{quote}
				
				Reste à savoir si un perroquet (dont la détention est souvent interdite par le Règlement de Copropriété) est ou non un animal familier.
			\end{itemize}
		
		\subsubsection{Les clauses contraires aux dispositions d'ordre public de la loi du 10 juillet 1965 et du décret \no67-223 du 17 mars 1967}
		
			L'article 43 de la loi répute en effet non écrites les clauses contraires aux articles 6 à 37 et 42 de la loi et ceux du décret pris pour leur application. Il s'agit, à titre d'exemple :
			\begin{itemize}
				\item clause de répartition des charges communes non conforme aux critères posés par l'article 10 de la loi ;
				
				\item clause interdisant à un copropriétaire de déléguer son droit de vote aux assemblées générales à une personne étrangère à la copropriété (art.22) ;
				
				\item clause refusant d'instituer un conseil syndical (art.21) ;
				
				\item clause limitant les pouvoirs du syndic tels qu'ils résultent de l'article 18 ;
				
				\item clause excluant toute indemnisation pour les copropriétaires devant subir dans ses parties privatives des travaux sur parties communes, contrairement aux dispositions de l’article 9 de la loi du 10 juillet 1965\footnote{Paris 23\degres Ch B, 28 mai 2009 ; Loyers et Copropriété nov 2009 \no 269}.
			\end{itemize}
		
		\subsubsection{Les clauses restrictives non justifiées par la destination de l'immeuble}
		
			\paragraph{Clauses restrictives au droit d’usage ou au droit de disposer}
			
			On peut citer à cet égard toutes les clauses limitant le droit des copropriétaires d'user de jouir et de disposer de leurs lots si de telles restrictions ne sont pas justifiées par la sauvegarde de la destination de l'immeuble :
			\begin{itemize}
				\item o clauses de préférence en cas de vente d'un lot dans un immeuble de standing non exceptionnel ;
				
				\item clauses interdisant la division d’un lot si le standing de l’immeuble ne le justifie pas\footnote{Civ 3\degres 26 mai 1988 JCP N 89 PRAT p 9.Dans le même sens et avec les mêmes termes voir Civ 3\degres 5 juillet 1989, Bull. \no 154.} ;
				
				\item interdiction de la circulation ou du stationnement de véhicules à moteur qui, dans un immeuble où sont autorisées les activités commerciales, empêchent les commerçants d'être livrés\footnote{Paris 8è ch.B, 9 avril 1987, D.1987, I.R.130 (en l’espèce la copropriété avait étayé un plancher haut dans un local privatif rendant ce local inhabitable pendant un mois.} ;
				
				\item clause imposant au copropriétaire de faire ses travaux privatifs par un entrepreneur déterminé ou de confier la gestion de son lot au syndic de l'immeuble.
				
				\item clause interdisant l'accès des parties communes d'un bâtiment aux copropriétaires des autres bâtiments alors que ces bâtiments n'ont pas été constitués en parties communes spéciales\footnote{Civ. 3° 30 juin 1992 - Bull. III p 40 n° 230} ;
				
				\item clause interdisant la location en meublée qui ne peut être justifiée par la destination bourgeoise de l’immeuble, dès lors que le Règlement de copropriété autorise les professions libérales ; « l'exercice d'une telle activité entraîne des inconvénients similaires à ceux dénoncés par le syndicat des
				copropriétaires pour la location meublée de courte durée et l'existence de nuisances fautives des locataires n'est pas établie »\footnote{Civ. 3\degres 8 juin 2011 – \no 694 FSPB}.
			\end{itemize}
			
			Étant observé toutefois qu’il ne peut exister de réponse de principe : la destination de l'immeuble doit s'apprécier \emph{in concreto}, en fonction d'éléments extérieurs au règlement de copropriété tenant à la situation de l'immeuble, son environnement ou ses caractéristiques\footnote{Civ. 3\degres - 9-6-2010 \no 09-14.206 : Bull. civ. III \no 116}.
			
			\paragraph{Clauses restreignant la concurrence}
			
			Ces clauses sont considérées par la jurisprudence comme << étrangères >> à la destination de l'immeuble dès qu’elles n'ont pas pour but de préserver un intérêt collectif, mais qu'elles sont stipulées pour la protection d'intérêts particuliers comme celui du promoteur vendeur ou celui des commerçants déjà installés dans l'immeuble.
			
			Ces clauses sont donc illicites\footnote{Civ. 3ème 11 mars 1971 (2 arrêts), J.C.P.1971 II 16722, concl. PAUCOT, note GUILLOT; Civ. 3ème 15 octobre 1974 J.C.P.1974 II note GUILLOT, Rep. Defrénois 1975, art.30965 obs. H.SOULEAU; Civ. 3ème 29 mai 1979, Rep. Defrénois l979, art.32162, obs.H.SOULEAU}.
			
			Mais la Cour de cassation admet qu'insérées dans l'acte de vente les clauses de non concurrence sont valables si elles n'ont pas pour but de réaliser une fraude à la loi. Est donc licite la clause de l'acte de vente stipulant que l'acquéreur ne pourra exploiter dans le lot vendu qu'un commerce déterminé, le vendeur s'interdisant pour sa part de vendre un autre lot à un commerçant concurrent\footnote{Civ. 3ème 9 novembre 1982, Bull. Civ. III \no215, Rep. Defrénois 1583, art.33093, obs. H.SOULEAU.}.
			
			Enfin, sont considérées comme licites les clauses d'intangibilité des commerces figurant dans les règlements de copropriété des centres commerciaux éloignés des villes, au motif qu'il est alors conforme à la destination de l'immeuble de maintenir la diversité des commerces qui y sont exploités\footnote{Civ. 3ème, 25 novembre 1980, Bull. Civ. III \no184, Rep. Defrénois 1981, art.32608 obs. H.SOULEAU}. Mais ici encore, eu égard aux circonstances, le juge peut annuler la clause en considérant que la concurrence est plutôt de nature à dynamiser le centre commercial.
			
			\paragraph{Clauses de solidarité}

			Il s'agit de la clause selon laquelle en cas de vente d'un lot l'acquéreur sera solidairement tenu avec le vendeur et vis à vis du syndicat de toutes sommes afférentes au lot vendu et non acquittées au moment de la mutation, par exemple de l'arriéré des charges dues par le vendeur. Cette clause facilitant le recouvrement des charges était fort appréciée des syndics.
			
			Après une vive discussion doctrinale, la Cour de cassation a décidé qu'une telle clause insérée dans le règlement était illicite, dès lors que le syndicat dispose de l'hypothèque légale instituée par l'article 20 de la loi pour recouvrer les sommes restant dues par le vendeur d'un lot\footnote{Civ. 3ème ler juillet 1980, D.1981 I note GIVERDON et CAPOULADE, Rep. Defrénois 1981, art.32608, obs. H.SOULEAU}.
			
			Cependant, dans les rapports entre le cédant et le cessionnaire, il est licite de stipuler la solidarité entre vendeur et acheteur dans l'acte de vente.
			
			Toutefois, si la vente a lieu sur adjudication l'adjudicataire ne peut être tenu au paiement de l'arriéré des charges, alors même que cette obligation figurait en termes non équivoques dans le Cahier des Charges (Civ. 3\degres 6 mars 1991; Civ 3\degres 17 juin 1992; Civ 2\degres 2 décembre 1992. Pour ces trois arrêts voir IRC 1993.342) :
			\begin{quote}
				<< {\itshape En matière de saisie immobilière, le cahier des charges ne peut modifier directement ou indirectement l'ordre dans lequel le prix des biens du débiteur, qui constitue le gage commun des créanciers, doit être réparti entre eux}. >>
			\end{quote}
		
			Est réputée non écrite la clause du règlement de copropriété qui exclut toute indemnisation pour le copropriétaire devant subir, dans ses parties privatives, des travaux sur les parties communes. Cette clause ne peut priver le copropriétaire de son droit à indemnisation pour le préjudice subi du fait des travaux\footnote{CA Paris 28 mai 2009 JD 2009-377938}.
			
			La clause du règlement de copropriété faisant supporter une surprime d’assurance par le copropriétaire d’un lot dans lequel est exploitée une discothèque, doit être réputée non écrite : le paiement des primes d’assurance souscrites pour les parties communes et privatives de l’immeuble constitue une charge relative à la conservation, à l’entretien et l’administration des parties communes\footnote{Cass. Civ. 3e 17 mars 2010}.
			
			Les primes d’assurances souscrites dans l’intérêt de l’ensemble des copropriétaires d’un immeuble comprenant une galerie marchande constituent des charges générales à répartir entre tous les lots : la répartition des charges spéciales à la galerie marchande, ne peut s’appliquer aux primes d’assurance.
			
			Il en est de même des dépenses afférentes au responsable du service de sécurité de l’ensemble immobilier\footnote{Cass. Civ. 3e 4 juin 2009}.

			Doit être réputée non écrite par application de l’article 43 alinéa 1er de la loi, la clause du règlement de copropriété donnant tous pouvoirs au syndic pour régulariser à première demande d’une Société et à son profit une convention d’occupation précaire sur un local, partie commune, pour une durée maximum de 10 ans moyennant une redevance annuelle déterminable ou pour lui vendre dans ce même délai ce local, pour un prix ferme et définitif.
			
			Cette clause a pour effet de priver par avance l’assemblée générale des pouvoirs de disposition et d’administration sur les parties communes qu’elle tient des règles d’ordre public des articles 17, 26 et 24 de la loi\footnote{Cass. Civ. 3e 11 février 2009}.
			
			La clause du règlement de copropriété prévoyant que les appartements ne pourront être consacrés à la location meublée sans l’autorisation de l’assemblée générale, laquelle autorisation pourra être retirée par l’assemblée sans que celle-ci ait à motiver sa décision et sans que le propriétaire visé puisse prétendre à une indemnité, donne à l’assemblée générale le pouvoir discrétionnaire d’autoriser un copropriétaire à louer ses lots en meublés et de retirer à tout moment cette autorisation.
			
			Cette clause restreignant les droits des copropriétaires sur leurs parties privatives, non justifiée par la destination contractuelle de l’immeuble, est réputée non écrite\footnote{CA Paris 3 février 201,0 \no 09/00448, JD 2010-380757}.
			
			Si chaque copropriétaire est libre de subdiviser son lot sans l’autorisation de l’assemblée générale, dès lors que cette subdivision n’est pas contraire à la destination de l’immeuble, c’est à la condition que le règlement de copropriété ne comporte ni interdiction, ni restriction ou que celles-ci aient été jugées inopérantes\footnote{Cass. Civ. 3e 25 février 2010}.
			
			La clause du règlement de copropriété subordonnant la possibilité de diviser les lots à la nécessaire autorisation de l’assemblée générale et en interdisant la location en chambre meublée ainsi que l’exploitation d’une pension de famille, a pour finalité de conserver à l’immeuble son caractère résidentiel tenant compte de son environnement et de son standing, auquel porterait atteinte notamment la réduction des surfaces des appartements et l’augmentation corrélative du nombre des appartements et des occupants est parfaitement conforme à la destination de l’immeuble de grand standing\footnote{CA Versailles 16 juillet 2009, \no 09/03168, JD 2009-379397}.
	
\section{Modification et adaptation du règlement de copropriété}

	\subsection{Les règles permanentes concernant la modification du règlement de copropriété}
	
		La loi a prévu la possibilité de modifier le règlement de copropriété. Mais, étant donné que le règlement constitue la charte de la copropriété, il est nécessaire qu'il présente une stabilité suffisante et que sa modification soit soumise à des conditions strictes.
		
		C'est pourquoi deux conditions sont exigées, l'une concernant la majorité requise, l'autre l'objet de la modification.
		
		Bien évidemment la modification du Règlement de copropriété ne peut être valablement décidée, en application de l’article 17 « d’ordre public » que si cette modification est votée en assemblée générale : il ne peut être suppléé à un vote de l’assemblée générale par aucun autre moyen, même un demande faite au notaire par l’ensemble des copropriétaires\footnote{Civ. 3\degres Ch. 8 juin 2011 (10-18.220) – Arrêt \no 700 FSPB}.
		
		\subsubsection{Les modifications relevant de l’article 26 de la loi}
	
			La modification du règlement de copropriété ne peut être décidée qu'à la double majorité de l'article 26, c'est-à-dire à la majorité des copropriétaires représentant les deux tiers des voix.
		
		\subsubsection{Les modifications relevant de l’unanimité}
			
			L'assemblée \textbf{ne peut modifier, à quelque majorité que ce soit, les droits des copropriétaires sur leurs parties privatives}, en modifiant la destination de ces parties ou les modalités de leur jouissance.
			
			A ce titre, l'assemblée ne peut, sauf unanimité :
			\begin{itemize}
				\item modifier les quotes-parts de droits des copropriétaires sur les parties communes ou le nombre de voix attaché aux lots ;
				\item supprimer un service collectif (service de conciergerie par exemple)
				\item supprimer un élément d'équipement commun sans prévoir son remplacement (Boite aux lettres).
			\end{itemize}

		\subsubsection{Les modifications des charges}
		
			L'article 11 de la loi pose le principe selon lequel l'état de répartition des charges ne peut être modifié qu'à l'unanimité des copropriétaires $\dots$
			
			Sauf les exceptions des articles :
			\begin{itemize}
				\item 11 (en cas de travaux ou d'actes d'acquisition ou de dispositions, en cas d'aliénation séparée d'une ou plusieurs fractions d'un lot) ;
				\item 25 f (en cas de changement d'usage d'une partie privative).
			\end{itemize}
			
			Dans tous les autres cas, la modification ne peut intervenir qu'à l'unanimité ou par décision du juge, que ce soit en cas de lésion (article 12 de la loi) ou en cas de nullité de la répartition (article 43 de la loi).
			
			Les modifications apportées, par une décision d’assemblée générale, au règlement de copropriété concernant la répartition des charges sont inopposables à l’acquéreur si elles n’ont pas été publiées au fichier immobilier ou s’il n’est pas expressément mentionné dans l’acte d’achat de l’acquéreur qu’il a adhéré aux obligations qui en résultent\footnote{Cass. Civ. 3e 8 septembre 2009}.
			
		\subsubsection{L’objet des modifications}
		
			Sauf unanimité, la modification ne peut concerner que la jouissance, l'usage et l'administration des parties communes. En application de ces prescriptions, le syndicat peut :
			\begin{itemize}
				\item établir ou modifier les périodes de chauffage de l'immeuble : T.G.I. Seine 13 mars 1952, Gaz. Pal. 1952 1.372 ;
				
				\item décider d'aménager un parking dans la cour commune : Civ. 3ème, 19 décembre 1978, J.C.P. 1979 IV p.72, ou autoriser les détenteurs de voitures à les garer dans la cour commune - T.G.I. Paris 23 avril 1976, D.1976 I.R. 312 ;
				
				\item réglementer l'accès de l'immeuble\footnote{Civ. 3ème, 19 décembre 1978, D.1979 I.R. obs. GIVERDON} ;
				
				\item réglementer la convocation et la tenue des assemblées générales sous réserve de ne pas porter atteinte aux dispositions impératives de la loi ;
				
				\item établir un \emph{Règlement Intérieur}, sous réserve que ce Règlement Intérieur ne soit pas en contradiction avec les dispositions du Règlement de Copropriété ou n'impose pas de restrictions nouvelles aux modalités de jouissance de son lot par le copropriétaire --- la pratique révèle en effet que le plus souvent les copropriétaires de locaux d'habitation votent un Règlement Intérieur pour << canaliser >> les << débordements >> qu'ils imputent à l'occupant des locaux commerciaux. Si de tels débordements peuvent faire l'objet de sanctions au << coup par coup >> et aboutir à une éventuelle interdiction d'exercer, ils ne sauraient justifier le vote de disposition restreignant ses droits sur les parties communes ;
				
				\item autoriser la mise en place d'étalages extérieurs dans une galerie marchande, partie commune, pendant les heures d'ouverture des magasins\footnote{Civ. 3ème 9 juillet 1986, Bull. Civ. III \no106, Rep. Defrénois 1986 art.33825 obs . H . SOULEAU}.
			\end{itemize}
			
			Les modifications du règlement de copropriété ne sont opposables aux acquéreurs qu’à dater de leur publication au fichier immobilier\footnote{Cass. Civ 3e 22 septembre 2009}.
		
	\subsection{L’adaptation du règlement (art 24 f) de la loi de 1965)}
	
		Article 24 f) de la loi du 10 juillet 1965 (Modifié par la loi ALUR)
		Sont notamment adoptées (à la majorité de l’article 24) f) les adaptations du règlement de copropriété rendues nécessaires par les modifications législatives et réglementaires intervenues depuis son établissement. La publication de ces modifications du règlement de copropriété sera effectuée au droit fixe.
		
		\subsubsection{Principe}
		
			Il s’agissait initialement d’une disposition provisoire – inscrite dans l’article 49 de la loi - devant permettre une « mise en conformité » des règlements de copropriété notamment aux modifications législatives intervenues lors de la loi SRU du 13 décembre 2000.
			
			De prorogation en prorogation, ce dispositif transitoire a fini par devenir permanent, a fin d’inciter les Syndicat à mettre leurs règlement de copropriété en « conformité » avec les évolutions législatives les plus récentes, pour la parfaite information des copropriétaires.
			
			Toutefois ce texte s’avère assez peu ambitieux dans ses objectifs et finalement insuffisant pour « nettoyer » un règlement de copropriété de toutes les clauses douteuses ou réputées non écrites du fait de l’évolution de la Jurisprudence. En effet, ce dispositif est encadré strictement par le législateur.
			
			\paragraph{Il s’agit d’une adaptation « nécessaire »}
			
				Ce terme ne prête pas à discussion : adapter, c’est rendre conforme en sorte, dans le cas présent de faire disparaître les contradictions entre le texte du règlement de copropriété et celui de la loi (et du règlement d’administration public du 17 mars 1967, pris en application de la loi). En outre, l’adaptation doit être « nécessaire » $\dots$
			
			\paragraph{Aux modifications législatives}
		
				26 lois successives du 28 décembre 1966 au 28 décembre 2016, dont les lois de 1985 (dite Loi \nom{Bonnemaison}), de 1994 (Loi sur l’Habitat) et 2000 (loi SRU) sont les principales $\dots$ ont modifié la loi de 1965.
				
				A ces modifications législatives il convient d’ajouter les 23 modifications apportées ar décret au Décret du 17 mars 1967 jusqu’au 30 décembre 2015 ; il convient également d’ajouter les 6 Ordonnances ayant directement modifié la loi \no65-557 du 10 juillet 1965 dont l’Ordonnance du 10 février 2016 portant réforme des contrats, du régime général et de la preuve des contrats ; il faut également ajouter les deux arrêtés dont l’arrêté comptable ! En effet, tous ces textes concernent directement le statut de la copropriété et font partie du corpus législatif visé par l’article 24 f).
			
			\paragraph{Intervenues depuis l’établissement du règlement de copropriété}
			
				Les adaptations autorisées sont seulement les adaptations rendues nécessaires par les lois publiées depuis l’établissement du règlement de copropriété.
		
				La date à prendre en compte est donc celle de la rédaction initiale du règlement de copropriété et non la date de ses modificatifs successifs. Si un règlement de copropriété a été établi postérieurement à la loi de 1965 sans respecter pour autant les dispositions de cette loi, l’article 24 f) ne permet pas de le rendre conforme à la loi.
			
		\subsubsection{Les adaptations possibles au titre de l’article 24 f}
		
			\paragraph{En présence d’un texte littéralement contraire à la loi postérieure}
			
				Il est des cas où l’adaptation nécessaire est évidente : lorsque le texte du règlement de copropriété est littéralement contraire au texte de la loi ou à celui du décret de 1967 pris en application de la loi.
				
				Si un règlement de 1950 dit que les assemblées sont convoquées huit jours avant la date prévue pour la réunion, il s’agit d’une disposition qui doit être adaptée pour tenir compte du texte adopté en 1965 soit postérieurement à l’établissement du règlement de copropriété d’origine.
				
				Mais si la portée de l’article 24 f) est réduite simplement à l’obligation de recopier la loi sur la copropriété dès lors que le texte en a été modifié par un nouveau texte législatif ou réglementaire, cette disposition est sans grand intérêt.
			
			\paragraph{En présence de dispositions nouvelles}
			
				Depuis la publication de la loi de 1965 de nombreuses dispositions ont été ajoutées au texte législatif :
				\begin{itemize}
					\item le privilège du syndicat ;
					\item la procédure applicable aux copropriétés en difficulté ;
					\item la procédure de recouvrement des provisions ;
					\item \etc
				\end{itemize}
			
				Si le Règlement de Copropriété est antérieur à ces nouvelles dispositions, il convient de les insérer au titre des modifications de l’article 24 f)
			
			\paragraph{En présence d’un texte contraire a la jurisprudence prise en application de la loi publiée après l’établissement du règlement de copropriété}

				Par exemple nous avons vu le sort que les tribunaux réservent aux clauses de solidarité contenues dans tous les bons règlements de copropriété depuis les deux arrêts du 1er juillet 1980. Ne doit-on pas dès lors supprimer ces clauses par l’adaptation prévue à l’article 24 f) ?
				
				Les copropriétaires auront du mal à comprendre que l’on a confié à un technicien le soin de mettre le règlement de copropriété en conformité avec la loi $\dots$ mais que ce technicien a laissé des clauses qui ne peuvent pas s’appliquer !
				
				Pourtant, il convient de ne pas céder à la tentation et ce pour deux raisons essentielles :
				\begin{itemize}
					\item L’article 24 f) ne traite que des adaptations rendues nécessaires par les lois postérieures ; c’est donc aller bien au-delà du texte que tenir compte également de la jurisprudence.
					\item S’il est admis que la jurisprudence est source de droit, il n’en demeure pas moins qu’il n’existe pas en droit français d’arrêts de règlement. Chaque décision s’applique à une espèce déterminée, contrairement à la loi, la jurisprudence n’a pas vocation à l’universalité.
				\end{itemize}
			
			\paragraph{L’adaptation du règlement de copropriété et répartition des charges}
			
				Nous arrivons ici à la question essentielle : peut-on, sous couvert d’adaptation nécessaire du règlement de copropriété, modifier la répartition des charges au motif que la répartition figurant au règlement de copropriété n’est pas conforme aux nouvelles règles de droit ?
				
				Notons tout d’abord que ces règles n’ont pas changé depuis 1965. En conséquence, la question de l’aggiornamento des charges dans le règlement de copropriété ne peut concerner que les règlements de copropriété antérieurs à la loi du 10 juillet 1965.
				
				En revanche les répartitions de charges antérieures à 1965, sauf hasard heureux, ne faisaient, par définition, pas application des critères de l’article 5 de la loi de 1965. Dès lors que ces critères n’ont pas été appliqués, la publication de la loi du 10 juillet 1965 semble « rendre nécessaire » l’adaptation de ces charges.
				
				Pourtant plusieurs arrêts de cours d’appel avaient condamné une telle révision, pour un règlement rédigé sous l’empire de la loi de 1938 (CA Aix en provence 23 avril 2010 JD 2010-012830,) et dans le même sens BORDEAUX 26 FEVRIER 2010 JD 2010-003758 ; VERSAILLES 12 AVRIL 2010 JD 2010-011329

				\subparagraph{La question a fait l’objet d’un arret de la cour de cassation du 23 mai 2012}\footnote{civile 3\degres Chambre, 23 mai 2012, \no de pourvoi: 10-28619 ; Loyers et Copropriété sep 2012 \no 244}
			
				Le Règlement de copropriété avait été établi en février 1965, donc avant la loi du 10 juillet 1965 et dispensait le propriétaire du rez-de-chaussée de participer aux charges d’ascenseur alors que celui-ci desservait les sous-sols. L’assemblée générale en application de l'article 49 (aujourd’hui 24 f) de la loi du 10 juillet 1965, a décidé d'adapter le règlement de copropriété aux dispositions législatives en vigueur et avait voté en 2006 cette adaptation comportant une nouvelle répartition de charges d’ascenseur. Un copropriétaire opposant avait demandé au Tribunal l’annulation de ce vote en soutenant que la « répartition des charges de copropriété ne peut être modifiée qu'à l'unanimité des copropriétaires et ne peut relever de la procédure simplifiée prévue à l'article 49 de la loi du 10 juillet 1965 ».
				
				La Cour d’Appel de Pau avait rejeté le recours du copropriétaire.
				
				La cour de cassation rejette le pourvoi, au motif que la Cour d’Appel « a exactement retenu qu'il y avait lieu, en application de la loi dite SRU du 13 décembre 2000, de réexaminer, à la majorité de l'article 24 de la loi du 10 juillet 1965, cette disposition au regard de l'article 10 de la même loi qui dispose que les copropriétaires sont tenus de participer aux charges entraînées par le services collectifs et les éléments d'équipement communs en fonction de l'utilité qu'ils présentent à l'égard de chaque lot ».
				
				Bien que cet arrêt ne soit pas publié au Bulletin de la Cour de Cassation, il retient nettement la possibilité pour l’assemblée générale de modifier les charges en application de l’article 24 f) lorsque ces charges ont été établies avant 1965 … en tout cas s’agissant de charges afférentes aux éléments d’équipement commun dès lors que le Règlement de copropriété d’origine ne faisait pas application du critère de l’utilité. La solution pourrait être différente s’agissant des charges communes générales dès lors qu’il y aurait seulement un calcul non conforme et non pas la méconnaissance de l’un des critères légaux.
				
				Dans le même sens on citera un arrêt (de rejet) de la 3ème chambre civile du 8 avril 2014 \no de pourvoi: 13-11633, ayant validé une ventilation des charges d’un Règlement de copropriété antérieur à 1965 faite en tantièmes généraux, dès lors que « que par application de l'article 10 de la loi du 10 juillet 1965, d'ordre public, cette globalisation des charges était devenue impossible, une distinction devant être opérée entre les charges relatives à la conservation, l'entretien et l'administration des parties communes et celles entraînées par les services collectifs et les éléments d'équipement communs, la cour d'appel, qui a relevé, procédant à la recherche prétendument omise, que l'article 16 du nouveau règlement de copropriété appliquait cette distinction et reprenait par ailleurs l'énumération des charges générales prévues dans le règlement initial sans modifier la clé de répartition qui y était prévue. » 
	
\section{Effets du règlement de copropriété}
	
	\subsection{Les effets a l'égard des copropriétaires}
	
		\subsubsection{Effet Obligatoire à l’encontre de tout copropriétaire}
		
			Le règlement de copropriété a un effet obligatoire pour tous les copropriétaires. Il s'ensuit que les copropriétaires sont tenus d'exécuter les diverses charges et obligations que leur impose le règlement sans pouvoir s'en dégager par une décision unilatérale, quelle que soit la raison invoquée : répartition prétendument inéquitable des charges, non-usage des services communs, infractions commises par certains copropriétaires.
			
			Il en sera de même de toute décision d’assemblée générale modifiant le Règlement de Copropriété\footnote{Du moins tant qu’un tribunal n’en aura pas relevé la nullité ou l’inexistence.} : une telle modification est opposable de plein droit aux copropriétaires, c’est à dire à ceux qui possèdent un lot lors du vote de la modification du Règlement de Copropriété, qu’ils aient voté ou non en faveur de cette modification.
			
			Inversement, tout copropriétaire peut se prévaloir des droits qu'il tient du règlement : par exemple la fourniture des services collectifs promis par le règlement auxquels les autres copropriétaires auraient renoncé.
			
			Les clauses du règlement s'imposent à partir de son entrée en vigueur, c'est-à-dire à la date de sa publication. {\bfseries Le règlement, bien que non-publié, est cependant opposable aux copropriétaires initiaux qui y ont personnellement adhéré}. Les modifications qui lui sont apportées sont opposables de plein droit à ceux qui avaient la qualité de copropriétaires lors du vote de la modification ; le défaut de publication n’a pas pour effet de rendre le modificatif inopposable au syndicat des copropriétaires\footnote{Civ 3\degres, 19 novembre 2008 \no 06-12.567}.
		
		\subsubsection{Les clauses réputées non écrites}
		
			Il convient de noter que les copropriétaires ne sont tenus que des dispositions licites du règlement.
	
			Or, nous savons qu'aujourd'hui les clauses du règlement contraires aux articles impératifs de la loi de 1965 sont réputées non-écrites. La Cour de cassation décide que << les clauses réputées non écrites par l'article 43 de la loi du 10 juillet 1965 sont non-avenues par le seul effet de la loi >>\footnote{Civ. 3ème, ler avril 1987, J.C.P. 1987 IV 201, Administrer août-septembre 1987, p.52 note GUILLOT.}.
			
			Il semble donc que de telles clauses soient considérées comme inexistantes plutôt que nulles. Il y a des différences entre nullité et inexistence :
			\begin{itemize}
				\item la nullité doit être demandée dans un certain délai (délai de prescription) alors qu'aucune limitation de temps n'existe pour faire constater l'inexistence d'une clause contraire aux dispositions d'ordre public de la loi de 1965 : Civ. 3eme, 26 avril 1989, Bull. Civ. III \no93, Rep. Defrénois 1989 art.34633, obs. H.SOULEAU ;
				
				\item la nullité doit nécessairement être constatée par un Tribunal, ce qui n'est pas le cas de l'inexistence.
			\end{itemize}
		
		\subsubsection{La prohibition de l'Atteinte aux Droits Acquis.}
		
			Le règlement ne peut avoir d'effet rétroactif en ce sens qu'il ne peut porter atteinte à des droits acquis antérieurement à sa mise en application, notamment à l'exercice d'un commerce exploité précédemment.
			
			De la même façon, le règlement de copropriété ne pourra être modifié en touchant aux droits acquis par un copropriétaire depuis la mise en copropriété de l’immeuble.
			
			Cette notion de droits acquis est particulièrement difficile à cerner : nous pensons avec le Professeur ATIAS que le droit acquis n’est pas un droit régulièrement constitué à l’origine. Par exemple, il n’y a pas atteinte aux droits acquis lorsque les droits invoqués sont dans le règlement de copropriété. Un droit acquis en copropriété doit s’analyser comme résultant d’une décision d’assemblée générale ou d’un fait juridique irréguliers mais qui n’ont pas été remis en cause légalement : par exemple, le fait pour un copropriétaire d’affecter son lot pendant plus de dix ans à un usage prohibé par le règlement de copropriété.
			
			De plus il ne faut pas confondre un « droit acquis » avec un « droit précaire » ; ce dernier s’analysant en une simple tolérance qui peut toujours être remise en cause.
			
			Sur l’ensemble de la question, lire la remarquable étude de Monsieur Jean Marc Le Masson\footnote{Les droits acquis en copropriété, Administrer mai 2002 p 11 et s.}
	
	\subsection{Les effets a l'égard des tiers}
	
		\subsubsection{Les Ayants-Cause à titre universel.}
		
			Conformément aux principes du droit commun, les effets obligatoires du règlement s'étendent à tous les ayants-cause à titre universel des copropriétaires : héritiers, légataires universels et à titre universel.
			
			Il en résulte qu’un nouveau Règlement de Copropriété voté par l’assemblée générale s’impose aux héritiers du copropriétaire, ce alors même que ce nouveau règlement n’a pas été publiée au fichier immobilier\footnote{Civ 3\degres, 22 nov. 2000 Construction Urbanisme 2001 \no 29, note Sizaire.}.
			
			La même règle s’applique sans aucun doute aux modifications du Règlement de Copropriété votées en assemblée générale mais non encore publiées.
		
		\subsubsection{Les ayants-cause à titre particulier}\footnote{Ce sont les donataires, légataires à titre particulier ou acquéreurs.} 
		
			\paragraph{Opposabilité par la publication au fichier immobilier}
			
			L'article 13 de la loi du 10 juillet 1965 dispose :
			<< {\itshape le règlement de copropriété et les modifications qui peuvent lui être apportées ne sont opposables aux ayants-cause à titre particulier des copropriétaires qu'à dater de leur publication au fichier immobilier}\footnote{Le fichier Immobilier est un terme général qui vise à la fois la Conservation des Hypothèques (pour la France de l’Intérieur) et le Service du Livre Foncier qui existe en Alsace-Moselle.}. >>
			
			Les ayants-cause à titre particulier sont des personnes qui ont acquis un droit réel déterminé sur le lot, en pratique les acquéreurs du lot : acheteurs, donataires, légataires particuliers, coéchangistes \etc
			
			En contrepartie, la cession du lot a pour effet de libérer le cédant des obligations imposées par le règlement de copropriété. Ces obligations sont attachées au lot. Elles se transmettent avec lui. Ce sont des obligations "propter rem" (à cause de la chose et non à cause de la personne).
	
			L'acte constatant le transfert de propriété du lot doit mentionner que l'acquéreur a eu préalablement connaissance du règlement de copropriété ainsi que des actes qui l'ont modifié, s'ils ont déjà été publiés (art.4 al .ler du Décret du 17 mars 1967).
			
			Le non respect de cette mention à l'acte de transfert est sans conséquence dans les relations du syndicat et de l'acquéreur puisque le Règlement de Copropriété lui est opposable du seul fait de sa Publication (article 13 de la loi); par contre une telle omission peut entraîner la responsabilité du notaire rédacteur de l'acte de transfert.
		
		- Opposabilité par adhésion expresse au règlement dénoncé à l’acquéreur
		Par ailleurs, l'alinéa 3 de l'article 4 du Décret du 17 mars 1967 admet l'opposabilité du règlement même non publié aux ayants cause à titre particulier si ces derniers en ont eu effectivement connaissance et ont adhéré aux obligations qui en résultent:
		" Le Règlement de Copropriété, l'État descriptif de division et les actes qui les ont modifiés, même s'ils n'ont pas été publiés au fichier immobilier, s'imposent à l'acquéreur ou titulaire du droit s'il est expressément constaté aux actes visés au présent article qu'il en a eu préalablement connaissance et qu'il a adhéré aux obligations qui en résultent".
		Le texte exige que soient remplies à la fois les deux conditions précitées,
		- Connaissance effective,
		- Adhésion exprès,
		A défaut de l'une seulement de ces deux conditions, et notamment lorsque l'acte précisé au titre des déclarations du vendeur "qu'à la connaissance de ce dernier il n'y a pas eu de modification au Règlement de Copropriété", l'acquéreur ne peut se voir opposer ces modifications.
		Un arrêt de la cour de Cassation du 8 septembre 2009 rappelle cette évidence à propos d’une modification de répartition de charges adoptée par une assemblée générale non publiée353 :
		...Attendu, selon l'arrêt attaqué (Aix en Provence, 25 janvier 2008), que la société civile immobilière Michel Ange (la SCI) a acquis par acte notarié du 20 septembre 1999, de la société civile immobilière Détroit plusieurs lots de copropriété ; que l'acte se réfère à un cahier des charges et règlement de
		353 8 septembre 2009 Cass.3è civ. pourvoi \no 08-15146
		droit de la copropriété année 2019-2020
		284
		copropriété établi le 12 septembre 1950, à un état descriptif de division établi le 2 mars 1960, à un état modificatif du 20 avril 1967 et à un état descriptif complémentaire établi le 15 février 1980, tous ces actes ayant été publiés au bureau des hypothèques ; que la SCI a assigné le syndicat des copropriétaires pour que les appels de charges qui lui ont été adressés depuis son acquisition lui soient déclarés inopposables ;
		Attendu que pour rejeter la demande de la SCI, l'arrêt retient que l'assemblée générale extraordinaire des copropriétaires réunis le 5 mai 1998 a approuvé les modifications aux tableaux de répartition des charges avec une nouvelle numérotation des lots, dans les termes joints au procès-verbal de l'assemblée générale, portant notamment modification des millièmes de parties communes afférents audits lots, que la société civile immobilière
		Le Détroit ex propriétaire des lots acquis par la société civile immobilière Michel Ange avait été invitée à participer à ladite assemblée générale et n'a pas contesté les délibérations prises et que le premier juge a retenu à juste titre que les décisions prises par cette assemblée sont devenues définitives et ce d'autant que contractuellement la société civile immobilière Michel Ange est tenue par les obligations de son auteur ;
		Qu'en statuant ainsi, sans constater que les modifications du règlement de copropriété votées lors de l'assemblée générale du 5 mai 1998 avaient été publiées au fichier immobilier ou expressément mentionnées dans l'acte d'achat de la SCI avec adhésion aux obligations qui en résultent, la cour d'appel n'a pas donné de base légale à sa décision.
		Toutefois, le juge du fond a trop souvent tendance à déclarer opposable le modificatif dont l'acte fait mention sans qu'ait été recueilli l'accord exprès de l'acquéreur354. .
		Cette opposabilité permet au syndicat d'exiger à l'encontre de l'acquéreur la démolition d'ouvrages réalisés par le précédent copropriétaire en violation des droits de la Copropriété. L'acquéreur est tenu au lieu et place du vendeur !
		3. Les locataires
		En droit strict, le locataire d'un lot en copropriété est étranger au règlement qui, dès lors, ne devrait pas lui être opposable, sauf si une clause du bail lui en imposait le respect. En ce dernier cas, c'est par l'action contractuelle née du bail que le propriétaire-bailleur peut obliger le locataire à se conformer aux dispositions du règlement.
		Cette conception a paru exagérément juridique et la jurisprudence a très vite considéré que les copropriétaires soumis au règlement ne peuvent conférer aux tiers plus de droits qu'ils n'en ont eux-mêmes dans l'organisation collective et que par conséquent le locataire "ne dispose pas de droits plus étendus que ceux dont jouit son bailleur".
		354 sur la notion d'adhésion cf. PARIS 19\degres 17 mai 1991 RTDI 91.384
		droit de la copropriété année 2019-2020
		285
		Les locataires sont assimilés à de véritables ayants cause à titre particulier des copropriétaires et doivent, lorsqu'ils ont été informés des clauses du règlement, s'y conformer. Ils peuvent être directement poursuivis par le syndic en cas d'inobservation de ces clauses : par exemple, il les fera condamner à cesser d'encombrer les parties communes en y déposant des marchandises355, à la démolition en cas de travaux irrégulièrement effectués sur les parties communes356 ou en cas d'empiétement ou de stationnement abusif sur les parties communes357 .
		Pour que l'opposabilité du règlement au locataire puisse fonctionner sans encourir la moindre objection, la loi QUILLOT (22 juin 1982), puis la loi MEHAIGNERIE (23/12/1986) ont imposé au bailleur d’un local à usage d’habitation d'annexer au contrat de location des extraits du règlement de copropriété concernant les clauses déterminant la destination de l'immeuble, la jouissance et l'usage des parties privatives et communes ainsi que la quote-part afférente aux lots loués dans chacune des catégories de charges. Ces dispositions sont reprises dans la loi du 6 juillet 1989.
		En outre, le copropriétaire reste responsable des infractions de son locataire
	
	\subsection{Les sanctions du règlement de copropriété}
	
		Les sanctions de l'inobservation du règlement de copropriété sont les mêmes que celles qui sont applicables en cas d'inexécution d'un contrat : exécution en nature et dommages intérêts, ces deux sanctions étant d'ailleurs cumulables.
		l) Les personnes qui peuvent demander que l'infraction au règlement soit sanctionnée sont le syndicat par la voix de son syndic ainsi que tout copropriétaire justifiant d'un intérêt à agir.
		Le syndicat est recevable à agir parce qu’il doit veiller à la conservation de l'immeuble et à l'administration des parties communes. Il a donc qualité pour exiger l'exécution des obligations stipulées au règlement.
		Chaque copropriétaire a également cette qualité en tant que partie contractante au règlement violé.
		2) La Cour de cassation se montre extrêmement rigoureuse à l'encontre de ceux qui auraient méconnu les dispositions du règlement de copropriété. Elle les condamne, au besoin sous
		355 Civ. 3ème, 20 octobre 1981, Bull. Civ. III \no162
		356 Paris, 18 mars 1987, D.1987 I.R.99
		357 Civ. 3ème, 5 avril 1968, Bull. Civ. III \no158; Grenoble 13 octobre 1965, D.1966 168
		droit de la copropriété année 2019-2020
		286
		astreinte, à effacer toutes les traces de leur comportement illicite, c'est-à-dire à l'exécution en nature des obligations résultant du règlement, alors même que cette exécution serait extrêmement difficile (Civ. 3ème, 13 octobre 1981, Bull. Civ. III \no152), ou que le préjudice causé n'aurait été que minime (Civ. 3ème, 18 juillet 1972, Bull. Civ. III \no39). Il suffit que par des entreprises illicites le règlement ait été violé pour que la responsabilité contractuelle du contrevenant soit engagée : ni la faute du contrevenant, ni l'existence d'un préjudice n'ont à être prouvés.
		Cette exécution en nature peut se traduire par la démolition de constructions irrégulièrement édifiées, le rétablissement d'installations enlevées sans droit. De telles condamnations ont été prononcées au cas d'empiétement sur des parties communes ou d'enlèvement d'une marquise ou d'une enseigne : Grenoble, 11 octobre 1965, D.1966 168 note GIVORD. La condamnation sous astreinte peut aussi être prononcée au cas de stationnement interdit dans les parties communes (Paris, 20 octobre 1970, J.C.P. 1971 II 16765 note DESIRY), ou la location d'une chambre de service en infraction avec le règlement (surpeuplement).
		Rappelons cependant que ces condamnations ont été prononcées en application de l’article 1143 du code civil aux termes duquel : Article 1143 - « le créancier a le droit de demander que ce qui aurait été fait par contravention à l'engagement soit détruit ; et il peut se faire autoriser à le détruire aux dépens du débiteur, sans préjudice des dommages et intérêts s'il y a lieu »
		Cet article sera remplacé à compter du 1er octobre 2016 par les articles 1221 et 1222 ainsi rédigés :
		« Art. 1221. – Le créancier d’une obligation peut, après mise en demeure, en poursuivre l’exécution en nature sauf si cette exécution est impossible ou s’il existe une disproportion manifeste entre son coût pour le débiteur et son intérêt pour le créancier.
		« Art. 1222. – Après mise en demeure, le créancier peut aussi, dans un délai et à un coût raisonnables, faire exécuter lui–même l’obligation ou, sur autorisation préalable du juge, détruire ce qui a été fait en violation de celle–ci. Il peut demander au débiteur le remboursement des sommes engagées à cette fin.
		« Il peut aussi demander en justice que le débiteur avance les sommes nécessaires à cette exécution ou à cette destruction.
		Toutefois ces dispositions ne s’appliqueront qu’aux contrats conclus après le 1er octobre 2016 en sorte que nous risquons d’avoir une justice à deux vitesses : très sévère pour les infractions aux Règlements de copropriété antérieurs à cette date et plus souples pour les Règlement de copropriété postérieurs !
		3) Si la violation du règlement par un copropriétaire provoque un trouble à un autre dans la possession de son lot, l'action possessoire ne peut être intentée. En effet, la protection
		droit de la copropriété année 2019-2020
		287
		possessoire n'est pas accordée à celui qui demande la cessation d'un trouble provenant de l'inexécution d'un contrat : Civ. 3ème, 22 juin 1976, Rep. Defrénois 1977 art.31350 note H.SOULEAU; Civ. 3ème, 10 juin 1980, Rep. Defrénois 1981 art.32686, note H.SOULEAU.
		4) La procédure d’injonction de faire prévue par le Décret du 4 mars 1988 (articles 1425-1 à 1425-9) peut se concevoir en cas d'inexécution d'obligations résultant du Règlement de Copropriété. Aux termes de cette procédure, le demandeur (le syndicat des copropriétaires par exemple), sollicite du juge du Tribunal d'Instance, par voie de requête déposée au greffe, que soit délivrée injonction au défendeur (le copropriétaire en infraction), en cas de non respect du contrat, de faire ce qui doit être fait pour parvenir à l'exécution de l'obligation contractuelle. Le juge fixe l'objet de l'obligation, le délai et les conditions dans lesquelles l'obligation doit être exécutée ainsi que la date à laquelle il entendra les parties, notamment pour s'assurer que son injonction a été suivie d'effet. En fait la jurisprudence ne révèle pas d'exemple d'application de l'injonction de faire à la Copropriété, dans la mesure où l'injonction de faire doit rester dans les limites du taux de compétence du Tribunal d'Instance et alors que le plus souvent l'obligation mise à la charge du copropriétaire est d'un montant non défini.
		5) Le règlement de copropriété comme toute convention peut contenir une clause pénale, c'est-à-dire la fixation d'une somme forfaitaire dissuasive, conformément à l'article 1153 du Code civil, pour le cas d'inexécution.358
		Il peut prévoir aussi des intérêts de retard au cas où un copropriétaire ne paierait pas ses charges ou appels de fonds en temps utile.359 Mais la question est discutée de savoir si l'intérêt stipulé peut être supérieur au taux légal.360.
		La prescription de 10 ans de l’article 42 alinéa 1e de la loi du 10 juillet 1965 court du jour où la violation du règlement de copropriété (exercice d’une activité contraire au règlement de copropriété) a été commise361.
		358 Civ. 3ème, 30 octobre l973, J.C.P. 1973, IV 402.
		359 Paris 30 octobre 1979, D.1980 I.R.24, obs. GIVERDON
		360 pour l'affirmative, voir Paris 30 octobre 1979 précité
		361 CA Paris 2 juillet 2009, JD 2009-378837 et Cass. Civ. 3e 8 septembre 2009
		\chapter{L'état descriptif de division}
		\chapter{La définition légale et la répartition des charges de copropriété}
		\chapter{La modification des charges}
		\chapter{La comptabilité du syndicat des copropriétaires}

Aux termes de l'article 18 de la loi du 10 juillet 1965 (disposition d'ordre public) le syndic de la copropriété
- tient la comptabilité du syndicat des copropriétaires,
- établit le budget prévisionnel,
- établit les comptes du syndicat et leurs annexes.
La comptabilité du syndicat des copropriétaires fait l'objet des articles 14-1 à 14-3 de la loi du 10 juillet 1965 issus de la loi S.R.U. du 13 décembre 2000 et des deux décrets d'application des 27 mai 2004 et 14 mars 2005.
Les articles de loi sont ainsi rédigés :
Art. 14-1 –Pour faire face aux dépenses courantes de maintenance, de fonctionnement et d'administration des parties communes et équipements communs de l'immeuble, le syndicat des copropriétaires vote, chaque année, un budget prévisionnel. L'assemblée générale des copropriétaires appelée à voter le budget prévisionnel est réunie dans un délai de 6 mois à compter du dernier jour de l'exercice comptable précédent.
Les copropriétaires versent au syndicat des provisions égales au quart du budget voté. Toutefois, l'assemblée générale peut fixer des modalités différentes.
La provision est exigible le premier jour de chaque trimestre ou le premier jour de la période fixée par l'assemblée générale.
Art. 14-2. Ne sont pas comprises dans le budget prévisionnel des dépenses pour travaux dont la liste sera fixée par décret en conseil d'état.
Les sommes afférentes à ses dépenses sont exigibles selon les modalités votées par l'assemblée générale.
Art. 14-3. Les comptes du syndicat comprenant le budget prévisionnel, les charges et produits de l’exercice, la situation de trésorerie, ainsi que les annexes au budget prévisionnel sont établis conformément à des règles comptables applicables spécifiques fixées par décret. Les comptes sont présentés avec comparatif des comptes de l’exercice précédent approuvé.
« Les charges et les produits du syndicat, prévus au plan comptable, sont enregistrés dès leur engagement juridique par le syndic indépendamment de leur règlement ou dès réception par lui des produits. L’engagement est soldé par le règlement. »
droit de la copropriété année 2019-2020
383
Ces nouveaux articles introduisent des règles de gestion comptable très strictes dans le statut de la copropriété qui en été jusqu’alors dépourvu (I). Ces règludgetes concernent l’approbation du budget (II) et des comptes (III)
SECTION I - HISTORIQUE DES REGLES COMPTABLES
A. LA LOI DU 10 JUILLET 1965
1. 1. L'absence de normes comptables dans la loi du 10 juillet 1965.
La comptabilité du syndicat des copropriétaires ne répondait à aucune obligation normative en sorte que si chaque syndic devait tenir la comptabilité des syndicats qu’il administrait, il le faisait à sa guise, selon ses propres recettes.
Les seules dispositions impératives dans le décret du 17 mars 1967 concernaient la reddition des comptes, puisqu'au terme de l'article 11 le syndic devait adresser aux copropriétaires avec la convocation à l'assemblée générale devant approuver les comptes et le budget :
- le compte des recettes et dépenses de l'exercice écoulé,
- l'état des dettes et créances,
- la situation de trésorerie,
- le budget prévisionnel.
La sixième recommandation de la commission relative à la copropriété avait tenté une certaine unification de présentation des comptes :
- recommandant au syndic l'utilisation d'appellations de catégories identiques pour tous les immeubles (charges communes générales, charges communes un groupe d'immeubles, charges de bâtiment, etc...),
- recommandant ensuite au syndic de faire une présentation identique des comptes d'un exercice sur l'autre et même de joindre à ses comptes une notice explicative,
- proposant des tableaux de présentation de l'état des dettes et des créances et de la situation de trésorerie.
droit de la copropriété année 2019-2020
384
2. 2. Comptabilité de trésorerie et comptabilité d'engagement.
Il existe essentiellement deux types de comptabilité :
- la comptabilité de trésorerie, selon laquelle les charges sont enregistrées lorsqu’elles sont décaissées et les recettes sont enregistrées lorsqu’elles sont encaissées. C'est le type même de la comptabilité familiale dite encore « comptabilité de ménagère ».
- La comptabilité d'engagement, selon laquelle les charges sont enregistrées le jour où elles font l'objet d'une obligation vis-à-vis du créancier, tandis que les produits sont enregistrés à la date d'acquisition de la créance. C'est le type même de la comptabilité commerciale.
Ni la loi du 10 juillet 1965 ni le décret du 17 mars 1967 n'imposaient aux syndics pour la tenue des comptes des copropriétés qu'ils administraient d'adopter l'un ou l'autre de ces deux types de comptabilité494. Étant simplement observés que les copropriétaires ne sont pas forcément commerçants ou polytechniciens et qu’il leur paraît plus facile de contrôler une comptabilité de ménagère qu’une comptabilité de type commercial.
La rédaction des articles de la loi et du décret faisaient plutôt référence à des notions tirées de la comptabilité de trésorerie plutôt que de la comptabilité d'engagement : ne serait-ce que l'obligation faite au syndic dans l'ancien article 11 du décret de joindre à la convocation le compte des « recettes » et le compte des « dépenses ». En sorte que le plus souvent, c'est effectivement une comptabilité de trésorerie qui était tenue par le syndic de la copropriété.
3. 3. Système de provision sur charges et système des charges échues.
A l’origine, les syndics de copropriété pouvaient opter entre deux systèmes opposés de comptabilité :
(1) Gestion en charges échues.
Selon ce système, le syndic paie au fur et à mesure les dépenses de la Copropriété. Trimestriellement, il demande aux copropriétaires de rembourser leurs quotes-parts de ces charges.
494 Étant observé que le syndic exerçant généralement sous forme commerciale doit tenir la comptabilité de son cabinet en la forme d'une comptabilité d'engagement.
droit de la copropriété année 2019-2020
385
Les copropriétaires ont un contrôle trimestriel des dépenses du syndicat. Les sommes appelées par le syndic sont les sommes dont les copropriétaires sont effectivement débiteurs après répartition dans les termes du Règlement de Copropriété.
En revanche, les charges trimestrielles varient en fonction des dépenses réalisées, et le syndic peut être amené à appeler à titre de charges des dépenses effectuées qui n'ont pas été préalablement décidées par l'Assemblée Générale.
(2) Gestion par provisions.
Dans ce type de système, le syndic appelle des provisions trimestrielles en quatre fractions d'un même montant, destinées à couvrir les dépenses d'un exercice annuel.
A l'issue de l'exercice le syndic compare le montant des provisions reçues et le montant des dépenses réellement réalisées. Cette comparaison permettra de constater si les provisions ont été supérieures aux dépenses ou si inversement les dépenses ont été supérieures aux provisions reçues.
En conséquence le syndic, après approbation des comptes par l'Assemblée, adressera aux copropriétaires un récapitulatif annuel aux termes duquel il sollicitera le complément nécessaire pour ajuster les dépenses et les appels de fonds... ou il créditera les copropriétaires des sommes appelées au-delà des dépenses réalisées.
Ce système présente l'avantage d'être parfaitement compris des copropriétaires; de plus il est sans surprise puisque dès le début de l'exercice chaque copropriétaire connaîtra le montant trimestriel des "provisions sur charges" qu'il devra adresser au syndic.
Il permet aux copropriétaires de recevoir un récapitulatif annuel des dépenses dont le contrôle sera facile.
Mais il présente aussi certains inconvénients :
- Le syndic n'a pas l'obligation d'adresser un récapitulatif des dépenses trimestrielles; en sorte que pendant toute une année le copropriétaire peut rester dans l'ignorance des dépenses réelles de la copropriété. Mais, en pratique les syndics qui pratiquaient le système des appels trimestriels envoyaient avec ces appels un récapitulatif des dépenses du trimestre.
- Les appels provisionnels sont appelés en tantièmes généraux et non au prorata de la participation de chacun dans les différentes dépenses de la Copropriété. En sorte qu'en fin d'exercice, certains copropriétaires auront trop payé et d'autres pas assez :
A titre d'exemple on évoquera la question des charges de chauffage : certains copropriétaires ne sont pas raccordés au chauffage central dont on sait qu'en copropriété il peut représenter 40 à 45 % de l'ensemble des charges. Dans le système des "provisions", le copropriétaire non raccordé se
droit de la copropriété année 2019-2020
386
verra appeler des charges tenant compte des dépenses de chauffage, en sorte qu'au moment de l'éclatement des dépenses, il apparaîtra comme ayant trop payé.
Mais il est vrai que rien n'interdit dans le système des appels provisionnels de répartir ces appels à proportion de la participation de chacun dans les différentes catégories de charges.
- Enfin, ce système ne permet pas de savoir instantanément les sommes dont un copropriétaire est débiteur envers le syndicat. Cette connaissance était cependant nécessaire lorsqu'en cas de mutation à titre onéreux le notaire interrogeait le syndic sur les sommes dues à la Copropriété par le copropriétaire vendeur.
B. LA LOI SRU DU 13 DECEMBRE 2000 ET SES TEXTES D'APPLICATION.
Dès 1985 l’idée avait été émise de créer une comptabilité spécifique pour les syndicats de copropriété (essentiellement en imposant un plan comptable aux syndicats). Mais la mise en oeuvre de cette idée parut trop compliquée pour être retenue à cette époque.
1. La loi du 13 décembre 2000.
La loi SRU va créer un véritable statut comptable du syndicat des copropriétaires en introduisant trois nouveaux articles dans la loi du 10 juillet 1965, ce sont les articles 14 – 1, 14 – 2 et 14 – 3.
Ces articles vont poser les principes suivants :
o une distinction doit être faite sans équivoque entre les dépenses sur budget (dépenses habituelles de fonctionnement et de maintenance courante de la copropriété) et les dépenses hors budget (correspondant essentiellement aux travaux et contrats exceptionnels) ;
o les provisions sur budget doivent être nécessairement appelées auprès des copropriétaires (normalement par trimestre) en fonction de la quote-part due par ces copropriétaires dans le budget voté ; la gestion en charges échues étant désormais interdite (article 14-1);
o les dépenses hors budget doivent faire l'objet d'un vote spécifique précisant la date d'exigibilité des provisions nécessaires à leur règlement (article 14-2);
o la comptabilité du syndicat des copropriétaires est une comptabilité d'engagement ;
o la comptabilité du syndicat est établie conformément à des règles comptables spécifiques qui seront fixées par décret (article 14-3).
Aux termes de la loi SRU (article 75 – III) l'article 14 – 3 relatif aux règles de comptabilité spécifique devaient entrer en vigueur le 1er janvier 2004.
droit de la copropriété année 2019-2020
387
Cette date d'entrée en vigueur a été retardée à plusieurs reprises :
o la loi du 2 juillet 2003 (Urbanisme et Habitat) a reporté l'application de cette disposition au 1er janvier 2005
o la loi du 18 janvier 2005 (Programmation pour la Cohésion Sociale) a reporté à son tour l'application de cette disposition au 1er janvier 2006
o la loi ENL du 13 juillet 2006, modifiant l’article 75 de la loi du 13 décembre 2000, a finalement imposé que les comptes du syndicat des copropriétaires soient tenus conformément aux normes édictées par le décret comptable « à partir du 1er exercice comptable commençant à compter du 1er janvier 2007 ».
En fait, ce retard était dû notamment495 à la discussion qui s'est instaurée sur l'utilité d'une comptabilité d'engagement pour les petites copropriétés.
Tant les syndicats professionnels de syndics (FNAIM, CNAB) que les associations de consommateurs (UNARC) ont souhaité en effet que les petites copropriétés soient dispensées d'une comptabilité d'engagement tenue en partie double. Les associations de consommateurs faisant valoir notamment que cette exigence était de nature à faire disparaître le syndic bénévole, tandis que les syndicats professionnels objectaient que le système mis en oeuvre constituait une véritable usine à gaz, de nature à augmenter les frais de gestion des copropriétés, sans véritable justification pour les copropriétaires.
Cette polémique devait être notamment entretenue par l'avis du Conseil National de la Comptabilité du 22 octobre 2002 qui invitait le gouvernement à définir une comptabilité simplifiée pour les petites copropriétés.
Si les services du ministre du logement devaient se montrer favorables à cette dualité de comptabilité distinguant entre les grandes et petites copropriétés, par contre la Chancellerie conservait une position hostile à toute dualité, estimant avec le président CAPOULADE qu’il n'existe qu'une seule copropriété pour toute la France496.
Finalement, à l’occasion du vote de la loi ENL du 13 juillet 2006, a été prévue une dispense de comptabilité en partie double des copropriétés comportant moins de dix lots à usage de logements, de bureaux ou de commerces et dont le budget annuel moyen sur trois ans est inférieur à 15.000 €.(art 92 de la loi ENL)
Quant à la dernière prorogation du délai d'entrée en vigueur des dispositions de l’article 14 – 3 de la loi du 10 juillet 1965 postérieurement 1er janvier 2006 , elle s'explique essentiellement par la tardivité de
495 La loi SRU devait être complétée sur ce point par deux décrets : le Décret modifiant le Décret du 17 mars 1967 pour adapter celui-ci aux dispositions de la loi SRU (décret devant être rédigé par la Chancellerie et approuvé par le Conseil d’Etat) qui ne sera finalement publié que le 27 mai 2004 et le Décret « comptable » (rédigé par le Ministère du Logement et devant être publié après avis du Conseil National de la Comptabilité).
496 La difficulté était surtout de déterminer le seuil à partir duquel une copropriété cessait d'être une petite copropriété pour devenir une grande copropriété.
droit de la copropriété année 2019-2020
388
publication du décret du 14 mars 2005 empêchant de mettre en place utilement la formation des professionnels et d'assurer la fiabilité des logiciels de gestion.
2. Le Décret du 27 mai 2004.
Le décret traite en premier lieu du budget prévisionnel en ce qui concerne son établissement et son contenu.
Il précise ensuite les dispositions relatives à l'ouverture du compte séparé.
Il donne par ailleurs des définitions précises de ce que l'on doit entendre s'agissant des provisions sur charges, des avances et des charges et redéfinit les sommes dont le syndic peut exiger le paiement auprès des copropriétaires.
Enfin, il redéfinit les documents qui doivent être annexés à la convocation de l'assemblée générale lorsque celle-ci est appelée à approuver les comptes et à voter le budget.
3. Le décret et l'arrêté du 14 mars 2005.
Il s'agit cette fois du décret comptable qui comporte 13 articles et 5 annexes correspondant ou documents de synthèse. Ce décret :
\begin{itemize}
	\item définit le champ d'application des règles comptables
	\item donne la définition des charges et des produits
	\item fixe la durée de l'exercice comptable du syndicat
	\item définit les pièces justificatives des enregistrements comptables
	\item présente l’organisation comptable des comptes de la copropriété
\end{itemize}
quant aux cinq annexes, elles sont ainsi définies :
\begin{itemize}
	\item annexe 1 : état financier qui comprend la situation financière et l'état des dettes et des créances.
	\item annexe 2 : compte de gestion générale et budget prévisionnel.
	\item annexe 3 : compte de gestion des opérations courantes du budget prévisionnel.
	\item annexe 4 : compte de gestion pour travaux et opérations exceptionnelles
	\item annexe 5 : état des travaux et dépenses exceptionnels votés et non clôturés en fin d'exercice.
\end{itemize}
droit de la copropriété année 2019-2020
389
Rappelons que pour la validité de l’approbation des comptes par l’assemblée générale, ces cinq annexes doivent être jointes à la convocation.
Mais leur compréhension par les copropriétaires est tellement problématique que le Décret \no 2010-391 du 20 avril 2010 exige que soit adressé de plus :
« 5 \no En vue de l’approbation des comptes par l’assemblée générale, le projet d’état individuel de répartition des comptes de chaque copropriétaire. »
SECTION II - LE BUDGET ET LES DEPENSES HORS BUDGET
A. LES DEPENSES SUR BUDGET (DEPENSES COURANTES).
1. CE QUE COMPRENNENT CES DEPENSES.
Les dépenses courantes, sont définies par la loi comme relatives à la maintenance, au fonctionnement et à l'administration des parties communes et des éléments d'équipement commun.
Comme par le passé, ces dépenses devront faire l’objet d’un budget adopté par l’assemblée Générale.
2. EPOQUE D’ADOPTION DU BUDGET.
CE QUE DIT LA LOI.
Pour que le dispositif nouvellement mis en place par le législateur soit efficace, il était nécessaire que le budget de l'année soit voté en début d'exercice et non pas dans le courant du troisième ou du quatrième trimestre de l'année.
L'idée était à l'origine que l'assemblée générale votant le budget devrait se réunir dans le courant du premier trimestre de l'exercice. Les syndics ont opposé la difficulté matérielle qui résulterait d'une telle exigence : soit toutes les assemblées générales annuelles se tenaient dans les trois premiers mois, soit il fallait prévoir au moins deux assemblées annuelles, l'une pour l'approbation du budget et l'autre pour l'adoption des questions courantes ou exceptionnelles de la copropriété.
Finalement, le parlement a donné un délai de six mois au syndic pour convoquer l'assemblée générale qui aura à adopter le budget :
L ’article 14-1 édicte en conséquence que : « L'assemblée générale des copropriétaires appelée à voter le budget prévisionnel est réunie dans un délai de six mois à compter du dernier jour de l'exercice comptable précédent ».
droit de la copropriété année 2019-2020
390
CE QUE DIT LE DECRET :
En réalité il apparaît inopportun de ne voter le budget qu’en cours d’exercice. C’est pourquoi, le décret, modifiant quelque peu la loi sur ce point pose les règles suivantes :
Article 43 (Modifié par Décret 2004-479 2004-05-27 art. 31, art. 32 JORF 4 juin 2004.)
« Le budget prévisionnel couvre un exercice comptable de douze mois. Il est voté avant le début de l'exercice qu'il concerne.
Toutefois, si le budget prévisionnel ne peut être voté qu'au cours de l'exercice comptable qu'il concerne, le syndic, préalablement autorisé par l'assemblée générale des copropriétaires, peut appeler successivement deux provisions trimestrielles, chacune égale au quart du budget prévisionnel précédemment voté. La procédure prévue à l'article 19-2 de la loi du 10 juillet 1965 ne s'applique pas à cette situation. »
« Le budget prévisionnel couvre un exercice comptable de douze mois », en sorte que bien évidemment les copropriétés peuvent prévoir un exercice comptable qui ne coïncide pas avec l’année civile : l’année peut commencer le 1er octobre par exemple.
En conséquence la pratique normale est de voter pendant l’année » n » le budget de l’année « n+1 », en sorte que le budget ne sera voté pendant l’année qu’il concerne que si l'assemblée générale n’a pu voter le budget avant le début de l’exercice.
A titre exceptionnel, si l'assemblée générale n’a pu adopter le budget avant le début de l’exercice qu’il concerne, le syndic pourra appeler jusqu’à deux fois ¼ du budget de l’année précédente… à condition toutefois que cette possibilité ait été donné au syndic par une assemblée générale préalable.
Comme on ne connaît jamais l’avenir, il serait bon que la résolution d’adoption du budget précise que « si l’assemblée générale ne peut être convoquée pour l’approbation de l’année n+2, le syndic pourra appeler successivement deux provisions trimestrielles, chacune égale au quart du budget prévisionnel présentement voté ».
Enfin, on peut penser qu’en cas de dérapage constaté au cours de l’année n, le syndic pourra inscrire à l’assemblée générale de l’exercice n+1 le vote d’un complément de budget. Mais il est vrai que le plan comptable ne comporte aucune ligne correspondant à un tel ajustement. D’où la question que posent certains experts comptables de la licéité d’un tel ajustement budgétaire.
droit de la copropriété année 2019-2020
391
3. EXIGIBILITE PAR QUART DU BUDGET.
article 14-1 al. 2 nouveau de la loi du 10 juillet 1965 :
« Les copropriétaires versent au syndicat des provisions égales au quart du budget voté. Toutefois, l'assemblée générale peut fixer des modalités différentes. »
Les dépenses courantes, telles que précédemment définies comme relatives à la maintenance, au fonctionnement et à l'administration des parties communes et des éléments d'équipement commun, font l'objet depuis le 1er janvier 2006 d'une comptabilité par provision.
De plus, le syndic ne peut plus demander aux copropriétaires le remboursement des dépenses acquittées ; il doit impérativement appeler les charges sur la base du budget prévisionnel.
Normalement, il y a quatre appels provisionnels par année. Mais la loi permet à l'assemblée générale de prévoir une périodicité différente. Cette décision est adoptée à la majorité de l'article 24 de la loi, puisqu’il n'est pas prévu de majorité particulière sur ce point.
On peut envisager que dans une petite copropriété l'assemblée générale prévoit deux appels provisionnels pour l'année.
Par contre, dans une copropriété importante, il peut être de l'intérêt de la personne morale du syndicat des copropriétaires de fixer des appels provisionnels mensuels. Auquel cas, le travail du syndic sera considérablement alourdi, sauf à prévoir en accord avec les copropriétaires un prélèvement mensuel automatique sur leur compte.
Article 14-1 alinéa 3 de la loi du 10 juillet 1965
« La provision est exigible le premier jour de chaque trimestre ou le premier jour de la période fixée par l'assemblée générale ».
Art 35-2 du Décret \no67-223 du 17 mars 1967 (Ajouté par le Décret \no 2004-479 du 27 mai 2004 – article 24)
Pour l'exécution du budget prévisionnel, le syndic adresse à chaque copropriétaire, par lettre simple, préalablement à la date d’exigibilité déterminée par la loi, un avis indiquant le montant de la provision exigible.
droit de la copropriété année 2019-2020
392
Si la provision est exigible comme il est dit à l’article 14-3 de la loi, il est nécessaire que le syndic adresse préalablement un avis d’exigibilité.
4. SANCTIONS AU DEFAUT DE PAIEMENT DE LA QUOTE-PART SUR BUDGET.
Cette sanction est édictée par l’article 19-2 de la loi qui prévoit une véritable déchéance de terme en sorte que le copropriétaire qui ne paie pas un appel trimestriel peut être condamné par le Président du TGI statuant comme en matière de référé à payer sa quote-part totale du budget annuel497.
B. LES DEPENSES POUR TRAVAUX : DEPENSES « HORS BUDGET ».
Article 14-2 de la Loi du 10 juillet 1965
« Ne sont pas comprises dans le budget prévisionnel les dépenses pour travaux dont la liste sera fixée par décret en Conseil d'Etat. Les sommes afférentes à ces dépenses sont exigibles selon les modalités votées par l'assemblée générale. »
1. Ce que comprennent les dépenses hors budget
Le « décret en Conseil d’Etat » du 27 mai 2004 n’a pas donné de liste à proprement parler, mais fixé des principes :
Art. 44. Les dépenses non comprises dans le budget prévisionnel sont celles afférentes :
1\no Aux travaux de conservation ou d'entretien de l'immeuble, autres que ceux de maintenance.
2\no Aux travaux portant sur les éléments d'équipement communs, autres que ceux de maintenance
3\no Aux travaux d'amélioration, tels que la transformation d'un ou de plusieurs éléments d'équipement existants, l'adjonction d'éléments nouveaux, l'aménagement de locaux affectés à l'usage commun ou la création de tels locaux, l’affouillement du sol et la surélévation de bâtiments ;
4\no Aux études techniques, tels que les diagnostics et consultations ;
5\no Et, d'une manière générale, aux travaux qui ne concourent pas à la maintenance et à l'administration des parties communes ou à la maintenance et au fonctionnement des équipements communs de l'immeuble
Bref, cet article ne permet pas de faire la distinction entre les travaux de maintenance courante et les autres travaux.
497 Cette question est traitée plus loin au titre des procédures de recouvrement.
droit de la copropriété année 2019-2020
393
2. L’exigibilité des appels de fonds pour travaux.
Bien évidemment, ces travaux sont « hors budget ». Par le passé nombreux étaient les syndics qui faisaient voter des travaux simplement en mentionnant leur coût provisionnel dans le budget. Cette Pratique était totalement contraire à la loi. L’’article 14-2 met les points sur les « i » !
L'assemblée générale votera spécialement chacun des travaux concernés et précisera en même temps les conditions de recouvrement des charges afférentes à ces travaux, c'est-à-dire qu’elle votera la date à laquelle les provisions sur travaux seront exigibles
L’article 35-2 du décret après avoir précisé dans le premier § que le syndic demandait le paiement de la quote-part du budget par lettre simple, ajoute dans un second paragraphe concernant les appels de fonds pour travaux « hors budget » :
« Pour les dépenses non comprises dans le budget prévisionnel, le syndic adresse à chaque copropriétaire, par lettre simple, préalablement à la date d'exigibilité déterminée par la décision d'assemblée générale, un avis indiquant le montant de la somme exigible et l'objet de la dépense ».
Comme par le passé, on admettra que « l’objet » de la dépense soit précisé d’un mot ou d’une simple expression « acompte sur ravalement » par exemple.
3. Le recouvrement des appels travaux se fait conformément au droit commun.
De plus, si le copropriétaire est bien débiteur des appels conformément à ce qui était décidé par l'assemblée générale pour le paiement des travaux, il ne peut pas cependant être condamné comme il est dit à l'article 19-1 de la loi.
Le syndic n'aura d'autre possibilité que de suivre la voie normale de la procédure à l'encontre du copropriétaire pour le paiement de sa quote-part des travaux votés en assemblée générale : il pourra demander la condamnation provisionnelle au juge des référés ou une condamnation au fond devant le tribunal.
SECTION III - LES COMPTES DU SYNDICAT.
Rappelons que ces règles sont fixées dans ler principe par l’article 14-3 de la loi du 10 juillet 1965 :
droit de la copropriété année 2019-2020
394
Art. 14-3 de la Loi du 10 juillet 1965 (Loi SRU du 13 décembre 2000, entre en vigueur le 1er janvier 2006)-
Les comptes du syndicat comprenant le budget prévisionnel, les charges et produits de l’exercice, la situation de trésorerie, ainsi que les annexes au budget prévisionnel sont établis conformément à des règles comptables applicables spécifiques fixées par décret. Les comptes sont présentés avec comparatif des comptes de l’exercice précédent approuvé ».
A. LES REGLES RELATIVES A L’ENREGISTREMENT DES ECRITURES
1. La comptabilité en partie double.
Article 1er de l’arrêté du 14 mars :
« les écritures sont passées selon le système dit « en partie double ». Dans ce système, tout mouvement ou variation enregistré dans la comptabilité est représenté par une écriture qui établit une équivalence entre ce qui est porté au débit et ce qui est porté au crédit des différents comptes affectés par cette écriture ».
2. La sincérité des écritures.
Aux termes de l’article 5 de l’arrêté du 14 mars 2005 :
« Les documents comptables sont tenus sans altération et sans blanc. Une écriture erronée est annulée par une écriture contraire.
« Une procédure de clôture destinée à figer la chronologie et à garantir l’intangibilité des enregistrements est mise en oeuvre à la date d’arrêté des comptes ».
En d’autres termes, une fois appuyé sur le bouton de validation de clôture de l’exercice il ne sera plus possible de modifier aucune écriture.
Les comptes 1 à 5 sont totalisés et font apparaître un solde qui sera le « report à nouveau » du début du nouvel exercice. Par contre les comptes 6 (charges par nature) et 7 (produits financiers) ne font pas l’objet d’un report à nouveau.
3. La comptabilité d’engagement.
droit de la copropriété année 2019-2020
395
L’article 14-3 de la loi précise que les charges et les produits sont enregistrés dès leur engagement juridique ; mais les articles 3 et 4 du décret définissent de façon plus libérale les notions d’engagement et de produit :
Article 3 du décret (dépenses sur budget) : Il faut comptabiliser les dépenses lorsque le syndicat a bénéficié de la fourniture ou du service
Article 4 du décret (dépenses hors budget) : il faut comptabiliser les dépenses au fur et à mesure de la réalisation des travaux ou de la fourniture des prestations
Article 3 du décret (produits): il faut enregistrer les sommes dues par les copropriétaires à la date d’exigibilité de ces sommes, que ce soit sur le budget ou sur les dépenses hors budget.
4. La comptabilité d’exercice.
L’article 5 al 1 du décret précise que la durée d’un exercice est normalement de 12 mois :
« L’exercice comptable du syndicat des copropriétaires couvre une période de douze mois. Les comptes sont arrêtés à la date de clôture de l’exercice. Pour le premier exercice, l’assemblée générale des copropriétaires fixe la date de clôture des comptes et la durée de cet exercice qui ne pourra excéder dix-huit mois ».
L’article 5 al 2 du décret autorise l'assemblée générale à modifier les dates de l’exercice annuel.
« La date de clôture de l’exercice pourra être modifiée sur décision motivée de l’assemblée générale des copropriétaires. Un délai minimum de cinq ans devra être respecté entre les deux décisions d’assemblées générales modifiant la date de clôture ».
Ce qui signifie que l'assemblée générale à la majorité de l’article 24 de la loi498 pourra par une décision motivée modifier non pas la durée de l’exercice (ce sera toujours un an) mais le point de départ de l’exercice : par exemple la résolution pourra être la suivante :
« L'assemblée générale pour faire coïncider l’exercice comptable avec le début de la campagne de chauffe décide que cet exercice comptable débutera le 1er octobre prochain pour s’achever au 30 septembre de l’année suivante ».
498 La loi étant muette sur ce point on peut se demander si le Règlement de copropriété ne peut pas prévoir une majorité différente. En tout cas dans le silence du Règlement de copropriété ce sera bien la majorité de l’article 24 qui s’appliquera.
droit de la copropriété année 2019-2020
396
De la même façon dans une résidence de vacances soumise au statut de la copropriété, on pourra motiver un exercice allant du 1er avril au 31 mars par la nécessité de tenir pendant la période estivale l'assemblée générale annuelle statuant sur les comptes et le budget.
Un nouveau changement des dates d’exercice ne pourra pas intervenir avant cinq années entières.
Bien évidemment chaque exercice est indépendant du précédent et du suivant, en sorte que si des dépenses sont échelonnées sur plusieurs exercices, elles devront apparaître dans un compte « solde en attente » reproduit à l’annexe V.
5. L’enregistrement de fait au fur et à mesure des engagements.
Cet enregistrement se fait dès l’engagement juridique par le syndic, indépendamment de leur règlement (article 2 al 2 du décret du 14 mars 2005). Sur ce fondement la Cour de Cassation a sanctionné un arrêt de Cour d'Appel ayant refusé l’annulation de la résolution d’approbation des comptes ne tenant pas compte d’une indemnité allouée au syndicat par le Tribunal et dont le règlement n’était pas intervenu au cours de cet exercice.499
6. L’enregistrement des écritures se fait TTC
L'article 3 de l'arrêté du 14 mars 2005 précise :
« les opérations sont enregistrées toutes taxes comprises dans les comptes dont l'intitulé correspond à leur nature. Le montant et le taux de taxe sont indiqués lorsque un ou plusieurs copropriétaires ont déclaré être assujettis à la TVA »
B. LE PLAN COMPTABLE DU SYNDICAT.
l'article 7 de l'arrêté du 14 mars 2005 donne la nomenclature des comptes.
1. Les cinq classes de comptes et leurs détails.
499 Civ. 3\no Ch. 5 février 2014, Pourvoi \no 12-19047, Au Bulletin.
droit de la copropriété année 2019-2020
397
Il y a cinq classes de comptes (les classes 1, et 3 à 7) :
CLASSE 1. Provisions, avances, subventions et emprunts
10 - Provisions et avances
102 - provisions pour travaux décidés 103 - avances 1031 - avances de trésorerie
1032 - avances travaux article 18 alinéa six
1033 - autres avances
12 - soldes en attente sur travaux et opérations exceptionnelles
13 - subventions
131 - subventions accordées en instance de versement
CLASSE 4. Compte des copropriétaires et des tiers
40 - fournisseurs 401 - factures parvenues
408 - factures non parvenues 409 - fournisseurs débiteurs
42 - personnel
421 - rémunérations dues
43 - sécurité sociale et autres organismes sociaux
431 - sécurité sociale
432 - autres organismes sociaux
44 - État et collectivités territoriales
441 - État et autres organismes -- subventions à recevoir 442 - État -impôts et versements assimilés
443 - collectivités territoriales -- aides
45 - collectivités des copropriétaires
450 - copropriétés individualisées 450-1 - copropriétaires -- budget prévisionnel
450-2 - copropriétaires -- travaux article 14 - 2
450-3 - copropriétaires -- avances
450-4 - copropriétaires -- emprunts
459 - copropriétaires -- créances douteuses
46 - débiteurs et créditeurs divers :
461 - débiteurs divers
droit de la copropriété année 2019-2020
398
462 - créditeurs divers
47 - compte d'attente :
471 - compte en attente d'imputation débiteur
472 - compte en attente d'imputation créditeur
48- comptes de régularisation :
486 - charges payées d'avance
487 - produits encaissés d'avance
49 - dépréciation des comptes de tiers :
491 - copropriétaires
492 - personnes autres que les copropriétaires
CLASSE 5 comptes financiers
50 - fonds placés
501 comptes à terme
502 - autre compte
51 -Banque, ou fonds disponibles en banque pour le syndicat
512 banques
514 - chèques postaux
53 - caisse
CLASSE 6 comptes de charges par nature
60 - achats de matières et fournitures :
601 - eau
602 - électricité
603 - chauffages, énergie et combustible
604 - achats produits d'entretien et petits équipements
605 - matériels
606 - fournitures
61 - services extérieurs :
611 - nettoyages des locaux
612 - locations immobilières
613 - locations mobilières
614 - contrats de maintenance
droit de la copropriété année 2019-2020
399
615 - entretien et petites réparations
616 - primes d'assurance
62 - frais d'administration et honoraires :
621 - rémunérations du syndic sur gestion copropriété
6211 - rémunérations du syndic
6212 - débours
6213 - frais postaux
622 - autres honoraires du syndic
6221 - honoraires travaux
6222 - prestations particulières
6223 - autres honoraires
623 - rémunérations de tiers intervenants
624 - frais du conseil syndical
63 - impôts -- taxes et versements assimilés :
632 - taxes de balayage
633 - taxes foncières
634 - autres impôts et taxes
64 - frais de personnel :
641 - salaires
642 - charges sociales et organismes sociaux
643 - taxes sur salaire
644 - autres (médecine du travail, mutuelle, etc.)
66 - charges financières des emprunts, agios ou autres :
661 - Remboursements d'annuité d'emprunts
662 - Autres charges financières et agios
67 - Charges pour travaux et opérations exceptionnelles :
671 - Travaux décidés par l'assemblée générale
672 - Travaux urgents
673 - Etudes techniques, diagnostic, consultation
677 - Pertes sur créances irrécouvrables
678 - Charges exceptionnelles
68 - Dotations aux dépréciations sur créances douteuses.
droit de la copropriété année 2019-2020
400
CLASSE 7 - Comptes de produits par nature
70 - Appels de fonds :
701 - Provisions sur opérations courantes
702 - Provisions sur travaux art. 14-2 703 - Avances
704 - remboursements d'annuités d'emprunt
71 - Autres produits:
711 - subventions
712 - emprunts
713 - indemnités d'assurance
714 - produits divers (dont intérêts légaux copro) 716 - produits financiers
718 - produits exceptionnels
78 - reprise de dépréciations sur créances douteuses
2. Il n'y a ni immobilisations ni stocks en copropriété.
On observera qu'il n'existe ni compte de la classe 2 (compte d'immobilisation) ni compte de la classe 3 (compte de stocks)., alors que ces comptes existent dans le plan comptable général. Il résulte de cette absence de comptes de ces deux classes que le syndicat des copropriétaires ne peut pratiquer aucune immobilisation et les stocks acquis en fin d'exercice sont considérés comme consommés au cours du même exercice (le fioul par exemple).
3. Des divisions de comptes sont possibles.
Le syndicat ne peut ouvrir aucun compte autre que ceux inclus dans la nomenclature. Par contre, rien n'interdit au syndicat de créer des subdivisions (électricité générale, électricité de l'ascenseur par exemple).
droit de la copropriété année 2019-2020
401
C. DOCUMENTS ET LES PIECES COMPTABLES.
1. Les pièces comptables
• Le livre journal
Le livre journal « enregistre chronologiquement les opérations ayant une incidence financière sur le fonctionnement du syndicat »,
• Le grand livre
Le grand livre des comptes « regroupe l'ensemble des comptes utilisés par le syndicat, opération par opération ».
Le grand livre de comptes regroupe les écritures chronologiques par catégories de comptes, en sorte que les totaux du grand livre pour un exercice déterminé correspondent aux totaux du livre journal pour le même exercice.
• La balance des comptes.
La balance des comptes synthétise le total des mouvements de chaque compte et le solde de chacun des comptes : pour une période donnée, les totaux de la balance générale en mouvement au débit et au crédit sont égaux entre eux et égaux au total des journaux relatifs à la même période.
L'article 2 de l'arrêté du 14 mars 2005 précise que le syndic édite deux balances générales des comptes :
- une balance éditée selon la nomenclature comptable,
- une balance éditée selon les clés de répartition des charges prévues par le règlement de copropriété.
2. Les archives comptables et leur transmission
Selon l’article 6 du décret du 14 mars 2005 :
« les pièces justificatives, document de base de toute écriture comptable, doivent être des originaux et comporter les références du syndicat (nom et adresse de l'immeuble). Elles doivent être datées et conservées par le syndic pendant 10 ans, sauf disposition contraire expresse »
Aux termes de l'arrêté du 14 mars 2005 :
droit de la copropriété année 2019-2020
402
« tout enregistrement comptable comporte un libellé permettant une identification de la pièce justificative qu'il appuie, notamment date et numéro de facture, actes et références supplément, période de l'appel de fonds et son objet.
La date à laquelle le paiement est intervenu peut être mentionné sur les factures, mémoires et situations »
L'article 33 du décret du 17 mars 1967 est consacré aux archives du syndic qui comprennent « les documents comptables du syndicat ».
Selon l'article 18 - 2 de la loi du 10 juillet 1965, en cas de changement de syndic, l’ancien syndic doit transmettre à son successeur dans le délai
- d'un mois de sa cessation de fonctions « la situation de trésorerie, la totalité des fonds immédiatement disponibles et l'ensemble des documents et archives du syndicat »
- et dans le délai complémentaire de deux mois (soit au total trois mois au plus après cessation des fonctions) l'ancien syndic doit remettre au nouveau syndic le solde des fonds disponibles après apurement des comptes et lui fournir l’état des comptes des copropriétaires ainsi que celui des comptes du syndicat ».
L'article 6 alinéa 2 du décret du 14 mars 2005 ajoute simplement :
"En cas de changement de syndic les documents comptables et les originaux des pièces justificatives sont transmis au successeur, le syndic sortant prenant ses propres dispositions afin de conserver les copies des pièces justificatives qu'il estime nécessaires pour la justification des opérations comptables qui lui incombaient ».
Le cumul de ces dispositions permet de conclure que le syndic doit transmettre les originaux à son successeur et qu'il lui incombe de garder les copies nécessaires à l'apurement des comptes.
3. La communication des comptes et de leur comparatif avec la convocation
Selon l’article 8 alinéa 5 du Décret du 14 mars 2005, développant l’article 14-3 de la loi :
« Les comptes de l’exercice clos sont à présenter pour leur approbation par les copropriétaires avec le budget voté correspondant à cet exercice et le comparatif des comptes approuvés de l’exercice précédent ».
Si les règles permettent un contrôle facilité des comptes de la copropriété, elles sont difficilement compréhensibles par le commun des copropriétaires pourtant appelés à approuver les comptes et le budget.
droit de la copropriété année 2019-2020
403
En fait le copropriétaire, s’il n’est membre du Conseil syndical, n’aura pas à examiner le détail des comptes. Pour autant il sera pleinement informé par l’annexion à la convocation de l'assemblée générale ayant à statuer sur les comptes des cinq annexes prévues par le décret du 14 mars 2005 :
		\chapter{Les appels de fonds aupres des coproprietaires}
		\chapter{Les mesures conservatoires en vue du recouvrement des charges}
		\chapter{Le recouvrement des provisions et des charges}
	
	\part{Deuxième partie}
	
	\tableofcontents
	
\end{document}