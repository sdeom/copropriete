\chapter{La modification des charges}

SECTION I - LA MODIFICATION CONVENTIONNELLE DE LA REPARTITION DES CHARGES.
Par application de l'article 11 de la loi du 10 juillet 1965 les charges de copropriété ne peuvent en principe être modifiées par l'Assemblée Générale autrement qu'à l'unanimité (A). Toutefois il existe trois exceptions au principe de l'unanimité (B) : lorsque des travaux ou des actes de disposition ou d'acquisition sont décidés par l'Assemblée (1), en cas d'aliénation séparée d'une ou plusieurs fractions d'un lot (2), en cas de changement d'usage d'une ou plusieurs parties privatives (3).
A. LE PRINCIPE D'UNANIMITE POSE PAR L'ARTICLE 11 DE LA LOI.
Aux termes de l'article 11 :
"Sous réserve des dispositions de l'article 12 ci-dessous, la répartition des charges ne peut être modifiée qu'à l'unanimité des copropriétaires (...)".
L'article 11 pose donc le principe de l'unanimité, et il s'agit d'une disposition d'ordre public.
En conséquence serait nécessairement annulée une modification de la répartition des charges votée autrement qu'à l'unanimité, c'est à dire par l'ensemble des propriétaires composant la Copropriété, ceci quelles que soient les dispositions du Règlement de Copropriété quant à la modification des charges.
Cette règle de l'unanimité est un obstacle pratiquement insurmontable à toute modification de la répartition des charges telle que fixée par le règlement de copropriété, même si la répartition du règlement de copropriété est contraire aux dispositions de l'article 10. En ce cas, aucune modification ne pourra être votée en assemblée générale, alors même qu'une majorité de copropriétaires manifesterait son sens de l'équité. De la sorte le seul recours pour établir une répartition conforme à la loi demeure - dans cette hypothèse - l'action en justice.
B. LES EXCEPTIONS AU PRINCIPE.
droit de la copropriété année 2019-2020
347
1. Article 11, premier alinéa de la loi du 10 juillet 1965 : modification consécutive à des travaux ou à des actes d’acquisition ou de disposition
Après avoir posé le principe de l'unanimité, cet article ajoute :
"Toutefois, lorsque des travaux ou des actes d'acquisition ou de disposition sont décidés par l'assemblée générale statuant à la majorité exigée par la loi, la modification de la répartition des charges ainsi rendue nécessaire peut être décidée par l'assemblée générale statuant à la même majorité".
Cette modification est soumise à une condition de fond : la modification des charges doit être nécessaire.
Par ailleurs, la modification des charges doit être votée à la même majorité que la décision de travaux ou la décision d’acquisition ou de disposition.
- Travaux décidés par l'Assemblée
LES TRAVAUX DECIDES PAR L’ASSEMBLEE GENERALE AYANT UNE REPERCUSSION SUR LA REPARTITION DES CHARGES
Nous verrons ultérieurement les travaux décidés en assemblée générale, donnons ici quelques exemples :
- Travaux rendus obligatoires par des dispositions légales ou réglementaires (majorité article 25).
- Travaux d'économie d'énergie (majorité article 25).
- Travaux de type particulier tels ceux destinés à faciliter l'accès à l'immeuble par les handicapés (majorité article 24).
- Travaux améliorant la sécurité de l'immeuble (majorité article 25).
- Travaux d'embellissement ou d'amélioration (majorité article 26).
- Travaux de surélévation (majorité article 26).
Dans toutes ces hypothèses, c'est le Syndicat des Copropriétaires qui décide d'effectuer des travaux aux frais de tous les copropriétaires ou de certains d'entre eux et ces travaux peuvent avoir pour conséquence d'imposer une modification dans la répartition des charges.
Par exemple lorsque le Règlement de Copropriété prévoit une répartition des charges d'eau froide aux tantièmes généraux, si le syndicat des copropriétaires décide d'installer des compteurs individuels, l'assemblée décidera en conséquence que les charges d'eau froide seront ensuite reparties en fonction
droit de la copropriété année 2019-2020
348
des consommations réelles de chaque lot : c'était le but même de la décision d'installation des compteurs individuels.
C'est donc de façon parfaitement logique que le législateur a prévu dans cette hypothèse que l'Assemblée votera la modification des charges à la majorité nécessaire pour l'adoption de ces travaux (en quelque sorte c'est l'application du principe selon lequel l'accessoire suit le principal.)
Relevons qu’une application stricte du texte impose que la modification n’est possible que dans la mesure où l’assemblée générale a concomitamment ou précédemment décidé la réalisation des travaux. En sorte qu’aucune modification des charges ne peut intervenir faute d’Assemblée ayant décidé les travaux.452
LES TRAVAUX SIMPLEMENT AUTORISES PAR L’ASSEMBLEE GENERALE
Il est fréquent que l'Assemblée, sans décider de faire elle-même des travaux, autorise un copropriétaire à effectuer des travaux affectant les parties communes de l'immeuble; ces travaux une fois réalisés rendront nécessaire une modification dans la répartition des charges.
Par exemple :
Un copropriétaire demande l'autorisation de réaliser des travaux affectant les parties communes qui lui permettront de transformer un grenier en local d'habitation. En ce cas il peut paraître logique d'augmenter les charges générales.
Ou encore, un copropriétaire demandera l'autorisation de raccorder son lot au chauffage collectif; un autre copropriétaire possédant plusieurs chambres de service au sixième étage jusque lors desservi exclusivement par l'escalier de service demandera l'autorisation de percer une trémie dans le plancher haut du cinquième étage de l'escalier principal et de créer une volée supplémentaire permettant à ses lots d'avoir l'accès à l'escalier principal.
Une fois encore il sera nécessaire de modifier la répartition des charges pour tenir compte de la réalisation de ces travaux.
Bien que la plupart des auteurs soient muets sur cette hypothèse à l'exception toutefois de MM LAFOND et ROUX453, du Conseiller GUILLOT454 et du Professeur ATTIAS455 qui font application du texte aux "travaux autorisés", il convient de considérer que l'article 11 de la loi doit recevoir application non seulement
452 Civ 3\degres 29 oct 1984, Administrer 1985 p. 37 note Guillot.
453 in Code de la Copropriété, Edition de 2018 p. 176 \no 6.
454 op cité p. 69 \no 11,9
455 Guide de la Copropriété des Immeubles Bâtis, 6\degres Ed., page 278 \no 618
droit de la copropriété année 2019-2020
349
lorsque la Copropriété réalise elle-même des travaux, mais encore lorsqu'elle autorise un copropriétaire à réaliser des travaux nécessitant son accord préalable (c'est l'hypothèse de l'article 25 b de la loi).
La Cour de Cassation sans se prononcer expressément sur ce point a cependant admis que des travaux modificatifs autorisés par l’Assemblée Générale impliquaient une modification de la répartition des charges par application de l’article 11 de la loi456 :
En l’espèce un copropriétaire avait été autorisé à réaliser des travaux de transformation de ses trois lots commerciaux en 28 lots à usage d’habitation. Ces travaux avaient eu pour conséquence de permettre aux copropriétaires des niveaux inférieurs d’accéder à un ascenseur et au couloir desservant cet ascenseur : « la cour d’appel en a exactement déduit qu’il y avait lieu en application de l’article 11 de la loi à une nouvelle répartition des charges relatives à l’ascenseur et au couloir commun entre les lots desservis par eux ».
Etant observé que le copropriétaire pourra avoir intérêt à une modification des charges à la suite de travaux autorisés par l'Assemblée : par exemple lorsqu'il obtient l'autorisation de se débrancher du chauffage collectif, il demandera à être dispensé de participer aux charges collectives de chauffage. Etant précisé que sauf cas très particulier, la Cour de Cassation estime que la Copropriété est libre de dispenser ou non le copropriétaire dans une telle hypothèse,457
Actes d'acquisition ou de disposition.
Les actes d'acquisition ou les actes de disposition de parties communes ne doivent avoir qu'exceptionnellement une influence sur la répartition des charges :
Par exemple si la Copropriété achète un terrain pour y faire un parking ou un lot privatif pour y installer une concierge, cela entraînera la création de nouvelles charges plutôt que la modification de la répartition existante.
De même si la Copropriété aliène une partie commune, volontairement ou par suite d’expropriation, cela ne devrait entraîner d’autre conséquence sur les charges que leur diminution ou la suppression des charges spécifiques à la partie d’immeuble ou à l’élément d’équipement qui disparaît de la sorte.
Par contre, l'aliénation de droits accessoires aux parties communes (droit de surélever, droit de construire, droit d'affouiller), permettant la réalisation de nouveaux lots privatifs, implique une modification dans la répartition des charges en sorte que le bénéficiaire de cette "aliénation" supporte des charges complémentaires. Modification qui sera votée à la majorité de l'article 26 ou de l’article 25 de la loi du 10 juillet 1965 selon que l’immeuble à surélever est situé ou non en zone « tendue ».
456 Civ 3\degres, 3 fév 1993 (Raudet).
457 Civ. 3\degres 13 mai 1980, Bull \no 97 p. 71.
droit de la copropriété année 2019-2020
350
Cette modification se conçoit également dans l’hypothèse où la copropriété cède un droit de jouissance d’une partie commune à un copropriétaire : Paris, 19\degres Chambre, 10 décembre 1992, Numéro JurisData : 1992-023902
« La répartition des charges qui ne peut en principe être modifiée qu'à l'unanimité des copropriétaires peut l'être par l'assemblée générale statuant à la majorité exigée par la loi ou, à défaut, par le tribunal de grande instance saisi par tout copropriétaire "lorsque des travaux ou des actes d'acquisition ou de disposition sont décidés par l'assemblée générale". La cession par la copropriété d'un droit de jouissance exclusive à titre définitif entre dans la catégorie des actes de disposition susvisés. »
2. Article 11, deuxième alinéa de la loi du 10 juillet 1965 : la division de lots458
CET ALINEA EST AINSI REDIGE :
"En cas d'aliénation séparée d'une ou plusieurs fractions d'un lot, la répartition des charges entre ces fractions est, lorsqu'elle n'est pas fixée par le règlement de copropriété, soumise à l'approbation de l'assemblée générale statuant à la majorité de l'article 24."
Cette disposition envisage l'hypothèse du copropriétaire qui divise un lot d'origine en deux ou plusieurs nouveaux lots. Il est rare en ce cas que le règlement de copropriété ait prévu comment se repartiraient alors les charges entre les nouveaux lots issus de cette subdivision. Aussi le copropriétaire demande-t-il alors à son notaire d'établir un projet modifiant l'état descriptif de division et la répartition des charges.
Cette répartition sera faite en sorte que le total des charges réparties entre les nouveaux lots issus de la division corresponde au total des charges du lot d'origine, sans augmentation ni diminution.
Enfin, cette répartition entre les nouveaux lots devra concerner toutes les catégories de charges : les charges générales comme les charges spéciales s'il en existe, telles par exemple les charges d'ascenseur ou de chauffage.
SEULES SONT CONCERNEES LES CHARGES DU LOT DIVISE.
L’assemblée générale, ne pourrait au prétexte de la division du lot imposer une modification dans la répartition des charges applicable aux autres lots de la copropriété459 :
458 Antérieurement à la loi ALUR il s’agissait de l’article 25 f) de la loi.
459 Civ 3\degres ch. 22 mars 1995. inédit.
droit de la copropriété année 2019-2020
351
En l’espèce le lot \no 42 de l’E.D.D. avait été divisé en deux lots 45 et 46. L’assemblée, au lieu d’approuver la nouvelle répartition des charges entre les deux lots, en a profité pour modifier la répartition des charges de l’ensemble de la copropriété; « en statuant ainsi alors que la nouvelle répartition adoptée ne concernait pas seulement les deux nouveaux lots issus de la division du lot \no 42, mais modifiait aussi la répartition des charges de l’ensemble de la copropriété, la cour d’appel a violé l’article 11. »
i le Règlement de copropriété ne prévoit pas par avance la répartition des charges entre les lots issus de la division, le syndic n’aura pas à appliquer l’éclatement des charges entre les lots issus de la division tant que l'assemblée générale ne se sera pas prononcée sur cette répartition des charges :
La 19\degres Chambre de la cour de Paris, dans un arrêt du 27 nov 1999, a très logiquement jugé :
« Il ne peut être fait grief au syndic de n'avoir pas modifié la répartition des charges spéciales telle qu'elle figure au règlement de copropriété alors que l'auteur de la division du lot, n'a pas cru bon de procéder à cette répartition ni de modifier en conséquence le règlement de copropriété. Jusqu'à ce qu'une nouvelle répartition de charges communes spéciales soit adoptée, soit par l'assemblée générale, soit à défaut, par la juridiction saisie, le syndic est donc fondé à appliquer le règlement de copropriété en affectant les tantièmes de copropriété afférents au lot divisé à l'ayant droit de l'ancien propriétaire de ce lot. »
Il convient d’observer qu’en cas de division de son lot par un copropriétaire, celui-ci doit seulement demander à l’assemblée générale d’approuver la répartition des charges entre les lots issus de la division ; il n’a pas à solliciter l’autorisation de l’assemblée générale des copropriétaires pour réaliser cette division, à moins qu’une telle autorisation ne soit expressément requise par le règlement de copropriété.
Toutefois il arrive fréquemment que le copropriétaire se méprenne sur l’étendue de ses obligations et sollicite une telle autorisation. La Cour de cassation vient de décider qu’en ce cas si l’assemblée générale refuse l’autorisation division, le copropriétaire doit respecter la décision d’assemblée générale460
3. L'article 25 e) de la loi du 10 juillet 1965 : le changement d’usage des parties privatives
- Le principe posé : changement d’usage des parties privatives
Cet article prévoit qu'est adoptée à la majorité des voix de tous les copropriétaires :
"la modification de la répartition des charges visées à l'alinéa 1er de l'article 10 ci-dessus rendue nécessaire par un changement d'usage d'une ou plusieurs parties privatives".
460 Civ. 3\degres Ch. 8 juin 2017, \no 16-16.566, JCP N 2017, \no 25, act. 655
droit de la copropriété année 2019-2020
352
Bien évidemment, il ne peut s'agir que d'un changement d'usage conforme au règlement de copropriété (i.e. qui n'est pas contraire à la destination de l'immeuble et ne porte pas atteinte aux droits des autres copropriétaires). Cependant un changement d’usage accordé à titre de tolérance justifie l’application de l’article 25 f461
Hypothèse fréquemment rencontrée : un local jusque lors affecté à usage d'habitation se trouve affecté à usage professionnel. Ce changement d'usage va entraîner une augmentation de certaines charges liées aux services collectifs ou équipements communs telle l'utilisation de l'ascenseur.
Dans ces différentes hypothèses l'assemblée générale déterminera à la majorité de l'article 25 de la loi les conditions de modification des charges462.S'agissant de l’hypothèse évoquée, le plus souvent les charges d'ascenseurs seront doublées.
- Conditions d'application de l'article 25 e) de la loi.
IL FAUT QU'IL Y AIT CHANGEMENT EFFECTIF D'AFFECTATION DU LOT :
L'article 25 e) de la loi ne peut recevoir application qu'en cas de modification dans l'affectation du lot : si le lot était depuis l'origine affecté à usage professionnel, on ne saurait augmenter les charges d'ascenseur au motif que la fréquentation du lot par la clientèle est plus importante que prévue
Cour d'Appel de Lyon 5 décembre 1979463 :
Une Compagnie d'assurances avait succédé à une société qui utilisait les locaux à usage de bureau sans réception de clientèle, d'où aggravation très sensible des charges d'ascenseur, mais il n'y avait pas de modification d'affectation du lot par rapport à ce que stipulait le Règlement de Copropriété).
cour de Paris en date du 6 mai 1998 464 :
« La modification de la répartition des charges ne peut donc intervenir à l'égard des copropriétaires sur la base de l'article 25 f de la loi du 10 juillet 1965 dès lors que leur lot n'a pas changé d'usage ».
461 Civ 3\degres Ch 20 juin 2001.
462 Pour une application de l’art. 25 f, cf. Civ 3\degres Ch. 3 juillet 1996 Loyers et Copropriété 1996 \no 404.
463 Cour d'Appel de Lyon 5 décembre 1979; DS 1980 Inf. Rap. 448, note Giverdon
464 Numéro JurisData : 1998-021120
droit de la copropriété année 2019-2020
353
Par contre la majoration des charges est possible dès qu’il y a changement effectif de l’usage du lot, quand bien même le Règlement de copropriété prévoit par exemple que le lot peut être affecté à usage d’habitation ou à usage professionnel.
Civ 3\degres Ch, 1\degres octobre 2014, Pourvoi \no 13-21745
« l'article 25 f) de la loi du 10 juillet 1965 est applicable alors même que le nouvel usage du lot est prévu par le règlement de copropriété »,
SEULES SONT CONCERNEES LES CHARGES DE L'ARTICLE 10 ALINEA 1.
L'Assemblée Générale ne saurait, au motif de changement d'affectation d'un lot, modifier la répartition des charges générales qui relèvent de l'article 10 alinéa 2 : elle ne peut qu'imposer une modification de la répartition des charges afférentes aux services collectifs et éléments d'équipement commun.
On peut regretter sur ce point la timidité du législateur, surtout lorsqu'on sait que la Cour de Cassation a tendance à classer en charges générales ce qui était considérée avant comme charges afférentes à des services collectifs (charges de concierge, essentiellement).
Certes, lorsque le changement d'affectation implique des travaux sur parties communes ou portant sur l'aspect extérieur de l'immeuble, le Syndicat pourra subordonner son autorisation à l'application de l'article 11 alinéa 1er (travaux autorisés par l'assemblée) et en conséquence modifier la répartition des charges dans les deux catégories.
Mais lorsque le changement d'usage ne s'accompagne d'aucuns travaux, si un tel changement est conforme à la destination de l'immeuble (par exemple, transformation d'un grenier en local d'habitation dans un immeuble essentiellement destiné à l'habitation), le syndicat ne pourra pas augmenter les charges générales afférentes au lot ainsi modifié.
4. Les recours judiciaires
Les recours judiciaires sont différents selon que la modification de répartition de charges est prévue par l’article 11 de la loi ou par l’article 25 e) de la même loi.
- Recours de l'article 11 dernier alinéa en l‘absence de décision.
droit de la copropriété année 2019-2020
354
"A défaut de décision de l'assemblée générale modifiant les bases de répartition des charges dans les cas prévus aux alinéas précédents, tout copropriétaire pourra saisir le tribunal judiciaire du lieu de situation de l'immeuble à l'effet de faire procéder à la nouvelle répartition rendue nécessaire".
Pour que l'action judiciaire soit recevable, il faut :
que l’Assemblée générale ait été préalablement invitée à se prononcer sur une modification de répartition des charges dans l’un des deux cas de l’article 11.
qu'il n'y ait pas eu de décision de l'assemblée générale modifiant les bases de répartition, soit parce que la majorité légale ne s'est pas dégagée en faveur de la modification des charges, soit parce que l’Assemblée a refusé cette modification.465
Le juge fixera alors la nouvelle répartition des charges rendue nécessaire par les travaux décidés (ou autorisés) ou par l’aliénation ou l’acquisition d’une partie commune.
- Le recours de l'article 42, al.4 en cas de modification illicite
Lorsque l'assemblée générale décide une modification des bases de répartition dans l'une des trois hypothèses (art. 11 al 1 et 2 et art. 25 e) ou lorsqu’elle refuse une modification sur le fondement de l’article 25 e), l'action de l'article 11 dernier alinéa n'est pas possible.
Dans ces cas là, il convient d'exercer le recours de l'article 42 quatrième alinéa, aux termes duquel :
« S’'il est fait droit à une action contestant une décision d'assemblée générale portant modification de la répartition des charges, le tribunal judiciaire procède à la nouvelle répartition ».
Il s’agit alors d’une contestation d’Assemblée générale de l’article 42 de la loi soumise comme telle au délai de prescription de deux mois.
465 Selon MM STEMMER et LAFOND, en l’absence de décision, le recours pourra être exercé pendant dix ans (ramené à 5 ans par la loi ELAN) ; en cas de refus, ce sera une contestation de l’article 42 alinéa 2 enfermée dans le délai de deux mois. Toutefois la Cour de Paris, 23\degres Ch. Le 15 décembre 1995 a estimé qu’en l’absence de décision alors que l’assemblée générale avait été saisie de la question, la demande devait être formulée à l’intérieur du délai de deux mois.
.
droit de la copropriété année 2019-2020
355
Le juge pourra annuler la décision de modification (ou de refus de modification), et fixer une nouvelle répartition.
SECTION II LA MODIFICATION JUDICIAIRE DES CHARGES
A. LES DEUX ACTIONS POSSIBLES POUR OBTENIR LA MODIFICATION DES CHARGES
1. Historique
- Nécessité d’un contrôle du juge sur la répartition des charges
Aux termes de l'article 10 alinéa 3 de la loi, le règlement de copropriété fixe la quote-part afférente à chaque lot dans chacune des catégories de charges; c'est l'état de répartition des charges prévu à l'article 1er du décret :
DECRET du 17 Mars 1967 art 1 alinéa 1er :
Le règlement de copropriété mentionné à l'article 8 de la loi du 10 juillet 1965 susvisé comporte (...) l'état de répartition des charges prévu au troisième alinéa de l'article 10 de ladite loi.
Cet état définit les différentes catégories de charges et distingue celles afférentes à la conservation, à l'entretien et à l'administration de l'immeuble, celles relatives au fonctionnement et à l'entretien de chacun des éléments d'équipements communs et celles entraînées par chaque service collectif.
L'état de répartition des charges fixe, conformément aux dispositions du troisième alinéa et, s'il y a lieu du dernier alinéa de l'article 10 de la loi du 10 juillet 1965, la quote-part qui incombe à chaque lot dans chacune des catégories de charges (…)
Cette répartition doit bien évidemment être conforme aux principes énoncés par l'article 10 de la loi :
alinéa 1er pour les charges relatives aux services collectifs et éléments d'équipements communs, réparties en fonction de l'utilité.
alinéa 2 pour les charges relatives à la conservation, l'entretien et à l'administration des parties communes, réparties proportionnellement aux valeurs relatives des parties privatives comprises dans les lots.
droit de la copropriété année 2019-2020
356
Fréquemment, il existe des erreurs ou même des injustices dans les réparations de charges :
o Erreur matérielle dans la rédaction du règlement de copropriété.
o Erreur d'appréciation de l'auteur du Règlement : par exemple dans l'échelle de répartition des charges d'ascenseur ou dans le mode de calcul des charges de chauffage.
o « Erreur » volontaire : le promoteur qui se réserve un appartement dans sa construction et qui sous-estime les charges qui seront demandées pour cet appartement.
La modification de la répartition des charges peut être nécessaire, non seulement dans les hypothèses que nous venons d'évoquer, mais encore, et essentiellement, dans le cas où cette répartition n'est pas conforme à la loi
Parce que les charges générales n'auront pas été calculées conformément aux trois critères de l'article 5 de la loi;
Parce que les charges afférentes aux services collectifs et éléments d'équipement communs n'auront pas été calculées en fonction de l'utilité,
Dans ces hypothèses, les erreurs de calcul auront pour conséquence que certains copropriétaires paieront trop de charges et d'autres pas assez.
La loi de 1938 ne permettait la modification des charges qu'à l'unanimité de tous les copropriétaires, sans recours au Tribunal.
Ce principe d'unanimité se retrouve dans les dispositions de l'article 11 de la loi; mais le législateur a permis, dans certains cas, et à certaines conditions, de remédier à des inégalités trop choquantes :
Toutefois, sans exclure totalement la possibilité de rectifier la répartition des charges le législateur de 1965 a entendu limiter cette faculté à une décision judiciaire et s'est essentiellement montré soucieux d'assurer la stabilité des relations au sein de la Copropriété, surtout sur le plan financier. Il a donc voulu limiter les hypothèses dans lesquelles les Tribunaux pourraient modifier ces charges en imposant des conditions rigoureuses de "révision judiciaire des charges" qui résultent de l'article 12 de la loi.
- La création de l’action en révision des charges par la loi du 10 juillet 1965
Article 11 de la loi du 10 juillet 1965
Sous réserve des dispositions de l'article 12 ci-dessous, la répartition des charges ne peut être modifiée qu'à l'unanimité des copropriétaires.
droit de la copropriété année 2019-2020
357
Article 12 de la loi du 10 juillet 1965
Dans les cinq ans de la publication du Règlement de Copropriété au fichier immobilier, chaque propriétaire peut poursuivre en justice la révision de la répartition des charges si la part correspondant à son lot est supérieure de plus d'un quart, ou si la part correspondant à celle d'un autre copropriétaire est inférieure de plus d'un quart, dans l'une ou l'autre des catégories de charges à celle qui résulterait d'une répartition conforme aux dispositions de l'article 10. Si l'action est reconnue fondée, le tribunal procède à la nouvelle répartition des charges.
Cette action peut également être exercée par le propriétaire d'un lot avant l'expiration d'un délai de deux ans à compter de la première mutation à titre onéreux de ce lot intervenue depuis la publication du règlement au fichier immobilier.
Lors d’une erreur dans le calcul des charges sur la surface du lot, les copropriétaires concernés peuvent demander la révision de la répartition des charges.
Cet article 12 constitue le seul moyen de révision judiciaire des charges de copropriété prévu par le législateur.
Mais nous verrons que sur la base du raisonnement développé par le Professeur GIVERDON sur le fondement de l’article 43 de la loi du 10 juillet 1965 qui répute non écrites les stipulations contraires à l’article 10 de la même loi, les Tribunaux ont admis la possibilité d’annuler des répartitions de charges contraires aux critères légaux. Faculté consacrée par le législateur en 1985.
L’action en nullité est disponible uniquement lorsque les bases de calcul de la répartition sont contraires aux dispositions d’ordre public de l’article 10 de la loi du 10 juillet 1965466.
L’action des copropriétaires tendant à faire déclarer « réputée non écrite » les stipulations du règlement de copropriété n’est recevable que si les copropriétaires précisent les clauses du règlement de copropriété qui seraient illicites et établissent le caractère illicite au regard des dispositions de l’article 10 de la loi du 10 juillet 1965467.
Si le demandeur invoque, non pas la méconnaissance des critères légaux de répartition des charges, mais la surélévation de celle-ci, en raison d’une interversion des surfaces entre l’un de ses lots et les autres lots, l’action engagée s’analyse comme une action en en révision de l’article 12 de la loi de 1965 qui est prescrite et non en une action en nullité468.
466 Cass. Civ. 3e 9 juin 2010
467 CA Paris 14 mai 2009, JD 2009-377895
468 Cass. Civ 3e 8 septembre 2009
droit de la copropriété année 2019-2020
358
- La création de l’action en « nullité » des charges par la jurisprudence, puis la loi du 31 décembre 1985
L'action en révision était donc la seule action en justice prévue par le législateur pour remédier aux répartitions de charges non conformes à la loi.
Cependant, dès la publication de la loi, GIVORD et GIVERDON ont fait observer que l'article 43 de la loi réputait non écrites les clauses contraires à l'article 10 de la loi et estimé que devait exister parallèlement à l'action en révision de l'article 12 de la loi, une action en nullité de l'article 43 de la loi.
Rappelons que l'article 43 de la loi répute non écrites toutes clauses contraires aux dispositions des articles 6 à 37; dès lors une répartition des charges contraires aux dispositions de l'article 10 de la loi constitue une clause qui doit être réputée non écrite.
La pratique des Tribunaux, consacrée par la révision légale du 31 décembre 1985, a ajouté une action dénommée -improprement- l'action en annulation des charges, par application de l'article 43 de la loi.
Dans ce cas, un copropriétaire peut faire constater, par application de l'article 43 de la loi que cette répartition est contraire aux dispositions impératives de la loi, telle que fixées par les articles 10 et 5, et en conséquence qu'il les déclare "non écrites" ou encore "inexistantes". Il en résultera un nouveau calcul des charges au bénéfice du demandeur.
2. Distinction entre les deux actions.
- Distinction proposée par Givord et Giverdon
Ces mêmes auteurs ont donné une distinction des deux types d'action :
L'action en révision réside dans le caractère lésionnaire d'une répartition, dont les bases sont néanmoins valables.
La cause de l'action en « nullité » repose sur le fait que ce sont les bases mêmes de la répartition qui sont contraires aux dispositions de la loi.
Il ne fait aucun doute qu'est contraire à la loi une répartition des charges relatives aux éléments d'équipement dès lors que cette répartition fait participer des lots qui n'ont aucune utilité de ces éléments d'équipement :
o Faire participer le rez-de-chaussée aux charges d'ascenseur est contraire aux dispositions mêmes de la loi.
o De même, faire participer un lot aux charges générales de l'immeuble pour une quote-part non proportionnel aux valeurs relatives des parties privatives comprises dans ce lot est
droit de la copropriété année 2019-2020
359
lésionnaire (pour lui si la quote part payée est supérieure à ce qu'elle devrait être, pour les autres si cette quote-part est inférieure à ce qu'elle devrait être).
- Critique de cette distinction
La distinction n'est pas aussi évidente qu'il y parait à première lecture :
- Dans le premier cas, la répartition est non seulement « nulle », mais elle est également lésionnaire : parce que tout paiement de charges non conforme à la loi est source de lésion.
- Dans le second cas, la répartition n'est pas seulement lésionnaire, mais elle est également nulle puisque "contraire aux dispositions de l'article 10 de la loi."
Pour que la distinction proposée par les Professeurs GIVORD et GIVERDON ait une justification réelle, il faudrait que les tribunaux adoptent une ligne de partage entre les deux types d'actions :
- Si les bases de la répartition impérativement fixées par l'article 10 de la loi n'ont pas été respectées, l'action en « nullité » peut être introduite.
- Si toutefois les bases de répartition ont été respectées, mais ont été mal appliquées, seule l'action en révision est possible.
De la sorte :
- Dispenser un lot de toute participation aux charges générales est contraire à la loi : il y a alors « nullité » de l'état de répartition des charges générales.
- Mais imputer au même lot des charges qui ne sont pas proportionnelles à la valeur relative de ce lot, n'est pas nul, mais lésionnaire.
- Faire participer un lot aux dépenses d'un service collectif ou à celles d'un élément d'équipement que ne présente aucune utilité pour ce lot justifie d'une action en « nullité ».
- Par contre, une erreur de calcul de cette utilité justifierait non une action en « nullité », mais une action en révision.
droit de la copropriété année 2019-2020
360
3. Modalités communes d’exercice des actions en modification judicaire des charges
- Le tribunal compétent.
Il s’agit du Tribunal Judiciaire du lieu de situation de l’immeuble
Article 62 du décret \no67-223 du 17 mars 1967
"Tous les litiges nés de l'application de la loi et du décret d'application sont de la compétence du lieu de situation de l'immeuble."
- Il n’y a pas lieu à publication de l'assignation.
Aux termes de l'article 30-5 du Décret du 4 janvier 1955 :
"Les demandes tendant à faire prononcer la résolution, la révocation, l'annulation ou la rescision de droits résultant d'actes soumis à publicité, ne sont recevables devant les tribunaux que si elles ont été elles-mêmes publiées conformément aux dispositions de l'article 28-4\degres, c, et s'il est justifié de cette publication par un certificat du conservateur ou la production d'une copie de la demande revêtue de la mention de publicité."
La Cour de Cassation, dans un arrêt du 18 décembre 1996469, au visa de l’article 30-5\degres du Décret du 4 janvier 1955 affirme : « l’action en révision du règlement de copropriété ne figure pas au rang de celles qui sont soumises à publicité en application des textes susvisés ».
- L'information des copropriétaires (article 59 du Décret \no67-223 du 17 mars 1967)
Aux termes de l'article 59 alinéa 1 du Décret \no67-223 du 17 mars 1967
"A l'occasion de tous litiges dont est saisie une juridiction et qui concernent le fonctionnement d'un syndicat ou dans lesquels le syndicat est partie, le syndic avise chaque copropriétaire de l'existence et de l'objet de l'instance".
Il s'agit là d'une disposition générale, donc non spécifique à l'action en révision des charges
469 Civ 3\degres 18 déc 1996, Lamarsalle, Loyers et Copropriété 1997 \no 85
droit de la copropriété année 2019-2020
361
Cet « avis », selon les termes du Décret, article 64 alinéa 2, est en fait une notification qui doit être faite valablement par lettre recommandée ou par remise contre émargement.
- La représentation par mandataire ad hoc et intervention volontaire (Art.54 du décret \no67-223 du 17 mars 1967).
"Chaque fois qu'une action en justice intentée contre le syndicat a pour objet ou peut avoir pour conséquence une révision de la répartition des charges, et indépendamment du droit pour tout copropriétaire d'intervenir personnellement dans l'instance, le syndic ou tout copropriétaire peut, s'il existe des oppositions d'intérêts entre les copropriétaires qui ne sont pas demandeurs, présenter requête au président du tribunal de grande instance en vue de la désignation d'un mandataire ad hoc. Dans ce cas, la signification des actes de procédure est valablement faite aux copropriétaires intervenants ainsi qu'au mandataire ad hoc".
Aux termes de cet article deux principes se dégagent
Intervention d'un copropriétaire.
Tout copropriétaire peut intervenir à l'action pour défendre son point de vue : si le demandeur triomphe en sa demande, les charges vont être modifiées et certains copropriétaires verront nécessairement leur quote-part augmenter. Il est bon de préserver à tous la possibilité d'intervenir dans ce type de procédure.
Il s’agira le plus souvent d’une intervention à titre accessoire, c’est à dire que le copropriétaire intervenant demandera au juge soit de débouter le copropriétaire demandeur, auquel cas il appuiera la défense du syndicat, soit de faire droit aux prétentions de ce copropriétaire, contre le syndicat des copropriétaires.
Cependant, on peut concevoir que ce copropriétaire fasse lui-même une intervention principale, en sorte qu’il ne s’alignera ni sur la position du copropriétaire demandeur, ni sur la position du syndicat défendeur, mais il invoquera une autre répartition. A notre sens, une telle intervention à titre principal ne sera recevable que si le copropriétaire intervenant respecte les conditions de délai posées par l’article 12 de la loi.
Le mandataire ad hoc
Le syndicat est représenté à la procédure par son syndic. Le syndic va donc orienter la position du syndicat, il risque en conséquence de devoir choisir entre les positions contradictoires des copropriétaires. Pour éviter un choix qui ne pourra le plus souvent qu'être source de critiques, un copropriétaire, ou le syndic lui-même pourront demander la désignation d'un mandataire ad hoc, c'est à dire d'un mandataire ad litem, chargé de représenter les intérêts du syndicat.
droit de la copropriété année 2019-2020
362
B. L'ACTION EN REVISION DE L'ARTICLE 12 DE LA LOI.
1. Les conditions d'ouverture :
- Lésion de plus du quart.
L'action en révision n'est recevable que dans la mesure où le copropriétaire demandeur justifie que la répartition actuelle dans l'une ou l'autre des catégories de charges est source de lésion.
soit parce qu'il paie plus du quart en trop par rapport à ce qu'il devrait payer si les principes de l'article 10 étaient respectés.
soit parce que la part incombant à un autre copropriétaire est inférieure de plus d'un quart à ce qu'elle devrait être.
Dans le premier cas, le copropriétaire s'estime directement lésé parce qu'il paie trop; dans le second cas, le copropriétaire se déclare indirectement lésé du fait qu'un autre est favorisé par une répartition contraire à la loi.
Cette proportion du quart a été fixée par la loi de 1965 après de longues discussions : le but du législateur était d'empêcher les actions en justice pour les écarts trop faibles.
Cette exigence de lésion de plus du quart (en plus ou en moins) est source d'aléa pour le copropriétaire demandeur : en effet, seul un expert judiciaire pourra déterminer s'il y a effectivement lésion de 25 %. Dans les cas "limites", ce n'est qu'à l'issue de la procédure que l'on saura si la demande était ou non fondée dans son principe (nous dirons plus juridiquement : si l'action était ou non recevable).
- Qui doit-on assigner ?
La procédure est différente selon que le copropriétaire demandeur se plaint de trop payer ou s'il estime qu'un autre copropriétaire ne paie pas assez :
• Article 52 du décret (le quart en trop) : le copropriétaire qui estime payer plus de 25 % de charges en trop assigne le syndicat et lui seul.
• Article 53 du décret (le quart en moins) : le copropriétaire qui estime qu'un autre copropriétaire paie plus du quart en moins que ce qu'il devrait payer par application des règles de l'article 10 de la loi assigne ce copropriétaire et appelle en la cause le syndicat des copropriétaires.
droit de la copropriété année 2019-2020
363
On note que dans tous les cas, la présence du syndicat des copropriétaires aux débats est nécessaire : c'est logique dès lors que le syndicat représente l'ensemble des copropriétaires et que l'action en révision, si elle est reconnue fondée aura pour conséquence de modifier les droits de tous les copropriétaires, soit parce qu'elle augmentera leurs charges, soit parce qu'elle les diminuera.
A cet effet, il ne fait aucun doute que le syndicat représentant chaque copropriétaire à la procédure, un copropriétaire ne saurait prétendre faire tierce opposition au jugement modifiant la répartition des charges, au motif qu'il n'avait pas été attrait à la cause.
En effet, s'il est maintenant de jurisprudence constante que le syndicat ne représente pas les copropriétaires dans leurs intérêts personnels, c'est à dire quant à leurs droits sur les parties privatives, par contre, il représente bien l'universalité des copropriétaires pour la défense des clauses du Règlement de Copropriété. Au demeurant les copropriétaires seront informés de l'existence de la procédure pendant le déroulement de celle-ci, en application de l'article 59 du Décret.
2. Les délais pour introduire l'action.
La loi article 12 prévoit deux délais : un délai de cinq ans et un délai de deux ans.
- Le délai de cinq ans à compter de la publication du règlement de copropriété
Ce délai court à compter de la publication du règlement de copropriété au fichier immobilier.
Rappelons l'article 13 de la loi aux termes duquel:
"Le règlement de copropriété et les modifications qui peuvent lui être apportées ne sont opposables aux ayants cause à titre particulier qu'à dater de leur publication au fichier immobilier".
Cette publication est donc prévue par la loi. Elle est également imposée par l'article 35 du décret du 4 janvier 1955 sur la publicité foncière.
- Le délai de deux ans à compter de la 1ère mutation du lot
L'action en révision peut être également introduite par un copropriétaire dans les deux ans à compter de la première mutation à titre onéreux de son lot intervenue depuis la publication du règlement de copropriété au ficher immobilier.
droit de la copropriété année 2019-2020
364
Ce délai de deux ans est distinct du délai de cinq ans ci-dessus
A titre d'exemple imaginons un Règlement de Copropriété publié le 25 janvier 1980. Le délai de cinq ans s’est achevé le 25 janvier 1985 (l'article 641 du N.C.P.C. précise que lorsqu'un délai est exprimé en années, ce délai expire le jour de la dernière année qui porte le même quantième que le jour de l’événement qui fait courir le délai).
Si un copropriétaire ayant acquis à l'origine de la mise en copropriété vend son lot le 15 mai 2010, son acquéreur disposera d'un délai expirant le 15 mai 2012 pour exercer l'action en lésion qui était prescrite pour son vendeur depuis le 25 janvier 1985. Ce qui nous permet d'affirmer que l'action en lésion dans le délai de deux ans de la première mutation sera beaucoup plus fréquente que l'action en lésion ouverte au premier acquéreur dans le délai de cinq ans de la publication du Règlement de Copropriété.
Imaginons par contre que le premier vendeur cède son lot un an après la mise en copropriété et en conséquence la publication du Règlement de Copropriété au Fichier Immobilier. Cela n'empêchera pas son acquéreur d'exercer l'action en révision trois ans après son achat puisque quand bien même le délai de deux ans sera écoulé, il sera encore dans le délai de cinq ans de la publication du Règlement de Copropriété (Aix 24 juillet 1984: Bull Aix 1984-2).
Mais seule la première acquisition du lot à titre onéreux ouvre ce délai de deux ans pour intenter l'action en révision : l'héritier - qui par définition n'est pas un acquéreur à titre onéreux - ne pourra pas invoquer l'ouverture du délai de deux ans pour intenter une action en révision. Par contre son acheteur sera le premier acquéreur à titre onéreux depuis la mise en copropriété de l’immeuble !
De la sorte, l'action en révision pourra être exercée de nombreuses années après la mise en copropriété de l'immeuble, parfois même après plusieurs successions.
- Sanction des délais de cinq et deux ans.
Le non respect de l'un ou l'autre de ces deux délais entraîne l'irrecevabilité de la demande (c'est une fin de non recevoir par prescription de l'action au sens de l'article 122 N.C.P.C.). L'article 12 étant lui même impératif, le juge peut soulever d'office l'irrecevabilité de la demande tirée du non respect des délais légaux.
3. Point de départ des délais
a. Immeubles vendus en l'état futur d'achèvement.
Point de départ du délai de cinq ans.
Lorsque l'immeuble est vendu en l'état futur d'achèvement, la société venderesse rédige, dépose au rang des minutes du notaire et fait publier le REGLEMENT DE COPROPRIETE avant même de procéder à la première vente.
droit de la copropriété année 2019-2020
365
On doit considérer en ce cas que le délai de cinq ans ne court qu’à compter de l’achèvement de l’immeuble puisque la loi ne s’applique qu’aux immeubles bâtis.
Toutefois, si le règlement de copropriété ne comporte pas d’état de répartition mais seulement « les bases selon lesquelles la répartition est faite pour une ou plusieurs catégories de charges », la question s’est posée dans une telle hypothèse de savoir si le délai de cinq ans commençait à courir de la publication du règlement de copropriété ne comportant que les bases de répartition (en l’espèce il était fait référence à la surface de chauffe) ou du jour où la quote-part de chaque lot avait été calculée.
La Cour de Paris a retenu cette dernière date comme point de départ du délai de cinq ans.470
Point de départ du délai de deux ans.
La loi sur la copropriété n'étant applicable qu'aux immeubles bâtis, doit-on considérer dans cette hypothèse que la première acquisition visée par l'article 12 de la loi est, non pas la vente en l'état futur d'achèvement, mais la première vente postérieure à l'achèvement de l'immeuble (en d'autres termes, le délai de deux ans s'appliquerait à la première revente du lot acquis par l'acquéreur en état futur d'achèvement)?
C’est cette dernière interprétation qui a été retenue par la Cour de Cassation.471
b. Location accession
Bien que la question ne se soit pas posée à ce jour, l’action en révision ne fait l’objet d’aucune disposition particulière au profit du locataire-acquéreur.
Le législateur de 1965 ignorait par principe la loi de 1984. Quant à la loi du 12 juillet 1994, si elle précise que vendeur-bailleur et locataire-accédant se répartissent les charges de copropriété selon la nature de ces charges, par contre, elle ne comprend aucune règle spécifique à la lésion.
C’est donc le vendeur-bailleur, et lui seul qui pourra exercer cette action jusqu’à l’arrivée du terme de la location-accession. Si le délai de cinq ans est dépassé à cette date, le locataire ne pourra pas intenter l’action en révision sur cette base. Par contre s’il est le premier sous-acquéreur, il bénéficiera alors du délai de deux ans.
470 PARIS 23\degres Ch. 6 juillet 1994, Supermarché Barbès, Loyers et Copropriété déc 1994 \no 485.
471 Civ. 3\degres 9 décembre 1980 JCP 81, IV, 77
droit de la copropriété année 2019-2020
366
c. Sociétés d'attribution et action en révision.
L'action en révision n'appartient qu'au copropriétaire : aux termes de l'article 12, c'est "chaque copropriétaire" qui peut poursuivre en justice la révision des charges.
En conséquence les associés d'une société d'attribution ne peuvent agir en révision dès lors que l'immeuble est en copropriété : en ce cas, seule la Société immobilière pourra exercer l'action pour le compte du lot concerné.
Etant observé cependant que tant que l'immeuble reste la propriété exclusive de la Société, chaque associé, en application de l'article L 212-6 code de construction et d'habitation pourra demander la révision judiciaire du règlement de jouissance quant à la répartition des charges.
4. Effets de l'action en révision.
a- La décision de justice exécutoire constitue le point de départ de la nouvelle répartition des charges
La procédure en révision judiciaire dure plusieurs années et bien souvent le jugement qui fixe la nouvelle répartition est frappé d'appel; procédure d'appel qui dure en moyenne près de deux ans !
Dans la mesure où la Cour infirme le jugement il n'y aura pas de difficulté : soit parce qu'elle déboute le demandeur, soit parce que tout en reconnaissant le bien fondé de la demande, elle adopte une nouvelle répartition différente de celle retenue par les premiers juges, auquel cas cette nouvelle répartition ne s'appliquera qu'à compter de l'arrêt de la Cour.
Mais si la Cour confirme le jugement la nouvelle répartition devra recevoir application à compter du prononcé du jugement, ceci par application des règles générales de la Procédure, et ceci alors même que le jugement n'était pas assorti de l’exécution provisoire. C'est d'ailleurs ce qui a été jugé par la Cour de Cassation Civ. 3\degres 23 avril 1992, AJPI 1994 p. 293.
"La Cour d'Appel qui confirme le jugement en toutes ses dispositions fixe justement la date d'effet de la nouvelle répartition à celle de la décision de première instance l'ayant ordonnée".
b- Publication de la décision de justice.
Pour être opposable aux acquéreurs successifs et par application de l'article 13 de la loi, la nouvelle répartition, telle que fixée par le juge devra être publiée au Fichier Immobilier.
droit de la copropriété année 2019-2020
367
A cet effet le demandeur à la modification judiciaire de la répartition des charges prendra soin de rédiger une assignation parfaitement claire dont le dispositif reproduira intégralement le nouveau tableau de répartition des charges entre les lots de Copropriété, en sorte qu'il n'y ait aucune difficulté lors de la publication du jugement ou de l'arrêt de la Cour.
Trop souvent en ce domaine, par suite de difficultés de publication tenant à des erreurs matérielles dans la rédaction de la demande, les parties doivent revenir en interprétation de jugement ou en rectification d'erreur matérielle.
Le copropriétaire demandeur pourra solliciter du juge qu’il ordonne la publication du jugement par le syndicat des copropriétaires et aux frais de ce dernier. Cependant le syndic est rarement diligent à cet égard, en sorte que le copropriétaire aura sans doute intérêt à prendre l’initiative de cette publication.
C. L'ACTION EN « NULLITE ».
Si l'action en révision est strictement réglementée par les textes, il en va tout différemment de l'action en « nullité », et ceci pour la raison très simple que nous avons exposée précédemment : le législateur n'avait pas envisagé qu'il fût possible de modifier judiciairement la répartition des charges par la voie de l'action en nullité.
Au demeurant les premières décisions de justice furent hostiles à l'action en « nullité » sur le fondement des dispositions de l'article 43 de la loi. La Cour de PARIS devait par exemple affirmer : " Les articles 10 et 12 ne peuvent être dissociés" et en conséquence les copropriétaires ne peuvent se soustraire aux dispositions de l'article 12 "qui sont l'expression de la volonté du législateur de limiter l'exercice du droit de révision aux fins d'assurer au régime de la copropriété la stabilité qui lui est nécessaire"472.
Rappelons que les termes "Action en nullité" sont inadéquats : le demandeur ne sollicite pas du Tribunal qu'il prononce la nullité de la répartition des charges, mais qu'il constate, par application de l'article 43 de la loi que cette répartition est contraire aux dispositions impératives de la loi, telle que fixées par les articles 10 et 5, et en conséquence qu'il les déclare "non écrites" ou encore "inexistantes". C'est donc par esprit de simplification que les auteurs parlent d'action « en nullité » et non d'action « en inexistence », terme qui eût été plus exact.
1. Délai d'exercice de l'action en « nullité »... Evolution jurisprudentielle.
a. La prescription de dix ans de l'article 42 alinéa 1.
472 PARIS 26 octobre 1970; A.J.P.I. 1971 p. 339 note Cabanac
droit de la copropriété année 2019-2020
368
Si les Tribunaux, changeant d'attitude vont admettre le principe de l'action en nullité venant s'ajouter à l'action en révision de l'article 12 de la loi, ils vont cependant tenter d'en limiter les conséquences.
Aussi feront-ils application de l'article 42 alinéa 1 de la loi de 1965 qui édicte que :
" Sans préjudice de l'application des textes spéciaux fixant des délais plus courts, les actions personnelles nées de l'application de la présente loi entre des copropriétaires ou entre un copropriétaire et le syndicat se prescrivent par un délai de dix ans ".
b. La prescription trentenaire.
Mais par plusieurs arrêts de cassation473, la Haute Juridiction devait admettre que les actions de l'article 43 de la loi se prescrivaient par le délai de droit commun de 30 ans au seul motif de la nullité de la répartition; en sorte que toutes les répartitions antérieures à la promulgation de la loi de 1965 auraient pu être contestées jusqu'au 10 juillet 1995 :
" Les clauses réputées non écrites par l'article 43 de la loi du 10 juillet 1965 étant non avenues par le seul effet de la loi, les copropriétaires demandeurs sont en droit de faire établir l'assiette et le mode de répartition des charges selon les critères légaux " (donc même au-delà du délai de l'article 42 de la loi).
c. Imprescriptibilité de l'action
C'est alors que les juristes ont voulu aller jusqu'au bout du raisonnement et ont relevé que l'action de l'article 43 n'était pas une action « en nullité », mais une action aux fins de faire constater l'inexistence de la répartition des charges.
Ils ont alors fait valoir que si une action en nullité se prescrit nécessairement - du moins à titre principal - parce que le législateur a limité à trente ans le délai pendant lequel on peut se faire reconnaître un droit (sauf cas de prescription plus courte); par contre une action qui n'a pas pour but de faire valoir la nullité d'un acte, mais de faire constater son inexistence n'est pas soumise à prescription. En sorte que l'action doit être déclarée imprescriptible.
Effectivement, par une série d'arrêts depuis 1987, la Cour de Cassation affirme le principe de l'absence de prescription :
• Civ 3\degres 1er avril 1987 (JCP N 1987 PRATIQUE p 624) :
"Les clauses réputées non écrites par l'article 43 de la loi du 10 juillet 1965 sont non avenues par le seul effet de la loi".
• Cf. également Civ. 3\degres 9 mars 1988 (D 89, 143 note Atias). Selon cet arrêt, la clause non écrite est censée "n'avoir jamais existé".
473 notamment Civ. 3\degres 11 jan 1983 in Revue de Dr. Imm. 1983 p. 423; Civ 13 juin 1984, D. 1984, IR 412
droit de la copropriété année 2019-2020
369
• Civ 3\degres 26 avril 1989 (D 89 IR 149) : Tout copropriétaire intéressé peut à tout moment attaquer l'absence de conformité des clauses du Règlement de Copropriété.
• Civ 3\degres 12 juin 1991 (Rev. DR. Immo. 1991, 379).
2. Les charges dont la nullité est demandée.
a. La nullité peut être demandée pour les charges générales ou les charges afférentes aux services collectifs et éléments d’équipements communs
Les Tribunaux ont également tenté de réduire le champ d'application de l'action en "nullité" en ne la déclarant recevable qu'en tant que dirigée à l'encontre de la répartition des charges de l'article 10 alinéa 1er (donc relatives aux seuls services collectifs et éléments d'équipement communs).
Nous avons rappelé précédemment (112) la distinction opérée par GIVORD \& GIVERDON entre action en révision (ayant pour fondement le caractère lésionnaire d'une répartition) et l'action « en nullité » (ayant pour fondement le fait que les bases mêmes de la répartition sont contraires aux dispositions de la loi).
Les Tribunaux en ont déduit dans un premier temps que si l'action « en nullité » pouvait parfaitement se concevoir pour les charges relatives à des équipements communs ou à des services collectifs qui n'étaient pas réparties conformément au critère de l'utilité, par contre elle ne pouvait être admise pour critiquer les charges générales qui peuvent seulement se voir reprocher un "mauvais calcul".
Le raisonnement se conçoit aisément avec des exemples : si le Règlement de Copropriété impute au lot du rez-de-chaussée des charges d'ascenseur, il ne respecte pas l'exigence du critère d'utilité; par contre s'il met à la charge d'un lot 200/1000 èmes de charges générales, alors que le respect des critères de l'article 5 aurait dû amener l'auteur du Règlement de Copropriété à n'imposer au lot concerné que 150/1000 èmes de charges, cela démontre une erreur de calcul entraînant lésion pour le copropriétaire du lot, mais ne démontre pas une nullité de la répartition.
En fait le raisonnement ne pouvait être tenu pour exact : en premier lieu lorsqu'on exonère un lot de charges générales, cela ne résulte pas d'une erreur de calcul, mais d'une disposition contraire à la loi. En second lieu, l'erreur de calcul ne se conçoit comme erreur et non comme atteinte aux critères légaux que si elle est purement matérielle; si l'erreur est "intellectuelle" cela signifie que son auteur n'a pas fait une juste application des critères de l'article 5 (superficie, consistance et situation).
droit de la copropriété année 2019-2020
370
Aussi la Cour de Cassation, à partir de 1990 (474) a admis que l'action en « nullité » était recevable à l'encontre de la répartition des charges de l'article 10 alinéa 2 (charges générales) comme elle l'est à l'encontre de la répartition des charges de l'article 10 alinéa 1er.
Pour autant l’assignation en justice devra être rédigée avec soin et de telle façon que le juge ne puisse requalifier une demande d’annulation (imprescriptible) en une demande de révision (soumise aux délais de cinq ans et deux ans).
La Cour d’Appel de Paris en donne un exemple intéressant le 15 septembre 2005475 :
« Les demandeurs ne démontrent pas, alors que cette preuve leur incombe en application de l’article 1315 du code civil que les critères de répartition de l’article 10 de la loi ont été méconnus, se bornant à soutenir, non utilement, que la quote-part des parties communes attachée à leur lot est trop élevée comparativement à celle affectée aux copropriétaires situés aux étages. S’agissant d’une demande en révision, cette demande est prescrite ».
b. La question particulière de la transformation des lots après la mise en copropriété de l’immeuble.
Il arrive qu’une répartition de charges soit conforme aux dispositions légales lors de la mise en copropriété de l’immeuble, mais que par la suite le lot se trouve modifié dans sa consistance, sa situation ou sa superficie.
Bien souvent en effet, le copropriétaire demandera à l’assemblée générale de l’autoriser à modifier la consistance de son lot, notamment en l’autorisant à se raccorder aux éléments d’équipement commun (canalisation d’eau pour créer un lavabo, une douche ou un wc) ou encore en annexant un couloir commun. Pour autant l’assemblée générale n’a pas nécessairement modifié la répartition des charges communes ou afférentes aux éléments d’équipement commun.
Peut-on en ce cas prétendre que la répartition des charges, conforme à la loi lors de la mise en copropriété est devenue contraire, « inexistante » après réalisation de ces travaux ?
En ce qui concerne les charges de l’article 10 al 2 (charges communes générales) la réponse devrait être négative puisqu’il faut tenir compte des critères de l’article 5 à l’époque de l’établissement de la copropriété476.
474 Civ. 3\degres 27 novembre 1990 Administrer mai 1991 p 53 note Guillot; Civ. 3\degres 6 mars 1991 Rev. Dr. Immo. 1991,101; Civ. 3\degres 12 juin 1991 Rev. Dr. Immo. 1991,379
475 Paris, 23\degres Ch B 15 septembre 2005, AJDI janvier 2006 p. 37
476 Article 5 : « telles que ces valeurs résultent lors de l'établissement de la copropriété »
droit de la copropriété année 2019-2020
371
Cependant on observera un arrêt de la cour de Paris du 16 mai 2001477 rendu dans les circonstances de fait suivantes :
En 1973 un copropriétaire est autorisé à réunir plusieurs lots en sorte que la surface totale des parties à jouissance privative est augmentée de 7 m². La cour d’appel retient :
« Il résulte en effet du rapport de l’expert Quignard que la répartition des charges telle qu’elle résulte du règlement de copropriété méconnaît actuellement les dispositions impératives de l’article 10 alinéa 2 de la loi du 10 juillet 1965, en ce qu’elle n’est plus en accord avec les valeurs relatives des parties privatives comprises ». En sorte que la cour d’appel a fait application de l’article 43 de la loi pour annuler la répartition des charges.
Cette décision paraît critiquable : la loi a prévu un mécanisme de modification des charges dans les dispositions de l’article 11 et le recours à l’article 43 devrait être exclu, s’agissant d’une irrégularité des charges générales postérieure à la mise en copropriété.
Rappelons également qu’aux termes de l’article 6 -3 de la loi du 10 juillet 1965 modifié par l’Ordonnance du 30 octobre 2020 : « Le règlement de copropriété précise, le cas échéant, les charges que le titulaire de ce droit de jouissance privative supporte. ». En sorte que l’assemblée générale qui accorde le droit de jouissance privatif pourra modifier le règlement de copropriété (application de l’article 11 alinéa 1) pour imposer au bénéficiaire de cette jouissance privative une participation complémentaire aux charges de copropriété ; ceci afin d’éviter une procédure ultérieure en modification judiciaire de la répartition des charges.
C’est également dans le sens d’une modification des charges que se prononcera la cour de cassation dans un arrêt du 22 juin 2005 (3\degres Chambre, Sanoyan) alors que la non-conformité aux critères de l’article 5 résulte d’une modification des lots :
En l’espèce un lot d’origine doté de 200/tantièmes est affecté à usage de réserve. Ce lot est voisin des emplacements de parkings dotés chacun de 20/tantièmes. Le propriétaire du lot « réserve » le divise en cinq emplacements de parkings d’une superficie égale pour chacun à la superficie des parkings voisins, affectant à chacun de ces nouveaux parkings 20 % des tantièmes du lot d’origine, soit … 40/tantièmes. Il résulte de ce découpage que chaque nouvel emplacement se voit doté du double des tantièmes de propriété … et en conséquence de charges des lots parkings d’origine. L’acquéreur d’un nouveau parking introduit une procédure en « annulation ». La cour d’appel lui donne raison et la cour de cassation rejette le pourvoi. Mais il est vrai que la cour de cassation reproche au syndicat des copropriétaires de ne pas avoir soutenu que l’aire de manoeuvre créé sur le lot d’origine était une partie commune spéciale aux seuls lots issus de la division du lot d’origine.
Cette jurisprudence s’est trouvée confirmée par un arrêt du 28 janvier 2016, publié au Bulletin478 : Tout copropriétaire peut, à tout moment, faire constater l'absence de conformité aux dispositions de l'article
477 Paris 23\degres Ch 16 mai 2001 Loyers et Copropriété oct 2001 \no 239
478 chambre civile 3 Audience publique du jeudi 28 janvier 2016 \no de pourvoi: 14-26921 Publié au bulletin Cassation
droit de la copropriété année 2019-2020
372
10, alinéa 1er et alinéa 2 de la loi du 10 juillet 1965, de la clause de répartition des charges, qu'elle résulte du règlement de copropriété, d'un acte modificatif ultérieur ou d'une décision d'assemblée générale et faire établir une nouvelle répartition conforme à ces dispositions. En l’espèce le non-respect des critères de l’article 5 résultait d’une transformation d’un appartement bien après la mise en copropriété de l’immeuble.
3. Les conditions d’ouverture de l’action en justice
a. Par qui l'action peut-elle être exercée ?
Par les copropriétaires.
Tout copropriétaire justifie d'un intérêt suffisant pour demander au juge de constater la nullité d'une répartition de charges. Il n'a pas à justifier d'une lésion quelconque, mais simplement que la répartition existante n'est pas conforme à la loi.
La différence avec l'action en révision est ici très appréciable : nous avons évoqué l'aléa de l'expertise dès lors que chaque expert a ses propres méthodes de calcul tant en ce qui concerne l'utilité d'un élément d'équipement qu'en ce qui concerne les coefficients à appliquer à chacun des critères de consistance et de situation des lots.
En sorte que telle répartition qui pourra paraître lésionnaire à un expert ne le sera pas nécessairement aux yeux d'un second expert.
Dans le cadre de l'action en "nullité" ce risque tenant au seuil de la recevabilité disparaît purement et simplement : l'action en nullité du copropriétaire sera recevable et fondée alors même qu'une répartition conforme à la loi ne modifie la répartition figurant dans le Règlement de Copropriété d'origine que de dix ou quinze %, par exemple.
Par le syndicat des copropriétaires.
Si l'article 12 de la loi n'ouvre qu'aux seuls "propriétaires" l'exercice de l'action en révision, par contre l'action en nullité peut être exercée par toute personne justifiant d'un intérêt à agir.
Or, le syndicat des copropriétaires a été lui-même considéré comme recevable à poursuivre en justice une action en "nullité" de la répartition des charges
On observera toutefois que si les tribunaux ont affirmé à plusieurs reprises que le syndicat pouvait invoquer l'illicéité d'une clause du Règlement de Copropriété tant par voie principale que par voie
droit de la copropriété année 2019-2020
373
d'exception479, en pratique les exemples cités sont relatifs à des demandes reconventionnelles formulées par le syndicat sur la demande principale d'un copropriétaire à l'effet de voir annuler une résolution d’assemblée générale refusant d'appliquer la répartition prévue au Règlement de Copropriété ; auquel cas le « défendeur » à la demande reconventionnelle en « nullité » se trouve tout désigné. Par contre, on peut se demander contre qui le syndicat des copropriétaires devrait introduire à titre principal une action en « annulation » d’une répartition de charges du Règlement de Copropriété.
b importance d'une bonne rédaction de la demande en justice.
Même si les demandes de révision et d'annulation sont régies par des règles de procédure identiques comme nous venons de le voir, il n'en demeure pas moins qu'il s'agit de deux demandes parfaitement distinctes et qui n'ont pas le même objet !
En conséquence il convient d'être extrêmement précis dans la formulation de la demande.
- Pratique judiciaire
L'action en « nullité » est apparue très rapidement comme l’action de repêchage que l'on tente d'exercer lorsque les délais de l'action en révision sont épuisés.
En effet si l'action en révision de l'article 12 de la loi est un véritable parcours d'obstacles, par contre l'action dite "en nullité" ne répond à aucun formalisme particulier, et surtout… est imprescriptible !
Le juge aura tendance à requalifier en action en révision une demande en annulation, surtout lorsque le requérant aura fait valoir dans sa demande les différences de charges affectées à son lot par rapport aux autres lots alors qu’il aurait dû expliquer en quoi l’un ou plusieurs des trois critères (consistance, situation, superficie) n’avaient pas été respectés.
La cour de cassation encourage les juges du fond dans cette attitude restrictive, ayant affirmé dans un arrêt de 2011 que « La cour d’appel n’a pas donné de base légale à sa décision en ne précisant pas en quoi la clause de répartition initiale était contraire aux critères légaux ».480
Il a été jugé par exemple que si le copropriétaire a demandé au Tribunal, en première instance, de constater la lésion de la répartition des charges, il ne peut en cour d'Appel y substituer une demande d'annulation : il s'agirait alors d'une demande nouvelle, comme telle irrecevable.
479 Civ 3\degres 9 mars 1988, JCP 89.II. 21248; PARIS 12 nov 1992 Loyers Copropriété, fev 93, \no 70 et en dernier lieu VERSAILLES 29 avr 1993; Dalloz 1994, Som. 197
480 Civ. 3\degres, 7 juin 2011, F-D ; Loyers et Copropriété sep 2011 \no 252
droit de la copropriété année 2019-2020
374
De la même façon a été rejetée une demande aux fins de constater la nullité d'une répartition des tantièmes de charges au motif que la répartition des tantièmes de propriété est intangible !481
En pratique le copropriétaire demandeur sollicitera du Tribunal, à titre principal qu'il déclare « nulle » la répartition des charges (article 43) et à titre subsidiaire qu'il déclare cette répartition lésionnaire (article 12) s’il est encore dans les délais pour ce faire.
4. L’obligation faite au juge de donner au Syndicat des Copropriétaires une nouvelle répartition.
a. La solution jurisprudentielle avant 1985.
Lorsque les Tribunaux ont admis "l'action giverdonienne", ils ont effectivement prononcé la nullité de la répartition des charges prévue par le Règlement de Copropriété, mais constatant qu'aucun texte de loi ne les autorisait à substituer une nouvelle répartition à l'ancienne, ils ont invité les syndicats de copropriété à adopter une nouvelle répartition, conforme à la loi, en Assemblée Générale, statuant à l'unanimité.
Tant et si bien qu'on a vu des copropriétés, faute d'unanimité, décider de continuer à faire application "à titre provisoire" de la répartition annulée par la Cour d'Appel, et ce avec la bénédiction des tribunaux !
Mais la Cour de Cassation a alors affirmé482 que le Tribunal annulant une répartition de charges devait impérativement procéder à la nouvelle répartition qui s'imposerait aux copropriétaires.
b. Consécration législative.
Cette solution a été consacrée par la loi du 31 décembre 1985 qui a complété l'article 43 de la loi en édictant :
Article 43 alinéa 2 de la loi du 10 juillet 1965 dans sa rédaction du 31 décembre 1985
"Lorsque le juge, en application de l'alinéa premier du présent article, répute non écrite une clause relative à la répartition des charges, il procède à leur nouvelle répartition".
481 Civ. 6 mars 1991, précité
482 Civ. 3\degres, 13 juin 1984 D 84, IR 386
droit de la copropriété année 2019-2020
375
Cet article est la consécration législative de l'action en nullité : il évite le renvoi à l'assemblée générale statuant à l'unanimité de l'article 11 pour adopter une nouvelle répartition; unanimité impossible à obtenir.
Cette réaction implique que le juge a l’obligation de fixer une nouvelle répartition. C’est ce qui a été jugé par la cour de cassation : Le juge ne peut pas réputer non écrite une clause relative à la répartition des charges de copropriété sans procéder à leur nouvelle répartition483
5. La nouvelle répartition a-t-elle effet pour l’avenir ou effet rétroactif ?
La jurisprudence ancienne.
Dès l’origine les juges ont fait application à l'action en "nullité" de la jurisprudence dégagée à propos de l'action en révision : la nouvelle répartition n'aurait d'effet que pour l'avenir (Cass. 3e civ., 10 oct. 1990 : Juris-Data \no 1990-003029. – 3 juill. 1996 : Bull. civ. III, \no 171).484
Il est vrai cependant que par application des principes généraux du droit, ce qui est nul n'a pas eu d'effet, et on pourrait en conséquence remonter les effets de la nullité à... la publication du Règlement de Copropriété.
En sorte que les effets de l’ « inexistence de la clause » ne concerneraient que l’imprescriptibilité de l’action.
Arrêt du 7 avril 2004 : en l’absence de répartition de charge préexistante, la répartition des charges par le juge n’a d’effet que pour l’avenir
Un arrêt du 7 avril 2004485 refusant de faire application rétroactive d’une répartition des charges imposée par le juge dans un immeuble dont le Règlement de Copropriété ne comportait pas jusque lors de répartition de charges, a été interprété par la doctrine comme une confirmation de la non-rétroactivité.
Un copropriétaire refuse de payer, il est poursuivi par le syndicat des copropriétaires et demande à être remboursé du trop-perçu. La cour d’appel le déboute. Pourvoi aux termes duquel il invoque que « lorsqu'une disposition du règlement de copropriété est réputée non écrite, comme étant contraire
483 Civ. 3\degres, 30 janvier 2008, \no 06-19773, Bulletin, III, \no 18
484 En dernier lieu Civ. 3\degres, 10 oct 1995 La Mongie:Dame Ara Loyers et copr. janvier 1996 \no 42.
485 Civ 3\degres Chambre \no2-13978
droit de la copropriété année 2019-2020
376
à une règle d'ordre public telle que l'article 10 de la loi \no 65-557 du 10 juillet 1965, la nullité vaut, non seulement pour l'avenir, mais également pour le passé, la disposition étant censée n'avoir jamais existé ; qu'en décidant le contraire, les juges du fond ont violé les articles 10 et 43 de la loi du 10 juillet 1965, ensemble le principe suivant lequel la nullité, sauf exceptions, a un effet rétroactif ».
La cour de cassation rejette le pourvoi : « Mais attendu qu'ayant relevé que le règlement de copropriété ne déterminait ni les tantièmes de répartition des charges de copropriété ni sur quelles bases cette répartition devait être faite et que le syndic n'avait pu jusqu'ici répartir les charges qu'en fonction des tantièmes de propriété des parties communes, d'où il résultait que la répartition des charges devait être faite judiciairement, une mesure d'instruction s'imposant à cet effet, la cour d'appel a retenu à bon droit que la nouvelle répartition qui sera judiciairement arrêtée n'aura d'effet que pour l'avenir » .
En réalité, il est possible que la « non rétroactivité » édictée par cet arrêt soit strictement limitée au cas d’espèce, à savoir l’hypothèse où il n’existait pas de tantièmes de charges de copropriété.
Les arrêts de 2005 : rétrocativité
Un arrêt … très remarqué de la 3ème chambre du 2 mars 2005 rendu par la cour de cassation486 casse une décision d’appel refusant l’effet rétroactif à l’annulation de la répartition des charges.
Vu l'article 43, alinéa 1er, de la loi du 10 juillet 1965 ; Attendu que toutes clauses contraires aux dispositions des articles 6 à 37, 42 et 46 de cette loi et celles du règlement d'administration publique prises pour leur application sont réputées non écrites
Attendu, selon l'arrêt attaqué (Basse-Terre, 10 février 2003), que la société Bazar des Iles (la société), condamnée à payer un arriéré de charges de copropriété, a assigné le syndicat des copropriétaires de la résidence La Darse en annulation des clauses du règlement de copropriété lui imposant de contribuer aux charges d'ascenseur pour le lot en rez-de-chaussée dont elle est propriétaire et à celles d'entretien pour des parkings qu'elle ne possède pas
Attendu que pour condamner la société à payer à ce titre une certaine somme au syndicat des copropriétaires, l'arrêt qui annule ces clauses énonce que sa décision n'a pas de caractère rétroactif et retient que la nouvelle répartition des charges ne prendra effet qu'après la signification de l'arrêt qui, au vu du résultat d'une mesure d'instruction, la déterminera
Qu'en statuant ainsi, alors qu'une clause réputée non écrite est censée n'avoir jamais existé, la cour d'appel a violé le texte susvisé ;
A la suite de cet arrêt, les commentateurs se sont interrogés sur la date qui devait être effectivement retenue comme point de départ de la nouvelle répartition.
486 Civ 3ème Chambre (Société Bazar des Iles c/syndicat des copropriétaires de la résidence La Darse) Juris-Data \no2005-027253; Loyers et copr. 2005, comm. 98 et la note
droit de la copropriété année 2019-2020
377
D’autant que par un nouvel arrêt la cour de cassation va affirmer le principe selon lequel une clause de répartition de charges contraire aux dispositions impératives de la loi est réputée non écrite et censée n’avoir jamais existé.
Civ. 3ème 27 septembre 2005487
« Sur le premier moyen
Vu l’article 43 de la loi du 10juillet1965;
• Attendu que toutes clauses contraires aux dispositions des articles 6 à 37, 42 et 46 de cette loi et celles du règlement d’administration publique prises pour leur application sont réputées non écrites
• Attendu, selon l’arrêt attaqué (Paris, 28 nov. 2002), que le syndicat des copropriétaires du 6 rue Emile-Duclaux a assigné M. K.., copropriétaire, en paiement de charges de copropriété
•Attendu que pour accueillir la demande, l’arrêt retient que M. K..soutient que les charges réclamées par le syndicat ne sont pas réparties selon l’article 10 de la loi du 10juillet1 965 et qu’il aurait dû attendre la décision de la cour d’appel de Versailles avant de lui réclamer de nouvelles charges, mais que l’arrêt de cassation du 21 novembre 2000 n’a cassé l’arrêt de la cour d’appel de Paris qu’en ce qu’il a dit les demandes irrecevables concernant les charges d’ascenseur, de tapis et de revêtement des escaliers, qu’en conséquence le syndicat est bien fondé à demander le paiement des charges, nonobstant l’arrêt de la cour d’appel de Versailles à intervenir et qu’en ce qui concerne les charges d’ascenseur, à les supposer mal réparties, la nouvelle grille ne s’appliquant que pour l’avenir, seule s’applique l’actuelle répartition en l’absence de décision définitive l’annulant.
Qu’en statuant ainsi, alors qu’elle avait constaté qu’une décision devait intervenir sur la demande de ce copropriétaire tendant à voir réputer non écrite la clause de répartition de certaines charges et qu’une clause réputée non écrite est censée n’avoir jamais existé, la cour d’appel a violé le texte susvisé. »
Dans cette affaire, la cour d’appel de Paris a condamné un copropriétaire au paiement de ses charges, considérant que la cour de Versailles, saisie par le défendeur, n’avait pas encore statué sur la légalité des clauses relatives à la répartition des charges, celles-ci demeuraient bien exigibles, puisqu’une éventuelle répartition judiciaire ne pourrait prendre effet que pour l’avenir. La Cour de cassation accueille le pourvoi, au motif qu’en écartant toute rétroactivité à l’annulation d’une clause atteinte de nullité, la cour de Paris a violé le texte de l’article 43 de la loi.
487 3ème Chambre civ.27 septembre 2005, \no 03-12402, Inédit, Synd. 6 rue Emile-Ducaux à Paris
droit de la copropriété année 2019-2020
378
Déjà certains Tribunaux ont déduit de cet arrêt qu’il fallait appliquer le principe de rétroactivité à la nouvelle répartition des charges. Les conséquences peuvent être graves : comment en effet récupérer les charges dues sur des copropriétaires ayant quitté leur immeuble depuis six ou sept ans ?
Pour autant M CAPOULADE propose une interprétation séduisante qui concilie les points de vue et qui peut être ainsi résumée488 :
Lorsque le juge est amené à déclarer non écrite une clause du Règlement de Copropriété par application de l’article 43 de la loi il rend une décision à caractère déclaratif puisqu’il constate le caractère non écrit de la clause. Mais lorsque sa décision porte sur une clause de répartition de charges, celle-ci devra en outre fixer une nouvelle répartition par application des dispositions de l’article 43 alinéa 2. Bien évidemment cette partie de la décision est constitutive de droit puisqu’elle retient une nouvelle répartition.
M CAPOULADE de conclure : « La rétroactivité se concilie avec le premier terme (jugement déclaratif) mais non avec le second (jugement constitutif) ».
Il est vrai que le juge peut déclarer non écrite une imputation de charges à un lot (faute d’utilité par exemple d’un élément d’équipement) sans fixer alors de nouvelle répartition. Auquel cas la rétroactivité jouerait nécessairement….
Enfin, il convient de citer un arrêt de la cour de cassation du 28 avril 2011 ayant cassé un arrêt de la Cour d’Aix ayant condamné un copropriétaire au paiement de charges pour des travaux votés en 2005 et appelés sur la base de la répartition du Règlement de copropriété homologué par le Tribunal mais ultérieurement annulé par la Cour. Cassation au motif « Qu'en statuant ainsi, sans répondre aux conclusions des copropriétaires selon lesquelles les charges réclamées par le syndicat avaient été calculées sur la base d'un règlement de copropriété qui leur était inopposable en raison de la rétractation de l'homologation de ce règlement ordonné par l'arrêt de la cour d'appel d'Aix-en-Provence du 20 février 2009, la cour d'appel n'a pas satisfait aux exigences du texte susvisé »489
La cause était-elle entendue en faveur de l’effet rétroactif de la nouvelle répartition ?
C’était l’opinion de Monsieur le Conseiller ROUZET dans un commentaire intitulé : « Le fantasme de la non-rétroactivité »490.
488 La Clause réputée non écrite en copropriété immobilière, Administrer \no 409, avril 2008
489 Civ. 3\degres, 28 avril 2011, \no 10-15264, au Bulletin
490 Mélanges en l’honneur du Professeur Jean HAUSER, décembre 2012, Editions Lexis Nexis Les Paradoxes du réputé non écrit appliqué au droit de la copropriété p. 1013
droit de la copropriété année 2019-2020
379
M ROUZET considèrait que la rétroactivité était la conséquence juridique inéluctable du réputé non-écrit, en sorte que seule le législateur pouvait intervenir : « Il revient au pouvoir législatif, s’il estime un texte mal interprété par l’autorité judiciaire, de le compléter, de l’amender ou de le modifier ».
Si rétroactivité il y avait, devait-elle s’appliquer à l’ensemble des répartitions depuis dix ans ou seulement aux sommes demandées au copropriétaire contestataire sous forme de répétition de l’indû ? Pour autant d’ailleurs que le paiement d’une somme perçue en application d’une disposition contractuelle en vigueur à l’époque constitue une somme « indue » ?
Une réponse semble avoir été donnée par un arrêt du 8 février 2012 de la troisième chambre de la cour de Cassation (Non publié au Bulletin), rendu en ces termes
« Vu l'article 2224 du code civil, ensemble l'article 26-II de la loi du 17 juin 2008 : Attendu que l'action en restitution de sommes indûment versées au titre des charges de copropriété, frais et honoraires de recouvrement, qui relève du régime spécifique des quasi-contrats, est soumise à la prescription qui régit les actions personnelles ou mobilières ; que les dispositions de la loi du 17 juin 2008 qui réduisent la durée de la prescription s'appliquent aux prescriptions à compter du jour de l'entrée en vigueur de la loi, sans que la durée totale puisse excéder la durée prévue par la loi antérieure »
Rappelons que l'action en recouvrement des charges était soumise à la prescription de dix ans prévue à l'article 42491 de la loi du 10 juillet 1965 pour les actions personnelles du syndicat à l'encontre des copropriétaires.492
Pourtant, s’agissant de l’action en restitution des charges trop payées engagées par un copropriétaire, la Cour de Cassation fait application de la prescription quinquennale de l’article 2224 du Code Civil, au motif que cette action « relève du régime spécifique des quasi contrats ».
\degres L’Arrêt du 10 juillet 2013 : retour à la non rétroactivité
L'arrêt493 du 10 juillet 2013 apporte une réponse qui va dans le sens du caractère constitutif de cette nouvelle répartition: « Mais attendu (…) qu'ayant exactement relevé que lorsqu'il répute non écrite une clause de répartition de charges, le juge doit procéder à une nouvelle répartition, la cour d'appel a retenu, à bon droit, que la décision de réputer non écrite une telle clause ne peut valoir que pour l'avenir et ne peut prendre effet qu'à compter de la date où la décision a acquis l'autorité de la chose jugée ».
Cette réponse paraît d'autant plus nette que la cour d'appel avait expressément rejeté les demandes de remboursement formées par les demandeurs au pourvoi en cassation.
La Cour de Cassation semble pourtant ferme désormais puisqu’elle a confirmé à nouveau sa position dans un arrêt du 21 janvier 2014 : « la décision de réputer non écrite une clause de
491 Délai ramené au délai de droit commun de 5 ans par la loi ELAN
492 PARIS 5 janvier 1988, Loyers et Cop. 1988 \no 138; PARIS 10 juillet 1990 Loyers et Copropriété 1990 \no 401; PARIS 19\degres Ch. 21 novembre 1994, Loyers et copropriété mai 1995 \no 238.
493 Civ 3\degres 10 juillet 2013, Pourvoi \no 12-14569, au Bulletin
droit de la copropriété année 2019-2020
380
répartition de charges ne vaut que pour l’avenir et ne prend effet qu’à compter de la date à laquelle la décision a acquis l’autorité de la chose jugée ».
(Cour de Cassation civ. 3ème 21 janvier 2014 \no12-26.689 ; JurisData \no2014-000811, Cour de Cassation).
Cette jurisprudence sera consacrée par la loi ELAN
6. Consécution législative de la non-rétroactivité de l’action en « nullité » L’Ordonnance du 30 octobre 2019 a consacré cette dernière jurisprudence en fixant le point de départ de la nouvelle répartition Article 43 Modifié par Ordonnance \no2019-1101 du 30 octobre 2019 - art. 38 « Toutes clauses contraires aux dispositions des articles 1er, 1-1, 4, 6 à 37, 41-1 à 42-1 et 46 et celles du décret prises pour leur application sont réputées non écrites. Lorsque le juge, en application de l'alinéa premier du présent article, répute non écrite une clause relative à la répartition des charges, il procède à leur nouvelle répartition. Cette nouvelle répartition prend effet au premier jour de l'exercice comptable suivant la date à laquelle la décision est devenue définitive. » Cette disposition est bienvenue, non seulement en ce qu’elle écarte les difficultés rencontrées pour récupérer la différence de charges sur la période antérieure au jugement mais également en ce qu’elle évite de diviser l’exercice comptable en cours.
Si la décision devient définitive en février 2021, la nouvelle répartition ne s'appliquera qu'au 1er janvier 2022 dès lors que l'exercice comptable commence 1er janvier de l'année.
Rappelons qu'en cas d'appel la décision ne devient définitive qu'à la date du prononcé de l'arrêt par la cour d'appel, ceci quand bien même depuis le 1er janvier 2020 les décisions de première instance sont assorties de l’exécution provisoire, sauf décision contraire du juge.