\chapter{Les travaux en copropriété}

Dans le cadre de la copropriété des travaux peuvent être entrepris soit par le syndicat des copropriétaires (section \ref{21_I}) et aux frais de la copropriété (SECTION 2) soit par des copropriétaires à titre personnel et à leurs frais (SECTION 3)

Le régime des travaux sera donc très différent puisque dans la première hypothèse il y aura décision des copropriétaires d'entreprendre des travaux alors que dans la seconde hypothèse il y aura décision du copropriétaire de faire ces travaux à titre personnel avec éventuellement la nécessité préalable d’une autorisation de l'assemblée générale des copropriétaires.

\section{Les travaux du syndicat des copropriétaires}\label{21_I}
	
	L' immeuble du syndicat des copropriétaires doit être entretenu. Il s'agira le plus souvent de maintenir en état d’usage ce qui existe; éventuellement de remplacer des équipements ou des éléments d’équipement qui sont devenus vétustes ou qui sont défaillants.
	
	Mais sans aller jusqu’à faire de la promotion - l’objet du syndicat est limité à l'entretien, à la conservation et à l’amélioration de l'immeuble1-, le syndicat des copropriétaires peut avoir intérêt à améliorer le bien commun; il peut même paraître opportun de modifier l'état des lieux initial. Il s'agira alors d'apporter une plus-value à l’immeuble.
	
	Sous l'empire de la loi de 1938 ces améliorations ne pouvaient intervenir qu'avec l'accord unanime de tous les copropriétaires. La loi de 1965 à créé un véritable régime des améliorations. Cette faculté de réaliser des améliorations au sein de la copropriété a été renforcée par l'adoption de la loi du 21 juillet 1994, puis de la loi SRU du 13 décembre 2000, et enfin par la loi ENE du 12 juillet 2010 qui ont soumis l'adoption de travaux d'amélioration à une majorité assouplie par rapport aux dispositions d'origine.
	
	Plus récemment, la loi ALUR du 24 mars 2014 et loi ELAN du 23 novembre 2018 et, enfin, l’ordonnance qui s’en suivit le 30 octobre 2019 a de nouveau abaissé les règles de majorités afin de faciliter les travaux. D’un point de vue symbolique2 l’ordonnance de 2019 a également inscrit dans l’objet du syndicat, à l’article 14 de la L. du 10 juill. 1965, l’amélioration de l’immeuble qui n’y figurait pas.
	
	Il convient d’étudier successivement les règles de conclusion des marchés de travaux, devenues beaucoup plus rigoureuses depuis la loi SRU du 30 décembre 2000 (SOUS-SECTION 1), le régime des travaux d’entretien (SOUS-SECTION 2) et le régime des travaux d’amélioration (SOUS-SECTION 3).
	1 Sous réserve des nouvelles dispositions de l’article 35 de la loi pour la réalisation de nouveaux locaux à usage privatif en surélévation.
	2 V\degres sur la portée symbolique de cet ajout, P.-e. Lagraulet, « L’administration de la copropriété réformée », AJDI 2019, p. 852 et s.
	
	\subsection{Règles générales de vote et de conclusion des marches de travaux}
	
		Les contraintes relatives à l’information et au contrôle des copropriétaires quant aux marchés de travaux se sont progressivement accrues.
		
		\subsubsection{L’avis obligatoire du Conseil Syndical}
		
			Article 21 de la Loi du 10 juillet 1965
			L'assemblée générale des copropriétaires, statuant à la majorité de l'article 25, arrête un montant des marchés et des contrats à partir duquel la consultation du conseil syndical est rendue obligatoire.
			Article 11 II 3 du décret du 17 mars 1967
			Sont notifiés au plus tard en même temps que l'ordre du jour :
			II – Pour l’information des copropriétaires
			«3\degres L'avis rendu par le conseil syndical lorsque sa consultation est obligatoire, en application du deuxième alinéa de l' article 21 de la loi [65-557] du 10 juillet 1965.
			La communication de cet avis impose une délibération du Conseil Syndical et un procès-verbal de cette délibération ou, à tout le moins, une lettre du Président du Conseil Syndical faisant connaître l’avis du conseil sur le marché proposé.
			Toutefois, le défaut de consultation n’est pas une cause de nullité3 de plein droit dans la mesure où sa notification n’est donnée, depuis le décret du 27 mai 2004, que pour « l’information des copropriétaires » et non pour la « validité de la décision »4. Il appartiendra au magistrat d’apprécier si ce défaut d’information a pu créer une irrégularité dans le vote.
			
		\subsubsection{La mise en concurrence des entreprises}
		
			L’article 21 de la loi, précité, a été complété par la loi du 13 décembre 2000 en ces termes
			Article 21
			L'assemblée générale des copropriétaires, statuant à la majorité de l'article 25, arrête un montant des marchés et des contrats à partir duquel la consultation du conseil syndical est rendue obligatoire. A la même majorité, elle arrête un montant des marchés et des contrats à partir duquel une mise en concurrence est rendue obligatoire.
			3 Pour des arrêts ayant retenu la nullité de la résolution, v\degres CA Paris Ch. 23 B, 23 mai 2001, no 1999/12364 et 1999/17229 : Jurisdata no 2001-146858 ; Loyers et copr. 2001, comm. 262 ; Vo également en ce sens CA Lyon ch. civ. 01 B, 28 mai 2013, no 12/00960
			4 En ce sens, J.-M. Roux, Le conseil syndical, LexisNexis, 2011, no 178, p. 106
			Le décret du 27 mai 2004 a ajouté un article 19-2 qui précise les modalités de cette mise en concurrence, mais restreignait cette mise en concurrence aux contrats de « travaux et fourniture », ce qui excluait les contrats de prestation de services (syndics, avocats, architectes …)
			Le décret du 20 avril 2010 est revenu sur cette restriction en adoptant une formule plus large visant « les marchés de travaux et les contrats autres que le contrat de syndic ». Il paraît en conséquence que tous les contrats, y compris d’avocats, architectes, etc. aient à être mis en concurrence lorsqu’un montant est fixé à partir duquel une mise en concurrence est rendue obligatoire est adopté par l’assemblée générale.
			Article 19-2 du décret (Modifié par Décret \no2010-391 du 20 avril 2010 - art. 11)
			La mise en concurrence pour les marchés de travaux et les contrats autres que le contrat de syndic, prévue par le deuxième alinéa de l’article 21 de la loi du 10 juillet 1965, lorsque l’assemblée générale n’en a pas fixé les conditions, résulte de la demande de plusieurs devis ou de l’établissement d’un devis descriptif soumis à l’évaluation de plusieurs entreprises.
			Ces modalités doivent préserver la transparence des marchés sans alourdir inutilement la passation des marchés et contrats : il suffit de justifier avoir demandé une pluralité de devis (même si un seul devis a été obtenu)5.
			Par contre selon la cour de Paris, en l’absence de document faisant une analyse des propositions reçues, le syndic doit communiquer avec la convocation à l’assemblée générale faisant état de trois consultations d’entreprises, l’ensemble des devis obtenus et non pas deux sur trois\footnote{6}
			
		\subsubsection{Le syndic et la transparence des marchés (art. 18 de la loi et 39 du décret)}
		
			Le syndic est assujetti à des obligations particulières de transparence lorsqu’une convention (marché, étude, contrat) est conclu avec une entité juridique dans laquelle il a des intérêts directs ou indirects.
			
			Après plusieurs élargissements du dispositif, le texte de l’article 18-1 A issu de l’Ordonnance du 30 octobre 2019 (applicable au 1er juin 2019) dispose
			ARTICLE 18-1 A MODIFIE PAR ORDONNANCE \no2019-1101 DU 30 OCTOBRE 2019 - ART. 16
			II.-Le syndic peut conclure avec le syndicat une convention portant sur des prestations de services autres que celles relevant de sa mission de syndic, après autorisation expresse de l'assemblée générale donnée à la majorité des voix exprimées de tous les copropriétaires présents, représentés ou ayant voté par correspondance. Ces prestations ne peuvent figurer dans le contrat de syndic.
			Le syndic soumet à l'autorisation de l'assemblée générale prise à la même majorité toute convention passée entre le syndicat et une personne ou une entreprise avec laquelle le
			5 En ce sens, Civ. 3e, 15 avr. 2015, no 14-13.255 ; Civ. 3e, 27 nov. 2013, no 12-26.395.
			6 Cour d'Appel Paris, pôle 4, 2\degres Ch, 20 janv. 2016, \no 13/23988 : Loyers et Copropriété, fev. 2016 \no 108.
			syndic a des liens de nature capitalistique ou juridique, en précisant la nature des liens qui rendent nécessaire l'autorisation de la convention.
			Les conventions conclues en méconnaissance de ces dispositions ne sont pas opposables au syndicat.
			La sanction prévue désormais par le texte est donc l’inopposabilité pure et simple au syndicat des copropriétaires de la convention conclue sans que les copropriétaires aient été informés de la nature des liens entre le syndic et la société choisie, donc l’obligation pour le syndic d’en supporter seul le coût, dès lors que le tiers est de bonne foi.
			Le décret actuel, précise les « liens capitalistiques » visés mais il devrait être modifié par suite de l’ordonnance
			Article 39 du décret modifié par Décret \no2010-391 du 20 avril 2010 - art. 21
			Doit être spécialement autorisée par une décision de l’assemblée générale toute convention entre le syndicat et le syndic, ses préposés, son conjoint, le partenaire lié à lui par un pacte civil de solidarité, ses parents ou alliés jusqu’au troisième degré inclus.
			Il en est de même des conventions entre le syndicat et une entreprise dont les personnes mentionnées à l’alinéa précédent sont propriétaires ou détiennent une participation dans son capital, ou dans lesquelles elles exercent des fonctions de direction ou de contrôle, ou dont elles sont salariées ou préposées.
			Ce texte bien évidemment a pour but d’éviter les collusions entre syndics et entreprises pour l’obtention des marchés… mais son efficacité est douteuse : il n’est pas nécessaire d’avoir des intérêts personnels ou familiaux dans une entreprise pour favoriser cette entreprise. La jurisprudence sanctionne néanmoins lourdement les syndics qui ont recours à l’usage de « devis de couverture »7 afin de faciliter une entreprise.
			Avant de soumettre à l'assemblée générale l’examen technique et financier du marché proposé, le syndic doit recueillir l’accord de l’assemblée générale sur le principe même de ce marché dans trois hypothèses :
			- le marché doit être signé entre le syndicat et lui-même ou entre le syndicat des copropriétaires et un membre de sa famille (directe ou par alliance) jusqu’au troisième degré ou encore entre le syndicat et la personne avec laquelle le syndic a signé un pacte civil de solidarité (PACS).
			- le marché doit être signé entre le syndicat des copropriétaires et une entreprise dans laquelle le syndic, un membre de sa famille telle que définie précédemment (« pacsé » inclus) est soit dirigeant, soit propriétaire de parts ou d’action en nombre suffisant pour lui donner un contrôle de la société, soit salarié ou préposé.
			- le marché doit être signé entre le syndicat et une entreprise qui détient elle-même une participation « directe ou indirecte »dans le capital de la société syndic.
			Pour que cette décision soit éclairée, le syndic doit joindre à la convocation :
			7 Crim. 23 mai 2007, no 01-85.045.
			- le projet de contrat dans son intégralité (et non seulement les conditions essentielles)
			- outre son propre contrat de syndic, en précisant les liens rendant nécessaire ce vote préalable
			La résolution doit être adoptée à la majorité de l’article 24.
			
		\subsubsection{L’information des copropriétaires}
		
			Aux termes de l’article 11 du Décret, le syndic doit joindre à la convocation :
			Art 11 I 3\degres/ du décret
			3\degres Les conditions essentielles du contrat ou, en cas d’appel à la concurrence, des contrats proposés, lorsque l’assemblée est appelée à approuver un contrat, un devis ou un marché, notamment pour la réalisation de travaux ;
			La jurisprudence est relativement exigeante quant à l’étendue des informations notamment quant à la description des travaux et quant à leur durée. (cf. supra, convocation de l’assemblée générale).
			
			De même, selon la cour de Paris, en l’absence de document faisant une analyse des propositions reçues, le syndic doit communiquer avec la convocation à l’assemblée générale faisant état de trois consultations d’entreprises, l’ensemble des devis obtenus et non pas deux sur trois\footnote{CA Paris, pôle 4, 2\degres Ch,, 20 janv. 2016, \no 13/23988 : Loyers et Copropriété, fev. 2016 \no 108.}.
			
			De même, l’assemblée générale ne pourra valablement déléguer au conseil syndical le choix de l’entreprise et la décision du coût définitif des travaux à engager que si ces travaux :
			\begin{itemize}
				\item relèvent de l’article 24 de la loi du 10 juillet 1965 (travaux d’entretien) ;
				\item sont décidés avec une enveloppe globale maximale et un calendrier d’appel des fonds
			\end{itemize}
			
			L’ordonnance du 30 oct. 2019 a ajouté, à ce même article et alinéa, les dispositions suivantes, relatives à l’obligation nouvelle de proposer un emprunt au syndicat des copropriétaires dès lors que des travaux sont proposés à l’assemblée générale\footnote{Art. 27 de l’ordonnance du 30 oct. 2019 modifiant l’article 25-1 de la L. du 10 juill. 1965.} :
			\begin{quote}
				Art 11 I 3\degres/ du décret complété :
				
				« {\itshape 3\degres Les conditions essentielles du contrat ou, en cas d'appel à la concurrence, des contrats proposés, lorsque l'assemblée est appelée à approuver un contrat, un devis ou un marché, notamment pour la réalisation de travaux ainsi que les conditions générales et particulières du projet de contrat et la proposition d'engagement de caution mentionné au deuxième alinéa de l'article 26-7 de la loi du 10 juillet 1965 lorsque le contrat proposé a pour objet la souscription d'un prêt bancaire au nom du syndicat dans les conditions prévues à l'article 26-4 de cette loi ;} »
			\end{quote}
		
		\subsubsection{La délégation donnée par l’assemblée générale pour passer un marché}
		
			La loi \no 65-557 du 10 juillet 1965 prévoit, en son article 25 a, la possibilité de donner au conseil syndical ou au syndic (voire à un tiers), à la majorité des voix de tous les copropriétaires « toute délégation du pouvoir de prendre l'une des décisions visées à l'article 24 » (article 25 a).
			
			Cette faculté de délégation peut être notamment utilisée pour permettre au conseil syndical ou au syndic de faire le choix final de l’entreprise pour la conclusion d’un marché de travaux, à la triple condition que :
			\begin{itemize}
				\item Les travaux votés relèvent de la majorité « simple » des voix exprimées ( article 24).
				
				\item L’objet de la délégation soit précis en son objet, et en son montant (montant maximal).
				
				\item Que la question ait été inscrite à l’ordre du jour, ce qui exclut le vote d’une telle délégation « par défaut » lorsque l’assemblée générale ne parvient pas à se décider. L’assemblée générale ne peut se contenter de prendre une décision de principe sur l’engagement de tel ou tel type de travaux si les informations nécessaires n’ont pas été transmises avec la convocation : les décisions de principe n’ont pas force de résolution, en sorte que, quels que soient les travaux envisagés, une décision de principe de faire des travaux n’engage ni la copropriété ni les copropriétaires.
			\end{itemize}
			
			La Loi \no 2014-366 du 24 mars 2014 dite ALUR avait en outre prévu, dans les petites copropriétés de 15 lots au plus, la faculté pour le syndic de se faire « substituer » (avec l’accord de l’assemblée générale) pour « la mise en application et le suivi des travaux et contrats financés dans le cadre du budget prévisionnel de charges » par le conseil syndical, à condition que celui-ci soit couvert par une assurance de responsabilité civile. Cette délégation était contestable, d’une part parce que le pouvoir ainsi délégué n’appartenait pas à l’assemblée générale mais bien au syndic, lequel ne peut se faire substituer par application de l’article 18 IV. \textbf{Cette faculté a été supprimée par l’ordonnance du 30 octobre 2019}.
			
			En revanche, l’ordonnance a institué un mécanisme de délégation plus général en faveur du Conseil Syndical :
			\begin{quote}
				{\bfseries Art.21 de l’ordonnance du 30 oct. 2019 instituant cinq nouveaux articles :}
				
				« {\textbf{Art. 21-1}.-Sans préjudice des dispositions du a de l'article 25, lorsque le conseil syndical est composé d'au moins trois membres, l'assemblée générale peut, par décision prise à la majorité des voix de tous les copropriétaires, lui déléguer le pouvoir de prendre tout ou partie des décisions relevant de la majorité des voix exprimées des copropriétaires présents, représentés, ou votant par correspondance.
				
				La délégation de pouvoirs ne peut toutefois porter sur l'approbation des comptes, sur la détermination du budget prévisionnel, ou sur les adaptations du règlement de copropriété rendues nécessaires par les modifications législatives et règlementaires intervenues depuis son établissement.
				
				\textbf{Art. 21-2}.-L'assemblée générale fixe le montant maximum des sommes allouées au conseil syndical pour mettre en œuvre sa délégation de pouvoirs.
				
				\textbf{Art. 21-3}.-La délégation de pouvoirs mentionnée à l'article 21-1 est accordée au conseil syndical pour une durée maximale de deux ans. Elle est renouvelable par une décision expresse de l'assemblée générale.
				
				\textbf{Art. 21-4}.-Le syndicat des copropriétaires souscrit, pour chacun des membres du conseil syndical, une assurance de responsabilité civile.
				
				\textbf{Art. 21-5}.-Les décisions du conseil syndical pour l'exercice de la délégation de pouvoirs mentionnée à l'article 21-1 sont prises à la majorité de ses membres. En cas de partage des voix, le président du conseil syndical a voix prépondérante.
				
				Le conseil syndical rend compte de l'exercice de sa délégation de pouvoirs devant l'assemblée générale votant l'approbation des comptes.
				
				Il établit un rapport en vue de l'information des copropriétaires.} »
			\end{quote}
			
			Ce nouveau mécanisme permettra au conseil syndical, s’il est mis en oeuvre par l’assemblée générale, de prendre toute décision de l’article 24 et donc de décider des travaux relevant de cette majorité\footnote{V\degres sur cette nouvelle délégation : Fl. Bayard-Jammes, « La réforme de la prise de décision au sein de la copropriété », AJDI ; A. Lebatteux, « Les dispositions de l’ordonnance \no 2019-1101 du 30 oct. 2019 applicables aux assemblées générales de copropriété », Loyers et copr. \no 1, janv. 2020, dossier 5 ; P.-e. Lagraulet, « La prise de décision au sein de la copropriété après l’ordonnance du 30 oct. 2019 », Lexbase, Hebdo éd. priv., N1578BY4, 11 déc. 2019.}. En particulier, il parait désormais possible de déléguer au conseil syndical la possibilité de décider de tous travaux d’entretien courant ( réparation des fontes cassées, par exemple), dans la limite d’un budget maximal, à condition que le conseil syndical soit composé de 3 membres au moins et assuré, et à charge uniquement de rendre compte de l’exercice de cette délégation à l’assemblée générale annuelle suivante.
		
		\subsubsection{Les honoraires du syndic}
		
			Le plus souvent le syndic perçoit des honoraires pour la gestion du marché. Ceci n'a rien d'anormal dès lors que le syndic engage sa responsabilité civile si cette gestion est défaillante.
			
			L’article 18-1-A à la loi \no 65-557 du 10 juillet 1965 dispose depuis 2009 précise que ces honoraires doivent être décidés par l’assemblée générale par un vote spécial (ils ne peuvent résulter de la simple aplication du contrat de syndic) et ne peuvent être demandés que pour des travaux « hors budget ». Jusqu’à l’ordonnance du 30 octobre 2019, le syndic avait même interdiction de mentionner dans son contrat un montant d’honoraires pré-déterminé pour ces travaux, même à titre indicatif, afin de laisser toute latitude de négociation à l’assemblée générale. Cependant, cette absence de mention dans le contrat des honoraires « travaux » constituait une lacune importante au moment de la mise en concurrence des contrats de syndic.
			
			L’article 18-1 A de la loi est désormais ainsi rédigé
			\begin{quote}
				ARTICLE 18-1 A (DIFFÉRÉ) MODIFIE PAR ORDONNANCE \No2019-1101 DU 30 OCTOBRE 2019 - ART. 16
				
				III.-Les travaux mentionnés à l'article 14-2 et votés par l'assemblée générale des copropriétaires en application des articles 24, 25, 26-3 et 30 peuvent faire l'objet d'honoraires spécifiques au profit du syndic. Ces honoraires sont votés lors de la même assemblée générale que les travaux concernés, aux mêmes règles de majorité.
				
				La rémunération fixée dans le projet de résolution soumis au vote de l'assemblée générale doit être exprimée en pourcentage du montant hors taxes des travaux, à un taux dégressif selon l'importance des travaux préalablement à leur exécution.
			\end{quote}

			Cette disposition entrera en vigueur le 1er juin 2020, à condition que l’arrêté portant le « contrat type » de syndic soit préalablement modifié afin que le syndic puisse, à titre indicatif, mentionner un barème.
			
			De plus, le texte de la loi impose que l’honoraire proposé aux copropriétaires ait un caractère dégressif en fonction de l’importance des travaux. Cet honoraire aura nécessairement pour assiette le montant des travaux votés … qui ne correspond pas toujours au montant des travaux exécutés : en cas d’approbation du supplément de prix le syndic ne pourra prétendre au réajustement de ses honoraires pour travaux.
			
			En contrepartie bien évidemment le syndic veillera à ce qu'il n'y ait pas de travaux réalisés sans ordre de service comme il veillera à la solidité financière de l'entreprise et aura pris soin de contrôler la réalité des assurances dont bénéficie l'entrepreneur.
			
			Une réponse ministérielle\footnote{Rép. Min : JO AN du 20.10.09} du 20 octobre 2009 précise qu’en pratique, il est nécessaire de procéder lors d’une même assemblée générale à deux votes distincts : un premier vote pour les travaux à réaliser et un second vote qui concerne le principe, le mode de calcul et le quantum pour les travaux concernés.
		
		\subsubsection{La signature du marché}
		
			Le syndic ne pourra signer le marché de travaux que si l’assemblée générale a adopté non seulement une \emph{résolution de principe, mais a décidé des modalités de réalisation et du cout des travaux} : pour des travaux, il n’y a de décision que lorsque l’assemblée a voté la réalisation, fixé leur coût et les modalités de la répartition de la dépense\footnote{CA Paris 14 octobre 2009 \no 08/16661 JD 2009-380166}, sauf à avoir donné délégation au syndic de passer commande des travaux à l’entreprise de son choix dans les limites d’un budget déterminé.
			
			Le contrat le marché avec l'entreprise devra toujours être passé au nom du syndicat des copropriétaires et non pas au nom du syndic. Dans cette dernière hypothèse en effet et en cas de défaillance du syndicat, l'entrepreneur poursuivra directement le syndic pour avoir paiement des sommes qui lui restent dues\footnote{Civ 3\degres 23 sep 2009 ; Loyers et Copropriété nov 2009 \no 207 : en l’espèce le syndic avait signé l’OS et apposé son cachet sans référence à la copropriété pour des travaux de déblaiement après incendie de l’immeuble administré par lui.}.
			
			Le syndic devra être particulièrement vigilant dans la passation des marchés. Les copropriétaires attendent de lui la signature d'un marché à forfait.
	
	\subsection{Les travaux de conservation de l’immeuble}
	
		\subsubsection{Les travaux concernés : distinction a faire entre les travaux de maintenance et les travaux de conservation.}

			Il convient de distinguer les « travaux de maintenance » (inclus dans le budget), les « travaux d’entretien » devant faire l’objet d’un vote spécifique en assemblée générale et les « travaux urgents » que le syndic peut commander directement.
			
			\begin{quote}
				Art. 14-1 –Pour faire face aux dépenses courantes de maintenance, de fonctionnement et d'administration des parties communes et équipements communs de l'immeuble, le syndicat des copropriétaires vote, chaque année, un budget prévisionnel. L'assemblée générale des copropriétaires appelée à voter le budget prévisionnel est réunie dans un délai de 6 mois à compter du dernier jour de l'exercice comptable précédent
				
				Art. 14-2. I - Ne sont pas comprises dans le budget prévisionnel des dépenses pour travaux dont la liste est fixée par décret en conseil d'état.
				
				Les sommes afférentes à ses dépenses sont exigibles selon les modalités votées par l'assemblée générale. »
			\end{quote}
			
			Cette distinction n’est pas inutile : il existait avant la loi SRU un non-dit sur les limites d’intervention du syndic pour la réalisation de travaux de maintenance\footnote{G.VIGNERON sous CA Paris, 22 mai 1998 (Loyers et Copropriété, dec. 1998).}.
			
			\paragraph{Travaux de maintenance courante.}
			
				\begin{quote}
					\textbf{L’article 45 du Décret du 17 mars 1967}
					
					\medskip{Les travaux de maintenance sont les travaux d'entretien courant, exécutés en vue de maintenir l'état de l'immeuble ou de prévenir la défaillance d'un élément d'équipement commun ; ils comprennent les menues réparations.
					
					Sont assimilés à des travaux de maintenance les travaux de remplacement d'éléments d'équipement communs, tels que ceux de la chaudière ou de l'ascenseur, lorsque le prix de ce remplacement est compris forfaitairement dans le contrat de maintenance ou d'entretien y afférent.
					
					Sont aussi assimilées à des travaux de maintenance les vérifications périodiques imposées par les réglementations en vigueur sur les éléments d'équipement communs.}
				\end{quote}
				
				Entrent donc dans la catégorie des travaux de maintenance :
				\begin{itemize}
					\item En premier lieu les petits travaux de maintenance ; c’est-à-dire ceux récurrents et d’un faible coût.
					
					\item Ensuite les travaux – quel qu’en soit le coût – qui sont réalisés en exécution d’un contrat d’entretien (contrats d’ascensoristes avec remplacement du matériel ou encore les contrats « P3 » ou « P4 » des chauffagistes).
					
					\item Enfin les vérifications périodiques imposées par les textes, à ne pas confondre avec les diagnostics techniques qui, même imposés par la règlementation, doivent faire l’objet d’un vote spécifique avec le financement nécessaire.
				\end{itemize}
				
				Ces travaux n’ont pas à être votés spécifiquement : ils sont considérés comme entrant dans le budget et à ce titre plusieurs lignes peuvent être consacrées dans le Budget à la prévision de ces petits travaux récurrents.
			
			\paragraph{Travaux d’entretien et de conservation non courants.}
			
				\begin{quote}
					\textbf{Article 44 du Décret \no 67-223 du 17 mars 1967 :}
					
					\medskip{Les dépenses non comprises dans le budget prévisionnel sont celles afférentes :
					
					1\degres Aux travaux de conservation ou d'entretien de l'immeuble, autres que ceux de maintenance ;
					
					2\degres Aux travaux portant sur les éléments d'équipement communs, autres que ceux de maintenance ;
					
					3\degres Aux travaux d'amélioration, tels que la transformation d'un ou de plusieurs éléments d'équipement existants, l'adjonction d'éléments nouveaux, l'aménagement de locaux affectés à l'usage commun ou la création de tels locaux, l'affouillement du sol et la surélévation de bâtiments ;
					
					4\degres Aux études techniques, telles que les diagnostics et consultations ;
					
					5\degres Et, d'une manière générale, aux travaux qui ne concourent pas à la maintenance et à l'administration des parties communes ou à la maintenance et au fonctionnement des équipements communs de l'immeuble.}
				\end{quote}
				
				A la différence des travaux de maintenance de l’article 14-1, ces travaux hors budget doivent être adoptés en assemblée générale. \textbf{Ils se caractérisent par la non récurrence de la dépense} : elle ne peut être programmée d’une année sur l’autre, et non par leur coût plus ou moins élevé.
		
		\subsubsection{Les travaux de conservation de l’immeuble relevant de la majorité de l’article 24 de la loi \no 65-557 du 10 juillet 1965}
		
			Tous les autres travaux de conservation de l’immeuble (ceux qui n’ont pas un caractère récurrent), relèvent de l’article 24 de la loi \no 65-557 du 10 juillet 1965, c’est-à-dire de la majorité des voix exprimées des copropriétaires présents ou représentés.
			
			Depuis la loi \no 2014-366 du 24 mars 2014 dite ALUR, l’article 24 précise les travaux dits de « conservation » de l’immeuble, bien que relèvent de l’article 24 toutes les décisions pour lesquelles il n’est pas prévu de majorité plus forte.
			
			\paragraph{Les travaux de conservation, de préservation de la santé et de la sécurité physique des occupants}
			
				\begin{quote}
					\textbf{Article 24 de la loi \no 65-557 du 10 juillet 1965}
					
					\medskip \II.-Sont notamment approuvés dans les conditions de majorité prévues au \I :
				
					\medskip a) Les travaux nécessaires à la conservation de l'immeuble ainsi qu'à la préservation de la santé et de la sécurité physique des occupants, qui incluent les travaux portant sur la stabilité de l'immeuble, le clos, le couvert ou les réseaux et les travaux permettant d'assurer la mise en conformité des logements avec les normes de salubrité, de sécurité et d'équipement définies par les dispositions prises pour l'application de l'article 1er de la loi \no 67-561 du 12 juillet 1967 relative à l'amélioration de l'habitat ;
				\end{quote}
				
				Le syndicat des copropriétaires ayant pour objet de pourvoir à la conservation et à l'entretien de l’immeuble, il est normal que les travaux nécessaires à cette conservation et cet entretien puissent être réalisés dans des conditions de majorité réduite. Ne pas entretenir l’immeuble ou
				ses éléments d’équipement, c’est engager la responsabilité de la copropriété toute entière (nous verrons ultérieurement les dispositions de l’article 14 de la loi aux termes duquel le syndicat est responsable des dommages causés aux copropriétaires ou aux tiers ayant leur origine dans les parties communes, sans préjudice de toutes actions récursoires).
				
				Mais cette disposition vise également certains travaux assimilés à la conservation de l’immeuble :
				
				a) Les travaux destinés à préserver la santé des occupants et leur santé physique qui vont porter, sans que cette liste soit exhaustive :
				- Sur la stabilité de l’immeuble
				- Sur le clos et le couvert de l’immeuble
				- Sur le couvert
				- Sur les réseaux
				Ainsi, il est possible de décider à la majorité de l’article 24 des travaux qui ajoute des matériaux ou des éléments n’existant pas présentement (ce qui caractérise normalement les travaux d’amélioration)en justifiant, notamment par une étude d’un architecte ou d’un BET que les travaux envisagés sont rendus nécessaires pour préserver la santé et la sécurité physique des occupants.
				
				b) Le même a) assimile aux travaux de conservation les travaux de l’article 1er de la loi du 12 juillet 1967, c’est à dire les « travaux destinés à adapter, totalement ou partiellement, les locaux d'habitation à des normes de salubrité, de sécurité, d'équipement et de confort ».
				Il s’agit également de travaux d’amélioration de l’immeuble en ce que ces travaux vont ajouter des éléments de confort qui n’existaient pas précédemment … mais de confort minimum puisqu’ils sont considérés comme de mise en conformité de l’immeuble aux normes d’habitabilité.
				La liste de ces travaux est donnée par le Décret \no 2002-120 du 30 janvier 2002 relatif aux caractéristiques des logements décents. Ces travaux concerneront essentiellement :
				- L’alimentation de l’immeuble en eau potable,
				- Le chauffage de la loge de la concierge
				- L’évacuation des eaux ménagères et eaux-vannes
				- Les réseaux d’électricité pour assurer l’éclairage des parties communes
			
			\paragraph{Les modalités de réalisation des travaux obligatoires}
			
				\begin{quote}
					\textbf{Article 24 b)} Les modalités de réalisation et d'exécution des travaux rendus obligatoires en vertu de dispositions législatives ou réglementaires ou d'un arrêté de police administrative relatif à la sécurité ou à la salubrité publique, notifié au syndicat des copropriétaires pris en la personne du syndic ;
				\end{quote}
				
				Constituent des travaux obligatoires les travaux qui sont imposés à la copropriété par des lois, décrets, arrêtés, comme par exemple :
				\begin{itemize}
					\item travaux de ravalement prescrits par l'autorité administrative,
					\item travaux de mise en conformité des ascenseurs, portes de garages, mesures de sécurité incendie dans les parkings, pose de compteurs d'eau chaude et pose de compteurs de chaleur.
				\end{itemize}
				
				L’assemblée générale ne décide donc pas de réaliser les travaux (la décision ne dépend plus d’elle), elle décide des modalités de réalisation de ces travaux.
				
				Si la décision est adoptée, mais ne peut être exécutée du fait de la résistance d’un copropriétaire, l’astreinte sera mise à la charge exclusive de ce dernier
				
				\begin{quote}
					\textbf{Article 10-1 (différé) Modifié par Ordonnance \no2019-1101 du 30 octobre 2019 - art. 10}
					
					Par dérogation aux dispositions du deuxième alinéa de l'article 10, sont imputables au seul copropriétaire concerné :
					
					d) Les astreintes, fixées par lot, relatives à des mesures ou travaux prescrits par l'autorité administrative compétente ayant fait l'objet d'un vote en assemblée générale et qui n'ont pu être réalisés en raison de la défaillance du copropriétaire.
				\end{quote}
				
				\begin{quote}
					Article 24-8 Modifié par LOI \no2018-1021 du 23 novembre 2018 - art. 194 (\V)
					
					Lorsque, en application des articles L. 1331-29-1 et L. 1334-2 du code de la santé publique ou des articles L. 129-2 ou L. 511-2 du code de la construction et de l'habitation, une astreinte applicable à chaque lot a été notifiée au syndicat des copropriétaires, pris en la personne du syndic, par une autorité publique, le syndic en informe immédiatement les copropriétaires.
					
					Lorsque l'inexécution des travaux et mesures prescrits par l'arrêté de police administrative résulte de la défaillance de certains copropriétaires, le syndic en informe l'autorité publique compétente, en lui indiquant les démarches entreprises et en lui fournissant une attestation de défaillance. Sont réputés défaillants les copropriétaires qui, après avoir été mis en demeure par le syndic, n'ont pas répondu aux appels de fonds nécessaires à la réalisation des travaux dans le délai de quinze jours après la sommation de payer. Au vu de l'attestation de défaillance, l'autorité publique notifie le montant de l'astreinte aux copropriétaires défaillants et procède à sa liquidation et à son recouvrement comme il est prévu aux mêmes articles L. 1331-29-1, L. 1334-2, L. 129-2 et L. 511-2.
					
					Lorsque l'assemblée générale des copropriétaires n'a pas été en mesure de voter les modalités de réalisation des travaux prescrits par un des arrêtés de police administrative mentionnés aux mêmes articles et que le syndicat des copropriétaires est lui-même défaillant, chacun des copropriétaires est redevable du montant de l'astreinte correspondant à son lot de copropriété notifié par l'autorité publique compétente.
				\end{quote}
			
			\paragraph{Les travaux réalisés dans le cadre des Opérations de Restauration Immobilière (ORI)}
			
				\begin{quote}
					24 c) Les modalités de réalisation et d'exécution des travaux notifiés en vertu de l'article L. 313-4-2 du code de l'urbanisme, notamment la faculté pour le syndicat des copropriétaires d'assurer la maîtrise d'ouvrage des travaux notifiés portant sur les parties privatives de tout ou partie des copropriétaires et qui sont alors réalisés aux frais du copropriétaire du lot concerné ;
				\end{quote}
			
				Il s’agit ici des travaux devant être réalisés dans le cadre d’opérations de restauration immobilière lesquelles consistent en des travaux de remise en état, de modernisation ou de démolition ayant pour objet ou pour effet la transformation des conditions d'habitabilité d'un immeuble ou d'un ensemble d'immeubles ; ces travaux peuvent être entrepris dans le cadre d’un Plan de Sauvegarde\footnote{Cf. Le dernier chapitre du cours sur les copropriétés en difficulté et le Plan de Sauvegarde.}. Ils sont obligatoires, la Collectivité publique pouvant procéder à l’expropriation de l’immeuble faute de mise en oeuvre des travaux définis.
				
				\begin{quote}
					Aux termes de l’article L. 313-4-2 du code de l’urbanisme :
					
					Après le prononcé de la déclaration d'utilité publique, la personne qui en a pris l'initiative arrête, pour chaque immeuble à restaurer, le programme des travaux à réaliser dans un délai qu'elle fixe.
					
					Lors de l'enquête parcellaire, elle notifie à chaque propriétaire le programme des travaux qui lui incombent. Si un propriétaire fait connaître son intention de réaliser les travaux dont le détail lui a été notifié, ou d'en confier la réalisation à l'organisme chargé de la restauration, son immeuble n'est pas compris dans l'arrêté de cessibilité ».
				\end{quote}
				
				Ces travaux peuvent être réalisés sous maîtrise d’ouvrage du syndicat des copropriétaires ou d’une AFUL auquel il adhère à l’article 24 :
				\begin{quote}
					Article 24-7 de la loi Créé par LOI \no2014-366 du 24 mars 2014 - art. 60
					
					Sauf dans le cas où le syndicat des copropriétaires assure la maîtrise d'ouvrage des travaux portant sur les parties communes et les parties privatives de l'immeuble en application du c du II de l'article 24, le syndicat des copropriétaires peut délibérer sur la création ou l'adhésion à une association foncière urbaine prévue au 5\degres de l'article L. 322-2 du code de l'urbanisme. Dans ce cas, par dérogation à l'article 14, l'association foncière urbaine exerce les pouvoirs du syndicat des copropriétaires portant sur les travaux de restauration immobilière relatifs aux parties communes de l'immeuble jusqu'à leur réception définitive.
				\end{quote}
		
		\subsubsection{Le vote des travaux urgents}
		
			Par définition, ces travaux urgents ne peuvent concerner que les travaux de conservation de l’immeuble.
			
			La commande de travaux urgents par le syndic, avant qu’il ait pu convoquer l’assemblée générale est encadré par l’article 37 du décret
			\begin{quote}
				{\bfseries Article 37 Modifié par Décret \no2010-391 du 20 avril 2010 - art. 20}

				Lorsqu’en cas d’urgence le syndic fait procéder, de sa propre initiative, à l’exécution de travaux nécessaires à la sauvegarde de l’immeuble, il en informe les copropriétaires et convoque immédiatement une assemblée générale.
				
				Par dérogation aux dispositions de l’article 35 ci-dessus, il peut, dans ce cas, en vue de l’ouverture du chantier et de son premier approvisionnement, demander, sans délibération préalable de l’assemblée générale mais après avoir pris l’avis du conseil syndical, s’il en existe un, le versement d’une provision qui ne peut excéder le tiers du montant du devis estimatif des travaux.
				
				Il ne peut demander de nouvelles provisions pour le paiement des travaux qu’en vertu d’une décision de l’assemblée générale qu’il doit convoquer immédiatement et selon les modalités prévues par le deuxième alinéa de l’article 14-2 de la loi du 10 juillet 1965.
			\end{quote}
			
			Le Syndicat des Copropriétaires ne peut donc engager, sans assemblée générale, de travaux urgents que si ceux-ci sont « nécessaires à la sauvegarde de l’immeuble » et il doit alors convoquer « immédiatement » une assemblée générale –laquelle bénéficiera d’un délai de convocation réduit par l’urgence-
			
			Le décret ouvre au syndic la possibilité de faire un premier appel de fonds immédiat pour l’approvisionnement du chantier, mais après avis du conseil syndical.
		
		\subsubsection{Modalités du vote et répartition du coût des travaux (rappel)}
		
			\paragraph{Calcul des voix et répartition des charges (rappel)}
			
				Si les travaux concernent des parties communes spéciales, seuls les copropriétaires détennant des droits indivis sur ces parties communes spéciales prennent part au vote et supportent le coût des travaux, à proportion de leurs tantièmes dans ces parties communes spéciales. Le vote peut avoir lieu lors d’une assemblée générale spéciale réunissant uniquement ces copropriétaires, comme le précise l’article 6-2 de la loi \no 65-557 du 10 juillet 1965 depuis la loi ELAN
				
				\begin{quote}
					{\bfseries Article 6-2 Créé par LOI \no2018-1021 du 23 novembre 2018 - art. 209 (V)}
					
					Les parties communes spéciales sont celles affectées à l'usage et ou à l'utilité de plusieurs copropriétaires. Elles sont la propriété indivise de ces derniers.
					La création de parties communes spéciales est indissociable de l'établissement de charges spéciales à chacune d'entre elles.
					
					Les décisions afférentes aux seules parties communes spéciales peuvent être prises soit au cours d'une assemblée spéciale, soit au cours de l'assemblée générale de tous les copropriétaires. Seuls prennent part au vote les copropriétaires à l'usage ou à l'utilité desquels sont affectées ces parties communes.
				\end{quote}
				
				Si les travaux concernent des éléments d’équipement commun dotés d’une clef de répartition spécifique, il résulte de l’article 10 dernier alinéa, dans sa rédaction issue de l’ordonnance du 30 octobre 2019 que
				\begin{quote}
					{\bfseries Article 10 dernier alinéa- O 30 octobre 2019} (différé au 1er juin 2020)
					
					Lorsque le règlement de copropriété met à la seule charge de certains copropriétaires les dépenses d'entretien et de fonctionnement entraînées par certains services collectifs ou éléments d'équipements, il peut prévoir que ces copropriétaires prennent seuls part au vote sur les décisions qui concernent ces dépenses. Chacun d'eux dispose d'un nombre de voix proportionnel à sa participation auxdites dépenses.
				\end{quote}
				
				Cette disposition signifie que lorsque le règlement de copropriété prévoit une clé de répartition spécifique, notamment pour les coûts afférents un élément d’équipement (la répartition de ces coups doit être faite en fonction du critère de l’utilité), les travaux seront répartis en fonction de la clé de charges spécifiques ainsi prévue.
				
				{\bfseries Toutefois, le vote restera un vote de tous les copropriétaires, à moins que le règlement de copropriété n’ait prévu une clause de « spécialisation des votes »}, auquel cas chaque copropriétaire vote avec un nombre de voix proportionnelles à la quote-part de charges qu’il supportera.
				
				Il existait avant l’ordonnance une disposition semblable, mais elle figurait à l’article 24 dernier alinéa dans les termes suivants :
				\begin{quote}
					{\bfseries 24 dernier alinéa (abrogé)}
					
					\III.-Lorsque le règlement de copropriété met à la charge de certains copropriétaires seulement les dépenses d'entretien d'une partie de l'immeuble ou celles d'entretien et de fonctionnement d'un élément d'équipement, il peut être prévu par ledit règlement que ces copropriétaires seuls prennent part au vote sur les décisions qui concernent ces dépenses. Chacun d'eux vote avec un nombre de voix proportionnel à sa participation auxdites dépenses.
				\end{quote}
				
				L’article 24 dernier alinéa était donc un peu plus large, puisqu’il envisageait la possibilité d’une spécialisation des charges pour l’entretien d’une partie de l’immeuble, ce qui me paraît plus possible aujourd’hui : en d’autres termes, les voies doivent se faire soient en fonction des tantièmes de parties communes spéciales, lorsqu’elles existent, soit en fonction des tantièmes de parties communes générales, soit encore à proportion de la participation aux dépenses prévue par le règlement de copropriété, mais uniquement pour les éléments d’équipements.
			
			\paragraph{Questions à poser à l’assemblée générale}
			
				Pour le vote des travaux, l’assemblée générale doit être saisie des questions suivantes :
				\begin{itemize}
					\item Décision de réaliser les travaux
					\item Choix de l’entreprise, après mise en concurrence, et définition du montant global des travaux (ou, éventuellement, délégation à donner au conseil syndical pour faire le choix de l’entreprise dans la limite d’un budget déterminé au titre de l’article 25 b de la loi \no 65-557 du 10 juillet 1965)
					\item Honoraires du syndic pour le suivi administratif et comptable
					\item Éventuel contrats « annexes » à souscrire du fait des travaux, tels que maîtrise d’œuvre, dommage-ouvrage, bureaux de contrôle, bureaux d’études \etc (Un vote séparé par contrat, avec mise en concurrence)
					\item Souscription éventuelle d’un emprunt collectif
					\item Date d’exigibilité des provisions pour travaux : faute de vote de date d’exigibilité des provisions pour travaux, les sommes ne peuvent pas être appelées auprès des copropriétaires
				\end{itemize}
		
	\subsection{Les travaux d’amélioration de l’immeuble}
	
		Par définition, les travaux d'amélioration de l'immeuble s'opposent aux travaux d'entretien et de conservation des parties communes qui doivent être considérés comme seulement destinés à assurer la sauvegarde de l'immeuble. Au contraire, les travaux d'amélioration ne sont pas indispensables à cette sauvegarde.
		
		Ces travaux ont donc un caractère superflu (au sens du dictionnaire ROBERT: « qui est en plus de ce qui est nécessaire »), alors que ces travaux peuvent affecter les possibilités ultérieures de financer les dépenses nécessaires. Imprévisibles dans leur ampleur, ils n’ont pu être anticipés par les acquéreurs et peuvent les contraindre à vendre leut lot, faut de pouvoir financer les travaux.
		
		La loi \no 65-557 du 10 juillet 1965 était donc plutôt hostile à de tels travaux : des garanties de forme et de fond figurent à l’article 30 de la loi du 10 juillet 1965, la principale d’entre elles était la double majorité des deux tiers des voix et la moitié en nombre de tous les copropriétaires.
		
		Mais depuis 1965, le législateur a progessivement créé de nouvelles catégories de travaux « d’amélioration » qui sont présumés conformes à la destination de l’immeuble, et qui, de ce fait, bénéficient d’un régime « de faveur » : la majorité requise, en particulier, a été abaissée à la majorité de l’article 25, voire 24.
		
		La théorie économique sur l’obsolescence des immeubles (et notamment l’obsolescence énergétique) a même ammené à un complet renversement de perspective : faute de réalisation de travaux d’amélioration de façon continue, l’immeuble risque de se trouver « déqualifié » par rapport au marché immobilier, et de n’être plus attractif que pour des acquéreurs de plus en plus pauvres, rendant la réalisation des travaux finalement impossible.
		
		C’est pourquoi, depuis l’ordonnance du 30 octobre 2019
		\begin{itemize}
			\item tous les travaux d’amélioration relèvent de l’article 25, avec possibilité d’un vote en seconde lecture à la majorité relative des voix exprimées des copropriétaires présents ou représentés, si le projet recueuille le tiers des voix de tous les copropriétaires au premier tour ( la loi LOI \no 2014-366 du 24 mars 2014 dite ALUR avait abaissé ces travaux de l’article 26 à l’article 25, mais excluait la majorité de repéchâge)
		
			\item l’objet du syndicat des copropriétaires a été étendu aux travaux d’amélioration
			\begin{quote}
				{\bfseries Article 14 (différé) Modifié par Ordonnance \no2019-1101 du 30 octobre 2019 - art. 11}
				
				La collectivité des copropriétaires est constituée en un syndicat qui a la personnalité civile. \lips Il a pour objet la conservation \textbf{et l'amélioration de l'immeuble} ainsi que l'administration des parties communes.
				
				Le syndicat est responsable des dommages causés aux copropriétaires ou aux tiers ayant leur origine dans les parties communes, sans préjudice de toutes actions récursoires.
			\end{quote}
		\end{itemize}
		
		Face à cet ambitieux programme, il faudra trouver de nouveaux modes de protection des copropriétaires minoritaires, et surtout, trouver les financements nécessaires pour éviter de mettre copropriétés et copropriétaires en difficulté
		
		\subsubsection{Condition de fond requises pour le vote des travaux d’amélioration article 30 de la loi du 10 juillet 1965}
		
			\begin{quote}
				ARTICLE 30 DE LA LOI ALINEA 1:
				
				« L’Assemblée générale des copropriétaires, statuant à la majorité prévue à l'article 25, peut, à condition qu'elles soient conformes à la destination l'immeuble, décider toute amélioration, telle que la transformation d’un ou plusieurs éléments d’équipement existants, l’adjonction d’éléments nouveaux, l’aménagement de locaux affectés à l’usage commun, ou la création de tels locaux $\dots$ >>
			\end{quote}
			
			Les travaux d’amélioration, tels que définis dans l’article 30, font l’objet de garanties au profit des copropriétaires minoritaires qui peuvent être dans l’incapacité de les financer
			
			\paragraph{Ces travaux doivent être conformes à la destination de l’immeuble}
			
				Il faudra faire appel aux notions générales déjà développées et que nous regrouperons ici sous trois thèmes essentiels :
				\begin{itemize}
					\item Le standing de l’immeuble, élément évolutif et objectif.
						\begin{itemize}
							\item La qualité des matériaux dans l'immeuble,
							\item La qualité des aménagements des locaux,
							\item Le degré de confort de l'immeuble,
							\item Le luxe plus ou moins grand des équipements,
							\item Le nombre et l'importance des espaces verts,
							\item La situation de l'immeuble dans le quartier,
							\item La situation écologique de l'immeuble,
							\item La catégorie sociale des occupants.
						\end{itemize}
					\item L'usage de l'immeuble, critère évolutif et objectif.
						\begin{itemize}
							\item L'affectation des lots
							\item Usages ou activités autorisés dans l’immeuble
							\item La sécurité des biens et des personnes.
						\end{itemize}
					\item Les clauses des actes, élément évolutif et objectif,
						\begin{itemize}
							\item Clauses du règlement de copropriété,
							\item Décisions antérieures d'assemblée
						\end{itemize}
				\end{itemize}
				
				\textbf{Exemples jurisprudentiels}
				
				La conformité ou la non-conformité avec la destination de l’immeuble est une question de fait tranchée par les seul juge du fond, au regard des standards ci-dessus rappelés.
				
				Seront normalement qualifiés de travaux d'amélioration conformes à la destination de l'immeuble ceux qui:
				\begin{itemize}
					\item améliorent objectivement la valeur du patrimoine commun
					\item répondent à l'évolution des techniques
					\item correspondent au confort moderne.
				\end{itemize}
			
				La jurisprudence s’est cristallisée autour de 3 catégorie de travaux « d’amélioration »\footnote{P. Lebatteux, « les travaux d’amélioration et destination de l’immeuble », RDI 1995, p. 429 et s.} :
				\begin{enumerate}
					\item Création d’un parking sur espaces verts.
					
					COUR D'APPEL DE LIMOGES EN DATE 1ER JUILLET 1966 : la destination de l’immeuble ne justifie pas dans le cas précis la préservation dans sa totalité de l'espace vert commun et ceci au motif que la vie moderne pose la nécessité de réaliser des parkings.
					
					TRIBUNAL DE GRANDE INSTANCE DE PARIS EN DATE DU 28 FEVRIER 1969 qui, à propos de la vente d'un terrain en vue de réaliser un parking public à la fois en surface et souterrain, refuse de prononcer la nullité de cette vente qui ne portait pas atteinte à la destination de l’immeuble.. Le tribunal retient notamment que << ces bâtiments ne sont pas des immeuble de luxe et ne bénéficient pas d'un environnement immédiat exceptionnel de calme et de verdure, mais se trouvent à proximité immédiate du métro et du terminus de plusieurs lignes d'autobus dans un cadre urbain d'une certaine densité, à la lisière d'une zone populeuse. >>
					
					\item L'installation d’un ascenseur dans un immeuble ancien
					
					L’installation d'un ascenseur dans un immeuble, aux frais du syndicat des copropriétaires, constitue une amélioration qui, sauf cas exceptionnels, doit être considérée comme parfaitement justifiée par la destination de l'immeuble et ce quand bien même de tels travaux emporteront certaines restrictions définitives aux modalités de jouissance des parties communes (réduction de ll’emmarchement de l'escalier par exemple) ou même à l’esthétique de la cage d'escalier\footnote{Paris 30 janvier 1979 administrer janvier 1979 page 36.}.
					
					Une décision de la cour de cassation\footnote{Paris 1re chambre 17 mars 1992 aux loyers copropriété 1992 \no 273.} considère même que :
					\begin{quote}
						« la suppression de l’escalier de service nécessitée par cette installation ne constitue pas l'aliénation de parties communes de l'immeuble dont la conservation est nécessaire au respect de la destination de celui-ci, mais consiste en une adjonction d’éléments d'équipement normaux et d'aménagement des locaux affectés à l'usage communs, conforme à la destination de l'immeuble ».
					\end{quote}
					
					Toutefois, il convient de vérifier si la création de l’ascenseur est compatible avec les normes de sécurité pour la circulation dans les couloirs et la cage d’escalier (CA AIX 30 MARS 199 D 99, SOMM P 317)
					
					\item La suppression du chauffage collectif
					
					Depuis un arrêt du 13 décembre 1983\footnote{Répertoire Defrénois 1984 article 33.379 note Souleau.} la Cour de cassation admet que soit considérée comme licite une telle suppression dès lors que certaines conditions matérielles ont été retenues par le juge du fond.
					
					On notera essentiellement un arrêt de la 3E CHAMBRE CIVILE DE LA COUR DE CASSATION EN DATE DU 25 JANVIER 1994 (Cass. 3e civ., 25 janv. 1994 : Inf. rap. copr. avr. 1994, p. 9)
		
					\begin{quote}
						<< Attendu que l'arrêt constate que la station collective de chauffage vapeur sous pression, très difficilement réglable et très ancienne, était à la fois archaïque, vétuste, inefficace, ne permettant plus de distribuer l'eau chaude, ainsi que le prévoit le règlement copropriété, et qu'elle est à refaire dans sa totalité; par ces motifs desquels il résulte que la suppression de chauffage individuel n'affectait pas les conditions de jouissance des parties privatives mais constituait, en conformité avec la destination de l'immeuble, une amélioration, ainsi qu'elle l’a retenu, la Cour d'appel a légalement justifié sa décision. >>
					\end{quote}
					
					\item La suppression de l’eau chaude collective
					Même jurisprudence pour l’abandon d’un service collectif de distribution d’eau chaude au profit de ballons électriques individuels voté à la majorité de l’article 26 dès lors que cette installation collective était vétuste et son fonctionnement aléatoire. Cette suppression permettant une économie d’énergie et de coût ainsi qu’une amélioration du confort des copropriétaires\footnote{Civ. 3ème, 9 mai 2012, \no 11-16226 : Loyers et Copropriété sep. 2012 Etude \no 12, Elodie Levacher.}.
				\end{enumerate}
		
			\paragraph{Absence d’atteinte aux droits des copropriétaires.}
			
			Les travaux décidés par l’assemblée générale ne doivent pas porter atteinte à la jouissance des parties privatives ou à la jouissance des parties communes à usage exclusif.
			
			\begin{quote}
				Art. 26 Loi du 10 juillet 1965
				
				L'assemblée générale ne peut, à quelque majorité que ce soit, imposer à un copropriétaire une modification à la destination de ses parties privatives ou aux modalités de leur jouissance, telles qu'elles résultent du règlement de copropriété
			\end{quote}
			
			\begin{quote}
				Article 9 de la loi \no 65-557 du 10 juillet 1965
				
				Chaque copropriétaire dispose des parties privatives comprises dans son lot ; il use et jouit librement des parties privatives et des parties communes sous la condition de ne porter atteinte ni aux droits des autres copropriétaires ni à la destination de l'immeuble.
			\end{quote}
		
		\subsubsection{Travaux du syndicat des copropriétaires impactant les parties privatives}
		
			\paragraph{Conditions d’intervention de la partie privative du lot}
			
				Les dispositions de l’article 9 de la loi permettent dans certaines limites cette atteinte aux parties privatives.
				
				\begin{quote}
					Article 9 de la Loi du 10 juillet 1965
				
					« Toutefois, si les circonstances l'exigent et à condition que l'affectation, la consistance ou la jouissance des parties privatives comprises dans son lot n'en soient pas altérées de manière durable, aucun des copropriétaires ou de leurs ayants droit ne peut faire obstacle à l'exécution, même à l'intérieur de ses parties privatives, des travaux régulièrement et expressément décidés par l'assemblée générale en vertu des e, g, h, et i de l'article 25 et des articles 26-1 et 30. »
				\end{quote}
				
				La Cour de cassation avait ainsi précisé, au visa de cet article, que cette atteinte n’était tolérable que si :
				\begin{itemize}
					\item \textbf{elle est temporaire ou limitée} : on peut concevoir une privation de jouissance totale mais limitée dans le temps, éventuellement une perte de valeur définitive (perte d’ensoleillement) mais limitée. Le Syndicat des Copropriétaires ne peut imposer à un copropriétaire, au titre de l’article 9, un empiètement définitif sur son lot, par exemple la mise en place d’une chaudière dans les parties privatives CA Paris 23 ch A 5 juillet 1995)
					\item \textbf{les circonstances l’exigent}, ce qui suppose qu’il n’existe pas une autre solution technique non contraignante (civ. 3ème 12 mars 1997) . Cependant, le nouveau texte ouvre notamment la possibilité d’imposer de tel
				\end{itemize}
				
				Cet article a été remanié par l’ordonnance du 30 octobre 2019 qui l’a scindé en trois parties, dont une consacrée à cette exception au principe :
				\begin{quote}
					Article 9 (différé) Modifié par Ordonnance \no2019-1101 du 30 octobre 2019 - art. 8
					
					« \II.-Un copropriétaire ne peut faire obstacle à l'exécution, même sur ses parties privatives, \textbf{de travaux d'intérêt collectif régulièrement décidés par l'assemblée générale des copropriétaires}, dès lors que l'affectation, la consistance ou la jouissance des parties privatives n'en sont pas altérées de manière durable. La réalisation de tels travaux sur une partie privative, \textbf{lorsqu'il existe une autre solution n'affectant pas cette partie, ne peut être imposée au copropriétaire concerné que si les circonstances le justifient}.
					
					Pour la réalisation de travaux d'intérêt collectif sur des parties privatives, le syndicat exerce les pouvoirs du maître d'ouvrage jusqu'à la réception des travaux. »
				\end{quote}
				
				Outre la précision relative à l’exercice des pouvoirs du maître d’ouvrage par le syndicat, le législateur a ainsi consacré les solutions jurisprudentielles relatives au limite de cette exception. Cependant, l’exception ainsi posée semble un peu plus large dans le cadre de la nouvelle rédaction :
				\begin{itemize}
					\item cette exception concerne maintenant tous les travaux « d’intérêt collectif », et non plus seulement certains travaux limitativement visés. Il peut s’agir de travaux d’entretien\footnote{le texte ancien ne visait que certains travaux$\dots$ oubliant les travaux d’entretien des parties communes, qui nécessitent pourtant fréquemment des interventions dans les lots (la jurisprudence admet toutefois que de telles interventions sont possibles CA Paris 23 ch B 20 avril 2000 Loyer et copr. 2000 comm 214).}, d’amélioration, voire de travaux qui seraient sollicités par un copropriétaire si l’intérêt collectif est caractérisé ;
					
					\item en revanche, il faut que ces travaux aient été « régulièrement décidés en assemblée générale » , ce qui exclut de travaux d’urgence avant le vote de l’assemblée générale ou encore des travaux décidés par le conseil syndical par application de l’article 21 nouveau de la loi ;
					
					\item les travaux peuvent être imposés aux copropriétaires alors même qu’il existerait une solution alternative « si les circonstances le justifient » c’est-à-dire notamment si le coût la solution alternative est excessif par rapport au inconvénient causé aux copropriétaires
				\end{itemize}
			
			\paragraph{Modalités d’intervention}
			
			Dans une telle hypothèse, le syndic doit respecter scrupuleusement la procédure prévue à l’article 9 al 1 :
			\begin{quote}
				Article 9 alinéa 1 dans sa rédaction issue de l’Ordonnance du 30 octobre 2019
				
				Les travaux \textbf{supposant} un accès aux parties privatives doivent être notifiés aux copropriétaires \textbf{concernés} au moins huit jours avant le début de leur réalisation, sauf impératif de sécurité ou de conservation des biens.
			\end{quote}
			
			Il faut en outre prévoir en outre une indemnisation du copropriétaire lésé :
			\begin{quote}
				{\bfseries Article 9 de la Loi du 10 juillet 1965 modifié par l’Ordonnance du 30 octobre 2019}
				
				\III.-Les copropriétaires qui subissent un préjudice par suite de l'exécution des travaux, en raison soit d'une diminution définitive de la valeur de leur lot, soit d'un trouble de jouissance grave, même s'il est temporaire, soit de dégradations, ont droit à une indemnité.
				
				{\bfseries En cas de privation totale temporaire de jouissance du lot, l'assemblée générale accorde au copropriétaire qui en fait la demande une indemnité provisionnelle à valoir sur le montant de l'indemnité définitive.}
				
				L'indemnité provisionnelle ou définitive due à la suite de la réalisation de travaux d'intérêt collectif est à la charge du syndicat des copropriétaires. Elle est répartie en proportion de la participation de chacun des copropriétaires au coût des travaux.
			\end{quote}
			
			Selon la jurisprudence, le trouble subi est indemnisable, mais à la condition qu’il soit :
			\begin{itemize}
				\item suffisamment grave s’il est temporaire (civ.3ème 22 juin 1994), ou définitif ;
				\item ne soit pas subi de façon égale par tous les copropriétaires.
			\end{itemize}
			
			L’indemnité est à la charge de tous les copropriétaires, à proportion de leur participation aux travaux.
			
			On notera enfin que le texte issu de l’ordonnance de 2019 a précisé la faculté pour un copropriétaire, en cas de privation totale temporaire de jouissance de demander une indemnité provisionnelle à valoir sur le montant de l’indemnité définitive.
		
		\subsubsection{Le vote de la répartition des coûts et charges}
			
			\paragraph{Répartition des dépenses de réalisation}
			
				\begin{quote}
					ARTICLE 30 ALINEA 2 de la Loi. Elle (l’assemblée générale) fixe alors, à la même majorité, la répartition du coût des travaux et de la charge des indemnités prévues à l'article 36 ci-après, en proportion des avantages qui résulteront des travaux envisagés pour chacun des copropriétaires, sauf à tenir compte de l'accord de certains d'entre eux pour supporter une part de dépenses plus élevée.
				\end{quote}
				
				Il faut noter cette double dérogation aux règles usuelles applicables en matière de répartition des charges :
				\begin{itemize}
					\item possibilité de prendre en compte l’accord d’un copropriétaire pour participer au-delà de sa part théorique ;
					\item calcul de la grille en fonction de l’\emph{avantage} –-- et non de l’utilité.
				\end{itemize}
				
				L’avantage dépend de la plus-value en capital (loyer ou vente) dont bénéficiera le lot après création de l’élément d’équipement. Ainsi la création d’un ascenseur constitue une plus-value immédiate importante (donc un \emph{avantage}) pour les lots à partir du 3ème étage, qui participeront de façon significative à l’installation alors qu’actuellement, les grilles de répartition des charges –-- établies selon le critère de l’utilité --- sont très « écrasées » au motif que le copropriétaire du 1\ier{} étage utilise régulièrement l’ascenseur quand celui-ci existe.
				
				La charge des indemnités de l’article 36 ne concerne que l’hypothèse de travaux de surélévation (cf. infra)
			
			\paragraph{La répartition des dépenses d’entretien future}
			
				\begin{quote}
					ARTICLE 30 ALINÉA 3. Elle fixe, à la même majorité, la répartition des dépenses de fonctionnement, d'entretien et de remplacement des parties communes ou des éléments transformés ou créés.
				\end{quote}
				
				S’agissant d’un nouvel élément d’équipement, le Syndicat des Copropriétaires devra fixer une grille de répartition des charges d’entretien, en fonction du critère de l’utilité (art. 10 de la L. du 10 juill. 1965) et impérativement publier cette nouvelle grille comme modificatif au règlement de copropriété.
			
			\paragraph{La participation différée (article 33)}
			
				Afin d’amortir les effets financiers du vote des travaux d’amélioration, le législateur a prévu la possibilité, pour les copropriétaires « non adhérents » de demander l’échelonnement du règlement de leur quote-part sur 10 ans.
				
				\begin{quote}
					Article 33 de la loi Modifié par LOI \no2012-387 du 22 mars 2012 - art. 103 (V)
					
					La part du coût des travaux, des charges financières y afférentes, et des indemnités incombant aux copropriétaires qui n'ont pas donné leur accord à la décision prise peut n'être payée que par annuités égales au dixième de cette part. Les copropriétaires qui entendent bénéficier de cette possibilité doivent, à peine de forclusion, notifier leur décision au syndic dans le délai de deux mois suivant la notification du procès-verbal d'assemblée générale. Lorsque le syndicat n'a pas contracté d'emprunt en vue de la réalisation des travaux, les charges financières dues par les copropriétaires payant par annuités sont égales au taux légal d'intérêt en matière civile.
					
					Toutefois, les sommes visées au précédent alinéa deviennent immédiatement exigibles lors de la première mutation entre vifs du lot de l'intéressé, même si cette mutation est réalisée par voie d'apport en société.
					
					Les dispositions qui précèdent ne sont pas applicables lorsqu'il s'agit de travaux imposés par le respect d'obligations légales ou réglementaires.
				\end{quote}
				
				Ces dispositions bénéficient aux copropriétaires « qui n’ont pas donné leur accord à la décision », c’est-à-dire les copropriétaires opposants, mais également aux abstentionnistes, absents, ou « n’ayant pas pris part au vote » .
				
				Un délai de deux mois seulement est laissé au copropriétaire pour faire part de sa décision, afin de ne pas bloquer la réalisation des travaux
				
				Il faut noter qu’en cas de vente du lot, la quote-part des travaux devient immédiatement exigible, ce qui permet au syndic de faire jouer le privilège du Syndicat des Copropriétaires sur le montant total des travaux. Mais il est conseillé d’inscrire en outre l’hypothèque légale sur le lot pour sûreté de la créance du Syndicat des Copropriétaires.
				
				Les dispositions ne sont pas applicables lorsqu'il s'agit de travaux imposés par le respect d'obligations légales ou réglementaires (mises en conformité, mise aux normes, réalisation de travaux d’ORI, \etc)
			
			\paragraph{Les travaux somptuaires (article 34 de la loi)}
			
				Si l'on retient l’idée déjà exprimée que ce qui est somptuaire correspond à un luxe excessif, ces travaux d'amélioration sont alors excessifs « eu égard à l'état, aux caractéristiques et à la destination de l'immeuble » selon les termes mêmes de l'article 34 de la loi qui est ainsi rédigé :
				\begin{quote}
					Article 34 de la loi
					
					La décision prévue à l'article 30 n'est pas opposable au copropriétaire opposant qui a, dans le délai prévu à l'article 42, alinéa 2, saisi le tribunal judiciaire en vue de faire reconnaître que l'amélioration décidée présente un caractère somptuaire eu égard à l'état, aux caractéristiques et à la destination de l'immeuble
				\end{quote}
				
				La loi considère bien que ces travaux d'amélioration ne sont pas conformes à la destination de l'immeuble.
				
				Le législateur aurait pu soumettre ces travaux somptuaires au seul droit commun des travaux d'amélioration : tout copropriétaire opposant ou défaillant aurait pu en conséquence faire juger leur non-conformité à la destination de l’immeuble et obtenir la décision d'annulation de la décision contraire aux dispositions de l’article 30 de la loi dont nous avons dit précédemment qu'il n'autorisait que le vote de « travaux d'amélioration conformes à la destination de l’immeuble »
				
				Pourtant de tels travaux d'amélioration somptuaires bénéficient d'un régime particulier. La loi laisse au copropriétaire opposant le choix entre deux attitudes :
				\begin{itemize}
					\item soit le copropriétaire opposant demande au juge l’annulation de la résolution sur le fondement de l'article 30 de la loi, auquel cas toute la copropriété sera privée des travaux que le syndicat des copropriétaires se propose de mettre en œuvre ;
					\item soit le copropriétaire opposant demande seulement aux juges de déclarer que la résolution lui est inopposable, également dans le délai de recours de 2 mois.
				\end{itemize}
				
				Ainsi, le copropriétaire qui exerce l'action aux fins d'inopposabilité obtient du juge qu'il constate que les travaux projetés ne sont pas conformes à la destination de l'immeuble mais ces travaux sont cependant maintenus à l'égard des autres copropriétaires dont les engagements se trouvent aggravés du fait que le copropriétaire opposant n’a pas à payer sa quote-part de ces travaux.
				
				Dans un arrêt\footnote{Civ 3ème, 4 nov 2004, \no 03-14.342 ; IRC 2006 \no 515 p 12, note Ritschy.} du 4 novembre 2004, la cour de cassation a considéré que le copropriétaire opposant ayant saisi le juge d’une demande sur le fondement de l’article 34 de la loi serait en droit de s’opposer au paiement de sa quote-part des travaux jusqu’à décision au fond du juge.
				
				Toutefois, si la demande n’est dans le délai de deux que relative à l’inopposabilité des travaux, la demande ultérieure d’annulation est considérée comme tardive\footnote{Civ. 3ème, 12 janv. 2010, \no 09-12.1999 : Administrer 4/2010. 37, obs. R. Bouyeure.}.
		
		\subsubsection{Les règles de majorité pour les travaux d’amélioration}
			
			\paragraph{Principe}
			
				{\bfseries Le vote des travaux de l’article 30 se fait désormais à la majorité de l’article 25 n),avec possibilité de passerelle à compter du 1er juin 2020.}
				
				Les conditions de majorité, très contraignantes à l'origine (la double majorité en nombre des copropriétaires représentant les 3/4 des voix de la copropriété) ont été assouplies par la loi du 31 décembre 1985 (réduction de la majorité des voix au 2/3 des voix de la copropriété et la moitié des copropriétaires), puis par la loi du 21 juillet 1994 (création, sous certaines conditions du majorité de repêchage, lors d’une seconde assemblée).
				
				La loi ALUR a purement et simplement abaissé la majorité nécessaire pour le vote de ces travaux à la majorité de l’article 25 n (majorité absolue des voix de tous les copropriétaires), {\bfseries mais sans recours possible à la passerelle de l’article 25-1, cet article excluant explicitement la faculté de « passerelle » pour les décisions de l’article 25 n (travaux d’amélioration) et 25 o ( individualisation des contrats d’eau)}.
				
				Ce dernier verrou a sauté avec l’ordonnance du 30 octobre 2019 : désormais, tous les travaux d’amélioration peuvent être adoptés à majorité des voix exprimées des copropriétaires présents ou représentés, si le projet recueuille lors du 1er vote le tiers des voix de tous les copropriétaires.
				
				\begin{quote}
					Article 25
	
					Ne sont adoptées qu'à la majorité des voix de tous les copropriétaires les décisions concernant :
					
					n) L'ensemble des travaux comportant transformation, addition ou amélioration ;
				\end{quote}
				\begin{quote}
					Article 25-1 (différé) Modifié par Ordonnance \no2019-1101 du 30 octobre 2019 - art. 26
					
					Lorsque l'assemblée générale des copropriétaires n'a pas décidé à la majorité des voix de tous les copropriétaires, en application de l'article 25 ou d'une autre disposition, mais que le projet a recueilli au moins le tiers de ces voix, la même assemblée se prononce à la majorité prévue à l'article 24 en procédant immédiatement à un second vote.
				\end{quote}
				
				Cet abaissement direct et indirect de majorité peut avoir des conséquences non négligeables : par exemple, jusque lors l’installation de l’ascenseur aux frais du syndicat des copropriétaires à la double majorité de l’article 26 constituait un obstacle sérieux à cette réalistaion qui n’intéresse véritablement que les copropriétaires des étages supérieurs. Désormais il suffira de la majorité des voix de la copropriété, plus facile à obtenir pour ces copropriétaires intéressés, et même la majorité simple par le jeu de passerelle lorsque le tiers des voix des copropriétaires est atteint.
				
				\textbf{Base du vote}
				
				Tous les copropriétaires, y compris ceux qui n’auront pas à participer à la dépense, participent au vote de travaux d’amélioration (CA DE PARIS 19 NOV.1992 ADM 93 P48), sauf si le règlement de copropriété en dispose autrement, conformément à l’article 10 de la L. du 10 juillet 1965 ou que les travaux concernent des parties communes spéciales à certains copropriétaires.
				
				L’article 9 de l'ordonnance du 30 octobre 2019 a ainsi précisé, par la modification de l’article 10 de la L. du 10 juill. 1965, que lorsque le règlement de copropriété prévoit que seuls les copropriétaires qui ont la charge de certaines dépenses, alors « chacun d’eux dispose d’un nombre de voix proportionnel à sa participation auxdites dépenses ».
				Le vote à la majorité de l’article 25 s’impose non seulement pour la décision de principe, mais également pour le choix des devis (CIV.3EME 9 AVR .2008).
			
			\paragraph{Première dérogation : les travaux d’amélioration relevant de la majorité de l’article 24 de la Loi}
			
				Certains travaux, jugés comme « indispensables », ont progressivement « glissés » de l’ancienne double majorité (art 26) pour se retrouver dans l’article 25… puis 24, et relèvent ainsi d’un vote à la majorité simple. En l’état actuel du texte ( Ordonnance du 30 octobre 2019) ces travaux sont les suivants
				24 e) La suppression des vides ordures
				L’ordonnance du 30 octobre 2019 a abaissé la majorité nécessaire pour supprimer les vides ordures pour des impératifs d’hygiène qui est transféré de l’article 25 g) à l’article 24 e). Comme précédemment, on peut s’interroger sur la limitation légale d’application du présent texte : la suppression du vide-ordures doit être imposée par des impératifs d’hygiène. Il sera donc prudent de prévoir, dans le projet de résolution, la motivation de la suppression de V.O.
				24 d) Les travaux d’accessibilité PMR
				Depuis la loi \no 2014-366 du 24 mars 2014 dite ALUR, ces travaux relèvent explicitement de l’article 24
				« 24 d) de la loi Les travaux d’accessibilité aux personnes handicapées ou à mobilité réduite, sous réserve qu’ils n’affectent pas la structure de l’immeuble ou ses éléments d’équipement essentiels ;
				Il s’agit ici de travaux d’amélioration qui relèvent normalement de l’article 25 n), puisqu’il s’agit de travaux d’amélioration travaux décidés par l’assemblée générale aux frais du syndicat. Mais, dès lors que ces travaux n’affectent pas la structure de l’immeuble, ils sont votés à la majorité de l’article 24.
				Les travaux d’accessibilité des immeubles aux handicapés font l’objet de dispositions réglementaires dans le Code de Construction et d’Habitation (C.C.H.) : art. R 111-18 à R 111-19. Mais ces dispositions ne s’appliquent qu’aux nouvelles constructions.
				Dans les constructions existantes, la mise aux normes PMR peut être assurée par le Syndicat des Copropriétaires lui-même, ou par le copropriétaire handicapé souhaitant réaliser des travaux pour accéder plus facilement à son lot (réalisation d’une rampe d’accès par réduction de la taille de l’escalier). Toutefois, il existe désormais une procédure spécifique de « non opposition » lorsque le copropriétaire prend l’initiative des travaux (cf. infra)
				24 g) La décision d’engager le diagnostic technique global à l’article L. 731-1 du code de la construction et de l’habitation ainsi que ses modalités de réalisation.
				Il ne s’agit pas de travaux à proprement parler mais seulement de donner un contrat à un technicien chargé d’établir le diagnostic technique (art L731-1 CCH, cf. infra).
			
			\paragraph{Articles 24-1 à 24-9 de la loi : les « travaux d’amélioration » devant obligatoirement être proposés par le syndic, et relevant par dérogation une décision à la majorité relative des voix exprimées}
			
				Ces articles subissent directement l’empreinte des objectifs de rénovation et de modernisation du parc privé imposés par le législateur : afin de permettre la réalisation dans les copropriétés de travaux jugés prioritaires, il est fait obligation au syndic d’inscrire la question de la réalisation de ces travaux à l’ordre du jour. Les copropriétaires peuvent alors décider de procéder à ces travaux à la majorité des voix exprimées des copropriétaires présents ou représentés ou ayant voté par correspondance.
				- Article 24-1 et 24 -3 : le raccordement à la TNT
				- Lorsqu'un réseau de communications électroniques interne à l'immeuble distribue des services de télévision (24-1), c’est-à-dire que la télévision est distribuée par le câble (payant)
				- Ou lorsque l'immeuble reçoit des services de télévision par voie hertzienne terrestre par une antenne collective (24-3)
				L'ordre du jour de l'assemblée générale comporte de droit, si l'installation ne permet pas encore l'accès aux services nationaux en clair de télévision par voie hertzienne terrestre en mode numérique et si le distributeur de services dispose d'une offre en mode numérique, l'examen de toute proposition commerciale telle que visée à la deuxième phrase du deuxième alinéa de l'article 34-1 de la loi \no 86-1067 du 30 septembre 1986 relative à la liberté de communication.
				Par dérogation au h de l'article 25 de la présente loi, la décision d'accepter cette proposition commerciale est acquise à la majorité prévue au premier alinéa du I de l'article 24.
				- Article 24-2 : raccordement de l’immeuble au haut débit
				De même, lorsque l'immeuble n'est pas équipé de lignes de communications électroniques à très haut débit en fibre optique, toute proposition émanant d'un opérateur de communications électroniques d'installer de telles lignes en vue de permettre la desserte de l'ensemble des occupants et au plus tard dans les douze mois à compter de la proposition reçue par le syndic est inscrite de droit à l'ordre du jour de la prochaine assemblée générale. La décision relève de l’article 24.
				Pour faciliter le vote la loi la Loi \no 2018-1021 du 23 nov. 2018 dite Macron a précisé que :
				« L'assemblée générale peut également, dans les mêmes conditions, donner mandat au conseil syndical pour se prononcer sur toute proposition future émanant d'un opérateur de communications électroniques en vue d'installer des lignes de communication électroniques à très haut débit mentionnées au premier alinéa du présent article. Tant qu'une telle installation n'a pas été autorisée, l'ordre du jour de l'assemblée générale comporte de droit un projet de résolution donnant au conseil syndical un tel mandat. »
				En effet, la décision pose des questions techniques qu’il est difficile de discuter dans le cadre de l’assemblée générale.
				- Article 24-4 : l’adoption du plan pluri-annuel de travaux ou du Plan de travaux
				Article 24-4
				Pour tout immeuble équipé d'une installation collective de chauffage ou de refroidissement, le syndic inscrit à l'ordre du jour de l'assemblée générale des copropriétaires qui suit l'établissement d'un diagnostic de performance énergétique prévu à l'article L. 134-1 du code de la construction et de l'habitation ou d'un audit énergétique prévu à l'article L. 134-4-1 du même code la question d'un plan de travaux d'économies d'énergie ou d'un contrat de performance énergétique.
				Avant de soumettre au vote de l'assemblée générale un projet de conclusion d'un tel contrat, le syndic procède à une mise en concurrence de plusieurs prestataires et recueille l'avis du conseil syndical.
				L'obligation prévue au premier alinéa est satisfaite si le plan pluriannuel de travaux, inscrit à l'ordre du jour de l'assemblée générale en application de l'article L. 731-2 du code de la construction et de l'habitation, comporte des travaux d'économie d'énergie.
				Un décret en Conseil d'Etat fixe les conditions d'application du présent article.
				- 24.5 les mobilités douces : vélos et IRVE (installation aux frais du syndicat des copropriétaires)
				Afin de faciliter la conversion des modes de transports en faveur des vélos des véhicules électriques, une loi d’orientation des mobilités, dite loi LOM a été adoptée le 24 décembre 2019, afin de favoriser l’adoption des travaux nécessaires dans les copropriétés.
				Cette loi s’applique aux assemblées générales de copropriétaires convoquées à compter du premier jour du troisième mois suivant la promulgation de la présente loi, c’est-à-dire aux assemblées générales qui sont convoquées à compter du 1er mars 2020.
				24.5 de la Loi du 10 juillet 65 issu de la loi \no 2019-1428 du 24 décembre 2019 d’orientation des mobilités, dite « loi LOM » (entrée en vigueur le 31 décembre 2020)
				Lorsque l'immeuble possède des emplacements de stationnement d'accès sécurisé à usage privatif et n'est pas équipé de stationnements sécurisés pour les vélos ou des installations électriques intérieures permettant l'alimentation de ces emplacements pour permettre la recharge des véhicules électriques ou hybrides ou des installations de recharge électrique permettant un comptage individuel pour ces mêmes véhicules, le syndic inscrit à l'ordre du jour de l'assemblée générale la question des travaux permettant le stationnement sécurisé des vélos ou la recharge des véhicules électriques ou hybrides et des conditions de gestion ultérieure du nouveau réseau électrique, ainsi que la présentation des devis élaborés à cet effet.
				III.-Quand les travaux permettant de réaliser les installations mentionnées au II n'ont pas été réalisés, le syndic inscrit à l'ordre du jour de l'assemblée générale des copropriétaires la question de la réalisation d'une étude portant sur l'adéquation des installations électriques existantes aux équipements de recharge et, le cas échéant, les travaux à réaliser à cet effet.
				IV.-Le syndic inscrit à l'ordre du jour de l'assemblée générale des copropriétaires la question des travaux et, le cas échéant, les conditions de gestion ultérieure du nouveau réseau électrique.
				Sont joints à la convocation à l'assemblée générale des copropriétaires :
				1\degres Le détail des travaux à réaliser ;
				2\degres Les devis et les plans de financement élaborés à cet effet ;
				3\degres Le cas échéant, le projet de contrat fixant les conditions de gestion du réseau électrique modifié ;
				4\degres Lorsqu'elle a été réalisée, l'étude mentionnée au III du présent article.
				Pour les vélos :
				L’autorisation donnée à certains copropriétaires de réaliser les travaux permettant ce stationnement, sous certaines conditions (pas d’impact sur structure de l'immeuble, sa destination ou ses éléments d'équipement essentiels et sur « la sécurité des occupants ») tout comme la décision de réaliser les travaux par le syndicat des copropriétaires mais aux frais des seuls demandeurs relèvent toutes deux de la majorité de l’article 24.
				Dans les autres cas (réalisation à frais communs , ou travaux affectant la structure de l’immeuble), la décision devra être prise selon le droit « commun », donc à la majorité de l’article 25
				article 24 j nouveau issu de la loi \no 2019-1428 du 24 décembre 2019 d’orientation des mobilités, dite « loi LOM » (entrée en vigueur le 31 décembre 2020)
				L'autorisation donnée à un ou plusieurs copropriétaires d'effectuer à leurs frais les travaux permettant le stationnement sécurisé des vélos dans les parties communes, sous réserve que ces travaux n'affectent pas la structure de l'immeuble, sa destination ou ses éléments d'équipement essentiels et qu'ils ne mettent pas en cause la sécurité des occupants.
				Article 24.5 in fine issu de la loi \no 2019-1428 du 24 décembre 2019 d’orientation des mobilités, dite « loi LOM » (entrée en vigueur le 31 décembre 2020)
				Un ou plusieurs copropriétaires peuvent demander au syndic d'inscrire à l'ordre du jour de l'assemblée générale des copropriétaires la question des travaux mentionnés au premier alinéa du présent IV pour qu'ils soient réalisés sous la responsabilité du syndicat des copropriétaires et aux frais des seuls copropriétaires demandeurs. Cette délibération est adoptée dans les conditions prévues au I de l'article 24.
				Pour les IRVE
				La question de l’installation d’IRVE doit obligatoirement être posée par le syndic, dès lors qu’il existe des emplacements de stationnement d’accès sécurisé à usage privatif. Le nouveau texte de l’article 24. 5 précise les documents à joindre à la convocation.
				Par ailleurs, il existe depuis la loi Grenelle II du 12 juillet 2010 l’obligation pour les syndics d’inscrire à l’ordre du jour la réalisation d’une étude portant sur la compatibilité de l’installation électrique existante avec les IRVE et les travaux nécessaires pour permettre l’utilisation de ces équipements. La loi mobilités douce a sorti cette obligation d’un délai, la question devant être posé avant le 1er janvier 2023.
				Par dérogation, l’article 24 i nouveau prévoit que la décision d’équiper les emplacements stationnement en IRVE relève de la majorité relative des voix exprimées. Par ailleurs, comme pour les emplacements de stationnement des vélos, les copropriétaires peuvent demander que le syndicat des copropriétaires a sur la réalisation de ces équipements, à leurs frais exclusifs, et la décision relève alors de l’article 24 (cf 24.5 in fine)
				article 24 i nouveau issu de la loi \no 2019-1428 du 24 décembre 2019 d’orientation des mobilités, dite « loi LOM » (entrée en vigueur le 31 décembre 2020)
				Ne sont adoptées qu'à la majorité des voix de tous les copropriétaires les décisions concernant :
				i) La décision d'équiper les emplacements de stationnement d'accès sécurisé à usage privatif avec des bornes de recharge pour véhicules électriques et hybrides rechargeables et de réaliser l'étude mentionnée au III de l'article 24-5 ;
				Toutefois, l’article 25j n’est pas modifié : la décision d’équiper l’immeuble en IRVE à frais collectifs relève donc de la majorité absolue des voix (avec passerelle)
				j) L'installation ou la modification des installations électriques intérieures ou extérieures permettant l'alimentation des emplacements de stationnement d'accès sécurisé à usage
				privatif pour permettre la recharge des véhicules électriques ou hybrides rechargeables, ainsi que la réalisation des installations de recharge électrique permettant un comptage individuel pour ces mêmes véhicules ;
				Ces dispositions se combinent avec l’article L 111 – 3-8 du CCH qui permettent à l'occupant de bonne foi d'emplacements de stationnement non équipés de réaliser à ses frais les travaux d’installation d’une IRVE permettant le décompte de la consommation, sans que le syndicat des copropriétaires puisse s’y opposer, sauf pour un motif sérieux et légitime.
				La loi mobilité précise que ce motif sérieux et légitime est la préexistence de telle installation de la décision prise par le syndicat des copropriétaires de réaliser de telles installations, en vue d’assurer l’équipement nécessaire dans un délai raisonnable (article L 111 – 3 –8 alinéa deux du code de la construction de l’habitation).
				La procédure de raccordement est prévue par le décret du 25 juillet 2011 (art R.136-2 et R.136-3 du CCH) et sera sans doute modifié à la suite de la loi Mobilité)24
				Art. L. 111-3-8, al. 3 et 4 CCH
				« Constitue notamment un motif sérieux et légitime au sens du premier alinéa la préexistence de telles installations ou la décision prise par le propriétaire ou, en cas de copropriété, le syndicat des copropriétaires de réaliser de telles installations en vue d'assurer l'équipement nécessaire dans un délai raisonnable.
				« Afin de lui permettre de réaliser une étude et un devis pour les travaux mentionnés au même premier alinéa, le propriétaire ou, en cas de copropriété, le syndic permet l'accès aux locaux techniques de l'immeuble concernés au prestataire choisi par le locataire ou l'occupant de bonne foi.
				24
				1. La demande est notifiée au copropriétaire avec copie au syndic de copropriété représentant le syndicat des copropriétaires, avec description des travaux.
				2. Dans le délai de trois mois suivant la réception de la demande, le copropriétaire notifie au syndic sa demande d'inscription de la question à l'ordre du jour de la prochaine assemblée générale.
				3. Le propriétaire, ou le cas échéant le syndicat des copropriétaires représenté par le syndic, qui entend s'opposer aux travaux doit, à peine de forclusion, saisir le juge de proximité du lieu de l'immeuble dans le délai de six mois suivant réception de la demande.
				4. Le propriétaire, ou le cas échéant le syndicat des copropriétaires représenté par le syndic, notifie une copie de la saisine au demandeur, ou le cas échéant au copropriétaire qui la notifie sans délai au demandeur, par lettre recommandée avec demande d'avis de réception.
				5. Le propriétaire, ou le cas échéant le syndicat des copropriétaires représenté par le syndic, peut dans le même délai décider de la réalisation de tels travaux afin d'équiper l'ensemble des places de stationnement de l'immeuble (vote à la majorité de l’article 25 l ).
				6. Si le propriétaire, ou le cas échéant le syndicat des copropriétaires représenté par le syndic, n'a pas réalisé les travaux dans les six mois suivant la décision visée à l'alinéa précédent, le demandeur pourra procéder à l'exécution des travaux qui ont fait l'objet de la notification au premier alinéa du présent article.
				« Les indivisaires, les copropriétaires et les membres des sociétés de construction peuvent se prévaloir du présent article et de l'article L. 111-3-9. »
			
			\paragraph{Les travaux d’amélioration de l’article 25 avec second vote à la majorité de l’article 25-1 (dispositions antérieures à l’ordonnance du 30 octobre et conservées)}
			
				Alors que les travaux d’amélioration relevaient encore de l’article 26, le législateur avait introduit dans 25 d’autres travaux d’amélioration dont il entendait faciliter l’adoption, au motif qu’un intérêt économique ou social supérieur commandait que ces travaux soient décidés par une majorité moins exigeante. Ces travaux pouvaient bénéficier de la passerelle de l’article 25-1.
				
				La « survie » de ces dispositions spécifiques après l’ordonnance du 30 octobre 2019 s’explique mal, sauf dans le cadre d’un régime « transitoire » en attente de la codification, puisque désormais tous les travaux d’amélioration (art 25 n) peuvent désormais faire l’objet d’un second vote.
				
				Relèvent donc, par mention spécifique dans l’article 25 :
				
				\begin{list}{}{}
					\item \textbf{25 f) Les travaux d’économie d’énergie ou de réduction des GES}
				(cf infra rénovation énergétique)
				
					\item \textbf{25 g) Les modalités d'ouverture des portes d'accès aux immeubles.}
					La majorité requise pour l’adoption des travaux « destinés à prévenir les atteintes à la sécurité des personnes et des biens » ( digicode, interphone, voir caméras de surveillance) a été modifiée à de multiples reprises à partir des années 2000.
					Depuis la loi ALUR, ces travaux (qui relevait jusqu’alors de la majorité l’article 25 par application des dispositions spéciales, mais qui ont « disparu » à l’occasion de cette réforme) sont susceptibles de révéler de différentes catégories :
					- soit l’article 24 a) permettant d’adopter à la majorité simple les travaux nécessaires « à la conservation de l'immeuble ainsi qu'à la préservation de la santé et de la sécurité physique des occupants »,
					- soit de la majorité de l’article 25 n, c’est-à-dire des « simples » travaux d’amélioration, pour lesquels un second vote est désormais possible.
					- Voire l’unanimité en cas d’atteinte aux modalités de jouissance du lot25
					Il y a là une véritable difficulté d’interprétation, qui n’a pas été tranché par l’ordonnance, laquelle ne vise pas spécifiquement le vote des travaux de fermeture de l’immeuble. Toutefois, avec les dispositions de l’ordonnance l’ange devient bien moindre puisque tous les travaux
					25 On relèvera avec intérêt l’arrêt de la 3\degres Ch. Civile du 18 février 2015 qui a considéré que la décision de fermer en permanence une barrière d’accès des véhicules, bien que n’interdisant pas le libre accès des piétons à l’immeuuble, implique un vote de l’article 26 c) (3\degres ch. Civ. 18 février 2015, \no 13-25974)
					d’amélioration peut faire l’objet d’un second vote par le biais de la passerelle, et donc être finalement adopté à la majorité l’article 24.
					En revanche, l’ordonnance traite une problématique qui avait été soulevée par la doctrine au sujet des modalités d’ouverture de la porte de l’immeuble, lorsque des travaux de sécurisation ont été adoptés. Jusqu’à l’ordonnance de 2019, continuaient de recevoir application pour la pose de digicodes les dispositions de l’article 26 c) de la loi aux termes desquelles l’assemblée générale vote à cette dernière majorité :
					« c) Les modalités d’ouverture des portes d’accès aux immeubles. En cas de fermeture totale de l’immeuble, celle-ci doit être compatible avec l’exercice d’une activité autorisée par le règlement de copropriété »
					Cette situation était extrêmement gênante, car il était fréquent que l’assemblée générale soit en mesure d’adopter les travaux de fermeture à la majorité l’article 25, mais dans l’incapacité de prendre une décision concernant les modalités d’ouverture compatible avec l’exercice d’une activité prévue dans le règlement de copropriété, faute d’un nombre suffisant de copropriétaires présents à l’assemblée générale.
					L’ordonnance de 2019 a finalement abaissé la majorité requise pour statuer sur les modalités d’ouverture de la porte, puisque celle-ci figure dorénavant à l’article 25 g)
					« g) Les modalités d'ouverture des portes d'accès aux immeubles. En cas de fermeture totale de l'immeuble, celle-ci doit être compatible avec l'exercice d'une activité autorisée par le règlement de copropriété ; »
					25 h) Les installations d’antenne hertzienne /câbles/antenne de téléphonie mobile
					L'installation d'une station radioélectrique nécessaire au déploiement d'un réseau radioélectrique ouvert au public ou l'installation ou la modification d'une antenne collective ou d'un réseau de communications électroniques interne à l'immeuble dès lors qu'elles portent sur des parties communes
					Il appartient à l'assemblée générale de décider à la majorité de l'article 25 l'installation ou la modification des antennes collectives ( antenne hertzienne, satellite collective) ou d’un réseau de communications électroniques interne (câble), sous réserve des dispositions propres à la TNT ou au haut débit (article 24)
					La loi MACRON, du 08 août 2015 a ajouté à cet article, le vote de la décision nécessaire à « l'installation d'une station radioélectrique nécessaire au déploiement d'un réseau radioélectrique ouvert au public », en d’autres termes une antenne relais pour la téléphonie mobile et le wifi. Bien que la jurisprudence ait évolué dans le même sens ces dernières années, cette disposition consacre l’abandon du « principe de précaution » qui avait permis un temps
					aux juridictions civiles d’exiger l’unanimité des copropriétaires pour cette décision, en raison de l’atteinte (éventuelle) aux modalités de jouissance des lots26.
					Toutefois, la double majorité de l’article 26 semble toujours applicable si l’installation de l’antenne relais implique la cession d’un droit réel (bail emphytéotique, constitution d’usufruit)
					25j) Locaux vélos et IRVE aux frais du syndicat des copropriétaires
					Cf supra ( loi LOM 29 décembre 2020)
					25 k) « L'installation de compteurs d'eau froide divisionnaires. »
					Très souvent il n’existe qu’un compteur unique d’eau froide par bâtiment, voire par copropriété et que dès lots les règlements de copropriété prévoient que la répartition des charges d’eau froide se fait en fonction des tantièmes bâtiment ou généraux.
					La loi S.R.U. a souhaité responsabiliser les consommateurs dans les immeubles collectifs, en facilitant la prise de décision d’installation de compteurs individuels d’eau froide par l'assemblée générale.
					Il ne faut pas confondre la décision de l’assemblée générale d’installer des compteurs d’eau froide divisionnaires en vue de spécialiser les charges d’eau froide avec la décision de l’article 25 o) qui porte sur l’autorisation donnée à un copropriétaire de bénéficier d’un contrat de fourniture d’eau individuel.
					25 l) L'installation de compteurs d'énergie thermique ou de répartiteurs de frais de chauffage ; Sur ce point, cf. Infra
					25 o) La demande d'individualisation des contrats de fourniture d'eau et la réalisation des études et travaux nécessaires à cette individualisation.
					26 En ce sens, voir l’arrêt –très médiatisé- de la 23ème Chambre B de la Cour d’Appel de PARIS 7 avril 2005, mais un arrêt de la 3ème chambre civile avait implicitement admis, au contraire, la validité du vote à l’article 25 « par défaut » puisqu’il n’existait pas à l’époque de disposition spéciale : Civ. 3ème 24 février 2009 \no de pourvoi: 08-11003.
					Parallèlement, un arrêt du Tribunal des conflits (14 mai 2012) a arbitré sur la compétence du juge liée aux antennes relais : seul le juge administratif est habilité à demander le démontage d’une antenne -relais en vertu de la police spéciale de l’Etat. Il ne reconnaît une compétence du juge judiciaire qu’en cas de trouble anormal de voisinage. Par deux décisions en date du 17 octobre 2012 (pourvois \no 10-26.854 et \no 11-19.259), la Première chambre civile de la Cour de cassation a confirmé l’incompétence du juge judiciaire pour connaître de toute « action portée devant le juge judiciaire, quel qu’en soit le fondement, aux fins d’obtenir l’interruption de l’émission, l’interdiction de l’implantation, l’enlèvement ou le déplacement d’une station radioélectrique ». Elle se reconnaît néanmoins compétente en matière d’indemnisation des dommages causés par une antenne relais. Elle n’a pas eu à statuer, pour le moment, sur la compétence du Juge judiciaire en cas d’annulation de l’autorisation d’assemblée générale.
					Jusqu’à l’ordonnance du 30 octobre 2019, le régime de l’individualisation des contrats de fourniture d’eau était aligné sur celui des travaux d’amélioration « sans passerelle possible ». Cette distinction a disparu avec l’ordonnance, néanmoins il faut être conscient que les travaux d’individualisation des contrats de fourniture d’eau sont extrêmement lourds, car il nécessite une individualisation complète des réseaux, chaque appartement devant être desservi par un réseau autonome afin de permettre une facturation directe du prestataire sur le copropriétaire.
				\end{list}
	
\section{Le financement des travaux par le syndicat des copropriétaires}

	\subsection{A. LES EMPRUNTS POUR TRAVAUX SOUSCRITS PAR LE SYNDICAT DES COPROPRIETAIRES (ARTICLES 26-4 A 26-8)}
	
		\subsubsection{1. Principe}
		
			La rédaction même du texte laisse bien entendre la réserve du législateur au regard de ce mode de financement, qui apparait comme un « mal nécessaire » : le principe est … la prohibition, ou plutôt la souscription à l’unanimité, mais avec dérogations.
			Article 26-4 al 1, 2 et 3 Créé par LOI \no2012-387 du 22 mars 2012 - art. 103 (V) et modifié par la Loi \no 2015-992 du 17 aoôut 2015 – art. 23 et l’ordonnance du 30 oct. 2019 – art. 30
			« L'assemblée générale ne peut, sauf à l'unanimité des voix des copropriétaires, décider la souscription d'un emprunt au nom du syndicat des copropriétaires pour le financement soit de travaux régulièrement votés concernant les parties communes ou de travaux d'intérêt collectif sur parties privatives régulièrement votés, soit des actes d'acquisition conformes à l'objet du syndicat et régulièrement votés.
			Par dérogation au premier alinéa, l'assemblée générale peut également, à la même majorité que celle nécessaire au vote des travaux concernant les parties communes ou de travaux d'intérêt collectif sur parties privatives, voter la souscription d'un emprunt au nom du syndicat des copropriétaires lorsque cet emprunt a pour unique objectif le préfinancement de subventions publiques accordées au syndicat pour la réalisation des travaux votés.
			Par dérogation au premier alinéa, l'assemblée générale peut, à la même majorité que celle nécessaire au vote soit des travaux concernant les parties communes ou de travaux d'intérêt collectif sur parties privatives, soit des actes d'acquisition conformes à l'objet du syndicat, voter la souscription d'un emprunt au nom du syndicat des copropriétaires au bénéfice des seuls copropriétaires décidant d'y participer. »
			Afin d’encourager la réalisation des travaux l’ordonnance a imposé l’inscription à l’ordre du jour la souscription d’un emprunt lorsque l’assemblée serait appelée à se prononcer sur les travaux mentionnées à l’article 26-4, c’est-à-dire les « travaux concernant les parties communes ou de travaux d'intérêt collectif sur parties privatives », soit la plupart si ce n’est tous les travaux…
			Art. 25-3 issu de l’Ordonnance du 30 octobre 2019 « Lorsque l'assemblée générale des copropriétaires est appelée à se prononcer sur les travaux mentionnés à l'article 26-4, la question de la souscription d'un emprunt collectif destiné à financer ces travaux est inscrite à l'ordre du jour de la même assemblée générale. »
			La difficulté de mise en oeuvre de ce texte résultera des seuils imposés par les organismes de crédit pour la souscription qui avoisinent généralement les 30 000 euros. Par ailleurs, on ignore si la sanction de cette disposition pourrait être la nullité des travaux votés sans emprunt…
		
		\subsubsection{2. Sur la majorité requise pour la souscription d'un emprunt collectif (art 26-4)}
		
			L'unanimité des voix des copropriétaires est requise pour la souscription d'un emprunt bancaire, qu'il s'agisse de financer des travaux régulièrement votés concernant des parties communes ou des travaux d'intérêt collectif sur parties privatives, ou des actes d'acquisition conforme à l'objet du syndicat est régulièrement votés.
			Cette exigence d'unanimité est logique, puisque nous sommes dans l'hypothèse où l'emprunt et non seulement souscrit par le syndicat des copropriétaires, mais encore au bénéfice de tous les copropriétaires, aucun ne réglant comptant sa quote-part des travaux. Or l'assemblée générale ne peut augmenter les obligations d'un copropriétaire des intérêts d'emprunt sans son consentement.
			Par dérogation, un emprunt collectif peut être souscrit à la même majorité que celle requise pour la décision de vote des travaux sur parties communes ou d'intérêt collectif concernant les parties privatives (c'est-à-dire l'article 25 ou l'article 26) lorsque que l'unique objet de cet emprunt est le pré financement de subventions publiques accordées au syndicat pour la réalisation de travaux votés.
			Pour ces prêts de « préfinancement des subventions » (SACICAP et Caisse d’Epargne), le remboursement de la banque se fait par le versement des subventions au profit de l’organisme bancaire: le prêteur est subrogé dans les droits du copropriétaire emprunteur et du Syndicat des Copropriétaires auprès de l’organisme versant les fonds.
			En revanche, la souscription d’un emprunt bancaire au nom du syndicat des copropriétaires, mais au bénéfice des seuls copropriétaires décidant d’y participer, doit être voté à la même majorité que celle nécessaire pour le vote des travaux où l'acte d'acquisition
		
		\subsubsection{3. Sur les modalités de souscription du prêt (art 26- 4 et 26-5)}
		
			Le texte de l'article 26-4 met en place la procédure qui permet l'individualisation de la souscription du prêt. En effet, le dernier alinéa dispose que :
			« Art 26-4
			Les copropriétaires qui décident de participer à l’emprunt doivent notifier leur décision au syndic en précisant le montant de l’emprunt qu’ils entendent solliciter, dans la limite de leur quote-part des dépenses. « A peine de forclusion, la notification au syndic doit intervenir dans le délai de deux mois à compter de la notification du procès-verbal d'assemblée générale, sans ses annexes, à tous les copropriétaires. »
			Ainsi, les copropriétaires ont un temps limité pour indiquer au syndic s'ils entendent ou non recourir à l’emprunt .
			Ce délai a été unifié par l’ordonnance du 30 octobre 2019 a modifié le mécanisme en prévoyant une notification de la décision à tous les copropriétaires afin d’unifier le point de
			départ, alors qu’auparavant, le procès verbal étant notifié par RAR aux seuls copropriétaires défaillants ou opposants, le délai courrait pour les copropriétaires ayant voté « pour » à compter de l’assemblée générale elle-même.
			Le texte de l'article 26-5 pose par ailleurs un certain nombre de garde-fous afin de permettre une information correcte des copropriétaires au moment de la souscription l'emprunt, information qui était jusqu'à présent nettement insuffisante.
			Article 26-5
			Le contrat de prêt conclu en application de l'article 26-4, conforme aux conditions générales et particulières du projet de contrat de prêt jointes à l'ordre du jour de l'assemblée générale, ne peut être signé par le syndic avant l'expiration du délai de recours de deux mois prévu au deuxième alinéa de l'article 42.
			Par conséquent
			- les conditions générales et particulières du projet de contrat de prêt doivent être jointes à l'ordre du jour de l'assemblée générale.
			Aux termes de l’article 11 – I – 3\degres du Décret du 17 mars 1967, le syndic doit joindre à la convocation à l’assemblée générale
			- du projet de contrat
			- et la proposition d'engagement de caution mentionné au deuxième alinéa de l'article 26-7 de la loi du 10 juillet 1965 lorsque le contrat proposé a pour objet la souscription d'un prêt bancaire au nom du syndicat dans les conditions prévues à l'article 26-4 de cette loi »
			- le contrat de prêt ne peut être signé avant l'expiration du délai de recours de deux mois prévus à l'article 42
			Cette limite temporelle s'impose pour deux raisons : d'une part le prêt ne peut être souscrit valablement que si les travaux ont été votés régulièrement, si bien qu'il est essentiel pour le syndic de savoir s'il existe ou non un recours sur la résolution, et d'autre part, ce délai correspond à celui donné aux copropriétaires pour demander leur adhésion à l’emprunt
		
		\subsubsection{4. Sur les modalités de remboursement de l'emprunt (art. 26-6)}
		
			Article 26-6
			Le montant de l'emprunt mentionné à l'article 26-4, qui ne peut excéder le montant total des quotes-parts de dépenses des copropriétaires décidant d'y participer, est versé par l'établissement bancaire au syndicat des copropriétaires, représenté par le syndic. Seuls les copropriétaires bénéficiant de l'emprunt sont tenus de contribuer : 1\degres A son remboursement au syndicat, en fonction du montant pour lequel ils participent à l'emprunt et selon la grille générale établie pour la répartition des quotes-parts de dépenses selon les principes prévus aux articles 10, 10-1 et 30 ; 2\degres Au paiement au syndicat des intérêts, des frais et des honoraires y afférents, en fonction du montant pour lequel ils participent à l'emprunt et selon la grille spécifique établie pour la répartition des accessoires.
			Le montant de l’emprunt souscrit ne peut excéder le total des quote-part des copropriétaires y participant
			Le capital doit être versé au syndicat, et en aucun cas aux copropriétaires qui pouvaient éventuellement l'affecter à un autre objet que le règlement des appels de fonds travaux : « le montant de l’emprunt mentionné à l’article 26-4, qui ne peut excéder le montant total des quotes-parts de dépenses des copropriétaires décidant d’y participer, est versé par l’établissement bancaire au syndicat des copropriétaires, représenté par le syndic. »
			En outre, cet article pose le principe de la spécialisation des remboursements de l'emprunt :
			- pour le capital, en fonction du montant emprunté selon « la grille générale établie pour la répartition des quote-part de dépenses »
			- pour les intérêts mais « selon la grille spécifique de répartition des accessoires »
			Si le syndic entend percevoir des honoraires spécifiques pour la gestion de cet emprunt, ces honoraires sont susceptibles de relever de l'article 18-1 A de la loi d'autant que faire l'objet d'un vote distinct. Le contrat de syndic stipule toutefois des honoraires hors forfait pour cette prestation à l’article 7.2.7. Il semble donc plus plausible que le syndic puisse, par application de son contrat, bénéficier d’une rémunération sans que celle-ci ne fasse l’objet d’un vote spécifique.
		
		\subsubsection{5. Sur la garantie de l’emprunt (art. 26-7 al 3 et 26-8)}
		
			Cet article rend obligatoire la souscription d'un cautionnement solidaire, auprès d'une assurance ou d'une banque, au profit du syndicat des copropriétaires, en cas de défaillance d'un des copropriétaires souscripteurs.
			« Art. 26-7. – Le syndicat des copropriétaires est garanti en totalité, sans franchise et sans délai de carence, par un cautionnement solidaire après constat de la défaillance d’un copropriétaire bénéficiant de l’emprunt mentionné à l’article 26-4 pour les sommes correspondant à son remboursement ainsi qu’au paiement des accessoires.
			« Le cautionnement solidaire ne peut résulter que d’un engagement écrit fourni par une entreprise d’assurance spécialement agréée, par un établissement de crédit ou une institution mentionnée à l’article L. 518-1 du code monétaire et financier.
			Néanmoins, si l'article précise que ce cautionnement doit être sans franchise sans délai de carence, le syndicat des copropriétaires devra être attentif aux clauses d'exclusion pouvant figurer dans le contrat.
			L'objectif de ce cautionnement solidaire et bien entendu de limiter autant que possible les effets de la solidarité de fait entre les copropriétaires
			Ce texte met fin à la situation fâcheuse créée par la jurisprudence antérieure de la Cour de Cassation\footnote{Civ. 3ème 15 mai 2002 bull, Cie Européenne d’opérations immobilières C Syndicat des Copropriétaires 8 place Bouchard),} sur les emprunts collectifs quant à l'impossibilité de faire jouer le privilège du syndicat pour les sommes dues par le copropriétaire au titre de l'emprunt souscrit par le
			syndicat. En effet, et de façon fort logique, il assimile purement et simplement les sommes correspondant au remboursement de l'emprunt au paiement de charges et travaux.
			«Art 26-7 alinéa 3
			Au regard du privilège prévu au 1\degres bis de l’article 2374 les travaux du code civil, les sommes correspondant au remboursement de l’emprunt ainsi qu’au paiement des accessoires sont assimilées au paiement des charges et travaux. Après mise en oeuvre de la caution, celle-ci est subrogée de plein droit dans l’exercice du privilège du syndicat des copropriétaires prévu au même 1\degres bis.
			Le dispositif est complété par la déchéance du terme : en cas de vote du lot de copropriété, les échéances de l'emprunt restant à courir deviennent immédiatement exigibles, ce qui permet au syndic de faire opposition sur le prix de vente pour la totalité des sommes restant dues.
			Le montant du prêt devra apparaître dans la Première partie de l’état daté, document qui est transmis par le syndic au notaire avant l’établissement de tout acte réalisant ou constatant le transfert ou la création d’un droit réel sur un lot.
			Cette déchéance du terme peut cependant être évitée si l’acquéreur accepte de s’engager au lieu et place de son vendeur, avec l’accord du prêteur et de la caution. La décision de l’acquéreur doit alors être notifiée en même temps que la vente par le notaire (art 6 du décret)
			« Art. 26-8. – Lors d’une mutation entre vifs du lot d’un copropriétaire bénéficiant de l’emprunt mentionné à l’article 26-4, même si cette mutation est réalisée par voie d’apport en société, les sommes restant dues par le copropriétaire au titre du remboursement de l’emprunt ainsi que du paiement des accessoires deviennent immédiatement exigibles. Toutefois, en cas d’accord du prêteur et de la caution, l’obligation de payer ces sommes peut être transmise au nouveau copropriétaire avec son accord. Le notaire informe le syndic de ces accords. » ;
	
	\subsection{B. LE FONDS DE TRAVAUX (ARTICLE 14-2 DE LA LOI \no65-557 DU 10 JUILLET 1965 ISSU DE LA LOI ALUR DU 24 MARS 2014)}
	
		Le fonds de travaux est présenté comme l’outil de financement privilégié des travaux importants, par la constitution d’une épargne obligatoire par le Syndicat des Copropriétaires
		L’idée de prévoir la constitution d’un « fonds de réserve » pour financer les travaux de conservation du bâti n’est pas complètement nouvelle. Depuis la loi du 21 juillet 1994, l’alinéa 6 de l’article 18 de la loi du 10 juillet 1965, prévoyait que « le syndic doit soumettre, lors de sa première désignation et au moins tous les trois ans, au vote de l'assemblée générale, la décision de constituer des provisions spéciales en vue de faire face aux travaux d'entretien ou de conservation des parties communes et des éléments d'équipement commun, susceptibles d'être nécessaires dans les trois années à échoir et non encore décidés par l'assemblée générale. Ce procédé étant facultatif, la disposition était très peu utilisée en pratique.
		Le nouveau dispositif mis en place par la loi du 24 mars 2014 s’applique de façon obligatoire à compter du 1er janvier 2017. Il remplace le dispositif incitatif précédent. Ces « provisions spéciales », qui n’existent plus, doivent être remboursées aux copropriétaires, quitte à être
		appelées à nouveau sous la forme de « fonds de travaux » : elles ne peuvent être transférées d’office sur ce fonds, car elles ne sont pas de la même nature
		Le nouveau mécanisme se distingue du précédant sur deux points :
		- il est obligatoire, constituant ainsi une « épargne forcée »
		- il demeure acquis à la copropriété, alors que les « provisions » de l’article 18 étaient des avances remboursables lors de la vente des lots.
		
		\subsubsection{1. Champs d’application}
		
			Le fonds de travaux concerne les immeubles soumis au statut de la copropriété qui comprennent au moins un lot à destination totale ou partielle d’habitation
			Pour les immeubles neufs, il s’applique à l’issue d’un délai de 5 ans après la date de réception des travaux, afin que, lors de l’expiration de la garantie décennale, l’immeuble dispose d’au moins 5 années d’épargne
			Les seuls tempéraments sont les suivants :
			o Pour les ESH
			Le nouvel article 443-14-2 du CCH dispense les entreprises sociales pour l’habitat (ESH) de régler leur contribution au fonds de travaux. Toutefois, la dispense ne porte que sur le règlement de la somme, et le syndic devra néanmoins appeler le fonds de travaux auprès de l’ESH, et intégrer le montant de sa quote-part dans le fonds de travaux. Cependant, au lieu d’acquitter cet appel de fonds, l’ESH pourra constituer une provision dans ses charges et fournir une caution bancaire pour le montant correspondant. La fourniture de cette caution bancaire peut poser quelques difficultés, dans la mesure où le montant du fonds de travaux évolue tous les trimestres. Cette exigence n’est cependant assortie d’aucune sanction.
			o En l’absence de travaux nécessaires sur les 10 prochaines années.
			Si le diagnostic global prévu à l’article L731-1 du code de la construction et de l’habitation a été réalisé et qu’il ne fait apparaitre aucun besoin de travaux dans les dix prochaines années, le syndicat est dispensé de l’obligation de constituer un fonds de travaux pendant la durée de validité du diagnostic.
			o Suspension de l’abondement au fonds travaux
			Lorsque le montant du fonds de travaux atteint un montant supérieur au budget prévisionnel annuel « pour dépenses courantes » de l’article 14-1 , le syndic inscrit à l’ordre du jour de l’assemblée générale :
			- la question de l’élaboration du plan pluriannuel de travaux mentionné à l’article L731-2 du code de la construction et de l’habitation ;
			- La question de la suspension des cotisations au fonds de travaux en fonction des décisions prises par l’assemblée générale sur plan pluriannuel de travaux. En d’autres termes, si le montant des travaux inscrits au plan pluriannuel de travaux n’excède pas le montant du « fonds de travaux » d’ores et déjà constitué, les cotisations peuvent être suspendues.
			Cette dernière disposition constitue une forte incitation en faveur de la réalisation d’un diagnostic technique global et de l’élaboration d’un plan pluriannuel de travaux. D’une part, dès lors qu’il est nécessaire de cotiser au fonds travaux, et que cette cotisation se fait à fonds perdus, les copropriétaires ont tout intérêt à entreprendre dans un délai raisonnable les travaux de réhabilitation de l’immeuble qui leur permettront de valoriser leur lot. D’autre part, en l’absence d’un plan pluriannuel de travaux, le fonds devra continuer à être abondé.
		
		\subsubsection{2. Les ressources du Fonds de travaux}
		
			Le fonds est alimenté par une cotisation annuelle dont le montant est exprimé en pourcentage du budget prévisionnel et décidé par l’assemblée générale à la majorité de l’article 25 (et éventuellement 25-1 de la loi) sans pouvoir être inférieur à \pourcent{5}  du montant du budget annuel.
			Compte tenu du caractère obligatoire du fonds, la résolution soumise à l’assemblée générale ne doit pas porter sur la constitution du fonds, mais sur le seuil, en précisant que celui-ci ne doit pas être inférieur à \pourcent{5}. Il n’est pas souhaitable de remettre le taux d’appel au vote chaque année mais de préciser dans le projet initial de résolution qui fixe le taux que celui-ci est valable jusqu’à une nouvelle décision d’assemblée générale le modifiant.
			La cotisation sera appelée par quart chaque trimestre en même temps que la provision pour charges, et versée par les copropriétaires « selon les mêmes modalités que les provisions du budget prévisionnel. « Cette précision implique, selon nous, que le fonds de travaux doit être appelé en charges communes générales.
			Les cotisations versées par les copropriétaires sont attachées aux lots et définitivement acquises au syndicat des copropriétaires. Elles ne donnent pas lieu à remboursement par le syndicat à l’occasion de la cession du lot.
			Les cotisations dues pour ce fonds bénéficient des mêmes privilèges que les provisions sur charges courantes : elles sont incluses dans le privilège immobilier spécial du syndicat des copropriétaires (article 19-1 de la loi du 10 juillet 1965) et peuvent faire l’objet d’une déchéance du terme. Ainsi, il peut être demandé au copropriétaire défaillant de verser par anticipation la totalité de sa cotisation au fonds de travaux après mise en demeure par lettre recommandée avec accusé de réception restée infructueuse pendant plus de 30 jours, et ces sommes peuvent être recouvrées devant le juge statuant au fond comme en matière de référé (art. 19-1 de la Loi du 10 juillet 1965).
			Afin de protéger l’épargne ainsi constituée, il est fait obligation au syndic d’ouvrir un compte bancaire séparé au nom du syndicat, donc distinct du compte bancaire sur lequel sont versés les autres fonds, destiné à recevoir les cotisations du fonds de prévoyance ou du fonds de provisions pour travaux. Les intérêts de ce compte restent bien évidemment acquis au syndicat des copropriétaires.
			Le défaut d’ouverture de ce second compte séparé, est sanctionné par la nullité de plein droit du mandat du syndic dans un délai de 3 mois suivant sa désignation, et ce quand bien même l’assemblée ce serait opposée --– illégalement --- à la constitution du fonds !
		
		\subsubsection{3. L’utilisation du fonds de travaux}
		
			Le syndic ne peut naturellement décider de son propre chef de transférer ces fonds sur le compte courant du syndicat, ou de les affecter au règlement de factures courantes : il s’agirait d’un détournement de l’affectation des fonds entraînant sa responsabilité civile et pénale. Par ailleurs, l’assemblée générale elle-même ne pourrait pas décider d’une telle affectation, même à l’unanimité, puisqu’il est exclusivement prévu la possibilité pour elle, d’affecter ces sommes à la réalisation de travaux.
			Deux exceptions seulement sont prévues :
			- Dans le cas où le syndic en application de l’article 18 de la loi, a dans un cas d’urgence fait procéder de sa propre initiative à l’exécution de travaux nécessaires à la sauvegarde de l’immeuble, l’assemblée générale votant dans les conditions de majorité prévues aux articles 25 et 25-1, peut affecter tout ou partie des sommes déposées sur le fonds de travaux au financement de ces travaux ;
			- L’administrateur judiciaire désigné dans le cadre de l’article 29-1 de la loi (copropriété en difficulté), peut être autorisé par le juge « 2\degres Autoriser l’administrateur provisoire à utiliser les sommes déposées sur le fonds de travaux pour engager les actions nécessaires au redressement de la copropriété ou permettre le maintien de la gestion courante. » (art. 29-14 de la loi).
			L’Assemblée générale elle-même ne peut pas disposer du fonds de travaux à sa guise, son emploi est affecté
			- aux travaux obligatoires prescrits par les lois et règlements
			- aux travaux votés par l’assemblée générale autres que ceux de maintenance et définis par l’article 44 du décret
			- en cas d’urgence, pour les travaux nécessaires à la sauvegarde de l’immeuble sur décision de l’assemblée générale, à la majorité de l’art 25 (et 25-1), d’affecter à leur exécution tout ou partie des sommes déposées sur le fonds de travaux
			L’assemblée générale doit décider, simultanément avec le vote des travaux, à la majorité de l’article 25 ou éventuellement 25 -1, d’affecter tout ou partie des sommes déposées sur le fonds travaux pour financer les travaux prescrits par les lois et règlements ainsi que des travaux décidés en assemblée générale
			La loi \no 2018-1021 du 23 novembre 2018 dite ELAN a complété l’article 14-2 de la loi par une phrase ainsi rédigée : « Cette affectation doit tenir compte de l’existence de parties communes spéciales ou de clefs de répartition des charges. » (article 204 de la Loi ELAN).
			Cette précision était attendue par les syndics pratiquement depuis la création du fonds de travaux par la loi \no 2014-366 du 24 mars 2014 dite ALUR. Ils avaient, en effet, fait observer que dans certaines copropriétés les parties communes générales sont pratiquement réduites au sol, si bien que toutes les dépenses de travaux concernent réalité les parties communes spéciales de chaque bâtiment. Or, si le fond de travaux est affecté à la réalisation de travaux sur des parties communes spéciales, cela revient à faire participer à cette dépense, en violation de
			l’article 10 de la loi du 10 juillet 1965, les copropriétaires qui n’ont aucune quote-part indivise dans les parties communes. En effet, leur contribution est acquise au syndicat, elle aura été définitivement dépensée dans un bâtiment dont il n’avait pas à supporter les dépenses. La problématique était strictement la même pour les dépenses relatives aux éléments d’équipement, tel l’ascenseur : l’utilisation du fonds travaux pour son amélioration de sa réfection revenait à faire participer à la dépense copropriétaires du rez-de-chaussée, qui pourtant n’avait aucune utilité de cet élément d’équipement.
			Malheureusement, l’article 14-2 ne précise pas comment le syndic pourra, sur un plan comptable, tenir compte de l’existence de parties communes spéciales ou de clés de répartition des charges dans l’affectation du fonds de travaux.
	
	\subsection{C. Les provisions pour « travaux »}
	
		Ce sujet a été abordé dans la première partie du cours.
		Rappelons simplement que :
		- sauf s’il s’agit de travaux de maintenance (ie. d’une récurrence annuelle), les travaux ne peuvent être intégrés dans le budget courant de l’immeuble
		- leur financement doit être prévu par une résolution séparée qui détermine la date calendaire d’exigibilité des appels provisionnels pour travaux. C’est cette date qui rend les appels de fonds travaux exigible. Le syndic n’a donc pas le pouvoir de « suspendre les appels de fonds » - par exemple en cas de recours contre les travaux, étant précisé qu’un recours en annulation de la résolution votant les travaux n’est pas suspensif de la résolution votée
		- s’agissant d’un appel « provisionnel », les travaux doivent faire l’objet d’une reddition de comptes après achèvement, reddition de compte qui fera apparaitre une plus ou moins- value, à la charge ou au profit du copropriétaire à la date de l’approbation des comptes
		- la clef de répartition des charges travaux n’a pas en principe à être fixée par l’assemblée générale, car elle résulte de l’application du règlement de copropriété. Cependant, s’il s’agit de travaux d’amélioration, il faudra déterminer une clef pour la réalisation des travaux en fonction de l’avantage pour chaque lot, puis une autre clef pour les futurs travaux d’entretien, en fonction de l’utilité de l’élément d’équipement ( cf. supra)
		- Aucune règle n’impose au syndic de détenir \pourcent{100} des fonds avant la signature du marché, d’autant plus que la plupart des entreprises ne demandent au lancement du chantier qu’un acompte de 10 à \pourcent{30}. Pour autant, en cas de défaillance du Syndicat des Copropriétaires en cours de chantier, la responsabilité du syndic peut être engagée s’il a signé le marché alors qu’il existait un risque d’insuffisance des provisions. De la même façon, il peut être prudent de budgéter un pourcentage supplémentaire au montant des devis pour les imprévus de chantier.
		- Le payement de ces appels de fonds peut être échelonné sur 10 ans à la demande du copropriétaire qui n’a pas « donné leur accord à la décision », uniquement pour les travaux d’amélioration de l’article 30 (cf. supra), à condition qu’ils ne soient pas imposés par une obligation légale ou réglementaire (art. 33 de la Loi \no65-557 du 10 juillet 1965)
	
\section{Les travaux réalisés par les copropriétaires}

	\subsection{A . TRAVAUX SUR PARTIES PRIVATIVES}
	
		Le copropriétaire peut être amené à réaliser des travaux sur parties privatives, soit parce qu’il en a l’obligation soit parce qu’il le souhaite.
		Dans la première hypothèse, le Règlement de copropriété met souvent à la charge du copropriétaire l’obligation de maintenir en bon état d’entretien ses parties privatives (portes et fenêtres par exemple). Le non-respect de cette obligation permet au syndicat des copropriétaires de demander en justice (en référé ou au fond) la condamnation du copropriétaire à réaliser ces travaux d’entretien de ses parties privatives, voire même d’être autorisé à réaliser lui-même ces travaux aux frais du copropriétaire\footnote{28}.
		
		La seconde hypothèse est de loin la plus fréquente.
		En théorie, le copropriétaire peut réaliser sans autorisation préalable de l'assemblée générale tous travaux d'aménagement de ses parties privatives, dès lors qu'ils ne touchent ni aux parties communes de l'immeuble ni à l'aspect extérieur de celui-ci.
		Toutefois, le copropriétaire tout en réalisant des travaux qui ne concernent que les parties privatives peut indirectement porter atteinte aux parties communes ou aux droits des autres copropriétaires :
		- Par exemple, le copropriétaire pose du marbre dans son appartement : le poids excessif de ce revêtement de sol peut porter atteinte à la solidité de l'immeuble.
		- Par exemple, le remplacement d'une moquette par du carrelage peut également être source de nuisance pour les autres copropriétaires.
		- Par exemple lorsque ses travaux sont susceptibles de déstabiliser l’immeuble\footnote{29}
		- Par exemple enfin, le copropriétaire supprime une cloison privative qui, avec le temps, est devenue porteuse.
		
		Souvent, le règlement de copropriété limite le droit du copropriétaire à réaliser ce type d'aménagement sur parties privatives. De telles clauses sont justifiées par la destination de l'immeuble et doivent être respectées par le copropriétaire. À défaut, le syndicat des copropriétaires pourra contraindre le copropriétaire à restituer le revêtement antérieur.
		28 Cour d'Appel VERSAILLES, 14 novembre 2011 JurisData : 2011-026741
		29 Paris 23\degres Ch B, 22 mars 2007, oy 2007 \no 161 à propos de travaux mixtes - privatifs et sur parties communes - d’un immeuble construit sur un sous-sol fragile (rue Gabrielle sur la Butte Montmartre à Paris).
		Si par contre le règlement de copropriété est muet, syndicat des copropriétaires ne pourra pas s'opposer à la réalisation de travaux modificatifs sur parties privatives mais il pourra une fois ces travaux réalisés et s'ils entraînent des nuisances pour la copropriété demander au juge d'obliger le copropriétaire à remettre ses locaux dans leur état d'origine.
		De même la qualification de travaux n’est pas toujours évidente : il a été jugé notamment que l’ouverture de baies dans un mur mitoyen ne constituait pas seulement des travaux affectant les parties communes de l’immeuble mais constituait également une appropriation de parties communes\footnote{30}
	
	\subsection{B. TRAVAUX SUR PARTIES COMMUNES OU AFFECTANT L’ASPECT EXTERIEUR.}
	
		\subsubsection{1. Principe : l’autorisation préalable de l’assemblée générale est nécessaire.}
		
			En principe, un copropriétaire ne peut réaliser aucun travail affectant les parties communes ou l'aspect extérieur de l'immeuble sans autorisation préalable de l'assemblée générale.
			Les tribunaux se montrent le plus souvent très vigilant quant au respect de cette obligation posée par l'article 25 de la loi : quel que soit l'importance de ces travaux, dès lors qu'il s'empiète sur les parties communes il doit faire l'objet d'une autorisation préalable de la copropriété.
			Il en est ainsi de pose de câbles sur les parties communes alors même que l'emprise est minime.
			Il est vrai cependant qu'un arrêt de la 3e chambre civile en date du 19 novembre 199731 a admis qu'un copropriétaire ayant la jouissance exclusive d'un emplacement de parking était en droit d'installer un dispositif destiné à en interdire l'accès, sans autorisation préalable de la copropriété dès lors que ce dispositif avait un aspect discret par ses formes de cette dimension et qu'il était fixé dans le sol par un ancrage léger et superficiel. Il convient cependant d'accueillir cet arrêt avec réserves.
			De plus cette autorisation doit être nominative : c’est ce qui a été jugée à propos d’une autorisation donnée à des copropriétaires non dénommés d’effectuer des travaux non définis d’installation d’un ascenseur32. A notre sens si les travaux doivent être précisément définis, il suffit que les copropriétaires autorisés soient identifiables. Mais ne satisferait pas à cette exigence l’autorisation donnée aux copropriétaires « qui le désirent ». Une telle rédaction correspondant à une résolution de principe sans valeur décisionnelle.
		
		\subsubsection{2. Les travaux concernés.}
		
			\paragraph{a) Travaux affectant les parties communes.}
			
			30 Civ. 3\degres 10 sep. 2008, Administrer déc. 2008 p. 65
			31 publié dans Loyers et Copropriété de janvier 1998 note nº 27
			32 Civ. 3\degres 24 sep 2008, Administrer déc. 2008 p. 66
				Il peut s'agir le plus souvent de travaux réalisés à l'intérieur des parties privatives mais qui touche aux parties communes :
				- Création d'une trémie pour relier deux appartements situés l'un au-dessus de l'autre et qui appartienne au même copropriétaire,
				- création d'un vélux dans la toiture de l'immeuble,
				- branchement sur des canalisations communes pour l'installation d'un lavabo ou d'un bac à douche,
				- création de fenêtres en façade ou de porte sur un palier,
				- installation d’un conduit d’extraction au profit d’un restaurant
				- etc.
				Il peut s'agir également de travaux qui seront réalisés à l'extérieur de la partie privative et directement sur les parties communes :
				- Pose d'une grille à l'entrée d'une cour a jouissance privative,
				- installation d'une enseigne ou d'une plaque professionnelle,
				- aménagement privatif sur un palier commun,
				- création d'un élément d'équipement sur les parties communes (réalisation d'un ascenseur).
				- Des travaux d'accessibilité aux personnes handicapées ou à mobilité réduite
				- \etc
				Tous les travaux affectant les parties communes doivent être autorisés, quand bien même ceux-ci permettent un aménagement normal des parties privatives, ne portent pas atteinte à la solidité de l’immeuble ou à sa destination et ne réduisent pas l’usage des parties communes\footnote{33}.
			
			\paragraph{b) Travaux affectant l’aspect extérieur de l’immeuble.}
				
				Bien souvent des travaux qui affectent l'aspect extérieur de l'immeuble sont réalisés directement sur les parties communes de celui-ci.
				
				Il en est ainsi par exemple de la pose d'une enseigne sur la façade de l'immeuble : c'est installation impliquera des travaux sur la façade même qui auront pour conséquence d'en modifier l'aspect.
				
				Mais ces travaux affectant l'aspect extérieur de l'immeuble peuvent ne concerner que des parties privatives par exemple lorsqu'ils touchent aux devantures (création d'une porte dans la devanture) ou vitrines des magasins. Par exemple encore lorsque le copropriétaire réalise une cloison en limite de la place de parking dont il est propriétaire.
				
				Il convient également d'observer que si l'article 25 b) fait état de l'aspect extérieur de l'immeuble, cette disposition s'applique à tous travaux réalisés en dehors du lot privatif, par exemple sur un palier commun à l'intérieur même de l'immeuble ou encore l'aménagement d'une terrasse toiture qui ne pourra pas être vu ni les autres copropriétaires ni même des voisins.
			33 Civ. 3\degres Ch. 4 décembre 2007, Loyers et Copropriété fev 2008 \no 47
			\paragraph{c) La mise en place des IRVE et des locaux vélos}
			
				Le stationnement sécurisé des vélos bénéficie d’un traitement spécifique depuis la loi LOM : les travaux permettant de sécuriser le stationnement des vélos relèvent de l’article 24 s’ils ne portent pas atteinte à la structure, sinon de l’article 25 (supra).
				En revanche, le raccordement du véhicule électrique relève toujours de l’article 25
				25 j L'installation ou la modification des installations électriques intérieures permettant l'alimentation des emplacements de stationnement d'accès sécurisé à usage privatif pour permettre la recharge des véhicules électriques ou hybrides, ainsi que la réalisation des installations de recharge électrique permettant un comptage individuel pour ces mêmes véhicules ;
		
		\subsubsection{2. La procédure d’autorisation.}
		
			Le copropriétaire qui veut réaliser des travaux affectant les parties communes de l'immeuble où l'aspect extérieur de celui-ci doit préalablement solliciter l'autorisation de l'assemblée générale.
			
			\paragraph{a) Majorité requise : article 25, éventuellement 25-1, voire 24.}
			
				Cette autorisation sera donnée au copropriétaire demandeur à la majorité de l'article 25 de la loi, éventuellement au bénéfice des dispositions de l'article 25-1 (majorité absolue puis relative si le projet recueille en sa faveur le tiers des voix de tous les copropriétaires). La majorité de l’article 24 est applicable dans certains pour les travaux de sécurisation des vélos.
			
			\paragraph{b) Conditions auxquelles les autorisations peuvent être subordonnées.}
			
				Le syndicat des copropriétaires peut parfaitement subordonner cette autorisation à diverses conditions comme par exemple une augmentation des charges ou encore le respect de diverses modalités dans la réalisation des travaux. Bien évidemment ces conditions devront être justifiées par l'intérêt collectif.
			
			\paragraph{c) Exigences de l’Administration.}
			
				De plus, la réalisation de nouveaux travaux qui vont affecter les parties communes ou l'aspect extérieur de l'immeuble ne pourra être entreprise qu'après autorisation de l'administration : déclaration de travaux dispensés de permis de construire ou autorisation de voirie pour les enseignes.
				
				En application d’un arrêt du CE du 22 mars 1985 (dit Arrêt TALBOT), jusqu’en 2007, l'administration n'acceptait d'instruire les demandes d'autorisation de travaux que si le copropriétaire justifiait d'une autorisation préalable de l'assemblée générale. Depuis la réforme du permis de construire\footnote{Ordonnance n) 2005-1527 du 8 décembre 2005 entrée en application le 1er octobre 2007}, le copropriétaire n’a plus l’obligation de produire l’autorisation de l’assemblée générale mais « atteste avoir qualité pour demander la présente autorisation » (formulaires Cerfa de demandes d’autorisation). En cas de fausse déclaration le permis de
				construire peut être retiré au bénéficiaire même au-delà du délai normal de retrait par l’Administration de trois mois.
				Cette modification résulte du nouveau texte de l’art. R 431-35 du code de l’urbanisme.
				En application de ce texte, le CE, par arrêt en date du 15 février 2012 \no 333631, publié au Recueil Lebon, a considéré que le copropriétaire ayant attesté être autorisé par le propriétaire à exécuter les travaux, n’avait pas à produire avec sa demande de permis de construire une autorisation de l’assemblée générale.
				
				Relevons un attendu de principe très clair dans un arrêt du CE en date du 23 mars 2015, \no 348261 :
				\begin{quote}
					« Les déclarations préalables doivent seulement comporter, comme les demandes de permis de construire en vertu de l’article R4 131–5 du code de l’urbanisme, l’attestation du pétitionnaire qu’il remplit des conditions définies à l’article R423–1 ; que les autorisations d’utilisation du sol qui ont pour seul objet de s’assurer de la conformité des travaux qu’elles autorisent avec la législation et la réglementation d’urbanisme étant accordées sous réserve du droit des tiers, il n’appartient pas à l’autorité compétente de vérifier dans le cadre de l’instruction d’une déclaration ou d’une demande de permis, la validité de l’attestation établie par le demandeur.
					
					« Considérant toutefois que lorsque l’autorité saisie d’une telle déclaration ou d’une demande de permis de construire vient à disposer au moment où elle statue, sans avoir à procéder à une instruction lui permettant de les recueillir, d’informations de nature à établir son caractère frauduleux ou faisant apparaître, sans que cela puisse donner lieu à une contestation sérieuse, que le pétitionnaire ne dispose, contrairement à ce qu’implique l’article R423–1 du code de l’urbanisme, d’aucun droit à la déposer, il lui revient de s’opposer à la déclaration ou de refuser la demande de permis pour ce motif »
				\end{quote}
				
				En sorte que si le copropriétaire ne justifie d’aucune autorisation de l’assemblée générale pour la réalisation de ces travaux ou pire encore dans l’hypothèse où l’autorisation lui a été refusée, il suffira que le syndic ou même un copropriétaire informe l’agent instructeur de cette absence d’autorisation pour que le permis de construire soit refusé au copropriétaire.
				
				Plus grave, une récente décision de la cour administrative d’appel (CAA) de Paris en date du 11 avril 2019 a prononcé l’annulation d’un arrêté de permis de construire sur le fondement de l’article R. 423-1 du code de l’urbanisme (CAA Paris 11 avril 2019, req. \no 18PA01038). L’existence d’une fraude peut donc être caractérisée lorsque le pétitionnaire a déposé en toute connaissance de cause une demande de permis attestant de son autorisation à exécuter les travaux tout en ayant, au préalable, présenté sa demande d’autorisation à la copropriété et s’être trouvé confronté à un refus de celle-ci.
				
				Bien évidemment c’est le copropriétaire bénéficiaire des travaux qui doit solliciter les autorisations administratives nécessaires et non le syndic mandataire du syndicat des copropriétaires\footnote{35}.
			
			\paragraph{d) Inscription de la demande à l’ordre du jour}
				
				Le copropriétaire demandera au syndic d'inscrire à l'ordre du jour de la prochaine assemblée générale, ou d'une assemblée générale spécialement convoquée à cet effet en application des dispositions de l'article 8 du décret du 17 mars 67, l'autorisation de travaux dont il souhaite être le bénéficiaire.
				
				L’ordonnance du 30 octobre 2020 a voulu faciliter cette inscription en introduisant la possibilité, pour tout copropriétaire de « solliciter du syndic la convocation et la tenue, à ses frais, d'une assemblée générale pour faire inscrire à l'ordre du jour une ou plusieurs questions ne concernant que ses droits ou obligations » (art 17-1 AA nouveau).
				
				L'assemblée générale ne pourra valablement délibérer sur cette demande d'autorisation de travaux que si les dispositions de l'article 11 du décret ont été respectées : ce qui implique que le copropriétaire doit proposer un projet de résolution sur lequel l'assemblée devra délibérer.
				
				Le décret du 10 avril 2010 a complété l’article 10 du décret en ce sens :
				\begin{quote}
					Le ou les copropriétaires ou le conseil syndical qui demandent l’inscription d’une question à l’ordre du jour notifient au syndic, avec leur demande, le projet de résolution lorsque cette notification est requise en application des 7\degres et 8\degres du I de l’article 11. Lorsque le projet de résolution porte sur l’application du troisième alinéa de l’article 24 et du b de l’article 25 de la loi du 10 juillet 1965, il est accompagné d’un document précisant l’implantation et la consistance des travaux.
					
					A l’occasion de chaque appel de fonds qu’il adresse aux copropriétaires, le syndic rappelle les dispositions de l’alinéa précédent.
				\end{quote}
		
		\subsubsection{4. Autorisation judiciaire}
		
			Si l'assemblée générale des copropriétaires refuse l'autorisation demandée, le copropriétaire peut solliciter du juge qu'il se substitue à l'assemblée générale pour donner cette autorisation.
			
			\paragraph{a) Conditions de recevabilité de la demande.}
			
				Il était admis par les Tribunaux que l’action aux fins d’autorisation judiciaire devait être assimilée à une contestation judiciaire de la décision d’assemblée générale, donc introduite dans le délai de deux mois de l’article 42 de la loi \no 65557 du 10 juillet 1965.
				
				Mais un arrêt\footnote{36}, qualifié de principe, de la 3ème Chambre Civile de la cour de cassation en date du 16 novembre 2009 a rejeté un pourvoi contre un arrêt de la Cour de Lyon ayant refusé de faire application du délai de deux mois à la recevabilité de la demande, en un attendu dénué de toute ambiguïté :
				35 CE, 11 février 2015, \no 366296
				36 Civ. 3\degres Ch. 16 nov 2009, Pourvoi \no 09 12654, publié au Bulletin.
				\begin{quote}
					« Mais attendu que l'arrêt retient exactement que l'action des époux X..., introduite non pas pour contester la décision d'une assemblée générale mais pour obtenir une autorisation judiciaire d'exécuter les travaux projetés malgré le refus opposé, n'est pas soumise au délai de deux mois de l'article 42 alinéa 2 de la loi du 10 juillet 1965 »
				\end{quote}
			
				L'action n'est également recevable que dans la mesure où le copropriétaire est à même de justifier du refus de l'assemblée générale. Il ne pourra donc former sa demande devant le juge avant d'avoir présenté cette demande à l'assemblée générale et à la condition que l'assemblée générale ait refusée l'autorisation demandée. Il est vrai que dans certains cas, le refus de statuer par l'assemblée générale sous un fallacieux prétexte doit être assimilé à un refus pur et simple de l'autorisation sollicitée : il en sera ainsi lorsque l'assemblée générale prétendra à tort ne pas avoir obtenu l'ensemble des renseignements nécessaires pour prendre une décision en connaissance de cause. En d'autres termes, les atermoiements de l'assemblée générale constituent un véritable refus !
			
			\paragraph{b) Conditions de fond posées par la loi.}
			
				L'article 30 de la loi permet le recours au juge à deux conditions :
				\begin{itemize}
					\item d'une part que les travaux envisagés constituent une amélioration ;
					\item d'autre part qu'ils soient conformes à la destination de l'immeuble.
				\end{itemize}
				
				La jurisprudence ajoute deux conditions complémentaires :
				\begin{itemize}
					\item le copropriétaire doit avoir des droits sur les parties communes concernées ;
					\item les travaux ne doivent pas avoir été réalisés.
				\end{itemize}
				
				\subparagraph{L'amélioration.}
				Aux termes d'une jurisprudence constante, l'amélioration s'entend au profit du copropriétaire demandeur. Il n'est pas nécessaire qu'elle bénéficie à l'immeuble tout entier\footnote{ civ. 3\degres 22 mars 1983 Administrer décembre 1983 p. 45.}
				
				\subparagraph{Conformité à la destination de l'immeuble.}
				Le tribunal appréciera cas par cas cette conformité au regard du règlement de copropriété, des caractéristiques physiques et de la situation de l'immeuble.
				
				\subparagraph{Les droits sur les parties communes.}
				Un copropriétaire ne peut exiger de la copropriété et en conséquence après refus de l'assemblée générale obtenir du juge la réalisation de travaux privatifs sur des parties communes dont il ne serait pas le propriétaire indivis : ce point a été jugé par un arrêt de la cour de cassation\footnote{Civ 3e chambre du 21 février 1978,J.C.P. 1979 II 19 149} dans les circonstances suivantes : le copropriétaire du sixième étage
				demandait l'autorisation de prolonger l'ascenseur qui ne desservait que les cinq premiers étages. Le règlement de copropriété précisait que l'ascenseur était une partie commune spéciale au copropriétaire des cinq premiers étages. Le copropriétaire du sixième étage n'avait donc aucun droit sur cet ascenseur il ne pouvait imposer par le biais d'une décision de justice la réalisation de travaux de prolongation de l'ascenseur jusqu'au sixième étage.
				
				\subparagraph{Les travaux ne doivent pas avoir été réalisés.}
				Si le copropriétaire, malgré le refus de l'assemblée générale, ou avant même d'avoir sollicité l'autorisation refusée par cette assemblée, a entrepris les travaux, le juge considère alors que sa demande est devenue sans objet. Cette jurisprudence est non seulement constante mais également particulièrement sévère : l'autorisation sera refusée alors même que les travaux ont seulement débuté.
				Cette jurisprudence s'explique également par le fait que lorsque le juge se substitue à l'assemblée générale pour autoriser les travaux, qui détermine les conditions auxquelles les travaux seront réalisés. Le juge mis devant le fait accompli ne peut donc plus remplir la mission qui lui est confiée par l'article 30 alinéa 4 de la loi.
				
				\subparagraph{Les travaux dont l’autorisation est demandée au juge peuvent ne pas être rigoureusement identiques à ceux soumis à l’approbation de l’assemblée générale.}
				Il arrive en effet que le projet évolue entre la demande présentée à l’assemblée générale et la demande présentée au juge, parfois même cette évolution peut se produire en cause d’appel, par exemple pour tenir compte des observations du syndicat des copropriétaires ou du jugement.
				
				Aux termes d’un arrêt\footnote{Civ 3\degres Ch. 4 juin 2014, pourvoi: 13-15400, Publié au bulletin} en date du 4 juin 2014 il a été jugé par la cour de cassation que l'article 30, alinéa 4, de la loi \no 65-557 du 10 juillet 1965 n'imposait pas que les travaux soumis au tribunal pour autorisation soient rigoureusement identiques à ceux refusés par l'assemblée générale, le projet pouvant comporter des évolutions limitées destinées à tenir compte des critiques du syndicat des copropriétaires sans toutefois porter sur un projet différent.
				
				Bien évidemment il s’agira là d’une question de fait et cette jurisprudence ne pourra pas être invoquée en cas de modification d’un élément essentiel du projet.
			
			\paragraph{c) L’autorisation du juge}
			
				Le plus souvent, le juge n'accordera l'autorisation sollicitée qu'après avoir obtenu l'avis d'un technicien désigné en qualité d'expert.
				
				Il donnera mission à l'expert de vérifier que les travaux ne sont pas susceptibles de porter atteinte à la solidité de l'immeuble, qu'ils ne risquent pas d'en compromettre l'esthétique. L'expert sera également chargé de s'assurer que les travaux envisagés ne sont pas disproportionnés et de rechercher éventuellement si le même résultat ne peut pas être atteint par la réalisation de travaux de moindre ampleur ou plus judicieusement placés.
				
				Enfin, et cela est expressément prévu par le texte de la loi, lorsque le juge donne son autorisation au copropriétaire il détermine les conditions auxquelles ces travaux vont être exécutés. Il peut en effet imposer au copropriétaire diverses modifications à ses projets d’origine ou soumettre la réalisation des travaux à diverses conditions. Bien évidemment, le copropriétaire pourra refuser ces conditions en renonçant au bénéfice des autorisations obtenues.
			
			\paragraph{d) Le raccordement des autres copropriétaires.}
				
				Bien souvent, l’autorisation demandée par le copropriétaire ne peut pas bénéficie aux autres : le percement d'une trémie entre deux appartements permettra la réalisation d'un duplex entre deux parties privatives et on voit mal comment les autres copropriétaires pourraient bénéficier de cette amélioration.
				
				Mais il arrive également que le copropriétaire sollicite l'autorisation de réaliser des travaux sur parties communes dont les autres copropriétaires pourront effectivement avoir l'utilisation ultérieure : ce sera le cas par exemple d'un copropriétaire ou de plusieurs copropriétaires qui auront sollicité de l'assemblée générale, puis après refus de cette dernière du juge l'autorisation d'installer un ascenseur dans la cage d'escalier de l'immeuble.
				
				Aussi le juge va-t-il fixer les conditions dans lesquelles les autres copropriétaires pourront utiliser les installations ainsi réalisées.
				\begin{quote}
					Article 30 alinéa 4 de la Loi \no65-557 du 10 juillet 1965
					
					Lorsque l'assemblée générale refuse l'autorisation prévue à l'article 25 b, tout copropriétaire ou groupe de copropriétaires peut être autorisé par le tribunal de grande instance à exécuter, aux conditions fixées par le tribunal, tous travaux d'amélioration visés à l'alinéa 1er ci-dessus ; le tribunal fixe en outre les conditions dans lesquelles les autres copropriétaires pourront utiliser les installations ainsi réalisées. Lorsqu'il est possible d'en réserver l'usage à ceux des copropriétaires qui les ont exécutées, les autres copropriétaires ne pourront être autorisés à les utiliser qu'en versant leur quote-part du coût de ces installations, évalué à la date où cette faculté est exercée.
				\end{quote}
				
				Se pose alors la question de la qualification des ouvrages ainsi réalisés : par application de l’article 3 modifié de la loi \no 65-557 du 10 juillet 1965, tel que modifié par la loi \no 2018-1021 du 23 novembre 2018 dite ELAN, il pourrait s’agir de parties communes (spéciales), puisqu’est réputé parties communes « tout élément incorporé dans les parties communes »
		
		\subsubsection{5. Le sort des travaux réalisés sans autorisation.}
		
			Lorsqu’un copropriétaire réalise des travaux nécessitant l’autorisation préalable de l'assemblée générale sans avoir sollicité cette autorisation, il commet une infraction au contrat (les dispositions de l’article 25-b sont reprises et la sanction de l’art. 1143 code civil lui est applicable : le syndicat des copropriétaires est en droit de demander au juge la condamnation du copropriétaire à remettre les lieux en leur état antérieur. Cette demande peut être faite également par un autre copropriétaire sans avoir à justifier d’un quelconque préjudice (cf. Chapitre sur les Actions du Syndicat des Copropriétaires).
			
			\paragraph{a) La démolition peut être demandée quelle que soit la gravité de
			l’atteinte aux parties communes.}
			
				Traditionnellement, la cour de cassation considérait que la réparation en nature, donc la remise en état pouvait toujours être exigée, même en l’absence d’atteinte à la solidité l’immeuble et étaient conformes à sa destination et à l’usage des parties communes par les autres copropriétaires\footnote{Civ 3\degres 4 déc 2007, \no 06-19.931, Loyers et Copropriété 2008 \no 47, p 28, note Vigneron.}.
				
				Toutefois, deux arrêts récents ont marqué une évolution
				\begin{enumerate}
					\item \textbf{Cass. 3e civ., 8 juin 2017, \no 16-16.677, F-D (pourvoi c/ CA Montpellier, 3 nov. 2015)}
					
					La cour d’appel a pu, dans le cadre de son pouvoir souverain d’appréciation, considérer que la remise en état était impossible dès lors que cela créerait un risque grave de fragiliser à nouveau la structure de l’immeuble, et rejeter par conséquent la demande du syndicat des copropriétaires.
					
					\item \textbf{Cass. 3e civ. 15 février 2018 \no 16-17.759 F-P-B, Sté Maison, au Bulletin}
					
					« la seule mesure nécessaire et \textbf{proportionnée à la cessation du trouble} était la remise en l'état des lieux »
				\end{enumerate}
				
				La démolition peut cependant toujours être exigée en cas d’empiètement (Cass. 3\degres civ 21 dec 2017 \no 16-25406 et civ.3ème 17 mai 2018 (16-15.792)
			
			\paragraph{b) La demande peut être formulée devant le juge du fonds ou devant le juge des référés.}
			
				Cette demande pourra être faite devant le juge du fond ou directement devant le juge des référés sur le fondement de l’article 809 du NCPC aux termes duquel le juge peut faire cesser un trouble manifestement illicite.
				
				Pour autant, si le juge du fond a l’obligation de faire droit à la demande de remise en état quelle que soit la gravité de la faute invoquée (sauf impossibilité matérielle ou juridique), par contre le juge des référés semble avoir un plus large pouvoir d’appréciation et peut décider que cette remise en état ne s’impose que si la faute invoquée présente un certain degré de gravité.
				
				C’est par exemple ce qu’a jugé la cour de Paris\footnote{Paris 14ème ch 1er sep 2005, Loyers et Copropriété 2006, comm 22} à propos d’aménagements réalisés par un copropriétaire sur une terrasse à jouissance privative (création de jardins suspendus !), en infraction avec les dispositions du Règlement de copropriété : en l’occurrence, alors même que le copropriétaire s’était vu refuser a-posteriori la régularisation par l'assemblée générale, le juge des référés a considéré que ces aménagements ne portaient pas à atteinte à l’harmonie extérieure de l’immeuble.
				
				Cependant, il a été jugé en sens inverse que les travaux effectués par un copropriétaire sur des parties communes (en l’espèce un conduit d’extraction pour commerce), sans autorisation de l’assemblée générale, constituent un trouble manifestement illicite, même s’ils sont exigés par des services administratifs, et autorisés par le syndic (Cour de cassation, Chambre civile 3, 22 mars 2018, 17-10.053, Inédit )
			
			\paragraph{c) La demande de remise en état doit être introduite dans le délai de prescription de cinq ans.}
			
				Par ailleurs, dans la mesure où ces travaux affectent seulement l’aspect extérieur de l’immeuble ou touche aux parties communes, le droit d’agir du syndicat des copropriétaires se prescrit par dix ans de la date à laquelle les travaux irréguliers ont été achevés\footnote{Civ 3\degres ch 7 mars 1990, IRC 1990 p. 136 ; Civ 3\degres 18 dec 2001 – Administrer mai 2002 p. 40} (délai réduit à cinq ans à par la loi \no 2018-1021 du 23 novembre 2018 dite loi \no 2018-1021 du 23 novembre 2018 dite ELAN).
				
				Cette nouvelle rédaction n'est accompagnée d'aucune mesure transitoire mais il est possible d'appliquer l'article 2222, alinéa 2, du Code civil qui prévoit qu'en cas de réduction du délai de prescription, le nouveau délai court du jour de l'entrée en vigueur de la loi, sans que la durée totale puisse excéder la durée prévue par la loi antérieure. Ainsi, les faits qui ne sont pas encore prescrits par application de l'ancien article 42, alinéa 1, le seront le 23 novembre 2023, sauf pour les actions atteintes par la prescription de 10 ans avant cette date, lesquelles s'éteindront à la date initialement prévue\footnote{(A. Lebatteux, L'amélioration des modalités de gestion de la copropriété par la loi \no 2018-1021 du 23 novembre 2018 dite Élan : Loyers et copr. févr. 2019, n \degres 2, p. 28).}.
				
				Bien évidemment il incombe au copropriétaire de démontrer cet achèvement par des documents objectifs (factures et règlements effectués à l’entreprise) ou des témoignages conformes aux prescriptions du code civil en matière de preuve.
				Le point de départ du délai de prescription est le jour où le syndicat des copropriétaires a connu ou aurait dû connaître les faits lui permettant d'agir (CA Paris, 4 juill. 2014, \no 12/16352 : JurisData \no 2014-015853 ; AJDI 2015, p. 125, obs. S. Porcheron).
				
				Si par contre les travaux irréguliers ont non seulement touché aux parties communes mais ont également eu pour conséquence d’absorber une partie commune ou d’utiliser le droit de construire, droit accessoire aux parties communes, la prescription de l’action est de trente ans : il s’agit alors d’une action de « droit réel » et non plus une action « personnelle »\footnote{Cass. Civ III : 11.1.89.}.
				
				La distinction n’est pas toujours évidente : tous travaux réalisés sur les parties communes a pour conséquence d’accaparer cette partie commune ; pour autant la cour de cassation fait bien la distinction pour considérer que des travaux, quelle que soit leur importance n’emportent pas « appropriation » des parties communes\footnote{Cass. Civ III : 22.10.08}.
				
				En l’espèce il s’agissait d’une installation de climatisation faite par un copropriétaire sur la toiture de l’immeuble partie commune dont le syndicat des copropriétaires demandait la dépose plus de dix ans après leur réalisation. Pour échapper à la prescription de dix ans il invoquait l’appropriation des parties communes du fait de ces travaux.
				
				La Cour de cassation rejette les prétentions du syndicat des copropriétaires en relevant que ces travaux n’ont pas conféré un caractère privatif au toit partie commune.
				
				Mais nous avons vu par ailleurs que l’ouverture de baies dans un mur mitoyen avait été jugée comme constitutive d’annexion de parties communes (Civ. 3\degres 10 sep. 2008) en sorte que l’action en remise en état a été déclarée recevable plus de dix ans après réalisation de ces travaux

		
		\subsubsection{6. Les travaux PMR}
		
			Avant l’ordonnance du 30 octobre 2020, les travaux d’accessibilité « PMR » relevaient d’une autorisation de l’assemblée générale, mais la majorité était abaissée, par faveur, à l’article 24
			\begin{quote}
				« e) L’autorisation donnée à certains copropriétaires d’effectuer, à leurs frais, des travaux d’accessibilité aux personnes handicapées ou à mobilité réduite qui affectent les parties communes ou l’aspect extérieur de l’immeuble et conformes à la destination de celui-ci, sous réserve que ces travaux n’affectent pas la structure de l’immeuble ou ses éléments d’équipement essentiels ;
			\end{quote}
			
			Cette disposition a toutefois été substantiellement modifiée par l’ordonnance du 30 oct. 2019. Le législateur a entendu réécrire le dispositif en créant un nouvel article 25-2 :
			\begin{quote}
				« Chaque copropriétaire peut faire réaliser, à ses frais, des travaux pour l'accessibilité des logements aux personnes handicapées ou à mobilité réduite qui affectent les parties communes ou l'aspect extérieur de l'immeuble. A cette fin, le copropriétaire notifie au syndic une demande d'inscription d'un point d'information à l'ordre du jour de la prochaine assemblée générale, accompagnée d'un descriptif détaillé des travaux envisagés.
				
				Jusqu'à la réception des travaux, le copropriétaire exerce les pouvoirs du maître d'ouvrage.
				
				L'assemblée générale peut, à la majorité des voix des copropriétaires, s'opposer à la réalisation de ces travaux par décision motivée par l'atteinte portée par les travaux à la structure de l'immeuble ou à ses éléments d'équipements essentiels, ou leur non-conformité à la destination de l'immeuble. »
			\end{quote}
			
			Ainsi, l’ancien système qui prévoyait une autorisation préalable nécessaire disparaît. Tout copropriétaire aura ainsi le pouvoir, librement exerçable, d’agir sur les parties communes, à charge, seulement, de notifier au syndic une demande d’inscription d’un « point d’information » à l’ordre du jour de la prochaine assemblée générale, accompagnée d’un descriptif détaillé des travaux envisagés\footnote{Cette mesure reprend la substance du projet de loi du 2 février 2016 qui n’avait pas été suivi d’effet ; Ce descriptif détaillé des travaux envisagés devraient être de même nature, et donc du même degré de précision, que celui exigé pour la réalisation des travaux de l’article 30.}.
			
			Le législateur a toutefois prévu un mécanisme d’opposition à ces travaux, par l’assemblée générale. Celle-ci devra, si elle entend s’y opposer, en prendre la décision motivée à la majorité des voix des copropriétaires.
			
			Ce mécanisme d’opposition permet de révéler que la notification faite par le copropriétaire soit celle d’un projet et non de travaux réalisés et qu’ainsi, s’il n’a pas à être « autorisé » il ne doit pas être « empêché ». Ainsi le « point d’information » est en réalité une question devant permettre à l’assemblée décider de s’opposer, ou non, au projet qui lui est soumis. L’assemblée ne pourra toutefois s’opposer qu’en raison d’une véritable justification, explicitement motivée, qui ne pourra être d’ordre simplement esthétique. En effet, pour être recevable, l’opposition devra, d’après le texte, être justifiée par l’atteinte portée à la structure de l’immeuble, ses
			éléments d’équipements essentiels ou par la non-conformité des travaux à la destination de l’immeuble.
			
			Ainsi, les article 10-2 et 10-3 modifiés par le décret 2020-xx disposent que
			\begin{quote}
				Article 10-2
				
				Pour l’application de l’article 25-2 de la loi du 10 juillet 1965, le syndic inscrit à l’ordre du jour de la même assemblée générale :
				\begin{itemize}
					\item le point d’information relatif aux travaux d’accessibilité ;
					\item la question de l’opposition éventuelle à la réalisation de ces travaux par décision motivée de l’assemblée générale.
				\end{itemize}
			\end{quote}
			\begin{quote}
				Article 10-3
				
				En l’absence d’opposition motivée de l’assemblée générale dans les conditions prévues au troisième alinéa de l’article 25-2 de la loi du 10 juillet 1965, le copropriétaire peut faire réaliser les travaux conformément au descriptif détaillé présenté à l’assemblée générale, à l’expiration du délai de deux mois mentionné au deuxième alinéa de l’article 42 de la loi du 10 juillet 1965.
			\end{quote}
			
			Si les travaux sont réalisés le copropriétaire assurera la maîtrise d’ouvrage jusqu’à la réception des travaux. Ces travaux, s’ils sont réalisés sur les parties communes seront donc, par incorporation, une propriété commune. Ceci implique que l’entretien des éléments d’équipements ou des éléments de structures nouveaux seront à la charge du syndicat des copropriétaires.
			
			Il semble alors que la mesure pourrait poser des difficultés, par exemple en cas d’installation d’un ascenseur répondant aux normes d’accessibilité aux personnes handicapées qui ne desservira qu’un étage. Il faudra alors modifier le règlement de copropriété afin de prévoir la répartition des charges de cet élément d’équipement, ce que le texte ne prévoit pas. Un mécanisme devrait être permis pour faciliter cette modification à défaut de quoi des problématiques importantes se poseront lorsque le syndicat, faute de pouvoir empêcher les travaux, refusera de modifier le règlement pour ne pas prendre en charge financièrement ces éléments dont un seul aurait l’utilité. »\footnote{P.-e. Lagraulet, « La prise de décision au sein de la copropriété », Lexbase Hebdo éd. priv. \no 806, 12 déc. 2019, N1578BY4}.
			
			L’autre difficulté est l’absence, en l’état actuel du texte, de renvoi à la procédure d’autorisation judiciaire de travaux, en cas « d’opposition » injustifiée : il semble donc que le copropriétaire devra faire « annuler » la résolution d’opposition, dans le délai de l’article 42, et non solliciter une autorisation judiciaire.
	
\section{Les travaux spécifiques}

	\subsection{SOUS SECTION I . LES TRAVAUX DE RENOVATION ENERGETIQUES}
	
	La loi du 12 juillet 2010 dite « Grenelle II » comporte un volet « copropriété ». Les Décrets d’application de ces textes ont été publiés en 2012.
	
	Par la suite, la loi TCEV est venue compléter et renforcer ce dispositif.
	
	L’idée est d’imposer aux copropriétés équipées d’une installation collective de chauffage ou de refroidissement :
	\begin{enumerate}
		\item De faire faire une étude, soit un Diagnostic de Performance Energetique (DPE), soit un Audit Energétique, et désormais un Diagnotic Technique Global (DTG).
		
		\item Après réalisation de cette Etude de soumettre aux copropriétaires l’adoption d’un contrat de performance énergétique (CPE) et/ou d’un Plan de travaux d’économie d’énergie.
		
		\item De voter enfin des travaux dits d’économie d’énergie.
	\end{enumerate}
	
	Nous reprendrons ces étapes successives.
	
	\subsubsection{A. Auditer l’immeuble}
	
		\paragraph{1. Le DPE ou l’audit énergétique sur parties communes est obligatoire, même si cette obligation n’est pas assortie de sanction.}
		
			\begin{quote}
				Article L.134-4-1 du CCH :
				
				« Un diagnostic de performance énergétique est réalisé pour les bâtiments équipés d'une installation collective de chauffage ou de refroidissement dans un délai de cinq ans à compter du 1er janvier 2012.
				
				Les bâtiments à usage principal d'habitation en copropriété de cinquante lots ou plus, équipés d'une installation collective de chauffage ou de refroidissement, et dont la date de dépôt de la demande de permis de construire est antérieure au 1er juin 2001, sont exemptés de la disposition de l'alinéa précédent.
				
				Dans ces bâtiments, un audit énergétique doit être réalisé. Le contenu et les modalités de réalisation de cet audit sont définis par décret en Conseil d'État. »
			\end{quote}
			
			Pour les immeubles en copropriété, le syndic doit donc inscrire à l’ordre du jour de l’assemblée générale la question de la réalisation, selon le cas, d’un DPE ou d’un audit énergétique.
			
			Le syndic a également l’obligation de poser en assemblée générale la question de la réalisation d’un DTG (Diagnostic Technique Global), mais la réalisation de celui-ci n’est pas obligatoire. Si une DTG est réalisé, il inclura nécessairement l’audit énergétique ou le DPE.
			
			Ce DPE/audit énergétique bénéficie d’une large diffusion
			- Affichage de la performance énergétique dans les annonces de vente ou de location à compter du 1er janvier 2011
			- Mise à disposition du DPE en cas de vente du lot à tout candidat acquéreur
			Le vendeur a l’obligation de remettre systématiquement à un candidat acquéreur ou locataire une copie du DPE. (Nouvel art. L.134-3 al 2 du Code de la construction et de l’habitation).
			- Annexion du DPE à tout contrat de location (art L.134-3-1 du CCH)
			Depuis la loi \no 2018-1021 du 23 novembre 2018 dite ELAN, le DPE (en principe intégré au DTG n’est plus remis « aux fins d’information », même si seules ses conclusions (et non les recommandations) sont opposables (L 134-3-1 du CCH). L’entrée en vigueur a été repoussée au 1er janvier 2021 "afin de laisser le temps nécessaire au plan de fiabilisation des diagnostics engagé par le gouvernement de produire tous ses effets". L'opposabilité ne concernera que le diagnostic et non les recommandations qui l'accompagnent.
			Précisons enfin qu’aux termes de l’article L. 134-4-2 le DPE doit être transmis à l’Agence de l’environnement et être mis à disposition de l’Agence nationale de l’habitat (Anah).
		
		\paragraph{2. Le diagnostic de performance énergétique}
		
			Le DPE (art L.134 et suivants du CCH) a pour objet de sensibiliser et d’informer l’occupant sur les consommations d’énergie, les coûts, les GES. Établi par un diagnostiqueur indépendant et valable 10 ans jusqu’à présent, il se traduit par un classement de l’immeuble en 5 catégories sur le plan de la consommation énergétique et des GES selon différents critères (source d’énergie, niveau d’isolation
			
			Dans les immeubles (en copropriété ou non) équipés d’un dispositif commun de chauffage ou de refroidissement, un diagnostic de performance énergétique devait être réalisé à compter du 1er janvier 2012 et dans un délai de 5 ans.
			
			L’article 179 de la loi \no 2018-1021 du 23 novembre 2018 portant évolution du logement, de l’aménagement et du numérique dite loi ELAN a rendu opposables, à partir du 1er janvier 2021 les diagnostics de performance énergétiques (DPE) annexés aux transactions et baux immobiliers. Jusqu’alors, ces diagnostics étaient fournis uniquement à des fins d’information. En dehors des recommandations de travaux qui garderont une valeur informative, le contenu des DPE réalisés à partir du 1er janvier 2021 aura donc la même opposabilité que les autres diagnostics fournis lors d’une mutation immobilière.
			
			En outre,
			
			Le Décret \no 2012-1342 du 3 décembre 2012 avait précisé les modalités de vote du DPE lors d’une première assemblée puis les modalités de présentation du diagnostic réalisé lors de l’assemblée suivante. Ce décret est profondément remanié par le décret \no 2020-xx modifiant les articles R134-2 et suivants du CCH, pour accompagner l’opposabilité du DPE à partir du 1er jancier 2021
			
			\subparagraph{a) Champs D’application}
			
				Le DPE est obligatoire, selon l’article L. 134-4-1 pour les bâtiments collectifs équipés de système de chauffage ou de refroidissement collectifs à l’exception des copropriétés de plus de 50 lots antérieurs à 2001, ces dernières devant réaliser un audit. Le DPE doit être établi pour l’immeuble, et pour le lot, avant la vente ou la location de celui-ci.
				
				Toutefois, si un DPE « volontaire » est réalisé, il devra répondre aux conditions prévues par le décret
			
			\subparagraph{b) Informations à transmettre au diagnostiqueur}
			
				Lorsque le diagnostic de performance énergétique porte sur un bâtiment ou une partie d'un bâtiment qui bénéficie d'un dispositif collectif de chauffage, de refroidissement ou de production d'eau chaude, le syndic fournit à la personne qui demande le diagnostic de performance énergétique sur un bâtiment ou une partie de bâtiment et aux frais de cette dernière :
				
				I a) La quantité annuelle d'énergie consommée pour ce bâtiment ou cette partie de bâtiment par le dispositif collectif ;
				
				b) Le calcul ou les modalités ayant conduit à la détermination de cette quantité à partir de la quantité totale d'énergie consommée par le dispositif collectif ;
				
				c) Une description des installations équipements collectifsves de chauffage, de refroidissement, ou de production d'eau chaude sanitaire, de ventilation, de leurs auxiliaires et de leur mode de gestion ;.
				
				d) Les modalités de répartition des frais liés aux consommations énergétiques de ces équipements.
				
				II. Dans la mesure où ces informations sont à sa disposition, pour les dispositifs collectifs passifs, tels l’enveloppe extérieure, la toiture, les planchers, plafonds et cloisons intérieures donnant sur des locaux non chauffés, leurs caractéristiques pertinentes ayant des incidences sur les consommations énergétiques.
			
			\subparagraph{c) Contenu}
			
			Le diagnostic de performance énergétique comprend :
			a. Les caractéristiques pertinentes du bâtiment ou de la partie de bâtiment et un descriptif de ses équipements de chauffage, de production d'eau chaude sanitaire, de refroidissement, de ventilation et, dans certains types de bâtiments, de l'éclairage intégré des locaux en indiquant, pour chaque catégorie d'équipements, les conditions de leur utilisation et de leur gestion ayant des incidences sur les consommations énergétiques ;
			b. L'indication, pour chaque catégorie d'équipements, de la quantité annuelle d'énergie consommée ou estimée selon une méthode de calcul conventionnel ainsi qu'une évaluation des dépenses annuelles résultant de ces consommations
			c. L'évaluation de la quantité d'émissions de gaz à effet de serre liée à la quantité annuelle d'énergie consommée ou estimée ;
			d. Une information sur les énergies d'origine renouvelable produites par les équipements installés à demeure et utilisées dans le bâtiment ou partie de bâtiment en cause ;
			e. Le classement du bâtiment ou de la partie de bâtiment en application d'une échelle de référence, prenant en compte la zone climatique et l’altitude, établie en fonction de la quantité annuelle d'énergie consommée ou estimée, pour le chauffage, le refroidissement, la production d'eau chaude sanitaire, l’éclairage et les auxiliaires de chauffage, de refroidissement, d'eau chaude sanitaire et de ventilation, rapportée à la surface du bâtiment ou de la partie du bâtiment ;
			f. Le classement du bâtiment ou de la partie de bâtiment en application d'une échelle de référence, prenant en compte la zone climatique et l’altitude, établie en fonction de la quantité d’émission de gaz à effet de serre, pour le chauffage, le refroidissement, la production d'eau chaude sanitaire, l’éclairage et les auxiliaires de chauffage, de refroidissement, d'eau chaude sanitaire et de ventilation, rapportée à la surface du bâtiment ou de la partie du bâtiment ;
			g. Des recommandations visant à améliorer la performance énergétique du bâtiment ou de la partie de bâtiment, accompagnées d'une évaluation de leur coût et de leur efficacité ;
			h. Le cas échéant, le dernier document en date mentionné à l'article R. 224-33 ou R. 224-41-8 du code de l'environnement.
			i. Des éléments d’appréciation sur la capacité du bâtiment ou de la partie de bâtiment à assurer un confort thermique en période estivale
			Le nouveau contenu du DPE « opposable » se caractérise, par rapport à l’ancien, par une simplification sur certains points ( il n’y a plus d’évaluation de la quantité d’énergie produite par des ressources renouvelables, l’information était peu fiable et peu lisible), par des compléments (prise en compte de l’énergie consommée par la ventilation, appréciation du « confort estival », prise en compte de la « zone climatique et de l’altitude »).
			d) DPE de l’immeuble et DPE du lot
			Selon le nouveau décret, le DPE de l’immeuble ne peut plus valoir d’office « DPE pour le lot »
			Néanmoins, le DPE collectif peut générer, dans certaines conditions, un DPE individuel « par défaut » dont les résultats seront cohérents avec un DPE réalisé uniquement à l’échelle du lot ou partie de bâtiment.
			e) Adoption du DPE « collectif » (R 134-4-3)
			Les dispositions, sur ce point, n’ont pas changé : Le syndic de copropriété inscrit à l'ordre du jour de l'assemblée générale des copropriétaires la décision de réaliser le diagnostic de performance énergétique. Il inscrit à l'ordre du jour de l'assemblée générale des copropriétaires qui suit cette réalisation la présentation du diagnostic par la personne en charge de sa réalisation. Ce document, qui comporte des explications détaillées, mentionne également les hypothèses de travail et les éventuelles approximations auxquelles il a donné lieu.
			Ce diagnostic vaut diagnostic de performance énergétique au sens des articles L. 134-1 à L. 134-4 pour chacun des lots.
			III. ― Les syndicats de copropriétaires ayant déjà fait réaliser un diagnostic de performance énergétique toujours en cours de validité et conforme aux exigences du I ne sont pas soumis à l'obligation de réaliser un nouveau diagnostic. « Dans le cas où un syndicat de copropriétaires a fait réaliser un diagnostic de performance énergétique toujours en cours de validité mais non conforme aux exigences du I, celui-ci est complété en vue de le rendre conforme à celles-ci. »
			Pour les DPE antérieurs au décret du xx -2020, ils ont cependant une durée de validité réduite, car ils n’étaient pas, lorsqu’ils ont été établis, « pleinement opposables ». (avant 2017 : caducité en 2023, 2018 : fin 2024).
			Par ailleurs, le DPE au lot devenant un document, non seulement obligatoire, mais opposable en cas de vente du lot, l’absence de réalisation d’un DPE sur l’immeuble, interdisant la réalisation d’un DPE fiable lot par lot, pourrait engager le syndicat des copropriétaires.
			f) Information des acquéreurs et locataires
			Depuis 2011, les annonces de vente et de location doivent comporter, de façon lisible, l’indication de la classe énergétique.
			A partir du 1er janvier 2022, cette information devra être complétée par une « estimation des coûts annuels d’énergie », suivie suivie d’une indication sur l’année de référence des prix de l’énergie utilisés pour établir cette estimation, indication qui figurera aussi dans le contrat de bail d’habitation (pour information)
			g) Interdiction des locations et mise en vente des logements relevant des catégories e et f
			La disposition principale de l’article 22 (1\degres du I) de la L. 2019-1147 du 8 nov. 2019 relative à l’énergie et au climat oblige les logements à ne pas excéder, à compter de 2028 ou 2033 le seuil de 330 kilowattheures d’énergie primaire par mètre carré et par an. Une obligation de d’affichage de l’obligation est instaurée dans les annonces immobilières (« logement à consommation énergétique excessive)
			Le nouvel article R. 134-5-3-1 traduit cette obligation de mention dans les différentes typologies d’annonce (presse écrite, presse en ligne et affichage en agence immobilière).
			Dans ce cas, le DPE au lot devra se transformer en véritable « audit énergétique », selon l’article L. 134-3-1 du CCH modifié par la L. 2019-1147 du 8 nov. 2019:
			« Dans le cas des logements qui ont une consommation énergétique primaire supérieure ou égale à 331 kilowattheures par mètre carré et par an, le diagnostic de performance énergétique mentionné au premier alinéa du présent article comprend également un audit énergétique.
			L'audit énergétique présente notamment des propositions de travaux dont l'une au moins permet d'atteindre un très haut niveau de performance énergétique du bâtiment et une autre au moins permet d'atteindre un niveau de consommation en énergie primaire inférieur à 331 kilowattheures par mètre carré et par an. Il mentionne à titre indicatif l'impact théorique des travaux proposés sur la facture d'énergie. Il fournit des ordres de grandeur des coûts associés à ces travaux et mentionne l'existence d'aides publiques destinées aux travaux d'amélioration de la performance énergétique.
			Le contenu de l'audit énergétique est défini par arrêté. »
		
		\paragraph{3. L’audit énergétique}
		
		Ces logements énergivore rejoindront ainsi les immeubles en copropriété à usage principal d’habitation de plus de 50 lots (la loi ne précise pas s’il s’agit des lots principaux ou de tous les lots) dont le dépôt de la demande de permis de construire est antérieur au 1er janvier 2001, emporte l’obligation est de réaliser un véritable audit énergétique.
		
		Le décret du 27 janvier 2012 a précisé le champ d’application et le contenu de l’audit énergétique
		\begin{quote}
			Art. R. 134-14. CCH
			
			« Dans les bâtiments à usage principal d’habitation d’un immeuble ou d’un groupe d’immeubles en copropriété de cinquante lots ou plus, quelle que soit l’affectation des lots, équipés d’une installation collective de chauffage ou de refroidissement et dont la date de dépôt de la demande de permis de construire est antérieure au 1er juin 2001, le syndic de copropriété inscrit à l’ordre du jour de l’assemblée générale des copropriétaires la réalisation d’un audit énergétique conformément aux dispositions des articles »
		\end{quote}
		
		Il résulte de la lettre de ce texte que les 50 lots doivent être comptés quelle que soit l'affectation de ceux-ci :
		- qu'il s'agisse de lots à affectation commerciale, tant que le bâtiment reste à usage principal d'habitation,
		- qu'il s'agisse d'un lot principal (appartement) ou d'un lot accessoire (caves parking).
		L'audit énergétique se généralise aux copropriétés « moyennes » (17 lots principaux), et les nombreux syndics qui avaient fait réaliser des diagnostics énergétiques pour les copropriétés de moins de 50 lots principaux doivent reprendre leur copie à zéro, d'autant plus qu'une simple actualisation ne paraît pas envisageable.
		En effet, le décret précise que les copropriétés qui ont fait réaliser un audit énergétique depuis moins de cinq ans doivent mettre cet audit à jour pour correspondre aux critères posés par le décret, mais n'envisage pas la possibilité de transformer un diagnostic de performance énergétique en audit compte tenu de l'écart qualitatif entre les deux.
		L’audit énergétique doit comprendre au minimum :
		a) un descriptif des parties communes et privatives ;
		b) une enquête auprès des occupants et des propriétaires non-occupants visant à évaluer leurs consommations énergétiques, leur confort thermique, l’utilisation et la gestion de leurs équipements et leurs attentes ;
		c) la visite d’un échantillon de logements sous réserve d’obtenir l’accord des occupants concernés.
		Le propriétaire de l’installation collective de chauffage ou de refroidissement (le syndicat des copropriétaires), son mandataire ou le syndic doit par ailleurs fournir à l’auditeur un certain nombre d’informations, à savoir :
		- la quantité annuelle d’énergie consommée pour la copropriété par ladite installation et, le cas échéant, la production d’eau chaude sanitaire ;
		- les documents en sa possession relatifs aux installations collectives de chauffage, de refroidissement ou de production d’eau chaude sanitaire et à leur mode de gestion ;
		- les contrats d’exploitation, de maintenance, d’entretien et d’approvisionnement en énergie ;
		- le dernier rapport de contrôle périodique de la chaudière ;
		- tout autre document nécessaire à l’établissement de l’audit dont la liste sera fixée par arrêté.
		Afin d’optimiser l’utilisation, l’exploitation et la gestion des équipements collectifs et notamment de l’installation collective de chauffage ou de refroidissement, l’audit énergétique comprend une liste de préconisations, en examinant leur faisabilité technique, juridique (autorisation administratives) et financières (plan de financement et mobilisation des aides), selon 3 grands schémas : maximal, centré sur les travaux à bref retour sur investissement (5 ans), ou a minima, c’est-à-dire les travaux indispensables.
		
		\paragraph{4. Le diagnostic technique global ou DTG (loi ALUR, entrée en vigueur au 1er janvier 2017)}
		
		La loi ALUR reprend le dispositif du diagnostic de performance énergétique mais élargit l’objet de ce diagnostic à l’ensemble du bâti, au-delà des perspectives énergétiques. En effet, pour que la programmation soit efficace, il faut qu’elle inclut la totalité des dépenses auxquelles les
		copropriétaires devront faire face dans les années à venir : il est irréaliste de prévoir la réhabilitation énergétique sur 5 ans si, dans le même intervalle, les copropriétaires doit faire face à des travaux très onéreux concernant par exemple la mise en conformité des ascenseurs ou le changement des canalisations en plomb.
		Pour autant, les dispositions concernant le diagnostic de performance énergétique restent en vigueur, et ne seront pas abrogées même après le décret définissant le contenu du diagnostic technique global.
		Il est simplement précisé dans l’article 24-4 de la loi du 10 juillet 1965, relatif au plan pluriannuel de travaux qui doit être adoptés à la suite de l’audit, que doit être soumis au vote un plan pluriannuel de travaux à l’issue du DPE ou de l’audit énergétique, mais que :
		« L’obligation prévue au premier alinéa est satisfaite si le plan pluriannuel de travaux, inscrit à l’ordre du jour de l’assemblée générale en application de l’article L. 731-2 du code de la construction et de l’habitation, comporte des travaux d’économie d’énergie. ».
		Ainsi les DTG vont progressivement absorber le DPE et l’audit énergétique.
		
		Contenu du DTG
		
		Aux termes de l’article L.731-1 du code de la construction et de l’habitation :
		\begin{quote}
			« afin d’assurer l’information des copropriétaires sur la situation générale de l’immeuble et, le cas échéant, aux fins d’élaboration d’un plan pluriannuel de travaux, l’assemblée générale des copropriétaires se prononce sur la question de faire réaliser par un tiers, disposant de compétences précisées par décret un diagnostic technique global pour tout immeuble à destination partielle ou totale d’habitation relevant du statut de la copropriété ».
		\end{quote}
		
		Ce diagnostic comportera :
		o Une analyse de l’état apparent des parties communes et équipements communs
		o Un état de la situation du syndicat des copropriétaires par rapport aux obligations légales et réglementaires de construction et d’habitation
		o Analyse des améliorations possibles de la gestion technique et patrimoniale de l’immeuble
		o Un DPE ou un audit énergétique de l’immeuble
		Ce diagnostic technique global doit faire apparaitre une évaluation sommaire du coût et une liste des travaux nécessaires à la conservation de l’immeuble en précisant ceux devant être menés dans les dix prochaines années.
		Ainsi malgré son intitulé, ce diagnostic ne sera pas que technique : il comportera également une analyse juridique par rapport aux normes existantes (2), et une appréciation sur la gestion technique (3) !
		Communication du DTG
		Le diagnostic technique global fera l’objet d’une large diffusion :
		- il sera porté à la connaissance des acquéreurs de lots de copropriétés, en même temps que le plan pluriannuel de travaux (article L. 721-2 CCH nouveau)
		- il devra être communiqué aux administrations (L. 731-5 CCH) qui en feront la demande, « pour vérifier l'état de bon usage et de sécurité des parties communes d'un immeuble collectif à usage principal d'habitation soumis au statut de la copropriété présentant des désordres potentiels », dans le cadre des procédures aboutissant à un arrêté d’insalubrité (Article L. 1331-26 du Code de la Santé Publique), à un arrêté de péril (Article L. 129-1 CCH), ou à une injonction relative à un bâtiment menaçant ruine (Article L. 511-1 CCH). Curieusement, le texte ne prévoit pas cette communication dans le cadre d’une procédure de carence engagée contre la copropriété. (L. 615-6 du CCH). À défaut de production du D.T.G dans le délai d’un mois, l’administration pourra faire réaliser elle-même le diagnostic en lieu et place du syndicat des copropriétaires et à ses frais
		- Les travaux dont la réalisation apparaît nécessaire aux termes du diagnostic seront intégrés au carnet d'entretien selon des modalités à préciser par décret. (L. 731-3CCH)
		- Les données essentielles relatives au bâti issues du DTG figureront au registre d’immatriculation des copropriétés (art L. 711-2 CCH).
		Champ d’application et date d’entrée en vigueur
		L’obligation de soumettre aux assemblée la question de la réalisation d’un DTG est entrée en vigueur le 1er janvier 2017.
		Contrairement à l’option prise pour l’audit énergétique, il n’a pas été instauré de « diagnostic allégé » pour les petites copropriétés de moins de 50 lots, toutefois pour ces dernières la partie « énergétique » du DTG sera valablement composé par un simple diagnostic de performance énergétique.
		Par ailleurs, l’article L. 731-1 du CCH ne prévoit pas pour le moment d’obligation de réaliser le D.T.G ou de sanctions en cas de non-réalisation (hormis la possibilité, pour les administrations, au syndicat des copropriétaires pour le faire réaliser.) Il est simplement prévu que l’assemblée générale « l'assemblée générale des copropriétaires se prononce sur la question de faire réaliser par un tiers, disposant de compétences précisées par décret, un diagnostic technique global pour tout immeuble à destination partielle ou totale d'habitation relevant du statut de la copropriété. » La décision est adoptée à la majorité relative de l’article 24 du 10 juillet 1965. Toutefois, ce caractère non coercitif est largement illusoire : d’une part il est probable que le législateur finira par un partir un délai de réalisation de ces D.T.G aux copropriétés, comme il l’a fait pour le diagnostic de performance énergétique, et d’autre part la réalisation de ce diagnostic est le seul moyen pour les copropriétés de fixer un montant maximal au « fond travaux » créé par ailleurs.
		Par ailleurs, selon l’article L. 731-5 II, relativement sévère, puisqu’à défaut de production de ce diagnostic dans un délai d'un mois après notification de la demande, l'autorité administrative peut faire réaliser d'office le diagnostic en lieu et place du syndicat des copropriétaires et à ses frais. » ;
		les décrets d’application
		Deux décrets étaient attendus : l’un pour déterminer les compétences du « tiers » diagnostiqueur, l’autre pour déterminer les modalités d’intégration du DTG au carnet
		d’entretien. Le tout devant être publié avant le 1er janvier 2017, date d’entrée en vigueur de ces nouvelles dispositions. Le gouvernement a respecté cette « dead line » en publiant un décret unique sur ces deux points le 28 décembre 2016.
		(i) Les compétences requises du diagnostiqueur.
		Le Décret ne confie pas le DTG à une profession déterminée (architecte par exemple) mais précise que les diagnostiqueurs doivent justifier d’un certain nombre de compétences … et pas des moindres :
		- Compétences techniques sur le mode de construction, les matériaux et éléments d’équipement
		- Compétences juridiques sur la terminologie et la connaissance des textes législatifs et réglementaires traitant de l’habitat et de la construction ainsi que sur les règles sanitaires et le droit de la copropriété
		- Compétences comptables sur la gestion financière des copropriétés
		Ces diagnostiqueurs, qu’ils établissent le DTG pendant la vie de la copropriété ou avant la mise en copropriété, doivent être indépendants : du syndic comme des entreprises et ne peuvent recevoir aucune rétribution des entreprises pour la réalisation des travaux (c’est à tout le moins une règle déontologique qui s’impose à tout conseil que celui-ci dépende ou non d’un ordre national). Ces exigences s’appliquent tant aux « patrons » qu’aux employés qui font les diagnostiques.
		Notons bizarrement que ces professionnels n’ont pas l’obligation de souscrire une assurance Responsabilité civile puisque le texte du Décret précise : « S'il a souscrit une assurance responsabilité civile professionnelle lui permettant de couvrir les conséquences d'un engagement de sa responsabilité en raison de ses interventions, il en justifie au syndicat des copropriétaires, et au conseil syndical s'il existe.
		L’exigence envers le CONSEIL SYNDICAL se comprend dans la mesure où le plus souvent ce diagnostiqueur sera soumis à concurrence en application de l’article 21 de la loi \no65-557 du 10 juillet 1965.
		Le DTG s’intègre au Carnet d’entretien.
		Aux termes du I de l’article 18 de la loi \no65-557 du 10 juillet 1965 : « le syndic est chargé (…) - d’établir et de tenir à jour et à disposition des copropriétaires un carnet d’entretien de l’immeuble conformément à un contenu défini par décret ».
		Ces dispositions font l’objet de deux articles du Décret du 17 mars 1967, l’article 4-4 qui permet au candidat acquéreur d’avoir connaissance du Carnet d’entretien et l’article 33 qui oblige le syndic à détenir le carnet d’entretien et d’en remettre copie aux copropriétaires qui lui en font la demande (disposition qui s’applique également aux diagnostics établis par le syndicat).:
		Le contenu du carnet d’entretien est fixé par le Décret \no 2001-477 du 30 mai 2001, articles 3 à 5
		Article 3 .- Le carnet d'entretien mentionne :
		l'adresse de l'immeuble pour lequel il est établi ;
		l'identité du syndic en exercice ;
		les références des contrats d'assurances de l'immeuble souscrits par le syndicat des copropriétaires, ainsi que la date d'échéance de ces contrats.
		Article 4 .- Le carnet d'entretien indique également :
		l'année de réalisation des travaux importants, tels que le ravalement des façades, la réfection des toitures, le remplacement de l'ascenseur, de la chaudière ou des canalisations, ainsi que l'identité des entreprises ayant réalisé ces travaux ;
		- la référence des contrats d'assurance dommages ouvrage souscrits pour le compte du syndicat des copropriétaires, dont la garantie est en cours ;
		- s'ils existent, les références des contrats d'entretien et de maintenance des équipements communs, ainsi que la date d'échéance de ces contrats ;
		- si le diagnostic technique global existe, la liste des travaux jugés nécessaires à la conservation de l'immeuble en précisant les équipements ou éléments du bâtiment concernés par ces travaux ainsi que l'échéancier recommandé ; (ajouté par le décret \no 2016-1965 du 28 déc. 2016)
		- s'il existe, l'échéancier du programme pluriannuel de travaux décidé par l'assemblée générale des copropriétaires.
		Article 5 .- Le carnet d'entretien peut en outre, sur décision de l'assemblée générale des copropriétaires, contenir des informations complémentaires portant sur l'immeuble, telles que celles relatives à sa construction ou celles relatives aux études techniques réalisées.
		
		\subsubsection{B. Programmer les travaux et les finances : CPE et Plan pluriannuel de travaux}
		
		Art 24-4 de la loi du 10 juillet 1965 créé par la loi 12 juillet 2010 - art. 7
		« Pour tout immeuble équipé d'une installation collective de chauffage ou de refroidissement, le syndic inscrit à l'ordre du jour de l'assemblée générale des copropriétaires qui suit l'établissement d'un diagnostic de performance énergétique prévu à l'article L. 134-1 du code de la construction et de l'habitation ou d'un audit énergétique prévu à l'article L. 134-4-1 du même code la question d'un plan de travaux d'économies d'énergie ou d'un contrat de performance énergétique.
		Avant de soumettre au vote de l'assemblée générale un projet de conclusion d'un tel contrat, le syndic procède à une mise en concurrence de plusieurs prestataires et recueille l'avis du conseil syndical. Un décret en Conseil d'Etat fixe les conditions d'application du présent article. »
		L’indicatif traduit une obligation pour le syndic de poser la question… sans toutefois que l’assemblée ait l’obligation de voter en faveur de la résolution
		
		\paragraph{1. le CPE (contrat de performance énergétique)}
		
		Lorsque l’Audit ou le CPE a été réalisé le syndic devra soit faire voter un Plan d’économie d’énergie (cf. ci-après), soit faire voter un Contrat de Performance énergétique.
		Objet du CPE
		L’'article 3 de la directive européenne 2006/32/CE du 5 avril 2006 Note 16, définit le CPE comme « un accord contractuel entre le bénéficiaire et le fournisseur d'une mesure visant à améliorer l'efficacité énergétique, selon lequel des investissements dans cette mesure sont consentis afin de parvenir à un niveau d'amélioration de l'efficacité énergétique qui est contractuellement défini »
		Il s’agit de substituer aux contrats d’exploitation classiques (P1 à P3) qui reposent sur des obligations de moyens, des contrats formulés en termes d’objectifs énergétiques, donc en termes d’obligations de résultat : un objectif est fixé en termes de DJU, et le prestataire conserve à sa charge les dépassements, sauf circonstances exceptionnelles (climat), mais partage en revanche les bénéfices liés aux économies réalisées si les consommations sont inférieures aux objectifs.
		Dans le cadre de ces contrats, les coûts d’investissement peuvent être répercutés pendant la durée d’exploitation de l’installation. Ces contrats obligent le professionnel à assurer, outre la mise en place, l’entretien et l’équilibrage fin de l’installation pendant toute la durée du contrat.
		Modalités d’adoption du CPE
		L’adoption du CPE se fera dans les conditions suivantes :
		- Réception du DPE ou de l’Audit par le syndic.
		- Mise en concurrence des Prestataires de services
		- Convocation de l’assemblée générale en joignant à la convocation le DPE ou l’Audit ainsi que les devis des Prestataires consultés
		- Vote par les copropriétaires du Contrat de Performance Energétique
		L’article 24-4 ne précise pas à quelle majorité ce CPE doit être adopté, mais on peut penser que ce texte étant placé à la suite de l’article 24 et avant l’article 25, il était de l’intention du législateur de faire adopter ce contrat à la majorité de l’article 24.
		2.1.Le plan pluriannuel
		Vote relatif au plan pluriannuel de travaux après établissement du DTG
		Une fois le diagnostic établi, celui-ci est présenté à la première assemblée générale qui suit sa réalisation ou sa révision (article L.731-2 du CCH). Selon ce texte :
		« Le syndic inscrit à l'ordre du jour de cette assemblée générale la question de l'élaboration d'un plan pluriannuel de travaux ainsi que les modalités générales de son éventuelle mise en oeuvre.
		II. ― Au regard des orientations décidées par les copropriétaires lors des assemblées générales précédentes, le syndic inscrit à l'ordre du jour de chaque assemblée générale
		soit la question de l'élaboration d'un plan pluriannuel de travaux, soit les décisions relatives à la mise en oeuvre du plan pluriannuel de travaux précédemment adopté. »
		Il faut comprendre que l’assemblée générale sera appelée à se prononcer, au cours d’assemblées générales successives, d’abord sur l’adoption (ou non) d’un plan pluriannuel de travaux, qui n’est qu’un outil de phasage et de financement, puis, au fur et à mesure de la « mise en oeuvre » du plan, sur les travaux eux même, avec le vote d’un budget actualisé et affiné.
		La majorité à laquelle le plan pluriannuel de travaux est la majorité de l’article 24 de la loi du 10 juillet 1965, comme pour l’adoption du plan pluriannuel de travaux de réhabilitation énergétique. En revanche, la mise en oeuvre du plan travaux relèvera, selon les cas, de l’article 24 ou 25 :
		- En fonction des majorités requises pour l’adoption des travaux
		- En fonction du mode de financement, car la décision d’affecter le « fond travaux » à la réalisation des travaux prescrits par les lois et les règlements ainsi que des travaux décidés en assemblée générale relève de l’article 25 ou 25-1 de la loi du 10 juillet 1965.
		Rôle et définition du plan pluriannuel de travaux
		Le plan pluri annuel a d’abord été conçu comme une modalité de constitution d’une épargne permettant de « lisser » les coûts de travaux lourds sur le bâti. Il avait en effet été introduit par le décret comptable du 14 mars 2005, à l’article 35-4\degres du décret du 17 mars 1967 qui dispose : « le syndic peut exiger le versement des avances correspondant à l’échéancier prévu dans le plan pluriannuel de travaux adoptés par l’assemblée générale ». Il a été généralisé par la suite avec la loi Grenelle II, l’audit énergétique devant être suivi d’une proposition d’adoption d’un plan pluriannuel .
		Ce plan pluriannuel permet de hiérarchiser les travaux à réaliser, de les programmer sur plusieurs années, et de constituer progressivement l’épargne nécessaire à leur réalisation. Par ailleurs, le plan pluriannuel servira de « butée » au fonds travaux : une fois que celui-ci attendra une année de budget « courant », l’abondement au fonds travaux ne sera plus obligatoire qu’à hauteur de l’échéancier fixé dans le plan pluriannuel (cf infra).
		Les sommes appelées au titre de l’échéancier du plan pluriannuel constituent des avances remboursables en cas de vente, il est donc impératif de prévoir que ces sommes seront reconstituées sur l’acquéreur. De même, il faut prévoir que ces sommes seront versées sur un compte rémunéré48, distinct du compte de la copropriété pour éviter tout « évaporation » des économies réalisées dans le payement des charges courantes.
		Modalités d’adoption et d’exécution du Plan de travaux d’économie d’énergie.
		Le Décret précité \no 2012-1342 du 3 décembre 2012 précise les conditions d’adoption … et d’exécution, du plan de travaux d’économie d’énergie ou du Contrat de performance énergétique. Il comprend des dispositions particulières à la réalisation de travaux d’intérêt collectif réalisés sur les parties privatives (article R 138-1 CCH)
		4848 NB : une loi du 1er juillet 2010 a autorisé l’ouverture du livret A au nom d’un syndicat de copropriétaire. (art. 221-3 du code monétaire et financier)
		- A réception du diagnostic ou de l’audit, le syndic prépare un plan de travaux et fait un appel à la concurrence.
		- Il convoque l’assemblée générale en joignant le diagnostic (DPE ou rapport de synthèse de l’audit énergétique ou DTG) à la convocation de l’assemblée générale ; il joint également les devis.
		- Sur la base des devis l’assemblée générale vote sur le principe d’un plan de travaux ou d’un contrat de performance énergétique.
		- Si l’assemblée générale décide d’adopter un plan de travaux elle doit procéder à un vote distinct selon que ces travaux portent sur les parties communes ou sur les parties privatives
		- A quelle majorité sera adopté le plan d’économie d’énergie ? Son adoption est prévue à l’article 24-4, ce qui permet de penser que la majorité applicable est celle de l’article 24.
		Notons enfin qu’à l’occasion de l’ordonnance de 2019, « En lien avec l’objet étendu du syndicat des copropriétaires le législateur avait initialement, d’après les rapports qui ont pu en être faits\footnote{49}, précisé le contenu du diagnostic technique global (DTG)\footnote{50} et avait considérablement développé l’obligation, embryonnaire jusqu’alors, du plan pluriannuel de travaux (PPT)\footnote{51}.
		Ce redéploiement des mesures existantes, qui étaient mises en corrélation avec le dispositif du fonds de travaux modifié en conséquence, devait s’imposer comme étant l’outil nécessaire et indispensable à la rénovation des immeubles en copropriété. Il était en somme prévu, d’après les sources précitées, qu’un plan pluriannuel de travaux soit obligatoirement élaboré par un professionnel disposant des mêmes compétences que celui pouvant réaliser le DTG. Ce plan devait ensuite être soumis à l’assemblée générale qui aurait eu la faculté de le mettre en oeuvre ou non. Le montant du fonds de travaux devait enfin être établi selon le montant des travaux proposé par le plan, avec un plancher de 2,5% de ce montant ou, à défaut de plan, d’un montant minimal par lot de copropriété fixé par décret, sans qu’il ne soit plus possible d’y déroger.
		Il n’en sera finalement rien puisque, contre toute attente\footnote{52}, ces mesures ont disparu au dernier moment ! ».\footnote{53}
		Le contenu du plan d’économies d’énergie sur parties communes et sur parties privatives :
		49 Toute la presse généraliste annonçait ces mesures selon une dépêche de l’AFP.
		50 V\degres sur ce diagnostic : P. Lebatteux et A. Lebatteux, « Les diagnostiqueurs et la modification corrélative du contenu du carnet d’entretien », Loyers et copr. mars 2017, \no 3, alerte 20 ; N. Figuière-Brocard, « Le diagnostic technique global : du principe à la pratique », IRC janv. 2017, p. 14.
		51 Cette mesure était assez similaire à un projet présenté en avril 2019 et actuellement débattu au Québec : projet de loi \no 16 « visant principalement l’encadrement des inspections en bâtiment et de la copropriété divise, le remplacement de la dénomination de la Régie du logement et l’amélioration de ses règles de fonctionnement et modifiant la Loi sur la Société d’habitation du Québec et diverses dispositions législatives concernant le domaine municipal ». L’article 33 du projet prévoit ainsi que « Tous les cinq ans, le syndicat obtient une étude du fonds de prévoyance établissant les sommes nécessaires pour que ce fonds soit suffisant pour couvrir le coût estimatif des réparations majeures et de remplacement des parties communes. Cette étude est réalisée par un membre d’un ordre professionnel déterminé par règlement du gouvernement. Ce règlement peut aussi déterminer le contenu et les modalités de l’étude. Les sommes à verser au fonds de prévoyance sont fixées sur la base des recommandations formulées à l’étude du fonds de prévoyance et en tenant compte de l’évolution de la copropriété, notamment des montants disponibles au fonds de prévoyance ».
		52 V\degres en ce sens le communiqué de presse de la FNAIM du 31 octobre, précit.
		53 P.e. Lagraulet, « L’administration du syndicat réformée », AJDI, précit.
		« Art. R. 138-2 (CCH) - Le plan de travaux d'économies d'énergie comprend : « I. ― Des travaux d'amélioration de la performance énergétique correspondant à une ou plusieurs des actions figurant dans l'une ou l'autre des deux catégories suivantes : « 1\degres Travaux portant sur les parties et équipements communs : « a) Travaux d'isolation thermique performants des toitures ; « b) Travaux d'isolation thermique performants des murs donnant sur l'extérieur ou sur des locaux non chauffés ; « c) Travaux d'isolation thermique performants des parois vitrées donnant sur l'extérieur ; « d) Travaux d'amélioration des installations d'éclairage des parties communes, « e) Travaux d'installation, de régulation, d'équilibrage ou de remplacement des systèmes de chauffage, de refroidissement ou d'eau chaude sanitaire ; « f) Travaux d'isolation des réseaux collectifs de chauffage, de refroidissement ou d'eau chaude sanitaire ; « g) Travaux de régulation ou de remplacement des émetteurs de chaleur ou de froid ; « h) Travaux d'amélioration ou d'installation des équipements collectifs de ventilation ; « i) Travaux d'installation d'équipements de chauffage, de refroidissement ou de production d'eau chaude sanitaire utilisant une source d'énergie renouvelable ; « 2\degres Travaux d'intérêt collectif portant sur les parties privatives : « a) Travaux d'isolation thermique des parois vitrées donnant sur l'extérieur comprenant, le cas échéant, l'installation de systèmes d'occultation extérieurs ; « b) Pose ou remplacement d'organes de régulation ou d'équilibrage sur les émetteurs de chaleur ou de froid ; « c) Equilibrage des émetteurs de chaleur ou de froid ; « d) Mise en place d'équipements de comptage des quantités d'énergies consommées. « II. ― Un programme détaillé indiquant l'année prévisionnelle de réalisation des travaux et leur durée. « III. ― Une évaluation du coût des travaux prévus au plan, fondée sur les devis issus de la consultation d'entreprises.
		On notera que cette liste répond à la définition de l’article 25 g et donne ainsi la liste des travaux d’économie d’énergie qui pourront être votés à la majorité de l’article 25 (on eut seulement souhaité que le Décret ne se contente pas de nous dire que cette liste correspond au contenu du plan d’économie d’énergie mais eut précisé qu’il s’agissait là de la liste des travaux pouvant être adoptés à la majorité de l’article 25 g).
		On notera enfin que le Plan de Travaux peut porter à la fois sur des travaux à réaliser sur les parties communes mais également sur des travaux à réaliser sur les parties privatives.
		
		\subsubsection{C. Décider les travaux : la majorité applicable}
		
		\paragraph{1. En principe, les travaux d’économie d’énergie relèvent de l’article 25 f (vote avec passerelle)}
		
		« f) Les travaux d'économies d'énergie ou de réduction des émissions de gaz à effet de serre. Ces travaux peuvent comprendre des travaux d'intérêt collectif réalisés sur les parties privatives et aux frais du copropriétaire du lot concerné, sauf dans le cas où ce dernier est en mesure de produire la preuve de la réalisation de travaux équivalents dans les dix années précédentes.
		Un décret en Conseil d'Etat précise les conditions d'application du présent f.» (avant l’ordonnance le texte était celui de l’article 25 g)
		La liste des travaux relevant de cette majorité est donnée par l’Art. R. 138-2 (CCH)
		« 2\degres Travaux d'intérêt collectif portant sur les parties privatives : « a) Travaux d'isolation thermique des parois vitrées donnant sur l'extérieur comprenant, le cas échéant, l'installation de systèmes d'occultation extérieurs ; « b) Pose ou remplacement d'organes de régulation ou d'équilibrage sur les émetteurs de chaleur ou de froid ; « c) Equilibrage des émetteurs de chaleur ou de froid ; « d) Mise en place d'équipements de comptage des quantités d'énergies consommées. »
		2. Les « travaux embarqués »
		La loi du 17 août 2015 a réécrit l’article L 111-10 du Code de la Construction et de l'Habitation en imposant à tout propriétaire les travaux favorisant une meilleure isolation des bâtiments qui doivent être mis en oeuvre en même temps que les travaux de ravalement ou de réfection de toiture, sous réserve que ces travaux soient réalisables techniquement et juridiquement et qu’il n’existe pas une disproportion manifeste entre ses avantages et ses inconvénients de nature technique, économique ou architecturale.
		Ces travaux ont été précisés par :
		- le Décret \no 2016-711 du 30 mai 2016 ajoutant au Code de la Construction et de l'Habitation les articles R. 131-28-7 (pour le ravalement de plus de 50 % d’une façade) et R. 131-28-8 (pour les travaux emplacement ou le recouvrement d'au moins 50 % de l'ensemble de la couverture).
		- L’article R 131-28-9 qui précise les cas où les dispositions des articles R. 131-28-7 et R. 131-28-8 ne sont pas applicables
		1\degres Il existe un risque de pathologie du bâti liée à tout type d'isolation. Le maître d'ouvrage justifie du risque technique encouru en produisant une note argumentée rédigée par un homme de l'art sous sa responsabilité ; 2\degres Les travaux d'isolation ne sont pas conformes à des servitudes ou aux dispositions législatives et réglementaires relatives au droit des sols, au droit de propriété ou à l'aspect des façades et à leur implantation ;
		3\degres Les travaux d'isolation entraînent des modifications de l'aspect de la construction en contradiction avec les prescriptions prévues pour les secteurs sauvegardés, les aires de mise en valeur de l'architecture et du patrimoine, les abords des monuments historiques, les sites inscrits et classés, ou avec les règles et prescriptions définies en application des articles L. 151-18 et L. 151-19 du code de l'urbanisme ;
		4\degres Il existe une disproportion manifeste entre les avantages de l'isolation et ses inconvénients de nature technique, économique ou architecturale, les améliorations apportées par cette isolation ayant un impact négatif trop important en termes de qualité de l'usage et de l'exploitation du bâtiment, de modification de l'aspect extérieur du bâtiment au regard de sa qualité architecturale, ou de surcoût.
		I.- Sont réputées relever de la disproportion manifeste au sens du 4\degres du I les situations suivantes :
		1\degres Une isolation par l'extérieur dégraderait significativement la qualité architecturale. Le maître d'ouvrage justifie de la valeur patrimoniale ou architecturale de la façade et de la dégradation encourue, en produisant une note argumentée rédigée par un professionnel mentionné à l'article 2 de la loi \no 77-2 du 3 janvier 1977 sur l'architecture ; 2\degres Le temps de retour sur investissement du surcoût induit par l'ajout d'une isolation, déduction faite des aides financières publiques, est supérieur à dix ans. L'assiette prise en compte pour calculer ce surcoût comprend, outre le coût des travaux d'isolation, l'ensemble des coûts induits par l'ajout d'une isolation. L'évaluation du temps de retour sur investissement s'appuie sur une méthode de calcul de la consommation énergétique du bâtiment référencée dans un guide établi par le ministre chargé de la construction et publié dans les conditions prévues à l'article R. 312-3 du code des relations entre le public et l'administration.
		Le maître d'ouvrage justifie du temps de retour sur investissement soit en produisant une note réalisée par un homme de l'art sous sa responsabilité, soit en établissant que sa durée est supérieure à dix ans par comparaison du bâtiment aux cas types référencés dans le guide mentionné au précédent alinéa.
		La loi TECV avait prévu que les travaux embarqués, considérés comme obligatoires, relevaient de l’article 24 alors que les autres travaux de rénovation énergétiques relèvent de l’article 25 -f . Cette incohérence dans les majorités retenues était d’autant plus préjudiciable que la plupart des travaux de rénovation énergétique comportent à la fois des travaux « embarqués » et d’autres travaux d’amélioration de la performance énergétique.
		L’article 212 de la loi ELAN a purement et simplement supprimé l’article 24 h, ainsi que la réserve qui figurait à ce sujet de l’article 25 f. Les travaux « embarqués » relèvent donc bien d’un vote à la majorité absolue (second tour possible).
		Compte tenu du faible niveau de participation des copropriétaires dans les grandes copropriétés, ce choix risque d’être bloquant précisément dans les copropriétés qui ont un besoin impérieux de rénovation énergétique. De plus, le choix de l’article 25 a pour conséquence l’impossibilité de procéder à des délégations au conseil syndical pour le choix de l’entreprise retenue.
		
		\subsubsection{D. Intervenir sur les parties privatives : les « travaux d’intérêt collectifs » sur parties privatives.}
		
		En outre, il résulte de cette disposition que l’assemblée générale peut décider de travaux sur les parties privatives, sans que l'accord du propriétaire du lot ne soit requis. Cette disposition est indispensable pour permettre les travaux de rénovation de l'étanchéité extérieure, qui touche fréquemment les huisseries et vitrages, lesquels peuvent être classés en parties privatives par le règlement de copropriété.
		Une atteinte au droit de propriété… et aux droits locatifs.
		Il s'agit toutefois d'une atteinte grave au principe de souveraineté du copropriétaire sur son lot, bien que la loi admît déjà qu'un copropriétaire pût être contraint de subir les conséquences d'une intervention du syndicat sur les parties communes mais en contrepartie d'une indemnisation du préjudice subi (art. 9 de la loi du 10/07/1965).
		Parallèlement, il a été introduit dans la loi du 6 juillet 1989 relatif au bail d'habitation une disposition (art 7e) imposant aux locataires de supporter les travaux d'amélioration de la performance énergétique à réaliser dans son logement.
		Le copropriétaire peut toutefois être dispensé.
		En revanche, et c’est là la seule exception, si le copropriétaire concerné par ces travaux sur ses parties privatives établit que des travaux équivalents à ceux votés en assemblée générale ont été réalisés, sur son lot, dans les dix dernières années, aucun desdits travaux ne pourra être effectué.
		Qui jugera de l’équivalence ?
		Dispositions spéciales aux travaux votés sur parties privatives :
		Article R 138-3 nouveau CCH :
		.-Les travaux d'économie d'énergie ou de réduction des émissions de gaz à effet de serre d'intérêt collectif réalisés sur les parties privatives mentionnés au g de l'article 25 de la loi \no 65-557 du 10 juillet 1965 fixant le statut de la copropriété des immeubles bâtis comprennent tout ou partie des travaux mentionnés au 2\degres du I de l'article R. 138-2. « Dès lors que de tels travaux sont votés, les copropriétaires concernés sont tenus de les réaliser dans un délai raisonnable en fonction de la nature et du coût des travaux, sauf s'ils sont en mesure de prouver la réalisation de travaux équivalents. « Le syndicat des copropriétaires procède à la réception des travaux en présence des copropriétaires concernés. En cas de réserves, le syndic de copropriété assure le suivi et la réception des travaux destinés à permettre la levée des réserves. Après réception définitive des travaux, le syndic de copropriété adresse aux copropriétaires concernés, par lettre recommandée avec avis de réception ou par voie de remise contre émargement, les pièces et documents relatifs aux travaux, notamment le contrat de l'entreprise, le ou les procès-verbaux de réception et, le cas échéant, les attestations des assurances prévues aux articles L. 241-2 et L. 242-1 du code des assurances afin que chaque copropriétaire puisse utilement mettre en oeuvre les garanties à la charge de l'entreprise. »
		De même l’article 25 f de la loi \no 65-557 du 10 juillet 1965 dans sa rédaction issue de l’ordonnance dispose :
		Ces travaux peuvent comprendre des travaux d'intérêt collectif réalisés sur les parties privatives et aux frais du copropriétaire du lot concerné, sauf dans le cas où ce dernier est en mesure de produire la preuve de la réalisation de travaux équivalents dans les dix années précédentes.
		Un décret en Conseil d'Etat précise les conditions d'application du présent f.
		Quant à l’article 10-1 c) de la loi \no65-557 du 10 juillet 1965 il précise que :
		« Par dérogation aux dispositions du deuxième alinéa de l'article 10, sont imputables au seul copropriétaire concerné : (…)
		c) Les dépenses pour travaux d'intérêt collectif réalisés sur les parties privatives notamment en application du c du II de l'article 24 et du f de l'article 25 ; »
		Si des dépenses sont imputables à un copropriétaire, cela signifie que le syndicat a lui-même assumé ces dépenses. Or, l’article R 138-3 CCH précité affirme que :
		« Dès lors que de tels travaux sont votés, les copropriétaires concernés sont tenus de les réaliser dans un délai raisonnable en fonction de la nature et du coût des travaux ».
		Qui « réalise » des travaux si ce n’est le maître d’ouvrage, voire le maître d’ouvrage délégué ?
		De plus que signifie ce renvoi à une « Réception définitive, après levée des réserves », qui implique une réception provisoire, alors même que ces notions de réception provisoire et de réception définitive ont fait place en 1978 a une réception unique avec ou sans réserves ?
		
		\subsubsection{E. Individualiser les consommations}
		
		Une loi du 29 octobre 1974 (portant RTE Réglementation Thermique) a posé le principe de l'installation dans tous les immeubles collectifs pourvus d'un chauffage commun de compteurs de chaleur et d'eau chaude. Ce texte fut à l’époque inséré dans le Code de la Construction et de l'Habitation (article L 131-3)\footnote{C’est aujourd’hui l’article L 241-9 du Code de l’Energie.}
		" Tout immeuble collectif pourvu d'un chauffage commun doit comporter quand la technique le permet, une installation permettant de déterminer la quantité de chaleur et d'eau chaude fournie à chaque local occupé à titre privatif ".
		Le but d'une telle disposition était bien évidemment de faire payer par le copropriétaire du lot l'énergie qu'il consomme. Aussi le même article L 131-3 CCH se poursuivait par les dispositions suivantes :
		" Nonobstant toute disposition, convention ou usage contraire, les frais de chauffage et de fourniture d'eau chaude mis à la charge des occupants comprennent, en plus des frais fixes, le coût des quantités de chaleur calculées comme il est dit ci-dessus".
		Il était généralement admis que la pose des compteurs de chaleur dans les immeubles où ceux-ci étaient obligatoires relevait d'une décision de la majorité de l'article 25 (travaux rendus obligatoires par les dispositions législatives ou réglementaires). En revanche, la pose de ces compteurs relevait de la double majorité de l'article 26 pour les immeubles dans lesquels leur pose n'était pas obligatoire.
		La pose de ces fameux compteurs de chaleur, malgré l'obligation légale et des textes réglementaires de décembre 1991, avait eu jusqu'à présent un succès très limité, alors qu'elle permet la réalisation d'économies allant jusqu'à \pourcent{20} de l'énergie consommée. C’est pourquoi dans le cadre de la nouvelle RTE élaborée par la loi ENE du 12 juillet 2010 (Grenelle II), l’article 25 s’était vu ajouter un petit o, aux termes duquel étaient votés à la majorité des voix de tous les copropriétaires (avec faculté de repêchage selon les dispositions de l’article 25-1) : « o) L'installation de compteurs d'énergie thermique ou de répartiteurs de frais de chauffage »

		En application de la loi Grenelle II un nouveau Décret a été publié le 23 avril 201255, disposait que cette installation de compteurs lorsqu’elle est obligatoire doit être faite avant le 31 mars 2017 et que les compteurs doivent être relevés sans avoir à pénétrer dans les locaux privatifs.
		
		Pour éviter que ces textes soient aussi peu respectés que les dispositions de 1991 qui étaient pourtant d’ordre public, la loi de Transition Energétique modifie la loi \no65-557 du 10 juillet 1965 en ajoutant un article 24-9 qui est ainsi rédigé : « Art. 24-9.-Lorsque l'immeuble est pourvu d'un chauffage commun à tout ou partie des locaux occupés à titre privatif et fournissant à chacun de ces locaux une quantité de chaleur réglable par l'occupant et est soumis à l'obligation d'individualisation des frais de chauffage en application de l'article L. 241-9 du code de l'énergie, le syndic inscrit à l'ordre du jour de l'assemblée générale la question des travaux permettant de munir l'installation de chauffage d'un tel dispositif d'individualisation, ainsi que la présentation des devis élaborés à cet effet. »
		Par ailleurs, la même loi du 17 août 2015 créée une sanction au non-respect de l’obligation de pose de compteurs individuels :
		\begin{quote}
			« Article L242-3 du Code de l’énergie
			
			En cas de manquement à l'article L. 241-9, l'autorité administrative met l'intéressé en demeure de s'y conformer dans un délai qu'elle détermine.
		\end{quote}
		\begin{quote}
			Article L242-4 du Code de l’énergie
			
			En l'absence de réponse à la requête mentionnée à l'article L. 242-2 dans le délai d'un mois ou lorsque l'intéressé ne s'est pas conformé à la mise en demeure prononcée en application de l'article L. 242-3 dans le délai fixé, l'autorité administrative peut prononcer à son encontre chaque année, jusqu'à la mise en conformité, une sanction pécuniaire par immeuble qui ne peut excéder \montant{1 500} par logement. Cette sanction est prononcée après que l'intéressé a reçu notification des griefs et a été mis à même de consulter le dossier et de présenter ses observations, assisté, le cas échéant, par une personne de son choix. L'amende est recouvrée comme les créances de l'Etat étrangères à l'impôt et au domaine. »
		\end{quote}
		
		Les dispositions de la loi TECV ont été complétées par le Décret \no 2016-710 du 30 mai 2016 (partie réglementaire du code de l’énergie). Ces dispositions ont à nouveau été modifiées par la loi \no 2018-1021 du 23 novembre 2018 dite ELAN puis par la loi \no 2019-1157 du 8 nov. 2019
		Article L241-9 du code de l’ énergie Modifié par LOI \no2018-1021 du 23 novembre 2018 - art. 71 puis par la Loi \no 2019-1147 du 8 nov. 2019 – art. 23 (modification formelle) :
		Tout immeuble collectif d'habitation ou mixte pourvu d'une installation centrale de chauffage doit comporter, quand la technique le permet, une installation permettant de déterminer et de réguler la quantité de chaleur et d'eau chaude fournie à chaque local occupé à titre privatif. Tout immeuble collectif d'habitation ou mixte pourvu d'une installation centrale de froid doit comporter, quand la technique le permet, une installation permettant de déterminer et de réguler la quantité de froid fournie à chaque local occupé à titre privatif. Le propriétaire de l'immeuble ou, en cas de copropriété, le
		55 Décret \no 2012-545 du 23 avril 2012 relatif à la répartition des frais de chauffage dans les immeubles collectifs
		syndicat des copropriétaires représenté par le syndic s'assure que l'immeuble comporte des installations répondant à ces obligations.
		Nonobstant toute disposition, convention ou usage contraires, les frais de chauffage, de refroidissement et de fourniture d'eau chaude mis à la charge des occupants comprennent, en plus des frais fixes, le coût des quantités de chaleur et de froid calculées comme il est dit ci-dessus.
		Un décret pris en Conseil d'État fixe les conditions d'application du présent article, et notamment la part des frais fixes visés au précédent alinéa, les délais d'exécution des travaux prescrits, les caractéristiques techniques et les fonctionnalités des installations prévues au premier alinéa ainsi que les cas et conditions dans lesquels il peut être dérogé en tout ou partie aux obligations prévues au même premier alinéa, en raison d'une impossibilité technique ou d'un coût excessif au regard des économies attendues.
		Lorsqu'il n'est pas rentable ou techniquement possible d'utiliser des compteurs individuels pour déterminer la quantité de chaleur, des répartiteurs des frais de chauffage individuels sont utilisés pour déterminer la quantité de chaleur à chaque radiateur, à moins que l'installation de tels répartiteurs ne soit ni rentable ni techniquement possible. Dans ces cas, d'autres méthodes rentables permettant de déterminer la quantité de chaleur fournie à chaque local occupé à titre privatif sont envisagées. Un décret en Conseil d'État précise le cadre de mise en place de ces méthodes. Dans sa version actuelle, résultant du décret de mai 2019, la partie règlementaire dispose que :
		Art. R. 241-7 :
		I.-Tout immeuble collectif à usage d'habitation ou à usage professionnel et d'habitation pourvu d'une installation centrale de chauffage ou alimenté par un réseau de chaleur est muni de compteurs individuels d'énergie thermique permettant de déterminer la quantité de chaleur fournie à chaque local occupé à titre privatif et ainsi d'individualiser les frais de chauffage collectif. II.-Les dispositions du I ne sont pas applicables :
		2\degres Aux immeubles dans lesquels, pour des motifs et dans des cas précisés par arrêté conjoint des ministres chargés de l'énergie et de la construction, il est techniquement impossible d'installer des compteurs individuels pour mesurer la chaleur consommée par chaque local pris séparément ou de poser un appareil permettant aux occupants de chaque local de moduler la chaleur fournie par le chauffage collectif ;
		3\degres Aux immeubles dont les valeurs de consommation en chauffage sont inférieures à un seuil fixé par arrêté conjoint des ministres chargés de l'énergie et de la construction ;
		4\degres Aux autres immeubles pour lesquels le propriétaire ou, le cas échéant, le syndicat des copropriétaires, représenté par le syndic, justifient que l'individualisation des frais de chauffage par l'installation de compteurs individuels se révèle techniquement impossible ou entraîne un coût excessif au regard des économies d'énergie susceptibles d'être réalisées. Dans ce cas, le propriétaire de l'immeuble ou, le cas échéant, le syndicat des copropriétaires représenté par le syndic établit une note justifiant de cette impossibilité technique ou de ce coût excessif. Cette note est jointe aux carnets numériques d'information, de suivi et d'entretien des logements, établis en application de l'article L. 111-10-5 du code de la construction et de l'habitation.
		III.-Dans les cas mentionnés aux 2\degres, 3\degres et 4\degres du II, dans lesquels l'installation de compteurs
		individuels d'énergie thermique ne serait pas techniquement possible, ou entraînerait des coûts excessifs au regard des économies d'énergie attendues, des répartiteurs de frais de chauffage sont installés pour mesurer la consommation de chaleur à chaque radiateur.
		Art. R. 241-8 :
		I-Tout immeuble collectif à usage d'habitation ou à usage professionnel et d'habitation pourvu d'une installation centrale de froid ou alimenté par un réseau de froid est muni d'appareils de mesure permettant de déterminer la quantité de froid fournie à chaque local occupé à titre privatif et ainsi d'individualiser les frais de refroidissement collectif. II.-Les dispositions du I ne sont pas applicables :
		(nb : les dérogations sont similaites)
		Art. R. 241-9 : Avant toute installation des appareils prévus à l'article R. 241-7, les émetteurs de chaleur, quand cela est techniquement possible, sont munis, à la charge du propriétaire, d'organes de régulation en fonction de la température intérieure de la pièce, notamment de robinets thermostatiques en état de fonctionnement.
		L’obligation entre progressivement en vigueur, selon la performance du Bâtiment (octobre 2020 au plus tard)
		Par ailleurs, depuis l’Ordonnance du 30 octobre 2019, l’article 10 de la loi \no 65-557 du 10 juillet 1965 est ainsi rédigé
		Les copropriétaires sont tenus de participer aux charges entraînées par les services collectifs et les éléments d'équipement commun en fonction de l'utilité objective que ces services et éléments présentent à l'égard de chaque lot, dès lors que ces charges ne sont pas individualisées.
		a. Procédure de pose des compteurs
		Le vote d’installation des compteurs, majorité article 25 f.
		Sur proposition faite par un Ingénieur spécialisé et /ou devis d’entreprises, dans tout immeuble en copropriété existant, doté d’un chauffage collectif, qu’il soit à usage d’habitation ou autre, et sauf dispense réglementaire ou impossibilité technique, (dalle chauffante, chauffage à air chaud, vapeur, …) l’assemblée générale doit voter l’installation d’appareils de mesure de la chaleur fournie et les appareils devront être posés avant le 31 mars 2017..
		Ces compteurs sont pris en location auprès d’une société spécialisée dans le comptage des fluides.
		Etant observé :
		- Que doit être normalement voté préalablement l’installation de robinets thermo-statiques (art. R 241-9 Code de l’énergie), c’est-à-dire d’un système à placer sur les radiateurs à eau qui régule la température dans chaque pièce (permettant d’y avoir une température constante), dont le coût est de l’ordre 40 à 50 € par appareil.
		- Il ne faut pas confondre le compteur individuel d’énergie thermique qui mesure directement la quantité consommée par le lot (mais dont la pose n’est possible qu’en cas de distribution d’eau chaude par un système horizontal) et le répartiteur de frais de chauffage qui est un boîtier placé sur le radiateur qui calcule la différence de température entre le radiateur et la pièce pour en déduire la quantité de chaleur effectivement consommée (seul système adapté à la distribution d’eau par colonnes verticales. Or la plupart des immeubles sont desservis par des canalisations verticales Etant bien compris que l’obligation de comptage individuel s’applique en présence de distribution verticale comme en présence de distribution horizontale de l’eau chaude.
		Le vote des compteurs ou répartiteurs de chaleur se fera à la majorité de l’article 25 et éventuellement de l’article 25-1 comme exposé précédemment. Il en ira de même du vote de la pose de robinets thermo statiques. En effet, et quand bien même les radiateurs sont privatifs, il s’agit là d’un vote portant sur des travaux d’intérêt collectif au sens de l’article 25 l de la loi \no65-557 du 10 juillet 1965 56
		l’adoption du mode de relevé des compteurs
		Dans tous les cas le relevé des compteurs se fait sans pénétrer dans les parties privatives. Mais il existe deux systèmes : soit par radio-relevé (un technicien relève les compteurs à l’immeuble), soit par télé-relevé qui nécessite un collecteur global dans l’immeuble dont les données peuvent être transmises à la société de location des compteurs et qui permet un contrôle beaucoup plus fin de la consommation.
		le contrat de location des compteurs et d’exploitation des données
		L’assemblée générale décidera également la souscription du contrat de location des compteurs et répartiteurs (à notre sens il conviendra de faire un vote aux mêmes conditions de majorité).
		la modification du Règlement de copropriété art 24-II-f
		Enfin, et quand bien même l’article L 241-9 Code de l’Energie édicte que :
		« Nonobstant toute disposition, convention ou usage contraires, les frais de chauffage et de fourniture d'eau chaude mis à la charge des occupants comprennent, en plus des frais fixes, le coût des quantités de chaleur calculées comme il est dit ci-dessus »
		56 La liste de ces travaux a été donnée par le Décret du 3 décembre 2012 (article R. 138-2 Code de la Construction et de l'Habitation) qui définit parmi les travaux d’intérêt collectif portant sur les parties privatives « b) Pose ou remplacement d'organes de régulation ou d'équilibrage sur les émetteurs de chaleur ou de froid ; « c) Équilibrage des émetteurs de chaleur ou de froid ; « d) Mise en place d'équipements de comptage des quantités d'énergies consommées.
		Il nous paraît donc nécessaire de faire un vote sur la modification du Règlement de copropriété au chapitre de la répartition des charges de chauffage, au titre de l’article 24-II-f de la loi \no65-557 du 10 juillet 1965 :
		« Les adaptations du règlement de copropriété rendues nécessaires par les modifications législatives et réglementaires intervenues depuis son établissement. La publication de ces modifications du règlement de copropriété est effectuée au droit fixe »
		Ceci pour que ce modificatif soit publié et opposable de plein droit aux acquéreurs qui – en l’absence de modification du Règlement de copropriété - pourront toujours être tentés de plaider que les dispositions du Règlement de copropriété s’appliquent tant qu’elles n’ont pas été modifiées per le juge ou par une décision d’assemblée générale qui lui serait opposable, quand bien même la loi rend impérative une répartition différente de celle prévue au Règlement de copropriété d’origine.
		La résolution modifiant le Règlement de copropriété sera très simple à rédiger puisqu’il conviendra de reprendre les dispositions de l’article R. 241-13 du code de l’énergie :
		La règle est la suivante (très proche de l’ancienne règle posée par l’arrêté de 1991) :
		- Immeubles nouvellement dotés de compteurs individuels :
		On prend le total des frais de chauffage que l’on divise en deux masses :
		Première masse
		Les frais de combustible ou d’énergie (Prestations de type P1)
		Frais communs : 30 % de la dépense totale sont répartis conformément aux dispositions du Règlement de copropriété (c’est la part conduite et entretien)
		Frais individuels : 70 % restant sont répartis en fonction des compteurs de calories avec possibilité de ventiler en fonction des configurations thermiques défavorables (ce sont les frais de combustible ou d’énergie).
		Deuxième masse
		Autres frais de chauffage (Prestations de type P2 et P3)
		Maintenance des installations (maintien en bon état de fonctionnement de l’installation) et conduite du chauffage (opérations de pilotage de la production et de distribution de la chaleur nécessaire pour obtenir les températures contractuelles dans les différents locaux et, le cas échéant, celle de l’eau chaude sanitaire) et les frais relatifs à l'utilisation d'énergie électrique (ou éventuellement d'autres formes d'énergie)
		pour le fonctionnement des appareillages, notamment les instruments de régulation, les pompes, les brûleurs et les ventilateurs
		Répartis conformément aux dispositions du Règlement de copropriété
		La modification du Règlement de copropriété sera d’autant plus nécessaire dans l’hypothèse où l’assemblée générale fera application des dispositions de l’article R. 241-13 selon lesquelles : « les situations ou configurations thermiquement défavorables des locaux pouvant être prises en compte ».
		Bien évidemment les travaux en chaufferie (renouvellement du matériel dans les contrats P3) continueront d’être répartis conformément aux dispositions du Règlement de copropriété et sous réserve d’actualisation on ne pourra pas modifier les charges générales de chauffage (autres que d’énergie) à la majorité de l’article 24 de la loi, l’unanimité restant nécessaire.
		La moyenne entre ces masses serait de l’ordre de 80 % à 90 % pour le P1 et de 10 à 20 % pour le P2 ; en sorte que ces nouvelles dispositions vont modifier très sensiblement la répartition des charges de chauffage entre les copropriétaires.
	
	\subsection{SOUS SECTION II CREATION DE NOUVEAUX LOCAUX PRIVATIFS ET SURELEVATION. (ART 35 DE LA LOI DU 10 JUILLET 1965 )}
	
	La loi du 24 mars 2014 a littéralement bouleversé les règles que la loi de 1965 avait initialement posées quant aux conditions et modalités de la surélévation, que celle-ci soit le fait du syndicat ou le fait de certains copropriétaires. Il convient d’observer en effet que pour lutter contre l’étalement des villes, le législateur moderne entend faciliter la surélévation des immeubles existants.
	
	Rappelons, au plan administratif, l’ordonnance \no2013-889 du 3 octobre 2013 relative au développement de la construction de logements, qui permet de déroger aux règles de gabarit, de densité et de stationnement lors de la délivrance de permis de construire dans le cadre d’un projet de surélévation ou encore aux règles d’isolation acoustique ou à la protection des personnes contre l’incendie. On citera également la loi \no2014-1545 du 20 décembre 2014 relative à la simplification de la vie des entreprises qui permet de déroger aux règles de retrait en cas de surélévation.
	
	On rappellera enfin que l'article 28 de la loi \no017-1775 du 28 décembre 2017 proroge jusqu'au 31 décembre 2020 le dispositif d'exonération des plus-values immobilières résultant de la cession d'un droit de surélévation, en vue de la réalisation de locaux destinés exclusivement à l'habitation
	"ARTICLE 35 DE LA LOI ALINEA PREMIER.
	"La surélévation ou la construction de bâtiments aux fins de créer de nouveaux locaux à usage privatif ne peut être réalisée par les soins du syndicat que si la décision en est prise à la majorité prévue à l’article 26". La décision d’aliéner aux mêmes fins le droit de surélever un bâtiment existant exige la majorité prévue à l’article 26, et, si l’immeuble comprend plusieurs bâtiments, la confirmation par une assemblée spéciale des copropriétaires des lots composant le bâtiment à surélever, statuant à la majorité indiquée ci-dessus.
	Les copropriétaires de l'étage supérieur du bâtiment surélevé bénéficient d'un droit de priorité à l'occasion de la vente par le syndicat des locaux privatifs créés. Préalablement à la conclusion de toute vente d'un ou plusieurs lots, le syndic notifie à chaque copropriétaire de l'étage supérieur du bâtiment surélevé l'intention du syndicat de vendre, en indiquant le prix et les conditions de la vente. Cette notification vaut offre de vente pendant une durée de deux mois à compter de sa notification.
	« Les copropriétaires de l'étage supérieur du bâtiment à surélever bénéficient du même droit de priorité à l'occasion de la cession par le syndicat de son droit de surélévation. Ce droit de priorité s'exerce dans les mêmes conditions que celles prévues au quatrième alinéa.
	a. La surélévation faite par le syndicat des copropriétaires et à ses frais.
	La loi du 10 juillet 1965 porte statut de la copropriété. Il s’agit d’un statut dont l’objet légal est de la gestion, la conservation et l’amélioration des immeubles.
	Il n’est donc pas de la compétence du syndicat des copropriétaires de réaliser des opérations de promotion immobilière, si ce n’est pour créer de nouvelles parties communes (locaux poubelles dans la cour, locaux sociaux, parking collectif, etc …)
	La loi de 1965 n’avait pas totalement interdit au syndicat des copropriétaires de « faire de la promotion » pour créer de nouvelles parties privatives, elle subordonnait cependant cette opération à l’accord unanime des copropriétaires.
	Le projet de loi ALUR et le texte adopté par l’Assemblée nationale en première lecture ne changeaient rien à cet état de choses.
	Un amendement parlementaire\footnote{De M Joël Labbé, sénateur de droite …} approuvé par la ministre a modifié le texte en sorte que le droit de surélever l’immeuble aux frais du syndicat des copropriétaires puisse être voté à la double majorité de l’article 26.
	La ministre a vu dans cette modification au texte de loi une possibilité de densification et de dégager des moyens nouveaux pour la copropriété.
	Tel est bien l’objet de cet amendement : les immeubles nécessitent d’importants travaux de conservation, mais faute de les avoir prévu en temps utile et en l’absence de fonds de prévoyance, les copropriétaires ne disposent pas des moyens financiers pour réaliser ces travaux. Leur mise en oeuvre par le syndicat des copropriétaires permettra de dégager des « bénéfices »\footnote{Sont-ils taxables ?} qui seront utilisés pour la conservation des parties communes de l’immeuble.
	Il est vrai cependant que le même résultat pourrait être atteint par la cession du droit de surélever qui certes serait moins rémunérateur pour le syndicat des copropriétaires mais lui permettrait malgré tout de dégager des fonds nécessaires à la réalisation des travaux de conservation sans prendre le risque d’une opération de promotion immobilière.
	b. L’aliénation du droit de surélever
	
	1. Majorité requise
	
	Sur ce point, la décision est toujours prise aux mêmes conditions de majorité : la double majorité de l’article 26 est en principe toujours nécessaire et si l’immeuble comporte plusieurs bâtiments une assemblée spéciale des copropriétaires des lots composant le bâtiment à surélever devra statuer à la même double majorité de l’article 26.
	L’exception apportée à cette exigence de la double majorité par la loi du 12 mai 2009 (loi MLLE) est maintenue : à savoir lorsque l’immeuble est situé dans un périmètre sur lequel est institué le droit de préemption urbain (DPU)\footnote{59}, la majorité applicable est alors la majorité des voix de tous les copropriétaires. C’est le cas de presque toutes les grandes agglomérations…
	Toute clause contraire du règlement de copropriété est réputée non écrite
	Cependant, l’unanimité reste requise si il existe, du fait de l’opération
	- atteinte à la destination de l’immeuble (transformation en IGH)
	- atteinte aux droits des autres copropriétaires (accès aux lots)
	Par ailleurs, s’il existe plusieurs bâtiments distincts, plusieurs bâtiments » ou si la toiture est une partie commune spéciale à deux corps de bâtiments : celle du bâtiment concerné et celle de tous les copropriétaires
	
	2. Champs d’application
	
	Dans certains, l’article 35 (et donc le vote à la majorité de l’article 25) ne trouvera pas à s’appliquer
	Techniquement, la surélévation n’existe qu’à trois conditions :
	- Une construction en dur ( les “vérandas mobiles” et “abris de terrasses) : à défaut, une simple autorisation de l’article 25 b est suffisant
	- La prolongation verticale des façades
	- Le rehaussement à un niveau plus élevé du faîtage de la toiture
	59 Le DPU résulte de l’initiative du conseil municipal ou de l’organe délibérant de l’établissement public de coopération intercommunale, dans les zones urbaines ou à urbaniser des communes dotées d’un POS ou d’un PLU opposable, dans les communes dotées d’un plan de sauvegarde et de mise en valeur (PSVM), dans les communes dotées d’une carte communale en vue de la réalisation d’une opération d’aménagement ou encore dans certains secteurs nécessitant une protection particulière.
	Tout projet d’aliénation dans une zone de préemption doit faire l’objet d’une déclaration d’intention d’aliéner (DIA). L’autorité titulaire du droit de préemption dispose alors d’un délai de deux mois pour notifier sa décision au propriétaire du bien. En cas de désaccord sur le prix, celui-ci est fixé par le juge de l’expropriation. Le propriétaire dispose de certaines garanties quant aux délais de règlement.
	Par ailleurs, le syndicat des copropriétaires ne possédait le droit de surélever que si l'on est propriétaire. Or, s'il s'agit d'exaucer un bâtiment constituant d'un lot privatif, le droit de surélever appartient au propriétaire du bâtiment. Par ailleurs, droit de surélever peut avoir été réservé un copropriétaire ou un tiers dans le cadre d'une convention de l'article 37 de la loi du 10 juillet 1965, il faut alors s'assurer que cette convention illicite est toujours en vigueur.
	3.2.Le droit de « priorité » des copropriétaires de l’étage supérieur du bâtiment surélevé.
	Les copropriétaires de l’étage supérieur bénéficiaient avant la loi ALUR d’un droit de veto, qu’ils ont perdu au profit d’un droit de priorité sur les locaux créés par la surélévation réalisée.
	Ce droit de « priorité »\footnote{60} pourra s’exercer dans les deux cas : soit lorsque le syndicat des copropriétaires réalise des nouveaux locaux et les met en vente, soit lorsque ce droit de surélever est exercé par un copropriétaire ou un tiers qui s’est vu céder le droit de surélever.
	Si l’on comprend parfaitement l’existence de ce droit de priorité lorsque le syndicat des copropriétaires réalise lui-même la surélévation ; on le comprend moins bien lorsque le droit de surélévation ne doit être exercé qu’au-dessus d’un lot existant dont le propriétaire souhaite réaliser la surélévation.
	En effet dans cette dernière hypothèse, le copropriétaire qui souhaite surélever au-dessus de son lot va demander au syndic d’inscrire à l’ordre du jour la création d’un nouveau lot surélévation au-dessus du lot dont il est déjà propriétaire.
	Mais compte tenu qu’au dernier étage il y a déjà plusieurs lots (plusieurs appartements, voire une ou deux chambres de service, appartenant à différents propriétaires), le syndic devra notifier à chacun de ces copropriétaires l’intention du syndicat de vendre le droit de surélévation et finalement ce droit pourra être acquis par un copropriétaire dont le lot se trouve à l’autre extrémité de l’étage supérieur !
	Enfin, comment est déterminée la valeur du droit de surélévation ? En cas de pluralité d’offres, qui sera déclaré acquéreur ? Le plus offrant … ou le copropriétaire le plus sympathique\footnote{61} ?
	Or, tel que le texte est aujourd’hui rédigé, dans cette dernière hypothèse le mécanisme devrait être le suivant :
	- Le copropriétaire d’un lot situé à l’étage supérieur qui souhaite surélever au-dessus de son lot demande au géomètre de constituer un nouveau lot à partir des parties communes qui sera dénommé droit de surélever et doté d’un certain nombre de tantièmes.
	60 Si ce droit de « priorité » s’apparente à un droit de préférence au sens du nouvel article 1123 du code civil, il a une nature distincte, le code civil en effet vise les contrats et non les obligations légales Pour autant on peut estimer que la sanction prévue à l’article 1123 du code civil (nullité de la cession ou substitution dans les droits de l’acquéreur de mauvaise foi) recevra application.
	61 Cf. l’arrêt PETERSON
	- Il demandera au syndic inscrire à l’ordre du jour la création du nouveau lot « droit de surélévation » et la cession de ce lot de surélévation à son profit.
	- Il demandera également au syndic d’inscrire à l’ordre du jour l’autorisation de faire les travaux de surélévation au-dessus de son propre lot (sans doute avec une trémie pour accéder à ce lot nouvellement créé)
	- le syndic, constatant qu’il y a plusieurs lots à l’étage supérieur, va notifier à chacun des autres copropriétaires de l’étage supérieur l’intention du syndicat des copropriétaires de céder son droit de surélévation en précisant le prix offert par le copropriétaire demandeur. D’après le texte cette notification vaut offre de vente pendant une durée de deux mois à compter de sa date.
	- Bien évidemment le copropriétaire intéressé au premier chef va faire son offre. Mais d’autres copropriétaires auront pu faire une offre. Qui sera déclaré bénéficiaire le jour de l’assemblée générale ?
	Il n’est pas certain que la rédaction du nouvel article 35 soit parfaitement saine ! Quitte à alourdir le texte il serait sans doute bon d’ajouter que ce droit de priorité n’existe pas si la demande de surélévation est faite par un copropriétaire souhaitant surélever au-dessus de son lot.
	.
	4.3.Les autres décisions requises par l’aliénation du droit de surélever
	La décision de surélévation doit nécessairement être assortie de plusieurs autres décisions :
	- La décision relative au sort du prix versé pour la surélévation
	- L’autorisation de travaux à l’article 25 b:
	Approbation du projet de surélévation lui-même(intervention sur parties communes et aspect extérieur) avec, en annexe à la convocation des plans, descriptif travaux, visuel
	- Les modifications de l’état descriptif de division et du règlement de copropriété consécutives à la cession :
	Création d’un lot transitoire issu des parties communes ; modificatif de la description de l’immeuble ; modificatif des charges ( à la même majorité que la décision de surélever
	- L’indemnisation due au copropriétaire:
	- Qui subit un trouble de jouissance grave, même s’il est temporaire
	- Qui subit une « diminution définitive de la valeur de son lot » (ensoleillement, vue)
	Le montant de cette indemnisation est à la charge de tous les copropriétaires (sic!) et doit être fixé par l’assemblée. Un rapport sur l’évaluation des indemnités devra être établi et joint à l’assemblée générale, et ce coût devra être intégré dans la cession, ou reversé à titre indemnitaire par le bénéficiaire.