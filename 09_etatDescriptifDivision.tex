\chapter{L'état descriptif de division}

\section{SECTION I - DEFINITION DE L'ÉTAT DESCRIPTIF DE DIVISION}

	\subsection{A. ORIGINES DE L'ETAT DESCRIPTIF DE DIVISION}
	
		1. Le Cadastre.
		La Direction Générale des Impôts, Service de l'Administration générale définit le Cadastre comme "une institution consacrée au recensement de toutes les propriétés, à la recherche de leurs propriétaires apparents ou réels, à la reconnaissance et à la définition de leurs limites, à leur description, à la constatation de leur mise en valeur, à leur évaluation". (Le Cadastre de la France, Paris novembre 1980).
		En nous épargnant le détour par l'Antiquité et l'analyse des tâches confiées par les Egyptiens aux harpédonaptes362, ancêtres de nos géomètres experts, nous retiendrons qu'en France, dès le XIII\degres siècle sont apparus des registres - tels le livre terrier ou le livre d'estimes en Languedoc - que l'on considère comme les premiers cadastres, dont la présentation ira en s'unifiant et qui ne décriront plus à partir de la fin du XV\degres siècle que la consistance des biens immobiliers possédés par différents "feux", consistance établie sur la base des indications obtenus par des "estimateurs" en interrogeant les voisins.
		Le Cadastre moderne résulte d'une loi de finances du 15 septembre 1807, modifiée en 1898. Il a été mis à jour par différents textes de 1930, 1941, 1955 et 1974.
		Présentement sont identifiés par le Cadastre près de 37 Millions de locaux qui sont la propriété de 20 millions de personnes !
		Chaque Commune est figurée sur un plan dit Plan Cadastral qui se compose lui-même de feuilles parcellaires (elles mesurent 75 x 105 cm) et d'un ou plusieurs tableaux d'assemblage. Ce plan est une description graphique de tous les biens de la Commune, en sorte que ces biens sont identifiés par une structure hiérarchique :
		- Commune
		- Section Lettre Majuscule ou couple de lettres majuscules
		- Lieudit
		- Parcelle Elément unitaire de propriété (\no ..)
		362 « Tendeurs de cordes » en grec ancien
		droit de la copropriété année 2019-2020
		289
		- Bâtiment
		Ce que l'on nomme les Références Cadastrales est constitué par le couple Section et Parcelle, par exemple A 25 ou BA 42.
		Ce que l'on appelle Matrice Cadastrale est le document qui récapitule par propriétaire les propriétés bâties ou non bâties qu'il possède dans la Commune. Elle est mise à jour une fois par an.
		Il est évident que si le Plan Cadastral permet d'identifier les parcelles et à l'intérieur des parcelles les bâtiments, par contre il est insuffisant pour identifier sur un même bâtiment les différentes fractions divises d'un immeuble en copropriété.
		Il était donc indispensable d'assurer une parfaite connaissance des différents droits réels en concours sur une même parcelle cadastrale ou à l'intérieur d'un même bâtiment.
		2. La création du Fichier Immobilier.
		Cette connaissance des droits réels concurrents sur une même parcelle a été obtenue par la réforme de la Publicité Foncière363 réalisée par le Décret du 4 janvier 1955.
		Ce Décret a en effet rendu obligatoire l'établissement et la publication au Bureau des Hypothèques364 dont dépend l’immeuble d’un état descriptif de division, c'est à dire la configuration des divers lots, leur numérotation, et la quote-part attachée à chaque lot dans les parties communes de l'ensemble immobilier :
		* Article 1er du Décret du 4 janvier 1955 :
		" Il est tenu pour chaque commune, par les conservateurs des hypothèques, un fichier immobilier sur lequel, au fur et à mesure des dépôts, sont répertoriés, sous le nom de chaque propriétaire, et, en ce qui concerne les catégories d'immeubles définies par décret en Conseil d'Etat, par immeuble, des extraits des documents publiés, avec référence à leur classement dans les archives ".
		363 Le principe de la Publicité Foncière remonte au « Code Colbert » (Edit du 21 mars 1673). L’idée étant d’inscrire sur chaque immeuble les créances existantes pour sécuriser les tiers. Abolie comme portant atteinte aux secrets des famille, elle fut rétablie par une loi du 23 mars 1855.
		364 La Conservation des Hypothèques a été supprimée pour être remplacée le 1er janvier 2013 par le Service de Publicité Foncière qui devient une administration comme les autres et le salaire du Conservateur est devenu la « contribution de sécurité immobilière »
		droit de la copropriété année 2019-2020
		290
		Encore fallait-il s'assurer que toutes les modifications juridiques concernant ces immeubles seraient effectivement transmises à la Conservation des Hypothèques.
		C'est pourquoi le même Décret, a prévu que :
		* Article 4 du Décret du 4 janvier 1955 :
		" Tout acte sujet à publicité dans un bureau des hypothèques doit être dressé en la forme authentique".
		Les actes soumis à publicité sont définis par l'article 28 du même Décret (essentiellement ce sont les actes de mutation ou de constitution de droit réel immobilier, les baux de plus de douze ans, les attestations de succession portant sur des biens immeubles, les conventions d'indivision immobilière, etc.).
		Ce qui signifie l'intervention obligatoire du notaire, Officier Public (ou du Préfet pour certains actes) ... et l'obligation faite au notaire de transmettre son acte au Service de Publicité Foncière.
		De la sorte toute mutation portant sur une fraction d'immeuble est nécessairement enregistrée sur le Fichier Immobilier qu'établit et détient Service de Publicité Foncière.
		Le non-respect de cette obligation est assorti de lourdes sanctions, ne serait-ce que l'effet relatif des formalités qui veut qu' "aucun acte ou décision judiciaire sujet à publicité dans un bureau des hypothèques ne peut être publié au fichier immobilier si le titre du disposant ou dernier titulaire n'a pas été préalablement publié."
		Lorsque le notaire envoie son acte au Service de Publicité Foncière "pour formalité", cet acte sera analysé et la formalité sera rejetée si les mentions obligatoires ne figurent pas dans l'acte, si l'acte comporte des inexactitudes ou des invraisemblances.
		Parmi ces mentions obligatoires figure l'identification certifiée des personnes concernées et la désignation cadastrale de l'immeuble.
		Tout ceci explique les difficultés rencontrées par les notaires lors de la "publication" de leurs actes. Souvent les notaires n’étaient pas rédacteurs des réglements de copropriété dont ils assuraient la publicationr : il en allait ainsi des Règlements de Copropriété directement rédigés par le vendeur en l'état futur d'achèvement et est simplement déposés au rang de leurs minutes pour pouvoir être publiés (la forme authentique requise par l'article 4 du Décret de 55). Pourtant ils devaient s'assurer que l'acte est publiable ... et parfois obliger le rédacteur à refaire sa copie !
		L’article 710-1 du code civil (rédaction du 28 mars 2011) pose le principe selon lequel le dépôt au rang des minutes d'un notaire d'un acte sous seing privé ne peut donner lieu aux formalités de publicité foncière.
	
	\subsection{B. FORMES DE L'ETAT DESCRIPTIF DE DIVISION}
	
		1. L’état descriptif de division est inclus ou non dans le règlement de copropriété
		Aux termes de l'article 8 de la loi de 65 et de l'article 2 du décret de 1967 : L'Etat Descriptif doit être inclus ou non dans le Règlement de Copropriété,
		Si l'Etat Descriptif est inclus dans le Règlement de Copropriété, il ne doit exister aucune confusion entre les clauses du Règlement de Copropriété et l'État descriptif de division :
		Selon l'article 3 du décret \no67-223 du 17 mars 1967:
		" Si le Règlement de Copropriété comprend un état descriptif de division et les conventions (prévues à l'article 37 de la loi365.) il doit être rédigé de manière à éviter toute confusion entre ses différentes parties et les clauses particulières au Règlement de Copropriété doivent se distinguer nettement des autres "
		2. L’état descriptif de division est obligatoire
		Prévu par le Décret de 55, l'état Descriptif de Division a été rendu obligatoire par le Décret du 7 janvier 1959, Et une loi du 2 janvier 1979 complétée par un Décret du 21 mai 1979 interdit aujourd'hui toute publication d'un lot non identifié dans l'état Descriptif.
		Il ne peut exister qu'un seul état descriptif de division par parcelle dont la propriété est placée globalement sous le régime de l'indivision forcée. Cette exigence résulte de l'article 71 du Décret du 7 janvier 1959.
		Aussi existe-t ‘il nombre d'immeubles placés antérieurement à cette date sous le régime de l'indivision forcée qui comprenaient autant d’états descriptifs de division qu'il y avait de bâtiments. Il s'agit alors d'un ou de plusieurs documents "trompeurs" qui peuvent faire croire que l'on n'est pas en présence d'une, mais de plusieurs copropriétés. Les Tribunaux considèrent alors qu'il y a en réalité une seule copropriété au sens de l'article 1er de la loi de 1965 et que les immeubles ainsi décrits indépendants les uns des autres doivent être assimilés à des syndicats secondaires.
		365 Nous avons vu précédemment que l’article 37 a vocation à disparaître après extinction des droits existant antérieurement à la publication de la loi ELAN. En sorte que désormais les nouveaux règlements de copropriété ne comportent plus de conventions de ce type.
		droit de la copropriété année 2019-2020
		292
		3. L’état descriptif de division doit être établi avant division de l’immeuble, ou à défaut à l’initiative de « tout intéressé »
		Ce sont les propriétaires ou copropriétaires de l'immeuble qui établissent l'état descriptif de division.
		Nous avons vu précédemment que l'état descriptif de division est nécessairement dressé en la forme authentique (acte notarié)
		Mais, l'état descriptif de division peut être établi par le Tribunal lorsqu'il procède au Partage d'un immeuble, créant de la sorte la Copropriété dans les conditions de l'art. 1 de la loi.
		Si l'immeuble est dépourvu d'état Descriptif, tout intéressé peut requérir un notaire d'en établir un (art. 50-1 du Décret de 1955).
		L'hypothèse est plus fréquente qu'on ne le croit : l’état descriptif de division existe, mais comporte une erreur matérielle (par exemple le même numéro de lot est attribué à deux lots distincts). Lors d'un partage, les héritiers ont fait établir le projet d’état descriptif de division mais sont en désaccord sur son contenu (essentiellement sur la valeur attribuée à chaque lot dans les tantièmes de parties communes).
		Le notaire rédigera son document, le soumettra pour observations aux intéressés puis, même en l'absence d'accord de ceux-ci il pourra publier son document ; sauf à l'un ou plusieurs des intéressés de demander au juge de trancher de la contestation.
	
	\subsection{C. CONTENU DE L’ETAT DESCRIPTIF DE DIVISION}
	
		Aux ternes de l'art. 71 du Décret de 1955 cet état descriptif de division doit :
		- Identifier l'immeuble ou l'ensemble immobilier.
		- Diviser l'immeuble en différentes fractions.
		- Identifier chaque fraction d'immeuble.
		- Indiquer la quote-part des parties communes comprises dans chaque fraction.
		1. Identification de l'immeuble ou de l'ensemble immobilier.
		Situation : Commune, lieu-dit, rue numéro.
		Contenance : Selon Cadastre ou Titres
		Désignation : Section, Numéro du Plan Cadastre.
		2. Division de l'Immeuble en fractions ou lots.
		droit de la copropriété année 2019-2020
		293
		L'immeuble est divisé en fractions. Chaque fraction comprend un lot et un seul lot
		Sont des lots distincts :
		Pour les bâtiments, chaque lot principal et chaque lot secondaire.
		Pour les terrains nus, chaque portion de terrain sur laquelle est réservé un droit réel privatif.
		3. Identification et numérotation de chaque fraction.
		Une fois constituée, chaque fraction doit être identifiée par son emplacement dans l'immeuble
		Bâtiment, escalier, étage, emplacement.
		Chaque fraction de l'immeuble ainsi déterminée constitue un lot. Dans l'état descriptif de division, chaque lot doit recevoir un numéro.
		Il est donc nécessaire d'identifier chaque fraction d'immeuble, en sorte qu'un même lot ne doit pas regrouper un local principal et un ou plusieurs locaux annexes.
		4. Indication de la quote-part des parties communes comprises dans chaque fraction.
		Chaque lot comprend deux éléments : l'un privatif, l'autre, consistant en une quote-part des parties communes (art. 1er al. 1 de la loi de 1965).
		Cette quote-part comprend celle de la propriété du sol.
		5. Le tableau récapitulatif des lots.
		La désignation des lots telle qu'elle est détaillée dans l'état descriptif de division est obligatoirement résumée dans un tableau incorporé à l'acte lui-même ou annexé à celui-ci. (Art. 71 du Décret du 14 octobre 1955).
		NB : Fin du Régime Transitoire quant aux lots comportant plusieurs fractions.
		droit de la copropriété année 2019-2020
		294
		Fréquemment dans les états descriptifs publiés avant 1956, on trouvait effectivement un lot unique regroupant l'appartement, la cave et éventuellement le garage.
		Pour ne pas contraindre les propriétaires à modifier les états descriptifs, un Régime Transitoire avait été instauré, permettant d'aliéner ces lots "complexes". Depuis un Décret du 21 mai 1979, il a été mis fin à ce régime transitoire. Désormais préalablement à la vente d'un lot "complexe", le notaire doit établir un modificatif à l'état descriptif de division, en sorte que le lot unique d'origine soit annulé et que soient créés des lots distincts pour chaque fraction : un lot appartement, un lot cave et un lot garage dans l'exemple que nous venons d'évoquer. Toutefois la plupart des conservations des hypothèques acceptaient encore la publication d’actes portant sur des lots « complexes »
		6. Le plan annexé à l’État descriptif de division
		Par le passé soit l’immeuble était destiné à être soumis au statut de la copropriété et en ce cas il était d’usage d’annexer un plan d’architecte mentionnant éventuellement les numéros de lots, soit l’immeuble préexistait et soit aucun document graphique n’était joint à l’état descriptif de division, soit on demandait au géomètre d’établir un tel plan.
		Aujourd’hui l’établissement d’un état descriptif de division s’accompagne systématiquement d’un plan graphique distinguant les parties communes (généralement figurées en jaune) des parties privatives (figurées en bleu, vert ou rose). Ce plan est établi par le géomètre et il est remis au notaire qui reçoit le Règlement de copropriété et l’état descriptif de division pour être conservé par lui (ce plan n’est pas déposé au Service de Publicité Foncière en même temps que le Règlement de copropriété et l’état descriptif de division). Si besoin est une copie certifiée sera demandée au notaire.
		Il est bien évident qu’un tel plan peut avoir une importance capitale pour déterminer si à l’origine telle ou telle partie d’immeuble était ou non privative.
		Au demeurant ce plan ne devrait pas être annexé à l’état descriptif de division (simple document de nature administrative) mais bien au Règlement de copropriété lui-même pour lui conférer une valeur contractuelle366

	\subsection{D. NATURE JURIDIQUE DE L’ETAT DESCRIPTIF DE DIVISION}

		La question a été très discutée en doctrine et a fait l'objet de décisions contradictoires : certaines reconnaissant une valeur contractuelle à l’état descriptif de division, d'autres refusant toute valeur
		366 En ce sens voir les propositions du 41\degres Congrès des Géomètres Experts de la Rochelle en 2012 et l’article de Daniel Labetoulle, Commissaire du gouvernement, Revue GEOMETRE, novembre 2012. On peut admettre en effet que ce plan, s’il est dressé par un géomètre conformément à l’article 1er de la loi du 7 mai 1946 avec délimitation de chaque lot, constitue la détermination contractuelle du fonds propriété de chaque copropriétaire.
		droit de la copropriété année 2019-2020
		295
		contractuelle à l’état descriptif de division. Aujourd'hui la question paraît tranchée dans le sens de l'absence de valeur contractuelle de ce document.
		1. La jurisprudence ancienne : valeur contractuelle de l’état descriptif de division
		L'État descriptif de division a d’abord été considéré comme ayant valeur contractuelle :
		A l'appui de cette thèse le fait que le plus souvent l’état descriptif de division est incorporé dans le Règlement de Copropriété, possibilité offerte par l'article 8 de la loi ("Un Règlement conventionnel de Copropriété incluant ou non l’état descriptif de division ...").
		De plus on observera que l'article 3 du Décret précise : " Les règlements, états et conventions énumérées aux articles qui précèdent peuvent faire l'objet d'un acte conventionnel ". Or les états visés sont ceux mentionnés à l'article 1er (état de répartition des charges) et à l'article 2 (état descriptif de division).
		Enfin on observera qu'aux termes de l'article 4 du Décret que nous avons étudié précédemment (cf. \no 79), le règlement de copropriété, l'état descriptif de division et les actes qui les ont modifiés, même s'ils n'ont pas été publiés s'imposent à l'acquéreur ou au titulaire du droit s'il est expressément constaté aux actes visés au présent article qu'il en a eu précédemment connaissance et qu'il a adhéré aux obligations qui en résultent.
		La Cour de Cassation par trois arrêts des 15 mai 1973 (Bull. civ. III \no 341 p 247); 17 mars 1976 (Bull. Civ. \no 125 p. 98); 26 juin 1979 (D. 80 IR p 234) a effectivement reconnu ce caractère contractuel et dit qu’il ne pouvait être modifié qu'à l'unanimité des copropriétaires.
		En conséquence un promoteur se voit refuser la possibilité de modifier la consistance des lots constituant un des bâtiments de la Copropriété : cette modification entraîne division des lots, donc modification de l'état descriptif de division.
		Cette position de la Cour de Cassation a été cependant critiquée avec vigueur par la majorité de la Doctrine qui a essentiellement constaté que ce document est établi pour les besoins de la publicité foncière et ne peut en conséquence être assimilé aux dispositions contractuelles du Règlement de Copropriété.
		2. La jurisprudence actuelle : absence de valeur contractuelle de l’état descriptif de division
		La Cour de Cassation a abandonné sa première position pour affirmer aujourd'hui que l’état descriptif de division n'est qu'un document à caractère administratif sans valeur contractuelle.
		droit de la copropriété année 2019-2020
		296
		Cassation, civ. 3ème 24 mars 1981367
		La question revêt toute son importance à propos du changement d'usage d'une partie privative : Si l'état descriptif est un document contractuel l'affectation du lot donnée par celui-ci ne pourra pas être changée, sauf disposition contraire du Règlement de Copropriété. Dans ce cas, un lot décrit comme garage, devra rester un garage et un grenier ne pourra être transformé en appartement.
		Si l'on considère au contraire, que l'état descriptif n'est qu'un simple document administratif, le changement d'affectation du lot sera possible dès lors que la nouvelle affectation est conforme à la destination de l'immeuble.
		Les arrêts de la cour de cassation (3ème ch) du 10 décembre 1986368 écartent le caractère contractuel de la définition donnée par l'Etat Descriptif quant à l'affectation des lots.
		Toujours à propos de l'affectation des parties privatives, les arrêts les plus récents ont marqué cette hostilité au caractère contractuel de l’état descriptif de division : arrêt du 8 juillet 1992369:
		" L'état descriptif de division dressé seulement pour les besoins de la publicité foncière n'a pas de caractère contractuel ".
		3. Vers une nouvelle évolution jurisprudentielle ?
		Cette jurisprudence est-elle pleinement satisfaisante ? Dénier tout caractère conventionnel à l’état descriptif de division a pour conséquence de remettre en cause la délimitation entre les parties communes et les parties privatives :
		En effet, ce n’est que dans l’état descriptif de division et dans le plan annexé que sont délimités précisément les lots privatifs et les parties communes. L’article 71 du décret du 14 octobre 1955 précise en effet que :
		Chaque fraction doit être identifiée par son emplacement, lui-même déterminé par la description de sa situation dans l'immeuble ou par référence à un plan ou croquis annexé à la minute de l'acte ou de la décision judiciaire
		Mais il est vrai que si par le passé les lots étaient définis avec une grande précision (le nombre de pièces et leur affectation étaient mentionnés) aujourd’hui la description comporte très souvent le mot « local » ou « appartement » sans aucune autre description !
		367 Administrer sept 81 \no 116 p. 32
		368 Administrer mars 1987 p 31 et J.C.P. 1987 IV 60
		369 RTDI 92.364 et le commentaire sans appel de Pierre CAPOULADE dans les IRC de janvier 1993 p. 10
		droit de la copropriété année 2019-2020
		297
		En définitive, ce n’est que le plan annexé qui permet de définir précisément ce qui est partie privative (généralement coloriée en rose, bleu ou vert) et ce qui est partie commune (toujours coloriée en jaune).
		Mais dans l’un et l’autre cas il s’agit bien de documents de nature administrative puisque visés dans le Décret sur la publicité foncière.
		L’arrêt du 3 décembre 2008 dans une affaire où n’existait qu’un état descriptif de division et pas de Règlement de Copropriété370 marque cependant une évolution sensible de la cour de cassation en l’absence de règlement de copropriété :
		Ayant relevé que l'état descriptif de division constituait en l'espèce le seul document de référence définissant la nature de chaque lot et donc sa destination, et retenu, par un motif non critiqué, qu'en l'absence de règlement de copropriété la modification de l'état descriptif de division avait pour objet le changement de destination d'une partie privative, la cour d'appel, répondant aux conclusions, en a exactement déduit que la modification devait être décidée par le syndicat en assemblée générale.
		Il convient de citer dans le sens de cette évolution l’arrêt rendu par la 3ème Chambre Civile de la Cour de Cassation le 6 juillet 2017371 qui reconnaît valeur contractuelle à l’état descriptif de division lorsque cette valeur contractuelle lui est expressément donnée par le Règlement de copropriété, que ses stipulations ne sont pas en contradiction avec les dispositions du Règlement de copropriété et sont plus précises que les stipulations de ce dernier. Cet arrêt est en contradiction formelle avec les décisions antérieurement rendues à propos de l’affectation des lots.
		En l’espèce, aux termes du Règlement de copropriété l'immeuble était destiné à un usage professionnel, de bureaux commerciaux ou d'habitation sans autre précision en ce qui concernait les locaux situés aux étages et combles ; l’état descriptif de division pour sa part définissait les lots du 1er étage à usage professionnel et les lots des étages supérieurs à usage d’habitation : « la cour d’appel a pu en déduire valablement que les lots du 6ème étage ne pouvaient pas être affectés à usage professionnel ».
	
\section{SECTION II - MODIFICATION DE L’ÉTAT DESCRIPTIF DE DIVISION}
	
	\subsection{A. NECESSITE D’ETABLIR UN MODIFICATIF A L’ETAT DESCRIPTIF DE DIVISION}
	
		370 Civ 3\degres Ch 3 décembre 2008, sélectionné, Syndicat des copropriétaires du 22 avenue Saint Jean Baptiste 06000 Nice
		371 3\degres civ., 6 juill. 2017, \no 16-16849, JCP N 2017, \no29, act. 729. publié au Bulletin
		droit de la copropriété année 2019-2020
		298
		Postérieurement à l'établissement de l'état descriptif de division, sa modification peut être rendue nécessaire par suite de modifications :
		o A la consistance de l'immeuble :
		Création de nouveaux lots ou transformation d'une partie commune en partie privative,
		Suppression d'un lot privatif acquis par la Copropriété (pour y installer la loge de la concierge)
		Acquisition d'un terrain contigu par le syndicat (par exemple pour créer un parking commun).
		o A la consistance des lots eux-mêmes :
		Réunion de plusieurs lots en un seul
		Ou au contraire division d'un lot en plusieurs
		o Rectification d'erreur matérielle portant sur l'attribution des quotes-parts d'un lot (le lot était doté de 250/1000 èmes alors qu'il aurait dû en avoir seulement 220).
		Quelle que soit la cause de cette modification il faut que soit établi un acte modificatif à l'état descriptif de division.
		Nous avons indiqué précédemment que chaque lot comprend deux éléments : l'un privatif, l'autre consistant en une quote-part de parties communes et nous venons de voir que chaque fraction d'immeuble doit être parfaitement identifiée en ses deux éléments.
		Dès lors, la modification d'une fraction de l'immeuble implique un travail d'identification complète :
		Par exemple, si les copropriétaires décident d'aliéner la loge de la concierge devenue inutile ensuite de la suppression de la fonction, cette vente devra être précédée d'une modification de l’état descriptif de l'immeuble en sorte que la loge sera constituée en un lot privatif propriété du syndicat qui sera ensuite vendu.
	
	\subsection{B. CONSEQUENCES SUR LES TANTIEMES DE PARTIES COMMUNES AVANT 1979}
	
		Jusqu'à la loi du 2 janvier 1979 cette modification entraînait un changement dans les quotes-parts de parties communes.
		Si nous reprenons l'exemple de la loge de la concierge partie commune, sa transformation en un nouveau lot privatif impliquait de doter ce nouveau lot de tantièmes de parties communes qui diminuaient d'autant les tantièmes de parties communes affectés aux lots préexistants.
		Ceci signifiait qu'une modification à une fraction d'immeuble en copropriété impliquait pratiquement une modification à la consistance de l'ensemble des lots de la Copropriété.
		droit de la copropriété année 2019-2020
		299
		La situation devenait complexe lorsque les lots à modifier étaient grevés d'hypothèques : en effet, et par application de l'article 2166 du code civil : "les créanciers ayant privilège ou hypothèque inscrits sur un immeuble, le suivent en quelques mains qu'il passe, (...)". (C'est ce que l'on appelle le droit de suite).
		Dans l'hypothèse de la loge privatisée, la création d'un lot nouveau avait donc pour conséquence d'amputer chacun des autres lots d'une faible partie des quotes-parts de parties-communes pour les reporter sur le lot nouvellement créé : par exemple le lot appartement qui avait 200/1.000 èmes des parties communes, pouvait se voir "retirer" 10/1.000 èmes reportés sur le lot nouveau constitué par la loge.
		Le droit de suite que nous venons d'évoquer avait alors pour conséquence de permettre au créancier inscrit sur ce lot ainsi amputé, de poursuivre l’exécution de son gage sur le lot nouvellement créé : l'article 6 de la loi interdisant l'aliénation séparée de tout ou partie d'une quote-part.
		En pratique et bien évidemment, le créancier renonçait à exercer son droit de suite sur le lot nouvellement créé, se contentant de ne poursuivre la vente que du lot d'origine ainsi amputé de partie de sa quote-part de parties communes.
		Essentiellement, cet édifice de droit aboutissait à la rédaction de modificatifs particulièrement délicats à établir ... et dont la justification passait largement au-dessus de la compréhension des copropriétaires pour qui un lot de copropriété est essentiellement un local dont il a la "propriété", les parties communes étant perçues non comme une propriété indivise, mais comme des lieux ... à jouissance partagée.
		C'est cette idée qui va inspirer le législateur de 1979 : "Il suffisait d'affirmer que la partie privative est l'élément fondamental de chaque lot et que la quote-part de parties communes n'est qu'un accessoire pour que tout le problème disparaisse". Comme il est écrit dans l'Instruction du 1er août 1979372.
	
	\subsection{C. REFORME DE LA LOI DU 02 JANVIER 1979 : ARTICLE 6-1 DE LA LOI DU 10 JUILLET 1965}
	
		Cette solution est ainsi exprimée dans l'article 6-1 alinéa premier de la loi de 1965 modifiée par la loi 79-2 du 2 janvier 1979 :
		" En cas de modification dans les quotes-parts des parties communes afférentes aux lots, quelle qu'en soit la cause, les droits soumis ou admis à publicité dont les lots font l'objet s'éteignent sur les quotes-parts qui en sont détachées et s'étendent à celles qui y sont rattachées ".
		372 Publiée notamment au JCP N 1979 - PRATIQUE p. 469 à 489
		droit de la copropriété année 2019-2020
		300
		La même loi du 2 janvier 1979 a également modifié le Code civil en ajoutant un article 2148-1 ainsi rédigé :
		" Pour les besoins de leur inscription, les privilèges et hypothèques portant sur les lots dépendant d'un immeuble soumis au statut de la copropriété sont réputés ne pas grever la quote-part de parties communes comprise dans ces lots.
		Néanmoins, les créanciers inscrits exercent leurs droits sur ladite quote-part prise dans sa consistance au moment de la mutation dont le prix forme l'objet de la distribution; cette quote-part est tenue pour grevée des mêmes sûretés que les parties privatives et de ces seules sûretés ".
		En d'autres termes, peu importe la quote-part de parties communes dont est doté le lot sur lequel l'inscription est prise, cette quote-part pouvant varier entre l'inscription et la vente du lot. Mais au moment de la vente du lot, l'inscription du créancier ne porte que la quote-part existant à cette époque.
		En pratique, ce "découplage" entre partie privative et quote-part de parties communes, a permis de simplifier considérablement la procédure de création de nouveaux lots. Désormais, il n'est plus nécessaire de prendre à l'ensemble des lots une quote-part de parties communes dont on va doter le nouveau lot. Il suffit de laisser chaque lot dans sa consistance et de créer de nouveaux tantièmes de parties communes pour le lot créé :
		Reprenons l'hypothèse de la loge de concierge, partie commune privatisée :
		Avant la réforme et à supposer que l'on ait eu cinq lots privatifs dotés chacun de 200/1.000 èmes des parties communes, on créait un sixième lot doté, par exemple de 50/1.000 èmes et l'on modifiait les lots 1 à 5 en sorte que chacun ne possède plus que 190/1.000 èmes des parties communes.
		Désormais le dénominateur sera calculé en 1.050/1.050èmes : le nouveau lot six est bien doté de 50/1.050èmes, tandis que chacun des lots 1 à 5 est désormais doté de 200/1.050 èmes.
		Certes, il conviendra d'établir un nouvel état descriptif de division pour tenir compte de cette modification du dénominateur commun, mais disparaît la difficulté de déterminer pour chaque lot la quote-part dont il doit être amputé.
	
\section{SECTION III - SECTION III. MODALITES DE CALCUL DES TANTIEMES DE PROPRIETE}
	
	droit de la copropriété année 2019-2020
	301
	L'État descriptif de division divise l'ensemble immobilier en lots qui, conformément à l'article ler de la loi du 10 juillet 1965 comprennent chacun une partie privative et une quote-part des parties communes. Cette détermination a des conséquences importantes pour la vie ultérieure de la copropriété (A).
	La valeur totale de l'immeuble étant répartie entre tous les lots, les tantièmes attribués à chaque lot sont fixés en fonction de la valeur relative de chaque partie privative par rapport à la valeur de l’ensemble des parties privatives de l’immeuble (B).
	
	\subsection{A. PRINCIPE : LES TANTIEMES DE PROPRIETE DEPENDENT DE LA VALEUR RELATIVE DU LOT}
	
		Article 5 de la loi du 10 juillet 1965373
		" Dans le silence ou la contradiction des titres, la quote-part des parties communes, tant générales que spéciales, afférente à chaque lot est proportionnelle à la valeur relative de chaque partie privative par rapport à l'ensemble des valeurs desdites parties, telles que ces valeurs résultent lors de l'établissement de la copropriété, de la consistance, de la superficie et de la situation des lots, sans égard à leur utilisation »
		L'article 5 énumère trois éléments :
		- La consistance
		- La superficie
		- La situation des lots
		Critères qui doivent être appréciés lors de l'établissement de la Copropriété374.
		1. La superficie.
		C'est le critère essentiel : il est bien évident que la valeur d'un local est avant tout fonction de sa superficie.
		Par exemple la S.H.O.N. (épaisseur des murs extérieurs comprise) ou la surface privative (épaisseur des cloisons comprise) ou surface réelle entre mur et cloison.
		373 Le texte d’origine a été modifié par l’Ordonnance du 30 octobre 2019 pour ajouter les mots « tat générales que spéciales », ceci dans un souci de cohésion avec les nouvelles dispositions de l’article 4 traitant des parties communes spéciales.
		374 On renverra sur ce point à l’ouvrage déjà ancien MM ARNAUD et BOUYEURE intitulée Millièmes et Charg374es de Copropriété (Actualité Juridique 1982) ou au livret beaucoup plus récent de l’Ordre des Géomètres Experts La Copropriété (Repères Experts) qui propose un mode de calcul des principaux éléments de l’article 5.
		droit de la copropriété année 2019-2020
		302
		Bien évidemment peu importe le mode de calcul retenu pour la superficie dès lors qu’il s’agit d’un mode de calcul identique pour l’ensemble des parties communes.
		2. La consistance.
		Un arrêt du 27 mai 1991 de la Cour de Paris définit ainsi la consistance :
		"Pour le calcul de répartition des tantièmes de parties communes, il est tenu compte de la consistance du lot, c'est à dire de l'existence ou non de balcons et de jardins et sa composition en habitations individuelles ou collectives, mais aussi de la différence de nature entre une cave, un garage ou un appartement et les éléments de confort ..."
		En fait il faut tenir compte de ces critères et d'autres375 :
		La qualité de la construction.
		P ex : 0,95 pour un bâtiment en briques et de 1 pour un bâtiment de maçonnerie pierre de taille.
		La hauteur sous plafond.
		P ex : 0,86 pour une hauteur sous plafond de 1,80 mètre et de 1,25 pour une hauteur de 4,00 mètres.
		La disposition des appartements (Bien ou mal distribué).
		C'est ce que les auteurs précités appellent la distribution intérieure. Il est évident que les grands appartements avec des locaux mal distribués (la cuisine à l'extrémité d'un long couloir) ont un coefficient de disposition inférieure aux locaux où les dégagements sont réduits au minimum.
		La composition du lot (avec ou sans balcons, loggias, etc.)
		L'aménagement ou le confort de chaque lot (lot desservi ou non par l'ascenseur, etc.)
		375 Cf une des rares études publiées à ce sujet : MM AZAIS, GOUVERNAIRE et MORAND la détermination des quotes-parts de droits et des diverses catégories de charges de copropriété, AJPI 1993 page 235, dont sont tirés les exemples figurant ici
		droit de la copropriété année 2019-2020
		303
		La nature du lot (principal ou secondaire.)
		Nos auteurs précités donnent une longue liste des coefficients de nature (selon leur expression) :
		Appartement = 1,00
		Cave en sous-sol = 0,10 à 0,15
		Grenier logeable = 0,25 à 0,30
		Emplacement de voiture découvert = 0,10 à 0,20
		Box = 0,40 à 0,45
		Balcon = 0,10 à 0,25
		Terrasse = 0,20 à 0,25
		Local secondaire (superficie supérieure à 7 m2 = 0,50 à 0,70
		Débarras (superficie inférieure à 7 m2 = 0,10 à 0,15)
		3. La situation.
		CA paris 27 mai 1991 prec.
		"Il est tenu compte également de la situation du lot. De ce fait, l'exposition, l'orientation, la vue et l'éclairement concourent également à définir la valeur du lot ".
		La vue
		De la vue la plus courte ou la plus "dépréciante" à la vue la plus dégagée ou « valorisante », la pondération sera comprise entre 0,6 et 1,1
		L'éclairement,
		L'exposition ou orientation
		Le coefficient variera entre 0,80 pour une vue plein Nord à 1,00 pour une vue plein Sud. (A l'évidence leur technique n'est pas destinée aux Territoires d'Outre-mer ou l'exposition Nord a plus de valeur que l'exposition Sud).
		La tranquillité,
		L'élévation du lot
		C'est le coefficient d'étage, partant du principe qu'un lot a d'autant plus de valeur, à superficie et situation identiques, qu'il est situé en élévation.
		droit de la copropriété année 2019-2020
		304
		Nos auteurs précités (Morand, Azaïs et Gouvernaire) distinguent l'immeuble avec escalier visible au rez-de-chaussée et l'immeuble dont l’escalier du rez-de-chaussée est situé derrière une porte :
		Etage
		Avec escalier visible
		Sans escalier visible
		3\degres sous-sol
		0,8
		0,85
		2\degres sous-sol
		0,83
		0,90
		1\degres sous-sol
		0,86
		0,95
		Rez-de-chaussée
		0,90
		0,90
		1er étage
		0,94
		0,97
		2\degres étage
		1,00
		1,00
		3\degres étage
		1,03
		1,03
		4\degres étage
		1,06
		1,06
		5\degres étage
		1,09
		1,09
		6\degres étage
		1,12
		1,12
		7\degres étage
		1,14
		1,14
		8\degres étage
		1,16
		1,16
		9\degres étage
		1,18
		1,18
		10\degres étage au 15\degres étage
		+ 0,2 par étage
		+ 0,2 par étage
		Au-dessus invariable
		1,25
		1,25
		Mais il faut essentiellement relever qu'il existe dans ces modes de calcul une part certaine de subjectivité due aux idées personnelles des experts en estimation.
	
	\subsection{B. PORTEE DES CRITERES DE L’ARTICLE 5 DE LA LOI}
	
		1. Liberté de répartition des tantièmes de propriété.
		Cet article n'est pas d'ordre public :
		- " Dans le silence ou la contradiction des Titres ..."
		- L'article 5 ne figure pas dans l'énumération de l'article 43 selon lequel "Toutes clauses contraires aux dispositions des articles 6 à 37, 42 ...sont réputées non écrites."
		Donc, rien n'interdit à l'auteur de l'état descriptif de division de répartir la quote-part des parties communes entre les différents lots sans respecter les trois critères de l'article 5 (consistance, superficie et situation des lots sans tenir compte de leur affectation).
		On trouve souvent dans des Règlements de Copropriété anciens une répartition des quotes-parts de parties communes qui, contrairement aux dispositions supplétives de l'article 5 de la loi, majorent la valeur relative des locaux commerciaux ... eu égard à leur affectation commerciale, c'est à dire en fonction de la valeur vénale des lots.
		Il est vrai que l'on a toujours considéré qu'un lot commercial, à superficie égale, avait beaucoup plus de valeur qu'un local d'habitation du fait que les loyers commerciaux ont une valeur au mètre carré beaucoup plus importante. D'où la tentation pour celui qui rédige l’État descriptif de division de
		droit de la copropriété année 2019-2020
		305
		répartir la valeur des parties communes de l'ensemble au prorata de la valeur locative de chaque lot.
		Quoi qu'il en soit l'auteur du Règlement de Copropriété aura toute latitude de répartir les tantièmes de copropriété comme il le souhaite, en tenant compte ou non des critères de l'article 5 de la loi, en tenant compte ou non de l'utilité des lots.
		2. Impossibilité d’une révision judiciaire des tantièmes de propriété.
		En conséquence de la liberté de répartir les tantièmes de copropriété dans le respect des dispositions de l'article 5 ou sans respecter ces dispositions, la répartition des tantièmes de copropriété est intangible : elle ne peut être modifiée ni par voie de justice, ni par décision d’Assemblée Générale (sauf unanimité).
		La Cour de Cassation ne manque pas de rappeler ce principe lorsqu'elle est saisie d'une demande judiciaire de modification des tantièmes de copropriété :
		Civ 3\degres, 6 mars 1991 : "Une cour d'appel a légalement justifié sa décision en retenant que la quote-part des parties communes afférente à chaque lot étant intangible et ne relevant des dispositions de l'article 5 de la loi du 10 juillet 1965 que dans le silence ou la contradiction des titres, la contestation portant sur l'évaluation comparée de la valeur des lots devait être rejetée". (JCP N 91, II, 336).
	
	\subsection{C. CAS DE MODIFICATION DES TANTIEMES DE PROPRIETE.}
	
		1. Division d'un lot.
		Lorsqu'un copropriétaire divise un lot d'origine, il s’en suit nécessairement une modification des tantièmes de copropriété.
		C'est ce qu'a décidé la Cour de Cassation Civ 3\degres 22 février 1989 :
		" Le fractionnement d'un lot de copropriété entraîne la modification de ses éléments constitutifs et par conséquent sa disparition et son remplacement par les différents lots issus de la division du lot primitif ".
		droit de la copropriété année 2019-2020
		306
		En effet les tantièmes antérieurement affectés à un seul lot devront être "éclatés" entre les deux lots (si ce n'est plus) issus de la division du lot d'origine.
		C'est d'ailleurs le principe même du lot transitoire que nous avons précédemment rencontré, lorsque par exemple le lot 101 constitué par le droit d'édifier un bâtiment de dix étages dotés de 2000/10.000 èmes des tantièmes généraux sera divisé en lots numérotés de 101 à 180 qui se partageront les 2.000/ 10.000 èmes du lot d'origine annulé.
		Toutefois cet éclatement se fera à la seule initiative du copropriétaire auteur de la division, sans intervention du syndicat des copropriétaires qui ne pourra s'y opposer du fait que l’état descriptif n'a pas de caractère conventionnel comme rappelé précédemment (sous réserve toutefois qu’il n’existe pas dans le règlement de copropriété de clause interdisant la division du lot : une telle clause devant être justifiée par la destination de l’immeuble et sous réserve également que la division ne porte pas atteinte à la destination de l’immeuble).
		2. Création de nouveaux lots.
		Si l'on créée de nouveaux lots non prévus dans l'état descriptif de division d'origine, il faudra bien en tenir compte dans la valeur relative de l'ensemble des lots :
		Il en est ainsi lorsque le Syndicat des Copropriétaires cède une partie commune à un copropriétaire, ou lorsqu'il autorise une surélévation, cette cession de partie commune ou de droits de construire, pour être identifiable selon les critères dégagés à propos du fichier immobilier nécessitera la création d'un nouveau lot (cf. \no 122) qui comportera une partie privative et des tantièmes de parties communes.
		Par exemple dans une copropriété dont les 30 lots constituant les parties privatives totalisent 1000/1000 èmes des parties communes générales, si le syndicat veut céder un couloir commun qui relie deux chambres de service (lots \no 20 et 21 dotés chacun de 30/1000 èmes des parties communes générales) appartenant à un seul copropriétaire, il fera dresser par un géomètre un plan de division créant un nouveau lot qui viendra à la suite des lots existant et qui sera doté de tantièmes de parties communes en fonction de la superficie, de la consistance et de la situation de ce nouveau lot. Ce nouveau lot pourra être dans notre exemple le lot 31 doté de 10/1010 èmes des parties communes générales, en sorte que désormais les tantièmes généraux seront répartis en 1010/1010 èmes au lieu des 1000/1000 èmes d'origine. Ce lot ainsi créé sera ensuite cédé au copropriétaire intéressé qui pourra soit ajouter le lot 31 à ses lots 20 et 21, soit annuler les trois lots 20, 21 et 31 pour créer au lieu et place un lot unique nouveau, le lot 32 doté de 70/1010 èmes des tantièmes généraux.
		La question se pose de savoir si ce modificatif à l’État descriptif de division doit être approuvé par l’assemblée générale.
		droit de la copropriété année 2019-2020
		307
		On pourrait penser en effet que l’état descriptif de division étant un document purement administratif376 l’assemblée générale doit certes approuver la cession de la partie commune à la double majorité de l’article 26 mais n’a pas à se prononcer sur la modification de l’état descriptif de division tenant compte de la création du nouveau lot.
		La réponse est cependant contraire : non seulement l’état descriptif de division doit être modifié mais la cour de cassation377 a décidé que cette modification ne pouvait se faire à la majorité de l’article 24 de la loi : Civ. 3\degres 29 oct. 2003
		Ne peut être adoptée à la majorité simple de l’article 24 de la loi la modification à l’État descriptif de division du fait de la création d’un nouveau lot ayant modifié les tantièmes de copropriété des demandeurs.
		En l’espèce l’assemblée générale avait voté la privatisation d’une partie commune avec création d’un nouveau lot doté de 8 tantièmes de copropriété, et approuvé la modification de l’État descriptif de division. Cette dernière résolution ayant été adoptée à la majorité de l’article 24 de la loi. Des copropriétaires, les époux Joab, demandaient l’annulation au motif que leurs tantièmes de propriété se trouvaient modifiés puisqu’ils détenaient préalablement 812/10.000èmes des parties communes alors que désormais ils ne possèdent plus que 812/10.008èmes des mêmes parties communes.
		Quelle est donc la majorité nécessaire ? L’unanimité puisqu’on touche aux tantièmes de propriété (ce qui rendrait quasiment irréalisable une cession de parties communes) ou la majorité de l’article 26 effectivement nécessaire à la cession des parties communes, la modification de l’État descriptif de division étant « l’accessoire » de cette cession ? Le bon sens (et non pas les dispositions légales) nous incite à conclure en ce sens.
		La division d’un lot relève de la liberté du copropriétaire intéressé, sauf en ce qui concerne la modification des charges : le modificatif à l’état descriptif de division, qu’il soit inclus ou non dans le règlement de copropriété, n’a pas à être approuvé par l’assemblée générale378.
		3. Disparition de lots.
		Une autre hypothèse se rencontre dans les grands ensembles immobiliers en cas de non réalisation d'un bâtiment prévu à l'origine.
		Le promoteur a-t ’il la faculté d'annuler purement et simplement le lot d'origine, ce qui aura pour conséquence de réduire le nombre des tantièmes généraux ?
		376 Cf ci-dessus La nature de l’ État descriptif de division
		377 Loyers et Copropriété mars 2004 \no 55
		378 Cass. Civ. 3e 7 octobre 2009, Pourvoi \no 08-18133, non publié
		droit de la copropriété année 2019-2020
		308
		Dans une affaire Volney Invest c/ syndicat des copropriétaires rue Raffet , la Cour de cassation a exclu un tel abandon – par référence à l’article 544 code civil - sans décision préalable de l’assemblée générale à la majorité de l’article 26379 :
		Civ. 3\degres Ch. 7 avril 2004
		La société ayant mis l’immeuble existant en copropriété avait créé un lot « droit de construire » plusieurs niveaux de parkings par affouillement sous l’immeuble, représentant plus de 20 % des tantièmes de la copropriété.
		Tenue de commercialiser l’immeuble avant d’avoir réalisé cet affouillement, elle voulut s’exonérer des obligations relatives à ce lot devenu inconstructible. L ‘assemblée générale ayant refusé l’annulation de ce lot, elle avait imaginé d’en signifier l’abandon à la copropriété380.
		La cour d’appel (avec rejet du pourvoi) a déclaré cet abandon impossible ; on retiendra l’attendu de la cour de cassation tout à fait significatif :
		« Ayant exactement relevé que le caractère contractuel du règlement de copropriété impliquait qu'un copropriétaire ne pouvait se dégager unilatéralement de ses obligations sans le consentement des autres copropriétaires, et constaté que les acheteurs des appartements rénovés par le marchand de biens avaient procédé à leur acquisition en tenant compte des charges qu'ils devaient acquitter et qu'ils ne pouvaient voir augmenter celles-ci de façon conséquente parce que leur vendeur avait estimé que la création d'emplacement des parkings en sous-sol serait moins rentable que ce qu'il avait cru pouvoir en espérer, la cour d'appel a retenu, à bon droit, que le syndicat des copropriétaires ne pouvait procéder à aucune acquisition immobilière sans que fût intervenue une décision de l'assemblée générale statuant à la majorité prévue à l'article 26 a) de la loi du 10 juillet 1965 »
		379 Civ. 3\degres Ch. 7 avril 2004, Pourvoi \no 02-14670, publié au Bulletin
		380 L’abandon est une procédure prévue à l’art. 544 du code civil qui permet à tout propriétaire d’immeuble de l’abandonner au profit de l’État