\chapter{La location du lot}

Il faut envisager successivement le droit de louer, c'est-à-dire la mesure dans laquelle le copropriétaire a la faculté de donner son lot en location, puis les effets produits par cette location.

\section{Le droit de louer}

	\subsection{La liberté de louer son lot}
	
		Le droit de louer, c'est-à-dire de retirer les fruits de son lot, est une prérogative appartenant à tout propriétaire. L'opération de location ne peut directement ou indirectement être interdite ni par le règlement de copropriété, ni par une décision d'assemblée générale.
		
		Ainsi, est considérée comme nulle la clause subordonnant la conclusion du bail à l'autorisation du conseil syndical\footnote{Lyon, 22 janvier 1969, A.J.P.I. 1969. 418, note BOUYEURE}. Il en serait incontestablement de même d'une clause exigeant l'accord du syndic préalablement à toute location.
		
		De même, sont réputées non-écrites les limitations à la liberté de louer qui ne sont pas justifiées par la sauvegarde de la destination de l'immeuble.
		
		Par exemple, l'interdiction générale de toute location sera presque toujours réputée non-écrite car il faudrait des circonstances tout-à-fait exceptionnelles pour qu'une telle stipulation puisse trouver sa justification dans une référence à la destination de l'immeuble.
	
	\subsection{Les restrictions licites pouvant figurer dans le règlement de copropriété}
	
	En revanche, cette destination de l'immeuble peut rendre licites certaines limitations :
	\begin{itemize}
		\item clause qui, dans un immeuble résidentiel, stipule que les lots ne pourront en aucun cas être divisés en vue de la location\footnote{Paris, 19 juin 1985, I.R. 452, obs. GIVERDON} ;
		\item clause qui, dans un immeuble de grand standing, interdit la location des chambres de service à des personnes étrangères à la copropriété\footnote{Paris, 12 février 1976, D.1977 I.R. 42} ;
		\item clause interdisant les locations meublées dans un immeuble d’habitation. Ce type de location à caractère commercial multiplie les allées et venues dans l'immeuble et sont source d'insécurité.
	\end{itemize}
	
\section{Le contrat de bail}
	
	\subsection{L’information du locataire sur la destination de l’immeuble}
	
		Outre les conditions de validité tenant au droit commun, le contrat de bail doit se conformer à la destination de l'immeuble en copropriété : ainsi dans un immeuble à usage d'habitation exclusive, il est interdit de louer à des fins professionnelles ou commerciales. Le copropriétaire qui agirait ainsi devrait indemniser les autres membres du syndicat de leur préjudice, mais si les parties avaient conclu en connaissance de cause, le bail demeurerait valable dans les rapports entre elles (bailleur / locataire).
		
		Cependant, la responsabilité du bailleur envers son locataire pourrait être engagée si le règlement comportait des contraintes dommageables pour lui et dont il aurait fallu qu'il fût informé.
		
		De plus, la nullité du contrat de location pourrait être envisagée si les stipulations du règlement de copropriété faisaient obstacle à une activité essentielle pour le locataire et formellement prévue par le contrat : logement loué à usage pour partie professionnel alors que le règlement contient une clause d'habitation exclusivement bourgeoise\footnote{AUBERT et BIHR, La location d'habitation, 1990, \no67 }.
		
		Afin de faciliter les actions directes du Syndicat contre le locataire du lot, et sans doute un meilleur contrôle des activités exercées (lesquelles doivent être conforme à la destination de l’immeuble), la loi \no 2018-1021 du 23 novembre 2018 dite ELAN avait prévu la communication au syndic des coordonnées de son locataire par le bailleur « avec l’accord de celui-ci ». Toutefois, la disposition a été censurée par le Conseil constitutionnel, comme cavalier parlementaire. Le syndic ne peut donc obtenir du propriétaire, ni la copie du bail ,ni l’identité du locataire, sauf si une clause du règlement de copropriété le prévoit.
	
	\subsection{L’obligation de communication du règlement de copropriété a la charge du bailleur}

		L'article 3 de la loi du 6 juillet 1989 (dite loi \nom{Méhaignerie}) dispose\footnote{La loi ALUR a réécrit l’article 3 de la loi du 6 juillet 1989, mais cette phrase a été conservée dans sa rédaction d’origine.} :
		\begin{quote}
			<< {\itshape lorsque l'immeuble est soumis au statut de la copropriété, le copropriétaire bailleur est tenu de communiquer au locataire les extraits du règlement de copropriété concernant la destination de l'immeuble, la jouissance et l'usage des parties privatives et communes et précisant la quote-part afférente au lot loué dans chacune des catégories de charges,
			
			\lips
			
			Le bailleur ne peut pas se prévaloir de la violation du présent article. } >>
		\end{quote}
		Ces extraits peuvent être indifféremment annexés au contrat de bail ou remis dans les mains du locataire. Dans ce cas, il est souhaitable que le contrat de bail mentionne que cette communication a eu lieu. Le bailleur pourra aussi demander à son locataire un reçu signé indiquant que les extraits ont été remis.
		
		Les sanctions du défaut de communication ne sont pas précisées par la loi; si ce n'est l'interdiction pour le bailleur de se prévaloir de la violation de ce texte.
		
		Il faut d'abord souligner que la sanction ne consiste pas dans l'inopposabilité du règlement au locataire. Ce dernier s'y trouve, on le sait, automatiquement soumis\footnote{Cf. Supra les effets du règlement de copropriété}.
	
	\subsection{La question des locations de courte durée}
	
		L’article 16 de la loi ALUR\footnote{C’est le seul texte concernant le droit de la copropriété qui a été remis en cause par les Députés devant le Conseil Constitutionnel qui y voient une atteinte à la propriété de l’immeuble pour le cas où l’assemblée générale refuserait son accord.} impose un certain nombre de contraintes nouvelles à la location dite « de courte durée » qui concerne le plus souvent les appartements ou studios meublés joliment situés dans les villes touristiques que les propriétaires, par l’intermédiaire du net ou d’agences spécialisées, mettent à la disposition de touristes pour une durée qui varie le plus souvent d’une nuit à une semaine.
		Cet article précise dans un premier temps que cette mise en location constitue un changement d’usage au sens de l’article L 631-7 du CCH.
		
		Il précise en deuxième lieu que cette pratique peut être soumise à l’autorisation du Conseil Municipal ; ladite autorisation fixant les modalités de cette mise en location (article 631-7-1 A nouveau du CCH), sauf dans le cas où cette mise à disposition porte sur le local qui constitue l’habitation principale du loueur.
		
		Enfin, il convient de signaler que la loi ALUR, article 19, adoptée par le Parlement avait prévu, si le local mis à disposition pour location de courte durée est un lot de copropriété que le syndicat pourrait décider que cette mise à disposition nécessiterait son accord préalable. Mais cette disposition a été censurée par le Conseil Constitutionnel : que ce texte « a ainsi, dans des conditions contraires à l'article 2 de la Déclaration de 1789, permis à l'assemblée générale des copropriétaires de porter une atteinte disproportionnée aux droits de chacun des copropriétaires ».
		
		Rappelons également les dispositions de l’article L 324-1-1 du Code du Tourisme :
		\begin{quote}
			I.-Toute personne qui offre à la location un meublé de tourisme, que celui-ci soit classé ou non au sens du présent code, doit en avoir préalablement fait la déclaration auprès du maire de la commune où est situé le meublé.
			
			Cette déclaration préalable n'est pas obligatoire lorsque le local à usage d'habitation constitue la résidence principale du loueur, au sens de l'article 2 de la loi \no 89-462 du 6 juillet 1989 tendant à améliorer les rapports locatifs et portant modification de la loi \no 86-1290 du 23 décembre 1986.
			
			II.-Dans les communes où le changement d'usage des locaux destinés à l'habitation est soumis à autorisation préalable au sens des articles L. 631-7 et L. 631-9 du code de la construction et de l'habitation une délibération du conseil municipal peut décider de soumettre à une déclaration préalable soumise à enregistrement auprès de la commune toute location pour de courtes durées d'un local meublé en faveur d'une clientèle de passage qui n'y élit pas domicile.
			
			Lorsqu'elle est mise en œuvre, cette déclaration soumise à enregistrement se substitue à la déclaration mentionnée au I du présent article.
			
			Un téléservice permet d'effectuer la déclaration. La déclaration peut également être faite par tout autre moyen de dépôt prévu par la délibération susmentionnée.
			
			Dès réception, la déclaration donne lieu à la délivrance sans délai par la commune d'un accusé-réception comprenant un numéro de déclaration.
			
			Un décret détermine les informations qui peuvent être exigées pour l'enregistrement.
		\end{quote}
		
		Et les articles D 324-1 et D 324-1-1 du Code de Tourisme modifiés pat le Décret \no 2017-678 du 28 avril 2017 sur les modalités de la Déclaration en mairie. Ce décret définit les données exigibles par les communes en cas de déclaration obligatoire des loueurs en meublé touristique.
		
		Sur la question on pourra se reporter utilement à la Réponse Ministérielle \no 97541 du 21 mars 2017, p. 2459 (JCP N – \no 14-15 – 7 avril 2017 p. 18) qui constitue une synthèse appréciable des difficultés engendrées par la location touristique de courte durée.
		
		Lorsque les communes (Paris, Lyon, Marseille) exigent une déclaration pour changement d'affectation, on note une tendance des juges du fond admettre que, si la location meublée touristique correspond aux conditions définies par la commune, il ne s'agit plus d'un local à destination d'habitation. Cela peut également poser des difficultés en termes de conformité de l'affectation du lot au regard de la destination de l'immeuble tel que défini par le règlement de copropriété.
		
		La cour de cassation a évolué dans le même sens. En effet, elle a d’abord considéré que l’exercice de cette activité n’était pas incompatible avec une destination de l’immeuble dite « simplement bourgeoise », c’est-à-dire tolérant l’exercice d’activités libérales.

		Un arrêt (non publié) de la 3ème chambre civile est revenue sur ce principe en 2018,assimilant plutôt cette activité à une affectation commerciale :
		\begin{quote}
			\textbf{Civ. 3ème, 8 mars 2018 – \no 14-15864 – non publié}
			
			« {\itshape En présence d’un règlement de copropriété autorisant seulement un usage mixte habitation-professionnel avec clause obligeant les bailleurs à aviser le syndic de l’existence d’un bail, le juge du fond peut souverainement décider que l’installation d’occupants pour de très brèves – ou plus longues - périodes dans des « hôtels studios meublés » avec prestations de services est contraire à la destination résidentielle de l’immeuble} »
		\end{quote}
	
\section{Les effets de la location}
	
	\subsection{Effets de droit commun du bail}
	
		Le bailleur a l'obligation de procurer la jouissance du bien loué, de garantir le locataire contre les troubles et d'entretenir le bien.
		
		Le locataire est tenu de payer les loyers et les charges récupérables, d'user normalement du bien loué et de procéder aux réparations locatives.
		
		En ce qui concerne la récupération des charges dues à la suite d'un bail de locaux d'habitation ou mixtes, il faut souligner qu'elle ne peut être exigée que sur justification et en contrepartie :
		\begin{itemize}
			\item des services rendus liés à l'usage des différents éléments de la chose louée;
			\item des dépenses d'entretien courant et des menues réparations sur les éléments d'usage commun dans la chose louée;
			\item du droit au bail et des impositions qui correspondent à des services dont le locataire profite directement (art.23 L.6 juillet 1989).
		\end{itemize}
		La liste de ces charges est fixée par décret en Conseil d'Etat : il s’agit du Décret du 9 novembre 1982 “ pris en application de l’article L. 442-3 du Code de Construction et de l’Habitation. Par conséquent, toutes les charges de copropriété ne sont pas nécessairement répercutables sur le locataire.
	
	\subsection{Rôle des locataires au sein de la copropriété}

		\subsubsection{Le locataire n’est pas substitué au propriétaire du lot}
		
			En principe, le locataire n'est pas membre du syndicat. Il n'a aucun lien de droit avec celui-ci.C'est pourquoi il ne participe pas en principe à la vie de la copropriété :
			\begin{itemize}
				\item Il n'est pas membre de l'assemblée générale; il ne peut y participer en son nom ni voter les délibérations qui lui sont soumises. Il ne peut davantage être membre du Conseil Syndical\footnote{ L’Avant Projet de loi de 1997 prévoyait cependant cette faculté pour les locataires.}.
				\item Il n'a pas qualité pour demander l'inscription d'une question à l'ordre du jour, ni pour agir en nullité de décisions votées par l'assemblée générale.
				\item Il ne peut demander l'autorisation d'exécuter des travaux sur les parties communes\footnote{PARIS 23\degres, 4/06:91 - Loyers et Copropriété octobre 1991}.
				\item Il ne peut faire tierce-opposition aux décisions de l’A.G., car il ne peut avoir plus de droits que le copropriétaire\footnote{Civ. 3\degres 18/11/92, Loyers et Copropriété oct. 1991}.
			\end{itemize}
		
			Corrélativement, il n'est pas redevable à l'égard du syndicat du paiement des charges communes qui restent dues par le bailleur en tant que copropriétaire. Le syndic ne peut donc l'assigner directement en paiement des charges, même celles qui font partie des charges récupérables.
			
			De même, le locataire n'est pas responsable des tiers (le médecin pour ses patients qui stationnent abusivement sur les parties communes\footnote{Civ. 3\degres 12/06/91, Bull Civ III \no 171}).
		
		\subsubsection{La représentation des locataires au sein des instances du Syndicat des Copropriétaires}
		
		Il a été institué certaines structures permettant aux locataires d'être informés du fonctionnement de la copropriété.
		
		Ces règles figurent aujourd'hui à l'article 44 dernier alinéa de la loi du 23 décembre 1986 (dans sa rédaction issue de la loi du 6 juillet 1989) :
		\begin{itemize}
			\item Il peut être constituée dans chaque immeuble ou groupe d'immeubles une association de locataires regroupant au moins 10\% des locataires ou des associations siégeant à la Commission Nationale de Concertation.	
			\item Ces associations désignent au syndic par lettre recommandée avec accusé de réception le nom de trois au plus de leurs représentants choisis parmi les locataires de l'immeuble.
			\item Ces représentants ont accès aux différents documents concernant la détermination et l'évolution des charges locatives.
			\item A la demande de ceux-ci, le syndic les consulte chaque semestre sur les différents aspects de la gestion de l'immeuble.
			\item Ils peuvent assister aux assemblées générales et formuler des observations sur les questions inscrites à l'ordre du jour. Mais ils n'ont pas le droit de vote.
			\item Le syndic informe les représentants de l'association par lettre recommandée avec demande d'accusé de réception de la date, de l'heure, du lieu et de l'ordre du jour de l'assemblée générale.
			\item Enfin, dans chaque bâtiment d'habitation, un panneau d'affichage doit être mis à la disposition des associations pour leurs communications portant sur le logement et l'habitat, dans un lieu de passage des locataires.
		\end{itemize}
	
		Cette loi a été complétée par une loi du 13 décembre 2000 (SRU) qui ajoute aux associations de locataires représentant ces derniers vis-à-vis du syndicat des copropriétaires les groupements de locataires affiliés à une organisation siégeant à la Commission Nationale de Concertation qui se voient reconnaître les mêmes droits que l’association de locataire\footnote{MM Lafond et Stemmer considèrent cependant que la rédaction de la loi de 2000 ne permet pas à ces groupements de locataires de participer aux assemblées générales (Cf op. cité p. 742)}.
		
		Toutefois, le locataire à titre individuel ne possède aucun de ces droits qui sont nécessairement exercés par un représentant d'association (représentant qui peut être lui-même locataire de l'immeuble concerné) ou membre d’un groupement de locataires.
	
	\subsection{Les obligations du locataire vis-a-vis du syndicat et leurs sanctions}
	
		\subsubsection{Opposabilité du règlement de copropriété}
		
			Le locataire vit dans l'immeuble et il n'est pas possible de ne pas tenir compte de ce fait.
			
			C'est pourquoi on sait que le règlement de copropriété est opposable de plein droit au locataire ainsi que les décisions du syndicat. (civ.3ème 14 avril 2010, Administrer août 2010, p40)
		
			De la même manière, le locataire est tenu de laisser exécuter dans les parties privatives louées les travaux régulièrement décidés par l'assemblée en vertu des articles 25 e), g), h), i), 26 et 30 de la loi du 10 juillet 1965.
			
			Il s'agit des travaux visés à l'article 9 de la loi : travaux légalement ou réglementairement obligatoires, travaux d'économie d'énergie, de mise aux normes de salubrité et de sécurité ou d'équipement, travaux d'accessibilité aux personnes handicapées, travaux de fermeture destinés à assurer la sécurité de l'immeuble, travaux d'amélioration de l'immeuble.
		
		\subsubsection{Action Oblique contre le locataire.}
		
			Le copropriétaire bailleur est responsable du non respect par son locataire des obligations résultant du Règlement de Copropriété. Souvent ce copropriétaire, pris entre deux feux, aura tendance à "protéger" son locataire contre les exigences du syndicat, et ce quand bien même le locataire a un comportement fautif.
			
			Le syndicat des copropriétaires n'a pas de lien de droit avec le locataire.
			
			Certes il peut demander sa condamnation sous astreinte à faire cesser les troubles. Mais le syndicat souhaiterait faire cesser définitivement la gêne ressentie par les copropriétaires, et en quelque sorte, se substituer au copropriétaire bailleur pour faire résilier le bail.
			
			Or, aux termes de l'article 1166 du Code Civil : << les créanciers peuvent exercer tous les droits et actions de leur débiteur, à l'exception de ceux qui sont exclusivement attachés à leur personne. >>
			
			En d'autres termes, le créancier peut agir directement contre le débiteur de son débiteur lorsque son débiteur n'exerce pas lui-même les actions qu'il possède contre ce débiteur\footnote{A est créancier de B qui est lui-même créancier de C. Si B n'agit pas contre C pour obtenir paiement, A intentera directement l'action en paiement contre C !}.
			
			Dans notre domaine, si le locataire ne respecte pas les dispositions du Règlement de Copropriété, il commet une faute contractuelle qui autorise le copropriétaire bailleur à l'assigner en résiliation judiciaire du contrat de bail.
		
			Avant la réforme du Code Civil, La Cour de Cassation avait admis que le syndicat des copropriétaires pouvait invoquer les dispositions de l'article 1166 du Code Civil et assigner directement le locataire en résiliation du bail \footnote{Civ. 3\degres 20 octobre 1981, Bull. III \no 162 p. 117; Civ. 3\degres 14 novembre 1985, JCP 86 IV 39.}.
			
			A l’époque, l’article 1166 du code civil disposait :
			\begin{quote}
				\textbf{Art. 1166.}- Néanmoins, les créanciers peuvent exercer tous les droits et actions de leur débiteur, à l’exception de ceux qui sont exclusivement attachés à la personne
			\end{quote}
			
			La question du maintien de cette jurisprudence après la réforme du code Civil se pose, car les conditions de l’action oblique ont été redéfinies et sont plus strictes
				
			\begin{quote}
				\textbf{Art. 1341-1.}- Lorsque la carence du débiteur dans l’exercice de ses droits et actions à caractère patrimonial compromet les droits de son créancier, celui-ci peut les exercer pour le compte de son débiteur, à l’exception de ceux qui sont exclusivement rattachés à sa personne. 
			\end{quote}»
			
			Or l’action en résiliation du bail pour violation du règlement de copropriété n’a pas réellement un « caractère patrimonial ». Pour autant, cette action a été admise récemment par la cour d’appel de Lyon :
			
			\begin{quote}
				\textbf{Cour d'appel Lyon Chambre civile 1, section B 14 Novembre 2017 \no 15/08882 R.G : 15/08882 Association MUTATION MUTUELLE REPUBLIQUE DEMOCRATIQUE DU CONGO NGO}
				
				\textit{Le syndicat des copropriétaires subissant les nuisances liées à l’exploitation, dans les locaux commerciaux loués, d’une activité d'établissement de nuit de nature à incommoder les copropriétaires, en violation des clauses du règlement de copropriété, est recevable et bien fondé à demander la résiliation du bail et par l’action oblique, outre des dommages et intérêts contre le copropriétaire bailleur.}
			\end{quote}
			
			Cependant l’action oblique ne peut être intentée par le syndicat qu’à condition de démontrer l’inertie du bailleur
			\begin{quote}
				\textbf{Civ.3ème 20 décembre 1994}
				
				\textit{Justifie sa décision, au regard de l'art. 1165 c. civ., la cour d'appel qui, sur l'action d'un syndicat de copropriétaires, ordonne l'expulsion d'un locataire qui exerce dans les lieux loués une activité nuisant à la tranquillité des copropriétaires, dès lors qu'elle relève que la carence du bailleur est une condition de recevabilité de l'action exercée par voie oblique, et que la mise en demeure du bailleur n'a pu mettre fin à cette contravention aux clauses du bail et au règlement de copropriété, le syndicat des copropriétaires agissant dans les seuls droits du copropriétaire-bailleur en poursuivant la résiliation du bail et l'expulsion du locataire.}
			\end{quote}
		
			Les même dispositions de l’article 1166 ont été appliquées par la 6ème chambre de la Cour d’appel de Paris alors que des occupants sans droit ni titre s'étaient installés dans un logement au décès de son propriétaire ; le syndicat des copropriétaires est déclaré recevable à agir, par voie oblique, en expulsion\footnote{Paris 6e ch. C 20 juin 2000 Recueil Dalloz 2001, Somm p. 351 voir aussi CA paris 31 mars 2001, AJDI 2001, p 806 ; CA paris 24 sept 2003, Jd \no223282}.
	
	\subsection{Les relations entre syndicat, locataire et bailleur}
	
		\subsubsection{Responsabilité du Syndicat vis-à-vis du locataire}
		
			Dans le domaine de la responsabilité civile, le syndicat peut être tenu de réparer les dommages causés aux locataires par le vice de construction ou le défaut d'entretien des parties communes (art.14 al.3 de la loi du 10 juillet 1965).
			
			Par ailleurs, les locataires peuvent agir sur le fondement de la responsabilité du fait des choses (art.1384 al 1. du Code civil), ce qui dispense le locataire de faire la preuve du vice de construction ou du défaut d’entretien de l’immeuble; sauf toutefois dans l’hypothèse de l’effondrement (total ou partiel) de l’immeuble, auquel cas l’action du locataire doit avoir pour fondement les dispositions de l’article 1386 du code civil, ce qui implique, comme sur le fondement de l’article 14 de la loi de 1965, de démontrer que cet effondrement est consécutif à un défaut d’entretien ou à un vice de construction.
		
		\subsubsection{Responsabilité du bailleur vis-à-vis du syndicat et des autres copropriétaires}
		
			A l'égard du syndicat, c'est le bailleur qui garde la qualité de copropriétaire avec toutes les prérogatives et charges qui y sont attachées. Participation à l'assemblée, droit de vote, droit de contester les décisions, obligation de payer les charges communes (quitte à se faire rembourser par le locataire des charges récupérables).
			
			Au cas de troubles ou de dommages causés par le preneur, le copropriétaire bailleur sera tenu pour responsable et disposera d'un recours contre le locataire fautif. ( PARIS, 20 janvier 1983, D 83 IR 335; CIV 3\degres 18 déc 1991, Loyers et Copropriété février 1992 \no 84).
			
			Le dommage qui porte atteinte aux parties communes peut aussi bien provoquer l'action du syndicat que celle d'un copropriétaire, dans la limite cependant du préjudice effectivement subi par ce dernier\footnote{Civ. 3\degres 12 mai 1993, Loyers et Copropriété Juillet 1993 \no 280}.
		
		\subsubsection{Recours entre Copropriété, Bailleur, Locataire}

			Un arrêt de la cour d’appel de Paris 23\degres Chambre A du 21 novembre 2000 (Loyers et Copropriété avril 2002 \no 102) permet de bien préciser les responsabilités respectives du syndicat des copropriétaires, du propriétaire bailleur et du locataire.
			L'arrêt apporte les précisions suivantes:
			\begin{itemize}
				\item Le contrat de location concerne exclusivement les rapports entre le bailleur et le preneur: il ne peut avoir aucune incidence dans la détermination des responsabilités encourues en raison de dommages mettant en cause la copropriété (Cass. 3 e ci v., 4 janv. 1991 : JCP N 199 1, /1, p. 269. - 16 juin 1993: Rev. Loyers 1994, p. 192. - 20 nov 1996 : Loyers et copr 1997, comm. no 127).
				\item Par voie de conséquence, le copropriétaire-bailleur demeure seul responsable vis-à-vis du syndicat ou de l'un de ses membres des troubles imputables au comportement de son locataire, qui doit lui-même se soumettre aux obligations inscrites dans le règlement de copropriété.
				\item Lorsque des dommages supportés par le copropriétaire ou son locataire sont dus à un vice de construction ou au défaut d'entretien des parties communes (L. art. 14), leur réparation incombe directement au syndicat (Cass. 3' civ, 3 juil. 1991 : Loyers et copr. 1991, comm. no 353. - le, avr. 1999 : Loyers et copr. 1999, comm. no 168).
				\item Si le locataire d'un lot est victime de troubles du voisinage provoqués par un autre copropriétaire ou locataire, la responsabilité de leur auteur doit être recherchée sur le fondement soit de la responsabilité contractuelle inhérente au bail (contre le bailleur), soit de la responsabilité quasi-délictuelle instituée aux articles 1382 et 1383 du Code civil (contre le voisin responsable).
				Cela étant, le copropriétaire-bailleur jugé responsable à l'égard du syndicat peut exercer une action récursoire à l'encontre de son locataire en application des stipulations du bail.
			\end{itemize}
	
\section{Interventions du locataire dans la vie de la copropriété}
	
	\subsection{La location accession}
	
		Rappelons qu'il s'agit d'une catégorie particulière de location avec promesse de vente régie par les dispositions de la loi du 12 juillet 1984.
		Aux termes de cette loi, le locataire-accédant est assimilé au copropriétaire dès lors que l'immeuble est déjà soumis au statut de la loi de 1965. En sorte que le locataire-accédant aux termes de l'article 32 de la loi << est subrogé dans les droits et obligations du vendeur >>.
	
		En conséquence :
		\begin{enumerate}
			\item Il participe aux assemblées générales et vote sur toutes questions autres que celles réservées au bailleur par la loi et qui concernent :
			\begin{itemize}
				\item les travaux qui sont à la charge du bailleur (sur éléments porteurs et concourant à la stabilité ou à la solidité des bâtiments, aux éléments d’équipement intégrés à eux, et aux éléments qui assurent le clos, le couvert et l’étanchéité, sauf parties mobiles) ;
				\item les actes de disposition et les travaux d’amélioration.
			\end{itemize}
			\item Il participe aux charges afférentes à l'entretien et aux réparations de l'immeuble (Toutefois le vendeur est garant des charges dues par le locataire-accédant).
			\item Il peut être membre du conseil syndical de l’immeuble.
		\end{enumerate}
	
	\subsection{Le bail réel et solidaire}
	
		Le bail réel est solidaire est un démembrement du droit de propriété, plus proche de la division entre nue-propriété et usufruit que d'un bail véritable. Il a pour objet la dissociation du foncier et de la construction. Le preneur est titulaire d'un véritable droit réel, pouvant faire l'objet d'une mutation.
		
		La loi \no 2018-1021 du 23 novembre 2018 a complété ce dispositif et précisé qu’il pouvait porter sur un lot de copropriété.
		
		C'est pourquoi l’article L 255-7-1 assimile le preneur d’un tel bail à un copropriétaire :
		\begin{quote}
			\textbf{Art. L. 255-7-1.}-Pour l'application de la loi \no 65-557 du 10 juillet 1965 fixant statut de la copropriété des immeubles bâtis, la signature d'un bail réel solidaire est assimilée à une mutation et le preneur est subrogé dans les droits et obligations du bailleur, sous réserve des dispositions suivantes :
			
			1\degres Le preneur dispose du droit de vote pour toutes les décisions de l'assemblée générale des copropriétaires, à l'exception de décisions prises en application des d et n de l'article 25 et des a et b de l'article 26 de la même loi ou de décisions concernant la modification du règlement de copropriété, dans la mesure où il concerne les spécificités du bail réel solidaire. Le bailleur exerce également les actions qui ont pour objet de contester les décisions pour lesquelles il dispose du droit de vote. Aucune charge ne peut être appelée auprès du bailleur y compris pour des frais afférents aux décisions prises par lui ou pour son compte
			
			2\degres Chacune des deux parties peut assister à l'assemblée générale des copropriétaires et y formuler toutes observations sur les questions pour lesquelles elle ne dispose pas du droit de vote.
		\end{quote}
		On notera que cette disposition a pour effet de faire payer par le preneur les travaux d’amélioration (art 25 n), alors qu’il n’aura pas le droit de vote sur ceux-ci.