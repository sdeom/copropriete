\chapter{Les appels de fonds auprès des copropriétaires}

SECTION I - L’IDENTIFICATION DU DEBITEUR
A. L’IDENTITE DU DEBITEUR DES CHARGES.
Il est évident que la personne tenue de payer les charges est le copropriétaire. Pourtant certaines difficultés doivent être énumérées tenant à la nature juridique du bien ou à la situation des personnes :
1. Le débiteur personne morale
COEXISTENCE D'UNE SCI D'ATTRIBUTION ET DE COPROPRIETAIRES
Si l'immeuble comprend une S.C.I. d'attribution et des copropriétaires retrayants, quel qu'en soit le nombre, la SCI sera seule tenue des charges de copropriété et non les associés à titre individuel.500
C’est ce que rappelait la Cour de Cassation Civ 3ème 9 juin 1993 D 1994 Somm p. 125
“Viole les art. 10 et 43 de la loi \no 65-557 du 10 juill. 1965 la cour d'appel qui, pour écarter la fin de non-recevoir opposée à la demande en paiement des charges de copropriété, formée par le syndicat contre les porteurs de parts d'une société civile immobilière d'attribution, attributaires de lots en jouissance, retient que l'art. 9 des statuts de la société stipule que le droit de jouissance immédiate s'exercera sous les charges et conditions définies par le règlement de copropriété, alors que le syndicat n'a pas qualité pour agir contre l'associé attributaire en jouissance».
cour d'appel de Paris 23e chambre du 24 septembre 1997 :
Le syndicat des copropriétaires n'a pas qualité pour agir contre l'associée attributaire en jouissance, la dette de la charge étant, selon l'article 10 de la loi du 10 juillet 1965, dont les dispositions sont d'ordre public, attachée à la qualité de copropriétaire.
500 PARIS 18 décembre 1979, D 1981 I.R. p. 98 \no 35; Civ. 3\degres 6 juillet 1988, JCP G 88,IV,328
droit de la copropriété année 2019-2020
405
En pratique on constate très souvent que le syndic de la Copropriété appelle directement les charges auprès des associés. C'est une pratique qui n'est pas conforme aux dispositions de la loi... même si elle rend service au gérant de la SCI. L'associé peut en conséquence refuser de payer les charges de son lot entre les mains du syndic pour le compte du Syndicat501
SOCIETE PROPRIETAIRE D’UN LOT.
Si le propriétaire du lot est une société, le syndicat ne pourra poursuivre le recouvrement qu’à l’encontre de la Société et non à l’encontre des associés, du moins dans un premier temps.
La société sera assignée à son siège social (toujours lever un K bis) prise en la personne de son représentant légal (il n’est pas utile de nommer ce représentant légal autrement que par son titre : Président, Gérant, etc.).
S’il s’agit d’une société commerciale, le syndicat ne pourra exécuter qu’à l’encontre de la Société, sans recours contre les associés à titre personnel, quelle que soit l’importance de leur participation dans le capital social.
S’il s’agit par contre d’une société civile, le syndicat aura la faculté après avoir vainement poursuivi la société, de solliciter condamnation des associés pris individuellement, à concurrence de leur participation dans le capital social502
2. Le débiteur personne publique
L’ADMINISTRATION PROPRIETAIRE DU LOT.
Le Conseil d’Etat a jugé qu’il y avait incompatibilité entre le Domaine Public et la Copropriété, aussi est-il est aujourd’hui unanimement admis que l’administration est un propriétaire comme les autres503, soumis à ce titre à l’ensemble de la loi de 1965 et en conséquence, susceptible d’être poursuivie devant la juridiction civile pour toute cause tirée de l’application du statut de la copropriété et en conséquence pour le paiement des charges.
501 Civ. 3\degres 9 juin 1993, RJDA 10/93 \no 830
502 Article 1858 code civil; Civ. 3\degres 26 mai 1994, Revue Dr. Imm. 1994 p. 493.
503 Civ 3\degres 11 mai 1994, Recueil Dalloz 1996, Somm. p. 159
droit de la copropriété année 2019-2020
406
L’ETAT ETRANGER PROPRIETAIRE D’UN LOT.
L’Etat étranger peut-il invoquer le bénéfice de l’immunité de juridiction ? La réponse semble négative pour deux raisons qui ont fait l’objet de deux décisions distinctes :
- d’une part la Convention de Vienne du 18 avril 1861 ne donne cette immunité qu’aux agents diplomatiques eux-mêmes et non aux Etats. Ce qui a été rappelé par la Cour d’Appel Administrative de Paris504 dans un arrêt du 16 juillet 1992,
- d’autre part, les immunités diplomatiques et juridictionnelles ont pour objet de permettre à leurs bénéficiaires l'exercice de leur mission et sont limitées aux nécessités de cet exercice, ce qui ne peut concerner le paiement des charges. C’est ce qui a été affirmé par un arrêt de la Cour de Paris505 en date du 25 janvier 2001.
Toutefois les autorités françaises invitent les syndicats de copropriété à aviser le service du protocole du ministère des affaires étrangères avant d’intenter une action en recouvrement contre un état étranger, pour que ces informations puissent être prises en compte dans le cadre des contacts bilatéraux entre États506...
Bien que nous ne disposions pas de jurisprudence sur ce point, il apparaît que par application de la Convention de Vienne, un agent diplomatique étranger propriétaire d’un lot de copropriété ne pourrait être poursuivi directement en paiement devant la juridiction française.
C’est ce qui a été jugé par exemple à propos d’une action sur le fondement du droit des baux à l’encontre d’un agent de l’UNESCO507 :
« L’action en validité de congé en expulsion et en paiement d'une indemnité d'occupation exercée par le bailleur contre le preneur bénéficie de l'immunité de juridiction, cette action n'entrant dans aucune des exceptions limitativement prévues ».
Le même arrêt rappelant par ailleurs que :
L’art. 688 nouv. C. pr. civ. dispose que l'acte notifié à un agent diplomatique étranger en France ou à tout autre bénéficiaire de l'immunité de juridiction est notifié au Parquet et transmis par l'intermédiaire du ministère de la Justice, à moins qu'en vertu d'un traité la transmission puisse être faite par une autre voie. La notification de congé et l'assignation délivrées directement aux intéressés en leur domicile n'ont pas respecté les prescriptions des art. 688 et 691 nouv. C. pr. civ.
504 CAA PARIS 16 juillet 1992 (Sat Bld Flandrin c/ Etat du Quatar) Loyers et Copr. Jan 93 \no 30.
505 Paris (23e Ch. B), 25 janvier 2001 DALLOZ., 2001, IR p. 1667
506 JOAN 20 mars 2000, p. 1774).
507 C. app., Paris (1re Ch. urg.), 30 juin 1981 30/06/81 EPOUX RIBEYRO C. DAME MASSARI
Rev. crit. dr. intern., 1982, p. 129, note P. Bourel.
droit de la copropriété année 2019-2020
407
et ont causé grief aux intéressés en portant atteinte à leur prérogatives diplomatiques. La nullité de ces actes doit être prononcée conformément à l'art. 114 du même Code
Les mêmes règles devraient recevoir application pour le paiement des charges de copropriété.
B. L’ADRESSE DU DEBITEUR DES CHARGES.
Bien souvent le copropriétaire débiteur de charges n’habite pas à l’immeuble en copropriété.
L’article 64, alinéa 1, du Décret 508est ainsi rédigé :
« Toutes les notifications et mises en demeure prévues par la loi du 10 juillet 1965 et par le présent décret, à l’exception de la mise en demeure visée à l’article 19 de ladite loi, sont valablement faites par lettre recommandée avec demande d’avis de réception ».
Aux termes de l’article 65, alinéa 1, du Décret509 :
« En vue de l’application de l’article précédent, chaque copropriétaire ou titulaire d’un droit d’usufruit ou de nue-propriété sur un lot ou une fraction de lot doit notifier au syndic son domicile réel ou élu soit en France métropolitaine si l’immeuble y est situé, soit dans le département ou le territoire d’outre mer de la situation de l’immeuble ».
Dans un arrêt de 1994510 la Cour de Paris a déduit de ces deux textes que le syndic a l’obligation de notifier tout acte au dernier domicile réel ou élu qui lui a été notifié par le copropriétaire débiteur.
En sorte que si ce copropriétaire n’a pas notifié régulièrement de changement de domicile réel ou élu au syndic, il engage sa responsabilité envers le syndicat des copropriétaires, et il pourra être condamné, outre le paiement de l’arriéré des charges avec intérêts de droit, à des dommages-intérêts complémentaires.
Dans le cas d’espèce, le copropriétaire «professionnel du droit» (avocat) avait déposé sa carte de visite avec nouvelle adresse chez le syndic. Ce n’était pas une notification régulière. Faute de notification régulière, il a causé un préjudice moral au syndicat « en ne collaborant pas loyalement au fonctionnement de celui-ci », en sorte qu’il a été condamné à 7.000 fr. de dommages-intérêts complémentaires.
508 Modification par le décret du 27 mai 2004
509 D\degres note 145
510 PARIS 8\degres Ch. 6 déc 1994; Loyers et copropriété mai 1995 \no 236.
droit de la copropriété année 2019-2020
408
C. CONCOURS DE DROITS SUR LE LOT DE COPROPRIETE ET OBLIGATION A LA DETTE
Aux termes de l'article 1202 du Code Civil :
" La solidarité entre débiteurs ne se présume point; il faut qu'elle soit expressément stipulée".
Ce qui signifie que les charges de copropriété sont dettes personnelles aux copropriétaires qui en sont tenus, en sorte que le syndic ne saurait demander aux autres copropriétaires de payer la quote-part du défaillant.
En fait comme l'ont exposé les auteurs depuis l'origine, s'il n'existe pas de solidarité légale (pas de solidarité sans texte), il existe malgré tout une solidarité de fait, en sorte que pour pallier à la défaillance d'un ou plusieurs copropriétaires, le syndic fera voter en Assemblée Générale un appel de fonds exceptionnel auquel tous les copropriétaires seront tenus au prorata de leurs tantièmes.
Mais la solidarité peut être expressément prévue par le Règlement de Copropriété : entre usufruitier et nu-propriétaire, entre indivisaires ou encore entre copropriétaires propriétaires de deux lots issus d'un lot d'origine.
Ces clauses de solidarité ont cependant connu un sort divers selon les personnes tenues :
1. Usufruitier et nu-propriétaire.
Selon le principe de l'article 1202 Code Civil, il n'y a pas solidarité entre usufruitier et nu-propriétaire.
En principe, l'usufruitier paie les charges de l'article 605 du Code civil (entretien) et le nu-propriétaire paie les charges de l'article 606 (grosses réparations).
" Les grosses réparations sont celles des gros murs et des voûtes, le rétablissement des poutres et des couvertures entières. Celui des digues et des murs de soutènement et de clôture aussi en entier. Toutes les autres réparations sont d'entretien".
Ceci signifie notamment que les travaux de ravalement, aussi coûteux soient-ils, sont dus par l'usufruitier.
En fait, la plupart des règlements de copropriété prévoient une clause de solidarité entre usufruitier et nu-propriétaire.
droit de la copropriété année 2019-2020
409
Cette clause a été validée par la Cour de Cassation 511
2. Indivision.
En cas de décès chacun des héritiers ne recueille qu'une part de la succession et n'est tenu qu'à concurrence de cette part (sous réserve toutefois que l'indivision ait été régulièrement dénoncée au syndic en application de l'article 6 du Décret de 1967).
La quasi-totalité des Règlements de Copropriété prévoit une solidarité entre les indivisaires.
La jurisprudence s’est trouvée partagée :
- la cour d'appel de Paris a condamné systématiquement cette clause de solidarité instituée entre co-indivisaires d'un lot :512; ces différentes décisions ont considéré en effet que la solidarité était contraire au principe selon lequel chacun ne peut être tenu d'acquitter une quote-part de dépenses qu'en proportion de ses droits dans l'indivision. Cependant depuis 2002 on assiste à un net revirement de la cour de Paris en faveur de la validité de la clause (Paris 21 janvier 2002513 ou encore Paris 29 sep 2005514)
- les cours de Lyon, Versailles, Pau se sont prononcées en faveur de la validité de la clause (cour d'appel de Lyon 15 mars 2000 Juris Data nº 121 580 ou la cour de Versailles du 11 juin 2003)
- La cour de cassation a adopté une position favorable à la validité de la clause de solidarité dans un arrêt du 1er décembre 2004 en rejetant le pourvoi contre l’arrêt précité de la cour de Versailles515 :
« Attendu que si la solidarité ne s’attache pas de plein droit ni à la qualité d’indivisaire, ni à la circonstance que l’un d’eux ait agi comme mandataire des autres, la clause de solidarité stipulée dans un Règlement de copropriété n’est pas prohibée entre indivisaires conventionnels tenus de désigner un mandataire commun »
511 Civ. 3\degres Ch. 30 novembre 2004 (Administrer nov. 2005 p. 12)
512 Paris 23e chambre 23 septembre 1994. loyers copropriété 1995 nº 84 ; Paris 26 mars 1997 loyers copropriété 1997 nº 211 ; Paris 24 février 1999 : loyers copropriété 1999 nº 217
513 Paris 23\degres Ch 21 jan 2002, Loyers et Copropriété 2002, comm 186
514 Paris 23\degres Ch 29 sep 2005, Loyers et Copropriété 2006, comm 19
515 Civ. 3\degres 1er dec 2004 – Loyers et Copropriété 2005, commentaire \no 36.
droit de la copropriété année 2019-2020
410
Pour autant on regrettera que la rédaction de cet arrêt ne constitue pas une adhésion sans restriction au principe de la validité de la clause de solidarité : pourquoi faire référence au mandataire commun ?
L'indivision peut être organisée auquel cas, les indivisaires sont représentés par le gérant auquel le syndic pourra s'adresser pour le paiement des charges dus par l'Indivision.
D. VENTE CONTRE RENTE VIAGERE.
C'est un contrat aléatoire mais c'est surtout un contrat de vente.
Ce contrat peut avoir deux formes distinctes :
- rente viagère sans réserve d'usufruit ou de droit d'usage et d'habitation. En ce cas c'est l'acquéreur (débirentier) qui doit les charges au syndicat des copropriétaires.
- rente viagère avec réserve d'usufruit ou de droit d'usage et d'habitation au profit du vendeur (crédirentier), auquel cas nous retrouvons la distinction entre grosses réparations à la charge du débirentier et autres dépenses qui demeurent à la charge du crédirentier.
E. LES EPOUX.
Lorsque le lot de copropriété est bien commun aux époux, il y a obligation solidaire des époux envers la copropriété pour le paiement des charges. En cas de divorce, cette solidarité ne cessera qu'à compter du jugement d'homologation du partage de la Communauté... quelles que soient par ailleurs les mesures prises par le juge pendant la procédure de divorce516.
C'est ce qui a été jugé également par la Cour de PARIS le 18 octobre 1993 :
« Lorsque deux époux sont titulaires de lots dans un immeuble en copropriété, chacun d'eux est tenu de la totalité des charges jusqu'à la liquidation et au partage de leurs intérêts patrimoniaux. Il s'ensuit qu'après règlement de la moitié des charges afférentes à ces lots, le mari reste débiteur de l'autre moitié envers le syndicat des copropriétaires » (Application de l'art. 1400 du C. Civ.)
Nous relevons ici que le jugement d’homologation du partage vaut transfert de propriété au sens de l’article 6 du Décret, en sorte que l’avocat qui obtient ce jugement doit le notifier au syndic (malheureusement, il s’agit d’un voeu pieu).
516 PARIS 23\degres 20 février 1991, Loyers et Copropriété 1991 \no 174
droit de la copropriété année 2019-2020
411
Lorsque le lot est propriété d'un seul époux, c'est en principe cet époux qui est copropriétaire et en conséquence débiteur des charges. Cependant l'autre époux peut être tenu si les charges ont été contractées pour l'entretien du ménage ou l'éducation des enfants, conformément à la solidarité édictée par l'article 220 du Code Civil517.
F. SIGNATAIRES D’UN PACTE DE SOLIDARITE
Les partenaires d’un Pacte de Solidarité (PACS) sont bien évidemment solidaires pour les dépenses de copropriété… dépenses de la vie courante (article 515-4 du code civil).
G. DROIT D’USAGE ET D’HABITATION.
L'application des règles posées par les articles 605 et 606 du Code civil, devrait conduire le syndicat à ventiler les charges entre propriétaires et titulaires d'un droit d'usage et d'habitation.
On notera cependant un arrêt de la cour de cassation du 23 février 2001 qui a rejeté le pourvoi du mari propriétaire d'un lot alors que le juge lors du divorce des époux avait accordé un droit d'usage et d'habitation à l'ex-femme au titre de la prestation compensatoire. Les charges étant demeurées impayées, le syndicat avait demandé la condamnation in solidum des anciens époux. La cour avait fait droit à cette demande et le mari avait régularisé un pourvoi contre cette décision. La cour de cassation rejette au motif que : « ayant relevé par motifs propres et adoptés qu'aucun texte légal ou réglementaire n'exonère un copropriétaire, sous prétexte qu'aurait été constitué un droit d'usage et d'habitation, de l'obligation de paiement des charges instituée par la loi du 10 juillet 1965, la cour d'appel, qui a retenu que M. B. avait conservé son droit de propriété et qui, en présence d'une demande de condamnation in solidum, n'était pas tenu de régler les rapports entre les titulaires des droits démembrés, a légalement justifié sa décision de ce chef. » (3e chambre civile 23 février 2001 ; Loyers et Copropriété juin 2000 nº 150)
H. LOCATION ACCESSION ET PAIEMENT DES CHARGES.
La location-accession s’analyse en droit comme un contrat de location qui n’est pas soumis aux règles concernant le bail mais à un régime propre.518
Dès lors que les trois conditions suivantes sont réunies, il y a contrat de location accession :
517 PARIS 8\degres Chambre 1er fév 80, Administrer Cop. 399
518 MALAURIE et AYNES, les Contrats spéciaux p. 433 \no 809
droit de la copropriété année 2019-2020
412
- Promesse de vente consentie à l’accédant,
-Période préalable de jouissance à titre onéreux,
-Paiement d’une redevance jusqu’à levée de l’option, comportant un loyer et la contrepartie « du droit personnel au transfert de la propriété du bien ».519
Dès lors que le contrat de location-accession porte sur un immeuble destiné à l’habitation ou à usage mixte d’habitation et professionnel, il est soumis aux dispositions de la loi du 12 juillet 1984.520
Aux termes de cette loi, le locataire accédant est tenu des charges d'administration, d'entretien et de réparations courantes, tandis que le propriétaire vendeur doit supporter les dépenses relatives aux grosses réparations et à l'amélioration de l'immeuble (article 29. al 2). Mais le vendeur est garant des charges qui incombent au locataire-acquéreur.
I. CREDIT-BAIL
Le crédit-bail peut porter sur un meuble ou sur un immeuble.
Lorsqu’il porte sur un immeuble, le schéma est le suivant : un industriel ou un commerçant souhaite utiliser cet immeuble pour les besoins de son activité sans avoir à l’acheter immédiatement. Il s’adresse à une société spécialisée, une SICOMI (Société Immobilière pour le Commerce et l’Industrie), qui achète l’immeuble. La société de crédit-bail va alors donner cet immeuble en jouissance au « crédit-preneur » qui lui paiera les loyers correspondant pendant cette période de location, dite irrévocable. A l’issue de cette période, le crédit-preneur aura le choix entre la restitution de l’immeuble, le renouvellement de la jouissance locative ou... (option le plus souvent adoptée), l’acquisition de l’immeuble pour une valeur dite résiduelle.
On le voit, le crédit-preneur est un simple locataire titulaire d’une promesse de vente unilatérale (il n’a pas l’obligation d’acheter). Même si en réalité l’opération de crédit-bail est essentiellement conçue comme un emprunt en vue d’acquérir avec garantie du vendeur sur l’immeuble... et que le prix payé par le crédit-preneur s’apparente davantage à un paiement en capital et intérêts qu’à un véritable loyer.
En fait, le crédit-bail est très proche de la location-vente, mais il fait intervenir trois personnes : le vendeur, l’acquéreur et... le banquier.
519 ibid.
520 Toujours selon les mêmes auteurs, ce type de contrat n’est qu’un pis-aller destiné à écouler les queues de programmes de promotion.
droit de la copropriété année 2019-2020
413
Dans le cadre d’un immeuble en copropriété, il est relativement fréquent que les locaux commerciaux ou industriels soient acquis par l’exploitant selon la technique du crédit-bail.
Vis à vis du syndicat des copropriétaires, le crédit-preneur n’est qu’un simple locataire qui ne bénéficie aucunement de la loi de 1984 sur la location-accession.
Certes, les contrats de crédit-bail comprennent le plus souvent un mandat général au profit du crédit-preneur, qui de la sorte est convoqué et participe aux assemblées générales de la Copropriété; mais il n’en demeure pas moins que le copropriétaire demeure le crédit-bailleur, en sorte que si les charges ne sont pas réglées par le crédit-preneur, le syndic doit poursuivre le paiement de ces charges exclusivement à l’encontre du crédit-bailleur.521
SECTION II - LA DETERMINATION DU DEBITEUR DANS LE TEMPS
A. LE COPROPRIETAIRE DEBITEUR EN CAS DE CESSION A TITRE ONEREUX.
Les règles ont été dégagées par les tribunaux avant d’être inscrites dans la loi de 1965.
1. Les principes dégagés par la jurisprudence.
LE COPROPRIETAIRE EST DEBITEUR JUSQU’A LA NOTIFICATION DE LA VENTE DE SON LOT.
En principe le copropriétaire vendeur reste débiteur de toutes les charges de copropriété jusqu'à la vente et l'acquéreur est débiteur des charges postérieurement à son acquisition.
Mais la vente est un acte entre vendeur et acquéreur sans participation du syndic.
Il en résulte tout d’abord que vendeur et acquéreur ne peuvent opposer au syndicat leurs conventions : par exemple est inopposable la convention par laquelle il est stipulé que le vendeur restera débiteur des charges du ravalement qui sera voté après la vente522. (Solution défavorable aux intérêts du syndicat
521 Cette délégation générale au profit du crédit-preneur ne règle pas tous les problèmes pratiques : certes, elle permet au crédit-preneur d’exercer directement l’action en vice caché à l’encontre du vendeur de l’immeuble; mais la garantie décennale appartient exclusivement, de par la loi, au maître d’ouvrage.
522 PARIS 23\degres Chambre 16 octobre 1996, Loyers et Copropriété 1997 \no 25.
droit de la copropriété année 2019-2020
414
puisque si elle était admise le syndic devrait rechercher le vendeur et ne bénéficierait pour ces sommes ni du privilège légal ni même de l’inscription d’hypothèque de l’article 19).
L'article 6 du décret de 67 prévoit que tout transfert de propriété doit être notifié sans délai au syndic par le notaire ou par l'avocat qui a obtenu la décision judiciaire en cas de vente à la barre du tribunal.
Aussi la jurisprudence en a déduit que l'ancien copropriétaire est tenu au paiement des charges jusqu'à ce que le syndic ait été avisé du transfert523
LA DATE D’EXIGIBILITE
Pour savoir qui est débiteur des charges entre vendeur et acquéreur, il ne suffit pas de connaître la date de mutation dans les relations entre vendeur et acquéreur d’une part et syndicat des copropriétaires d’autre part, mais il convient également de savoir si la créance du syndicat des copropriétaires est où non exigible à la date de mutation.
La jurisprudence a sensiblement évolué quant à la définition des sommes exigibles :
A l’origine, le copropriétaire vendeur était tenu des charges votées, même lorsque, s’agissant de travaux, ceux-ci n’avaient pas été réalisés avant la mutation Civ. 3\degres, 19 juillet 1983 Civ 3\degres 8 janvier 1992524..
Cf. la Douzième Recommandation de la Commission sur la Copropriété:
"Constituent des créances liquides et exigibles les appels de fonds votés par l'assemblée générale avant la cession, fût-ce pour le financement de travaux non encore exécutés".
Cette position a été abandonnée, en sorte que pour que la somme votée fut exigible, il ne suffisait pas que la dépense ait été adoptée par l'assemblée générale, encore fallait-il que la même assemblée générale ait décidé des appels de fonds :
CA paris 1ère chambre 16 juin 1995525 :
« Si l’assemblée générale a effectivement voté des travaux de changement de vannes, en revanche ces travaux n’ont donné lieu à aucun calendrier fixant les appels de fonds qui n’ont pas été lancés, de sorte que la somme réclamée ne correspondant pas à un appel de fonds décidé par délibération de l’assemblée générale antérieure à la cession et devenue définitive, le syndicat ne peut se prévaloir
523 Civ 3\degres 3 oct 1972
524 Loyers et Copropriété mars 1992 \no 134
525 PARIS 23\degres Ch. 16 juin 1995, Loyers et Copropriété novembre 1995 \no 486.
droit de la copropriété année 2019-2020
415
à l’encontre des vendeurs d’une créance liquide et exigible à la date de mutation du lot au titre des travaux ».
C’est l’appel de fonds qui rendait la créance du syndicat exigible.
L’EXIGIBILITE IMPOSE NON SEULEMENT QUE LA DEPENSE AIT ETE VOTEE EN ASSEMBLEE AVEC UN CALENDRIER PRECIS, MAIS ENCORE QU’UN APPEL DE FONDS AIT ETE FAIT AVANT LA MUTATION..
Cour d’appel de Paris 21 nov. 1996526 :
«Considérant qu’est débiteur des charges au regard du syndicat des copropriétaires celui qui est propriétaire au moment où l’appel de fonds est lancé».
23e chambre section B de La Cour d’appel de Paris 1er juillet 1999(Loyers et Copropriété janvier 2000 nº 17). :
« considérant que la personne redevable du coût des travaux entrepris dans un immeuble régi par le statut de la copropriété est, en application des articles 20 de la loi du 10 juillet 1965 et 5 - 1 du décret du 17 mars 1967, celle qui est copropriétaire à la date à laquelle la créance du syndicat est devenue liquide et exigible ; que doivent être considéré comme liquides et exigibles les sommes qui ont fait l'objet d'appel de fonds soit sur l'initiative du syndic soit sur décision de l'assemblée générale des copropriétaires ; qu'il convient donc de recherche à quelle date les appels de fonds relatifs aux travaux considérés sont devenus effectivement liquides et exigibles ».
2. Les dispositions du décret du 27 mai 2004.
Trois articles du décret du 27 mai 2004 apportent d’importantes précisions quant aux sommes qui sont respectivement dues par vendeur et acquéreur et quant à leur exigibilité : l’article 6-2 du Décret de 67, l’article 35-2 et l’article 45-1 du même décret.
ARTICLE 6 – 2 DU DECRET
A l'occasion de la mutation à titre onéreux d'un lot :
1\degres Le paiement de la provision exigible du budget prévisionnel, en application du troisième alinéa de l'article 14-1 de la loi du 10 juillet 1965, incombe au vendeur ;
526 PARIS 8\degres Ch. 21 novembre 1996, Loyers et Copropriété 1997 \no 83
droit de la copropriété année 2019-2020
416
2\degres Le paiement des provisions des dépenses non comprises dans le budget prévisionnel incombe à celui, vendeur ou acquéreur, qui est copropriétaire au moment de l'exigibilité ;
3\degres Le trop ou moins perçu sur provisions, révélé par l'approbation des comptes, est porté au crédit ou au débit du compte de celui qui est copropriétaire lors de l'approbation des comptes.
Cet article ne donne pas de réponse réglementaire à la question tranchée par la jurisprudence de la date de la mutation à l’égard du syndicat des copropriétaires et il convient de considérer que c’est bien la date de signification de la mutation et non celle de la signature de l’acte de vente entre vendeur et acquéreur qui constitue la date de mutation au regard du syndicat des copropriétaires.
Par contre, cet article donne une définition précise des charges liquides et exigibles lors de la cession.
En premier lieu, ce texte ne s’applique qu’en cas de mutation à titre onéreux, donc il ne s’applique pas réglementairement en cas de donation … pour autant ce texte ne faisant que reprendre la jurisprudence, on doit admettre que le syndic appliquera les mêmes règles entre donateur et donataire.
Ensuite, il convient d’observer que le texte de l’article 6-2 ne précise :
- ni à quel moment une personne est propriétaire,
- ni ce qu’est la date d’exigibilité.
QUI EST PROPRIETAIRE ?
Nous en resterons à la règle jurisprudentielle prise en application de l’article 6 du décret et que nous venons d’évoquer : vis à vis du syndicat des copropriétaires est propriétaire celui qui figure comme tel dans les documents du syndic. En d’autres termes, c’est la notification prévue à l’article 6 du décret qui vaut changement de propriétaire.
Tant que le syndic n’a pas reçu cette notification peu importe que la vente ait été faite, le syndic doit considérer que le vendeur est encore propriétaire !
On admettra cependant, même si le dernier alinéa de l’article 6 dit le contraire, que la notification de la vente faite par le notaire en application de l’article 20 de la loi vaut également comme notification de l’article 6 du Décret.
En sorte que pour l’application de l’article 6-2 du décret, le syndic prend une date unique : celle à laquelle il reçoit la lettre recommandée du notaire (et non pas la date de l’acte notarié de vente).
QUAND LA CREANCE EST ELLE EXIGIBLE ?
droit de la copropriété année 2019-2020
417
S’agissant des provisions exigibles du budget, rappelons que l’article 14-1 de la loi précise que :
« La provision est exigible le premier jour de chaque trimestre ou le premier jour de la période fixée par l'assemblée générale ».
S’agissant des appels « hors budget », l’article 14-2 de la loi précise que :
« Les sommes afférentes à ces dépenses sont exigibles selon les modalités votées par l'assemblée générale »
Bien évidemment cette dernière disposition est capitale : si l'assemblée générale qui vote les travaux et le montant de la dépense ne vote pas les conditions d’exigibilité … le syndic ne pourra pas exiger le paiement de la part du copropriétaire (vendeur ou pas).
L’ARTICLE 35-2 DU DECRET.
Pour l'exécution du budget prévisionnel, le syndic adresse à chaque copropriétaire, par lettre simple, préalablement à la date d’exigibilité déterminée par la loi, un avis indiquant le montant de la provision exigible.
Pour les dépenses non comprises dans le budget prévisionnel, le syndic adresse à chaque copropriétaire, par lettre simple, préalablement à la date d'exigibilité déterminée par la décision d’assemblée générale, un avis indiquant le montant de la somme exigible et l'objet de la dépense.
En sorte que la date d’exigibilité de la créance ne résulte plus de l’appel de fonds, mais :
- Du 1er jour du trimestre pour le quart du budget (sauf exception)
- Du jour fixé par l’assemblée générale pour les travaux.
Au demeurant, le texte de l’article 35-2 précise bien que la lettre du syndic au copropriétaire avise le copropriétaire que la somme est devenue exigible ; ceci quand bien même la pratique a conservé les termes d’appels de fonds, plus facilement compréhensibles pour les copropriétaires !
- Le sort du compte prorata temporis.
droit de la copropriété année 2019-2020
418
Il ne peut y avoir de compte prorata temporis en copropriété : le syndic ne peut pas répartir la dette entre vendeur et acquéreur en tenant compte des périodes d’engagement de la dépense ou au-prorata du nombre de jours pendant lesquels le vendeur a été propriétaire au cours de l’exercice.
Le dernier alinéa de l’article 6-2 le dit désormais expressément en mettant le trop ou le moins perçu à la charge de l’un ou de l’autre !
Cela dit, rien n’empêche le syndic d’accepter comme prestataire de services des vendeur et acquéreur de leur faire un compte de ce type … mais ce faisant il ne tiendra pas compte du résultat de ce compte dans la comptabilité du syndicat des copropriétaires, il ne fera que collaborer aux conventions des parties.
LE SORT DES CONVENTIONS PARTICULIERES (ARTICLE 6-3)
L’article 6-3 du décret est ainsi rédigé :
« Toute convention contraire aux dispositions de l'article 6-2 n'a d'effet qu'entre les parties à la mutation à titre onéreux. ».
Ajoutons simplement que non seulement ces conventions particulières ne sont pas opposables au syndicat des copropriétaires mais que le syndic commettrait une faute en les appliquant si cette application l’amenait à répartir les dettes de façon différente de ce qui est dit à l’article 6-2 du Décret !
C’était l’opinion de CAPOULADE (reprenant ici une jurisprudence de la cour de cassation) lorsqu’il écrivait : « L’article 6-2 est impératif. L’article 6-3 n’y déroge pas et n’autorise pas la dérogation ».
Par contre, rien n’interdit aux parties (vendeur et acquéreur) d’avoir des conditions particulières qui obligent contractuellement les parties à l'acte.
droit de la copropriété année 2019-2020
419
B. LE DEBITEUR DES CHARGES EN CAS DE DISPARATION DU TITRE DE PROPRIETE OU .. DU COPROPRIETAIRE
1. La résolution de la vente.
Lorsque la vente d'un lot en copropriété entraîne une procédure en résolution de la vente, ni le vendeur ni l'acquéreur ne se considèrent tenus de payer les charges :
le vendeur parce qu'il estime qu'il n'est plus copropriétaire,
l'acquéreur parce qu'il est convaincu que la vente sera annulée et en conséquence qu'il sera considéré comme n'ayant jamais eu la qualité de copropriétaire.
En fait la réponse est double :
Tant que la vente n'est pas résolue, elle est opposable à l'acquéreur qui sera seul tenu du paiement des charges vis à vis du syndicat des Copropriétaires.
Si la vente est effectivement annulée, le syndic devra recouvrer les charges impayées du lot sur le propriétaire d'origine527, du moins à compter du jour où la résolution de la vente lui aura été notifiée. Tant que cette notification ne lui aura pas été faite, l'acquéreur "résilié" demeurera tenu de payer les charges de Copropriété afférente au lot litigieux528.
2. L’expropriation
En cas d’expropriation d’un lot privatif, l’ordonnance d’expropriation opère transfert de propriété en sorte que les charges sont dues par l’expropriant dès l’ordonnance d’expropriation et non pas seulement lors du paiement ou de la consignation du prix, ce quand bien même le copropriétaire exproprié conserve la
527 C.A. VERSAILLES 25 juin 1992, Gaz Pal 12 janvier 1993 somm. j. p 15, Revue Trimestrielle de Droit Immobilier 93.122
528 Civ. 3\degres 31 mars 1993, Informations Rapides de la Copropriété juillet-août 1993
droit de la copropriété année 2019-2020
420
jouissance de son lot jusqu’à l’expiration du délai d’un mois suivant le paiement ou la consignation du prix.529
3. Le décès du copropriétaire
Si un copropriétaire débiteur de charges vient à décéder, quatre hypothèses peuvent être envisagées :
En principe la dette se divise de plein droit entre les héritiers et l'on retrouve l'hypothèse précédemment évoquée de l'indivision (il peut d'ailleurs y avoir coexistence d'une indivision de la nue-propriété et d'une indivision de l'usufruit !).
Les héritiers peuvent renoncer à la succession. Auquel cas, le syndicat pourra demander en justice la désignation d'un administrateur judiciaire de la succession auquel il s'adressera pour le paiement des charges antérieures au décès et échues depuis le décès.
La succession est vacante (art 809-1 et s.)
Les héritiers n’ont pas pris parti (malgré mise en demeure des créanciers, ils ne se sont pas manifesté dans le délai légal de 6 mois), ils sont inconnus ou n’existent pas, ou encore tous les héritiers connus ont renoncé à la succession. En cette hypothèse, la succession est dite vacante.
A titre provisoire, tout intéressé, dont le syndic, peut demander au Tribunal de Grande Instance la désignation d'un curateur, qui sera l’Administration des domaines. Le curateur dresse l’inventaire de la succession, et règle les créanciers. Pour accélérer les choses, il est possible de s’adresser à un généalogiste pour qu’il fasse la recherche d’héritiers, étant précisé que les honoraires du généalogiste incombent à l’héritier … une fois qu’il aura été retrouvé.
La succession est définitivement en déshérence
L’état peut se faire envoyer en possession par les Tribunaux et liquidera la succession, donc procédera à la vente du lot, le syndic pouvant alors faire opposition sur le prix de vente du lot pour récupérer les charges des 5 dernières années
529 Civ 11 mai 1994 – inédit ; Civ 3\degres Ch 2 mars 2010 \no 09-13.724 Inédit
droit de la copropriété année 2019-2020
421
C. LE DEBITEUR DES CHARGES DANS LES IMMEUBLES EN COURS DE CONSTRUCTION
1. Vente d’un immeuble achevé
Si l'immeuble est un immeuble achevé lors de la vente des lots, l'acquéreur est tenu du jour de son entrée en jouissance du lot vendu.
A cet effet, le syndic est en droit de réclamer paiement des charges au nouveau propriétaire dès la date d'entrée en jouissance fixée à l'acte de vente530, et ce quand bien même la notification de la vente ne lui sera faite que postérieurement à cette date d'entrée en jouissance ou si l'acquéreur en désaccord avec son vendeur a refusé de prendre possession des locaux531.
2. La vente en Etat Futur d’achèvement
Si par contre, il s'agit d'une vente en l'état futur d'achèvement, l'acquéreur sera tenu de la date de remise des clés, la Société venderesse restant tenue des charges échues entre l'achèvement de l'immeuble et la remise des clés ou la date pour laquelle l'acquéreur a été mis en demeure de recevoir les clés.
Tant que l’immeuble n’est pas achevé, le syndicat ne naît pas à la vie civile, puisqu’aux termes de l’article 1er de la loi, celle-ci ne s’applique qu’aux immeubles bâtis. :
Paris 23\degres Chambre, 25 avril 1997532 :
La loi ne peut recevoir application avant édification, la période de construction étant exclue de son champ d’application. Il incombe au syndicat des copropriétaires demandeur de démontrer la date précise de l’achèvement qui conditionne la recevabilité de la demande de paiement des charges.
Quand bien même l’immeuble est-il construit, il semblerait que l’acquéreur en l’état futur ne sera débiteur des charges que du jour où ses lots sont achevés : en l’espèce les lots avaient été vendus en VEFA en 1975. Le vendeur n’avait pas satisfait à son obligation de délivrance et n’avait été condamné à procéder à cette délivrance en 1983. Assigné par le syndicat des copropriétaires pour des charges antérieures à 1983, la Cour de Cassation casse l’arrêt d’appel faisant droit à la demande du syndicat des copropriétaires au motif : « l'acquéreur n'est tenu des charges de copropriété qu'à partir de l'achèvement des lots acquis et
530 Rappelons qu’à défaut de stipulation à l’acte, l’acquéreur a la jouissance du bien dès accord sur la chose et sur le prix.
531 AIX 28 février 1983, Bull. Cour d'Aix 1983, 1 p. 45
532 Loyers et Copropriété 1997 \no 269.
droit de la copropriété année 2019-2020
422
sans rechercher si les lots étaient achevés à la date d'exigibilité des charges, la cour d'appel n'a pas donné de base légale à sa décision »533.
3. Le lot transitoire et le paiement des charges.
C'est surtout dans l'hypothèse de la construction d’ensembles immobiliers livrés par tranches successives que se posera la question de la participation du vendeur aux charges de copropriété :
Le statut de la copropriété s'applique dès l'achèvement de la première tranche. Mais s'applique t-il à tous les lots, même ceux qui ne sont pas encore construits ou seulement au bâtiment achevé ?
La question a donné lieu à une abondante jurisprudence marquée par une évolution certaine :
Un arrêt de la Cour de cassation du 13 mai 1987534 a en premier lieu admis que dans cette hypothèse, le vendeur devait payer certaines charges pour les lots non construits :
« Ne donne pas de base légale à sa décision, au regard de l'article. 10 de la loi \no 65-557 du 10 juill. 1965, la cour d'appel qui, pour condamner une SCI à payer les charges de copropriété au syndicat des copropriétaires, énonce que la SCI, propriétaire d'un lot non construit, auquel sont affectés 36 700/100 000e des parties communes générales, devait supporter les charges de copropriété dans cette proportion, sans avoir recherché la nature des charges réclamées. »
Monsieur ATTIAS, dans un commentaire de cet arrêt, a alors estimé que les charges de l'article 10 alinéa 1 ne seraient pas dues; par contre, seraient dues les charges de l'article 10 alinéa 2.
Les décisions postérieures, même si elles sont implicites, permettent d'affirmer que telle est bien la solution de la Cour de Cassation : le titulaire du lot transitoire est un copropriétaire a part entière et est en conséquence tenu au paiement des charges générales et des charges se l'article 10 alinéa 1 qui présentent une utilité objective pour son lot535.
“Viole l'art. 1er de la loi du 10 juillet 1965, ensemble l'art. 10 de cette loi, la cour d'appel qui, pour déclarer fondée en son principe la demande en paiement des charges relatives à la conservation, l'entretien et l'administration des parties communes, afférentes à des lots non construits (et ordonner une expertise pour vérifier les étapes de construction des différents bâtiments, en vue de faire les comptes entre les parties), énonce qu'en l'absence de modalités pour une mise en oeuvre progressive du règlement de copropriété, il y a lieu de s'en tenir à l'application du statut qui ne retient l'obligation au paiement des charges qu'à l'égard des immeubles bâtis, alors que, en conséquence des premières ventes, la propriété du groupe d'immeubles est répartie entre plusieurs
533 3e civ., 22 janv. 2014, \no 12-29.368, Loyers et Copropriété fev 2014
534 Civ 3\degres 13 mai 1987, JCP N 1987 II p 271
535 Civ. 3 mai 1990, Bull \no 107 p. 53
droit de la copropriété année 2019-2020
423
personnes et que l'application du statut de la copropriété entraîne l'obligation de participer aux charges suivant les distinctions de la loi “.
Au demeurant un arrêt de 1995536 casse un arrêt de cour d'appel qui avait affirmé que certaines charges générales ne présentant pas d'utilité pour les lots non bâtis, ces lots devaient être dispensés de payer les charges correspondantes. Pour la Cour de Cassation, la cour d'appel aurait dû faire application de l'article 10 alinéa 2 sans référence à l'utilité des charges générales.
SECTION III - LA SUSPENSION DES POURSUITES ET LES OBSTACLES AU RECOUVREMENT
A. A. REDRESSEMENT JUDICIAIRE OU LIQUIDATION DE BIENS.
Rappelons les principaux textes codifiés au livre VI du Code de Commerce depuis 2006
Article L621-43 (ancien article 50 loi du 25/01/85)
A partir de la date de l’ouverture de la procédure collective, il est interdit de régler des sommes appartenant au passif, en dehors des cas et conditions prévus par la loi. Les créanciers doivent produire leur créance auprès du représentant des créanciers.
1. 1. S’agissant des créances antérieures au Jugement
Article L621-43
A partir de la publication du jugement, tous les créanciers dont la créance a son origine antérieurement au jugement d'ouverture, à l'exception des salariés, adressent la déclaration de leurs créances au représentant des créanciers. Les créanciers titulaires d'une sûreté ayant fait l'objet d'une publication ou d'un contrat de crédit-bail publié sont avertis personnellement et, s'il y a lieu, à domicile élu.
Faute de production et en l’absence de relevé de déchéance, les créanciers perdent leur créance.
Si le Redressement judiciaire est transformé en liquidation de biens, les créanciers doivent produire de nouveau, cette fois entre les mains du liquidateur, ou ils perdent leur créance.
Il est donc indispensable que le Syndicat des Copropriétaires produise à la liquidation, pour conserver sa créance.
536 Civ 8 fév. 1995, Loyers et Copropriété 1995 \no 290
droit de la copropriété année 2019-2020
424
Le syndicat des copropriétaires est titulaire d’une sûreté s’il inscrit l’hypothèque légale de l’article 19 de la loi de 1965. Par contre il n’est pas privilégié en application de l’article 19-1 de la loi, tant que le lot du débiteur n’est pas vendu537 ; il ne peut donc produire qu’en qualité de créancier chirographaire s’il n’a pas pris l’hypothèque légale de l’article 19.
Cela n’empêchera nullement son privilège immobilier spécial de jouer lors de la vente du lot.
2. 2. S’agissant des créances postérieures au jugement
S’agissant des charges échues postérieurement au jugement déclaratif, les dispositions de l’article L 621-
32 reçoivent application :
Les créances nées régulièrement après le jugement d'ouverture sont payées à leur échéance lorsque l'activité est poursuivie. En cas de cession totale ou lorsqu'elles ne sont pas payées à l'échéance en cas de continuation, elles sont payées par priorité à toutes les autres créances, assorties ou non de privilèges ou sûretés, à l'exception des créances garanties par le privilège établi aux articles L. 143-10, L. 143-11, L. 742-6 et L. 751-15 du code du travail.
II. - En cas de liquidation judiciaire, elles sont payées par priorité à toutes les autres créances, à l'exception de celles qui sont garanties par le privilège établi aux articles L. 143-10, L. 143-11, L. 742-6 et L. 751-15 du code du travail, des frais de justice, de celles qui sont garanties par des sûretés immobilières ou mobilières spéciales assorties d'un droit de rétention ou constituées en application du chapitre V du titre II du livre 5.
III. - Leur paiement se fait dans l'ordre suivant :
1\degres Les créances de salaires dont le montant n'a pas été avancé en application des articles L. 143-11-1 à L. 143-11-3 du code du travail ;
2\degres Les frais de justice ;
3\degres Les prêts consentis par les établissements de crédit ainsi que les créances résultant de l'exécution des contrats poursuivis conformément aux dispositions de l'article L. 621-28 et dont le cocontractant accepte de recevoir un paiement différé ; ces prêts et délais de paiement sont autorisés par le juge-commissaire dans la limite nécessaire à la poursuite de l'activité pendant la période d'observation et font l'objet d'une publicité. En cas de résiliation d'un contrat régulièrement poursuivi, les indemnités et pénalités sont exclues du bénéfice de la présente disposition
537 Civ. 3\degres, 15 fev 2006 - La Semaine Juridique Notariale et Immobilière \no 36, 8 Septembre 2006, 1278 – Note Stéphane Piédelièvre
droit de la copropriété année 2019-2020
425
4\degres Les sommes dont le montant a été avancé en application du 3\degres de l'article L. 143-11-1 du code du travail
5\degres Les autres créances, selon leur rang.
Le Syndicat des Copropriétaires se retrouve alors au 5ème rang, parmi les « autres créances » : il s’agit là d’un « privilège » bien mince.
Mais le liquidateur est responsable s’il ne fait pas application des dispositions qui précèdent sur les fonds qu’il reçoit.
En tous cas il a été jugé qu’on ne saurait opposer au syndicat des copropriétaires la cessation d'activité du débiteur pour lui refuser le paiement des charges : la seule occupation des lots constitue une poursuite d'activité, même si aucun chiffre d'affaires n'est réalisé par le copropriétaire en état de redressement judiciaire538.
Enfin, il convient de citer l'arrêt de la cour d'appel de Paris du 3 mai 2001 sur le tribunal compétent pour le recouvrement de ces charges : aux termes de cet arrêt, même en cas d'une procédure collective engagée à l'encontre d'un copropriétaire, le tribunal de grande instance du lieu de situation de l'immeuble demeure seul compétent pour statuer sur une action en recouvrement des charges restant dues par le débiteur. Dans le cas d'espèce une société civile immobilière était débitrice du syndicat des copropriétaires et la cour considère que c'est à tort que les premiers juges ont décliné leur compétence au profit du tribunal de commerce de Bobigny. (Paris première chambre 3 mai 2000 ; Loyers et Copropriété novembre 2000 nº 253)539
B. LE COPROPRIETAIRE SURENDETTE.
Le juge saisi d’une demande de recouvrement de charges a la faculté en application de l’article 1244-1 « compte tenu de la situation du débiteur et en considération des besoins du créancier » de rapporter, rééchelonner, le paiement des sommes dues dans la limite de deux années.
538 GRENOBLE 6 novembre 1991, Revue des Loyers 1993 p 262; Voir également Ch; LEVINSOHN "Recouvrement des charges et faillite du copropriétaire in A.J.P.I., 1994, 26
539 Sur l’ensemble de la question se reporter à l’excellent article de Me Christophe Mounet Les droits du syndicat des copropriétaires et du propriétaire d’immeuble en cas de procédure collective du copropriétaire ou du locataire débiteurs, in Administrer \no 502, Octobre 2016, . 14 et s .
droit de la copropriété année 2019-2020
426
De plus et en application de la loi Neiertz du 31 décembre 1989, modifiée en dernier lieu par la loi du 1er août 2003, le débiteur surendetté bénéficie de la faculté d’obtenir un plan de redressement étalant le règlement de ses créanciers sur plusieurs années et, dans les cas extrêmes d’être placé sous le régime dit du «rétablissement personnel ».
Cette procédure ne peut être mise en oeuvre que pour les dettes personnelles du débiteur, personne physique qui doit être de bonne foi et qui se trouve dans l'impossibilité manifeste de faire face à ses dettes non professionnelles.
1. Procédure avec plan de redressement
La procédure est introduite devant la commission départementale de surendettement qui, après avoir déclaré la demande recevable, déterminé l'état d'endettement du débiteur , communiqué aux créanciers l'état du passif déclaré par le débiteur, élaborera un plan de redressement comportant des mesures ordinaires et extraordinaires (effacement partiel des dettes du débiteur).
Ce plan de redressement (sur une durée maximale de dix ans) est adressé au débiteur et aux créanciers qui sont invités à signer ce plan.
En cas d’échec de la proposition d’un plan de redressement le juge de l’exécution donne force exécutoire aux recommandations de la commission. Bien évidemment la procédure sera contradictoire entre débiteur et créanciers informés de cette procédure.
2. Le rétablissement personnel.
Il peut être mis en oeuvre à l’initiative de la Commission de Surendettement, du débiteur ou des créanciers, si « le débiteur est dans la situation irrémédiablement compromise ».
Le juge de l'exécution doit procéder à deux vérifications essentielles :
• Le caractère irrémédiablement compromis de la situation du débiteur,
• La bonne foi du débiteur.
Dans le cadre de cette procédure, tous les créanciers vont devoir produire leurs créances.
Cette production a lieu entre les mains du mandataire s’il en est désigné un (sinon, cette production se fera au secrétariat du JEX). Par contre, le législateur reconnaît au juge la faculté de prononcer le relevé de forclusion.
droit de la copropriété année 2019-2020
427
Cette procédure de production des créances est extrêmement importante : tant que l'on est dans le cadre de la procédure ordinaire, si le créancier demeure inconnu ou ne se fait pas spontanément connaître, le plan de redressement ne lui est pas opposable. Dans le cadre de la procédure de rétablissement personnel le défaut de production entraîne la disparition de la créance !
Le jugement de clôture de la procédure de rétablissement personnel entraîne l'effacement des dettes du débiteur : le syndicat des copropriétaires n’est qu’un créancier comme les autres !
C. CREATION D'UN SYNDICAT SECONDAIRE OU D’UNE UNION DE SYNDICATS.
Du jour de la création d'un Syndicat Secondaire il y a spécialisation des charges pour le bâtiment constitué en syndicat secondaire : il est procédé à la nouvelle répartition des charges en accord avec le syndicat principal ou à défaut, le tribunal pourra imposer cette nouvelle répartition540.
Il peut y avoir parfois course de vitesse : lorsque le syndicat envisage la réalisation de travaux importants et qu'il n'existe pas de ventilation des charges par bâtiment, un bâtiment (généralement de moindre importance que les autres) peut avoir intérêt à se constituer en syndicat secondaire. C'est la situation tranchée par un arrêt de la Cour de PARIS.541 Retenant que la spécialisation des charges prenait effet à compter de la création du syndicat secondaire, la Cour constate que les travaux de ravalement décidés par le syndicat principal l'ont été après création du syndicat secondaire, en sorte que ce dernier ne paiera que ses propres travaux de ravalement et non sa quote- part sur l'ensemble de la Copropriété.
Lorsqu’il existe une Union de Syndicats le même principe va recevoir application : l’Union de Syndicats ne peut agir directement à l’encontre du copropriétaire d’un des syndicats membre de cette Union pour avoir paiement de sa quote-part de charges de l’Union de syndicats542.
D. ABANDON DE SON LOT PAR LE COPROPRIETAIRE
Le copropriétaire ne peut pas abandonner son lot au syndicat des copropriétaires pour échapper au paiement des charges543 :
540 PARIS 2\degres Chambre 16 mars 1973, Gaz. Pal. 73.2 p. 729
541 PARIS 1er déc 1989; JURIS CLASSEUR CONSTRUCTION Fasc. 92 -B \no 41
542 (Cass. Civ III : 19.5.09)
543 Civ. 3\degres Ch. 7 avril 2004 6 Defrénois 2005, art . 38282
droit de la copropriété année 2019-2020
428
Dans le cas d’espèce le propriétaire était titulaire d’un lot transitoire jamais construit … devenu irréalisable. Pour échapper à ses obligations il avait fait acte chez un notaire d’abandon de ce lot au profit du syndicat. Cet abandon a été jugé impossible !
En outre, il est de jurisprudence constante que les charges de copropriété sont attachées au propriétaire du lot par les dispositions d'ordre public de l'article 10 de la loi et qu'elles sont dues, même si le lot est inoccupé.
Ceci alors même que le lot aurait été rendu inutilisable par suite de désordres provenant des parties communes de l'immeuble… ou démoli !
Rappelons cependant la faculté pour le copropriétaire de délaisser son lot situé en zone couverte par un Plan de Prévention des Risques Technologiques (article L 515-16 du code de l’environnement) avec obligation pour le syndic d’en informer l’assemblée générale (article 24-6 de la loi et 11-II-6 du Décret de 1967.
SECTION IV - LA DETERMINATION DES SOMMES EXIGIBLES
La loi ayant posé les principes, c’est le décret qui précise les sommes que le syndic peut appeler dans l’article 35 totalement remanié et qui est désormais ainsi rédigé :
Article 35 du Décret (rédaction du 27 mai 2004, complétée le 10 avril 2010 par l’ajout des deux derniers paragraphes) :
« Le syndic peut exiger le versement :
1\degres De l'avance constituant la réserve prévue au règlement de copropriété, laquelle ne peut excéder 1/6 du montant du budget prévisionnel ;
2\degres Des provisions du budget prévisionnel prévues aux deuxième et troisième alinéas de l'article 14-1 de la loi du 10 juillet 1965 ;
3\degres Des provisions pour les dépenses non comprises dans le budget prévisionnel prévues au I de l'article 14-2 de la loi du 10 juillet 1965 et énoncées à l'article 44 du présent décret ;
4\degres Des avances correspondant à l'échéancier prévu dans le plan pluriannuel de travaux adopté par l'assemblée générale ;
5\degres Des cotisations au fonds de travaux prévues au II de l’article 14-2 de la loi du 10 juillet 1965.
»
droit de la copropriété année 2019-2020
429
« Lors de la mise en copropriété d’un immeuble, le syndic provisoire peut exiger le versement d’une provision, lorsque celle-ci est fixée par le règlement de copropriété, pour faire face aux dépenses de maintenance, de fonctionnement et d’administration des parties et équipements communs de l’immeuble.
« Lorsque cette provision est consommée ou lorsque le règlement de copropriété n’en prévoit pas, le syndic provisoire peut appeler auprès des copropriétaires le remboursement des sommes correspondant aux dépenses régulièrement engagées et effectivement acquittées, et ce jusqu’à la première assemblée générale réunie à son initiative qui votera le premier budget prévisionnel et approuvera les comptes de la période écoulée. »
Cet article devrait être complété pour tenir compte des dépenses pour travaux décidées par le conseil syndical prévues à l’article 21-2 et énoncées à l’article 26-1 alinéa 2 du Décret : l’assemblée générale donnant délégation détermine les sommes afférentes qui sont exigibles selon les modalités fixées par la même assemblée générale.
A. A. L'AVANCE PERMANENTE DE TRESORERIE
1. Objet de l’avance permanente
Le syndic (s’il n’est syndic provisoire544) ne peut faire d’avances à la copropriété. Il a été jugé à plusieurs reprises qu’en faisant une telle avance le syndic commettait une faute par excès de gentillesse545 !
L’Ordonnance du 30 octobre 2019 a d’ailleurs complété le I de l’article 18 de la loi en édictant : « A l'exception du syndic provisoire et de l'administrateur provisoire désigné en application des articles 29-1 et 29-11, le syndic de copropriété ne peut avancer de fonds au syndicat de copropriétaires »
La Cour de cassation en 2009 a ajouté que le syndic ne pouvait, pour demander le remboursement de telles avances, invoquer la gestion d’affaires
Il est donc évident que si le syndicat des copropriétaires ne veut pas être à découvert, les copropriétaires devront laisser entre les mains du syndic de la copropriété une "avance permanente" ou ("fonds de roulement") suffisante pour payer les dépenses pendant que les "sommes exigibles" rentrent dans la caisse du syndicat.
A défaut il y aura rupture qui peut être extrêmement préjudiciable aux intérêts du syndicat... et du syndic : certaines dépenses doivent impérativement être réalisées à date fixe, parfois à bref délai, sous peine de majorations forfaitaires (les impôts) ou même de sanctions pénales (charges sociales).
544 Voir le 4 ci-après
545 CA PARIS 23\degres Chambre 16 octobre 1992 ADMINISTRER mars 1993 p. 61
droit de la copropriété année 2019-2020
430
Cependant une discussion s'est instaurée sur l'objet de l'avance permanente : Selon MM LAFOND et STEMMER 546les sommes ainsi recueillies ne sont pas destinées à faire face aux dépenses courantes; cette thèse a d'ailleurs reçu l'appui d'un arrêt de la Cour d’Appel de PARIS 547, aux termes duquel :
"Le fonds de roulement qu'il convient de distinguer des avances sur charges, s'analyse en une avance permanente de trésorerie, dont le syndic peut exiger le versement par chaque copropriétaire, soit en vertu du Règlement de Copropriété, soit, à défaut, par application de l'article 35 du décret, pour parer aux dépenses urgentes ou imprévues, dont on ne peut déterminer par avance la nature"
Cette interprétation n'est pas convaincante, car aboutissant à geler inutilement des fonds pendant de longues périodes où le syndic n'aurait pas à faire face à des dépenses imprévues ou urgentes et alors que ces dépenses imprévues et urgentes sont spécialement visées par l'article 37 du Décret.
2. Origine de l’avance permanente.
Le plus souvent, le Règlement de Copropriété prévoit cette avance de Trésorerie et l'Assemblée vote le "réajustement du fonds de roulement".
Si l’avance permanente n’est pas prévue dans son principe par le Règlement de Copropriété, elle pourra être instituée par l'assemblée générale statuant à la double majorité de l’article 26 de la loi (ou à la majorité de repêchage prévue à l’article 26-1 rétabli par l’Ordonnance du 30 octobre 2019.
Si le Règlement de Copropriété ou une assemblée générale ont prévu la constitution d’une avance permanente, le syndic sera en droit, aux termes de l'article 35 alinéa 1 du Décret d'exiger des copropriétaires qu’ils versent leur quote-part de cette avance permanente.
Mais le décret du 27 mai 2004, tenant compte de ce que le syndic doit avoir en caisse les appels provisionnels, a décidé de réduire le fonds de roulement à un maximum de deux mois du budget prévisionnel voté.
D’où l’importance de la clause inscrite dans le Règlement de Copropriété concernant cette avance permanente :
• Si le Règlement de Copropriété fixe le montant de l’avance permanente, le syndic appellera ce montant dès lors qu’il ne sera pas supérieur à deux mois de budget ; dans le cas contraire, il limitera son montant aux deux mois du budget voté.
546 op. cit. p. 440
547 23\degres chambre du 18 oct 1993, Loyers et Copropriété jan 1994 \no 32
droit de la copropriété année 2019-2020
431
• Le mieux est que désormais le Règlement de Copropriété précise simplement quee l’avance permanente correspondra à deux mois du budget annuel, afin d’éviter à l’assemblée générale de devoir réajuster son avance permanente chaque année.
3. Comment est appelée l’avance permanente de Trésorerie ?
Tout comme les provisions, le fonds de roulement sera appelé en charges communes générales et non sur la base de répartition des charges définitives. C’est ce que la cour de Paris548 avait décidé par le passé ; il n’y a pas de raison qu’elle change sa jurisprudence avec la nouvelle rédaction de l’article 35 du décret.
B. LES PROVISIONS DU BUDGET PREVISIONNEL PREVUES AUX DEUXIEME ET TROISIEME ALINEAS DE L'ARTICLE 14-1 DE LA LOI DU 10 JUILLET 1965 ;
En application de l’article 14-1 de la loi l'assemblée générale a voté son budget que le syndic va appeler par quarts tous les trimestres, à moins que l'assemblée générale ait décidé d’une périodicité différente.
Le syndic doit en effet appeler sa quote-part auprès de chaque copropriétaire, ceci en application du nouvel article 35-2 du décret qui est ainsi rédigé :
« Pour l'exécution du budget prévisionnel, le syndic adresse à chaque copropriétaire, par lettre simple, préalablement à la date d'exigibilité déterminée par la loi, un avis indiquant le montant de la provision exigible ».
C. LES PROVISIONS POUR LES DEPENSES NON COMPRISES DANS LE BUDGET PREVISIONNEL PREVUES A L'ARTICLE 14-2 DE LA LOI DU 10 JUILLET 1965 ET ENONCEES A L'ARTICLE 44 DU PRESENT DECRET ;
Il s’agit donc ici des provisions pour les travaux décidés par l’assemblée générale dont la « liste » est donnée par l’article 44 du décret du 17 mars 1967.
Sous réserve que l’assemblée générale ait préalablement voté ces travaux et à condition d’avoir fixé le calendrier des appels de fonds (exigence de l’article 14-2 de la loi : "Le syndic peut exiger le versement (...) de provisions spéciales destinées à permettre l'exécution (de ces travaux), dans les conditions fixées par ladite assemblée") le syndic pourra appeler aux échéances fixées par l'assemblée générale les sommes ou provisions afférentes à ces travaux.
548 CA PARIS 23\degres Ch. 18 oct 1993, Loyers et Copropriété 1994 \no 31
droit de la copropriété année 2019-2020
432
Le syndic adresse à chaque copropriétaire, par lettre simple, préalablement à la date d'exigibilité déterminée par la décision d'assemblée générale, un avis indiquant le montant de la somme exigible et l'objet de la dépense. En sorte que le syndic ne fait plus des « appels de fonds » pour ces sommes mais adresse aux copropriétaire des « avis d’exigibilité »549.
D. DES AVANCES CORRESPONDANT A L'ECHEANCIER PREVU DANS LE PLAN PLURIANNUEL DE TRAVAUX ADOPTE PAR L'ASSEMBLEE GENERALE
Le Président Capoulade affirmait que le plan pluriannuel correspondait à des « travaux décidés dans leur principe et dont l’exécution est programmée à court ou moyen terme » et rappelait qu’ils avaient été préconisés par la 16\degres recommandation de la Commission relative à la copropriété.
Cette disposition n’est pas récente : elle figure dans le Décret du 20 avril 2010 et a été conservée dans les mêmes termes depuis lors.
Le plan pluriannuel de travaux fait également partie du contenu obligatoire du carnet d’entretien que tout immeuble en copropriété doit posséder : « le carnet d’entretien mentionne (…) s’il existe, l’échéancier du programme pluriannuel de travaux décidé par l’assemblée générale des copropriétaires ».
La loi ALUR (articles L 731-1 et suivants du CCH) a imposé aux copropriétaires de voter à la majorité de l’article 24 de la loi, sur la réalisation d’un DTG (Diagnostic Technique Général) destiné à faire connaître aux copropriétaires les travaux qui seront nécessaires dans les dix prochaines années et à en chiffrer approximativement le coût.
Le technicien ainsi désigné, disposant des compétences nécessaires, établit le document qu’il remet au syndic. Le syndic doit alors obligatoirement soumettre à l’assemblée générale un plan pluriannuel de travaux ainsi que les modalités de son éventuelle mise en oeuvre.
Sur la base de ce plan sont donc appelées chaque année des « avances » qui permettront d’étaler dans le temps le financement des travaux projetés.
Pour autant le dispositif est presque systématiquement paralysé dans la mesure où si le syndic est obligé de mettre à l’ordre du jour la décision d’établir un DTG, rien n’oblige les copropriétaires à voter celui-ci, en sorte que faute d’évaluation des travaux, l’assemblée générale ne peut adopter de plan pluriannuel et voter les avances correspondantes.
Lors de la préparation de l’Ordonnance du 30 octobre 2019, le gouvernement avait élaboré un ensemble de textes rendant obligatoire le DTG et le vote d’un plan pluriannuel de travaux.
549 Il est vrai que cette distinction est quelque peu théorique, les syndics continuant le plus souvent de qualifier ces avis d’exigibilité d »appels de « provisions »
droit de la copropriété année 2019-2020
433
Ce projet, accepté par tous les intervenants a cependant été écarté à la demande du Conseil d'Etat qui a craint qu’une telle obligation soit retoquée par la Conseil Constitutionnel comme portant atteinte au droit de propriété (il est vrai que le projet du gouvernement prévoyait une cotisation annuel de 2.5 % du montant des travaux et que ces cotisations auraient été des provisions (non remboursables en cas de vente du lot) et non plus des avances qui sont remboursables.
Le Ministre du Logement a affirmé que d’une façon ou d’une autre cette dispositions serait prochainement introduite dans la loi550.
Reste à savoir si ces travaux doivent être préalablement définis dans leur quantum ou s’ils peuvent faire l’objet d’une simple estimation qui servira de base aux appels de fonds correspondant.
E. DES COTISATIONS AU FONDS DE TRAVAUX PREVUES AU II DE L’ARTICLE 14-2 DE LA LOI DU 10 JUILLET 1965.
La loi du 21 juillet 1994 avait créé ces provisions pour travaux futurs éventuel : le syndic devant lors de sa première désignation et au moins tous les trois ans, soumettre au vote de l'assemblée la décision de constituer des provisions en vue de travaux d'entretien des parties communes.
On sait que cette disposition n’a connu aucun « succès » dans la plupart des régions de France, les copropriétaires refusant de voter ces avances.
Le Rapport Braye avait proposé pour prévenir les difficultés dans les copropriétés de planifier les travaux sur plusieurs années et de créer un fonds de travaux obligatoire qui resterait attaché au lot en cas de vente. Cette idée a été reprise dans le projet de loi ALUR et après quelques péripéties parlementaires finalement retenue ; elle fait l’objet du II de l’article 14-2 :
- Pour les immeubles comprenant au moins un lot à usage d’habitation (quelle que soit sans doute la qualification donnée à ce lot dans l’état descriptif de division).
- Le montant doit être au moins de 5 % du budget annuel .
- En sont cependant exonérés les immeubles dont le diagnostic technique global ne fait apparaître aucun besoin de travaux dans les dix prochaines années
- En sont également exonérés les immeubles qui comportent moins de dix lots (principaux ou accessoires) lorsque les copropriétaires ont décédé à l’unanimité de ne pas constituer de fonds de travaux
- Lorsque le montant du fonds de travaux atteint le montant du budget provisionnel l’assemblée générale peut décider de faire les travaux ou de suspendre les cotisations annuelles.
- Les fonds reçus seront obligatoirement placé sur un compte bancaire distinct du compte « général » ouvert au nom du syndicat des copropriétaires .
550 Le Monde du 27 novembre 2019, Article intitulé « critiqué sur sa réforme, le gouvernement promet de revoir sa copie ».
droit de la copropriété année 2019-2020
434
- Les fonds reçus sont destinés à financer les travaux hors budget, les travaux prescrits par les lois et règlements et, après décision d’assemblée générale, aux travaux urgents551
- Les fonds reçus ne sont pas des avances (remboursables) mais des sommes qui demeurent acquises au syndicat des copropriétaires (attachées au lot).
Le syndic pourra donc appeler les sommes correspondantes au poucentage du budget retenu par l’assemblée générale.
En fait ce « fonds de travaux »a posé d’emblée un certain nombre de questions: Essentiellement peut-on utiliser ce fonds de travaux pour la réalisation de travaux sur parties communes spéciales alors qu’il est appelé sur la base du budget global du syndicat des copropriétaires ? Cette question est particulièrement importante pour les grandes copropriétés où l’ensemble ne détient en commun que le sol et les équipements généraux tandisque les immeubles sont parties communes spéciales. Certes, s’il existe des syndicats secondaires la question d’affectation du fonds de travaux sera secondaire. Dans le cas contraire il apparaît choquant d’utiliser un fonds abondé par tous les copropriétaires pour entretenir certains bâtiments parties communes spéciales. C’est pourquoi la loi ELAN a ajouté une phrase au texte de la loi ALUR : « Les sommes versées au titre du fonds de travaux sont attachées aux lots et définitivement acquises au syndicat des copropriétaires. ».
Nous avons précédemment indiqué qu’il s’agit de provisions non remboursables en cas de vente du lot.
F. LES PROVISIONS EXIGIBLES PAR LE SYNDIC PROVISOIRE.
Le Décret de 1967 ne traitait pas du syndic provisoire , c'est-à-dire de la personne désignée au Règlement de Copropriété pour assurer la gestion de la copropriété jusqu’à la désignation d’un syndic conventionnel par la première assemblée générale.
Or ce syndic provisoire est pourtant nécessaire et sa mission peut se poursuivre pendant une durée maximum d’un an de la mise en copropriété de l’immeuble.
Inconnu du décret, ce syndic provisoire ne pouvait –légalement – demander aucune somme aux copropriétaires. Le Décret du 20 avril 2010 a voulu combler ce vide juridique. Il l’a fait dans les termes suivants :
« Lors de la mise en copropriété d’un immeuble, le syndic provisoire peut exiger le versement d’une provision, lorsque celle-ci est fixée par le règlement de copropriété, pour faire face aux dépenses de maintenance, de fonctionnement et d’administration des parties et équipements communs de l’immeuble.
551 La rédaction de la loi est redondante : qu’il s’agisse de travaux décidés spontanément par l’assemblée générale, de travaux prescrits ou de travaux urgents, uil s’agira dans tous les cas de travaux hors budget décidés par l’assemblée générale.
droit de la copropriété année 2019-2020
435
« Lorsque cette provision est consommée ou lorsque le règlement de copropriété n’en prévoit pas, le syndic provisoire peut appeler auprès des copropriétaires le remboursement des sommes correspondant aux dépenses régulièrement engagées et effectivement acquittées, et ce jusqu’à la première assemblée générale réunie à son initiative qui votera le premier budget prévisionnel et approuvera les comptes de la période écoulée. »
On relève que deux hypothèses sont distinguées, selon que le Règlement de Copropriété a ou non prévu la faculté pour le syndic provisoire d’appeler une provision :
- Si le Règlement de copropriété prévoit cette faculté, il doit en préciser l’objet qui ne peut être que le règlement de dépenses de maintenance, de fonctionnement et d’administration des parties et équipements de l’immeuble.
- Si le Règlement de copropriété ne prévoit pas cette faculté, le syndic provisoire pourra cependant exiger le remboursement des sommes dépensées jusqu’à la date de la première assemblée générale. Il en sera de même si la provision prévue au Règlement de Copropriété est épuisée avant cette date, pour les sommes dépensées au-delà de ce montant.
Dans tous les cas le syndic provisoire devra faire approuver ses comptes par la première assemblée générale.
G. AUTRES SOMMES QUI PEUVENT ETRE DUES PAR LES COPROPRIETAIRES
1. Les appels pour travaux urgents.
L’ancien article 35 du décret permettait au syndic d’appeler une partie des fonds nécessaires aux travaux urgents. Le nouveau texte de l’article 35, issu dans sa rédaction du décret 20 avril 2010, n’en fait plus état.
En fait, il s’agit d’une clarification dans la mesure où le sort des avances pour travaux urgents est régi par l’article 37 du décret qui n’a pas été modifié et qui est ainsi rédigé :
« Lorsqu'en cas d'urgence le syndic fait procéder, de sa propre initiative, à l'exécution de travaux nécessaires à la sauvegarde de l'immeuble, il en informe les copropriétaires et convoque immédiatement une assemblée générale.
Par dérogation aux dispositions de l'article 35 ci-dessus, il peut, dans ce cas, en vue de l'ouverture du chantier et de son premier approvisionnement, demander, sans délibération préalable de l'assemblée générale mais après avoir pris l'avis du conseil syndical, s'il en existe un, le versement d'une provision qui ne peut excéder le tiers du montant du devis estimatif des travaux
droit de la copropriété année 2019-2020
436
Il ne peut demander de nouvelles provisions pour le paiement des travaux qu'en vertu d'une décision de l'assemblée générale qu'il doit convoquer immédiatement et selon les modalités prévues par le deuxième alinéa de l'article 14-2 de la loi du 10 juillet 1965.».
2. Les intérêts de retard…. A vos calculettes !
Selon l'article 36 du Décret \no67-223 du 17 mars 1967
Sauf stipulation contraire du règlement de copropriété, les sommes dues au titre de l'article 35 portent intérêt au profit du syndicat. Cet intérêt, fixé au taux légal en matière civile, est dû à compter de la mise en demeure adressée par le syndic au copropriétaire défaillant. »
Le syndic doit donc veiller à adresser une lettre recommandée (article 64 du Décret) : "
Toutes les notifications et mises en demeure prévue par la loi et le présent Décret (...) sont valablement faites par lettre recommandée".
A défaut, le syndicat pourra lui reprocher sa négligence et demander réparation du préjudice subi.
Rappelons que l’intérêt légal était fixé par périodes annuelles jusqu’au 31 décembre 2014 :
2004 = 2,27 % 2005 = 2,05 % 2006 = 2,11 %
2007 = 2,95 % 2008 = 3,99 % 2009 = 3,79 %
2010 = 0,65 % 2011 = 0,38 % 2012= 0, 71 %
2013 = 0, 04 % 2014 = 0, 04 % 552
\degres Un choc de simplification :
Depuis le 1er janvier 2015 le taux d’intérêt légal553 est :
- Fixé semestriellement et non plus annuellement
- Différent selon que le créancier est une personne physique ou un professionnel
- Le syndicat des copropriétaires n’est pas un particulier personne physique ; en conséquence, que le copropriétaire débiteur soit un particulier n’agissant pas pour des besoins professionnels, un professionnel ou une personne morale, le taux de l’intérêt légal le plus bas s’appliquera pour les charges dues au syndicat des copropriétaires.
552 Décret du 2014-98 du 4 février 2014 553 Décret du 2 octobre 2014
droit de la copropriété année 2019-2020
437
On peut regretter que le syndicat des copropriétaires se trouve de facto assimilé à un professionnel alors qu’il ne poursuit aucune activité lucrative et qu’au sens du code de la consommation le syndicat des copropriétaires est assimilé à un … non professionnel !
Certes, le Règlement de Copropriété peut écarter l'application de l'intérêt légal... pour fixer un taux inférieur ou (le plus souvent) supérieur au taux légal. Cette dernière hypothèse sera réputée non écrite : le taux légal s’appliquera de plein droit si le taux contractuel lui est supérieur.
Relevons enfin que les intérêts joueront sur les sommes allouées au syndicat des copropriétaires depuis la mise en demeure, quand bien même le juge ne lui accorderait pas le plein de sa demande554
554 Civ 3\degres 15 février 2005 – IRC 2006 \no 515 p 13
Date du JO
Semestre
Créances des
Personnes Physiques
Toutes les autres
créances
JO du 23/12/2019
1 S 2020
3.28 %
0.87 %
JO du 26/06/2019
2 S 2019
3.40%
0.87 %
JO du 21/12/2018
1 S 2019
3.28 %
0,86 %
JO du 27 juin 2018
2 S 2018
3.60%
0.88%
JO du 30/12/2017
1 S 2018
3.73%
0.89%
JO du 27/12/2017
1 S 2018
3.73%
0,89%
JO du 30/06/2017
2 S 2017
3.94%
0.90%
JO du 30/12/2016
1S 2017
4.16 %
0.90 %
JO du 24/06/2016
2S 2016
4,35 %
0,93 %
JO du 27/12/2015
1S 2016
4,54 %
1,01 %
JO du 28/06/2015
2S 2015
4,29 %
0.99 %
JO du 27/12/2014
1S 2015
4,06 %
0,93 %
droit de la copropriété année 2019-2020
438
SECTION V - LE CALCUL DES CHARGES DUES PAR LE COPROPRIETAIRE DEBITEUR
A. LA VENTILATION DES CHARGES
Les provisions et charges sont dues en fonction de la ventilation prévue au règlement de copropriété (éventuellement modifiée en assemblée) ou par décision de justice.
1. La ventilation résultant du règlement de copropriété ou d’une modification régulière des charges
Les charges dues sont celles prévues au règlement de copropriété ou celles qui résultent de l'application de décisions de justice définitives.
S’il s’agit d’une décision ayant réputé la répartition de charges non écrites, par application de la nouvelle rédaction de l’article 43 (Dernière phrase) issue de l’Ordonnance du 30 octobre 2019 : « Cette nouvelle répartition prend effet au premier jour de l'exercice comptable suivant la date à laquelle la décision est devenue définitive ».
Les charges dues sont également celles qui résultent de décisions d’Assemblées Générales ayant modifié la répartition des charges fixée par le Règlement de Copropriété, peu importe au demeurant que l’Assemblée Générale se soit régulièrement ou irrégulièrement prononcé. En tout cas on ne peut que conseiller aux syndics de proposer à l’assemblée générale de décider que la nouvelle répartition n’entreta en application qu’au premier jour de l’exercice comptable à venir.
2. La ventilation résultant d’une modification irrégulière des charges
Nous avons dit précédemment que la ventilation des charges telle que prévue par le Règlement de Copropriété ne peut, sauf exception, être modifiée que par voie de justice ou par l'unanimité des copropriétaires. Cependant, il arrive assez fréquemment que l'Assemblée vote irrégulièrement une modification de la répartition des charges (c'est à dire sans que l'unanimité soit réunie).
En ce cas, le syndic devra appliquer la nouvelle répartition, même votée irrégulièrement.
Mais si dans un premier temps il a été jugé que passé le délai de deux mois de contestation d’une assemblée générale ayant modifié une répartition de charges, un copropriétaire ne pouvait agir sur le fondement de l’article 43 pour faire juger « inexistante » la nouvelle répartition 555, il est admis
555 Civ 17 juillet 1991; CIV 3\degres 19/02/92 Loyers et Copropriété mai 1992 \no 223;; PARIS 23\degres 12 sep 1994 Loyers et Copropriété janvier 1995 \no 34
droit de la copropriété année 2019-2020
439
aujourd’hui que l’on ne peut opposer à une action en « inexistence » le fait que la répartition contestée avait fait l’objet d’une décision d’assemblée générale non contestée. d’une répartition de charges que cette répartition ayant été votée par une assemblée générale non contestée dans le délai de deux mois :
Civ. 3\degres Chambre, 27 septembre 2000
« Le délai de deux mois pour contester une décision d'assemblée générale ne s'applique pas à l'égard d'une résolution modifiant irrégulièrement une répartition des charges, s'agissant d'une action relative aux clauses réputées non écrites »556
3. La ventilation résultant de clauses du règlement de copropriété contraires aux règles d’ordre public de répartition des charges
LE SYNDIC N’A PAS A SE FAIRE JUGE.
Le syndic n'a à se faire juge, ni de la régularité de la répartition des charges telle que fixée par le Règlement de Copropriété, ni de la régularité de la délibération d' Assemblée Générale modifiant la répartition des charges : il doit faire application des documents existants.
Bien souvent le Règlement de Copropriété comporte des erreurs flagrantes : un lot qui n'est pas raccordé au chauffage se voit doté de charges de chauffage, etc... et d'un accord tacite entre copropriétaires et syndic, ce dernier appliquant la loi de préférence au règlement de copropriété appelle les charges en "redressant" les anomalies du règlement de copropriété.
Une telle pratique, aussi généreuse soit-elle, doit être condamnée sans réserve car source de procédure ultérieure et de redressement des comptes quasiment impossible à réaliser si des mutations sont intervenues.
L’ASSEMBLEE GENERALE N’A PAS NON PLUS A SE FAIRE JUGE :
Civ. 3ème 21 juin 2006
« Les clauses d'un règlement de copropriété doivent recevoir application tant qu'elles n'ont pas été déclarées non écrites par le juge ; dès lors, viole les dispositions des articles 10 et 43 de la loi du 10 juillet 1965 une cour d'appel qui rejette une demande d'annulation d'une décision d'assemblée générale de copropriétaires relative à la répartition des charges d'une porte cochère automatisée en retenant que cette répartition est conforme aux dispositions d'ordre public de l'article 10 sur
556 Loyers et Copropriété décembre 2000 nº 279. Dans le même sens Civ. 3\degres Ch. 12 mai 2016, Pourvoi \no 15-15166, Inédit
droit de la copropriété année 2019-2020
440
lesquelles les mentions contraires du règlement de copropriété rédigé en 1958 ne peuvent l'emporter».
B. LES CHARGES ET LA PRESCRIPTION.
1. Historique : la prescription de dix ans
Aux termes de l'article 2277 du Code civil, se prescrivent par cinq ans :
"les loyers et les fermages (...) et généralement tout ce qui est payable par année ou à des termes périodiques plus courts".
C’est ainsi qu’il est jugé en matière locative557 que :
« Selon l'art. 2277 C. civ., la prescription quinquennale concerne les actions en paiement non seulement des loyers mais généralement de tout ce qui est payable par année ou à des termes périodiques plus courts. Tel est notamment le cas des charges stipulées payables «au moyen d'une provision mensuelle payée en même temps que le loyer, l'apuration des comptes se faisant annuellement ou trimestriellement».
Dès lors, cette prescription doit s'appliquer non seulement à la demande concernant les loyers mais également à celle concernant les charges locatives.
Mais la Cour de cassation ajoute une condition à l'application de ce texte qui est la "fixité des sommes dues". Les charges étant l’accessoire du loyer présentent ce caractère de fixité.
S’agissant des charges de copropriété dues par le copropriétaire, par définition, ces charges sont variables. C'est ce qu'a jugé la Cour de Paris.558
L'action en recouvrement des charges est en conséquence soumise à la prescription prévue à l'article 42 de la loi du 10 juillet 1965 pour les actions personnelles du syndicat à l'encontre des copropriétaires.559
557 Paris (2e Ch. B), 5 janvier 1990 05/01/90 Paris (2e Ch. B), D., 1990, IR p. 44
558 Paris 26 oct 1979 (D. IR. p. 100).
559 PARIS 5 janvier 1988, Loyers et Cop. 1988 \no 138; PARIS 10 juillet 1990 Loyers et Copropriété 1990 \no 401; PARIS 19\degres Ch. 21 novembre 1994, Loyers et copropriété mai 1995 \no 238.
droit de la copropriété année 2019-2020
441
La cour de cassation a pris position de façon très nette en 1999 et en 2000560 :
« L'action en recouvrement des charges n'est pas soumise à la prescription quinquennale de l'art. 2277 C. civ. » et « l'action en recouvrement des charges de copropriété, qui sont nécessairement indéterminées et variables, ne rentre pas dans le champ d'application de l'art. 2277 C. civ. et constitue une action personnelle née de l'application de la loi du 10 juillet 1965. »
L’action en paiement au titre des charges et arriérés et la contestation des sommes réclamées sont des actions personnelles qui se prescrivent par un délai de 10 ans et non de 5 ans561.
2. L’Ordonnance du 18 septembre 2019 : Prescription de droit commun.
Les mêmes principes continuent de s’appliquer : il convient donc toujours de faire application du délai de prescription de l’article 42 de la loi.
Mais l’Ordonnance du 18 septembre 2019, applicable immédiatement, a remplacé le délai de dix ans par le délai de droit commun de
L’article 2224 du code civil
Les actions personnelles ou mobilières se prescrivent par cinq ans à compter du jour où le titulaire d'un droit a connu ou aurait dû connaître les faits lui permettant de l'exercer.
Bien évidemment le syndicat des copropriétaires a la possibilité d’exercer ses droits du jour où les sommes sont exigibles, en sorte que l’action en paiement des charges devra être introduite à l’intérieur de ce délai de 5 ans de l’exigibilité de chaque créance.
Cette disposition étant d’application immédiate, à compter du 18 septembre 2019, le syndicat ne dispose plus que d’un maximum de cinq ans pour récupérer sa créance, quand bien même celle-ci remontrait à plus de 5 ans.
560 3\degres Ch civ 17 novembre 1999 Gaz. Pal., Rec. 2000, somm. p. 261, J. \no 39 et 3\degres Ch. civ. 11 mai 2000 11/05/2000 (Loyers et copr., 2000, comm. \no 281)
561 Cass. Civ. 3e 9 février 2010